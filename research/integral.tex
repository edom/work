\chapter{Integral}

\section{Measure}

A \emph{Jordan content} is...

\index{content (measure theory)}%
A \emph{content} \(m : 2^A \to \Real\) satisfies
\begin{enumerate*}[label={(\arabic*)}]
    \item \(\forall x \in A : m~x \ge 0\),
    \item \(m~\emptyset = 0\), and
    \item \(\forall X \subseteq A, Y \subseteq A, X \cap Y = \emptyset : m~(X \cup Y) = m~X + m~Y\).
\end{enumerate*}

\index{measure}%
\index{countably additive}%
\index{countable additivity}%
A \emph{measure} \(m\) on a set \(A\) is a content that is \emph{countably additive}.

\index{Lebesgue measure}%
\index{measure!Lebesgue}%
A \emph{Lebesgue measure} for \(\Real\) is
\(m~(a,b) = m~[a,b) = m~(a,b] = m~[a,b] = |b-a|\).
From a measure \(m : \Real \to \Real\)
we can define a measure \(\mu : \Real^n \to \Real\)
where
\(\mu \left( \prod_{i=1}^n X_i \right) = \prod_{i=1}^n m(X_i)\).
\index{measure space}%
\index{measure!space}%
\index{space!measure}%
A \emph{measure space} is a set and a measure on that set.

\index{measurable function}%
\index{function!measurable}%
A \emph{measurable function} is ...

\section{Integral}

\index{integral}%
\index{integral!Lebesgue}%
\index{integral!Riemann}%
\index{Lebesgue integral}%
\index{Riemann integral}%
Let \((\Real,m)\) be a measure space and \(f\) be a function.
Let \(X \subseteq \Real\) and \(Y = \{ f(x) ~|~ x \in X\}\).
Let \(X_1,\ldots,X_n\) be a partitioning of \(X\) and \(Y_1,\ldots,Y_n\) be a partitioning of \(Y\).
For each \(k\), let \(x_k \in X_k\) and \(y_k \in Y_k\).
An \emph{integral of \(f\) in \(X\)}
is \(\int_X f = \lim_{n\to\infty} \sum_{k=1}^{n} a_k\)
where each \(a_k = m(X_k) \cdot m(Y_k)\) is a rectangular part of the total area.
See Table \ref{tab:integral}.

\begin{table}[h]
    \caption{How integrals partition spaces}
    \label{tab:integral}
    \centering
\begin{tabular}{lll}
    Name & \(X_k\) & \(Y_k\)
    \\
    \hline
    Riemann integral & \( [x_k,x_{k+1}]\) & \( [0,f(x_k)]\)
    \\
    Stieltjes integral with respect to \(g\) & \( [g(x_k),g(x_{k+1})]\) & \( [0,f(x_k)]\)
    \\
    Lebesgue integral & \( \{x ~|~ f(x) \in Y_k\}\) & \( [y_k,y_{k+1}]\)
\end{tabular}
\end{table}

\section{Calculus}

\index{fundamental theorem of calculus}%
\emph{Fundamental theorem of calculus}:
Iff \(D(F) = f\), then \(I(f) = F + c\) where \(c\) is a constant function,
\(D\) is the derivative operator, \(I\) is the antiderivative operator.
