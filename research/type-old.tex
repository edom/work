\chapter{Type theory (old computation book chapter draft)}

The type theory described here is inspired by
Martin-L\"of type theory
and the Haskell programming language.
The type theory is constructive (intuitionistic) and dependent.

\section{Notation}

Iff \(a\) is a type then \(|a|\) is the \emph{cardinality} of \(a\).
It is the number of inhabitants of that type.

\section{Common types}

\begin{itemize}
    \item
        $\Void$ is the empty type.
        It has no inhabitant.
    \item
        $\Bool$ is the type of boolean values.
        This type has two inhabitants: $\true$ and $\false$.
    \item
        $\Bit$ has two inhabitants: $0$ and $1$.
    \item
        $\Fun{a}{b}$ is the type of functions from $a$ to $b$.
        The function can be partial.
        This type is also be written $a \to b$.
    \item
        $\Pred{a}$ is the type of
        logical predicates about objects of type $a$.
        We define
        \[ \Pred{a} = \Fun{a}{\Bool}. \]
    \item
        $\Nat$ is the type of natural numbers.
        This type is defined using Peano axioms.
    \item
        $\Set{a}$ is the type of the set of elements of type $a$.
    \item $\List{a} = \Kleene{a}$ is the Kleene closure of $a$.
        A list of $a$ is an ordered collection of elements of type $a$ with duplicates allowed.
        so we can write $[x,y,z] 1 = y$.
    \item $\Bits$ is the type of bitstrings.
        \[
            \Bits = \Kleene{\Bit}
        \]
    \item $\Either{a}{b}$ is the sum type or the union type
        that consists of $\Left{x}$ for all $x : a$
        and $\Right{y}$ for all $y : b$.
    \item $\Pair{a}{b}$ is the product type
        that consists of $(x,y)$ for all $x : a$ and $y : b$.
    \item
        $\Vect{a}{n}$ is the type of vectors
        whose element type is $a$ and whose length is $n : \Nat$:
        \[ \Vect{a}{n} = \Fun{(I n)}{a} \]
        where
        \[ I n = \{ 0, 1, 2, \ldots, n - 1 \} \]
        or alternatively using predicate logic
        \[ \Fa{x} (x : \Nat \wedge x < n \iff x : I n) \]
    \item $\Typ$ is the type of types.
        This implies
        \[ \Typ : \Typ \]
        (Does $\Typ : \Typ$ imply inconsistency?
        Russell's paradox?
        Unrestricted comprehension?)

        This means that $\sfSet$ and $\sfRel$ can be thought as the \emph{type functions}
        \begin{align*}
            \sfSet &: \Typ \to \Typ,
            \\
            \sfRel &: \Typ \to \Typ \to \Typ.
        \end{align*}
\end{itemize}

The cardinality of $\Nat$ is $\aleph_0$ (aleph-null).

\section{Cardinality of types}

$|\List{a}| = |\Fun{\Nat}{a}|$.

\(|a \to b| = |b|^{|a|}\)

If a type has finitely many inhabitants,
then the cardinality of that type is the number of its inhabitants.

If there is a bijection between two types,
then those types have the same cardinality.

Type-theoretic restatement of Cantor's theorem?
There is no bijection between $a$ and $\Set{a}$.
$|a| < |\Set{a}|$.

Kind of ordering on cardinalities:
\begin{itemize}
    \item Iff there is an injection from $a$ to $b$, then $|a| \le |b|$.
    \item Iff there is an surjection from $a$ to $b$, then $|a| \ge |b|$.
    \item Iff there is a bijection between $a$ and $b$, then $|a| = |b|$.
\end{itemize}

Two sets are equinumerous (have the same cardinality) if and only if there is a bijection between them.

$|\sfT a| = |\sfT (\Set{a})|$?

\section{Cardinality theorems}

\begin{mthm}
    \[
        |a| \lneq |\Set{a}|
    \]
\begin{proof}
    Has been proved by Georg Cantor using Cantor's diagonal argument
    that the cardinality of a set is strictly less than its power set.
    Beth numbers.
    $\beth_n \lneq \beth_{n+1}$ for each natural number $n$.
\end{proof}
\end{mthm}

\begin{mthm}[Equinumerosity among one-parameter types]
    For each $a$, all these types have the same cardinality:
    $\Pred{a}$, $\Set{a}$.
    \begin{proof}
        Let $p : \Pred{a}$ be a predicate and $s : \Set{a}$ be a set.
        We define $p$ and $s$ such that each object that satisfies the predicate $p$ is an element of the set $s$
        and also such that each element of the set $s$ satisfies the predicate $p$.
        \begin{align*}
            F p &= \{ x \,|\, p x \}
            \\
            G s &= \lambda x \to x \in s
        \end{align*}
        The relationship is
        \[ \FA{x} (p x \iff x \in s) \]

        But what if $p x = x \not\in S$.
        Or what if $p x = \neg\exists S ~ x \in S$?
        Or what if $p x = \Fa{S} x \in S$?
        Or what if $p x = x \in x$?
        What if $p x = \neg (p x)$?
        Isn't this prone to Russell's paradox?
        Unrestricted comprehension?
        FIXME?

        Or is this not prone?
        $p$ cannot refer to $s$?
        Can it?
    \end{proof}
\end{mthm}

Thus a predicate is a set and a set is a predicate.
It turns out that there is a name for this concept:
that set is the \emph{extension} of that predicate.
If $p$ is a predicate, then $p$ is also a set,
so we can write $x \in p$ to mean that $p x$ is true.
What if we assume that a predicate is equal to its own extension?
Now we make a bold but reasonable claim:
a predicate \emph{is} a set and a set \emph{is} a predicate.
This has some interesting consequences.

If we assume the equality, then $p$ becomes a fixed point of $\phi \mu$.
To see this, we have to define several functions.
Let $\phi$ be the flip combinator, that is $\phi f x y = f y x$.
Let $\mu$ be the set membership function, that is $\mu x y = x \in y$.
Recall that the $\eta$-reduction transforms $p x = q x$ to $p = q$.
\begin{align*}
    p x &= x \in s
    \\
    &= \mu x s
    \\
    &= \phi \mu s x
    \\
    p &= \phi \mu s
    \\
    p &= s
    \\
    p &= \phi \mu p
    \\
    p &= \phi \mu (\phi \mu p)
    \\
    &= \phi \mu (\phi \mu (\phi \mu p))
    \\
    &= \ldots
\end{align*}
That implies that we can write strange but provable things like these:
\begin{align*}
    1 \in \{0,1,2\} &= \{0,1,2\} 1 = \true
    \\
    3 \in \{0,1,2\} &= \{0,1,2\} 3 = \false
    \\
    (\lambda x \to x = 1) 1 &= 1 \in (\lambda x \to x = 1) = \true
\end{align*}
but this can be confusing at first.
Should we distinguish predicate and set?
Should we treat them as the same thing?
The membership operator $\in$ becomes swapped function application.

We can even generalize the notation $f x = x \in f$ to every function $f : a \to b$, not just predicates.
Let $f x = x + 1$. Then $f 0 = 0 \in f = 1$.
This may need some effort and time to get accustomed to,
but once you master it, you will be another mathematician.

\begin{mthm}[Equinumerosity among two-parameter types]
    All these types have the same cardinality:
    \begin{itemize}
        \item $\Relab{a}{b}$, $\Relab{b}{a}$
        \item $\Pred(a,b)$, $\Pred(b,a)$
        \item $\Set{(a,b)}$, $\Set{(b,a)}$
        \item $\Fun{a}{(\Set{b})}$, $\Fun{(\Set{a})}{b}$
    \end{itemize}
    \begin{proof}
        Proving $|\Relab{a}{b}| = |\Relab{b}{a}|$ is simple.

        Proving $|\Pred(a,b)| = |\Pred(b,a)|$ is simple.

        $r : \Relab{a}{b}$ and $p : \Pred(a,b)$ and $f : a \to b \to \Bool$.
        $r$ relates $x$ to $y$ if and only if $p(x,y)$ is true.
        \begin{align*}
            p z &= z \in r
             \\ &= \mu z r
             \\ &= \phi \mu r z
            \\
            p &= \phi \mu r
        \end{align*}
        Then let $p = r$.

        Since there is a bijection between $\Pred{(a,b)}$ and $\Relab{a}{b}$
        and between $\Pred{a}$ and $\Set{a}$,
        there is a bijection between $\Relab{a}{b}$ and $\Set{(a,b)}$.

        To prove that there is a bijection between $\Relab{a}{b}$ and $\Fun{a}{(\Set{b})}$,
        we choose any $r : \Relab{a}{b}$ that is a relation
        from objects of type $a$ to objects of type $b$.
        Define the \emph{image of $x$ in $r$} as
        $i r x = \{ y \,|\, \text{$r$ relates $x$ to $y$} \}$
        where the type of $i$ is $\Relab{a}{b} \to a \to \Set{b}$.
        We define the \emph{relation functionization} function $F$ as
        \[ F r = \{ (x,Y) \,|\, i r x = Y \} \]
        we capitalize $Y$ to highlight the fact that it is a set.
        $G : \Fun{a}{(\Set{b})} \to \Relab{a}{b}$ is the \emph{function relationization} function.
        \[ G f = \{ (x,y) \,|\, y \in f x \} \]
        $G f$ relates $x$ to $y$ iff $y \in f x$.
        We can see that $F(G f) = f$ and $G(F r) = r$.
        Thus $F \circ G$ is the identity of $\Fun{a}{(\Set{b})}$
        and $G \circ F$ is the identity of $\Relab{a}{b}$.
        Thus $F$ and $G$ are inverses of each other.

        ???
    \end{proof}
\end{mthm}

There is a mapping from $\Fun{a}{b}$ to $\Relab{a}{b}$.
There is a bijection between $\Relab{a}{b}$ and $\Relab{b}{a}$.
There is a bijection between $\Relab{b}{a}$ and $\Fun{b}{(\Set{a})}$.
This means that there is a bijection between $\Fun{a}{(\Set{b})}$ and $\Fun{b}{(\Set{a})}$.
