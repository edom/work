\chapter{Number}

\section{Natural number}

\index{natural number}%
The set of all \emph{natural numbers} is \(\Nat\).
The symbol \(\Nat\) is also used for the type of natural numbers.

\section{Peano axioms}

\index{Peano axioms}%
The \emph{Peano axioms} are
\begin{enumerate*}[label={(\arabic*)}]
    \item \(0 : \Nat\),
    \item \(n : \Nat \implies S(n) : \Nat\),
    \item \(S(x) = S(y) \iff x = y\),
    \item the
\index{axiom of induction}%
\index{induction!axiom of}%
\emph{axiom of induction}:
\end{enumerate*}
\[
    \phi(0) \wedge (\forall n \in \Nat : [\phi(n) \implies \phi(S(n))]) \implies (\forall n \in \Nat : \phi(n))
\]
\index{successor}%
where \(S\) is the \emph{successor symbol}.
Thus, \(1 = S(0), ~ 2 = S(1) = S(S(0)), ~ 3 = S(2) = S(S(1)) = S(S(S(0)))\), and so on.

\section{Construction}

\index{Zermelo construction of natural numbers}%
\index{natural number!Zermelo construction}%
Zermelo construction:
\(S(n) = \{n\}\).
Thus, \(0 = \emptyset\), \(1 = \{\emptyset\}\), \(2 = \{\{\emptyset\}\}\), and so on.

\index{von Neumann construction of natural numbers}%
\index{natural number!von Neumann construction}%
Von Neumann construction:
\(0 = \{\}\),
\(S(n) = n \cup \{n\}\).
Thus \(0 \leftrightarrow \{\}\), \(1 \leftrightarrow \{\{\}\}\), \(2 \leftrightarrow \{\{\}, \{\{\}\}\}\), and so on.
Property: \(n \subseteq S(n)\).

\section{Closure under successor}

\(\Nat\) is the smallest set closed under successor.

Assert: \(\Nat = S(\Nat)\).

Equation:
\(\Nat = \Nat \cup \{\Nat\}\).

Consequence:

\(x \in \Nat \iff x \in \Nat \vee x \in \{\Nat\}\)

\(x \in \Nat \iff x = \Nat\)
?

\section{Integer}

\index{integer}%
\index{number!integer}%
\(\Int\) is the set of all \emph{integers}.
\begin{enumerate*}[label={(\arabic*)}]
    \item \(\Nat \subseteq \Int\),
    \item \(\forall x \in \Int : (-x) \in \Int\), and
    \item \(x + (-x) = 0\).
\end{enumerate*}

\(\Int\) is the smallest superset of \(\Nat\)
such that \(\Int\) is closed under negation.

\section{Rational number}

\index{number!rational}%
\index{rational number}%
The set of all \emph{rational numbers} is
\(\Rational = \{ a/b ~|~ a : \Int, b : \Int, b \neq 0 \}\).
Iff \(x : \Rational\) and \(x \neq 0\) then \(x/x = 1\).

\section{Real number}

\index{number!real}%
\index{real number}%
\index{digit}%
\index{base (positional notation)}%
\(\Real\) is the \emph{set of all real numbers}.
The \emph{base-\(b\) positional notation} of a real number
is \(\ldots a_2 a_1 a_0 . a_{-1} a_{-2} \ldots\).
The value of that number is \(\sum_{k \in \Int} a_k b^k\)
where each \(a_k\) is a \emph{digit} and \(b\) is the \emph{base},
where \(b \in \Nat\), \(b > 1\).
\(\Real\) is isomorphic to \(B^\infty\) where \(1 < |B| < \infty\).

\section{Complex number}

\index{complex number}
\(\Complex = \{ a + bi ~|~ a \in \Real, b \in \Real \}\)
is the set of all \emph{complex numbers} where \(i = \sqrt{-1}\).

\index{Euler's identity}%
\emph{Euler's identity} is \( e^{i \pi} + 1 = 0 \).

Let \(z = a+bi = re^{it}\).
\index{complex number!rectangular form}%
\index{rectangular form!of a complex number}%
The \emph{rectangular form} of \(z\) is \(a+bi\).
\index{complex number!polar form}%
\index{polar form!of a complex number}%
The \emph{polar form} of \(z\) is \(re^{it}\).
\index{conjugate!of a complex number}%
\index{complex number!conjugate}%
\index{complex conjugate}%
The \emph{conjugate} of \(z\) is \(\conjbar{z} = \conjstar{z} = a-bi = re^{-it}\).
\index{complex magnitude}%
\index{complex number!magnitude}%
\index{magnitude!of a complex number}%
The \emph{magnitude} of \(z\) is \(\abs{z} = r = \sqrt{z\conj{z}} = \sqrt{a^2+b^2}\).
\index{complex argument}%
\index{complex number!argument}%
\index{argument!of a complex number}%
The \emph{argument} of \(z\) is \(\arg(z) = \{ t ~|~ z = \abs{z} e^{it} \}\).
\index{complex argument!principal value}%
\index{complex number!principal value of argument}%
\index{principal value of the complex argument}%
The \emph{principal value of the argument} of \(z\) is \(\Arg(z)\)
which is the element of \(\arg(z)\) closest to zero.

\section{Sequence}

\index{sequence}%
A \emph{sequence} \(x\) is a list of things \(x_1, x_2, x_3\), and so on.

\section{Series}

\index{series}%
If \(x\) is a sequence, then \(\sum_k x_k\) is a \emph{series}.
The series \emph{converges} iff it is a number.

\section{Prime}

\index{coprime numbers}%
\index{number!coprime}%
Two numbers are \emph{coprime} iff their only common divisor is 1.
\index{prime number}%
\index{number!prime}%
A \emph{prime number} is divisible by 1 and itself only, except 1.

Let \(P\) be the set of all prime numbers.

\paragraph{Nontrivial multiples}
Let \(\{\Nat\ge 2\} = \Nat - \{0,1\}\).
Let \(M(n) = \{ k n : k \in N_2 \}\) be the set of nontrivial multiples of \(n\).
Then \( \Nat - P = \bigcup_{n \in \{\Nat\ge 2\}} M(n) \).
Property: \(M(ab) \supseteq M(a) \cap M(b)\).
Property: \(M(ab) \subseteq M(a)\).
Property: iff \(gcd(a,b)=1\) then \(M(a) \cap M(b) = \emptyset\).

Define \(Q(A) = \bigcup_{a \in A} M(a)\).
Property: \(Q(\emptyset) = \emptyset\).
Property: \(Q(\{a\}) = M(a)\).
Property: \(Q(A \cap B) \subseteq Q(A)\).
Property: \(Q(A \cup B) \supseteq Q(A)\).

Define \(R(n) = \bigcup_{k=2}^n Q(n)\).

Define \(M_n(d) = M(d) \cap \{\Nat \le n\}\).
Define \(Q_n(A) = Q(A) \cap \{\Nat \le n\}\).
Define \(R_n(m) = R(m) \cap \{\Nat \le n\}\).
Property: \(|M_n(d)| \le \lfloor n/d \rfloor\).
Define \(u_n(m) = |R_n(m)| / |\{\Nat \le n\}|\).
Define \(u(m) = \lim_{n \to \infty} u_n(m)\).

Example: \(u(2) = 1/2\), \(u(3) = u(4) = 4/6\), \(u(5) = u(6) = 23/30\), \(u(7) = u(8) = u(9) = u(10) = 86/105\).

\section{Primality and monoid}

The ordered pair \((a,b)\) is a \emph{2-factorization} of \(x\) iff \(ab = x\).
The \emph{2-factorization set} of \(x\) is \(F(x) = \{ (a,b) ~|~ ab = x \}\).

In \(\Nat\):
If \(x \ge 1\) then \(|F(x)|\) is even.
Let \(G(n) = \{ x : |F(x)| = n \}\).

\paragraph{Factor}
Iff \(ab = x\) then
\(a\) is a \emph{left factor} of \(x\),
\(a\) is a \emph{left divisor} \(x\),
\(a\) \emph{left-divides} \(x\),
\(b\) is a \emph{right factor} of \(x\),
\(b\) is a \emph{right divisor} of \(x\),
\(b\) \emph{right-divides} \(x\),
\(a\) is a \emph{factor} of \(x\),
and \(b\) is a \emph{factor} of \(x\).
Iff \(ab = x\) and \(ac = y\),
then \(a\) is a \emph{common left factor} of \(x\) and \(y\).
Iff \(ab = x\) and \(cb = x\),
then \(b\) is a \emph{common right factor} of \(x\) and \(y\).
A \emph{common factor} is a common left factor or common right factor.

\paragraph{Prime}
\(x\) is
\emph{prime} iff \(F(x) = \{(1,x),(x,1)\}\),
\emph{composite} iff it is not prime.

What is the maximum number of primes in a monoid with n elements?
Let the answer be a(n).
a(1)=0. a(2)=1. a(3)=2. a(n)=n-1?

An \emph{ordered monoid} is a monoid and an order \(\le\).

A \emph{monotone monoid} is a monoid where \(a \le ab\) and \(b \le ab\).

The element \(g\) is a
\index{generator}%
\emph{generator} of \((S,\cdot,1)\) iff \(\{g^n ~|~ n \in \Nat\} = S\)
where \(g^0 = 1\) and \(g^{n+1} = g \cdot g^n\).

\section{R2 = R}

There is a bijection between \(\Real^2\) and \(\Real\).
We do this by interleaving the digits.
\[
    (x, y) \leftrightarrow \ldots X_2 Y_2 X_1 Y_1 X_0 Y_0 . x_1 y_1 x_2 y_2 x_3 y_3 \ldots
\]

\section{Collatz conjecture}

Let \(f(n) = n/2\) if \(n\) is even or \(3n+1\) if \(n\) is odd.

\(f(n) = (n \bmod 2) (n/2) + (1 - n \bmod 2) (3n + 1)\).

\(f(n) = (n \bmod 2) (n/2) + (3n + 1) - (n \bmod 2) (3n + 1)\).

\(f(n) = (n \bmod 2) (n/2 - (3n + 1)) + (3n + 1)\).

\(f(n) = (n \bmod 2) (2 - 5n)/2 + (3n + 1)\).

\(2 f(n) = (n \bmod 2) (2 - 5n) + 6n + 2\).

\(f(2n) = n\)

\(f(2n+1) = 3(2n+1)+1 = 6n+4 = 2(3n+2)\)

\(f(2n+2) = n+1\)

\(f(2n+3) = 3(2n+3)+1 = 6n+10 = 2(3n+5)\)

\(f(2n+2k) = n+k\)

\(f(2n+2k+1) = 2(3n+2+3k)\)

\(f(2n+2k+1) = 2(3(n+k)+2)\)

\(f(2(n+k)+1) = 2(3(n+k)+2)\)

\(f(2m+1) = 2(3m+2)\)... back to square 0

Let \(q\) be odd.
\begin{align*}
    f(2^{p+1} q) = 2^p q
    \\
    f(2^0 q) = 3q + 1
    \\
    f(2k+1) = 3(2k+1) + 1 = 6k+4 = 2(3k+2)
    \\
    f(2(2k)+1) = 3(4k+1) + 1 = 12k+4 = 4(3k+1)
    \\
    f(2(2k+1)+1) = 3(4k+3) + 1 = 12k+10 = 2(6k+5)
\end{align*}

\begin{align*}
    f(6n) &= 3n
    \\
    f(6n+1) &= 3(6n+1)+1 = 18n+4
    \\
    f(6n+2) &= 3n+1
    \\
    f(6n+3) &= 3(6n+3) + 1 = 18n+10
    \\
    f(6n+4) &= 3n+2
    \\
    f(6n+5) &= 3(6n+5) + 1 = 18n+16
\end{align*}

Let \(g(n) = f(2n)\) and \(h(n) = f(2n+1)\).

Let \(a(n,0) = n\) and \(a(n,k+1) = f(a(n,k))\).

Corollary: \(a(n,k+1) = a(f(n),k)\).

Corollary: If \(n\) is odd, then \(f(n)\) is even.

\section{Recursion}

Lemma: For every \(n\) and \(d \neq 0\), there exists \(q,d\) such that \(n = qd+m\) and \(0 \le m < d\).

\begin{align*}
    f(2n+0) &= n+0
    \\
    f(2n+1) &= 6n+4 = 2(3n+2)
    \\
    f(4n+0) &= 2n+0
    \\
    f(4n+1) &= 12n+4 = 4(3n+1)
    \\
    f(4n+2) &= 2n+1
    \\
    f(4n+3) &= 12n+10 = 2(6n+5)
    \\
    f(8n+0) &= 4n+0
    \\
    f(8n+1) &= 24n+4 = 4(6n+1)
    \\
    f(8n+2) &= 4n+1
    \\
    f(8n+3) &= 24n+10 = 2(12n+5)
    \\
    f(8n+4) &= 4n+2 = 2(2n+1)
    \\
    f(8n+5) &= 24n+16 = 8(3n+2)
    \\
    f(8n+6) &= 4n+3
    \\
    f(8n+7) &= 24n+22 = 2(12n+11)
    \\
    f(16n+0) &= 8n+0
    \\
    f(16n+1) &= 48n+4 = 4(12n+1)
    \\
    f(16n+2) &= 8n+1
    \\
    f(16n+3) &= 48n+10 = 2(24n+5)
    \\
    f(16n+4) &= 8n+2 = 2(4n+1)
    \\
    f(16n+5) &= 48n+16 = 16(3n+1)
    \\
    f(16n+6) &= 8n+3
    \\
    f(16n+7) &= 48n+22 = 2(24n+11)
    \\
    f(16n+8) &= 8n+4 = 4(2n+1)
    \\
    f(16n+9) &= 48n+28 = 4(12n+7)
    \\
    f(16n+10) &= 8n+5
    \\
    f(16n+11) &= 48n+34 = 2(24n+17)
    \\
    f(16n+12) &= 8n+6 = 2(4n+3)
    \\
    f(16n+13) &= 48n+40 = 8(6n+5)
    \\
    f(16n+14) &= 8n+7
    \\
    f(16n+15) &= 48n+46 = 2(24n+23)
    \\
    f(16n-1) &= 48n-2 = 2(24n-1)
    \\
    f(16n-5) &= 48n-14 = 2(24n-7)
\end{align*}

\begin{align*}
    f^2(4n) &= (gg)(n) = n
    \\
    f^2(4n+1) &= (gh)(n) = 6n+2 = 2(3n+1)
    \\
    f^2(4n+2) &= (hg)(n) = 6n+4 = 2(3n+2)
    \\
    f^2(4n+3) &= (gh)(n) = 6n+5
    \\
    f^3(8n) &= (ggg)(n) = n
    \\
    f^3(8n+1) &= f^2(24n+4) = (ggh)(n) = 6n+1
    \\
    f^3(8n+2) &= f^2(4n+1) = (ghg)(n) = 6n+2 = 2(3n+1)
    \\
    f^3(8n+3) &= f^2(24n+10) = (hgh)(n) = 36n+16 = 4(9n+4)
    \\
    f^3(8n+4) &= f(2n+1) = (ggg)(n) = 6n+4 = 2(3n+2)
    \\
    f^3(8n+5) &= f^2(24n+16) = (ggh)(n) = 6n+4 = 2(3n+2)
    \\
    f^3(8n+6) &= f^2(4n+3) = (hgh)(n) = 6n+5
    \\
    f^3(8n+7) &= f^2(24n+22) = f(12n+11) = (ghg)(n) = 36n+34 = 2(18n+17)
    \\
    f^p(2^p q+0) &= f^{p-1}(2^{p-1} q+0)
    \\
    f^p(2^p q+1) &= f^{p-1}(3 \cdot 2^{p-1} q+4)
\end{align*}

Lemma: \(f^3(8n+4) = f^3(8n+5)\).

Let \(g(n) = n/2\) and \(h(n) = 3n+1\).

\section{Equation}

\(2^p q = 3 n + 1\) where \(q\) is odd.

\(2^{p_0} q_0 = 3 n_0 + 1\) where \(q\) is odd.

\(2^{p_1} q_1 = 3 n_1 + 1\) where \(q\) is odd.

\(n_{k+1} = 2^{p_k} q_k\)

\section{Sequence}

Consider three mutual sequences \(p\), \(q\), and \(n\).

Input: \(q_0\).

Constraint: \(q_k\) odd.

\begin{align*}
    n_0 &= q_0
    \\
    n_{k+1} &= 3 q_k + 1
    \\
    n_k &= 2^{p_k} q_k
\end{align*}

Collatz conjecture: For every odd \(q_0 \in \Nat\), there exists \(k\) such that \(q_k = 1\).

Alternative: For every odd \(a\), there exists \(k\) and odd \(q_0\) such that \(q_k = a\).

Lemma: If \(n\) is even, then \(3n+1\) is odd.

Lemma: If \(n\) is odd, then \(3n+1\) is even.

\section{Composition}

\(u(n) = 1\) where \(u \in (f|g)^*\)

\(3(2n)+1 = 3(2n+1)-2\)

\section{Composition}

\((h \circ g)(n) = 3(n/2) + 1 = (3n+2)/2\)

\((g \circ h)(n) = (3n + 1)/2 = (3n+1)/2\)

\((h \circ g - g \circ h)(n) = 1/2\)

\section{Path}

Let \(P(a,b)\) mean that there is a path from \(a\) to \(b\) in the forward Collatz graph.

(Domain of discourse is \(\Nat\).
All unbound variables are universally quantified.)

Corollary: If \(P(f(a),b)\) then \(P(a,b)\).

Corollary: If \(n\) is odd and \(P(3n+1,a)\) then \(P(n,a)\).

Corollary: \(P(6n+4,a)\) then \(P(2n+1,a)\).

Corollary: If \(P(n,a)\) then \(P(2n,a)\).

Corollary: If \(n \bmod 3 = 1\) and \(P(n,a)\) then \(P(\frac{n-1}{3},a)\).

Lemma: There is no \(n\) such that \(n \bmod 2^p = 1\) for all \(p > 0\).

Lemma: \(P(2^p n, n)\).

Corollary: \(P(2^p,1)\).

Lemma: If \(P(a,2^p)\) then \(P(a,1)\).

Lemma: If \(P(q, 1)\) then \(P(2^p q,1)\).

Conjecture: If \(P(a,1)\) and \(P(b,1)\) then \(P(ab,1)\)?

Conjecture: If \(P(a,1)\) and \(P(b,1)\) then \(P(a+b,1)\)?

Lemma: If \(P(3n+2,1)\) then \(P(2n+1,1)\).

Lemma: If \(P(3n+1,1)\) then \(P(4n+1,1)\).

Lemma: If \(n\) is odd, then \(f(4n+1) = 4f(n)\).

Lemma: If \(p \ge 2\) then \(f(2^p n + 3) = 2(3 \cdot 2^{p-1} + 5)\).

Suppose \(P(n,1)\). Prove \(P(n+1,1)\).

If \(P(k,1)\) then \(P(2k,1)\).

If \(P(6k+4,1)\) then \(P(2k+1,1)\).

\section{Equivalence}

Lemma: Iff every \(6n+4\) has a path to \(2^p\), then the Collatz conjecture is true.

If \(3n+1 < 4\) and \(3n \bmod 2 = 1\) then \(P(n+1,1)\).

If \(3n+1 < 8\) and \(3n \bmod 4 = 1\) then \(P(3n+1,1)\).

If \(3n+1 < 16\) and \(3n \bmod 8 = 1\) then \(P(3n+1,1)\).

If \(3n+1 < 32\) and \(3n \bmod 16 = 1\) then \(P(3n+1,1)\).

\(\{ (p,N) ~|~ n \in N, ~ 3n \bmod 2^p = 1 \} = \{ (1,\{1,3,5,\ldots\}), (2,\{3,7,11,\ldots\}), (3,\{3,11,19,\ldots\}) \}\)

For all \(n,p\): Statement \(S(p)\) is: if \(3n+1 < 2^{p+1}\) and \(3n \bmod 2^p = 1\) then \(P(3n+1,1)\).

An inductive proof of \(S(p)\) is a proof of the Collatz conjecture.

\section{Representation}

In this section, the domain of discourse is \(\Nat\).

Assume \(n > 0\).
Define \(t(n)\) as the largest \(p\) such that \(n = 2^p q\).
Equivalent:
Define \(t(n)\) as the \(p\) such that \(n = 2^p q\) and \(q\) is odd.

Lemma:
For every positive \(n\), there exists \(p\) and odd \(q\) such that \(n = 2^p q\).

Lemma:
For every positive \(n\), there exists \(p\) and \(q < 2^p\) such that \(n = 2^p + q\).

Lemma: If \(b < 2^a\) then \(t(2^a + b) = \min(a,t(b))\).

Lemma: For every \(n\), there exist \(p,q\) such that \(3(2n+1)+1 = 2^p q\).

Corollary: Iff \(n\) is a power of two, then \(q = 1\).

If \(m = 2^a b\) and \(n = 2^c d\) then \(mn = 2^{a+c} (bd)\).

\begin{align*}
    (2^a b + 2^c d) \bmod 2^p = ?
\end{align*}

\(2^p \cdot 1 + 1 = 2^0 (2^p + 1)\).

If \(k\) is even then \(2^0 (2k + 1) + 1 = 2k + 2 = 2^1 (k+1)\).

\(2^0 (2(2k) + 1) + 1 = 4k+2 = 2^0(2k+1)\)

If \(k\) is even then \(2^0 (2(2k) + 3) + 1 = 4k+4 = 2^2(k+1)\)

\(t(1,2,3,4,5,\ldots) = 0,1,0,2,0,1,0,3,0,1,0,2,0,1,0,4,\ldots\)

\(t(2,4,6,8,10,\ldots) = 1,2,1,3,1,2,1,4,1,2,1,3,1,2,1,5,\ldots\)

\(t(k)\) is the number of trailing zero bits in the binary representation of \(k\)

Lemma: \(t(k) \cdot t(k+1) = 0\)

Lemma: \(t(ab) = t(a) + t(b)\)

If \(t(k) \ge 1\) then \(2^0 (2^p k+1) + 1 = 2^p k + 2 = 2^p (k+1)\)

If \(t(k) \ge 1\) then \(t(2^a k+1) = a\)

\(t(3k) = t(k)\)

Under what circumstances are \(3(2n+1)+1 = 2^p\) (that is \(q=1\))?

\(6n+4 = 2^p\).

\(1 \equiv 2^p \pmod 3\).
Corollary: Iff \(p\) is even then \(2^p+2\) is even and divisible by 3.

Corollary: If \(t(n) > 0\) then \(t(f(n)) = t(n)-1\).

\[
    g(n) =
    \begin{cases}
        \{2n\} & \text{if } n \bmod 3 \in \{0,2\}
        \\
        \{2n, \frac{n-1}{3}\} & \text{if } n \bmod 3 = 1
    \end{cases}
\]

Let \(G(N) = \bigcup_{n \in N} g(n)\).

Let \(H_0 = \{1\}\) and \(H_{k+1} = H_k \cup G(H_k)\).

\(H \sim \{1\}, \{2,1,0\}, \{4,2,1,0\}, \{8,4,2,1,0\}, \{16,8,5,4,2,1,0\}, \{32,16,10,8,5,4,2,1,0\}\)

\(\forall k : H_k \subset H_{k+1}\)

Collatz conjecture: for every \(N \subseteq \Nat - \{1\}\), there exists \(p \in \Nat\) such that \(N \subseteq G^p(\{1\})\).

Conjecture: \(G(\Nat) = \Nat\).

\section{Collatz graph}

Graph \((V,E)\)

\((2n,n) \in E\)

\((2n+1,3(2n+1)+1) \in E\), that is
\((2n+1,6n+4) \in E\)

\((2,1),(4,2),(6,3),(8,4),(10,5),\ldots\)

\((1,4),(3,10),(5,16),(7,22),(9,28),\ldots\)

Prove that there is a path from every positive \(n\) to 1.

Inverse graph

\((n,2n) \in E'\)

\((6n+4,2n+1) \in E'\)

Prove that there is a path from 1 to every positive \(n\).

Can we transform \(f\) into a monotonically increasing function?
