\chapter{Number}

\section{Natural number}

\index{natural number}%
The set of all \emph{natural numbers} is \(\Nat\).
The symbol \(\Nat\) is also used for the type of natural numbers.

\section{Peano axioms}

\index{Peano axioms}%
The \emph{Peano axioms} are
\begin{enumerate*}[label={(\arabic*)}]
    \item \(0 : \Nat\),
    \item \(n : \Nat \implies S(n) : \Nat\),
    \item \(S(x) = S(y) \iff x = y\),
    \item the
\index{axiom of induction}%
\index{induction!axiom of}%
\emph{axiom of induction}:
\end{enumerate*}
\[
    \phi(0) \wedge (\forall n \in \Nat : [\phi(n) \implies \phi(S(n))]) \implies (\forall n \in \Nat : \phi(n))
\]
\index{successor}%
where \(S\) is the \emph{successor symbol}.
Thus, \(1 = S(0), ~ 2 = S(1) = S(S(0)), ~ 3 = S(2) = S(S(1)) = S(S(S(0)))\), and so on.

\section{Zermelo construction}

\index{Zermelo construction of natural numbers}%
\index{natural number!Zermelo construction}%
\(S(n) = \{n\}\).
Thus, \(0 = \emptyset\), \(1 = \{\emptyset\}\), \(2 = \{\{\emptyset\}\}\), and so on.

\section{Closure under successor}

\(\Nat\) is the smallest set closed under successor.

\section{Von Neumann construction}

\index{von Neumann construction of natural numbers}%
\index{natural number!von Neumann construction}%
Define:
\(0 = \{\}\),
\(S(n) = n \cup \{n\}\).

Thus \(0 \leftrightarrow \{\}\), \(1 \leftrightarrow \{\{\}\}\), \(2 \leftrightarrow \{\{\}, \{\{\}\}\}\), and so on.

Property: \(n \subseteq S(n)\).

Assert: \(\Nat = S(\Nat)\).

Equation:
\(\Nat = \Nat \cup \{\Nat\}\).

Consequence:

\(x \in \Nat \iff x \in \Nat \vee x \in \{\Nat\}\)

\(x \in \Nat \iff x = \Nat\)

\section{Integer}

\index{integer}%
\index{number!integer}%
\(\Int\) is the set of all \emph{integers}.
\begin{enumerate*}[label={(\arabic*)}]
    \item \(\Nat \subseteq \Int\),
    \item \(\forall x \in \Int : (-x) \in \Int\), and
    \item \(x + (-x) = 0\).
\end{enumerate*}

\(\Int\) is the smallest superset of \(\Nat\)
such that \(\Int\) is closed under negation.

\section{Rational number}

\index{number!rational}%
\index{rational number}%
The set of all \emph{rational numbers} is
\(\Rational = \{ a/b ~|~ a : \Int, b : \Int, b \neq 0 \}\).
Iff \(x : \Rational\) and \(x \neq 0\) then \(x/x = 1\).

\section{Real number}

\index{number!real}%
\index{real number}%
\index{digit}%
\index{base (positional notation)}%
\(\Real\) is the \emph{set of all real numbers}.
The \emph{base-\(b\) positional notation} of a real number
is \(\ldots a_2 a_1 a_0 . a_{-1} a_{-2} \ldots\).
The value of that number is \(\sum_{k \in \Int} a_k b^k\)
where each \(a_k\) is a \emph{digit} and \(b\) is the \emph{base},
where \(b \in \Nat\), \(b > 1\).
\(\Real\) is isomorphic to \(B^\infty\) where \(1 < |B| < \infty\).

\section{Complex number}

\index{complex number}
\(\Complex = \{ a + bi ~|~ a \in \Real, b \in \Real \}\)
is the set of all \emph{complex numbers} where \(i = \sqrt{-1}\).

\index{Euler's identity}%
\emph{Euler's identity} is \( e^{i \pi} + 1 = 0 \).

Let \(z = a+bi = re^{it}\).
\index{complex number!rectangular form}%
\index{rectangular form!of a complex number}%
The \emph{rectangular form} of \(z\) is \(a+bi\).
\index{complex number!polar form}%
\index{polar form!of a complex number}%
The \emph{polar form} of \(z\) is \(re^{it}\).
\index{conjugate!of a complex number}%
\index{complex number!conjugate}%
\index{complex conjugate}%
The \emph{conjugate} of \(z\) is \(\conjbar{z} = \conjstar{z} = a-bi = re^{-it}\).
\index{complex magnitude}%
\index{complex number!magnitude}%
\index{magnitude!of a complex number}%
The \emph{magnitude} of \(z\) is \(\abs{z} = r = \sqrt{z\conj{z}} = \sqrt{a^2+b^2}\).
\index{complex argument}%
\index{complex number!argument}%
\index{argument!of a complex number}%
The \emph{argument} of \(z\) is \(\arg(z) = \{ t ~|~ z = \abs{z} e^{it} \}\).
\index{complex argument!principal value}%
\index{complex number!principal value of argument}%
\index{principal value of the complex argument}%
The \emph{principal value of the argument} of \(z\) is \(\Arg(z)\)
which is the element of \(\arg(z)\) closest to zero.

\section{Sequence}

\index{sequence}%
A \emph{sequence} \(x\) is a list of things \(x_1, x_2, x_3\), and so on.

\section{Series}

\index{series}%
If \(x\) is a sequence, then \(\sum_k x_k\) is a \emph{series}.
The series \emph{converges} iff it is a number.

\section{Prime}

\index{coprime numbers}%
\index{number!coprime}%
Two numbers are \emph{coprime} iff their only common divisor is 1.
\index{prime number}%
\index{number!prime}%
A \emph{prime number} is divisible by 1 and itself only, except 1.
