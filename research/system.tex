\chapter{Systems}

This chapter defines \emph{system}.
Later chapters discuss interesting systems.
We classify systems, hoping to gain some insight.
We can classify systems into two big classes:
\emph{time-dependent} and \emph{time-independent}.
Time-dependent systems are big enough
so we put it in the next chapter.

\section{What is a system?}

We define a system as an input, a state, and an output.
The input is \(x\), the output is \(y\), and an equation relates them.
The state is implied by the equation.
Such equation can be written \(f~y~x = y\).
The equation can be quite arbitrary.
The terms \(f,x,y\) may appear on both sides of the equation.

\begin{m:def}[System]
A system is \((x,y,f)\) where \(y=f~y~x\).
\end{m:def}

\section{What equation describes a system?}

\section{Invariant}

An invariant of a system is a property that stays
the same throughout the evolution of the system.

\section{What is the behavior of a system?}

The behavior of a system is its output, especially the observable part of the output.

\section{Composition}

\section{Continuous system}

\section{Discrete system}

\section{Finite system}

\section{What is an embedded system?}

A system in another system.
The outer system feeds the inner system's output back to the inner system's input,
possibly with some change.

Don't confuse this with embedded systems in computer engineering.

\section{How do we measure system complexity?}

\section{Ignoring degenerate feedback: feedforward}

Every function \(f\) is a special case of the general feedback equation \(f~x = g~f~x\)
where \(g\) is an identity function.
This suggests that feedforward is a degenerate case of feedback.
To simplify the writing, from this point on,
we always assume that a feedback is non-degenerate
unless written otherwise.

\section{Finding feedback: the inverse fixed point problem}

Given \(f\), find a \(g\) such that \(f~x = g~f~x\) and \(g\) is not an identity function.

The forward fixed point problem:
Given \(f\), find an \(x\) such that \(x=f~x\).

The inverse fixed point problem is
``Given \( x \), find an \( f \) such that \( x = f~x \) and \(f\) is not an identity function.''
This problem arises when we want to determine
if \(f\) has a feedback.

Example of non-feedback: linear functions.
Consider a function of the form \(f~x = a \cdot x + b\) where \(a\) and \(b\) are non-zero constants.
The only \(g\) that satisfies \(g~f = f\) is the identity function \(g~x=x\).

Example of feedback: functional equation.
Consider a function of the form \(f~x = x \cdot f~(x-1)\).

Recursive functions are special cases of feedback.
Searching in list.
\(f~N~e = 0\).
\(f~(C~h~t)~e = h \equiv e \vee f~t~e\).
\(g\) is the Y-combinator.

We have a problem: there are infinitely many wildly discontinuous functions satisfying that.
We want smooth functions.

\section{Feedback based on differentiability-preserving map}

We want a map that preserves differentiability.
Formally, given \(f=g~f\), we want \(g\) to have the property
that iff \(f\) is differentiable then \(g~f\) is also differentiable.
Surely if \(f\) is differentiable and \(g\) is differentiable then \(g~f\) is also differentiable?
Surely if \(f\) is differentiable \(g\) is a polynomial then \(g~f\) is differentiable?

We begin with the generalized differential on a field:
\( g~(f+h) = g~f + h \cdot d~g~f \)
where \( (f + g)~x = f~x + g~x \) and \( (f \cdot g)~x = f~x \cdot g~x \).
Thus \( h \cdot d~g~f = g~(f+h) - g~f \).
This is like computing the gradient of a vector function,
but the vector is infinite-dimensional.

Is it time to learn topology? Smooth manifolds?

\section{Measuring feedback}

Given a system \( f~x = g~f~x \),
we're interested in measuring how much feedback it has.

Assume that \(f\) is a vector.
We can measure the feedback by measuring \( d_f ~ g \):
the differential of \(g\) with respect to \(f\).
Using non-standard analysis, we define the gradient \( d~f \)
as something satisfying \( f~(x + h) = f~x + h \cdot d~f~x \)
where \(h\) is an infinitesimal.

\section{Linear feedback and function classes}

If \(f\) is linear and \(g\) is linear, then \(f \circ g\) is linear.
A linear feedback does not add anything interesting to a linear function.
