\chapter{Feedback gives rise to fixed point equation.}

What happens if we feed the output
of a function \(f\) back to its input?
But firstly, what do we mean by feeding it back?
We mean we wire the input and the output together so that they are the same,
so that the function's input \(x\) is the same as the function's output \(f~x\).
That statement immediately translates to the equation \( x = f~x \),
which is a \emph{fixed point equation}.
Therefore equalizing the output of a function to its input gives rise to a fixed point equation.

But feedback is more general than that.
Feedback is the case where the input also depends on the output.
If the input is \(x\) and the output is \(f~x\),
we can express feedback as the equation \( x = g~(f~x)~x \),
which we can generalize to \( x = g~f~x \)
by letting \(g\) apply \(x\) to \(f\) internally.
We shall call the equation \( x = g~f~x \) the \emph{general feedback equation}.

But it turns out that we can always write \( x = g~f~x \) as \( x = h~x \)
where \( h = g~f \).
We can always write \( f~x = x \) as \( g~x = 0 \) where \( g~x = f~x - x \).
We can also always write \( x = g~f~x \) as \( 0 = g~f~x \).
What about squared fixed point equation \( (f~x)^2 - x^2 = 0 \)?
The equation \( g~x = 0 \) doesn't mean that \(g\) is zero everywhere.
It means that \(g\) is zero at \(x\), and nothing more, unless quantified.

A fixed point equation is an example of functional equations.
Other examples are Cauchy's functional equations.

\section{Feedback is a special case of fixed point.}

The \emph{general fixed point equation} is \(x = f ~ x\).
In this case, we say that \(x\) is a fixed point of \(f\).

\begin{m:thm}[Feedback as special case of fixed point]
The general feedback equation is a special case of the general fixed point equation.
\begin{proof}
Start with the general feedback equation \( f ~ x = g ~ f ~ x \).
Do eta conversion, producing \( f = g ~ f \).
Then, rename \(f\) to \(x\) and rename \(g\) to \(f\),
producing \( x = f ~ x \),
which is the general fixed point equation.
\end{proof}
\end{m:thm}

That theorem relates cybernetics and fixed point theory.
Thus a feedback is a fixed point of a higher-order function
in the sense that if \(f = g f\), then \(f\) is a fixed point of \(g\).

\begin{m:lem}
    If \( f~x = g~x \) for all \(x\), then \(f = g\).
    \begin{proof}
        We can prove this using set theory.
        A function is an injective relation.
        A relation is a set of ordered pairs.
    \end{proof}
\end{m:lem}

Generally, the equation \(f~x = x\) doesn't hold for all \(x\),
because if it does, \(f\) becomes the identity function.
There are two points of view:
you can fix \(f\) and let \(x\) vary, or you can fix \(x\) and let \(f\) vary. 

The set of fixed points of \(f\) is \(\sbn{x}{x = f~x}\).

\begin{m:lem}
    If \( x = f~x \) for all \(x\),
    then \(f\) is an identity function.
\begin{proof}
    The step from the second to the third equation
    comes from the previous lemma.
    \begin{align*}
        x &= f~x & \text{from the antecedent of this lemma}
        \\
        \fid~x &= f~x & \text{due to the antecedent of this lemma}
        \\
        \fid &= f
    \end{align*}
\end{proof}
\end{m:lem}

\section{Recurrence relation is an example feedback.}

Consider the function \( f~x = y \).
If we know \(x\), we can compute \(y\).
If we know \(y\), we can limit the possibilities of \(x\).
They depend on each other.
But they aren't feedback.
Why?
What do we mean by `depend'?

Usually we can rewrite a recurrence relation to a fixed point equation.
For example, we can rewrite \( x_k = f~x_{k-1} \)
to another equation \( x = f \circ (d~x) \),
and then to \( x = (f \circ d)~x \),
where \( d \) is the discrete unit delay function,
which is defined as \( d~x~k = x~(k-1) \).
Recurrence relations describe feedback.
Difference equation is a special case of recurrence relations.
Difference equation is a discrete version of differential equation.

\[
    f~x = c \cdot (s - x)
\]

\section{Closed-loop control is feedback.}

A control system has input signal \(x\), control signal \(y\), output parameter \(z\),
control function \(c\),
response function \(r\),
and follows the equations
\( z~t = r~z~y~t \)
and
\( y~t = c~y~x~t \)
where \(t\) represents time.
We use two equations to model the case
that changing the control signal
does not instantaneously change the output parameter.
The control function affects the output parameter indirectly through the control signal.
The control function cannot directly change the output parameter.
The response function models physical limitations.
\begin{align*}
    z &= r~z~y
    \\ y &= c~y~x
\end{align*}
which can also be stated as the vector feedback equation
\begin{align*}
    \bmat{z \\ y} &= \bmat{r~z~y \\ c~y~x}
\end{align*}

That system of equations suggests that a control system is just two feedback loops.
A control system is a system of feedback equations.
However, system of feedback equations is still a special case of the general feedback equation.

\section{Is this a general feedback equation?}

An equation \(f~x = y\) is a feedback equation iff \(f\) appears in \(y\),
that is if \(f\) appears in both sides of the equation.
A feedback equation is just another name for functional equation.
A feedback equation describes a subset of implicit functions.

The \emph{general feedback equation} is \(f ~ x = g ~ f ~ x\).
It can model every feedback system.
We give the name \emph{system function} to \(f\)
and \emph{wiring function} to \(g\).

An \emph{open} system or a \emph{pure feedforward} system
is a system whose \(g\) does not use \(f\).

\section{What are examples of feedbacks?}

The differential equation \( f~x = d~f~x + d~(d~f)~x \)
is a special case of the general feedback equation \( f~x = g~f~x \)
with \( g~f~x = d~f~x + d~(d~f)~x \).
We can also eta-convert \(g\) to \( g = d + d^2 \).

The Fibonacci equation \( f~x = f~(x-1) + f~(x-2) \)
is a special case of the general feedback equation \( f~x = g~f~x \)
with \( g~f = f \circ (s~1) + f \circ (s~2) \)
where \(s~a~b = b - a\) is flipped subtraction.

Indeed all differential equations and difference equations
are special cases of the general feedback equation.

PID (proportional-integral-derivative)
control systems are special cases of the general feedback equation
with \(g~f~x = c_p \cdot x + c_i \cdot i~f~0~x + c_d \cdot d~f~x\).

IIR (infinite impulse response) filters are special cases of the general feedback equation.

Functional equations are special cases of the general feedback equation.

Thermostats are special cases of the general feedback equation.

Organisms, organizations, and planets are
also special cases of the general feedback equation.

\section{What are positive and negative feedback?}

We say that a system \( f~x = g~f~x \) has positive feedback iff \( d~f~x > 0 \).
We say that it has a negative feedback iff \( d~f~x < 0 \).

\section{What is feedback from dynamical systems of view?}

According to \r{A}str\"om and Murray \cite{AstromMurrayFeedback}:
\begin{quote}
A \emph{dynamical system} is a system whose behavior changes over time, often in response
to external stimulation or forcing. The term \emph{feedback} refers to a situation in which
two (or more) dynamical systems are connected together such that each system
influences the other and their dynamics are thus strongly coupled.
\cite[p. 1]{AstromMurrayFeedback}
\end{quote}

\section{What are examples of fixed points?}

Everything is a fixed point of the identity function \(\fb{x}{x}\).
The constant \(c\) is a fixed point of the constant function \(\fb{x}{c}\).
The function \(\exp\) is a fixed point of the real derivative operator \(d\)
because \(d ~ \exp = \exp\).

Consider \(d^4\), the real derivative operator applied four times.
The function \(\exp\) is a fixed point of \(d^4\).
The sine function is another fixed point.
The cosine function is yet another fixed point.

Eigenvalue-eigenvector pairs are related to fixed points.
If \(a \cdot x = c \cdot x\) where \(a\) is a matrix, \(x\) is a vector, and \(c\) is a scalar,
then \(x\) is a fixed point of \(g\) where \(g~x = (a \cdot x) / c\).
