\chapter{Foundation}

If you are already familiar with the things here,
please at least skim this chapter to verify that
we are speaking the same language.

\section{Type}

\index{type}%
\(x : T\) means that the \emph{type} of \(x\) is \(T\).
\index{function type}%
\index{type!function}%
A \emph{function type} looks like \(a \to b\).

Function application:
Iff \(x:A\) and \(f:A\to B\), then \(f(x):B\).

Types and sets are different:
\(x \in A\) and \(x : A\) are different.

\section{Logic}

\index{iff}%
\index{if and only if}%
``Iff'' is ``if and only if''.
It is bi-implication.

\index{connectives}%
\index{logical connectives}%
Logical connectives:
\index{conjunction}%
\index{and}%
\emph{Conjunction} is \(a \wedge b\) (\(a\) and \(b\)).
\index{disjunction}%
\index{or}%
\emph{Disjunction} is \(a \vee b\) (\(a\) or \(b\)).
\index{implication}%
\emph{Implication} is \(a \implies b\) (if \(a\) then \(b\)).
\index{bi-implication}%
\emph{Bi-implication} is \(a \iff b\) (iff \(a\) then \(b\)).
\index{negation}%
\index{not}%
\emph{Negation} is \(\neg a\) (not \(a\)).

\section{Set}

\index{set}%
Practically, a \emph{set} is an unordered collection of unique things.%
\index{Russell's paradox}%
\footnote{Theoretically, a ``set'' cannot be defined this way due to Russell's paradox.}
If \(a \neq b\), then the notations \(\{a,b\}\), \(\{b,a\}\), and \(\{a,a,b\}\)
all describe the same set of two elements.
The \emph{cardinality} of a set \(A\), written \(|A|\),
is the number of elements of \(A\).

\index{set builder notation}%
\index{notation!set builder}%
The \emph{set builder notation}
\( \{ M ~|~ F \} \)
describes the set of every \(M\) where \(F\) is true
where \(M\) is an expression and \(F\) is a logic formula.
For example, the set \(\{ x^2 ~|~ x \in \Nat \}\) is \(\{ 0^2, 1^2, 2^2, \ldots \}\).
The ellipsis (\ldots) is not an element of the set;
there are just too many things to write.

Set relationships:
\index{subset}%
\(A \subseteq B\)
(\(A\) is a \emph{subset} of \(B\))
iff \(\forall x (x \in A \implies x \in B)\).
\index{disjoint sets}%
\index{sets!disjoint}%
Two sets \(a\) and \(b\) are \emph{disjoint} iff \(a \cap b = \emptyset\).

Set operations:
\index{intersection}%
\index{intersection  (set theory)}%
The \emph{intersection of \(A\) and \(B\)} is
\(A \cap B = \{ x ~|~ x \in A \wedge x \in B \}\).
\index{union}%
\index{union (set theory)}%
The \emph{union of \(A\) and \(B\)} is
\(A \cup B = \{ x ~|~ x \in A \vee x \in B \}\).
\index{subtraction (set theory)}%
\index{sets!subtraction}%
The \emph{subtraction of \(A\) and \(B\)} is
\(A - B = \{ x ~|~ x \in A \wedge x \not\in B \}\).
\index{Cartesian product}%
The \emph{Cartesian product of \(A\) and \(B\)} is \(A \times B = \{ (a,b) ~|~ a \in A, b \in B \}\).
The \emph{Cartesian product of \(A_1,\ldots,A_n\)} is
\(\prod_{k=1}^n A_k = A_1 \times \ldots \times A_n = \{ (a_1,\ldots,a_n) ~|~ a_1 \in A_1, ~\ldots, ~a_n \in A_n \}\).
\index{power set}%
\index{set!power}%
The \emph{power set} of \(A\) is \(2^A = \{ X ~|~ X \subseteq A \}\).

\section{Set partitioning}

\(A_1,\ldots,A_n\) is a \emph{partitioning of \(A\)} iff
\(\bigcup_{k=1}^n A_k = A\) and
\(\forall i \neq j (A_i \cap A_j = \emptyset)\).

\section{Relation}

\index{relation}%
A \emph{relation} is a subset of a Cartesian product.
\index{binary relation}%
\index{relation!binary}%
\(R\) is a \emph{binary relation} between \(A\) and \(B\) iff \(R \subseteq A \times B\).
Iff \(A = B\), then \(R\) is an \emph{endorelation}.

Properties of an endorelation:
\index{reflexive relation}%
\index{relation!reflexive}%
An endorelation \(R \subseteq A \times A\) is \emph{reflexive} iff
\(\forall x \in A : (x,x) \in R\),
\index{symmetric relation}%
\index{relation!symmetric}%
is \emph{symmetric} iff
\(\forall x,y: (x,y) \in R \iff (y,x) \in R\),
\index{antisymmetric relation}%
\index{relation!antisymmetric}%
is \emph{antisymmetric} iff
\(\forall x,y: (x,y) \in R \wedge (y,x) \in R \implies x = y\),
\index{transitive relation}%
\index{relation!transitive}%
is \emph{transitive} iff
\(\forall x,y,z: (x,y) \in R \wedge (y,z) \in R \implies (x,z) \in R\),
\index{total relation}%
\index{relation!total}%
is \emph{total} iff
\(\forall x \in A , y \in A : (x,y) \in R \vee (y,x) \in R\).

\index{equivalence relation}%
\index{relation!equivalence}%
An \emph{equivalence relation} is a symmetric transitive reflexive relation.
\index{partial order}%
\index{total order}%
\index{order!partial}%
\index{order!total}%
A \emph{partial order} is a antisymmetric transitive reflexive relation.
A \emph{total order} is an antisymmetric transitive total relation.

\section{Function}

\index{function}%
A \emph{function} \(f\) is a relation such that
\(\forall a, b, c : (a,b) \in f \wedge (a,c) \in f \implies b = c\).
The type of a function is \(a \to b\).
If \(f\) is a function, then \(f~a = b\) iff \((a,b) \in f\).
The type of a relation between \(a\) and \(b\) is \(a \to 2^b\).

\section{Indicator function}

\(f(x) = [F]\) means \(f(x) = 1\)
iff \(F\) is true
and \(0\) iff \(F\) is false.

Example: \(f(x) = [x \in [0,1]]\)

\section{Unnamed function}

\(f(x) = x + 1\) and \(f = x \to x+1\) mean the same thing.
When an expression is expected, the notation \(X \to Y\)
means a function that evaluates to \(Y\) if given \(X\).
When a type is expected, the notation \(X \to Y\) describes a function type.
Expressions like \(x \to x + 1\) are called \emph{unnamed functions}.

\section{Currying}

\index{curry}%
\index{currying}%
\emph{Currying} \(f : (a_1,\ldots,a_n) \to b\) turns it into \(f' : a_1 \to \ldots \to a_n \to b\)
such that \(f(x_1,\ldots,x_n)
= f'~x_1~\ldots~x_n
= f'(x_1)(x_2)\ldots(x_n)
= (((f'~x_1)~x_2)\ldots)~x_n
= (((f'(x_1))(x_2))\ldots)x_n\).
In this book, all functions are curried.

\section{Order}

Limit

Minimum

Maximum

Extremum

Infimum, greatest lower bound

Supremum, least upper bound

\index{argmin}%
Define \(m = \arg\min_{x \in X}(f~x)\) iff \(\forall x \in X : f~m \le f~x\).

\index{min}%
Define \(\min_{x \in X}(f~x) = f~(\arg\min_{x \in X}(f~x))\).

\section{Numbers}

\index{natural number}%
The set of all \emph{natural numbers} is \(\Nat\).
\index{Peano axioms}%
The \emph{Peano axioms} are
\begin{enumerate*}[label={(\arabic*)}]
    \item \(0 \in \Nat\),
    \item \(\forall n (n \in \Nat \implies S(n) \in \Nat)\), and
    \item nothing else is in \(\Nat\).
\end{enumerate*}
\index{successor function}%
\(S\) is the \emph{successor function}.
Thus \(1 = S(0), ~ 2 = S(1) = S(S(0)), ~ 3 = S(2) = S(S(1)) = S(S(S(0)))\), and so on.

\(\Nat\) is also the type of natural numbers.

\index{integer}%
\index{number!integer}%
The set of all \emph{integers} is \(\Int\).
\begin{enumerate*}[label={(\arabic*)}]
    \item \(\Nat \subseteq \Int\),
    \item \(x : \Int\) iff \((-x) : \Int\), and
    \item \(x + (-x) = 0\).
\end{enumerate*}

\index{number!rational}%
\index{rational number}%
The set of all \emph{rational numbers} is
\(\Rational = \{ a/b ~|~ a : \Int, b : \Int, b \neq 0 \}\).
Iff \(x : \Rational\) and \(x \neq 0\) then \(x/x = 1\).

\index{number!real}%
\index{real number}%
\index{digit}%
\index{base (positional notation)}%
\(\Real\) is the \emph{set of all real numbers}.
The \emph{base-\(b\) positional notation} of a real number
is \(\ldots a_2 a_1 a_0 . a_{-1} a_{-2} \ldots\).
The value of that number is \(\sum_{k \in \Int} a_k b^k\)
where each \(a_k\) is a \emph{digit} and \(b\) is the \emph{base},
where \(b \in \Nat\), \(b > 1\).
\(\Real\) is isomorphic to \(B^\infty\) where \(1 < |B| < \infty\).

\index{complex number}
\(\Complex = \{ a + bi ~|~ a \in \Real, b \in \Real \}\)
is the set of all \emph{complex numbers} where \(i = \sqrt{-1}\).

\index{Euler's identity}%
\emph{Euler's identity} is \( e^{i \pi} + 1 = 0 \).

Let \(z = a+bi = re^{it}\).
\index{complex number!rectangular form}%
\index{rectangular form!of a complex number}%
The \emph{rectangular form} of \(z\) is \(a+bi\).
\index{complex number!polar form}%
\index{polar form!of a complex number}%
The \emph{polar form} of \(z\) is \(re^{it}\).
\index{conjugate!of a complex number}%
\index{complex number!conjugate}%
\index{complex conjugate}%
The \emph{conjugate} of \(z\) is \(\conjbar{z} = \conjstar{z} = a-bi = re^{-it}\).
\index{complex magnitude}%
\index{complex number!magnitude}%
\index{magnitude!of a complex number}%
The \emph{magnitude} of \(z\) is \(\abs{z} = r = \sqrt{z\conj{z}} = \sqrt{a^2+b^2}\).
\index{complex argument}%
\index{complex number!argument}%
\index{argument!of a complex number}%
The \emph{argument} of \(z\) is \(\arg(z) = \{ t ~|~ z = \abs{z} e^{it} \}\).
\index{complex argument!principal value}%
\index{complex number!principal value of argument}%
\index{principal value of the complex argument}%
The \emph{principal value of the argument} of \(z\) is \(\Arg(z)\)
which is the element of \(\arg(z)\) closest to zero.

\section{Metric}

\index{distance}%
\index{distance!between hyperplane and point}%
Let \(\dist\) stand for \emph{distance}.
If \(h\) is a hyperplane,
then \(\dist~h~X = \min_{x \in X} (\dist~h~x)\)
where \(\dist~h~x = |x-q|\) where \(q\) is on \(h\) and \(x-q\) is parallel to \(n\).
\index{metric}%
A \emph{metric} is...

\section{Norm}

\index{norm}%
A \emph{norm} is...
\index{normed space}%
\index{space!normed}%
A \emph{normed space} is...
\index{metric space}%
\index{space!metric}%
A \emph{metric space} is...
\index{Euclidean space}%
\index{space!Euclidean}%
An \emph{Euclidean space} is ...
\index{pre-Hilbert space}%
\index{space!pre-Hilbert}%
A \emph{pre-Hilbert space} is ...
\index{Hilbert space}%
\index{space!Hilbert}%
A \emph{Hilbert space} is ...
Usually we can pretend that a Hilbert space is an infinite-dimensional Euclidean space.
%A Hilbert space is an infinite-dimensional vector space with inner product,
%that is also a complete metric space with respect to the distance function induced by the inner product.
% TODO rephrase using pre-Hilbert space

\index{Banach space}%
\index{space!Banach}%
A \emph{Banach space} is a complete normed vector space.

\section{Measure}

A \emph{Jordan content} is...

\index{content (measure theory)}%
A \emph{content} \(m : 2^A \to \Real\) satisfies
\begin{enumerate*}[label={(\arabic*)}]
    \item \(\forall x \in A : m~x \ge 0\),
    \item \(m~\emptyset = 0\), and
    \item \(\forall X \subseteq A, Y \subseteq A, X \cap Y = \emptyset : m~(X \cup Y) = m~X + m~Y\).
\end{enumerate*}

\index{measure}%
\index{countably additive}%
\index{countable additivity}%
A \emph{measure} \(m\) on a set \(A\) is a content that is \emph{countably additive}.

\index{Lebesgue measure}%
\index{measure!Lebesgue}%
A \emph{Lebesgue measure} for \(\Real\) is
\(m~(a,b) = m~[a,b) = m~(a,b] = m~[a,b] = |b-a|\).
From a measure \(m : \Real \to \Real\)
we can define a measure \(\mu : \Real^n \to \Real\)
where
\(\mu \left( \prod_{i=1}^n X_i \right) = \prod_{i=1}^n m(X_i)\).
\index{measure space}%
\index{measure!space}%
\index{space!measure}%
A \emph{measure space} is a set and a measure on that set.

\index{measurable function}%
\index{function!measurable}%
A \emph{measurable function} is ...

\section{Integral}

\index{integral}%
\index{integral!Lebesgue}%
\index{integral!Riemann}%
\index{Lebesgue integral}%
\index{Riemann integral}%
Let \((\Real,m)\) be a measure space and \(f\) be a function.
Let \(X \subseteq \Real\) and \(Y = \{ f(x) ~|~ x \in X\}\).
Let \(X_1,\ldots,X_n\) be a partitioning of \(X\) and \(Y_1,\ldots,Y_n\) be a partitioning of \(Y\).
For each \(k\), let \(x_k \in X_k\) and \(y_k \in Y_k\).
An \emph{integral of \(f\) in \(X\)}, written \(\int_X f\),
is \(\lim_{n\to\infty} \sum_{k=1}^{n} a_k\)
where each \(a_k = m(X_k) \cdot m(Y_k)\) is a rectangular part of the total area.
See Table \ref{tab:integral}.

\begin{table}[h]
    \caption{How integrals partition spaces}
    \label{tab:integral}
    \centering
\begin{tabular}{lll}
    Name & \(X_k\) & \(Y_k\)
    \\
    \hline
    Riemann integral & \( [x_k,x_{k+1}]\) & \( [0,f(x_k)]\)
    \\
    Stieltjes integral with respect to \(g\) & \( [g(x_k),g(x_{k+1})]\) & \( [0,f(x_k)]\)
    \\
    Lebesgue integral & \( \{x ~|~ f(x) \in Y_k\}\) & \( [y_k,y_{k+1}]\)
\end{tabular}
\end{table}

\section{Calculus}

\index{fundamental theorem of calculus}%
\emph{Fundamental theorem of calculus}:
Iff \(D(F) = f\), then \(I(f) = F + c\) where \(c\) is a constant function,
\(D\) is the derivative operator, \(I\) is the antiderivative operator.

Derivative
