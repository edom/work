% xelatex: use "\ " instead of "~", and use "\textendash" instead of "--"
\usepackage{microtype}
% mathematics
\usepackage{amsmath}
\usepackage{amsfonts}
% amsthm must be loaded before newtxmath due to \openbox conflict
\usepackage{amsthm}
%\usepackage{amssymb}
% no need to load amsfonts and amssymb if using mathdesign (included via format-lato)
% <fonts>
% Latin Modern is thinner than Libertine.
% Libertine is thinner than New Century (fouriernc).
% See also http://www.icl.utk.edu/~mgates3/docs/latex-fonts.pdf
%\usepackage{lmodern}
% URW Nimbus Sans: baseline of small I and L don't line up
%\usepackage[scaled]{helvet}
%\renewcommand*\familydefault{\sfdefault}
\let\OriginalSfdefault=\sfdefault% newtx clobbers sfdefault
\usepackage{newtxtext}
\usepackage{newtxmath}
%\usepackage{newpxtext}
%\usepackage{newpxmath}
\let\sfdefault=\OriginalSfdefault
%% http://www.tug.dk/FontCatalogue/erewhon/
% (not in ubuntu 14.04)
\usepackage[proportional,scaled=1.064]{erewhon}
\usepackage[erewhon,vvarbb,bigdelims]{newtxmath}
\renewcommand*\oldstylenums[1]{\textosf{#1}}

%\usepackage{libertine}
\usepackage[libertine]{newtxmath}

%\usepackage{fouriernc}
%\usepackage{fourier}
% sans-serif fonts
%\usepackage[default,scale=0.96875]{lato}
%\usepackage[default]{lato}
%\usepackage[sfdefault]{quattrocento}
%\usepackage[sfdefault]{cabin}
%\usepackage[sfdefault]{cabin}
%\usepackage[default,scaled=0.96875]{roboto}
%\usepackage[bitstream-charter,scaled=1.03125]{mathdesign}
%\usepackage[default,scaled=0.9375]{roboto}
% <main-font>
% we pick Computer Modern Bright because it is sans-serif and it has math support
%\usepackage{lmodern}%thicker than Computer Modern
%\usepackage{cmbright}
% arev has bold-italic; cmbright doesn't have bold-italic
%\usepackage{arev}%the other sans-serif fonts with math support
%\linespread{1.15}% expand line spacing to improve readability of fonts with high x-height; arev is one such font
%\renewcommand{\sfdefault}{cmbr}% Computer Modern Bright, a sans-serif font
%\renewcommand{\familydefault}{\sfdefault}
\usepackage[T1]{fontenc}
%\usepackage{fontspec}
%\usepackage{unicode-math}
% <workaround>
% https://tex.stackexchange.com/questions/85696/what-causes-this-strange-interaction-between-glossaries-and-amsmath
%\usepackage{mathspec}
%\makeatletter % undo the wrong changes made by mathspec
%\let\RequirePackage\original@RequirePackage
%\let\usepackage\RequirePackage
%\makeatother
% </workaround>
%\setmainfont[Scale=1]{Arial}
%\setmainfont[Scale=1]{Arimo}
%\setmainfont[Scale=1]{Lato}
%\setmainfont[Scale=1]{Nimbus Sans L}%thinner than Arial
%\setmainfont[Scale=1]{TeX Gyre Heros}%enhanced Nimbus Sans
%\setmainfont[Scale=1]{Liberation Sans}%comes with Ubuntu, but thinner than Nimbus Sans
%\setmainfont[Scale=1]{Roboto}
%\setmainfont[Scale=1]{CMU Sans Serif}
%\setmainfont[Scale=1]{Latin Modern Sans}
% </main-font>
%\setmonofont[Scale=0.9375]{Liberation Mono}
%\setmathfont{Latin Modern Math}
%\DeclareMathSizes{12}{13}{9.75}{7.3125}%tfs, ms, ss, sss; tfs = 12; R = 3/4; R*ms = ss; R*ss = sss;
%\usepackage[T1]{fontenc}
%\usepackage[scaled=0.9375]{inconsolata}%15/16
% </fonts>
% <if pdflatex> load fontenc before inputenc
%\usepackage[T1]{fontenc}
%\usepackage[utf8]{inputenc}
% </if>
\usepackage[strict]{csquotes}% \enquote{something} instead of ``something''
% citation
\usepackage{cite}
% indexing
% put \makeindex before \begin{document}
% put \printindex where you want the index to be printed
\usepackage{imakeidx}
% notes for indexing:
% (1) end your \index line with a percent sign %, just in case it inserts a rogue space
% (2) if we use singlepar, we should make sure that there are no more than one exclamation mark in the argument of \index
\usepackage[columns=2,font=footnotesize,justific=raggedright,columnsep=1em,indentunit=1em,itemlayout=abshang]{idxlayout}
%
%
% hyperlinking; usually should be loaded last
%
% final undoes draft given in documentclass
\usepackage[hidelinks,final]{hyperref}
%
% epigraph
\usepackage{epigraph}
\renewcommand\epigraphsize{\scriptsize}
%\setlength\epigraphwidth{0.625\textwidth}
\setlength\epigraphwidth{0.75\textwidth}
%
%
% for \( and \) inside \section
\usepackage{fixltx2e}
%
% layout
%
% page size
\setstocksize{9in}{6in}
\settrimmedsize{9in}{6in}{*}
\setlrmarginsandblock{1.000in}{0.750in}{*}% spine, edge, ratio
\setulmarginsandblock{0.875in}{0.875in}{*}
\setheadfoot{2em}{0em}% headheight, footskip
\setheaderspaces{*}{1em}{*}% headdrop, headsep, ratio
%\skip\footins=1em
\checkandfixthelayout
% SECTION 1.1 BLA BLA -> Section 1.1 Bla bla
\nouppercaseheads
%\raggedright
%\setlength\parindent{2em}
\raggedbottom
%%
%% toc
%%
\setlength\cftpartnumwidth{3.0em}
\setlength\cftchapternumwidth{3.0em}
\setlength\cftsectionnumwidth{3.0em}
\setlength\cftsubsectionnumwidth{3.5em}
\setlength\cftsectionindent{0em}
\setlength\cftsubsectionindent{3.0em}
%%
%% toc arev (big font)
%%
%\setlength\cftpartnumwidth{4.0em}
%\setlength\cftchapternumwidth{4.0em}
%\setlength\cftsectionnumwidth{4.0em}
%\setlength\cftsubsectionnumwidth{4.0em}
%\setlength\cftsectionindent{0em}
%\setlength\cftsubsectionindent{4.0em}
% list, itemize, enumerate vertical spaces
\tightlists
% beforeparaskip leaves too much empty space
\setbeforeparaskip{0em}
% table of contents
\setsecnumdepth{subsection}
\maxtocdepth{subsection}
% section font size
\setsecheadstyle{\normalsize\bfseries}%\normalsize because we have long section titles
\setsubsecheadstyle{\normalsize\bfseries}% same size?
\setbeforesecskip{0.5em}
\setaftersecskip{0.5em}
\setbeforesubsecskip{0.5em}
\setaftersubsecskip{0.5em}
% chapterprecis
\setlength\prechapterprecisshift{-1.5em}
% exercise
% http://mirror.pregi.net/tex-archive/macros/latex/contrib/exercise/exercise.pdf
%\usepackage{exercise}
%\renewcommand\ExerciseName{Exercise}
%\renewcommand\ExerciseListName{Exercise}
\usepackage{../exercise}
%
% part; memman.pdf p. 80
%
\renewcommand\beforepartskip{\vspace*{1em}}
\renewcommand\midpartskip{\par\vspace*{0em}}
\renewcommand\afterpartskip{\vspace*{1em}}
\renewcommand\partnamefont{\Large\bfseries}
\renewcommand\partnumfont{\Large\bfseries}
\renewcommand\parttitlefont{\huge\bfseries}
