\chapter{Probability}

\section{Probability space}

\index{probability measure}%
\index{probability!measure}%
\index{measure!probability}%
A \emph{probability measure} \(\Pr : 2^\Omega \to \Real\) on a set \(\Omega\) is a measure where \(\Pr(\Omega) = 1\).

\index{probability space}%
\index{probability!space}%
\index{space!probability}%
A \emph{probability space} is a measure space \((\Omega,\Pr)\) where
\(\Omega\) is a set called the
\index{sample space}%
\index{sample!space}%
\index{space!sample}%
\emph{sample space}
and \(\Pr : 2^\Omega \to \Real\) is a probability measure.

\index{event}%
An \emph{event} is a subset of a sample space.

Relationship between two events \(A\) and \(B\):
They are
\index{events!equiprobable}%
\index{equiprobable events}%
\emph{equiprobable} iff \(\Pr(A) = \Pr(B)\),
\index{mutually exclusive events}%
\index{events!mutually exclusive}%
\emph{mutually exclusive} iff \(A \cap B = \emptyset\),
\index{independent events}%
\index{events!independent}%
\emph{independent} iff \(\Pr(A \cap B) = \Pr(A) \cdot \Pr(B)\).

\section{Density}

\index{density of a measure}%
The \emph{density of the measure \(m : 2^U \to \Real\)}
is the \(f : U \to \Real\) where \(\forall A \subset U : \int_A f = m(A)\).

\index{probability density function}%
A \emph{probability density function} (pdf) is a function \(f : \Real \to \Real\) where \(\int_\Real f = 1\).

Let \((\Omega,\Pr)\) be a probability space.
The density of \(\Pr\) is a pdf.
The density of that space is the density of its measure.

\section{Distribution}

\index{distribution function}%
\index{cumulative distribution function}%
The \emph{distribution function} of \(\Pr:2^\Real\to\Real\) is \(F(a) = \Pr(\{ x~|~x \le a\})\),
which is also called the \emph{cumulative distribution function} of \(\Pr\).

Relationship between density and distribution:
Iff \(\Pr : 2^\Real \to \Real\),
\(f\) is the density of \(P\), and
\(F\) is the distribution of \(P\),
then \(F(a) = \int_{(-\infty,a]} f\).
We say that \(f\) is the pdf of \(F\).

\section{Conditional probability}

\index{conditional probability}%
\index{probability!conditional}%
The \emph{conditional probability of \(R\) given \(C\)}
is \(\Pr(R|C) = \Pr(R \cap C) / \Pr(C)\).

\(C\)-conditionalization maps \((\Omega,\Pr)\) to \((C,Q)\)
where \(Q(R) = \Pr(R|C)\).

Let \(f\) be the density of \(\Pr\).
Recall that \(\Pr(A) = \int_A f\).
Therefore \(\Pr(R|C) = \Pr(R \cap C)/\Pr(C) = \int_{R \cap C} f / \int_C f\).

\index{conditional density}%
\index{density!conditional}%
The conditional density \(f(r|C)\) satisfies \(\int_R (r \to f(r|C)) = \Pr(R|C) = Q(R)\).
\(\int_{R \cap C} f / \int_C f\).
The right side of the vertical bar is a set.

\index{conditional distribution}%
\index{distribution!conditional}%
The conditional distribution
\(F(r \le a|C) = \int_{(-\infty,a]} (r \to f(r|C))\).

\section{Example: Fair six-faced dice roll}

A fair six-faced dice roll can be modeled with
\(\Omega = \{1,2,3,4,5,6\}\) and \(\Pr(\{1\}) = \Pr(\{2\}) = \Pr(\{3\}) = \Pr(\{4\}) = \Pr(\{5\}) = \Pr(\{6\}) = 1/6\).
The probability of rolling an odd face is \(\Pr(\{1,3,5\}) = \Pr(\{1\}) + \Pr(\{3\}) + \Pr(\{5\}) = 3/6\).

\section{Random variable}

\index{random variable}%
\index{random variable!distribution of}%
\index{distribution!of random variable}%
A \emph{random variable} \(X : \Omega \to V\)
is a measurable function.
\index{real random variable}%
\index{random variable!real}%
Iff \(V = \Real\), then \(X\) is a \emph{real random variable}.
\index{probability space!of random variable}%
Such \(X\) maps
\((\Omega,\Pr)\) to \((V,Q)\)
where \(Q(A) = \Pr(\{\omega~|~X(\omega)\in A\}) = \Pr(X\in A)\).
Note that \(\Pr(X \in A)\) is an abuse of notation.

The \emph{distribution of \(X\)} is the distribution of \(Q\),
that is \(F(a) = \Pr(X \le a)\).

\(X \sim D\) means that \emph{the distribution of \(X\) is \(D\)}.

\index{probability mass function}%
A \emph{probability mass function} (pmf) \(f : \Omega \to \Real\)
is a function where \(\forall A \subseteq \Omega : \sum_{x \in A} f(x) = \Pr(A)\).
\index{probability density function}%
A \emph{probability density function} (pdf) \(f : \Omega \to \Real\)
is a function where \(\forall A \subseteq \Omega : \int_A f = \Pr(A)\).
A statement about pdfs can usually be translated into
a statement about pmfs by replacing integral with summation.

If \(A \subseteq \Real\), then a pmf \(f : A \to \Real\)
can be turned into a pdf \(g : \Real \to \Real\)
where
\(g(x) = \sum_{c \in A} f(c) \cdot \delta(x-c)\)
where
\index{Dirac delta function}%
\index{function!Dirac delta}%
\(\delta\) is the \emph{Dirac delta function}
where
\( \int_X \delta = [0 \in X] \)
where \([F]\) is 0 iff \(F\) is false and 1 iff \(F\) is true.

\paragraph{Example}
If \(\Omega\) is the set of all people on Earth,
then \(X(\omega)\) might be the age of the person \(\omega\) in years.
If \(\Omega\) is the set of all cookies produced by a machine,
then \(X(\omega)\) might be the sugar amount in the cookie \(\omega\) in grams.
If \(\Omega\) is the set of all tasks,
then \(X(\omega)\) might be the time required to finish the task \(\omega\) in seconds.

\section{Joint and marginal}

\index{joint probability space}%
\index{probability space!joint}%
\index{probability!joint}%
Let there be \(n\) probability spaces \(S_k = (\Omega_k,\Pr_k)\).
The \emph{joint probability space} of \(S_1,\ldots,S_n\) is
\(S = (\Omega,\Pr)\)
where \(\Omega = \prod_{k=1}^n \Omega_k\).

Define the set \(W_k(A) = \{ \omega ~|~ \omega \in \Omega, ~ \omega_k \in A \}\).
Relationship: \(\Pr_k(A) = \Pr(W_k(A))\).
We say that \(\Pr_k\) is obtained by \emph{retaining} the \(k\)th component of \(S\).

Let \(N = \{1,\ldots,n\}\) be the set of the indexes of \(S\)
and let \(K \subseteq N\) be the set of the indexes we want to retain.
The result of
\index{retain}%
\emph{retaining} \(K\) in \(S\)
is \((\Omega,\Pr_K)\) where
\(\Pr_K(A) = \Pr\left(\bigcap_{k \in K} W_k(A_k)\right)\)
where \(A_k = \{ a_k ~|~ a \in A \} \).
\index{marginalize out}%
To \emph{marginalize out} \(K\) from \(S\) is to retain \(N-K\) in \(S\).
Note the relationship among ``retain'', ``marginalize out'', and ``all but'':
To retain \(x\) is to marginalize out all but \(x\),
and to marginalize out \(x\) is to retain all but \(x\).

\index{marginal probability space}%
\index{probability space!marginal}%
\index{probability!marginal}%
Marginal probability space?
Example?

Many random variables:
The notation \(\Pr(X_1\in A_1 \wedge \ldots \wedge X_n\in A_n)\) means
\(\Pr(\{(\omega_1,\ldots,\omega_n) ~|~ X_1(\omega_1) \in A_1, ~ \ldots, ~ X_n(\omega_n) \in A_n\})\)
where \(\Omega = \prod_{k=1}^n \Omega_k\)
and \(\forall k (X_k : \Omega_k \to V_k)\).
Such \((\Omega,\Pr)\) is a \emph{joint probability space}.

\section{Iid random variables}

Let there be two random variables \(X\) and \(Y\).
They are
\index{random variables!independent}%
\index{independent random variables}%
\emph{independent} iff \(\forall A, B : \Pr(X \in A \wedge Y \in B) = \Pr(X \in A) \cdot \Pr(Y \in B)\),
\index{random variables!identically distributed}%
\index{identically-distributed random variables}%
\emph{identically distributed} iff
\(\forall A : \Pr(X \in A) = \Pr(Y \in A)\),
\index{random variables!independent and identically-distributed}%
\index{independent and identically-distributed random variables}%
\index{random variables!iid}%
\index{iid!random variables}%
\emph{iid} iff they are independent and identically-distributed.

\section{Central tendency}

\index{average!weighted}%
\index{weighted average}%
The \emph{weighted average} of \(f:\Real\to\Real\) is
\(w(f) = \int_\Real (x \to f(x) \cdot x) / \int_\Real f\)
where \(\int_\Real f\) is the
\index{normalizing constant}%
\index{constant!normalizing}%
\emph{normalizing constant},
which is 1 iff \(f\) is a pdf.

\index{mean!of a pdf}%
The \emph{mean} of a pdf \(f\) is \(\mu(f) = w(f)\),
\index{mean!of a distribution}%
of a distribution \(D\) whose pdf is \(f\) is \(\mu(D) = \mu(f)\),
\index{mean!of a probability measure}%
of a probability measure \(\Pr\) whose distribution is \(D\) is
\(\mu(\Pr) = \mu(D)\),
\index{mean!of a probability space}%
of a probability space \(S = (\Real,\Pr)\) is \(\mu(S) = \mu(\Pr)\).

\index{expectation}%
The \emph{expectation} of a random variable \(X \sim D\) is \(\Expect{X} = \mu(D)\).
\index{linearity of expectation}%
\index{expectation!linearity of}%
Expectation is linear:
\(\Expect{aX} = a \Expect{X}\) and
\(\Expect{X+Y} = \Expect{X} + \Expect{Y}\).

\index{moment}%
The \emph{\(k\)th moment of \(X\)} is \(m_k(X) = \Expect{X^k}\)
where superscript means power and not composition.
Expectation is first moment.

\section{Dispersion tendency}

\index{central moment}%
The \emph{\(k\)th central moment of \(X\)} is \(\mu_k(X) = \Expect{(X - \Expect{X})^k}\).

\index{variance}%
\emph{Variance} is the second central moment.
The variance of \(X\) is \(\sigma_X^2 = \mu_2(X)\).

\index{standard deviation}%
\emph{Standard deviation} is square root of variance.

\index{skewness}%
\emph{Skewness} is the third central moment.

\section{Mixed moments}

Covariance is \(Cov(X,Y) = \Expect{(X-\Expect{X})(Y-\Expect{Y})}\).
Correlation is \(Cor(X,Y) = Cov(X,Y) / (\sigma_X \sigma_Y)\).

\section{Bernoulli trial}

\index{Bernoulli trial}%
A \emph{Bernoulli trial}
is a probability space \((\Omega,\Pr)\) where \(\Omega = \{0,1\}\),
\(\Pr(\{0\}) = 1 - p\), and \(\Pr(\{1\}) = p\).

\section{Bernoulli distribution}

\index{distribution!Bernoulli}%
\index{Bernoulli distribution}%
The pmf of \(\Bernoulli(p)\) (the \emph{Bernoulli distribution with parameter \(p\)})
is \(f(0) = 1 - p\) and \(f(1) = p\),
which can also be written \(f(k) = (1-p)^{1-k} p^k\) where \(k \in \{0,1\}\).

\section{Binomial distribution}

Bernoulli distribution is a special case of
\emph{binomial distribution} \(\Binomial(n,k,p)\) whose pmf is
\(f(k) = \binom{n}{k} (1-p)^k p^{n-k}\) where \(k : \Nat, k \le n\) with \(n=1\).
\(\Binomial(n,k,p)\) is the distribution of \(n\) iid instances of a Bernoulli trial.

The \emph{binomial formula} is \(\binom{n}{k} = \frac{n!}{k!(n-k)!}\).

\(\binom{n}{k}\) is the number of all \(k\)-sized subsets of an \(n\)-sized set.
The cardinality of \(\{ S ~|~ S \subseteq A, |S| = k \}\) is \(\binom{n}{k}\)
where \(|A| = n\).

The binomial formula is involved in the expansion \((x+y)^n = \sum_{k=0}^n \binom{n}{k} x^k y^{n-k}\).

\section{Multivariate random variable}

\index{random variable!multivariate}%
\index{multivariate random variable}%
\index{random vector}%
\index{random!vector}%
\index{vector!random}%
A \emph{multivariate random variable} or \emph{random vector}
is a vector of random variables,
or a vector-valued random variable,
or a random variable whose sample space is a vector space.

Define \(\{ X < a \}\) as \(\{ x ~|~ x \in u, ~ X~x < a \}\).
If \(f\) is a bijection and \(Y = f~X\),
then \(P~\{f~X < a\} = P~\{X < f^{-1}~a\}\).

Algebra on random variables:
If \(k : a\) and \(X : RV~a\) then \(k X : RV~a\).
If \(X : RV~a\) and \(Y : RV~a\) then \(X+Y : RV~a\).

\index{population}%
\index{population!to sample a}%
\index{population!sample space of}%
\index{population!probability space}%
\index{sample (noun)}%
\index{sample (verb)!a population}%
\index{observation}%
\index{unit!statistical}%
A \emph{population} is a probability space.
Let \(p\) be a population and \(s\) be its sample space.
A \emph{sample} of \(p\) is a subset of \(s\).
To \emph{sample} \(p\) is to pick a subset of \(s\).
An \emph{observation} is a member of \(s\).
A \emph{unit} is an observation.

\section{What are these rules for?}

\index{chain rule}%
\index{probability!chain rule}%
\index{conditional probability!chain rule}%
The \emph{chain rule} is \(\Pr(a \cap b) = \Pr(b|a) \cdot \Pr(a) = \Pr(a|b) \cdot \Pr(b)\).

\index{law of total probability}%
\index{probability!law of total probability}%
\index{conditional probability!law of total probability}%
The \emph{law of total probability} is \(\Pr(b) = \sum_{k=1}^n \Pr(b|a_k) \cdot \Pr(a_k) \)
where \(a_1,\ldots,a_n\) is a partitioning of \(a\).

\index{Bayes rule}%
The \emph{Bayes rule} is \(\Pr(a|b) = \Pr(b|a) \cdot \Pr(a)\)
where \(a_1,\ldots,a_n\) is a partitioning of \(a\).

\section{Materials looking for place to belong}

\index{Gaussian distribution}%
\index{normal distribution}%
\index{distribution!Gaussian}%
\index{distribution!normal}%
\index{Gaussian probability density function}%
\index{normal probability density function}%
\index{probability density function!Gaussian}%
\index{probability density function!normal}%
The \emph{Gaussian pdf} or the \emph{normal pdf}
with mean \(\mu\) and variance \(\sigma^2\) is
\(\frac{1}{\sqrt{2\sigma^2\pi}}\exp\left(-\frac{(x-\mu)^2}{2\sigma^2}\right)\).

\index{Poisson distribution}%
\index{Chi-square distribution}%
\index{Cauchy distribution}%
\index{Beta distribution}%
The \emph{Poisson distribution}...
The \emph{Chi-square distribution}...
The \emph{Cauchy distribution}...
The \emph{Beta distribution}...

\section{Random process}

The type of a \emph{random process} is \(\Nat \to \Omega \to R\).

\emph{Random process} is also called \emph{stochastic process}.

\section{Wiener process}
