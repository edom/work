\chapter{Designing a lovable government}

\epigraph{The Navy is a master plan designed by geniuses for execution by idiots.}{Herman Wouk, \emph{The Caine Mutiny} (1951)}

\section{Purpose}

Government is to do things that benefit many but do not benefit few,
and prevent things that benefit few but do not benefit many.

The highest law is the will of the people.
The people can change the constitution.

\begin{table}[h]
    \caption{Examples of actions according to beneficiaries}
    \begin{tabular}{lll}
        & benefits few & does not benefit few
        \\
        benefits many & good business & road-building
        \\
        does not benefit many & crime & self-defeating behavior
    \end{tabular}
\end{table}

Economically, government is to maximize positive externality,
and minimize negative externality.

Example negative externality:
road congestion due to residential space transforming to commercial space.

\section{Policy}

Do the simplest intervention with the least effort,
the greatest effect, the least gameability,
the least unintended consequences.
But those are conflicting requirements.

Example: even-odd license plate based on day of month.
People will game this by buying two license plates or buying two vehicles.

Avoid wasting resources on policies that
do not attack the root cause of the problem.

\section{Murder}

Why is something illegal?
Perpetrators are going to perpetrate anyway,
and non-perpetrators are not going to perpetrate,
regardless of its illegality.

Some people commit crime unwillingly because they
want to continue living but they no see other way.
Some people are wired differently and commit crime like a pastime.

If something is illegal, why does a perpetrator still do it?
Is it futile to make anything illegal?
If it were not illegal, will it still be done?

Law cannot change human nature.
If you want to change human nature, study biology, not law.
Lawmaking must defer to human nature.

It is impossible to enforce a law if millions of people violate it at the same time.
If a law is against human nature, of course people will violate it.

Is it moral to force someone who enjoy murder not to murder?

\section{Morality}

Government should not interfere in private matter
such as religion, insurance, and consensual sex.

If something does not harm anyone else other than the doer,
it should not be illegal.

We should help others,
but we should to the greatest extent avoid forcing
others to do anything against their will
even though doing that would be good for them.
Consequence: procreation is immoral because
it forces people to exist without their consent.

We can persuade, but we should not force, unless we are in war.

Consumption should be taxed, not income.

Alternative to taxation: donation, state-owned enterprise.

Stability.

In a republic, in an election, you don't pick the best candidate;
you pick the one that will do the least damage.

Holding a government position should not be profitable.
How do we prevent abuse of power?
By not giving power in the first place?
The people must have the final say.

\section{Banning}

Banning something does not eliminate it.
Worse, banning it may also create a black market for it.

Regulate, don't ban.

\section{Sex}

Blocking access to Internet porn sites does not work.

Banning prostitution creates black market.

Banning sex makes more people more curious.

It's part of human nature.

Family education.

\section{Drugs}

Some people are curious.

\section{Patents}

Alternative to patents:
Prize system:
People collect money for a problem,
and the sum is awarded to the inventors who solve the problem.

Another alternative: People can donate to inventors.

Another alternative: Assume that inventors do things
because they love to do that, and they want money, but don't want to get filthy rich.

How was the first people who invented spears rewarded,
if they were rewarded at all?

\section{Prevention of mass destruction}

How does a government prevent
nuclear briefcase detonation,
water source poisoning,
asteroids hitting Earth,
and so on?

The amount of security is inversely proportional to the amount of trust.
If you could trust everybody, security would be unnecessary.
Republics are complicated because you can't trust the ruler.
Republics are designed to minimize the damage done by an evil ruler,
not to maximize the benefit done by a competent ruler.

The most efficient government is
benevolent competent kind laissez-faire minimalist absolute monarch,
but it is also the most risky.
Anything can happen if the ruler dies.

\section{Differences}

Conventional government assumes that most humans are fundamentally evil.
New government assumes that most humans are fundamentally non-evil.

\section{Office}

If you tell people to do something they don't want,
they will do their best to avoid it.
If they want to do it,
they will do it anyway without your telling them.
Thus, you don't tell people.
You simply let them do things,
help them do things,
get out of their way.

If you hire the people who don't want to do the things you want,
no amount of salary will fix it.
When you hire people, there is an intersection between what they want,
what you want, and what your organization wants.

In the office, there are so many things happening that it is too much
for one person to keep track what everyone else is doing.
You are the only person who knows what you really do.
Your supervisor, if any, is the second person to know what you do.
Others have even vaguer ideas about what you do.
But people can instinctively see whether you're doing well or not,
so if you're not doing well,
you must pretend that you were doing well,
and then suddenly you will do well.
If you think you're not doing well, if you doubt yourself,
then it will be a self-fulfilling prophecy.
Even if you are absolutely sure that your job is unimportant,
even if you keep stumbling,
you must unquestioningly believe that you are doing well.

In the office, what you do is less important
than what others think you do.

If you don't know what to do, walk around, and talk to people.

If you don't fit in one company,
maybe that's the company's fault.
But if you have tried three much-different companies
and still can't fit, maybe it's your fault.

Make friends.
People first.
Jobs second.

The number of processes is inversely proportional to the amount of trust.

How much trust are you convenient with?

There are people you naturally gravitate towards,
and people you naturally gravitate away from.
