\chapter{Government}

\section{Purpose}

Government does things that benefit many but do not benefit few,
and prevent things that benefit few but do not benefit many.

\begin{table}[h]
    \caption{Examples of actions according to beneficiaries}
    \begin{tabular}{lll}
        & benefits few & does not benefit few
        \\
        benefits many & good business & road-building
        \\
        does not benefit many & crime & self-defeating behavior
    \end{tabular}
\end{table}

\section{Intervention}

Do the simplest intervention with the least effort and the greatest effect.

\section{Morality}

Government should not interfere in private matter.
Religion is private matter.
Insurance is private matter.
Consensual sex is private matter.

Everything that does not harm anyone else other than the doer should not be illegal.

We should help others,
but we should to the greatest extent avoid forcing
others to do anything against their will
even though doing that would be good for them.
Consequence: procreation is immoral because
it forces people to exist without their consent.

We can persuade, but we should not force, unless we are in war.

Consumption should be taxed, not income.

Alternative to taxation: donation, state-owned enterprise.

Stability.

In a republic, in an election, you don't pick the best candidate;
you pick the one that will do the least damage.

Holding a government position should not be profitable.
How do we prevent abuse of power?
By not giving power in the first place?
The people must have the final say.
