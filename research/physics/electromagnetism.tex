\chapter{Electromagnetism}

The important things:
\UnorderedList{
\item Magnetic field induced by an electric current.
\item Lorentz force on a charge moving in magnetic field.
}

\emph{Electric current} is a flow of electric charges.
The unit of current is \emph{ampere}.
One ampere is one coulomb per second.

\section{William Gilbert's research of magnets}

\section{Electrical objects}

\emph{Electric charge} is similar to mass.

\emph{Point charge} is similar to point mass.

A point charge exists in a \emph{medium}.
An example medium is vacuum.
A \emph{homogeneous medium} is a medium that is the same everywhere.
Every part of a homogeneous medium has the same property as every other part of the same medium.

An \emph{electron} has negative electric charge.
A \emph{proton} has positive electric charge.

The \emph{oil drop experiment} measures the charge of an electron.

\section{Coulomb's law of electrostatics}

Let there be two objects \(C_1\) and \(C_2\).
Let both of them exist in the same homogeneous medium.
Let \( k \) be a \emph{constant} that depends on the medium.

Let \( q_1 \) be the charge of \(C_1\).

Let \( q_2 \) be the charge of \(C_2\).

Let \( x_{12} \) be the position of \(C_2\) as seen from \(C_1\).
In other words, let \( x_{12} \) be the vector from \(C_1\) to \(C_2\).

Let \( F_2 \) be the force exerted by \(C_1\) on \(C_2\).

Let both \(C_1\) and \(C_2\) be \emph{stationary} (let their velocities be zero).

\emph{Coulomb's law} states that \( F_2 = k q_1 q_2 x_{12} / |x_{12}|^3 \),
and, by symmetry, \( F_1 = k q_2 q_1 x_{21} / |x_{21}|^3 = -F_2 \) because \( x_{12} = -x_{21} \).
This is the \emph{electrostatic force}.
Coulomb's law is similar to Newton's law of universal gravitation,
but Coulomb's can repel or attract, depending on the sign of the charges.
Newton's has only been found to attract because negative mass has not been found.

% https://en.wikipedia.org/wiki/Dielectric
A \emph{dielectric} is ...

% https://en.wikipedia.org/wiki/Force_between_magnets#Gilbert_Model
% https://en.wikipedia.org/wiki/Magnetic_dipole%E2%80%93dipole_interaction
% https://en.wikipedia.org/wiki/Magnetism#History
% https://en.wikipedia.org/wiki/Faraday%27s_law_of_induction

% Where does magnetism come from?
% Magnetic moment? Electric current? Spin magnetic moment?
% https://en.wikipedia.org/wiki/Magnetism#Sources_of_magnetism
% https://en.wikipedia.org/wiki/Magnetism#Quantum-mechanical_origin_of_magnetism

\emph{Faraday's law of induction}:
A changing magnetic field through a closed loop causes an electric current?

Force between magnets, Gilbert model, magnetic pole?

\section{Pieces of electromagnetism}

% https://en.wikipedia.org/wiki/Lorentz_force

In 1820, Hans Christian {\O}rsted found that electric current deflects magnetic needle.

Thomson: \(F = q v \times B\).

\emph{Lorentz force law}:
A particle of charge \(q\) moving with velocity \(v\)
in an electric field \(E\) and a magnetic field \(B\)
experiences a force \( F = q E + q v \times B \).
This force is called \emph{Lorentz force}.
("Lorentz force", Wikipedia)
The equation can be factored to \( F/q = E + v \times B \).
If \(E = 0\) and \(q > 0\), then the Lorentz force can be visualized as follows:
Use your right hand, point your thumb right (this is the direction of \(v\)),
your index finger forward (this is the direction of \(B\)),
and your middle finger up (this is the direction of \(F\)).

\emph{Biot-Savart law}:
A constant electric current causes a magnetic field that does not vary with time.
Right-hand rule.
Use your right hand.
Raise your thumb.
Curl the other four fingers.
The thumb is the direction of the current.
The other four fingers is the direction of the magnetic field lines.

\emph{Ampere's law}?

\emph{Gauss's law of magnetism}?

\section{Maxwell's equations}

\section{Heaviside's vector-calculus formulation}

\section{Electromagnetic radiation, electromagnetic wave}

\section{Lorentz invariance}

\section{Electricity, electrochemistry}

An \emph{electrochemical cell} has chemical reaction and voltage.
A \emph{battery} is a collection of electrochemical cells.

% https://en.wikipedia.org/wiki/Electricity
\emph{Electricity} is the set of physical phenomena associated with the presence of electric charge ("Electricity", Wikipedia, unhelpful?).

Why can't battery be more efficient?

Why can't solar cells be cheaper and more efficient?

% https://en.wikipedia.org/wiki/Solar_cell
% https://en.wikipedia.org/wiki/Photovoltaic_effect
The \emph{photovoltaic effect} is ...
A \emph{photovoltaic cell}, also called \emph{solar cell}, is ...

Energy storage: flywheel, battery, reserve hydroelectric reverse water pumping back upstream, thermal energy storage, etc.

\section{Units}

1 ampere is ...

1 farad is ...

1 tesla is ...

1 weber is ...

\section{Electric circuits}

\section{Resistance}

\section{Ohm's law}

\emph{Ohm's law}: \( V = IR \).

\section{Hertz's antenna experiment}

\section{Electromagnetic wave}

\section{Light is electromagnetic wave}

\section{Electromagnetic radiation}

\section{Magnetism}

Magnetism is caused by electron spin?%
\footnote{\url{https://en.wikipedia.org/wiki/Magnetism\#Sources_of_magnetism}}

Tesla is the unit of magnetic flux density (magnetic field strength).
1 tesla is 1 weber per \si{m^2}.
\enquote{A particle, carrying a charge of one coulomb, and moving perpendicularly through a magnetic field of one tesla,
at a speed of one metre per second, experiences a force with magnitude one newton, according to the Lorentz force law.}%
\footnote{\url{https://en.wikipedia.org/wiki/Tesla_(unit)\#Definition}}%
\footnote{\url{https://en.wikipedia.org/wiki/Orders_of_magnitude_(magnetic_field)}}
