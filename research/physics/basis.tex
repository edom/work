\chapter{Bases}

The plural of \enquote{basis} is \enquote{bases}.
Don't confuse it with the plural of \enquote{base}.

\Table{
    \Columns{ll}
    \Caption{Basis-related notations}
    \Head{object & type}
    \Body{
        vector & \(V\)
        \\ tuple & \( \Real^n \)
        \\ parametric curve & \(\Real \to V\)
        \\ basis & \( \Real^n \to V \)
        \\ covector, scalar field & \(V \to \Real\)
        \\ cobasis & \( V \to \Real^n \)
        \\ vector change, vector field & \(V \to V\)
        \\ basis change & \( \Real^n \to \Real^n \)
    }
}

A basis invertibly maps a tuple to a vector.

Let \(V\) be a vector space.

A basis of \(V\) is an invertible function with type \( \Real^n \to V \).

With a basis, we can describe a vector by writing a tuple of numbers instead of drawing an arrow.

\section*{Example basis}

Let \(V\) be the space of all two-dimensional Euclidean vectors.

Thus the basis we describe here will have the type \( \Real^2 \to V \).

Let \(i\) be the unit vector pointing east (right).

Let \(j\) be the unit vector pointing north (up).

Note that \(i\) and \(j\) are orthogonal.

Then we can choose a basis \(e\) such that \(e(x,y) = xi+yj\).

In this basis, the tuple \((1,1)\) describes the vector
that has length \(\sqrt{2}\) and points northeast.

\section*{Example basis change}

Suppose that we rotate the basis \(e\)
(we rotate the coordinate axes) by 90 degrees counterclockwise.
Let the new basis be \(f\).
Then \(f(x,y) = -yi + xj\).

\section*{Don't confuse vectors and tuples}

The tuple is not the vector itself.
The tuple describes the vector.
The tuple and the vector are two different things.
They are related by the basis.

But we can indeed form an abstract-algebraic vector space \(\Real^n\) over the abstract-algebraic field \(\Real\),
so a real tuple is a vector.

\section{Linear basis}

A basis \(e : \Real^n \to V\) is linear iff \(e(a+b) = e(a) + e(b)\).
Note that we overload the plus sign.
The left plus sign is tuple addition.
The right plus sign is vector addition.

We can explode a linear basis \(e : \Real^n \to V\) to \(n\) vectors \(e_1,\ldots,e_n\),
each of the type \(V\), in this way:
\begin{align*}
    e(x_1,\ldots,x_n) &= e_1 x_1 + \ldots + e_n x_n
    \\ &= \Matrix{e_1 & \ldots & e_n}\Matrix{x_1 \\ \vdots \\ x_n}
    \\ &= EX
\end{align*}
and thus the linear basis \(e\) can be represented by the matrix \(E\) where
\[
    E = \Matrix{e_1 & \ldots & e_n}
\]

\section{Describing every vector as a linear combination of basis vectors}

We are talking about the two-dimensional Euclidean space here.
We can imagine it as an unbounded flat sheet of paper.

Let \(V^2\) be the set of all two-dimensional Euclidean vectors.

From \(V^2\), pick any two non-collinear vectors \(e_1\) and \(e_2\).

Let \(x_1,x_2\in\Real\).

The linear combination \(x_1 e_1 + x_2 e_2\) describes a vector in \(V^2\).

If we pick a basis,
we can represent every vector in \(V^2\) using two real numbers.
We can describe the entire \(V^2\) using \(\Real^2\)
as \( V^2 = \{ x_1 e_1 + x_2 e_2 ~|~ (x_1,x_2) \in \Real^2 \} \).

We say that \(E = \{e_1,e_2\}\) is a basis of the two-dimensional Euclidean space.

We say that \((x_1,x_2)\) is the coordinate tuple of vector \(v\) according to basis \(E\).
We can also say that \((x_1,x_2)\) is the \(E\)-coordinates of \(v\).

A basis of \(V^n\) is a set of \(n\) basis vectors
in which every pair of basis vectors are non-collinear.
With such basis, we can describe every vector in \(V^n\)
as a linear combination of those basis vectors.

\section{Representing a coordinate tuple by a column matrix}

We can write \((x,y,z)\) or we can write
\[
    \Matrix{x \\ y \\ z}
\]

\section{Scaling a vector}

If \(k\) is a number and \(v\) is a vector,
then \(kv\) is a vector that has the same direction as \(v\),
but the length of \(kv\) is \(k\) times the length of \(v\),
that is, \( \norm{k v} = k \norm{v} \).

We can think of \(-v\) (the negation of \(v\)) as scaling \(v\) by \(-1\).

If \(v = \sum_k x_k e_k\) then \(cv = \sum_k (c x_k) e_k \).

\section{The relationship between vectors and coordinates}

We have two choices

\(V \to \Real^n\)

\(\Real^n \to V\)

\section{Exploding a cobasis to covectors}

We can explode a cobasis \( e : V \to \Real^n \) to \(n\) covectors \( e_1, \ldots, e_n \),
each having type \( V \to \Real \), in this way:
\[
    e(v) = (e_1(v), \ldots, e_n(v))
\]

\section{Why are the basis and cobasis not the other way around?}

\enquote{A choice of an ordered basis for \(V\) is equivalent to a choice of a linear isomorphism \(\varphi\)
from the coordinate space \(F^n\) to \(V\).}%
\footnote{\url{https://en.wikipedia.org/wiki/Basis_(linear_algebra)\#Ordered_bases_and_coordinates}}

\section{Radial basis? Polar coordinates?}

Let \(e\) be the radial basis. Then \(e(v+w) \neq e(v) + e(w)\) in general.

\section{The cross product? The Levi-Civita symbol?}

The vector \( a \times b \) is the vector that is orthogonal to \(a\), orthogonal to \(b\).
Follow the right hand rule.
If \(a\) is represented by the thumb pointing right,
and \(b\) is represented by the index finger pointing forward,
then \(a \times b\) is represented by the middle finger pointing up.

\( a \times b = \sum_i\sum_j\sum_k \epsilon_{ijk} e_i a^j b^k \) ?

\footnote{\url{https://en.wikipedia.org/wiki/Levi-Civita_symbol\#Cross_product_(two_vectors)}}

\section{The Pythagorean theorem}

Let there be a right triangle.
Let \(c\) be its hypothenuse.
Let \(a\) and \(b\) be the other two sides.
Then, \( a^2 + b^2 = c^2 \).

Many proofs of this theorem are on the Internet.

\section{Closing}

With a basis, we can define vectors with numbers without drawing.
After we have a basis, we can use the calculus of infinitesimals on vectors.
