\chapter{Geometry kitchen sink}
\label{chp:geometry}

\section{Combining synthetic and analytic geometry? Do we still need this distinction?}

Synthetic geometry uses axioms; it does not use formulas.%
\footnote{\url{https://en.wikipedia.org/wiki/Synthetic_geometry}}

Analytic geometry uses formulas from mathematical analysis.

Analytic geometry is a marriage between analysis and geometry.

When asked "What is a line in a 2-dimensional Euclidean space?",
a student of
synthetic geometry
may answer "It is the infinite extension (in both direction)
of the shortest geometric object connecting two points,"
while a student of
analytic geometry
may answer
``It is $\{ (x,y) ~|~ ax+by=c \text{ where } a,b,c\in\Real \}$.''
% https://en.wikipedia.org/wiki/Synthetic_geometry
% https://en.wikipedia.org/wiki/Analytic_geometry
% http://en.wikipedia.org/wiki/Foundations_of_geometry
% http://en.wikipedia.org/wiki/Timeline_of_geometry
% https://en.wikipedia.org/wiki/Algebraic_geometry
Algebraic geometry studies the solution of equations.

Topology studies topological spaces.
Analytic geometry uses coordinates to define spaces,
while topology uses neighborhood to define spaces.

\section{Understanding the advantages of synthetic geometry?}

Why do we begin with synthetic geometry?
Why don't we just go straight to analytic geometry?

\section{Rambling, too abstract}

What makes geometry interesting is not the spaces,
but the mapping between spaces?

(Why are we telling the reader this?)

There are many kinds of spaces we will talk about:

* algebraic structures (especially fields),
* coordinate space (a space of tuples),
* real coordinate space (a space of tuples of real numbers),
* vector space (a coordinate space with vector operations),
* inner product space (a space with an inner product),
* covector space (a space whose member is a covector, defined below).
* tensor space.

There are many ways to make a mapping between two spaces,
as we will see:

* physical field (mapping between two spaces, usually from a coordinate space),
* coordinate system (surjective mapping between two spaces),
* natural vectorization of a ring (mapping from a ring to one of its natural vector spaces),
* basis (linear coordinate system for a vector space),
* covector (linear mapping from a vector space to its ring space),
* cobasis (basis for a covector space).

Then, we will eat our own tail by stating that all those mappings are members of their respective function spaces
(a function space is a space whose members are functions):

* coordinate system transformation (a mapping between two mappings),

\section{Measuring distance using metrics}

With a metric, we can define
a (generalized) circle with radius $r$ and center $c$
to be the set of points
$\{x ~|~ d~c~x = r \}$.

An astronomer may need to tell another astronomer where the latter can find a celestial body.
He can give its coordinates in a coordinate system they have agreed on.
The astronomer can use a celestial coordinate system.
% https://en.wikipedia.org/wiki/Celestial_coordinate_system

\subsection{Measuring distance with Euclidean metrics}

\index{definitions!Euclidean space}%
The two-dimensional Euclidean space
may be imagined as an unbounded sheet of flat paper.

The \emph{Euclidean metric} for \(\Real^n\) is \( d(x,y) = \sqrt{\sum_{k=1}^n (y_k-x_k)^2} \).
It is the length of the shortest straight line segment from \(x\) to \(y\) in Euclidean geometry.
The two-dimensional Euclidean metric is the Pythagorean theorem.

Do not confuse Euclidean metric \(d\) and derivative operator \(d\).

\section{Not yet read}

\enquote{Two Approaches to Modelling the Universe: Synthetic Differential Geometry and Frame-Valued Sets},
John L. Bell\footnote{\url{http://citeseerx.ist.psu.edu/viewdoc/download?doi=10.1.1.114.1930&rep=rep1&type=pdf}}

\enquote{The power of analytic geometry derives very largely from the fact
that it permits the methods of the calculus, and, more generally, of
mathematical analysis, to be introduced into geometry.} (p.~1)

Foundations of geometry\footnote{\url{https://en.wikipedia.org/wiki/Foundations_of_geometry}}

Hilbert's axioms\footnote{\url{https://en.wikipedia.org/wiki/Hilbert\%27s_axioms}}

Birkhoff's axioms\footnote{\url{https://en.wikipedia.org/wiki/Birkhoff\%27s_axioms}}

\section{Mapping \(\Real^n\) to itself? Why?}

We have shown that we can use a real $n$-tuple as a coordinate to the real $n$-space.
In other words, we can map the real $n$-space to itself.
The identity coordinate system for the real $n$-space is
$I : \mathbb{R}^n \to \mathbb{R}^n$ where $I(x) = x$;
this $I$ is also the standard basis for the real $n$-space.

We also call a vector space a linear space.

There is another way to think of a coordinate in $\mathbb{R}^n$:
as a function $N \to \mathbb{R}$ where $N = \{1,2,3,\ldots,n\}$.
We can think of an $n$-dimensional $F$-vector as a function $N \to F$ where $N = \{1,2,3,\ldots,n\}$.
We can forget about dimensions and think of a $F$-vector as a function $\mathbb{N} \to F$.

We can also use vector spaces to talk conveniently about geometric objects.
With vector spaces, we can define a line by vector collinearity,
and define a plane by its normal vector.

Let there be two bases $J$ and $K$.
Let $T$ be coordinate transformation from $J$ to $K$
(that is $J~x = K~(T~x)$)
and $U$ be basis transformation from $J$ to $K$
(that is $K = U~J$).
\[
J~x = (U~J)~(T~x)
\]
\[
J = U \circ J \circ T
\]

Interesting things happen when we change the basis.
$T : E \to F$.

\section{Projective spaces (Why are we talking about this?)}

To grasp projective space, see real projective space.

Descartes's Cartesian coordinate system.

Klein's erlangen program.

Using algebra, we can describe a circle whose radius is $r$
and whose center is the origin in a 2-dimensional Euclidean space with Cartesian coordinate system
as $\{C_2~(x,y) ~|~ x^2+y^2 = r^2\}$.
We can find the intersection of two geometric objects
by solving the system of their equations.







% Synthetic differential geometry
% http://home.sandiego.edu/~shulman/papers/sdg-pizza-seminar.pdf
% Synthetic Differential Geometry: An application to Einstein’s Equivalence Principle
% http://www.math.ru.nl/~landsman/scriptieTim.pdf
% see the references in
% https://ncatlab.org/nlab/show/synthetic+differential+geometry
% https://mathoverflow.net/questions/186851/synthetic-vs-classical-differential-geometry

\section{Avoiding the tensor notation in Ricci calculus and the Einstein summation convention?}

% https://en.wikipedia.org/wiki/Ricci_calculus
We try not use the tensor notation in Ricci calculus,
of which the Einstein notation is a part.

Every text should be more readable than writable.

\section{Using equations to constrain a system, reducing its degree of freedom? Why are we talking about this?}

% https://en.m.wikipedia.org/wiki/Constraint_(classical_mechanics)

A \emph{constraint} is?
A constraint is modeled by an equation.
A system is modeled by a set of equations.
The system satisfies all equations in that set.

Example:
The position of an object moving in straight line is modeled by \(x(t) = k t\).

Constraint force is the cause of the constraint.

\subsection{Understanding a system's degree of freedom}

Bead sliding in a straight wire at an angle to gravitational field line.

Motivate degree of freedom with analytic geometry.
To describe a circle, we need two parameters c and r.

The degree of freedom of a system is the minimum number of parameters required to describe it.
Each parameter has type \(\Real\).

% https://en.wikipedia.org/wiki/Classical_central-force_problem
% https://en.wikipedia.org/wiki/Central_force

\section{Bilinear function? Why are we talking about this?}

A bilinear function \(f\) is left-linear and right-linear:
\(
f(x+z,y) = f(x,y) + f(z,y)
\)
and
\(
f(x,y+z) = f(x,y) + f(x,z)
\).

We can generalize that to multilinearity, but we won't.

* A linear function is multilinear.
* Otherwise a function is multilinear iff every partial application of it is also multilinear.

\section{Measuring the distance between two points on a sphere}

The distance from \(P\) to \(Q\) is \(\angle PCQ \cdot r\)
