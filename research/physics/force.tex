\chapter{Weights and forces}

Pretend that the concept of \emph{mass} has not been invented.

\emph{Weight} is what a weight balance measures.

A weight balance has two arms.

Put a weight on an end of a weight balance.
Push the other end with your hand until the balance comes to rest.
When they reach equilibrium,
both of them exerts the same amount of \emph{force}.

\section{Law of the lever}

% https://en.wikipedia.org/wiki/Virtual_work#Law_of_the_lever
% https://en.wikipedia.org/wiki/Lever

\index{definitions!lever}%
\index{lever!definition}%
\index{simple machine!lever|see{lever}}%
A \emph{lever} has a fulcrum and two ends.

Let \(r_1\) be the distance between the first end to the fulcrum.

Let \(r_2\) be the distance between the second end to the fulcrum.

Let \(F_1\) be the weight placed at the first end.

Let \(F_2\) be the weight placed at the second end.

\index{Archimedes!law of the lever}%
\index{laws named after people!Archimedes's law of the lever}%
\index{laws!lever}%
\index{lever!law of the lever}%
\index{statics!Archimedes's law of the lever}%
\emph{Law of the lever}:
Such lever at equilibrium satisfies \(F_1 \cdot r_1 = F_2 \cdot r_2\).

We take this law as evident.
Doubt can be removed by a simple experiment.

Thus, a weight balance is a lever whose arms have equal length.
