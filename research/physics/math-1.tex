\chapter{Mathematics 1}

Lambda calculus is a set of rules for manipulating expressions.

\section*{Accepting mathematics}

If you love mathematics, skip this section.
This is written for people who think
they can understand physics without mathematics.

\paragraph{Mathematics is not only formulas}
Mathematics is about thinking.
Ideas are important.

\paragraph{Mathematics is an unavoidable tool for physics}

Mathematics has advanced physics.
Mathematics is here to stay.

If you don't want to learn at least some mathematics,
don't waste your time learning physics.
Without math, you will forever be physicist-wannabe.

For physicists, math is a tool.
Physicists use math.
Carpenters use saws.
Blacksmiths use hammers.

You don't need to love math.
Just don't hate it.
A carpenter neither loves nor hates his hammer;
he is simply not attached to it.
He is dispassionate, indifferent, neutral, uncaring.
He has no reason to hate it as long as it does what he expects it to do.

If you can understand English,
then you can understand the math used in this book.
It will take time, and you will have to think hard, but it is surmountable.
If you have tried but you still can't understand,
let us know that we have failed to write clearly.

\paragraph{Mathematics has become a necessity for physics}

We can really understand 20th century physics only if
we are somewhat fluent in the needed mathematics.
Without mathematics, our understanding is superficial and vague,
and all we can do is impress others who are even more clueless.

\paragraph{Mathematics needs mental effort}

Understanding triangles is like lifting a 1 kg weight.
Understanding vectors is like lifting a 2 kg weight.
Understanding manifolds is like lifting a 20 kg weight.
Lifting is uncomfortable,
but getting used to it makes it less comfortable,
until it becomes no big deal.
The only way to get used to something is by repeating it.
Don't try too hard to understand mathematics; just try to get used to it.
If you are hell-bent on building your body,
you won't minding lifting weights, no matter how heavy they are.
Now you just have to get yourself hell-bent on building your brain.
Physical training builds muscles.
Mental training builds brains.
Muscles can be sexy.
Brains can be sexy too.
A book is a mental gym.
If that motivates you.

Fluffy text ends here.
The rest of the book requires hard thinking.
Begin the journey!

\section{Reading mathematical notation}

\paragraph{Mathematical notation is concise English}

When you read an English sentence mixed with mathematical notation,
replace the notation with an English fragment that makes the sentence grammatical.

\index{definitions!overloading a symbol}%
\paragraph{Overloading}

To overload a symbol is to give it different meanings in different contexts.
For example, the same letter \(\Real\) means a set in \enquote{\(x \in \Real\)}
but means a type in \enquote{\(x : \Real\)}.
We write \(\Real\) for both the set of all real numbers and the type of a real number.

\TableRef{tab:overloading} shows overloading.
Note how the same notation \(x,y:\Real\) reads differently.

\Table{
    \Label{tab:overloading}
    \Caption{Reading mathematical notation}
    \Columns{p{0.4\textwidth}p{0.4\textwidth}}
    \Head{notation & English}
    \Body{
        Let \(x,y:\Real\).
        &
        Let \(x\) be a real number and \(y\) be a real number.
        \\
        \midrule
        Given \(x, y:\Real\), we can compute \(z:\Real\) between them.
        &
        Given a real number \(x\) and a real number \(y\), we can compute a real number \(z\) between them.
    }
}

\section{Lambda calculus: rules for forming and reducing expressions}

(Do we really need to tell the reader about lambda calculus?)

In mathematics, a calculus is a set of rules.
(\enquote{Calculus} means \enquote{pebble}.
How did its meaning change that much?)

Lambda calculus is a set of rules for forming and reducing expressions.
An expression is a description of a mathematical object.

\paragraph{Formation rules}

An expression is a variable, a function, or an application.

A variable is an irreducible expression.

If \(a\) is a variable and \(b\) is an expression,
then \(a \to b\) is a function.

If \(a\) and \(b\) are expressions,
then \((a)(b)\) is an application.
We omit the parentheses if such omission doesn't confuse us.
We do application from left to right: \(abcd\) means \(((ab)c)d\).

\paragraph{Substitution}

The notation \( E[x:=y] \) means the expression \(E\)
but with all free occurrences of \(x\) replaced by \(y\).

\paragraph{Reduction rules}

We can reduce an application \((a)(x \to y)\) to \(y[x:=a]\).

We can't reduce anything else.

\paragraph{Question}

Why do we bother to make up expressions, only to reduce them?
(Hint: Computers.)

\paragraph{Typed lambda calculus}
We can further constraint the rules using types in the next section.

\section{Describing existence using type theory}

Every value has a type.

Every expression has a type.

Every variable has a type.

The type of an expression reminds us about what it is.
For example, we write \(x : \Nat\) to mean that \(x\) is a natural number.

\index{definitions!type}%
We write \( x : T \) to mean ``the value \(x\) has the type \(T\)''.
We can read \(x:T\) as ``\(x\) is a \(T\)''.

We write \( x, y : T \) to shorten ``\( x : T \) and \( y : T \)''.

Variable...

Free variable and bound variable...

Let \(x:T\).
Then the answer to the question \enquote{What is \(x\)?} is \enquote{\(x\) is a \(T\)}.
We say that \(x\) inhabits \(T\).
We also say that \(x\) is an inhabitant of \(T\).

\paragraph{Evaluating an expression to a value}

Example: the expression \(1+2\) evaluates to the value \(3\)
according to high school arithmetic.

\paragraph{Tuples}

\index{definitions!tuple}%
A tuple has a fixed number of components.
An \(n\)-tuple has \(n\) components.
Some examples of tuples are \( (1,2,3) \) and \( (t_1,t_2) \).

\index{definitions!tuple component}%
The \(k\)th component of the tuple \(t\) is written \( t_k \) with \(k\) beginning from one.
For example, if \( t = (0,1) \), then \( t_1 = 0 \) and \( t_2 = 1 \).

\section{Converting between English and logic}

Every popular human language contains logic.

\( p \wedge q \)

\section{Describing collections using set theory}

\index{definitions!``practically''}%
In this book, \enquote{practically} means \enquote{wrong but useful}.
We can be practical when we aren't dealing with the foundation of math.

\index{definitions!set}%
We can think of a \enquote{set} as an unordered collection of things without duplicates.%
\footnote{Mathematicians don't define \enquote{set} as such because it would lead to Russell's paradox.}
\enquote{Unordered} means that \( \{a,b\} \) and \( \{b,a\} \) are the same set.
We say that \( a \) is an element of that set.
We notate that as \( a \in \{a,b\} \).

We can also define a set by a predicate or an English description.
An example is the set of all 26 Latin capital letters;
some of its elements are A, B, and C.

\paragraph{Working with sets}

TODO intersection \(A \cap B\),
union \(A \cup B\),
subtraction \(A - B\), ...

\paragraph{Using the set-builder notation to define sets using predicates}

\footnote{\url{https://en.wikipedia.org/wiki/Set-builder_notation}}

\( \{ x ~|~ P(x) \} \)

\( \{ x : P(x) \} \)

\paragraph{Relating set operations and logic operations}

Set intersection corresponds to logical conjunction.
\[
\{ x ~|~ a(x) \} \cap \{ x ~|~ b(x) \} = \{ x ~|~ a(x) \wedge b(x) \}
\]

Set union corresponds to logical disjunction.
\[
\{ x ~|~ a(x) \} \cup \{ x ~|~ b(x) \} = \{ x ~|~ a(x) \vee b(x) \}
\]

\section{Working with equations}

An equation \(a = b\) means that every \(a\) can be replaced with \(b\)
and every \(b\) can be replaced with \(a\).

Equations help us program computers.

An equation relates several parameters.

An equation can also be thought as describing a relationship, a constraint, something that may or may not be satisfied.

\paragraph{Substituting equal things}

\section{Working with numbers}

\paragraph{Natural numbers}

\paragraph{Real numbers}

% https://www.quora.com/Why-are-real-numbers-called-%E2%80%98real%E2%80%99
\index{definitions!real number}%
We write \( \Real \) to mean the set of real numbers\footnote{%
It's not that other numbers are fake. It's just a name that has stuck.%
}.
Example elements are \( 123 \), \( -12.34 \), \( 1/2 \),
\( \sqrt{2} \), \( 2^{0.1} \), \( \log 5 \), \( \sin 1 \), and \( \sum_{k=0}^\infty 3^{-k} \).
We assume that you know real number arithmetics.

\paragraph{Basic arithmetics}

We expect you to be able to add, subtract, multiply, and divide two one-digit numbers (from 0 to 9).
Primary education should have taught you this.

\footnote{\url{https://en.wikipedia.org/wiki/Arithmetic\#Arithmetic_operations}}%
\footnote{\url{https://en.wikipedia.org/wiki/Addition\#Addition_table}}%
\footnote{\url{https://en.wikipedia.org/wiki/Multiplication_table}}%

\paragraph{Using the summation notation}

If \(a < b\), then
\[
\sum_{k=a}^b f(k) = f(a) + f(a+1) + \ldots + f(b-1) + f(b)
\]

If \(K = \{ k_0, k_1, \ldots, k_n \}\) is a set, then
\[
\sum_{k \in K} f(k) = f(k_0) + f(k_1) + \ldots + f(k_n)
\]

\( \sum_k f(k) \) is a notation of the sum of all \(f(k)\) for each value that \(k\) can take. The meaning depends on context. For example, if \(v\) is a tuple with three components, then \(\sum_k v_k\) means \(v_1 + v_2 + v_3\).
