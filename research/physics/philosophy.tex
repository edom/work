\chapter{Philosophy}

\section{Thinking as a physicist (What are we trying to say here?)}

Knowledge is a tower.
It is built from the bottom upward.
Concepts stand on other concepts.
The bottom of that tower is everything we know from our senses.

All models are wrong, but some are useful. (George E. P. Box)

% http://chem.tufts.edu/answersinscience/relativityofwrong.htm
% https://en.wikipedia.org/wiki/The_Relativity_of_Wrong
% https://en.wikipedia.org/wiki/Wronger_than_wrong
Asimov's relativity of wrong:
Both round earth theory and flat earth theory are wrong,
but believing that they are equally wrong
is wronger than both of them combined.

Popper's falsifiability (testability)?

\emph{Abductive reasoning}:
We know from experiment that \(B\), \(C\), and \(D\).
Find a theory \(A\) that explains them.

% https://en.wikipedia.org/wiki/Problem_of_induction
% https://en.wikipedia.org/wiki/Uniformitarianism
\index{laws!uniformity of nature}%
We assume the \emph{principle of the uniformity of nature}:
The laws of nature is the same everywhere everytime \cite{hume1793inquiry}.

\section{Ontology and epistemology (What are we trying to say here?)}

\emph{Ontology} is about what exists in a domain of discourse.
Ontology defines objects, their properties, and their relationships.
\emph{Object} is another word for \emph{thing}.
\emph{Entity} is distinguishable object.
An entity has identity.
An entity can be distinguished from other entities.

\emph{Epistemology} is about how we know something.
In physics, we know things from experiments and inference.
% http://www2.phy.ilstu.edu/pte/publications/scientific_epistemology.pdf
\cite{wenning2009scientific} explains scientific epistemology (about 15 pages).
% https://en.wikipedia.org/wiki/Scientific_method
The \emph{scientific method}?

How can something be said to exist?

% An Ontology for Engineering Mathematics
% http://www-ksl.stanford.edu/knowledge-sharing/papers/engmath.html

% https://www.technologyreview.com/s/429561/the-measurement-that-would-reveal-the-universe-as-a-computer-simulation/
% ontological problem
If we write a computer simulation, then there \emph{is} an absolute position of space.
If we believe that we are a simulation, then there should be a special origin.

\section{Question, epistemology, ramble}

How do we answer ``What is \(A\)?''?

What is a question?
What are its answers?%
\footnote{\url{https://plato.stanford.edu/entries/questions/}}

``\(A\) is \(B\)'' means that \(A\) is another word for \(B\).

``\(A\) is \(B\) that \(C\)''
means that every \(A\) is an instance of \(B\) which satisfies \(C\).
Example: ``a city car is a car designed for traveling inside a city''.
But ``designed for'' is intensional.
How do we know?
If I'm a bad designer, and I make a car I call a ``city car'',
and it fares poorly in a city, then can it be called a ``city car''?

But we can then ask, ``What is \(B\)?'',
but we can't go on forever.
We have to end at something that has to be accepted as is, as a definition, as an axiom.

\(A\) has \(B\) (example: a car has doors).
A big bus is a bus that is big (a bus whose size is great).

A description of reality is not reality.
A theory is a description of reality, a model of reality, a representation of reality.
If reality and theory disagree, then reality wins and theory must change.

To \emph{know} an object is to be familiar with that object.
To \emph{know} an object is to be able to \emph{use} that object.
To \emph{understand} is to \emph{know}.
To \emph{comprehend}.

If \(A\) is tangible (such as a thing described by a concrete noun), it is understandable by sensing (perception).
\emph{Tangible} does not mean \emph{exist}.
A ``red swan'' is tangible, but it doesn't have to exist.

If \(A\) is intangible, it is understandable by thinking (inference).

If \(A\) is a tangible verb (a verb that can be done by actuator (as opposed to sensor)), it is understandable by moving one's body.

If \(A\) is understandable, then ``\(B\) is \(A\)'' is understandable.

If \(A\) is understandable, then ``\(B\) is an \(A\)'' is understandable.

If \(A\) and \(C\) are understandable,
then ``\(B\) is an \(A\) that \(C\)'' (``\(B\) is a \(C\) \(A\)'') is understandable.
Not always the case, for example: \(B\) is an invisible color.
Both ``color'' and ``invisible'' are understandable,
but ``invisible color'' is not understandable due to contradiction in terms.
Thus, in order to be understandable, a statement must be logically satisfiable.

``Why did you do that?''
If the answer is \(A\), then if \(A\) were false, then he would not do that.

``How did you do that?''
If the answer is \(A\), then if we do \(A\), we will accomplish the same.
