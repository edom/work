\chapter{Measurement}

\paragraph{Quantity, magnitude, and unit}

The quantity 1 kg has magnitude 1 and unit kg (kilogram).
The magnitude is a number.
The quantity 2 kg is twice 1 kg.
The quantity 10 kg is ten times 1 kg.

\section{SI units and prefixes}

SI stands for \emph{syst\`eme international d'unit\'es}
(international system of units).

The units are grouped into two kinds: base units and derived units.
Base units don't depend on other units.
Derived units depend on other units.
Example base units are kilogram (kg), meter (m), and second (s).
Example derived units are newton (N, which is \si{kg.m.s^{-2}})
and joule (J, which is \si{N.m}).

A prefix can be put before a unit to enlarge or shrink it.
Example prefixes are
\si{\micro} (micro, \(10^{-6}\)),
m (milli, \(10^{-3}\)),
k (kilo, \(10^3\)),
M (mega, \(10^6\)).
The quantities \SI{1}{kg}, \SI{1000}{g}, and \SI{1000000}{mg} are the same quantity.

Some units are rarely used.
An example rare unit is megagram (1 million grams).

For the complete list of SI units and prefixes, search the Internet%
\footnote{\url{https://en.wikipedia.org/wiki/International_System_of_Units\#Units_and_prefixes}}%
\footnote{\url{https://physics.nist.gov/cuu/Units/units.html}}%
.

\paragraph{A psychological effect}
Someone sounds heavier in grams than in kilograms.
For example, one friend of mine weighs one hundred \emph{thousand} grams.
We compare numbers more easily than we compare units.

\section{Writing numbers in scientific notation}

% https://en.wikipedia.org/wiki/Significant_figures
An example number in scientific notation is \( 1.23 \times 10^{50} \).
Without scientific notation, we would need to write out that number with its 47 trailing zeros.

The number \( 1.23 \times 10^{50} \) has three significant figures.

\section{Reporting a measurement and its uncertainty}

How do we measure quantities?
How do report a measurement?

Every measurement has an uncertainty.
Tools have limited precision.
We report a measurement by writing \SI{1.00(5)}{cm} or \SI[separate-uncertainty]{1.00+-0.05}{cm}
to mean that the actual quantity is somewhere between 0.95 cm and 1.05 cm.
We don't know the actual quantity.
We only know that it's between those.

\section{Sanity check with dimensional analysis}

Dimensional analysis can sanity-check a calculation.
If the unit is wrong, the calculation is wrong.
If the unit is right, the calculation may be right.

Dimensional analysis is a test.
It can detect falsehood.
It can't prove correctness.
