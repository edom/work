\chapter{Statistical physics}

\section{Appreciating the relevance of statistics}

Consider a box of gas with 1 billion particles.
It is impractical to model that by 1 billion equations of motion.
However, we can still say something useful,
because \emph{statistics} allows us to \emph{summarize} the gas.
With statistics, we can talk about \emph{macroscopic behavior},
but we can't talk about individual particles;
we get the summary and we sacrifice the details.

Now we can talk about the \emph{distribution} of the velocity of the particles,
such as \emph{50\% of the particles are slower than something}.

Statistical physics is macro-physics.
The idea is we consider a statistics of the system.
We look at the big picture instead of looking at each particle.
There are many particles.
We cannot say anything about one particle.
What is an example of \emph{statistical ensemble}?

\section{Getting used to probability and statistics}

\ExerciseAnswer{(Discrete probability) Roll a fair six-faced die once. What is the probability of getting the three-dotted face?}{\(3/6\).}
\ExerciseAnswer{(Joint probability of independent events) Roll a fair six-faced die three times. What is the probability of getting the three-dotted face three times?}{%
\((3/6) \times (3/6) \times (3/6)\).}

\ShowAnswers

A \emph{distribution} of a set \(\Omega\) describes how members of \(\Omega\) are distributed.
Let \(f\) be the density of that distribution.
Then \(f(x)\) describes the tendency of values to gather around \(x\)?
Values tend to gather near the peaks of \(f\).

\section{Using the Maxwell\textendash{}Boltzmann distribution of speed (for what?)}

An example question that statistical physics (statistical mechanics) can answer is
``What is the probability of finding a particle with a given speed?''
For example, see the probability density function of the Maxwell\textendash{}Boltzmann distribution.

Maxwell distribution is a chi-distribution with 3 degrees of freedom.

Don't remember the equation.
To be a physicist, you don't need to remember this; you can always go to Wikipedia or open a book.
The important thing is that you know \emph{what it means} and \emph{what it's useful for}.
The density of \emph{Maxwell\textendash{}Boltzmann distribution} is \(f(v)\).
The number \(\int_A f\) describes the \emph{probability of finding a particle
whose speed is in the set (the range) \(A\)}.
Let that sink for a moment, especially if you aren't yet comfortable with probability theory.
The density of \emph{Maxwell\textendash{}Boltzmann distribution} is
???
\Formula{
    \NoNumber
    f(v) = \parenthesize{ \frac{m}{2\pi k T} }^{3/2} 4 \pi v^2 \exp \parenthesize{ - \frac{mv^2}{2kT} }
}
Who got that? How?

% https://en.wikipedia.org/wiki/Temperature
TODO paraphrase this Wikipedia text:
Based on the historical development of the kinetic theory of gases, temperature is proportional to the average kinetic energy of the random motions of the constituent microscopic particles

% https://en.wikipedia.org/wiki/Maxwell–Boltzmann_distribution

Statistical mechanics explains thermodynamics.

% https://en.wikipedia.org/wiki/Thermodynamics

\emph{mole} is

Chemistry?

Entropy?

Canonical ensemble?

Statistical ensemble?

% http://demonstrations.wolfram.com/BoseEinsteinFermiDiracAndMaxwellBoltzmannStatistics/
