\chapter{Vectors}
\label{chp:vector}
\label{sec:vector}

Pick two points \(A\) and \(B\).

The vector \(AB\) tells us how to go from \(A\) to \(B\).

The vector \(AB\) is the subtraction \( B - A \).

We draw the vector \(AB\) as a straight arrow from \(A\) to \(B\).

If we know \(A\) and \(B\), then we can find \(AB\).

If we are at \(A\), then following \(AB\) leads us to \(B\).
We notate this fact as \( A + AB = B \).
We write the point on the left and the vector on the right:
write \( A + AB \), and don't write \( AB + A \).

The zero vector \(0\) tells us how to get to the same point:
\(A + 0 = A\).

The length of a vector \(v\) is written \(\norm{v}\).

A unit vector is a vector whose length is one.

\section*{Adding two vectors}

If \(AB\) tells us how to go from \(A\) to \(B\)
and \(BC\) tells us how to go from \(B\) to \(C\),
then \(AC = AB + BC\)
tells us how to go from \(A\) to \(C\).

We can see that \( a + b = b + a \) by drawing a parallelogram.
We say that vector addition is commutative
because commuting (swapping) the arguments doesn't change the result.

\section*{Negating a vector}

The negation of a vector \(v\) is \(-v\).
It is the vector such that the sum \(v + (-v)\) is the zero vector \(0\).

If we know the vector \(AB\), then we know \(BA\).
We write \(BA = -AB\).
If \(AB\) tells us how to go from \(A\) to \(B\),
then \(BA\) tells us how to go from \(B\) to \(A\).
\(BA\) is the reverse of \(AB\).

\(BA\) is also called the negation of \(AB\) because \(AB + BA = 0\) is the zero vector.
\(A + 0 = A\).

Negating a vector preserves its length:
\(\norm{-AB} = \norm{BA} = \norm{AB}\).

\section{Vector spaces}

A vector space is a set of vectors.

An example vector space is the set of all two-dimensional Euclidean vectors.
We can imagine this vector space as the set of all arrows that we can draw on an unbounded flat sheet of paper.
We don't care about where the arrow is.
We only care about its length and direction.
If two arrows have the same length and direction,
then they represent the same vector.

An example vector space is the space of all two-dimensional Euclidean vectors.

\section{Collinearity and linear combination}

Two vectors \(a\) and \(b\) are collinear
iff they have the same direction or the opposite direction.
If we place them so that their origins coincide, they form a straight line segment.

The zero vector is collinear with every vector.
% This simplifies the definition of basis later.
% Now we can write \enquote{non-collinear} instead of \enquote{non-zero non-collinear}.

See also Wikipedia\footnote{\url{https://en.wikipedia.org/wiki/Collinearity}}.

\paragraph{Linear combination of two vectors}

Let \(p\) and \(q\) be vectors.
A linear combination of \(p\) and \(q\) is a vector \( ap+bq \)
where \(a\) and \(b\) are real numbers.
In other words, we say that \(r\) is a linear combination of \(p\) and \(q\)
iff there exists \(a,b\in \Real\) such that \(r = ap+bq\).

\section{External resources}

See Wikipedia%
\footnote{\url{https://en.wikipedia.org/wiki/Euclidean_vector}}%
\footnote{\url{https://en.wikipedia.org/wiki/Vector_(mathematics_and_physics)}}%
.

\section{What?}

\paragraph{Finding the angle between two vectors}

For every pair of vectors,
we can always find a plane such that both of them lie on that plane.
The plane defines the angle.

\paragraph{Deciding orthogonality by dot product}

Let \(a\) be a vector.

Let \(b\) be vector.

Let the angle from \(a\) to \(b\) is \(\theta\). Positive means counterclockwise.

The dot product is \(a \cdot b = \norm{a} \cdot \norm{b} \cdot \cos \theta\).

Note that the dot symbol \(\cdot\) is overloaded.
It may mean real multiplication,
vector scaling,
vector dot product,
or something else.
Its meaning depends on the things around it.

If \(\norm{a} \cdot \norm{b} \neq 0\) and \(a \cdot b = 0\),
then \(a\) and \(b\) are orthogonal.
Two orthogonal vectors form a right angle.
