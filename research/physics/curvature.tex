\chapter{Curvature}

Having defined \emph{spaces} \ParenRef{chp:manifold},
we now think about their \emph{curvature},
which we have to understand for general relativity%
\footnote{\url{https://en.wikipedia.org/wiki/Mathematics_of_general_relativity}}%
\footnote{\url{https://en.wikipedia.org/wiki/Introduction_to_the_mathematics_of_general_relativity}}%
.

There are several ways to express the curvature of a space.%
\footnote{\url{https://en.wikipedia.org/wiki/Curvature_of_Riemannian_manifolds\#Ways_to_express_the_curvature_of_a_Riemannian_manifold}}
What are they?
What are their benefits and drawbacks?
\enquote{the most standard one is the curvature tensor}

\section{Finding the normal bundle of a space}

\subsection{Finding the normal line at a point of a curve in \(\Real^2\)}

\subsection{Finding the normal plane at a point of a surface in \(\Real^3\)}

\section{The curvature of a curve at a point, and the circle that osculates the curve at that point}

\paragraph{Understanding curvature intuitively}

We have learned to describe shapes using equations and functions.
We have seen that some shapes are more sharply bent than others.
\enquote{Higher curvature} means \enquote{more sharply bent}.

\paragraph{Visualizing the osculating circle}

The \emph{center of curvature} at \(P\) is the point where two normal lines very close to \(P\) intersect.

Read the Wikipedia article about \emph{osculating circle}%
\footnote{\url{https://en.wikipedia.org/wiki/Osculating_circle}}.
\emph{Osculate} means kiss.

Read the Wikipedia article about \emph{curvature}\footnote{\url{https://en.wikipedia.org/wiki/Curvature}}.

The \emph{curvature} of a curve at point \(P\) is the reciprocal of the radius of the osculating circle of the curve at \(P\).
Formally, \( K = 1/R \).

\paragraph{Finding the center of curvature}

\paragraph{Finding the radius of curvature}

\paragraph{Defining the scalar curvature as the reciprocal of the radius of curvature}

\section{Measuring the curvatures of a parabola}

\section{Riemann tensors}

Other names: Riemann curvature tensor,
Riemann-Christoffel curvature tensor.%
\footnote{\url{http://mathworld.wolfram.com/RiemannTensor.html}}

Do we need to understand the Riemann tensor?
Is the Ricci tensor not enough?

\footnote{\url{https://en.wikipedia.org/wiki/Curvature_of_Riemannian_manifolds}}
\footnote{\url{https://en.wikipedia.org/wiki/Riemann_curvature_tensor}}

Levi-Civita connection, affine connection, metric connection

Riemann tensor generalizes Ricci tensor and scalar curvature.
We can contract a Riemann tensor into a Ricci tensor.%
\footnote{\url{http://mathworld.wolfram.com/RicciCurvatureTensor.html}}%
\footnote{\url{http://mathworld.wolfram.com/RiemannTensor.html}}

\footnote{\url{https://en.wikipedia.org/wiki/Ricci_curvature}}

\paragraph{Understanding the two-dimensional case}

\section{Why is the second derivative not enough for describing curvature?}
