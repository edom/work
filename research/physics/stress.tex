\chapter{Linear elastostatics}

\section{Hooke's law of spring restoring force}
\label{sec:hooke-s-law}

Consider a spring at rest.
One end is fixed.
The other end is movable.

Let \(k\) be the spring's \emph{stiffness}.

Suppose that we pull the movable end,
displacing it by \(x\) from its resting position.

\index{laws named after people!Hooke's law of spring restoring force}%
\index{laws!spring restoring force}%
\index{Hooke!law of spring restoring force}%
\index{statics!Hooke's law of spring restoring force}%
\paragraph{Hooke's law}
For small displacements,
the spring's \emph{restoring force} at the movable end is \( F = - k x \).

It also works for rubbers and metals.%
\footnote{\url{http://www.continuummechanics.org/hookeslaw.html}}

\section{Seeing a rod as a stiff spring}

A solid rod behaves like a very stiff spring.

Let there be one such rod with length \(L\) and cross-section area \(A\).

Let one end be fixed.

If we push the other end,
that is if we apply a force \(F\) that is parallel
to the length of the rod (normal to the cross-section),
then we \emph{compress} the rod.
In this case, the \emph{stress} at the fixed end is \(F/A\).

The many definitions of \emph{strain}%
\footnote{\url{http://www.continuummechanics.org/strain.html}}

\footnote{\url{https://en.wikipedia.org/wiki/Saint-Venant\%27s_Principle}}

\footnote{\url{https://en.wikipedia.org/wiki/Linear_elasticity}}

\footnote{\url{https://en.wikipedia.org/wiki/Continuum_mechanics\#Car_traffic_as_an_introductory_example}}

\footnote{\url{https://en.wikipedia.org/wiki/Continuum_mechanics\#Major_areas}}

\section{Pressure, normal stress, shear stress, and stress}

Both pressure and stress are force per unit area.
But pressure is a scalar and stress is a tensor.%
\footnote{\url{https://physics.stackexchange.com/questions/107824/what-is-the-difference-between-stress-and-pressure}}

\section{Deformation}

A soft object is modeled by a subspace of a Euclidean space.
We call this subspace a \emph{material space}.

A \emph{deformation} assigns a vector to every point in the material space.

\section{Understanding the one-dimensional case}

\subsection{Understanding stress and strain}

Let \(\Delta x\) be the change in length.
Let \(x\) be the rest length.
The \emph{strain} is \( \Delta x / x \).

\subsection{Modeling the tension of a cord}

\subsection{Understanding Young's modulus (elastic modulus)}

Young's modulus (elastic modulus) is?

Relationship between length change and the exerted force.

Force that does not cause acceleration.

\subsection{Modeling continuous deformation using continuum mechanics}

Continuum mechanics?

How do we model stress in several dimensions?

\section{Understanding the three-dimensional case}

\subsection{Understanding Cauchy stress tensor}

The \emph{Cauchy stress tensor} is ...
Example: consider a cube:

% https://en.wikipedia.org/wiki/Stress_(mechanics)#General_stress
% https://en.wikipedia.org/wiki/Cauchy_stress_tensor

\section{Motivating tensor}

% https://en.wikipedia.org/wiki/Hooke%27s_law#General_tensor_form
Generalizing Hooke's law to continuous medium?

Generalizing Ohm's law to continuous medium?
