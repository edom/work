\chapter{Statics}

% https://en.wikipedia.org/wiki/Timeline_of_fundamental_physics_discoveries

\emph{Thermodynamics} began as a theory of steam engines.

\emph{Volume} is how much space something occupies.

\emph{Density} is weight per volume.

% https://en.wikipedia.org/wiki/Work_(physics)
\index{definitions!work}%
\index{work!definition}%
\emph{Work}: If one lifts a weight \(F\) so that its height increases by \(h\),
then he does a \emph{work} of \( W = F \cdot h \).
Coriolis defined this in 1826 \cite{coriolis1829calcul}
when steam engines lifted buckets of water out of flooded ore mines.
We shall generalize this definition later,
if the force and the displacement make an angle.

% https://en.wikipedia.org/wiki/Power_(physics)
\index{definitions!power}%
\index{power!definition}%
\emph{Power} is work done per unit time: \( P = W / t \).
This means that a steam engine with twice the power
will clean the same mine in half the time.

% https://hsm.stackexchange.com/questions/414/when-were-the-modern-notions-of-work-and-energy-created
% Helmholtz 1847?

\section{Archimedes's principle of buoyancy}

% https://en.wikipedia.org/wiki/Archimedes%27_principle
% https://en.wikipedia.org/wiki/On_Floating_Bodies

Put a solid into a container full of liquid.

The volume of the spilled part of the liquid is equal to
the volume of the submerged part of the solid.

\index{Archimedes!principle of buoyancy}%
\index{laws named after people!Archimedes's principle of buoyancy}%
\index{laws!buoyancy}%
\paragraph{Archimedes's principle of buoyancy}
Equal are the weight of the object and the buoyant force on the object.
(???)

\section{Pascal's law of fluid pressure transmission}

Blaise Pascal 1647

Pascal's law: Incompressible fluid spreads pressure evenly.

\index{Pascal!law of fluid pressure transmission}%
\index{laws named after people!Pascal's law of fluid pressure transmission}%
\index{laws!fluid pressure transmission}%
\index{statics!Pascal's law of fluid pressure transmission}%
\( P = \rho g h \)

\paragraph{Appreciating Pascal's barrel demonstration}

Counterintuitive: The hydrostatic pressure
does not depend on \emph{how much} fluid.
It depends on \emph{how deep}.
\footnote{\url{https://www.youtube.com/watch?v=EJHrr21UvY8}}

\section{Understanding the zeroth law of thermodynamics}

Put hot iron into cold water.
Eventually both become equally warm.

\index{laws!thermodynamics, zeroth}%
\emph{Zeroth law of thermodynamics}:
Heat never spontaneously flows from cold to hot.

\section{Unstructured content}

TODO Pendulum

\index{definitions!pendulum}%
\index{pendulum!definition}%
A pendulum is a bob hung on a string.

\emph{Conservation of mechanical energy}:
A released pendulum comes back to the same height.

TODO
Interplay between potential and kinetic energy:
Galileo's interrupted pendulum

TODO Vacuum

Boyle showed that objects of different masses fall with the same acceleration.

TODO Toricelli manometer

TODO von Guericke, Magdeburg

TODO Boyle

TODO Pascal

Boyle's experiments

\index{laws named after people!Lavoisier's law of conservation of mass}%
TODO Lavoisier's law of conservation of mass

\section{Understanding energy}

Conservation of energy

Kinetic energy

\emph{Kinetic energy} is \( \frac{1}{2} m |v|^2 \) which can also be written as \( |p|^2 / (2m) \).
This is explained by energy conservation and work by a constant force \(F\) that accelerates an initially resting mass.
\(F = ma\) and \(s = \frac{1}{2}at^2\) and \( W = Fs \) and \( v = at \) therefore \( W = E_k = \frac{1}{2} m(at)^2 = \frac{1}{2}mv^2 \).

\section{Understanding gases}

% https://en.wikipedia.org/wiki/Perfect_gas
% https://en.wikipedia.org/wiki/Gas#Historical_synthesis

A \emph{gas} is ...

\emph{Pressure} is measured by a manometer.

In statics, the \emph{volume} of a gas is the volume of its container.
Statics assumes that a gas fills its container evenly.

\emph{Temperature} is measured by a thermometer.
The unit of temperature is \emph{kelvin} (K).

% ?
Gas and piston at equilibrium:
Gas and a piston with weight \(F\).

\section{Using gas laws}

Let there be a container of gas with pressure \(P_1\) and volume \(V_1\).
Let this gas expand or shrink without changing its temperature
so that its pressure becomes \(P_2\) and its volume becomes \(V_2\).

\index{laws!gas pressure and volume}%
\index{laws named after people!Boyle's law of gas pressure and volume}%
\index{Boyle!Boyle's law of gas pressure and volume}%
\emph{Boyle's law}: \( P_1 V_1 = P_2 V_2 \).

Other gas laws

\emph{Charles's law}?
\emph{Dalton's law}?

% https://en.wikipedia.org/wiki/Dalton%27s_law
% https://en.wikipedia.org/wiki/Combined_gas_law
% https://en.wikipedia.org/wiki/Gay-Lussac%27s_law#Pressure-temperature_law
% https://en.wikipedia.org/wiki/Avogadro%27s_law

\index{laws!ideal gas}
\emph{Ideal gas law}: \( PV = nRT \).

Kinetic energy of one mole of gas is \( \frac{3}{2} RT \).

Statistical thermodynamics: kinetic theory of gases?

\section{Understanding Boltzmann's constant}

% https://en.wikipedia.org/wiki/Boltzmann_constant
\emph{Boltzmann's constant} relates the average kinetic energy of particles in a gas and the temperature of the gas?

% https://en.wikipedia.org/wiki/Gas_constant
The \emph{gas constant} (molar gas constant, universal gas constant, ideal gas constant)?

\section{Understanding Avogadro's number}

\emph{Avogadro's number} is?

Terms?

System and environment

Thermodynamic equilibrium

\section{Understanding heat}

Heat capacity

\emph{Black's principle}:
When two liquids are mixed, the heat released by one equals the heat absorbed by the other.
???

???
If \(m_1\) amount of water at temperature \(T_1\) is mixed with \(m_2\) amount of water at temperature \(T_2\),
then the result, after equilibrium, is \(m_1+m_2\) amount of water at temperature \(\frac{m_1 T_1 + m_2 T_2}{m_1+m_2}\).

Specific heat

Latent heat

\section{Understanding thermodynamic process and cycle?}

Isobaric?
Isochoric?
Adiabatic?
Expansion of gas?
Work done by a gas?

Carnot engine?

Thermodynamic efficiency?

\section{Understanding the laws of thermodynamics}

% https://en.wikipedia.org/wiki/Laws_of_thermodynamics
% https://en.wikipedia.org/wiki/History_of_entropy

\section{Working with simple machines}

% https://en.wikipedia.org/wiki/Simple_machine

\UnorderedList{
\item Lever
\item Wheel and axle
\item{Pulley}
\item{Tilted plane}
\item{Wedge}
\item{Screw}
}

TODO:
Modern machine theory: Kinematic chains

\section{On ignorance}

In the 18th century, occasionally, steam boilers and coal mines exploded, killing tens of people.

Then nuclear power plants exploded.

What if a Dyson sphere exploded...

% Chemistry
% Thermostatics
% Heat and fluids
