\chapter{Particles}

\section{Black body, temperature, and radiation}

\index{black body!definition}%
\index{definitions!black body}%
A \emph{black body} absorbs all radiation that hits it.

A black body radiates.
The spectrum depends on the body's temperature.

Every temperatured object radiates
because it consists of electrons and
accelerating electron emits electromagnetic radiation?
Because heat is vibration of particles?

% https://en.wikipedia.org/wiki/Bremsstrahlung
\index{braking radiation}%
\index{bremsstrahlung}%
Braking radiation (\emph{bremsstrahlung}) happens for the same reason:
deceleration of charged particle.
Deceleration is negative acceleration.

% https://www.quora.com/Why-does-accelerating-charge-radiate-electromagnetic-radiation
% https://en.wikipedia.org/wiki/Li%C3%A9nard%E2%80%93Wiechert_potential
An accelerating charge radiates because of Li\'enard\textendash{}Wiechert potential?

Does an accelerating charge always radiate?

\section{de Broglie wavelength}

Every particle of momentum \(p\) is also a wave of wavelength \(\lambda = h / p\),
and vice versa: every wave is also a particle?

The \emph{Planck constant} is \( h \approx \SI{6E-34}{J.s} \).

\section{Condensed matter physics}

Bose\textendash{}Einstein condensate

Spontaneous emission

\section{Laser}

Laser cooling: using laser to cool atoms.

Thought experiment: Schr\"odinger's cat

Cosmology

Big bang

Cosmic microwave background radiation

Tunneling

Josephson effect

% https://en.wikipedia.org/wiki/Josephson_effect

Macroscopic quantum phenomena

% https://en.wikipedia.org/wiki/Macroscopic_quantum_phenomena

Superfluidity

% https://en.wikipedia.org/wiki/Superfluidity

Photon energy and momentum

% how are photons produced?
% https://science.howstuffworks.com/light7.htm
% > A photon is produced whenever an electron in a higher-than-normal orbit falls back to its normal orbit. (?)
% https://en.wikipedia.org/wiki/Laser
% https://en.wikipedia.org/wiki/Spontaneous_emission
% https://en.wikipedia.org/wiki/Stimulated_emission


% https://physics.stackexchange.com/questions/2229/if-photons-have-no-mass-how-can-they-have-momentum
% Energy of photon E = hf = pc (because photon rest mass is zero)
% Momentum of photon p = E/c = hf/c

\section{Mirror}

% mirror
% https://en.wikipedia.org/wiki/Radiation_pressure
% https://en.wikipedia.org/wiki/Optical_amplifier

Conservation of momentum in light reflected by mirror? Quantum explanation of how mirror works?

% https://en.wikipedia.org/wiki/Franck%E2%80%93Hertz_experiment
% https://en.wikipedia.org/wiki/Thermionic_emission

Half-silvered mirror?

How optical diode works

% "optical diode"
% https://en.wikipedia.org/wiki/Optical_isolator
% http://physicsworld.com/cws/article/multimedia/2015/jul/08/how-do-you-produce-a-single-photon
% https://www.scientificamerican.com/article/how-do-mirrors-reflect-ph/

Hardy's paradox

% https://en.wikipedia.org/wiki/Hardy%27s_paradox

Argument: the universe is computer simulation

% http://www.bottomlayer.com/bottom/argument/Argument4.html

\section{Radiant exitance}

\index{radiant power}%
\index{power!radiant}%
\index{units!radiant power (\si{W})}%
The unit of \emph{radiant power} is watt.

\index{definitions!radiant exitance}%
\index{radiant exitance!definition}%
The \emph{radiant exitance} (radiant emittance)
of a surface is the radiant power
emitted by that surface
per unit area of that surface.
\index{radiant exitance!unit (\si{W/m^2})}%
\index{units!radiant exitance (\si{W/m^2})}%
The unit of radiant exitance is \si{W/m^2}.

% https://en.wikipedia.org/wiki/Larmor_formula
\emph{Larmor formula}?

Rayleigh\textendash{}Jeans law

Wien approximation

% https://en.wikipedia.org/wiki/Stefan%E2%80%93Boltzmann_law

\index{black body!Stefan\textendash{}Boltzmann law of radiant exitance}%
\index{laws!black body radiant exitance}%
\index{laws named after people!Stefan\textendash{}Boltzmann law of black body radiant exitance}%
\emph{Stefan\textendash{}Boltzmann law:}
The total radiant exitance of a black body at temperature \(T\) is \( j^\star = \sigma T^4 \).

% https://en.wikipedia.org/wiki/Josef_Stefan#Work
1879, Stefan derived from Dulong and Petit.
1884, Boltzmann: heat engine with light as working matter (opposed to the usual steam).
Stefan\textendash{}Boltzmann law can be derived from Planck's law?

\section{Planck's law of black body radiation}

% https://en.wikipedia.org/wiki/Photon_gas
% Stefan–Boltzmann law - the total flux emitted by a black body

% https://en.wikipedia.org/wiki/Rayleigh%E2%80%93Jeans_law
% https://en.wikipedia.org/wiki/Wien_approximation
% https://en.wikipedia.org/wiki/Planck%27s_law

Let \(h\) be Planck's constant.

Let \(k\) be Boltzmann's constant.

\index{laws!black body radiation}%
\index{laws named after people!Planck's law of black body radiation}%
\emph{Planck's law of black body radiation}:
The \emph{radiance} of a black body for frequency \(f\) at temperature \(T\) is
\begin{equation}
    B(f,T) = \frac{2hf^3}{c^2} \cdot \frac{1}{\exp\left(\frac{hf}{kT}\right) - 1}.
\end{equation}
