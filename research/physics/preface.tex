\chapter{Preface}

% https://en.wikipedia.org/wiki/First_principle
% https://en.wikipedia.org/wiki/Cartesian_doubt

\paragraph{Target audience}
We assume that you have always wanted to study physics, but life gets in the way.
We assume that you have passed high school (twelve years of formal education).

\paragraph{Required time}
We expect you to focus 10 minutes re-reading the chapter you read yesterday.
Then, we expect you to focus 30 minutes on the next chapter.
We expect you to read 100 words per minute
with 100\% comprehension.

\paragraph{Why physics feels hard}
Because we fantasize that we could gain 500 years of human knowledge
by ten minutes of browsing the Internet.
In reality, we have only been able to
condense some of those 500 years of research
into 4 years of undergraduate education.

\section{How to use this book}

Skim this book until you don't understand.

Then, read from there.

We try to order the materials for optimal learning.
Read this book sequentially.

You will need an Internet connection.

You should do all the exercises.

Don't look at the answers before you come up with your own.
Only use them to check your answer.

Exercises stimulate the mind.
They help you understand concepts,
apply formulas for engineering,
think about theoretical physics,
or do a combination of those.

\section{How this book came into being}

I began writing this book on 2017-11-01.

I wanted to understand physics.

To understand something, I usually write a book

Thus, I wrote this book for myself.

\section{There are too many things to learn}

Human knowledge is always growing,
but the human brain is not.

Research is accelerating.

It's year 2017.
A person needs 22 years to catch up
with the last 500 years of human knowledge.
A person can expect to live 80 years.

It's year 3000.
We may need 50 years to catch up with the last 1,500 years of research.

We must compress knowledge.
We must remove all accidental complexity in learning.
What is left is the essential complexity.

We must make sure that something is hard because it is really hard,
and not because it is presented poorly.
How we explain a concept is accidental complexity.
