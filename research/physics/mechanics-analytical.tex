\chapter{Analytical mechanics}

% https://halshs.archives-ouvertes.fr/halshs-00116768/file/chap5.pdf
\emph{The origins of analytical mechanics in 18th century}, Marco Panza
\cite{panza2003origins}

About the name:
Why is it ``analytical mechanics'' and not ``analytic mechanics''?

\emph{Analytical mechanics} is mathematical analysis applied to mechanics.
Calculus is a part of mathematical analysis.
Calculus is about limit, derivative, and integral.
Several \emph{formulations} of mechanics are
Newtonian, Lagrangian, Hamiltonian, Routhian,
one due to Appell,
and one due to Udwadia\textendash{}Kalaba.

% https://en.wikipedia.org/wiki/Analytical_mechanics
% https://en.wikipedia.org/wiki/Mathematical_analysis
% https://en.wikipedia.org/wiki/Gauss%27s_principle_of_least_constraint

\section{Principle of economy}

\section{Variational principles}

\emph{D'Alembert's principle of virtual work}?
Virtual work?
Virtual displacement?
% https://en.wikipedia.org/wiki/D%27Alembert%27s_principle
% https://en.wikipedia.org/wiki/Virtual_displacement
% http://fy.chalmers.se/~tfemc/mekanikkompendium.pdf
% https://en.wikipedia.org/wiki/Hamilton%27s_principle
% https://en.wikipedia.org/wiki/Principle_of_least_action
% https://en.wikipedia.org/wiki/Routhian_mechanics
% https://en.wikipedia.org/wiki/Appell%27s_equation_of_motion
% https://en.m.wikipedia.org/wiki/Udwadia–Kalaba_equation

\section{Lagrangian mechanics}

Introduction to analytical mechanics \cite[p.~43]{varvoglis2014history}

% https://en.wikipedia.org/wiki/Lagrangian_mechanics#From_Newtonian_to_Lagrangian_mechanics

% https://archive.org/details/springer_10.1007-978-94-015-8903-1
% https://en.m.wikipedia.org/wiki/Mécanique_analytique
Lagrange's \emph{M\'ecanique analytique} was published in 1788 \cite{lagrange1997analytical}.

Lagrangian mechanics only works for conservative forces?

The \emph{Lagrangian} of a mechanical system is?

% https://en.m.wikipedia.org/wiki/Lagrangian_mechanics

Let \( e_k \in \Real^n \) be the \emph{\(k\)th basis vector of \( \Real^n \)};
it is \( e_k = [\delta_{ik}]_{i=1}^n \);
the \(k\)th component of \( e_k \) is one; every other component of \( e_k \) is zero;
\( (e_k)_k = 1 \) and \( (e_k)_i = 0 \) if \( i \neq k \).

For example, assume \( \Real^3 \), and let there be a particle \( M \)
whose mass is \( m \)
and whose position is \( (0,0,0) \).
The \emph{gravitational field} of \( M \) at \( x \) is \( g(x) = - G m x / |x|^3 \).
Such \( g \) is a vector field.
The \emph{gravitational potential} of \( M \) is the \( \phi \) such that \( \nabla \phi = g \).
Such \( \phi \) is a scalar field.
The \emph{potential energy of \( N \) due to \( M \)} is \( K_{MN} = m_N \cdot \phi(x_N) \).

Every particle in a system translates to two things in the \emph{phase space}: a position and a momentum.
We can describe a system without explicit reference to time.
We describe each particle by a set of \emph{position-momentum pairs}.
This set is the \emph{phase space} of the system.

Why bother using phase space if systems of equations work just fine?

We can \emph{describe} the trajectory of a particle using a function
whose input is a real number representing relative time
and output is a three-dimensional real vector representing relative position.
This is straightforward to imagine.

However, we can also describe the same thing by a set of ordered pairs
\( \{ (t,x) ~|~ \text{the object is at \(x\) at time \(t\)} \} \).

Newton's law of gravity describes the force that a
\emph{point mass} exerts on another point mass.
It still applies to planets even if when we assume that a planet is a point mass.
\[
    F_{ab} = \frac{G m_a m_b}{|r_{ab}|^2} \hat{r}_{ab}
\]

% https://en.wikipedia.org/wiki/Gauss%27s_law_for_gravity

Newton's second law:
\(F = dp\) where \(F(t)\) is force acting on the point mass at time \(t\)
and \(p(t)\) is the momentum of the point mass at time \(t\).

The \emph{degree of freedom} of a system is the minimum number of parameters required to describe that system.

A system of \emph{equations of motion} has the form:
\begin{align*}
    x_1(t) &= \ldots
    \\
    & \vdots
    \\
    x_n(t) &= \ldots
\end{align*}

Imagine that in front of you there is a \emph{pendulum} hanging on a thread attached to the roof.
To model that system, we could pick the XYZ coordinate system
where, from your point of view,
the positive X axis is rightward, the positive Y axis is forward, and the positive Z axis is upward.
Thus, at all times, the force acting on the pendulum is \( F = (0,0,-mg) \).
This should be straightforward to imagine.
But you have to determine the tension of the thread that constrains the pendulum's motion.

But we can pick another coordinate system where a point is described by \( (\theta) \).
Let \(\theta\) be the angle from the vertical axis to the thread:
\( \theta = 0 \) means that the thread is vertical,
and positive \( \theta \) means that the pendulum is to your right.
Let \( K \) be the line length.
\(h = K - (1 - \cos \theta) K = K \cos \theta\).
\(P = m g h\).
\(K = \frac{1}{2} m v^2\).
Generalized coordinates: \((h,\theta)\) instead of \((x,y)\).
Translation rules: \(x = K \sin \theta\) and \(y = K \cos \theta\).
The point \(x,y=0,0\) is the lowest point of the pendulum.

\section{Example: Two rigid bodies}

Assume constant mass.
\begin{align*}
    m_1 \cdot (d^2 x_1)(t) &= G m_1 m_2 \cdot (x_2(t) - x_1(t)) / \norm{x_1(t) - x_2(t)}^3
    \\
    m_2 \cdot (d^2 x_2)(t) &= G m_1 m_2 \cdot (x_1(t) - x_2(t)) / \norm{x_1(t) - x_2(t)}^3
\end{align*}
Matrix form:
\begin{align*}
    \bmat{
        m_1 \cdot (d^2 x_1)(t)
        \\
        m_2 \cdot (d^2 x_2)(t)
    }
    &=
    \frac{G m_1 m_2}{\norm{x_1(t) - x_2(t)}^3}
    \bmat{
        x_2(t) - x_1(t)
        \\
        x_1(t) - x_2(t)
    }
\end{align*}

\paragraph{Example}
Uniform gravitation field.
Phase space coordinate \(h\) where \(h\) is height.
\(P(h) = m g h\).
\(K(h) = m (v(h))^2 / 2\).
Conservation of energy: \(P + K = E\).
\(d_h E = 0 = m g + \frac{1}{2} m \cdot (d_h v)(h) \cdot 2 v(h)\).
\(0 = g + (d_h v)(h) \cdot v(h)\).
\(- g = d_h v \cdot v\).

\section{Canonical coordinates}

\section{Poisson bracket}

The \emph{Poisson bracket} is ...

\section{Hamiltonian mechanics}

With physical laws, we can predict the state of physical systems.

A
\index{configuration space}%
\emph{configuration space} is a vector space where each vector is a generalized coordinate tuple.

Example of Hamiltonian mechanics:
In two-body problem,
the state space is ...,
the configuration space is ...,

\section{Noether's theorem}

\subsection{Conservation of energy}

\subsection{Newton's third law of motion}
