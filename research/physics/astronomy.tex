\chapter{Astronomy}

\section{Reading sky map to find celestial objects}

% https://en.wikipedia.org/wiki/Celestial_sphere

Celestial sphere.
The sky looks as if it were projected to a spherical screen.
If a star is far enough, it will look as if it were fixed in the sky.

You find a star in the sky.
You write a letter to your friend.
How do you write where that star is?
How do you explain to him which direction he should look at?

You use the \emph{equatorial coordinate system}.

\emph{Right ascension}, \emph{declination}, and \emph{epoch}.

Example: Alpha Centauri A.
Right ascension 14 h 39 m 35.06311 s.
Declination \(-60\deg\) 50' 15.0992".
Epoch J2000.

% https://en.wikipedia.org/wiki/Epoch_(astronomy)#Julian_years_and_J2000
J2000 is the Gregorian date 2000-01-01 12:00 TT (terrestrial time).

Star chart, star map, sky map

% https://en.wikipedia.org/wiki/Celestial_coordinate_system

Celestial coordinate system

% https://en.wikipedia.org/wiki/Star_chart

% https://en.wikipedia.org/wiki/Celestial_coordinate_system

Equatorial coordinate system

% https://en.wikipedia.org/wiki/Alpha_Centauri

\section{Distance}

% https://en.wikipedia.org/wiki/Parsec

1 au (astronomical unit) is roughly the distance between the Sun and the Earth.
It is about 150 million km.

Parsec is a unit of length.
\( 648000/\pi \).
1 pc is about 3.26 ly.

A \emph{light year} is the distance traveled by light in one year.
\emph{Light year} (ly) is a unit of \emph{distance}, not time.
1 au is about 6 light minutes.

\section{Objects}

A \emph{planet} is?

% https://en.wikipedia.org/wiki/Stellar_evolution
A \emph{star} is a luminous sphere of plasma held together by its own gravity.
("Star", Wikipedia)
Every star begins from collapsing clouds of gas and dust.
A \emph{protostar} is ...
A \emph{main-sequence star} is ...

A \emph{solar system} is?

A \emph{galaxy} is?

A \emph{nebula} is?

A \emph{constellation} is?

A \emph{satellite} is?

A \emph{moon} is?

A \emph{comet} is?

An \emph{asteroid} is?

A \emph{supernova} is?

A \emph{brown dwarf} is?

A \emph{white dwarf} is?

A \emph{black hole} is?

\section{Cosmology}

\section{Cosmogony}
