\chapter{Relativity}

\paragraph{Needed before reading this chapter}
Make sure that you have understood Chapter~\ref{chp:geometry}~(p.~\pageref{chp:geometry}).

\UnorderedList{
\item We see every moving clock ticking slower than our clock.
}

When reality and theory disagree, reality always wins.
We celebrate every time we falsify a theory.
It means that our knowledge is building.

To talk about relativity, we need \emph{analytic geometry},
which allows us to use numbers to describe locations, shapes, and other geometrical concepts.

After reading this chapter, you should be able to:
\UnorderedList{
\item craft a coordinate system
\item use vector to describe relative positions/displacements
\item calculate the time dilation, length contraction, momentum, and kinetic energy of an object moving in spacetime
\item describe the curvature of spacetime due to a point mass
\item visualize the Schwarzschild metric
}

\section{Calculating the Lorentz boost}

\index{Lorentz transformation}%
\index{Lorentz boost}%
Let \(c\) be the speed of light.
The \emph{Lorentz boost} of an object moving with constant speed \(v\) is \(\gamma(v)\) where
\Formula{
    \gamma(v) = \frac{1}{\sqrt{1 - \frac{v^2}{c^2}}}.
}
What does it mean?\footnote{\url{https://en.wikipedia.org/wiki/Lorentz_transformation\#Physical_implications}}

If you are interested in the derivation of the Lorentz transformation,
see \S\ref{sec:derive-lorentz-transform} (p. \pageref{sec:derive-lorentz-transform}) \emph{later}.
Don't do it now.

% https://en.wikipedia.org/wiki/Lorentz_transformation#boost
\ExerciseAnswer{Compute the slowing down of a clock that is moving at half the speed of light.
How many tick does the moving clock make for each tick that a stationary clock makes?
What if the moving clock's speed is \(c/4\)?
What if the moving clock's speed is \SI{100}{km/h}?
Hint: A handy conversion factor to remember is \(\SI{18}{km/h} = \SI{5}{m/s}\).
}{TODO}

\ExerciseAnswer{Exercise? Which clock is slower?
\(A\) is resting and \(B\) is moving.
\(A\) sees that \(B\)'s clock ticks slower.
But from \(B\)'s point of view,
\(A\) is moving and \(B\) is resting,
so \(B\) sees that \(A\)'s clock ticks slower.
Thus, which clock ticks slower?
}{Is this question even valid?}

\ShowAnswers

Time dilation: the faster a clock moves, the slower it ticks.

% this has a point?
% http://alternativephysics.org/book/index.htm

% Disproved by Schutz:
% Why Time Dilation must be Impossible:
% because it's impossible to tell which is moving faster than which,
% because the frame of reference itself could be moving.
% http://alternativephysics.org/book/TimeDilation.htm

% https://physics.stackexchange.com/questions/18867/how-does-time-dilation-work-without-a-privileged-reference-frame
% https://www.reddit.com/r/askscience/comments/1oej1u/how_do_we_pick_the_right_frame_of_reference_for/
% https://en.wikipedia.org/wiki/Twin_paradox#Resolution_of_the_paradox_in_special_relativity

\emph{Minkowski spacetime}?

\emph{World line} is ...

Proper time?

\subsection{Calculating time dilation}

\subsection{Calculating length contraction}

\subsection{Calculating the vector Lorentz transformation}

\section{Understanding rotation in Minkowski space}

\section{???}

\section{Understanding the stress-energy tensor}

\section{???}

\section{Understanding Einstein field equations}

\paragraph{Key idea}
Mass bends spacetime.

As energy at a point increases, spacetime bends more there.

% https://en.wikiversity.org/wiki/General_relativity
% https://en.wikipedia.org/wiki/Riemann_curvature_tensor
% Notation
% https://en.wikipedia.org/wiki/Ricci_calculus

A \emph{tensor} is ...

% FIXME copied from wikipedia
The tensor form of the Einstein field equations is%
\footnote{\url{https://en.wikipedia.org/wiki/Einstein_field_equations}}
\Formula{
    R_{ij} - \frac{1}{2} R g_{ij} + \Lambda g_{ij} = \frac{8\pi G}{c^4} T_{ij}
}
where \(R_{ij}\) is the Ricci curvature tensor, \(R\) is the scalar curvature, \(g_{ij}\) is the metric tensor,
\(\Lambda\) is the cosmological constant, \(G\) is Newton's gravitational constant, \(c\) is the speed of light in vacuum,
and \(T_{ij}\) is the stress–energy tensor.

\paragraph{Exercise}
Describe the bending of space due to a point mass.

\subsection{Calculating black hole event horizon radius?}

% https://en.wikipedia.org/wiki/John_Michell_(natural_philosopher)#Black_holes

In 1783, John Michell wrote about \emph{dark stars},
an object so heavy that light can't escape it?

\section{Calculating gravitational lensing}

% \section{Hawking radiation}

% \section{Black hole thermodynamics}

\section{Deriving Einstein field equations}

(To be done.)

\section{Relativity}

\subsection{Some basic terms}

% Read Landau & Lifshitz relativity volume?

A \emph{spacetime} \(M\) is a four-dimensional space.

An \emph{\(M\)-event} is a point in \(M\).

An \emph{\(M\)-frame} is an invertible function \(\Real^4 \to M\).
Such frame assigns a unique coordinate tuple to every event.

Let \(A : \Real^4 \to M\) be an \(M\)-frame.

Let \(B : \Real^4 \to M\) be an \(M\)-frame.

Let \(E : M\) be an \(M\)-event.

Let \(t \in \Real\).

Let \(x \in \Real^3\).

The notation \(E = A(t,x)\) means that
the event \(E\) \emph{happens} at \emph{\(A\)-time} \(t\) and \emph{\(A\)-position} \(x\).
It means that, \emph{from \(A\)'s point of view},
the event happens at time \(t\) and position \(x\).
We also say that such \(t\) is the \emph{\(A\)-time of \(E\)}
and such \(x\) is the \emph{\(A\)-position of \(E\)}.

We can write unqualified \emph{event} and \emph{frame} to mean \emph{\(M\)-event} and \emph{\(M\)-frame}
if it's clear from context that the spacetime being discussed is \(M\).
Similarly, we can write unqualified \emph{time} and \emph{position} to mean \emph{\(A\)-time} and \emph{\(A\)-position}
if it's clear from context that the frame being discussed is \(A\).

\emph{Observer} is another word for \emph{frame}.

\emph{Time} is measured by a \emph{clock}.

Time orders (sequences) events.

The \emph{position} of an object is where it is in a frame.

\emph{Distance} is measured by a \emph{ruler}.

% https://en.wikipedia.org/wiki/Inertial_frame_of_reference

\subsection{Motion and trajectory}

How do we describe motion?

Let \(A\) be an \(M\)-frame.

Let \(B\) be an \(M\)-frame.

Let \(P\) be a moving object with negligible size.

We describe \emph{A's account} of the motion of \(P\) as
\enquote{The \emph{\(A\)-position} of \(P\) at \emph{\(A\)-time} \(t\) is \(x(t)\).}

% TODO clear up

The \emph{\(M\)-trajectory} of the object \(P\)
is the set \( \{ A(t,x(t)) ~|~ t \in \Real \} \subseteq M \).
It means that at \(A\)-time \(t\), the \(A\)-position of \(P\) is \(x(t)\).
The \emph{\(A\)-velocity} of \(P\) is the derivative of that \(x\).
For example, if \(x(t) = vt + x_0\),
then \(A\) sees that \(P\) is moving with constant velocity \(v\).
The function \(x\) is \(A\)'s description of \(P\)'s motion.
Thus, velocity is also relative, because time and position are relative,
and velocity is defined in terms of time and position.

\index{definitions!stationary}%
\index{stationary!definition}%
An object is \emph{\(A\)-stationary} iff its \(A\)-velocity is constantly zero.

\(A\) and \(B\) may disagree about the motion of \(P\), but \(P\)'s \(M\)-trajectory is the same:
\( \{ A(t,x(t)) ~|~ t \in \Real \} = \{ B(u,y(u)) ~|~ u \in \Real \} \).
The problem is to state \(u\) and \(y\) in terms of \(t\) and \(x\).
It can be thought that the trajectory is the unknowable objective existence of the object?

\emph{\(M\)-world-line} is synonym of \emph{\(M\)-trajectory}.

\section{Deciding whether two events are simultaneous}

What happens at the same time according to an observer?

Two \(M\)-events are \emph{\(A\)-simultaneous} iff they have the same \(A\)-time.
Formally, $A(t,x)$ and $A(u,y)$ are \emph{\(A\)-simultaneous} iff $t = u$.

The set of all events that happen at \(A\)-time \(t\) is \(S_A(t) = \{ A(t,x) ~|~ x \in \Real^3 \}\).

The simultaneity relationship forms an equivalence class?
Isochrone?

Let the \(A\)-velocity of \(B\) be constant \(b\).

From \(A\)'s point of view, at time \(t\),
the light has traveled a distance \(ct\).
Thus, \(S_A(t) = \{ (t,x) ~|~ \norm{x} = ct \}\).
\(X_A(t) = \{ x ~|~ \norm{x} = ct \}\).

From \(B\)'s point of view, at time \(t\),
the light has traveled a distance \(ct\),
but every point in space has moved by \(-vt\).
Thus, the set of all \(B\)-simultaneous events at \(B\)-time \(t\) is
\( S_B(t) = \{ (t,x) ~|~ \norm{x - vt} = ct \} \).
\(X_B(t) = \{ x ~|~ \norm{x - vt} = ct \}\).

Space homogeneity principle:
\(d(x+h,y+h) = d(x,y)\).
translating a rod preserves its length.
The distance between two points depends only on the relative position of the points.

Time homogeneity principle is similar to space homogeneity principle.

Plot \((t,x)\) satisfying \(\norm{x-vt} = ct\).
In two-dimensional space, it is a pair of lines.
In three-dimensional space, it is a cone.
\begin{align*}
    \norm{x-vt}^2 &= \sum_{k \in \{1,2,3\}} (x_k - v_kt)^2
\end{align*}

From \(B\)'s point of view, at time \(t\),
after duration \(dt\),
the light has traveled a distance \(c~dt\),
but every point in space has moved by \(-v(t)~dt\).
Thus, the set of all \(B\)-simultaneous events at \(B\)-time \(t + dt\) is
\( S_B(t + dt) = \{ (t,x) ~|~ \norm{x - v(t)~dt} = c~(t + dt) \} \).

How do we measure the distance to somewhere?
We fire a light at it.
The time light takes to get there is proportional to the distance.

A stationary observer \(A\) fires light everywhere.
According to \(A\), everything in \(\{ x ~|~ \norm{x} = ct \}\)
happens at the same \(A\)-time \(t\).

Two events are \emph{\(A\)-simultaneous} iff their \(A\)-times match.

TODO
What does `synchronized' mean?
What does it mean that two clocks are synchronized?

% from TED history of the world: Matter is congealed energy.

The water waves we see propagate in water,
the sound waves we hear propagate in air,
so it's tempting to guess that the light (the optical waves) we see propagate in
\emph{luminiferous aether} (``fresh air that carries light''), who coined that name?
The wave in moving water?
The \emph{Michelson\textendash{}Morley aether interferometry experiment} tried to measure
the velocity of the Earth with respect to the aether,
but they found no significant differences.
Who threw away the aether idea?
Light does not need a medium to propagate.

\subsection{Inertial frame}

What is a \emph{moving frame}?
How do we formalize it?
Given what \(A\) sees, how do we compute what \(B\) sees?

Let \(A\) and \(B\) agree on the origin: \(A(0,0) = B(0,0)\).

Let the \(A\)-velocity of \(B\) be constant \(b\).

Let the \emph{target} \(T\) be an \(A\)-stationary object at \(A\)-position \(x\).

Let \(E = A(0,0) = B(0,0)\) be the event \enquote{the light fired from the origin}.

Let \(F = A(t,x) = B(t',x')\) be the event \enquote{the light arrived at the target}.

\(A(0,x-ct) = A(0,0)\)

\(B(0,x'-c't') = B(0,0)\)

Let \(c\) be the \(A\)-velocity of the light fired from \(E\) to \(F\).

Thus \(x = ct\). Thus \(c\) and \(x\) have the same direction.

Then \(A(0,x) = B(0,x')\) is the initial event of \(T\).

Then \(F = A(t,x) = B(t',x' - vt')\).

Then \(F = A(t,ct) = B(t',x' - vt')\).

Every \emph{object} has a position in space.

\emph{Law of the velocity of two observers}:
If \(A\) sees \(B\) moving with velocity \(v\),
then \(B\) sees \(A\) moving with velocity \(-v\).
(The \(A\)-velocity of \(B\) is the negation of the \(B\)-velocity of \(A\).)

How do we measure distance?

To measure distance, we use light.

Let \(P\) be an \(A\)-position.

Let \(Q\) be an \(A\)-position.

In frame \(A\),
if light traverses \(PQ\) in duration \(t\),
then the \(A\)-length of \(PQ\) is \(d_A(P,Q) = c t\).

If light takes the same \(A\)-duration to traverse \(OA\) and \(OB\),
then \(OA\) and \(OB\) have equal \(A\)-length: \(d(O,A) = d(O,B)\).

Define \(S(O,q)\) as the set of every point whose distance from \(O\) is \(q\).
Formally, \(S(O,q) = \{ P ~|~ d(O,P) = q \}\).

Law: \(v_{AB} = -v_{BA}\).

To derive Lorentz transformation, we use three observers \(A,B,C\)
where \(A\) sees \(B\) moving but \(A\) sees \(C\) stationary.

An \emph{event} is a point \((t,x)\) in spacetime \(\Real \times \Real^3\).
Let \(A\) and \(B\) agree at \((0,0)\).
An observer is a frame is a chart is a coordinate system.
Let \(C_A(e)\) be the \(A\)-coordinates of \(e\).

Let \(E_A(t)\) be the set of all events that happen in \(A\)-time \(t\).
Then it is \(E_A(t) = \{ e ~|~ \exists x ~ C_A(e) = (t,x) \}\).

Let the position of \(M\) at \(O\)-time \(t\) be \(x(t)\).
If \(O\) emits light, and object \(M\) is moving with velocity \(v\),
then the light and \(M\) will meet at \(d(O,x_\text{meet})\)
where there exists \(t\) such that \(x(t) = x_\text{meet}\).
It is the intersection of a sphere and a line.

The distance takes two objects, not two points.
Let there be objects \(S, M, T\) (stationary, moving, and target).
According to \(S\), \(M\) is moving with velocity \(v\), and \(T\) is stationary.
Suppose that when \(S\) and \(M\) both coincided at \((t_0,x_0)\),
both of them fired a light at \(T\) (target).
\(S\) sees that \(d(S,T) = d(x_0,x_T)\).
\(M\) sees that \(d(M,T) = d(x_0,x_T - vu)\).

\subsection{Old text}

All lights that originate at \(A(t,x)\) reach \(A\)-position 0 simultaneously at \(A\)-time \(t + x/c\).
The set of all $\{ A(t,x) ~|~ \norm{x} = ct \}$ is the \(A\)-simultaneity surface at \(A\)-time $t$.
Two events in the same \(A\)-simultaneity surface happens at the same \(A\)-time.

Let O's firing light in all directions happens at $A=O(0,0)$
and the detection of that light by Q happens at $B = O(\vec{x},t)$.
If $\vec{x}$ is always the O-position of Q,
then
\[
t = x/c.
\]

Now suppose that $Q$ has constant O-velocity $\vec{v}$,
and the O-position of $Q$ at O-time 0 is $\vec{x}$.
Q's detection of light now happens at $B'=O(\vec{x}',t')$ instead, where
\begin{align*}
\vec{x}' = \vec{x} + \vec{v} t'
\\ c t' = \norm{\vec{x}'}
\end{align*}
due to the light speed constancy principle.

Suppose $\vec{v}$ and $\vec{x}$ are perpendicular to each other.
\begin{align*}
(x')^2 = x^2 + (vt')^2 = (ct')^2
\\ x^2 = (c^2-v^2)(t')^2
\\ t' = \frac{x}{\sqrt{c^2-v^2}}
\\ t' = \frac{x}{c \sqrt{1-(v/c)^2}} = \alpha x/c = \alpha t
\\ x' = \alpha x
\end{align*}

To help visualize this, we can use the spacetime diagram.
The x-axis is the distance from O-position zero.
The y-axis is the interval from O-time zero.

Suppose that they weren't perpendicular.
\begin{align*}
\vec{x}' = \vec{x} + \vec{v} t'
\\ c t' = x'
\\ x_k' = x_k + v_k t'
\\ c^2 (t')^2 = (x')^2
\\ = \norm{\vec{x} + \vec{v} t'}^2
\\ = x^2 + 2 t' \vec{x} \cdot \vec{v} + v^2(t')^2
\\ = x^2 + 2 t' \vec{x} \cdot \vec{v} + (v^2 - c^2)(t')^2
\\ t' = \frac{- 2 \vec{x} \cdot \vec{v} \pm \sqrt{(2 \vec{x}\cdot\vec{v})^2 - 4 (v^2 - c^2) x^2}}{2 (v^2 - c^2)}
\\ t' = \frac{\vec{x}\cdot\vec{v} + \sqrt{(\vec{x}\cdot\vec{v})^2 + (c^2 - v^2) x^2}}{c^2 - v^2}
\\ t' = \frac{\vec{x}/c\cdot\vec{v}/c + \sqrt{(\vec{x}/c\cdot\vec{v}/c)^2 + (1 - (v/c)^2) (x/c)^2}}{1 - (v/c)^2}
\\ t' = \frac{(x/c)(v/c) \cos \theta + \sqrt{((x/c)(v/c)\cos\theta)^2 + (1 - (v/c)^2) (x/c)^2}}{1 - (v/c)^2}
\\ t' = \frac{t(v/c) \cos \theta + \sqrt{(t(v/c)\cos\theta)^2 + (1 - (v/c)^2) t^2}}{1 - (v/c)^2}
\end{align*}

A light fired at $O(\vec{x},t)$ can be at any $O(\vec{x}+\vec{dx},t+dt)$
where $dx = c~dt$.

A light that O detects at $O(\vec{x},t)$ could come from any $O(\vec{x}-\vec{dx},t-dt)$ where
$dx = c~dt$.

Let M has constant O-velocity $\vec{v}$.
Let M pass $O(0,0)$.

A light that M detected at $O(\vec{v}t,t)$ could come from any $O(\vec{v}t-\vec{dx},t-dt)$ where $dx = c~dt$.
Therefore O observes that M must detect all such lights at the same M-time that corresponds to O-time $t$.
Therefore O observes that all such things must be M-simultaneous.

A light that O detected at $O(0,t)$ and M detected at $O(\vec{v}\tau,\tau)$
could come from any $O(-\vec{dx},t-dt)$ and $O(\vec{v}\tau-\vec{d\xi},d\tau)$ where $dx = c~dt$ and $d\xi = c~d\tau$.

A light that O detected at $O(0,0)$ and M detected at $O(0,\tau)$
could come from any $O(-\vec{dx},t-dt)$ and $O(\vec{v}\tau-\vec{d\xi},d\tau)$ where $dx = c~dt$ and $d\xi = c~d\tau$.

A light fired from $O(\vec{x},t)$ will be detected by O at $O(0,t)$ and by M at $O(\vec{v}\tau, \tau)$
where
\[
x = ct
\\
\norm{\vec{x} - \vec{v}\tau} = c\tau 
\]

A light that M detected at $M(0,\tau)$ could come from any $M(\vec{d\xi},\tau-d\tau)$ where $d\xi = c~d\tau$.
A light that O detected at $M(-\vec{v}\tau,\tau)$ could come from any $M(-\vec{v}\tau+\vec{d\xi},\tau-d\tau)$ where $d\xi = c~d\tau$.

\[
O(0,0) = M(0,0)
\\ O(\vec{v}t,t) = M(0,\tau)
\\ M(-\vec{v}\tau,\tau) = O(0,t)
\]

This is the same case where everything else has constant M-velocity $-\vec{v}$.

A light fired at $M(\vec{x},t)$ can be at any $M(\vec{x}+\vec{dx},t+dt)$ where
\[
dx = c~dt
\]
according to light speed constancy principle.

However, $M(\vec{x},t)$ and $M(\vec{x}+\vec{v}u,t+u)$ are the same point in space for all $u$.
If, at $M(0,t+u)$, M detects a light fired from $M(\vec{x},t)$ (thus $u=x/c$),
then the firer actually was $M(\vec{x}+\vec{v}u,t)$.

Let $O(0,0) = M(0,0)$.
Let $O(\vec{v}t,t) = M(0,\tau)$.

Suppose that $O$ observes $M$ to be moving according to $\vec{p}$
where $\vec{p}(t) = \vec{a} + \vec{v}t$ is the O-position of $M$ at O-time $t$.
A light that $M$ detects at $O(\vec{p}(t),t)$ could come from any $O(\vec{p}(t)-\vec{dx},t-dt)$
where $dx = c~dt$.
A light from $O(\vec{x},t)$ will be detected by O at $O(0,t+x/c)$
and will be detected by M at $O(\vec{a}+\vec{v}(t+u)-\vec{x},t+u)$
where $\norm{\vec{a}+\vec{v}(t+u)-\vec{x}} = cu$.
If $\vec{x}=0$ and $t=0$, then it will be detected by M at $O(\vec{a}+\vec{v}u,u)$ where $\norm{\vec{a}+\vec{v}u}=cu$.

A light from $O(0,t)$ will be detected by M at $O(\vec{a}+\vec{v}u,u)$ where $\norm{\vec{a}+\vec{v}u} = c(u-t)$
A light from $O(\vec{x},0)$ will be detected by M at $O(\vec{a}+\vec{v}t-\vec{x},t)$ where $\norm{\vec{a}+\vec{v}t-\vec{x}} = ct$

\subsection{Frame transformation}

We \emph{describe frame transformation}.
We can write the equation \(A(t,x) = B(t',x')\) as \(A(t,x) = B(T(t,x), X(t,x))\)
with unknown functions \(T\) and \(X\).
We are trying to state \(t'\) in terms of \(t\) and \(x\).
We write \(A(t,x) = B(t',x')\) to mean that the \(A\)-coordinates \((t,x)\)
and the \(B\)-coordinates \((t',x')\) refer to the same \(M\)-event.
The same \(M\)-event can have different \(A\)-coordinates and \(B\)-coordinates.

\subsection{(being written)}

What is the \(A\)-distance traveled in \(A\)-duration \(dt\) by an object moving with constant \(A\)-velocity \(v\)?
It is \(\norm{v ~ dt} = \norm{v} ~ dt\).

How do we formalize the constancy of the speed of light?

\section{Special relativity}

Special relativity can be derived from Galileo's principle of relativity
and Einstein's principle of the universality of the speed of light.
\cite[p.~1]{schutz2009first}

\index{laws!relativity}%
\index{laws named after people!Galileo's principle of relativity}%
\emph{Galileo's principle of relativity}: No experiment can measure the absolute velocity of an observer.
There is no preferred inertial frame.
There is no special inertial frame.
There is no absolute motion.

\index{laws!universality of the speed of light}%
\index{laws named after people!Einstein's principle of the universality of the speed of light}%
\emph{Einstein's principle of the universality of the speed of light}:
The speed of light is the same for all inertial (non-accelerating) observers.
The origin of this postulate is \enquote{the law of electromagnetics is the same at all reference frames}
or \enquote{the law of physics is the same in all reference frames}.
Everyone sees the same Maxwell's equations.
Everyone measures the same speed of light,
regardless of how fast they are moving.
Did Einstein know the Michelson\textendash{}Morley experiment?

So what?
What are the consequences?

Lorentz transformation can be derived from the principle of the universality of the speed of light,
with the aid of a spacetime diagram?
What is a spacetime diagram?
Minkowski diagram.
% https://en.wikipedia.org/wiki/Minkowski_diagram
Hermann Minkowski introduced in 1908.

\section{Ramble}

Spacetime metric

% http://en.wikibooks.org/wiki/Special_Relativity/Simultaneity,_time_dilation_and_length_contraction

% http://en.wikisource.org/wiki/On_the_Electrodynamics_of_Moving_Bodies_(1920_edition)

Let $O$ be the type of frames and $M$ be the type of motions.
Define $m~x~y$ as the motion of $y$ as observed by $x$:
\[
m : O \to O \to M.
\]
Define $a~x~y$ as adding the motion $x$ to the frame $y$:
\[
a : M \to O \to O.
\]
A motion has spacetime distance elapsed as measured by a frame:
\[
s : O \to M \to S.
\]

A frame always looks stationary to itself.
\[
m~x~x = 0
\]
where $0$ is the null motion.

Inverse motion:
\[
m~x~y = -m~y~x
\]

$m$ looks like subtraction.

Transformation of frames.
\begin{align*}
m~(T~x)~y &= T^{-1}~(m~x~y)
\\ m~x~(T~y) &= T~(m~x~y)
\end{align*}

If the same transform is applied to both frames:
\[
m~x~y = m~(T~x)~(T~y)
\]

The order of transformation does not matter:
\begin{align*}
m~(T~x)~(U~y) &= T^{-1}~(U~(m~x~y)) = U~(T^{-1}~(m~x~y))
\end{align*}

Frame seeing another frame that sees yet another frame.
\[
m~x~z = m~x~y + m~y~z
\]
where $+$ is motion addition.

We see that motions form a monoid.

This motion is straightforward to imagine: $x~t = 3 t$.

A motion seen from another reference frame.

The space picture:
We can think of motion as a path in spacetime.
\[
\begin{bmatrix}t\\x\end{bmatrix}
\]

Force-view of pendulum.

\begin{align*}
F_x &= 0
\\
F_y &= -mg
\end{align*}

It is easier if we change the coordinate system.
We look from the bob's point of view
so that $(0,0)$ is always the bob
and $(0,1)$ is always the direction from the bob to the pivot.

$\theta=0$ represents negative y-axis.
Positive increment is counterclockwise.

\begin{align*}
F_{x'} &= -m g \sin \theta
\\ a_{x'} &= -g \sin \theta
\\ \frac{\partial^2 \theta}{\partial t^2} &= -g \sin \theta
\end{align*}

Field-view of pendulum.

\begin{align*}
P &= m g h
\\
K &= \frac{1}{2} m v^2
\end{align*}

We can describe a manifold by its tangent space.

% https://physics.stackexchange.com/questions/305857/lorentz-transformation-without-constant-speed-of-light-in-vacuum-reasonable/305861#305861

% Cite this paper: Nothing but Relativity
% https://arxiv.org/pdf/physics/0302045v1.pdf
% This paper is already cited by the above paper
% "One more derivation of the Lorentz transformation"
% https://fenix.tecnico.ulisboa.pt/downloadFile/3779571248372/Levy-Leblond_(76).pdf

% ramble

OpenStax University Physics Volume 3 \cite{openstaxphysics3}

TODO
Should we move this chapter earlier to introduce coordinates?
But this chapter should come after electromagnetism?

TODO
How do we describe the bending of space?
How do we describe an almost-Euclidean curved space?
The curvature of space?
By the metric?

(Rewrite: simultaneity hypersurface)
An event is a point in spacetime.
Let \(A,B,O\) be spatial points.
Two points \(A\) and \(B\) are \emph{\(O\)-simultaneous} iff a light fired from \(O\) reaches \(A\) and \(B\) at the same \(O\)-time.
If we are stationary, then our simultaneity hypersurface is a hypersphere.

A world line is a one-dimensional manifold in \(M\).

If \(A\) measures that a light has moved by spatial distance \(dx\), then \(A\)'s clock will have advanced by \(dt = dx / c\).

If \(A\) emits a light, and waits until a time interval of \(dt\) elapsed in his clock, then he will measure that the light has traveled a distance of \(c~dt\).

If \(A\)'s clock measures that a time interval of \(dt\) has elapsed, then the light will have moved by spatial distance \(dx = c ~ dt\).

%https://en.wikipedia.org/wiki/Postulates_of_special_relativity#Mathematical_formulation_of_the_postulates
Pseudo-Riemannian manifold
%https://en.wikipedia.org/wiki/Relativity_of_simultaneity#Accelerated_observers
radar-time/distance definition

The distance of an event \(A\) from another event \(B\) is the time required by light to reach \(B\) from \(A\).

From the point of view of the moving observer, he is stationary, and it is the world (everything else) that is moving towards him with velocity \(-v\). If he fire a ball of light, then after he measures time \(t\), then...

%https://en.wikipedia.org/wiki/Bondi_k-calculus
%https://en.wikipedia.org/wiki/Rapidity

%https://physics.stackexchange.com/questions/305857/lorentz-transformation-without-constant-speed-of-light-in-vacuum-reasonable/305861#305861

\section{Spacetime diagram and world-lines}

What is the coordinate system of an observer moving with constant velocity \(v\)?

A spacetime diagram represents the worldview of an observer?
Consider a spacetime diagram with one time dimension and one space dimension.
A point \((t,x)\) represents an event that occurs at time \(t\) and position \(x\) according to \(O\).
The horizontal is space coordinate \(x\).
The vertical is time coordinate \(t\).
Each event has a spacetime coordinate \((t,x,y,z)\).

Let \(A\) be at 0.
A light coming from a distance \(x\) from \(A\) will reach \(A\) after \(x/c\) time.

If \(A\) fires a light, let the light \(A\)-velocity be \(c\)
(a vector whose magnitude is the speed of light),
then after \(A\)-time-interval \(dt\),
the light will have traveled an \(A\)-distance of \(\norm{c ~ dt}\).
But \(B\) sees \(A\)-velocity \(c'\) and the light has traveled an \(A\)-distance of \(\norm{c' ~ dt - v ~ dt}\) from \(B\).

Speed is distance divided by time.

We can measure distance and time:
fire a light, and wait for its reflection to come back.
If \(t\) time elapsed until we measure the reflection, then the round-trip distance traveled by the light is \(ct\).

\emph{The same point in spacetime is assigned different coordinates.}

How do we know that two different coordinate tuples refer to the same point in spacetime?

Let \(M\) be the event space.
An event is a 4-tuple \((t,x,y,z) \in \Real^4\).
A coordinate system is a bijective mapping \(\Real^4 \to M\).
Let \(C_A\) and \(C_B\) be coordinate systems.
The same event \(e = C_A^{-1}(p_A) = C_B^{-1}(p_B)\).
How can we know that \(p_A\) and \(p_B\) map to the same event?

If \(A\) measures a constant-velocity object at \(C_A^{-1}(t_1,x_1)\) and \(C_A^{-1}(t_2,x_2)\),
then the velocity of the object is \(C_A^{-1}\left(\frac{x_2-x_1}{t_2-t_1}\right)\).

Two observers \(B\) and \(C\) move with the same \(A\)-velocity \(v\).
\(B\) fires a light to \(C\).

Let \(e_k\) be \(k\)'s coordinate system.
Thus \(e_A p = e_B q\) means that \(p\) and \(q\) happen at the same point in spacetime.

An observer is a coordinate system?

The \emph{\(A\)-world-line} of an object \(P\) moving with constant \(A\)-velocity \(v\)
is \( wlPA = \{ (t, x_0 + v t) ~|~ t \in \Real \} \)
where \(x_0\) is where the object is at \(A\)-time 0.
That line is the trajectory of \(P\) in \(A\)'s coordinate system.
Iff \(P\) is a light, then \(\norm{v} = c\).

TODO How do we add velocities in special relativity?

\section{What?}

An observer brings a clock and a light detector with him.
He determines the location
of another object by measuring how long his clock elapses
between his firing a light and his detecting its reflection.
If he fires a light with direction $\hat{x}$
when his clock says $t$ and
detects its reflection with direction $-\hat{x}$ when his clock says $u$,
then he infers that his light hit the object when his clock said $(t+u)/2$,
and he infers that the object was at $\hat{x} c(u-t)/2$ when his light hit the object;
he assumes that the speed of light is constant.

An observer measures the velocity of another object by firing two lights:
the first with direction $\hat{x}$ when his clock says $t$,
and the second with direction $D\hat{x}$ ($\hat{x}$ rotated a little) when his clock says $t + dt$.
If he detects the first reflection when his clock says $u$
and he detects the second reflection when his clock says $u + du$,
then he infers that his first light hits the object at $\hat{x} c(u-t)/2$ when his clock says $(t+u)/2$
and his second light hits the object at $D\hat{x} c(u-t+du-dt)/2$ when his clock says $(t+dt+u+du)/2$,
and therefore he infers that the object has displaced by
$D\hat{x} c(u-t+du-dt)/2 - \hat{x} c(u-t)/2$
while his clock has elapsed by $(dt+du)/2$.
We rearrange the displacement:
\begin{align*}
& D\hat{x} c(u-t+du-dt)/2 - \hat{x} c(u-t)/2
\\ &= \frac{c\hat{x}}{2} ((u-t+du-dt)D - (u-t)I)
\\ &= \frac{c\hat{x}}{2} ((u-t)(D-I) + (du-dt)D)
\end{align*}
therefore he measures that the object's average velocity between $t$ and $t+dt$ was:
\begin{align*}
&\frac{\frac{c\hat{x}}{2} ((u-t)(D-I) + (du-dt)D)}{(dt+du)/2}
\\ &= \frac{c\hat{x} ((u-t)(D-I) + (du-dt)D)}{dt+du}
\end{align*}
which becomes this if $D=I$ ($\vec{v} \cdot \hat{x} = 0$):
\begin{align*}
&\frac{\frac{c\hat{x}}{2} ((u-t)(D-I) + (du-dt)D)}{(dt+du)/2}
\\ \vec{v} &= \frac{du-dt}{du+dt} c \hat{x}
\\ -v/c &= \frac{du-dt}{du+dt}
\\ (du+dt) v/c &= dt-du
\\ (1+v/c)du &= (1-v/c)dt
\\ \frac{du}{dt} &= \frac{1-v/c}{1+v/c} = \frac{c-v}{c+v}
\end{align*}

When he detects a light when his clock says $t$,
he knows the direction where the light is going ($\hat{x}$),
and thus he knows the direction the light came from ($-\hat{x}$),
but he does not know where exactly in his spacetime whence the light originates;
the light could come from any $\hat{x}u$ when his clock said $t-u$, for all real $u$.

Now we introduce coordinate systems.
O can describe an event in his O-spacetime coordinate $(t,x,y,z)$,
and P can describe the same event in his P-spacetime coordinate $(t',x',y',z')$.

An event happens at O-spacetime point P iff
O would instantaneously observe the event if he were at P.
If a light is fired from $(t,x,y,z)$ towards O,
then he will detect it at $(t+\sqrt{x^2+y^2+z^2}/c,0,0,0)$.
The set of points that are O-simultaneous at O-time $t$ is $\{(t,x,y,z) ~|~ x,y,z\in\mathbb{R}\}$.

Suppose that M has a ruler.
One tip, A, is at M.
The other tip, B, is at $\vec{x}$.
Event C: At M-time $t_0$, M fires a light from A.
Event D: At M-time $t_1$, the light hits B and reflects.
Event E: At M-time $t_2$, the reflection arrives at A.
The M-length of the ruler can be found from $\norm{\vec{x}} = c~(t_2-t_0)/2$.

Suppose that O sees M with constant velocity $\vec{v}$.

% http://www.pbs.org/wgbh/nova/blogs/physics/2014/04/how-many-dimensions-does-the-universe-really-have/

If something travels in O-spacetime from $(t,x,y,z)$ to $(t+dt,x+dx,y+dy,z+dz)$,
then its velocity can be found as follows:
\[
(dx)^2 + (dy)^2 + (dz)^2 = v^2~(dt)^2.
\]
If that thing is light then $v=c$.

The light-distance between two points in spacetime:
\[
(ds)^2 = (dx)^2 + (dy)^2 + (dz)^2 - c^2~(dt)^2.
\]

An observer exists as a point in space.
Given a point, he can measure his distance to it.
Given two points, he can measure his distance to each;
he can measure the distance between them;
he can measure the angle formed by the points through him;
he can determine if they lie in a line;
he can calculate the area enclosed by those points and him.

% European Journal of Physics
% Relativity, H. Bondi
% http://iopscience.iop.org/article/10.1088/0034-4885/22/1/304
% Nothing but relativity, redux
% http://iopscience.iop.org/article/10.1088/0143-0807/28/6/011/meta
% Lorentz transformations with arbitrary line of motion
% http://iopscience.iop.org/article/10.1088/0143-0807/28/2/004

% The principle of relativity and the indeterminacy of special relativity
% http://iopscience.iop.org/article/10.1088/0143-0807/29/1/004

\footnote{\url{https://en.wikipedia.org/wiki/Orbital_decay}}%
\footnote{\url{https://en.wikipedia.org/wiki/Gravitational_wave}}%
\footnote{\url{https://en.wikipedia.org/wiki/Two-body_problem_in_general_relativity}}

\footnote{\url{https://en.wikipedia.org/wiki/Newtonian_motivations_for_general_relativity}}

Somatogravic illusion
\footnote{\url{http://aviationknowledge.wikidot.com/aviation:somatogravic-illusion}}

Brans-Dicke theory

\footnote{\url{https://en.wikipedia.org/wiki/Hafele\%E2\%80\%93Keating_experiment}}%
\footnote{\url{https://en.wikipedia.org/wiki/Ives\%E2\%80\%93Stilwell_experiment}}%
\footnote{\url{https://en.wikipedia.org/wiki/Kennedy\%E2\%80\%93Thorndike_experiment}}
