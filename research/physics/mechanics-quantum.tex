\chapter{Quantum mechanics}

\cite{manousakis2016practical}

\section{Understanding the need for quantum mechanics}

Classical mechanics works for slow big things.

Relativity works for fast big things.

Non-relativistic quantum mechanics works for slow small.

Relativistic quantum mechanics works for fast small things.

We abbreviate \emph{quantum mechanics} to QM.

\begin{table}[h]
    \centering
    \begin{tabular}{l|ll}
        & slow & fast
        \\
        \hline
        small & non-relativistic QM & relativistic QM
        \\
        big & Galilean relativity & Einsteinian relativity
    \end{tabular}
    \caption{Four theories}
\end{table}

% https://en.wikipedia.org/wiki/History_of_quantum_mechanics

% http://theorie2.physik.uni-erlangen.de/index.php/Papers_from_the_beginning_of_quantum_mechanics

% Schrodinger's paper
% https://journals.aps.org/pr/abstract/10.1103/PhysRev.28.1049 (paywall)
% http://web.archive.org/web/20081217040121/http://home.tiscali.nl/physis/HistoricPaper/Schroedinger/Schroedinger1926c.pdf

Dirac's 1978 4th edition \emph{Principles of quantum mechanics} \cite{dirac1978principles}

Self-interference? Counterintuitive.

% https://en.wikipedia.org/wiki/Mathematical_formulation_of_quantum_mechanics#Postulates_of_quantum_mechanics

% https://en.wikipedia.org/wiki/Matter_wave
% https://en.wikipedia.org/wiki/Matter_wave#de_Broglie_relations
% https://en.wikipedia.org/wiki/Pilot_wave
% causal interpretation of quantum mechanics
% https://en.wikipedia.org/wiki/De_Broglie%E2%80%93Bohm_theory
% https://en.wikipedia.org/wiki/Louis_de_Broglie

% https://en.wikipedia.org/wiki/Bra%E2%80%93ket_notation

% https://en.wikipedia.org/wiki/Uncertainty_principle
The \emph{Heisenberg uncertainty principle}?

% https://en.wikipedia.org/wiki/Wave_function#Definition_.28one_spinless_particle_in_1d.29
Example wave function?

Dirac's relativistic quantum mechanics

Positron

Quantum tunneling.
It boosts nuclear decay.
With it, a particle can escape a trap.
But, it prevents smaller computers.

\section{Knowing the history of quantum mechanics? Not the best way to learn?}

Bohr model of hydrogen atom

From Wikipedia\footnote{\url{https://en.wikipedia.org/wiki/Photoelectric_effect\#History}}.
1887: Heinrich Hertz discovered the photoelectric effect.
1902: Philipp Lenard observed that the energy of an emitted electron increased with the frequency of the light.
% https://en.wikipedia.org/wiki/Annus_Mirabilis_papers#Photoelectric_effect
1905: Albert Einstein explained that light is a particle, and the energy of a light particle is \(E = hf\).
Bohr inferred that because hydrogen emission spectrum line is discrete,
then hydrogen electron can only occupy certain orbitals with discrete spacing between orbitals.

% https://en.wikipedia.org/wiki/Photoelectric_effect
\index{effect!photoelectric}
\emph{Photoelectric effect}:
High-frequency light shone onto metal dislodges electron.
Low-frequency light has no effect.
The voltage required to stop the dislodged electron
depends on the frequency of the light.
The current depends on the intensity of the light.
The \emph{work function} of the metal is \( \phi = h f_0 \),
the energy required to dislodge its electron.
The rest of the energy becomes the kinetic energy of the electron.
The kinetic energy of an ejected electron is \( E = h f - \phi \).

The energy of a photon with frequency \(f\) is \( E = hf \).

\emph{Stern\textendash{}Gerlach experiment}

% https://en.wikipedia.org/wiki/Stern%E2%80%93Gerlach_experiment

\emph{Zeeman effect: electron spin splits spectral lines}

\emph{Stark effect: electric field splits spectral lines}

The 1914 \emph{Franck\textendash{}Hertz experiment}\footnote{\url{https://en.wikipedia.org/wiki/Franck\%E2\%80\%93Hertz_experiment}}
hurled electrons at mercury vapor.
No electron lost less than \SI{4.9}{eV} of kinetic energy.
This experiment supports Bohr's idea of quantized orbitals?
\cite{franck1967zusammenstosse}

% https://en.wikipedia.org/wiki/Spectral_line
Every atom has a unique spectral line, an emission pattern.
To identify the atom, we put it in a \emph{spectrometer},
and match the emitted spectral line
to known patterns\footnote{We can find some known patterns in \url{https://en.wikipedia.org/wiki/Spectral_line}.}.
Such activity is called \emph{spectrometry}.
Spectral lines also tell us what stars are made of.

It was 1925.
Heisenberg was trying to explain hydrogen \emph{spectral lines}.
While the \emph{old quantum theory} was trying to fix classical mechanics,
Heisenberg suggested that we start afresh instead \cite{heisenberg1925quantum}.
\footnote{\url{http://www.vub.ac.be/CLEA/IQSA/history.html}}

\Hyperlink{https://en.wikipedia.org/wiki/Matrix_mechanics}{Understanding matrix mechanics}

Born\textendash{}Heisenberg\textendash{}Jordan formulation:
\footnote{\url{http://fisica.ciens.ucv.ve/~svincenz/SQM333.pdf}}
infinite matrices.
What?

% https://en.wikipedia.org/wiki/Davisson%E2%80%93Germer_experiment
1923\textendash{}1927:
Davisson\textendash{}Germer experiment of de Broglie wavelength:

de Broglie wavelength is about the relationship between the momentum and the wave vector (and thus the wave number) of free particles.
\( p = \hbar k \).
\cite{okun2012abc}

1927: Davisson and Germer's electron diffraction experiment:

% https://en.wikipedia.org/wiki/History_of_quantum_field_theory
Meanwhile, Paul Dirac was trying to quantize the electromagnetic field.

% https://en.wikipedia.org/wiki/Quantum_electrodynamics
% https://en.wikipedia.org/wiki/Precision_tests_of_QED

% quantum mechanics in one dimension
% http://www.tcm.phy.cam.ac.uk/~bds10/aqp/handout_1d.pdf

% QM exercises
% https://www.math.temple.edu/~prisebor/qm1.pdf

% https://plato.stanford.edu/entries/qt-nvd/#3
% In a letter to Birkhoff from 1935, von Neumann says: “I would like to make a confession which may seem immoral: I do not believe in Hilbert space anymore”

%Heisenberg 1925 paper
%scan of English translation
%http://www.mat.unimi.it/users/galgani/arch/heis25ajp.pdf
%backup link
%http://fisica.ciens.ucv.ve/~svincenz/SQM261.pdf
%scan of German original?
%http://www.chemie.unibas.ch/~steinhauser/documents/Heisenberg_1925_33_879-893.pdf

%on that paper
%The 1925 Born and Jordan paper “On quantum mechanics”
%http://people.isy.liu.se/icg/jalar/kurser/QF/references/onBornJordan1925.pdf
% Heisenberg's 1924 paper
% Heisenberg\textendash{}Born\textendash{}Jordan "On quantum mechanics"
% http://people.isy.liu.se/icg/jalar/kurser/QF/references/onBornJordan1925.pdf


%2004 paper
%Understanding Heisenberg's 'Magical' Paper of July 1925: a New Look at the Calculational Details
%https://arxiv.org/pdf/quant-ph/0404009.pdf
%https://arxiv.org/abs/quant-ph/0404009

% https://en.wikipedia.org/wiki/Heisenberg_picture

1961: Clauss J\"onsson electron double-slit experiment:

% https://en.wikipedia.org/wiki/Interaction-free_measurement
% https://en.wikipedia.org/wiki/Elitzur%E2%80%93Vaidman_bomb_tester
% https://en.wikipedia.org/wiki/Mach%E2%80%93Zehnder_interferometer

Quantum physics and nuclear physics\footnote{\url{https://www.patana.ac.th/secondary/science/anrophysics/ntopic13/commentary.htm}}

\section{Working with complex numbers}

The \emph{imaginary number}\footnote{There is nothing imaginary about the imaginary number.
There is nothing real about the real number either.
Descartes coined these names, and they have stuck.}
is \( i = \sqrt{-1} \).
Thus, \(i^2 = -1\).

The set of \emph{complex numbers} is \( \Complex = \{ a + bi ~|~ a, b \in \Real \} \).

Example of complex number: \( 1 + 2i \).

Every real number is a complex number:
a real number is a complex number with zero imaginary part.
Thus, \( \Real \subset \Complex \).

\section{Working with complex vectors}

A \emph{complex vector} is a vector where every element is a complex number.
The set of \(n\)-dimensional complex vectors is \(\Complex^n\).

The \emph{dual} of a vector space \(V\) is \(V^*\)
which is obtained by transposing every element of \(V\).
Thus, every column vector in \(V\) becomes a row vector in \(V^*\).
Also, \((V^*)^* = V\).

The \emph{complex Hilbert space} is a subset of \(\Complex^\infty\).

\emph{Dirac bra-ket notation}:

\(\bra{A}\) (read ``bra-\(A\)'') is a row vector in the dual space of \(\Complex^\infty\).

\(\ket{B}\) (read ``ket-\(B\)'') is a column vector in \(\Complex^\infty\).

\(\braket{A}{B}\) is matrix multiplication \(\bra{A} \cdot \ket{B}\).
The result of \(\bra{A} \cdot \ket{B}\) is a complex number.

A ket represents a \emph{quantum state}.

\section{Working with operators}

% FIXME COIK
% https://en.wikipedia.org/wiki/Projective_Hilbert_space
% https://en.wikipedia.org/wiki/Quantum_state
A \emph{pure quantum state} is a ray in complex Hilbert space.

% https://en.wikipedia.org/wiki/Configuration_space_(physics)
A \emph{configuration} of a \(d\)-dimensional system with \(n\) particles consists of \(d \times n\) real numbers.
The \emph{configuration space} of the system is the set of all such points.

An \emph{operator} is a higher-order function?

\footnote{\url{https://en.wikipedia.org/wiki/Operator_(physics)\#Examples_of_applying_quantum_operators}}

Linear operator, adjoint, Hermitian, matrix

% https://en.wikipedia.org/wiki/Wave%E2%80%93particle_duality

For example, consider a ball falling near the ground.
How do we say it in the language of operators?

\section{Working with wave functions}

The \emph{wave function} of a system is a function that takes a configuration of the system and gives a complex number.

Example: the wave function of a system with \(n\) particles has the shape \(\psi(x_1,\ldots,x_n,t)\)
where every \(x_k : \Real^3\) and \(t:\Real\).

\emph{Born's statistical interpretation}:
The value of \( |\psi(c)|^2 \) is the \emph{probability density}
of measuring/finding/observing the configuration \(c\)?

Sobolev space?
\(L^2\) space?

The momentum of the object?

The position of the object?

% FIXME COIK
An \emph{observable} is a linear self-adjoint operator on a Hilbert space.

% https://en.wikipedia.org/wiki/Schr%C3%B6dinger_picture
The \emph{Schr\"odinger equation} constrains the wave function of a system?

% E.T. Jaynes's book
% http://bayes.wustl.edu/etj/prob/book.pdf

% https://en.m.wikipedia.org/wiki/Old_quantum_theory

% https://en.m.wikipedia.org/wiki/Transformation_theory_(quantum_mechanics)

% https://physics.stackexchange.com/questions/1417/general-relativity-gravitation-in-time-and-one-spatial-dimension

% Schutz's book?


An Introduction to Relativistic Quantum
Mechanics
I. From Relativity to Dirac Equation

% https://arxiv.org/pdf/0708.0052.pdf


% https://ocw.mit.edu/courses/physics/8-323-relativistic-quantum-field-theory-i-spring-2008/lecture-notes/


% https://en.wikipedia.org/wiki/Quantum_field_theory

Introduction to
Relativistic Quantum Field Theory
% http://www.tep.physik.uni-freiburg.de/lectures/QFT14/qft.pdf

% quantum mechanics made simple, weng cho CHEW
% http://wcchew.ece.illinois.edu/chew/course/QMALL20121005.pdf

% rather unstructured
% http://studylib.net/

\subsection{Modeling a lone one-dimensional free particle}

The wave function of one free (not influenced by any external fields) spinless one-dimensional particle is
\( \psi(x,t) = A \exp(k x - \omega t) \).

Momentum? Energy? Position?

\subsection{Modeling two one-dimensional particles}

Simulation hypothesis.

Assume the simulation hypothesis: the Universe is a simulation.

How do we overload it?
How do we crash it?
How do we glitch it?
How do we hack it?
How do we debug it?

A ket is a column vector. A bra is a row vector. The dual of a ket is its \emph{conjugate transpose}.
A ket represents a \emph{quantum state}?

quantum locking\footnote{\url{https://www.ted.com/talks/boaz_almog_levitates_a_superconductor/transcript}}
