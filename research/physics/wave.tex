\chapter{Wave and optics}

Differential equation? Diffusion? Oscillation? Wave behavior?

This chapter has two purposes:
introduce differential equations to the reader,
and prepare the reader for quantum mechanics.

\section{Wave}

% https://en.wikipedia.org/wiki/Wave
("Wave", Wikipedia):
A \emph{wave} is an oscillation accompanied by a transfer of energy?
A wave is a disturbance that transfers energy through matter or space?

\section{Detecting waves}

We can detect a wave by a diffraction slit.
If it's a wave, it diffracts.

We assume the converse: if it diffracts, it's very likely a wave.

\emph{Undulation} is an old term for \emph{wave}.

\section{Oscillation of a loaded spring}

If a spring is loaded, pulled, and released, then it will oscillate.

The equation of motion can be derived from Hooke's law of spring restoring force (\S\ref{sec:hooke-s-law}).

Wave equation

Second-order differential equation

Water wave

D'Alembert's waves?

How do we describe oscillation?
Periodic motion?
Harmonic motion?

How do we describe a diffusion?
How do we derive the wave equation from the diffusion equation?

% https://en.wikipedia.org/wiki/Fick%27s_laws_of_diffusion
% https://en.wikipedia.org/wiki/Continuity_equation
% https://en.wikipedia.org/wiki/Diffusion_equation#Derivation
Diffusion equation is derived from continuity equation and Fick's laws of diffusion.
Assume homogeneous (made of the same thing everywhere)
isotropic (behaving the same everywhere) medium?
Diffusion:
The rate of diffusion is proportional to the gradient?
\[
    f(x,t+h) - f(x,t) = c \cdot \frac{[f(x - h, t) - f(x,t)] + [f(x + h, t) - f(x,t)]}{2}
\]
Divide both sides by \(h\)
\[
    D_t f(x,t) = c \cdot D_x f(x,t)
\]
\[
    \frac{\partial f}{\partial t} = c \cdot \frac{\partial f}{\partial x}
\]
\[
    \frac{\partial f}{\partial t} = \vec{c} \cdot \nabla f
\]
???

How do we describe waves?

How do we describe waves on a string?
Pulse on a string?
Pulse on a chain of springs?
Replace the springs with more smaller springs?

% https://en.wikipedia.org/wiki/D%27Alembert%27s_formula

A function \(f\) has \emph{period} \(p\) iff \(f(x+p) = f(x)\) for all \(x\).

Let \(f(x,t)\) be the \emph{amplitude} of the wave at position \(x\) and time \(t\).

Let the oscillator be at position \(0\).

Let \(g\) be an unknown function.

Flow:
\(f(x,t + dt) - f(x,t) = g(c,f(x,t),f(x-dx,t),f(x+dx,t))\)

\(f(x,t + dt) - f(x,t) = [f(x-dx,t)-f(x,t)] + [f(x+dx,t)-f(x,t)]\)

\section{Velocities}

Propagation velocity

Phase velocity

Group velocity

\section{Light wave}

\section{Fermat's principle of least time}

Light takes the path that takes the least time.

\section{Snell\textendash{}Descartes law of refraction}

% https://en.wikipedia.org/wiki/Snell%27s_law
% https://en.wikipedia.org/wiki/Snell%27s_law#History
\begin{equation}
    \frac{\sin \theta_1}{\sin \theta_2} = \frac{v_1}{v_2} = \frac{\lambda_1}{\lambda_2} = \frac{n_2}{n_1}
\end{equation}

Descartes 1637 \emph{Dioptrics},

Huygens 1678: Huygens\textendash{}Fresnel principle.

Snell's law can be derived from Fermat's principle?

Snell's law can be derived from Huygens\textendash{}Fresnel principle?

% https://en.wikipedia.org/wiki/Huygens%E2%80%93Fresnel_principle

\section{Optics}

\emph{Wavenumber} is?
\emph{Wavelength} is?
\emph{Frequency} is?
\emph{Phase speed} is?
\emph{Group velocity} is?

\emph{Dispersion relation} is?

\emph{Doppler effect} is

Expanding universe?

\section{Reflection}
\section{Diffraction}
\section{Diffusion}
\section{Dispersion}
\section{Interference}
\section{Superposition}
\section{Fresnel spot}
\section{Transversal wave}
\section{Longitudinal wave}
\section{Young's double-slit experiment}
\section{Isochronic oscillation of a pendulum}

\section{Camera obscura}

\section{Newton's 1672 prism splits white light into colors?}

\section{Young's 1803 double-slit experiment}
