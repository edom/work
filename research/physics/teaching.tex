\chapter{Teaching physics}

CERN physics skill inventory\footnote{\url{http://ais.web.cern.ch/ais/apps/sti/examples_e.html}}

\section{Cognitive load}

It is important to minimize cognitive load.%
\footnote{\url{http://billblondeau.com/techwriter/good_technical_writing.htm}}
The cognitive load is minimum if it's something the student already knows.

Extraneous and germane cognitive load in psychology.
\footnote{\url{https://hobbolog.wordpress.com/2016/06/09/cognitive-load-theory-the-right-kind-of-load/}}
Similar to Fred Brooks' accidental and essential complexity in software.

\section{Exercises}

Exercises are vital for understanding.
Students must think for themselves.
They don't get anything by passive listening.

Exercises are checkpoints.

The teacher must make the student ask the question the teacher wants in the sequence the teacher wants.
The teacher must predict the student's question.
The book must predict the student's questions in the most intuitive sequence possible.
The book must cause the students to ask a question and answer it themselves.

The job of the teacher is
to plan a journey,
to stimulate the students,
to guide the students.

The readers can't ask the book if there is something they don't understand.
Therefore the book must handle all possibilities of the readers's failing to understand.

Study is doing the optimal sequence of \emph{exercises}.
The essence of teaching is \emph{crafting} such sequence.

\section{Introduce mathematics as we go}

It is easier for the student if the teacher
presents a concrete instance before the abstraction.

Avoid unused mathematics; don't just generalize a concept just because you can.
For example, in physics, vector spaces are mostly over \(\Real\) or \(\Complex\),
so we don't need to define vector space over a field (the algebraic structure).
We don't want the readers to take a long detour to abstract algebra.
We don't want to teach the readers anything that they are not likely to use.
We want to be as practical as possible, don't dumb down the material;
if the physics unavoidably requires mathematics in all its generality, then so be it.
We assume that the readers don't know mathematics, but want to learn mathematics,
but only want to learn mathematics that they will use.
We should not introduce mathematics that is not used anywhere else in this book.
We should not teach the wrong thing.
Where we diverge from mathematics, we shall indicate.

\section{How do we measure the student's mastery?}

Know? Able to use?

\section{What should we avoid?}

Avoid \enquote{popular} science books.
You are not their target audience.
The material is watered down.
They exist to entertain laypeople, not to teach researchers.

Ignore all books that have \emph{no} mathematics.

Ignore all books that have \emph{only} mathematics.

\section{How to teach yourself}

Learn how to learn.

Learn how to teach.

Alternate between being a teacher and being a student.

\section{People's ways of teaching}

\cite{kusse2010mathematical} motivates tensor by generalizing Ohm's law of electrical resistance
from discrete element to continuous medium.

\cite{scott2015student} is an exercise book for general relativity.
It's designed as a companion to \cite{schutz2009first}.

% Teaching general relativity to undergraduates
%http://physicstoday.scitation.org/doi/10.1063/PT.3.1605
%http://web.mit.edu/edbert/GR/gr1.pdf
%https://www.space.com/17661-theory-general-relativity.html

% Teaching special relativity
%https://manyworldstheory.com/2012/11/16/get-off-that-train-a-different-way-to-teach-special-relativity/

\section{Book reviews}

\subsection{Susskind\textendash{}Hrabovsky\textendash{}Friedman 2014 books}

Susskind and Friedman published \cite{susskind2014quantum}.
Susskind and Hrabovsky published \cite{susskind2014theoretical}.
Both books were published in 2014.

Leonard Susskind and George Hrabovsky's 2014 book
\emph{The theoretical minimum: what you need to know to start doing physics}
\cite{susskind2014theoretical}
begins by defining \emph{state spaces} and \emph{dynamical laws}.
Then it wastes at least five pages mathematicizing Aristotle.\footnote{\url{https://en.wikipedia.org/wiki/Chekhov\%27s_gun}}

It also teaches differential calculus, integral calculus, partial derivatives.

At page 90, it begins talking about \emph{phase spaces}.

What it does right is that it spreads lots of exercises throughout the book.

\section{What you need for what?}

If you want to build a nuclear fusion power plant, study nuclear physics.
What is nuclear physics?

What is energy? Ability to do work?

\section{Mathematical notation}

If \(f : \Real \to \Real\), then don't write \enquote{the function \(f(x)\)}
The function is \(f\).
The result of applying \(f\) to \(x\) is \(f(x)\).

\section{Physics curriculum around the world}

% https://en.wikipedia.org/wiki/Advanced_Placement
% https://en.wikipedia.org/wiki/AP_Physics_1
% https://en.wikipedia.org/wiki/AP_Physics_2

The USA have \emph{Advanced Placement} (AP).
High-school students can take AP exams to earn college credits while they are still in high school.

MIT? Harvard? Stanford?

\section{Indonesian}

\footnote{\url{https://www.itb.ac.id/news/read/5164/home/dosen-itb-luncurkan-buku-fisika-dasar-siap-unduh}}%
\footnote{\url{https://drive.google.com/file/d/0B3b8pBt2LxtWSkhCeC1nWmNXNFE/view}}

\section{Opinions}

On teaching mathematics, V. I. Arnold,
translated by A. V. Goryunov.%
\footnote{\url{https://www.uni-muenster.de/Physik.TP/~munsteg/arnold.html}}

The evolution of mathematics in the 20th century,
Michael Atiyah.%
\footnote{\url{https://web.math.rochester.edu/people/faculty/cmlr/Advice-Files/Atiyah-Mathematics.pdf}}

\section{Theory of learning}

\footnote{\url{https://en.wikipedia.org/wiki/Instructional_scaffolding}}%
\footnote{\url{https://en.wikipedia.org/wiki/Zone_of_proximal_development}}
