\chapter{Ramble}

\section{Books}

Study books:

Lev Okun's 2012 \emph{ABC of physics: a very brief guide} \cite{okun2012abc} has less than 200 pages.
The \enquote{formulas selected for this book are so simple that the knowledge of
elementary mathematics taught at high schools is sufficient
for understanding them} \cite[p.~vi]{okun2012abc}.
However, some parts are clear only if known.

\emph{History and evolution of concepts in physics}, 2014, Harry Varvoglis \cite{varvoglis2014history}.
With history, we can appreciate the work of others.
History explains why things are the way they are.
% It also has slides.
% http://www.tat.physik.uni-tuebingen.de/~varvoglis/

Thick books:

% http://www.lightandmatter.com/area1sn.html
% http://www.lightandmatter.com/books.html
\emph{Simple nature}, by Benjamin Cromwell.
This is a free book.

\emph{Fundamentals of physics extended}
(10th edition, by Halliday, Resnick, and Walker)
\cite{halliday2013fundamentals}

Book about physics experiments and experimental physics?

Landau\textendash{}Lifshitz books?

Reference books:

% http://202.38.64.11/~jmy/documents/ebooks/Hassani%20Mathematical%20Physics%20A%20Modem%20Introduction%20to%20Its%20Foundations%20-%20S.%20Hassani%20%5B0-387-98579-4%5D.pdf
The book \emph{Mathematical physics: a modern introduction to its foundations} (2013, Sadri Hassani)
\cite{hassani2013mathematical}
is for refreshing yourself about mathematical concepts that you have understood but you have forgotten,
not for studying physics from scratch.

\section{Unprocessed online resources}

Introduction to theoretical physics%
\footnote{\url{https://en.wikibooks.org/wiki/Introduction_to_Theoretical_Physics}}

Timeline of fundamental physics discoveries.%
\footnote{\url{https://en.wikipedia.org/wiki/Timeline_of_fundamental_physics_discoveries}}

Gerard 't Hooft has some advices on what to study to be a good theoretical physicist.%
\footnote{\url{http://www.staff.science.uu.nl/~gadda001/goodtheorist/index.html}}

% Landau\textendash{}Lifshitz book?
% https://www.reddit.com/r/Physics/comments/1dmxq7/our_beloved_landaulifshitz_books_are_available/
% https://archive.org/details/QuantumMechanics_104

% https://en.wikipedia.org/wiki/List_of_experiments#Physics

Quantum field theory%
\footnote{\url{http://www.thphys.uni-heidelberg.de/~weigand/QFT2-14/SkriptQFT2.pdf}}%
\footnote{\url{ftp://ftp.theorie.physik.uni-goettingen.de/pub/papers/rehren/qft13.pdf}}

Table of contents for College physics : a strategic approach / Randall D. Knight, Brian Jones, Stuart Field.
almost optimal order of materials
% http://catdir.loc.gov/catdir/toc/ecip072/2006032583.html

'big ideas'
% http://assets.pearsonschoolapps.com/asset_mgr/current/201412/pdf_145323.pdf

Wikipedia physics portal%
\footnote{\url{https://en.wikipedia.org/wiki/Portal:Physics/Navigation}}









Are the books recommended here concise enough?%
\footnote{\url{https://www.susanjfowler.com/blog/2016/8/13/so-you-want-to-learn-physics}}

philosophy of learning
amount learned is proportional to time put in
best way to learn is to figure out ideas yourself or teach them to someone else
the object of  a lecture is not so much to inform you of important facts, but rather to stimulate you to try to learn about some concept
\footnote{\url{https://ocw.mit.edu/ans7870/18/18.013a/textbook/chapter01/section01.html}}


Biochemistry: Wohler synthesized urea.

% https://en.wikipedia.org/wiki/Timeline_of_physical_chemistry
1869 Mendeleev periodic table

Alternating current flowing in wire emits EM radiation.






Constraint force acting on a ball rolling down a tilted plane.

Does powder/grain (such as sand) behave like an incompressible fluid? I guess they should. If they do, then liquid is a collection of tiny solids.

U-tubes

Stevin's law

% https://en.m.wikipedia.org/wiki/Communicating_vessels






Strong force keeps the atomic nucleus together. Electrostatic repulsion.


Constraint force

Every constraint force does zero work?

Example: The tension of pendulum string. Door hinge. Normal force on a box sliding down a tilted plane.

Related are the law of the lever and the conservation of energy.

The work done by \(F_k\) is \(W_k = F_k s_k\) where \(s_k = r_k da_k\).
W1 = W2
F1 r1 da1 = F2 r2 da2
F1 r1 da1 = F2 r2 da2



% https://physics.stackexchange.com/questions/12435/einsteins-postulates-leftrightarrow-minkowski-space-for-a-layman/13621#13621
\cite{dyson1972missed}



A Radically Modern Approach to Introductory Physics
% http://kestrel.nmt.edu/~raymond/books/radphys/book1/book1.html#x1-390004.1


physics compendium
Compendium of theoretical physics
% http://202.38.64.11/~jmy/documents/ebooks/Compendium%20of%20Theoretical%20Physics,Springer2005,529p,0387257993.pdf

concepts of modern physics
% http://web.pdx.edu/~pmoeck/lectures/beiser%206.pdf

Introduction to the Basic Concepts of Modern Physics
% http://www.springer.com/la/book/9788847006072


1910 Rutherford scattering:
Atom is mostly empty space?

hacking the quantum:
how anyone can become an amateur quantum physicist
% https://blogs.scientificamerican.com/critical-opalescence/hacking-the-quantum-a-new-book-explains-how-anyone-can-become-a-amateur-quantum-physicist/

% https://www.quora.com/What-is-the-best-way-to-self-teach-physics

% https://en.wikipedia.org/wiki/Double-slit_experiment

modern physics
% https://cnx.org/contents/rydUIGBQ@5.1:_dS0E2kQ@2/Introduction

openstax university physics 1, 2, 3%
\footnote{\url{https://cnx.org/contents/1Q9uMg_a@6.3:Gofkr9Oy@9/Preface}}%
\footnote{\url{https://cnx.org/contents/eg-XcBxE@4.1:Gofkr9Oy@9/Preface}}%
\footnote{\url{https://cnx.org/contents/rydUIGBQ@5.1:Gofkr9Oy@9/Preface}}

thermostatics%
\footnote{\url{http://www.ueltschi.org/teaching/chapthermostatics.pdf}}

\section{Discarded sources}

History of physics (abandoned work?)%
\footnote{\url{http://www.historyworld.net/wrldhis/PlainTextHistories.asp?groupid=2506&HistoryID=ac25&gtrack=pthc}}

Not useful, too wordy, no mathematics:%
\footnote{\url{http://lesswrong.com/lw/r5/the_quantum_physics_sequence/}}

\section{Ramble}

History of concepts%
\footnote{\url{http://www.springer.com/us/book/9783319042916}}

Understanding energy%
\footnote{\url{https://en.wikipedia.org/wiki/History_of_energy}}%
\footnote{\url{https://en.wikipedia.org/wiki/Conservation_of_Energy\#historical_information}}%
\footnote{\url{https://hsm.stackexchange.com/questions/414/when-were-the-modern-notions-of-work-and-energy-created}}%
\footnote{\url{https://en.wikipedia.org/wiki/Theory_of_heat}}

Synthetic physics?
Physics from first principles?

("What If Quantum Theory Violates All Mathematics?")%
\footnote{\url{https://www.degruyter.com/view/j/phys.2017.15.issue-1/phys-2017-0069/phys-2017-0069.xml?format=INT}}
What are the Bell inequalities?
Why are they famous?

The principle of relative locality%
\footnote{\url{https://arxiv.org/PS_cache/arxiv/pdf/1101/1101.0931v2.pdf}}
What is this?

\section{Irrelevant information}

% https://en.wikipedia.org/wiki/Mousetrap_car
% https://en.wikipedia.org/wiki/Torsion_spring
Torsion spring, angular form of Hooke's law

% https://en.wikipedia.org/wiki/Screw_theory

\section{Professions and their concerns}

\subsection{Mathematicians}

\subsection{Physicists}

Physicists aim to find out the truth.

\subsection{Engineers}

Engineers cares about models that allow them to do their work with acceptable error.

Engineers apply knowledge to solve practical problems.

\section{Civil construction}

\section{Burying water pipes and electrical cables}

Why do we bury pipes?
To avoid freezing.

However, in Jakarta, why do we bury pipes?
There is no reason.
Thus pipes should, for easier maintenance.
My house had an unknown pipe leak.
To trace that, one would have to dig 2 meters below the ground.
One would have to demolish all the floor.

Thus, in tropical climates, pipes should be above the ground.

The same logic goes for cables.
Cables should go below the ceiling, above the ground, and on the wall, not in the wall.
Don't bury cable in the wall.

If you mind the sight, use a portable temporary wall.

\section{Consciousness}

Consciousness requires memory.

Consciousness requires feedback.

I move my hand, and I see my hand move,
therefore I infer that I can control my hand,
and therefore I infer that my hand is part of my self.

My \emph{self} is \emph{everything} I can control.

If another person \(B\) absolutely obeys my orders,
then \(B\) is a part of my self.

I know the extent of my self by experimenting.
I find out what I can control and what I can't.
The part I can control, I call my \emph{self}.

The self of \(A\) is everything that \(A\) can directly control.
If \(A\) is driving a car, the car becomes part of \(A\)'s self, until he stops driving.

\section{Digression: law of physics}

A \emph{law of physics} is a mathematical statement.
\enquote{The laws of physics are the same in all frames} means that the statement has the same forms.
For example, \(F = m a\) has the shape \(F' = m' a'\) in another inertial frame.

Two people can see the same thing from different point of view.
What does the other person see?
Where and when are things, from another point of view?

\section{Structure}

Each section title must be the goal of that section.

Bloom's taxonomy of learning\footnote{\url{http://edutechwiki.unige.ch/en/Learning_level\#Blooms_taxonomy}}

\section{Other resources on the Internet}

If an Internet resource (such as Wikipedia)
happens to teaching something well,
then we should refer to it instead of duplicating the effort.

outline of physics\footnote{\url{https://en.m.wikipedia.org/wiki/Outline_of_physics}}

Theoretical physics\footnote{\url{https://en.wikipedia.org/wiki/Theoretical_physics}}
\footnote{\url{https://en.m.wikipedia.org/wiki/Conceptual_physics}}

\Hyperlink{https://en.wikipedia.org/wiki/Physicist}{Physicist, career, award}

\Hyperlink{https://en.wikipedia.org/wiki/Branches_of_physics}{Branches of physics}

Forget about prizes.
Science is a collaboration, not a competition.
The only way to win the prize is to win the opinion of the judges,
and you don't know the judges.

What do physicists think about nature in 2017?

How can we contribute to physics?
