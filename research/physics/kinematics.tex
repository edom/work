\chapter{Kinematics, and the calculus of infinitesimals}
\label{sec:derivative}

Kinematics is the mathematics that describes motions.

A \emph{frame} defines \emph{where} and \emph{when}.

\emph{Motion} is change of position.%
\footnote{\url{https://en.wikipedia.org/wiki/Motion_\%28physics\%29}}

An object \emph{moves} iff its position changes.

The \emph{speed} of an object is how fast it moves:
how far it moves in how much time.
\emph{Fast} means high speed,
going far in little time,
traveling much distance in little time.

\emph{Average speed} is distance traveled divided by time required.

\emph{Velocity} is the rate of change of position.
Speed is the magnitude of velocity.
\emph{Rate of change} is defined by \emph{derivative} (\S\ref{sec:derivative}).

\section{Describing motions using equations relating position and time}

We can use a position function.
Its type is \( \Real \to V \).

An example of an equation of motion is \( x(t) = 2 t e_1 \).
It describes an object that moves with constant velocity \(2\) towards the positive x-axis.
An \emph{equation of motion} is an equation that describes
the motion of an object by relating time and position.

Each equation of motion corresponds to a moving object.

To describe more objects, use more equations.

Let \(e\) be a linear basis.
Suppose that the position of an object at time \(t\) is
\(x(t) = e(x_1(t), \ldots, x_n(t))\).
Then the velocity at time \(t\) is \(v(t) = \der(x,t) = e(v_1(t), \ldots, v_n(t)) \).
Can we say that \(v_k(t) = \der(x_k,t)\)?

Moral of the story:
If we have a linear basis,
then doing calculus on the coordinates
is doing calculus on the vectors.

\section{Notating derivatives}

Let \(f : \Real \to \Real\).

\paragraph{\der}

We can describe the derivative by a function \(\der : (\Real \to \Real) \to (\Real \to \Real)\).
\[
    \der(f,x) = \StandardPart\parenthesize{\frac{f^*(x+\delta)-f^*(x)}{\delta}}
\]
where \(f^*\) is the natural extension of \(f\) to the hyperreals.%
\footnote{\url{https://en.wikipedia.org/wiki/Non-standard_calculus\#Definition_of_derivative}}
Then \(\der(f,x)\) is the slope of the tangent line of \(f\) at \(x\).

Strictly, \(\der(f)(x)\), not \(\der(f,x)\).

Thus \(f' = \der(f)\).

\(\dd{x}\) means an infinitesimal change in \(x\)?
What does that mean?
What is \(x\)?

\paragraph{Euler\textendash{}Arbogast D-notation}

% https://en.wikipedia.org/wiki/Notation_for_differentiation#Euler.27s_notation

\( D_x E \) is the derivative of expression \(E\) with respect to variable \(x\).
The variable \(x\) should occur free in \(E\).

Typing rule:
If \(x:\Real\) and \(E:\Real\), then \(D_x E:\Real\).

The notation \(E[x:=y]\) means \(E\) but with each free occurrence of \(x\) replaced with \(y\).

Then \( D_x E = \lim_{h \to 0} \frac{E[x := x+h] - E}{h} \).

Then \( D_x E = d(x \to E) \).

Example: \( D_x (4x^2) = d(x \to 4x^2) = 8x \).

Example: \( D_y (x + y) = d(y \to x + y) = x + 1 \).

Advantage: With \(D_x\) notation, we can refer to an input by name;
With \(d_k\) notation, we can refer to an input by index only.

\(D\) is a custom syntax, not an ordinary function.

% Multivariate differential calculus
% Vector calculus

\section{Relating velocities, tangent lines, and derivatives}

There are several ways of understanding \(f'(x)\) (the derivative of \(f\) at \(x\)):
\UnorderedList{
    \item rate of change of \(f\) at \(x\); instantaneous velocity
    \item slope of the tangent line of \(f\) at \(x\)
    \item best linear approximation (not discussed here)
}

\paragraph{Average velocity and the secant line}

Let there be an object.

Let \(x(t) : V^2\) be a vector that describes its position at time \(t : \Real\).

The \emph{average velocity} of that object in the time interval \([t,t+\Delta t]\) is
\[ \frac{x(t+\Delta t) - x(t)}{\Delta t}. \]

If at time \(t_1\) its position is \(x_1\)
and at time \(t_2\) its position is \(x_2\),
then its \emph{average velocity} in the time interval between \(t_1\) and \(t_2\)
is \((x_2 - x_1) / (t_2 - t_1)\).

A \emph{secant line of \(f\)} is a line that passes \((x_1,f(x_1))\) and \((x_2,f(x_2))\).
Think of average velocity.

\paragraph{Instantaneous velocity and the tangent line}

If the position of an object at time \(t\) is \(x(t)\),
then its \emph{instantaneous velocity} at time \(t\) is \(v(t) = (d(x))(t)\).
The velocity function is the derivative of the position function.

The term \emph{instantaneous velocity} is often shortened to just \emph{velocity}.

The unqualified \emph{velocity} means \emph{instantaneous velocity}.

A car's speedometer measures its instantaneous speed.

Derivative is about \emph{rate of change}:
how fast a function changes value,
how big is the change in output compared to the change in input.

Consider a function \(f : \Real \to \Real\).
If the input is \(x\), then the output is \(f(x)\).
If you change the input by \(\dd{x}\), the output changes by \(\dd{y}\).
Formally, \(f(x+\dd{x}) = f(x)+\dd{y}\).

A \emph{tangent line of \(f\) at \(x\)} is what the secant line converges to
if both \(x_1\) and \(x_2\) converge to \(x\).
Think of instantaneous velocity.

\paragraph{Understanding the derivative as the slope of the tangent line}

The \emph{derivative of \(f\) at \(x\)} is the slope of the tangent line of \(f\) at \(x\).
Reminder: The line \(y = mx + c\) has slope \(m\).

\section{Describing motions using implicit equations}

An example of \emph{implicit} equation is \( x(t) = - (d(d(x)))(t) \).
This is also an example of a \emph{differential equation} because it contains the derivative operator \(d\).
Differential equations are discussed in \S\ref{sec:diff-eqn}.

\section{Integrals}

We can think of an integral in several ways:
\UnorderedList{
\item area under a curve
\item slicing and summing
}

\paragraph{Integrating by slicing and summing}

\section{Calculating derivatives and integrals quickly}

\paragraph{Calculating derivatives}

Symbolic calculation is enabled by
the \emph{constant rule} \eqref{der-constant-rule},
the \emph{power rule} \eqref{der-power-rule},
the \emph{product rule} \eqref{der-product-rule},
and the \emph{chain rule} \eqref{der-chain-rule}.
\begin{align}
    d (x \to y) &= 0 \text{ if \( y \) is a constant} \label{der-constant-rule}
    \\ d (x \to x^p) &= p \cdot x^{p-1} \text{ if \( p \) is a non-zero real number} \label{der-power-rule}
    \\ d(f \cdot g) &= d(f) \cdot d(g) \label{der-product-rule}
    \\ d(f \circ g) &= d(g) \cdot (d(f) \circ g) \label{der-chain-rule}
\end{align}

The \(d\) operator is linear:
If \(c\) is a constant, then \(d(c \cdot f) = c \cdot d(f)\).
Also, \(d(f+g) = d(f) + d(g)\).

\paragraph{Calculating integrals}

With the \emph{fundamental theorem of calculus},
we can compute integrals using antiderivatives.
With this shortcut, we can skip the slicing and summing.

\Formula{
    \int_{[a,b]} f(x) \dd{x} \dd{y} \dd{z} = F(b) - F(a)
}

\section{Exercise}

Compute the derivative of each of these functions:
\( x \to 1 \), \( x \to x \), \( x \to 2x \), \( x \to x^2 \),
\( x \to 3e^x \), \(x \to e^x \cdot x \).

Using the power rule, show that \( d(x \to x^2) = 2x \).

Show that \( d(x \to x^3 + x^2) = 3x^2 + 2x \).
