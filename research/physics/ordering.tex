\chapter{Ordering of materials}

You don't have to read this chapter which
explains the reason for the ordering of materials in this book.

\section{Appetizer}

High school math.

Kinematics is discussed before the rest of mechanics.
Kinematics and dynamics motivate infinitesimal calculus and systems of differential equations.
It is more productive to think in infinitesimals because infinitesimals motivate differential geometry.

The historical ordering of materials is not always optimal for learning.

\section{Tensor}

Tensors have to be introduced early because all modern theories seem to use it.
Example theories are quantum field theories (quantum electrodynamics, quantum chromodynamics).

Thus we begin the book with relativity.

There are several routes to tensors:

\UnorderedList{
\item from differential geometry, via Riemann curvature tensor
\item from vectors and matrices by generalization as multi-dimensional array of numbers
\item from statics, via Cauchy stress tensor
}

\section{Electromagnetism}

Perhaps the easiest to electromagnetics is
from gravitostatics to electrostatics by analogy between mass and electric charge,
then to magnetostatics,
then to electrodynamics,
then to electromagnetism.

Another route is from hydrostatics to electrostatics by analogy flow-current pressure-voltage.

However, all such analogies break down when we have to explain electromagnetic radiation.

\section{Quantum mechanics}

There are several routes to quantum mechanics.

The historical way:
old quantum theory, de Broglie wavelength, Bohr hydrogen atom model, Heisenberg matrix mechanics.
This isn't good for teaching.
The old quantum theory isn't coherent.

From state spaces, quantum states, spins, qubits \cite{susskind2014quantum}.
The student can skip classical mechanics.

From wave functions, Schr\"odinger equation.
From studying the wave function of one-dimensional spinless particle.

From Hamiltonian mechanics, operator theory.

From Poisson brackets? Commutators?

From photoelectric effect, from black-body radiation.

The Wikipedia way\footnote{\url{https://en.m.wikipedia.org/wiki/Introduction_to_quantum_mechanics}}

Tensors are introduced before QFT because QFT is formulated in tensors.
