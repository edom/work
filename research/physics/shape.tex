\chapter{Shapes}

\section{Angles}

Two intersecting lines form an angle\footnote{\url{https://en.wikipedia.org/wiki/Angle}}.

Two lines are parallel iff they don't intersect.

Two lines coincide iff the angle between them is zero.

Two lines are orthogonal iff the angle between them is a right angle.

We use a protractor\footnote{\url{https://en.wikipedia.org/wiki/Protractor}} to measure an angle in degrees.
A full circle is 360 degrees,
a half circle is 180 degrees,
a quarter circle is 90 degrees,
and so on.

A full turn is 360 degrees,
a half turn is 180 degrees,
a quarter turn is 90 degrees,
and so on.

\section*{Degrees and radians}

Degrees are convenient for manual calculation
because 360 is divisible by several small integers.

Radians\footnote{\url{https://en.wikipedia.org/wiki/Radian}} simplify formulas.%
\footnote{\url{https://en.wikipedia.org/wiki/Radian\#Advantages_of_measuring_in_radians}}
For example,
let \(c\) be the circumference of
a circular sector of angle \(a\) and radius \(r\).
If \(a\) is in radians, then \(c\) has the simple formula \(c = a \cdot r\),
but if \(a\) is in degrees, the formula becomes \(c = (a / \ang{360}) \cdot r\).

We prefer the unit that simplifies our jobs.
Engineers prefer degrees.
Mathematicians prefer radians.
Both units are widely used,
so let's learn to convert one to the other.

An angle of \(2\pi\) radians is equal to an angle of \(360\) degrees.%
\footnote{\url{https://en.wikipedia.org/wiki/Radian\#Conversion_between_radians_and_degrees}}
Both of them are equal to one full turn:
\Formula{
    \frac{d}{360} = \frac{r}{2\pi}
}

To convert \(r\) radians to \(d\) degrees:
\Formula{
    d = \frac{360}{2\pi} \cdot r
}

To convert \(d\) degrees to \(r\) radians:
\Formula{
    r = \frac{2\pi}{360} \cdot d
}

\ExerciseAnswer{Convert \ang{360} to radians?}{\(2\pi/1\) radians.}
\ExerciseAnswer{Convert \ang{180} to radians?}{\(2\pi/2\) radians.}
\ExerciseAnswer{Convert \ang{90} to radians?}{\(2\pi/4\) radians.}
\ExerciseAnswer{Convert \ang{45} to radians?}{\(2\pi/8\) radians.}

\ShowAnswers

\section{Circles}

\footnote{\url{https://en.wikipedia.org/wiki/Circular_sector}}

\section{Triangles}

A vertex is a point where two sides meet.

A triangle is called a triangle because it has three angles.
A triangle also has three vertices and three sides.

The sum of all interior angles of a triangle is 180 degrees.

An equilateral triangle is a triangle whose sides have the same length.%
\footnote{\url{https://en.wikipedia.org/wiki/Equilateral_triangle}}
Each interior angle of such triangle is \ang{60}.

\enquote{Equilateral} is the Latinate of \enquote{same-sided}.

The study of triangles is called \enquote{trigonometry}.

\paragraph{Labeling a triangle}

A capital letter labels an interior angle.
The corresponding small letter labels the side across the angle.
For example, \(a\) is the side across the angle \(A\).

\subsection*{Drawing a standard right triangle}

A right angle is 90 degrees.%
\footnote{\url{https://en.wikipedia.org/wiki/Right_angle}}

A right triangle is a triangle that has a \ang{90} interior angle.

See the footnote\footnote{\url{https://commons.wikimedia.org/wiki/File:Rtriangle.svg}}
for the picture of a standard right triangle.
Here we describe that triangle.

Let there be a triangle \(ABC\).

Let the angle \(C\) be a right angle.

Let \(a\) be the side across angle \(A\).

Let \(b\) be the side across angle \(B\).

Let \(c\) be the side across angle \(C\).
Thus, \(c\) is the hypotenuse\footnote{\url{https://en.wikipedia.org/wiki/Hypotenuse}},
the longest side of a right triangle,
the side across the right angle.

Such triangle is called a standard right triangle.

\section*{Triangle side ratios}

Consider a standard right triangle.

The sine of the angle \(A\) is \(\sin(A) = a/c\).

The cosine of the angle \(A\) is \(\cos(A) = b/c\).

The tangent of the angle \(A\) is \(\tan(A) = a/b\).

\section*{Trigonometric identities}

We can show these by drawing:
\( \sin(0) = 0 \),
and \( \cos(0) = 1 \),
and \( \sin(\pi/2) = 1 \),
and \( \cos(\pi/2) = 0 \).

The Pythagorean theorem implies \((\sin(A))^2 + (\cos(A))^2 = 1\).

The sine function has a period of \(2\pi\).
It means that \(\sin(a + 2\pi) = \sin(a)\) for every real number \(a\).

We have \(\cos A = \sin\left(\frac{\pi}{2}-A\right)\)
because \(A+B+C = \pi\) and \(C = \pi/2\) and \(A+B = \pi/2\).

See also Wikipedia\footnote{\url{https://en.wikipedia.org/wiki/Special_right_triangle}}%
\footnote{\url{https://en.wikipedia.org/wiki/Right_triangle}}.

\section*{Inverse trigonometric functions}

We can measure an angle by the ratio of the sides of the right triangle formed by the angle.
We use inverse trigonometric functions (inverse sine, inverse cosine, inverse tangent).
