\chapter{Transformations}

\section{How transforming a vector transforms its coordinates}

We can imagine rotating a vector by 30 degrees,
but how do the coordinates change?

Let \(f : V \to V\) be a vector transformation.

Let \(T : V \to V\) be a linear function.

Let \(U : \Real^n \to \Real^n\) be a linear function.

Let \(v:V\) be a vector.

Let \(e:\Real^n \to V\) be a basis.

If \(v\) changes to \(T(v)\), then the \(e\)-coordinates change from \(x\) to \(U(x)\).

Let \(x = (x_1,\ldots,x_n)\).

We can explode \(U : \Real^n \to \Real^n\) into \(n\) functions
\(U_1,\ldots,U_n\), each of type \(\Real^n \to \Real\),
so that we can write \(U\) out as
\Formula{
    U(x) = (U_1(x), \ldots, U_n(x))
}

\(T\) is a vector transformation.
\(U\) is the coordinate transformation that corresponds to \(T\).

Let \(E\) be the basis.

Remember that we can write a vector as \(v = E(x)\).

Suppose that we transform a vector from \(v\) to \(T(v)\).

Thus \(T(v) = T(E(x)) = E(U(x))\).

Thus \(T \circ E = E \circ U\).

\section{Rotation in two-dimensional orthonormal basis}

The vector \(a\) rotated by \(t\) radians counterclockwise is the vector \(b\) having the same length
but such that the angle from \(a\) to \(b\) is \(t\).

Let \(\{i,j\}\) be an orthonormal basis.

It can be geometrically shown that
the result of rotating the vector \(xi+yj\) by \(a\) radians
is the vector \(x'i+y'j\) where
\begin{align}
    x' &= \cos(a) \cdot x - \sin(a) \cdot y
    \\ y' &= \sin(a) \cdot x + \cos(a) \cdot y
\end{align}
which we can write as the matrix equation
\Formula{
    \Matrix{x' \\ y'}
    = \Matrix{
        \cos(a) & -\sin(a)
        \\ \sin(a) & \cos(a)
    }
    \Matrix{x \\ y}
}
for which we can also write \( \text{rotate}(a, xi + yj) = x'i + y'j \).
We can also write \( \text{rotate}(e, a, (x,y)) = (x',y')\).
We can also write \( \text{rotate}(e, a, X) = R(a) \cdot X\).

If the basis is orthonormal, then we can define the rotation matrix
\Formula{
    R(a) = \Matrix{
        \cos a & -\sin a
        \\ \sin a & \cos a
    }
}
so that we can write the rotation as matrix multiplication
\( X' = R(a) \cdot X \).

Thus changing the vector \(v\) to \(\text{rotate}(a,v)\)
changes the \(e\)-coordinate tuple \(x\) to \(R(a) \cdot x\).

See also Wikipedia%
\footnote{\url{https://en.wikipedia.org/wiki/Rotation_(mathematics)\#Two_dimensions}}%
\footnote{\url{https://en.wikipedia.org/wiki/Rotation_matrix\#In_two_dimensions}}%
.

\section{Transforming the basis}

We have just rotated a vector.
What if we rotate the basis (the coordinate axes) instead?

See also Wikipedia%
\footnote{\url{https://en.wikipedia.org/wiki/Active_and_passive_transformation}}%
\footnote{\url{https://en.wikipedia.org/wiki/Change_of_basis}}%
\footnote{\url{https://en.wikipedia.org/wiki/Rotation_of_axes}}%
.

\section{Change of basis}

Let \( e : \Real^n \to V \) be a basis.

For example, if \(e(x,y) = xi + yj\).

Let \(t\) be rotation of 90 degrees counterclockwise.

We can think of a change of basis as an invertible function \( t : \Real^n \to \Real^n \).
Then, we can change the basis from \(e\) to \(e \circ t\).

We have a vector \(v : V\).
Its coordinate tuple under basis \(e\) is \(x\).
Its coordinate tuple under basis \(e \circ t\) is \(x'\).
\begin{align*}
    v &= v
    \\ e(x) &= (e \circ t)(x')
    \\ e(x) &= e(t(x'))
    \\ x &= t(x')
    \\ t^{-1}(x) &= x'
\end{align*}

Thus a vector is contravariant.

\section{Coordinateless and coordinateful description of scalar fields}

Let \(f : V \to \Real\) be a scalar field.
We call \(f\) coordinateless because \(f\) does not use any coordinates.

A coordinateful description of \(f\) under basis \(e : \Real^n \to V\) is
another function \(g : \Real^n \to \Real\)
such that \(g = f \circ e\).

How do we describe derivatives coordinatelessly?

\section{Keeping a function while changing basis: contravariance and covariance}

Let there be two functions \( p : \Real \to V \) and \( \phi : V \to \Real \).
Observe how their types mirror each other.
We will explain how \( p \) is contravariant and \( \phi \) is covariant.

An example of \(p\) is a parametric curve.

An example of \( \phi \) is a scalar field such as a temperature field,
which maps each point in space to the temperature at that point.
Another example is a height map,
which maps each point in space to the height of the terrain at that point.

Let \( p_e, p_f : \Real \to \Real^n \) be coordinateful descriptions of \( p \).
Let \(\phi_e : \Real^n \to \Real \) describe \(\phi\) using the basis \(e\).
Let \(\phi_f : \Real^n \to \Real \) describe \(\phi\) using the basis \(f\).
The subscript denotes the basis.
\begin{align*}
    e(p_e(t)) &= p(t)
    \\ f(p_f(t)) &= p(t)
    \\ \phi(v) &= \phi_e(e^{-1}(v))
    \\ \phi(v) &= \phi_f(f^{-1}(v))
\end{align*}

Even though \(p_e\) and \(p_f\) are different functions,
they describe the same coordinateless function \(p\), only with different bases.
We want to change the basis from \(e\) to \(f\), but we want \(p_f\) to describe \(p\).

We write \(f \equiv g\) to mean that \(f\) gives the same result as \(g\) for all parameters.
We have discussed this in \SectionRef{sec:function-equivalence}.
Thus, we can tidy up the equations as
\begin{align*}
    e \circ p_e &\equiv p
    \\ f \circ p_f &\equiv p
    \\ \phi &\equiv \phi_e \circ e^{-1}
    \\ \phi &\equiv \phi_f \circ f^{-1}
\end{align*}
Observe the inverses.

Let \( m : \Real^n \to \Real^n \) be the function that changes the basis from \( e \) to \( f \).
This means \( f \equiv e \circ m \).
(The other possibility \(m \circ e\) does not make sense because the types conflict.)

Then, observe how we have to change the coordinates to keep \(p\) and \(\phi\) the same:
\begin{align*}
    p &\equiv p
    &
    \phi &\equiv \phi
    \\
    e \circ p_e &\equiv f \circ p_f
    &
    \phi_e \circ e^{-1} &\equiv \phi_f \circ f^{-1}
    \\
    e \circ p_e &\equiv (e \circ m) \circ p_{e \circ m}
    &
    \phi_e \circ e^{-1} &\equiv \phi_{e \circ m} \circ (e \circ m)^{-1}
    \\
    e \circ p_e &\equiv e \circ m \circ p_{e \circ m}
    &
    \phi_e \circ e^{-1} &\equiv \phi_{e \circ m} \circ m^{-1} \circ e^{-1}
    \\
    p_e &\equiv m \circ p_{e \circ m}
    &
    \phi_e &\equiv \phi_{e \circ m} \circ m^{-1}
    \\
    m^{-1} \circ p_e &\equiv p_{e \circ m}
    &
    \phi_e \circ m &\equiv \phi_{e \circ m}
\end{align*}

Therefore, summarizing, we get
\begin{align}
    m^{-1} \circ p_e &\equiv p_{e \circ m}
    \\ \phi_e \circ m &\equiv \phi_{e \circ m}
\end{align}

We say that \(p\) is contravariant because of the \(m^{-1}\).

We say that \(\phi\) is covariant because of the \(m\).

A vector is contravariant.

Functions of type \( \Real \to V \) are contravariant.

Functions of type \( V \to \Real \) are covariant.

\section{Einstein notation?}

\paragraph{Subscripts and superscripts for coordinate tuple components?}

A vector is \(v = e(x^1,\ldots,x^n)\).

A parametric curve is \(p(t) = e((p_e(t))^1, \ldots, (p_e(t))^n)\).
We lift it to \(p = E(q^1, \ldots, q^n)\).

A covector is \(\phi(v) = \phi_e(x_1,\ldots,x_n)\).

We overload the superscript, which unfortunately has been used to mean raising a number to a power.
Don't confuse contravariant tuple component notation \(x^3\)
with power notation \(2^3 = 2 \times 2 \times 2\).
