\chapter{One-dimensional spaces}

We can divide one-dimensional spaces into two categories:
\emph{lines} and \emph{curves}.
Lines are straight.
Curves may bend.
Thus every line is a curve.

\ResearchQuestion{%
Drawing a line is easy,
but how do we describe a line \emph{algebraically}?
}

\ResearchQuestion{%
A line is a straight one-dimensional space.
\emph{Straight} is defined by the ambient space that contains the line.
How do we define \emph{straight}?
}

\section{Defining a line}

If two points \(a\) and \(b\) are on a line,
then their midpoint \((a + b) / 2\) is also on the line.

(These aren't obvious to the uninitiated?)

If both \(a\) and \(b\) are on a line,
then the point \(k \cdot (b - a) + a\) is also on the line, for every \(k:\Real\).

If three points \(a,b,c\) are on a line,
then the displacements \(b-a\) and \(c-b\) are parallel.

\paragraph{Infinite extension of a line segment}
A \emph{line segment} is what is drawn using a straightedge.
A \emph{line} is obtained by infinitely extending a line segment in both directions.

\paragraph{Embedding of \(\Real\)}
\emph{A line is a straight embedding of \(\Real\).}
A line is something straight-shaped and isomorphic to \(\Real\).

But there is a problem with that definition.
That definition may include a pathological sheet, which should be a two-dimensional object.
This is a mapping from \(\Real^2\) to \(\Real\):
Let there be two real numbers \(a\) and \(b\).
Define the number \(c\) as \(\ldots a_1 b_1 a_0 b_0 . a_{-1} b_{-1} a_{-2} b_{-2} \ldots\).%
\footnote{\url{https://math.stackexchange.com/questions/75107/injective-map-from-mathbbr2-to-mathbbr}}%
\footnote{\url{https://math.stackexchange.com/questions/183361/examples-of-bijective-map-from-mathbbr3-rightarrow-mathbbr}}

We can define a line by a parametric equation: \( x(k) = k g + p \).

We can define a line by an algebraic equation: \( a \cdot x + b = 0 \).

\subsection{Defining a line as a curve with constant velocity}

The \emph{velocity} of the curve \( x : \Real \to \Real^n \) is the derivative of \(x\).
The velocity of \(x\) is the rate of change of \(x\).

\emph{A line is a curve whose velocity is constant.}

\subsection{Defining a line as a geodesic}

\section{Describing lines in a two-dimensional ambient space}

Every line can be described as the set
\( \{ (x,y) ~|~ (x,y) \in \Real^2, ~ a x + b y = c \} \).
(Why?)

\subsection{Describing a line that passes two points}

Describe a line that passes \((x_1,y_1)\) and \((x_2,y_2)\).

The description is
\begin{align*}
    a x_1 + b y_1 &= c
    \\
    a x_2 + b y_2 &= c
\end{align*}
Rearrange:
\begin{align*}
    x_1 a + y_1 b &= c
    \\
    x_2 a + y_2 b &= c
\end{align*}
Solve for \(a,b,c\).
\begin{align*}
    \Matrix{x_1 & y_1 \\ x_2 & y_2} \Matrix{a \\ b} = \Matrix{c \\ c}
\end{align*}
We can solve it using GNU Octave by typing \verb@[x1,y1;x2,y2] \ [c;c]@
but we have to substitute the variables with numbers first.

\subsection{Finding the angle formed by two lines}

\subsection{Translating lines and describing parallel lines}

Translating a line produces another line that is parallel to the original line.

Two lines \(ax+by=c\) and \(a'x+b'y=c'\) are parallel iff \(\abs{a/b} = \abs{a'/b'}\)?

\subsection{Describing orthogonal lines}

This is important for tangents, normals, and osculating circles.

\section{Describing higher-dimensional lines}

Every \(n\)-dimensional line can be described as
\( \{ k g + p ~|~ k \in \Real \} \)
if \(g, p : \Real^n\).

Describe a line that passes \(x_1\) and \(x_2\).
The description is
\begin{align*}
    x_1 &= k_1 g + p
    \\
    x_2 &= k_2 g + p
\end{align*}

\begin{align*}
    x_i - p &= k_i g
\end{align*}

How do we solve the equation \(a = kb\) if \(a,b\) are vectors and \(k\) is a scalar?
