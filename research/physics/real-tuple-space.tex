\chapter{Real tuple spaces}

Earlier, we defined \enquote{space} as another word for \enquote{set} because we want
that definition to include every real tuple space.

The set \(\Real^n\) is the set of all real \(n\)-tuples.

The dimension of \(\Real^n\) is \(n\).

An example element of \(\Real^3\) is \((1,2,3)\).

\section{An example basis}

Imagine a flat sheet of paper.

Draw a point \(A\).

Draw a vector named \(i\), from \(A\), \SI{1}{cm} long, pointing right.

Draw another vector named \(j\), also from \(A\), \SI{1}{cm} long, but pointing up.

Thus, the vectors \(i\) and \(j\) are orthogonal.

Then, we declare the basis
\( e : \Real^2 \to E^2 \) as \( e(x,y) = xi + yj \).%

\section{Cartesian coordinate system}

Let \(E^n\) mean the \(n\)-dimensional Euclidean space.

The \(n\)-dimensional Cartesian coordinate system matches
a point in \(E^n\) and a point in \(\Real^n\).

\footnote{\url{https://en.wikipedia.org/wiki/Cartesian_coordinate_system}}

For example...

What mapping is drawn in \FigureRef{fig:a-2d-cartesian-coordinate-system}?

\begin{figure}[h]
    \centering
    \url{https://en.wikipedia.org/wiki/File:Cartesian-coordinate-system.svg}
    \caption{A two-dimensional Cartesian coordinate system}
    \label{fig:a-2d-cartesian-coordinate-system}
\end{figure}

\section{Example of geometry with numbers: describing a circle in \(\Real^2\)}

Every variable is a real number unless specified otherwise.

Assume a two-dimensional Cartesian coordinate system.

Consider a circle \(C\) with center \((0,0)\) and radius 1.
That circle can be described in two ways:
the algebraic description \eqref{eq:set-circle-algebraic},
and the parametric description \eqref{eq:set-circle-parametric}.
Both sets have the same members and describe the same circle.
\begin{align}
    \label{eq:set-circle-algebraic}
    C_{\text{algebraic}} &= \{ (x,y) ~|~ x^2 + y^2 = 1 \}
    \\
    \label{eq:set-circle-parametric}
    C_{\text{parametric}} &= \{ (r \cos t, r \sin t) ~|~ t \in [0,2\pi] \}
\end{align}

The parametric description \eqref{eq:set-circle-parametric}
tells us that the dimension of \(C\) is one
because the description uses one parameter \(t\).

\FIXME{Rename \enquote{parametric description} to \enquote{chart}?}

The derivative of the parametric function is the curve's velocity.

A shape's dimension is the number of parameters in its parametric description.

To see that both sets does describe a circle (and the same circle),
we can plot the graph of both sets.

\section*{Describing a curve parametrically or algebraically}

\paragraph{Using parametric equations}

We can describe a curve with a function of type \( \Real \to \Real^n \).
For example, we can write \(x(t) = (r \cos t, r \sin t)\) to describe a circle of radius \(r\).
We can also write the same equation as a set of parametric equations:
\(x(t) = r \cos t\) and \(y(t) = r \sin t\).

\paragraph{Using algebraic equations}

We can describe a circle as \( \{ (x,y) ~|~ (x,y) \in \Real^2, ~ x^2 + y^2 = r^2 \} \).
The equation describes a circle because the set of all points satisfying the equation forms a circle.

\paragraph{Comparing the approaches}

The parametric equation of a circle simplifies computing the derivative.

The algebraic equation of a circle simplifies computing the distance.

\section*{Calculating the length of a parametric curve}

Let the function \(x : \Real \to \Real^n\) describe a curve.

Define \(X(T) = \{ x(t) ~|~ t \in T \}\) as the \emph{image of \(T\) under \(x\)}.
\footnote{\url{https://en.wikipedia.org/wiki/Image_(mathematics)}}

The segment between \(a\) and \(b\) is \( X([a,b]) = \{ x(k) ~|~ a \le k \le b \} \).
What is the length of this segment?

Divide the curve into many segments.
Approximate each segment as a straight line segment.

The \emph{arc length}.

The length of the segment of \(x\) in \(T\) is
\begin{align*}
    L(x,T) &= \sum_k \norm{x(t_{k+1}) - x(t_k)}
    \\ &= \sum_k \frac{\norm{x(t_k + h_k) - x(t_k)}}{h_k} \cdot h_k
    \\ &= \int_T \abs{x'(t)} \dd{t}
\end{align*}

Wikipedia\footnote{\url{https://en.wikipedia.org/wiki/Arc_length\#Definition_for_a_smooth_curve}}
explains how to derive that equation.
The key is to multiply by \(h_k / h_k\).

\footnote{\url{https://en.wikipedia.org/wiki/Arc_length}}%
\footnote{\url{https://en.wikipedia.org/wiki/Curve\#Length_of_a_curve}}%
\footnote{\url{http://mathworld.wolfram.com/ArcLength.html}}

\section*{Describing a surface}

An example of a space is \( \{ (x,y,z) ~|~ x^2+y^2+z^2 = 1 \} \), the skin of the unit sphere in a 3-dimensional Euclidean space.

A \emph{smooth space} looks smooth (no discontinuities).
It has something to do with differentiability.

\section*{Describing spaces using vectors}

\section*{Describing a line, a plane, and a hyperplane}

A line is a two-dimensional hyperplane.
A plane is a three-dimensional hyperplane.

The \emph{line} that connects point \(A\) and point \(B\) is the set
\( \{ A + k \cdot AB ~|~ k \in \Real \} \).

To define a hyperplane, we need a \emph{normal vector} \(n\) and a point \(C\) on the plane.
The \emph{hyperplane} is then the set of every point \(P\) such that \(CP\) is orthogonal to \(n\).

Try convincing yourself that a line is a two-dimensional hyperplane.
In two-dimensional Euclidean space, fix a point \(C\),
and draw a vector \(n\) whose origin is \(C\).
Pick a \(P\) such that \(CP\) and \(n\) form a right angle.
Pick another such \(P\).
Pick yet another such \(P\).
Pick as many such \(P\) as you need to see that those points form a line.

\section*{Defining lines as cotangent spaces and tangent spaces?}

We have just shown that there are two ways to define a line:
by \emph{cotangent space} and by \emph{tangent space}?

\section*{Describing a circle, a sphere, and a hypersphere}

A circle has a center and a radius.
The circumference of a circle is the set of all points
whose distance from the center is the radius.

A circle with center \(C\) and radius \(r\)
is described by the set \( \{ P : \norm{CP} = r \} \).
Note that this set only describes the circumference.

With coordinates: \( \{ (x,y) : \norm{xi + yj} = r \} \).

With coordinates and orthonormal basis: \( \{ (x,y) : x^2 + y^2 = r^2 \} \).

The two-dimensional circle generalizes to the three-dimensional sphere,
which generalizes to the higher-dimensional hypersphere.

\section*{Describing a shape}

\paragraph{As a set of points}

A shape is a set of points.

\paragraph{With numbers}

Here we get used to using numbers to describe shapes.

Let's say we have a line that connects \((0,0)\) and \((1,1)\).
Some other points on that line are \((2,2)\) and \((3,3)\).
However, we want to describe \emph{all} points on that line.
The way we do it is: ``For every \((x,y)\) in \(\Real^2\), iff \(x=y\), then \((x,y)\) is on the line.''
We can write it in math notation as \( \{ (x,y) ~|~ (x,y) \in \Real^2, ~ x = y\} \).

We generalize.
A line that passes \((0,c/b)\) and \((c/a,0)\)
is described by the set \(\{ (x,y) ~|~ (x,y) \in \Real^2, ~ ax + by = c \}\).
If this is not obvious to you, try replacing \(x\) and \(y\) in the equation with some numbers
without violating the equation.

The set \(\{ (x,y) ~|~ (x,y) \in \Real^2, ~ x^2 + y^2 = r^2 \}\) describes a circle.

The set \(\{ (x,y,z) ~|~ (x,y,z) \in \Real^3, ~ x^2 + y^2 + z^2 = r^2 \}\) describes a sphere.

\index{definitions!dimension of a shape}%
\index{dimension of a shape}%
The number of the tuple component is the \emph{dimension} of the shape.

\paragraph{With functions}

A function \(\Real \to \Real^3\) describes a curve in a three-dimensional space.

\index{definitions!dimension of a shape}%
\index{dimension of a shape}%
The function's parameter count is the \emph{dimension} of the shape.

We can use a function \(\Real^2 \to \Real^3\) such as \( (u,v) \to (0,u,v^2) \).
This describes a \emph{surface} in a three-dimensional space.

We can use a vector equation such as \( m \cdot x + n = 0 \).

We can use an equation such as \( x^2 + y^2 + z^2 = 1 \).

The distance of two points on the surface is the length of the shortest one-dimensional submanifold
of that surface such that this submanifold connects those points.

What is the length of a one-dimensional submanifold?
The curve is described by \( x : \Real \to \Real^n \).
The length of a small segment around \(t\) is \(\norm{x(t+dt) - x(t)}\).

\section{We haven't understood yet}

\subsection{Using charts and atlases}

TODO
Motivate differential geometry:
How do we make a map of the Earth?
How do we project the surface of a sphere to a paper?
Curvilinear coordinates. Space curvature.
How do you describe a (curved) surface?
What is curvature?

Manifold generalizes surface.
A surface is a two-dimensional thing in a three-dimensional space.
There can be \(m\)-dimensional manifold in \(n\)-dimensional space, if \(m \le n\).

We describe the surface of the unit sphere as \( \{ (x,y,z) ~|~ x^2 + y^2 + z^2 = 1, ~ (x,y,z) \in \Real^3 \} \).
We can also in cylindrical coordinates \( \{ (1,a,b) ~|~ 0 \le a,b < 2\pi \} \).
The \emph{surface} of that sphere is a \emph{two}-dimensional manifold.

How do we describe a line on the sphere?
A great circle?

A \emph{chart} is a mapping (a function) between two manifolds?

An \emph{atlas} is a set of charts?

\subsection{Describing a cotangent bundle}

The cotangent bundle of a space \(S\) is the vector bundle of all the cotangent spaces at every point in \(S\).

The cotangent bundle of a space \(S\) is the dual bundle of the tangent bundle of \(S\).

\subsection{Treating a phase space as a cotangent bundle}

This system of two equations describes a system that consists of one free particle: \( x(t) = t \) and \( v(t) = 1 \).
The phase space of that system is \( \{ (t,mv) ~|~ t \in \Real \} \).
A set of \emph{canonical coordinates} describes a point in the phase space.
What physicists call \emph{phase space},
mathematicians call \emph{cotangent bundle of a manifold} (\enquote{Phase space}, Wikipedia).

\subsection{Understanding \enquote{locally}}

\enquote{locally}\footnote{\url{https://en.wikipedia.org/wiki/Local_property}}

\subsection{Understanding parallel transports}

To understand Riemann curvature tensor.
\cite{arnold1989mathematical}

\subsection{Describing smooth deformation}

A map from a sheet to a bent sheet.

\subsection{Describing strain using tensor}

\footnote{\url{https://en.wikipedia.org/wiki/Infinitesimal_strain_theory}}
\footnote{\url{https://en.wikipedia.org/wiki/Continuum_mechanics}}

\section{Generalizing two-dimensional shapes to higher dimensions}

\subsection{Generalizing line to plane and hyperplane}

\subsection{Generalizing circle to sphere and hypersphere}

\section{The word \enquote{manifold}}

You can skip this section.

Riemann defines manifold?
Poincar\'e's \emph{Analysis situs} defines \emph{manifold}%
\footnote{\url{http://www.maths.ed.ac.uk/~aar/papers/poincare2009.pdf}}

Manifolds are confusing
because the word \enquote{manifold} means \enquote{variety},
which doesn't help students guess anything about locally flat spaces.
History can explain this mess.%
\footnote{\url{https://en.wikipedia.org/wiki/Manifold\#History}}%
\footnote{\url{https://en.wikipedia.org/wiki/History_of_manifolds_and_varieties}}

\section*{Replacing \enquote{manifold} with \enquote{space}}

Sometimes we can replace \enquote{manifold} with \enquote{space}.
For example, if we are discussing about spaces that are smooth and locally flat,
then we can write \enquote{cotangent bundle of a space}
instead of \enquote{cotangent bundle of a manifold}.

\section*{Redefining \enquote{manifold} to mean \enquote{locally flat spaces}}

Every time we meet the word \enquote{manifold},
we should by reflex think \enquote{locally flat space}.

A manifold is a space that looks flat if we zoom close enough.
\enquote{Flat} means Euclidean, that is resembling an Euclidean space.

\ExerciseAnswer{Say to yourself until it becomes a reflex: \enquote{A manifold is a locally flat space.}}{A manifold is a locally flat space.}

\ShowAnswers

Some examples of one-dimensional manifolds are lines and circles.%
\footnote{\url{https://en.wikipedia.org/wiki/Manifold}}
Some examples of two-dimensional manifolds are flat sheets and bent seets.

\section{Raw thought}

A line is a set of points.
A sheet is a set of lines.
A cube is a set of sheets.
Is this thought useful?

We will study mappings of the form \(\Real^a \to \Real^b\).
For example, \( \Real \to \Real^n \) is the type of a curve,
and \( \Real^n \to \Real \) is the type of a scalar field.
Is there a deeper connection between curves and scalar fields?
Curves and embeddings?
Scalar fields and projections?

With the real tuple space and the Cartesian coordinate systems,
we can marry infinitesimal calculus and geometry into differential geometry.

With analytic geometry, we can describe shapes using real numbers.

\section{Embedding a space in another space}

An \(n+1\)-dimensional space is bigger than an \(n\)-dimensional space.

An \(n+1\)-dimensional space can contain an \(n\)-dimensional space.

An \(n\)-dimensional space can be embedded in an \(n+1\)-dimensional space.

\footnote{\url{https://en.wikipedia.org/wiki/Embedding}}

\section*{Ambient spaces}

An \(m\)-dimensional object in an \(n\)-dimensional ambient space is a function \(\Real^m \to \Real^n\).
It tells you how to embed the \(m\)-dimensional object into the \(n\)-dimensional ambient space.
