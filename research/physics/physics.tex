% draft is faster than final?
\documentclass[12pt,twoside,openany,draft]{memoir}
% set to 10pt if using fonts with big x-height such as arev
%\documentclass[10pt,openany]{memoir}
% deprecated
\newcommand\N{\mathbb{N}}
\newcommand\Nat{\mathbb{N}}
% deprecated
\newcommand\R{\mathbb{R}}
\newcommand\Real{\mathbb{R}}
\newcommand\Bool{\textsf{Bool}}
\newcommand\Int{\mathbb{Z}}
\newcommand\Rational{\mathbb{Q}}
\newcommand\Complex{\mathbb{C}}
% complex conjugate
\newcommand\conjstar[1]{#1^*}
\newcommand\conjbar[1]{\bar{#1}}
\newcommand\conj\conjbar
\newcommand\abs[1]{|#1|}
\newcommand\Arg{\mathrm{Arg}}
\theoremstyle{definition}
\newtheorem{m:def}{Definition}
\theoremstyle{plain}
\newtheorem{m:thm}{Theorem}
\newtheorem{m:lem}{Lemma}
\newcommand\bmat[1]{\begin{bmatrix}#1\end{bmatrix}}
\newcommand\fun[1]{\textsf{#1}}
% deprecated
\newcommand\func[1]{\textsf{#1}}
\newcommand\fid{\func{id}}
% set builder notation
\newcommand\sbn[2]{\{#1~|~#2\}}
\newcommand\SysTmp{\textsf{SysTmp}}
\newcommand\Regex{\fun{Regex}}
\newcommand\fnull{\fun{null}}
\newcommand\just{\fun{just}}
\newcommand\fempty{\fun{empty}}
\newcommand\concat{\fun{concat}}
\newcommand\union{\fun{union}}
\newcommand\many{\fun{many}}
\newcommand\match{\fun{match}}
\newcommand\InfList{\fun{InfList}}
\newcommand\List{\fun{List}}
\newcommand\false{\fun{false}}
\newcommand\true{\fun{true}}
\newcommand\isNull{\fun{isNull}}
\newcommand\dist{\fun{dist}}
\newcommand\stay{\fun{stay}}
\newcommand\fleft{\fun{left}}
\newcommand\fright{\fun{right}}
\newcommand\head{\fun{head}}
\newcommand\uncons{\fun{uncons}}
\newcommand\fwrite{\fun{write}}
\newcommand\jump{\fun{jump}}
\newcommand\after{\fun{after}}
\newcommand\phase{\fun{phase}}
\newcommand\fpath{\fun{path}}
\newcommand\dir{\fun{dir}}
\newcommand\removeLast{\fun{removeLast}}
\newcommand\cons{\fun{cons}}
\newcommand\snoc{\fun{snoc}}
\newcommand\last{\fun{last}}
\newcommand\tail{\fun{tail}}
\newcommand\tape{\fun{tape}}
\newcommand\halting{\fun{halting}}
\newcommand\foutput{\fun{output}}
\newcommand\flist{\fun{list}}
\newcommand\species[1]{\textit{#1}}
\newcommand\ob{\fun{ob}}
\newcommand\norm[1]{\lVert#1\rVert}
\newcommand\fhom{\fun{hom}}
\newcommand\normalize{\fun{normalize}}
\newcommand\Var{\fun{Var}}
\newcommand\map{\fun{map}}
\newcommand\invmap{\fun{invmap}}
\newcommand\Set{\fun{Set}}
\newcommand\any{\fun{any}}
% probability
\renewcommand\Pr{P}
\newcommand\Expect[1]{\mathbb{E}[#1]}
\newcommand\estimate[1]{\hat{#1}}
\newcommand\Bernoulli{\text{Bernoulli}}
\newcommand\Binomial{\text{Binomial}}
\newcommand\RV{\fun{RV}}
\newcommand\kolmogorov{\fun{kolmogorov}}
\newcommand\minimum{\fun{minimum}}
\newcommand\minimumBy{\fun{minimumBy}}
% deprecated; function body
\newcommand\fb[2]{\ensuremath{\lambda #1. #2}}
\newcommand\emptystr{\epsilon}

\usepackage{physics}
\usepackage{siunitx}
% workaround
% cmbr doesn't have this symbol?
% https://tex.stackexchange.com/questions/33965/siunitx-%C2%B5-doesnt-work
\usepackage{textcomp}
%\sisetup{math-micro=\text{µ},text-micro=µ}% if Unicode font
\sisetup{math-micro=\textmu,text-micro=\textmu}% if legacy font
% end workaround
% deprecated
\newcommand\N{\mathbb{N}}
\newcommand\Nat{\mathbb{N}}
% deprecated
\newcommand\R{\mathbb{R}}
\newcommand\Real{\mathbb{R}}
\newcommand\Bool{\textsf{Bool}}
\newcommand\Int{\mathbb{Z}}
\newcommand\Rational{\mathbb{Q}}
\newcommand\Complex{\mathbb{C}}
% complex conjugate
\newcommand\conjstar[1]{#1^*}
\newcommand\conjbar[1]{\bar{#1}}
\newcommand\conj\conjbar
\newcommand\Arg{\mathrm{Arg}}
\theoremstyle{definition}
\newtheorem{m:def}{Definition}
\theoremstyle{plain}
\newtheorem{m:thm}{Theorem}
\newtheorem{m:lem}{Lemma}
\newcommand\bmat[1]{\begin{bmatrix}#1\end{bmatrix}}
\newcommand\fun[1]{\textsf{#1}}
% deprecated
\newcommand\func[1]{\textsf{#1}}
\newcommand\fid{\func{id}}
% set builder notation
\newcommand\sbn[2]{\{#1~|~#2\}}
\newcommand\SysTmp{\textsf{SysTmp}}
\newcommand\Regex{\fun{Regex}}
\newcommand\fnull{\fun{null}}
\newcommand\just{\fun{just}}
\newcommand\fempty{\fun{empty}}
\newcommand\concat{\fun{concat}}
\newcommand\union{\fun{union}}
\newcommand\many{\fun{many}}
\newcommand\match{\fun{match}}
\newcommand\InfList{\fun{InfList}}
\newcommand\List{\fun{List}}
\newcommand\false{\fun{false}}
\newcommand\true{\fun{true}}
\newcommand\isNull{\fun{isNull}}
\newcommand\dist{\fun{dist}}
\newcommand\stay{\fun{stay}}
\newcommand\fleft{\fun{left}}
\newcommand\fright{\fun{right}}
\newcommand\head{\fun{head}}
\newcommand\uncons{\fun{uncons}}
\newcommand\fwrite{\fun{write}}
\newcommand\jump{\fun{jump}}
\newcommand\after{\fun{after}}
\newcommand\phase{\fun{phase}}
\newcommand\fpath{\fun{path}}
\newcommand\dir{\fun{dir}}
\newcommand\removeLast{\fun{removeLast}}
\newcommand\cons{\fun{cons}}
\newcommand\snoc{\fun{snoc}}
\newcommand\last{\fun{last}}
\newcommand\tail{\fun{tail}}
\newcommand\tape{\fun{tape}}
\newcommand\halting{\fun{halting}}
\newcommand\foutput{\fun{output}}
\newcommand\flist{\fun{list}}
\newcommand\species[1]{\textit{#1}}
\newcommand\ob{\fun{ob}}
\newcommand\fhom{\fun{hom}}
\newcommand\normalize{\fun{normalize}}
\newcommand\Var{\fun{Var}}
\newcommand\map{\fun{map}}
\newcommand\invmap{\fun{invmap}}
\newcommand\Set{\fun{Set}}
\newcommand\any{\fun{any}}
% probability
\renewcommand\Pr{P}
\newcommand\Expect[1]{\mathbb{E}[#1]}
\newcommand\estimate[1]{\hat{#1}}
\newcommand\Bernoulli{\text{Bernoulli}}
\newcommand\Binomial{\text{Binomial}}
\newcommand\RV{\fun{RV}}
\newcommand\kolmogorov{\fun{kolmogorov}}
\newcommand\minimum{\fun{minimum}}
\newcommand\minimumBy{\fun{minimumBy}}
% deprecated; function body
\newcommand\fb[2]{\ensuremath{\lambda #1. #2}}
\newcommand\emptystr{\epsilon}
% vector calculus
\newcommand\fgrad{\fun{grad}}
\newcommand\fdiv{\fun{div}}
\newcommand\fcurl{\fun{curl}}
% other
\newcommand\Bit{\fun{Bit}}
\newcommand\Bits{\fun{Bits}}
\newcommand\argmin{\operatorname{arg\,min}}
\newcommand\sgn{\operatorname{sgn}}
% transpose
\newcommand\tp{\operatorname{tp}}
\newcommand\fspan{\operatorname{span}}
% derivative (der : (R -> R) -> (R -> R))
\newcommand\der{\ensuremath{\operatorname{der}}}
\newcommand\StandardPart{\ensuremath{\operatorname{st}}}

\title{Physics}
\author{Erik Dominikus}
\date{2017 Working Draft Edition}%2017-11-01
\makeindex
\begin{document}
% equation spacing; can't precede \begin{document}
% https://tex.stackexchange.com/questions/21687/avoid-large-spaces-between-text-and-equations
\def\MathSkip{0.25em}
\setlength\abovedisplayskip{\MathSkip}
\setlength\belowdisplayskip{\MathSkip}
\setlength\abovedisplayshortskip{\MathSkip}
\setlength\belowdisplayshortskip{\MathSkip}
%
% chapter style
%
\chapterstyle{crosshead}
\setlength\beforechapskip{1em}
\setlength\afterchapskip{1em}
\renewcommand\chapterheadstart{\par\vspace{\beforechapskip}}
\renewcommand\clearforchapter{}% don't start new page
\indentafterchapter
% https://tex.stackexchange.com/questions/122574/globally-changing-math-line-spacing
\setlength\jot{0em}% math line spacing = text line spacing
\frontmatter
\begin{titlingpage}
    {
        \centering
        \def\usefontsize#1{\fontsize{#1}{#1}\selectfont}
        {\noindent\usefontsize{6em}\bfseries\thetitle\par}
        \vspace*{2em}
        {\noindent\usefontsize{1.5em}\bfseries\thedate\par}
        \vspace*{\fill}
        {\noindent\usefontsize{1.25em}\bfseries\theauthor\par}
    }
    \pagebreak
{
    \setlength\parindent{0em}

    \textcopyright{} 2017 Erik Dominikus

    \vspace{1em}

    This book is licensed under CC BY-SA 4.0 (Creative Commons Attribution-ShareAlike 4.0 International).
    More information about the license is at

    \url{https://creativecommons.org/licenses/by-sa/4.0/}

    \vspace{1em}

    The source code of this book is available at

    \url{https://github.com/edom/research}
}
\end{titlingpage}

\pagebreak

\long\def\disabled#1{}

\pagebreak

\chapter*{Preface}

How much mathematics does one need to understand artificial intelligence?
On Tuesday, 28th February 2017, I set out to answer that.
I wanted to learn about artificial intelligence as fast as possible,
and I thought that the best way to do that was to dive cold turkey into the mathematics.

I wrote this book for people who use but don't research mathematics.
I want to help the readers get used \emph{quickly} to the mathematics they need to do their job.
Easy and correct are \emph{not} parts of the goal.

The approach is formal, but not rigorous.
Wrong-but-useful is preferred to correct-but-irrelevant.
Proofs are omitted.

This book only covers a very small part of mathematics.
For an overview of how big mathematics is,
see the Mathematics Subject Classification.


{
    \makeatletter
    {
        \pagebreak
        \renewcommand\contentsname{List of parts}
        \maxtocdepth{part}
        \tableofcontents
    }
    {
        \pagebreak
        \renewcommand\contentsname{List of chapters}
        \maxtocdepth{chapter}
        \tableofcontents
    }
    \disabled{
        \pagebreak
        \renewcommand\contentsname{List of sections}
        \maxtocdepth{section}
        \tableofcontents
    }
    {
        \pagebreak
        \renewcommand\contentsname{Table of contents}
        \maxtocdepth{subsection}
        \tableofcontents
    }
    \listoftables
    \listoffigures
    \makeatother
}

\mainmatter
{

    \part{Foundations}

The foundations consist of
lambda calculus\footnote{\url{https://en.wikipedia.org/wiki/Lambda_calculus}},
type theory\footnote{\url{https://en.wikipedia.org/wiki/Type_theory}},
and logic\footnote{\url{https://en.wikipedia.org/wiki/Logic}}.
All math in this book is based on them.

    \chapter{Outline}

Branch of physics according to Arxiv.
Arxiv frontend.%
\footnote{\url{http://front.math.ucdavis.edu/physics}}

Wikipedia%
\footnote{\url{https://en.wikipedia.org/wiki/Outline_of_physics}}%
\footnote{\url{https://en.wikipedia.org/wiki/Timeline_of_fundamental_physics_discoveries}}%
\footnote{\url{https://en.wikipedia.org/wiki/Timeline_of_theoretical_physics}}%
\footnote{\url{https://en.wikipedia.org/wiki/Timeline_of_developments_in_theoretical_physics}}%
\footnote{\url{https://en.wikipedia.org/wiki/List_of_important_publications_in_physics}}

    \chapter{Mathematics 1}

Lambda calculus is a set of rules for manipulating expressions.

\section*{Accepting mathematics}

If you love mathematics, skip this section.
This is written for people who think
they can understand physics without mathematics.

\paragraph{Mathematics is not only formulas}
Mathematics is about thinking.
Ideas are important.

\paragraph{Mathematics is an unavoidable tool for physics}

Mathematics has advanced physics.
Mathematics is here to stay.

If you don't want to learn at least some mathematics,
don't waste your time learning physics.
Without math, you will forever be physicist-wannabe.

For physicists, math is a tool.
Physicists use math.
Carpenters use saws.
Blacksmiths use hammers.

You don't need to love math.
Just don't hate it.
A carpenter neither loves nor hates his hammer;
he is simply not attached to it.
He is dispassionate, indifferent, neutral, uncaring.
He has no reason to hate it as long as it does what he expects it to do.

If you can understand English,
then you can understand the math used in this book.
It will take time, and you will have to think hard, but it is surmountable.
If you have tried but you still can't understand,
let us know that we have failed to write clearly.

\paragraph{Mathematics has become a necessity for physics}

We can really understand 20th century physics only if
we are somewhat fluent in the needed mathematics.
Without mathematics, our understanding is superficial and vague,
and all we can do is impress others who are even more clueless.

\paragraph{Mathematics needs mental effort}

Understanding triangles is like lifting a 1 kg weight.
Understanding vectors is like lifting a 2 kg weight.
Understanding manifolds is like lifting a 20 kg weight.
Lifting is uncomfortable,
but getting used to it makes it less comfortable,
until it becomes no big deal.
The only way to get used to something is by repeating it.
Don't try too hard to understand mathematics; just try to get used to it.
If you are hell-bent on building your body,
you won't minding lifting weights, no matter how heavy they are.
Now you just have to get yourself hell-bent on building your brain.
Physical training builds muscles.
Mental training builds brains.
Muscles can be sexy.
Brains can be sexy too.
A book is a mental gym.
If that motivates you.

Fluffy text ends here.
The rest of the book requires hard thinking.
Begin the journey!

\section{Reading mathematical notation}

\paragraph{Mathematical notation is concise English}

When you read an English sentence mixed with mathematical notation,
replace the notation with an English fragment that makes the sentence grammatical.

\index{definitions!overloading a symbol}%
\paragraph{Overloading}

To overload a symbol is to give it different meanings in different contexts.
For example, the same letter \(\Real\) means a set in \enquote{\(x \in \Real\)}
but means a type in \enquote{\(x : \Real\)}.
We write \(\Real\) for both the set of all real numbers and the type of a real number.

\TableRef{tab:overloading} shows overloading.
Note how the same notation \(x,y:\Real\) reads differently.

\Table{
    \Label{tab:overloading}
    \Caption{Reading mathematical notation}
    \Columns{p{0.4\textwidth}p{0.4\textwidth}}
    \Head{notation & English}
    \Body{
        Let \(x,y:\Real\).
        &
        Let \(x\) be a real number and \(y\) be a real number.
        \\
        \midrule
        Given \(x, y:\Real\), we can compute \(z:\Real\) between them.
        &
        Given a real number \(x\) and a real number \(y\), we can compute a real number \(z\) between them.
    }
}

\section{Lambda calculus: rules for forming and reducing expressions}

(Do we really need to tell the reader about lambda calculus?)

In mathematics, a calculus is a set of rules.
(\enquote{Calculus} means \enquote{pebble}.
How did its meaning change that much?)

Lambda calculus is a set of rules for forming and reducing expressions.
An expression is a description of a mathematical object.

\paragraph{Formation rules}

An expression is a variable, a function, or an application.

A variable is an irreducible expression.

If \(a\) is a variable and \(b\) is an expression,
then \(a \to b\) is a function.

If \(a\) and \(b\) are expressions,
then \((a)(b)\) is an application.
We omit the parentheses if such omission doesn't confuse us.
We do application from left to right: \(abcd\) means \(((ab)c)d\).

\paragraph{Substitution}

The notation \( E[x:=y] \) means the expression \(E\)
but with all free occurrences of \(x\) replaced by \(y\).

\paragraph{Reduction rules}

We can reduce an application \((a)(x \to y)\) to \(y[x:=a]\).

We can't reduce anything else.

\paragraph{Question}

Why do we bother to make up expressions, only to reduce them?
(Hint: Computers.)

\paragraph{Typed lambda calculus}
We can further constraint the rules using types in the next section.

\section{Describing existence using type theory}

Every value has a type.

Every expression has a type.

Every variable has a type.

The type of an expression reminds us about what it is.
For example, we write \(x : \Nat\) to mean that \(x\) is a natural number.

\index{definitions!type}%
We write \( x : T \) to mean ``the value \(x\) has the type \(T\)''.
We can read \(x:T\) as ``\(x\) is a \(T\)''.

We write \( x, y : T \) to shorten ``\( x : T \) and \( y : T \)''.

Variable...

Free variable and bound variable...

Let \(x:T\).
Then the answer to the question \enquote{What is \(x\)?} is \enquote{\(x\) is a \(T\)}.
We say that \(x\) inhabits \(T\).
We also say that \(x\) is an inhabitant of \(T\).

\paragraph{Evaluating an expression to a value}

Example: the expression \(1+2\) evaluates to the value \(3\)
according to high school arithmetic.

\paragraph{Tuples}

\index{definitions!tuple}%
A tuple has a fixed number of components.
An \(n\)-tuple has \(n\) components.
Some examples of tuples are \( (1,2,3) \) and \( (t_1,t_2) \).

\index{definitions!tuple component}%
The \(k\)th component of the tuple \(t\) is written \( t_k \) with \(k\) beginning from one.
For example, if \( t = (0,1) \), then \( t_1 = 0 \) and \( t_2 = 1 \).

\section{Converting between English and logic}

Every popular human language contains logic.

\( p \wedge q \)

\section{Describing collections using set theory}

\index{definitions!``practically''}%
In this book, \enquote{practically} means \enquote{wrong but useful}.
We can be practical when we aren't dealing with the foundation of math.

\index{definitions!set}%
We can think of a \enquote{set} as an unordered collection of things without duplicates.%
\footnote{Mathematicians don't define \enquote{set} as such because it would lead to Russell's paradox.}
\enquote{Unordered} means that \( \{a,b\} \) and \( \{b,a\} \) are the same set.
We say that \( a \) is an element of that set.
We notate that as \( a \in \{a,b\} \).

We can also define a set by a predicate or an English description.
An example is the set of all 26 Latin capital letters;
some of its elements are A, B, and C.

\paragraph{Working with sets}

TODO intersection \(A \cap B\),
union \(A \cup B\),
subtraction \(A - B\), ...

\paragraph{Using the set-builder notation to define sets using predicates}

\footnote{\url{https://en.wikipedia.org/wiki/Set-builder_notation}}

\( \{ x ~|~ P(x) \} \)

\( \{ x : P(x) \} \)

\paragraph{Relating set operations and logic operations}

Set intersection corresponds to logical conjunction.
\[
\{ x ~|~ a(x) \} \cap \{ x ~|~ b(x) \} = \{ x ~|~ a(x) \wedge b(x) \}
\]

Set union corresponds to logical disjunction.
\[
\{ x ~|~ a(x) \} \cup \{ x ~|~ b(x) \} = \{ x ~|~ a(x) \vee b(x) \}
\]

\section{Working with equations}

An equation \(a = b\) means that every \(a\) can be replaced with \(b\)
and every \(b\) can be replaced with \(a\).

Equations help us program computers.

An equation relates several parameters.

An equation can also be thought as describing a relationship, a constraint, something that may or may not be satisfied.

\paragraph{Substituting equal things}

\section{Working with numbers}

\paragraph{Natural numbers}

\paragraph{Real numbers}

% https://www.quora.com/Why-are-real-numbers-called-%E2%80%98real%E2%80%99
\index{definitions!real number}%
We write \( \Real \) to mean the set of real numbers\footnote{%
It's not that other numbers are fake. It's just a name that has stuck.%
}.
Example elements are \( 123 \), \( -12.34 \), \( 1/2 \),
\( \sqrt{2} \), \( 2^{0.1} \), \( \log 5 \), \( \sin 1 \), and \( \sum_{k=0}^\infty 3^{-k} \).
We assume that you know real number arithmetics.

\paragraph{Basic arithmetics}

We expect you to be able to add, subtract, multiply, and divide two one-digit numbers (from 0 to 9).
Primary education should have taught you this.

\footnote{\url{https://en.wikipedia.org/wiki/Arithmetic\#Arithmetic_operations}}%
\footnote{\url{https://en.wikipedia.org/wiki/Addition\#Addition_table}}%
\footnote{\url{https://en.wikipedia.org/wiki/Multiplication_table}}%

\paragraph{Using the summation notation}

If \(a < b\), then
\[
\sum_{k=a}^b f(k) = f(a) + f(a+1) + \ldots + f(b-1) + f(b)
\]

If \(K = \{ k_0, k_1, \ldots, k_n \}\) is a set, then
\[
\sum_{k \in K} f(k) = f(k_0) + f(k_1) + \ldots + f(k_n)
\]

\( \sum_k f(k) \) is a notation of the sum of all \(f(k)\) for each value that \(k\) can take. The meaning depends on context. For example, if \(v\) is a tuple with three components, then \(\sum_k v_k\) means \(v_1 + v_2 + v_3\).

    \chapter{Functions}

\footnote{\url{https://en.wikipedia.org/wiki/Lambda_calculus}}%
\footnote{\url{https://en.wikipedia.org/wiki/Lambda_calculus_definition}}

\index{definitions!function}%
We can think of a function as a computer.

The type \( a \to b \) is inhabited by functions that take an inhabitant of \(a\) and give an inhabitant of \(b\).

A function \(f : a \to b\) maps an \(a\) to a \(b\).

The expression \(f(x)\) means the result of applying \(f\) to \(x\).

If \(x : a\), then \( f(x) : b \).

For example, if \(f(x) = x^2 + x + 1\), then \(f(2+3) = (2+3)^2 + (2+3) + 1\) by substituting \(x\) with \(2+3\).

Don't write \(f(x)\) to mean the function \(f\).
The function is \(f\).
The expression \(f(x)\) means the result of applying \(f\) to \(x\).

\paragraph{Defining functions}

An \emph{unnamed function expression} is written like \( x \to x^2 + 1 \).

Example application: the expression \( (x \to x + 1)(5) \) reduces to \( 5 + 1 \).

\paragraph{Composing functions}

We write \(f \circ g\) to mean \( x \to f(g(x)) \).

\paragraph{Applying functions}

\paragraph{The inverse of an invertible function}

If \(f(x) = y\), then \(f^{-1}(y) = x\),
but only if \(f\) does not map anything else to \(y\).
We say that \(f^{-1}\) is the inverse of \(f\).

An invertible function is a function whose inverse is also a function.

\((f \circ g)^{-1} \equiv g^{-1} \circ f^{-1}\)

\section{Function-returning functions}

\(x \to (y \to x+y)\).

\section{Currying: one parameter is enough}

A two-input function has a type like \( (a,b) \to c \).

\index{definitions!currying}%
\emph{Currying} is the transformation from \( f(x,y) \) to \( (f'(x))(y) \) (from \(f\) to \(x \to y \to f(x,y)\)).
\emph{Uncurrying} is the inverse of currying.

We conflate \( (f'(x))(y) \) and \( f(x,y) \).

We assume that every function takes one input.

Do not confuse:
\begin{itemize}
    \item a function that takes \(n\) inputs, and
    \item a function that takes \emph{one} input that is an \(n\)-tuple.
\end{itemize}

\section{Plotting the graph of a function}

If you type \verb@sin(x)@ into Google\footnote{\url{https://www.google.com/}},
it will plot the graph of \(y = \sin(x)\) for you.

You can also type \verb@y=sin(x)@ into Wolfram Alpha\footnote{\url{https://wolframalpha.com/}}.

You can use GNU Octave.

You can use Gnuplot.

\paragraph{Thinking of an operator as a function}

An infix operator is an operator that is placed between two things.
For example, the dot in \(a \cdot b\) is an infix operator.

An operator does not have to be a symbol.
It can be a letter.
It can be a complicated notation.
For example, \(\pdv{f}{x}\).

We can treat an infix operator as another way of writing a function application.
We can think of \(a \cdot b\) as a convenient way of writing \((\cdot)(a,b)\).

\footnote{\url{https://en.wikipedia.org/wiki/Operator_(mathematics)}}%
\footnote{\url{https://en.wikipedia.org/wiki/Operator_(physics)}}

\section{Equivalence of functions, eta-reduction}
\label{sec:function-equivalence}

We write \(f \equiv g\) to mean that \(f\) gives the same result as \(g\) for all parameters.

\section{Exercises}

To evaluate something is to reduce it to normal form.

\ExerciseAnswer{Evaluate the expression \((x \to x + 1)(5)\), assuming the usual arithmetic.}{Substitute: \(5 + 1\). Reduce: \(6\).}

\ShowAnswers

}
{
    \part{Geometry}

    \chapter{Spaces}
\label{chp:manifold}

A space is a set of points.%
\footnote{\url{https://en.wikipedia.org/wiki/Space_(mathematics)}}%
\footnote{\url{https://en.wikipedia.org/wiki/Topological_space\#Definition}}

\enquote{Space} is another word for \enquote{set}.

\enquote{Point} is another word for \enquote{element}.

Because spaces are sets,
we can use set operations such as intersection and union.
For example, if \(A_1\) is the set of all points on the line \(L_1\),
and \(A_2\) is the set of all points on the line \(L_2\),
then the set intersection \(A_1 \cap A_2\)
is the set of all points where \(L_1\) and \(L_2\) intersect.
A point in \(A_1 \cap A_2\) must be both on \(L_1\) and \(L_2\) simultaneously.
If those lines are parallel (and don't coincide),
then \(A_1 \cap A_2\) is empty,
because those lines don't intersect.

The dimension of a space is a natural number.

A point is an example of a zero-dimensional space.

A line is an example of a one-dimensional space.

A sheet is an example of a two-dimensional space.

A cube is an example of a three-dimensional space.

    \chapter{Shapes}

\section{Angles}

Two intersecting lines form an angle\footnote{\url{https://en.wikipedia.org/wiki/Angle}}.

Two lines are parallel iff they don't intersect.

Two lines coincide iff the angle between them is zero.

Two lines are orthogonal iff the angle between them is a right angle.

We use a protractor\footnote{\url{https://en.wikipedia.org/wiki/Protractor}} to measure an angle in degrees.
A full circle is 360 degrees,
a half circle is 180 degrees,
a quarter circle is 90 degrees,
and so on.

A full turn is 360 degrees,
a half turn is 180 degrees,
a quarter turn is 90 degrees,
and so on.

\section*{Degrees and radians}

Degrees are convenient for manual calculation
because 360 is divisible by several small integers.

Radians\footnote{\url{https://en.wikipedia.org/wiki/Radian}} simplify formulas.%
\footnote{\url{https://en.wikipedia.org/wiki/Radian\#Advantages_of_measuring_in_radians}}
For example,
let \(c\) be the circumference of
a circular sector of angle \(a\) and radius \(r\).
If \(a\) is in radians, then \(c\) has the simple formula \(c = a \cdot r\),
but if \(a\) is in degrees, the formula becomes \(c = (a / \ang{360}) \cdot r\).

We prefer the unit that simplifies our jobs.
Engineers prefer degrees.
Mathematicians prefer radians.
Both units are widely used,
so let's learn to convert one to the other.

An angle of \(2\pi\) radians is equal to an angle of \(360\) degrees.%
\footnote{\url{https://en.wikipedia.org/wiki/Radian\#Conversion_between_radians_and_degrees}}
Both of them are equal to one full turn:
\Formula{
    \frac{d}{360} = \frac{r}{2\pi}
}

To convert \(r\) radians to \(d\) degrees:
\Formula{
    d = \frac{360}{2\pi} \cdot r
}

To convert \(d\) degrees to \(r\) radians:
\Formula{
    r = \frac{2\pi}{360} \cdot d
}

\ExerciseAnswer{Convert \ang{360} to radians?}{\(2\pi/1\) radians.}
\ExerciseAnswer{Convert \ang{180} to radians?}{\(2\pi/2\) radians.}
\ExerciseAnswer{Convert \ang{90} to radians?}{\(2\pi/4\) radians.}
\ExerciseAnswer{Convert \ang{45} to radians?}{\(2\pi/8\) radians.}

\ShowAnswers

\section{Circles}

\footnote{\url{https://en.wikipedia.org/wiki/Circular_sector}}

\section{Triangles}

A vertex is a point where two sides meet.

A triangle is called a triangle because it has three angles.
A triangle also has three vertices and three sides.

The sum of all interior angles of a triangle is 180 degrees.

An equilateral triangle is a triangle whose sides have the same length.%
\footnote{\url{https://en.wikipedia.org/wiki/Equilateral_triangle}}
Each interior angle of such triangle is \ang{60}.

\enquote{Equilateral} is the Latinate of \enquote{same-sided}.

The study of triangles is called \enquote{trigonometry}.

\paragraph{Labeling a triangle}

A capital letter labels an interior angle.
The corresponding small letter labels the side across the angle.
For example, \(a\) is the side across the angle \(A\).

\subsection*{Drawing a standard right triangle}

A right angle is 90 degrees.%
\footnote{\url{https://en.wikipedia.org/wiki/Right_angle}}

A right triangle is a triangle that has a \ang{90} interior angle.

See the footnote\footnote{\url{https://commons.wikimedia.org/wiki/File:Rtriangle.svg}}
for the picture of a standard right triangle.
Here we describe that triangle.

Let there be a triangle \(ABC\).

Let the angle \(C\) be a right angle.

Let \(a\) be the side across angle \(A\).

Let \(b\) be the side across angle \(B\).

Let \(c\) be the side across angle \(C\).
Thus, \(c\) is the hypotenuse\footnote{\url{https://en.wikipedia.org/wiki/Hypotenuse}},
the longest side of a right triangle,
the side across the right angle.

Such triangle is called a standard right triangle.

\section*{Triangle side ratios}

Consider a standard right triangle.

The sine of the angle \(A\) is \(\sin(A) = a/c\).

The cosine of the angle \(A\) is \(\cos(A) = b/c\).

The tangent of the angle \(A\) is \(\tan(A) = a/b\).

\section*{Trigonometric identities}

We can show these by drawing:
\( \sin(0) = 0 \),
and \( \cos(0) = 1 \),
and \( \sin(\pi/2) = 1 \),
and \( \cos(\pi/2) = 0 \).

The Pythagorean theorem implies \((\sin(A))^2 + (\cos(A))^2 = 1\).

The sine function has a period of \(2\pi\).
It means that \(\sin(a + 2\pi) = \sin(a)\) for every real number \(a\).

We have \(\cos A = \sin\left(\frac{\pi}{2}-A\right)\)
because \(A+B+C = \pi\) and \(C = \pi/2\) and \(A+B = \pi/2\).

See also Wikipedia\footnote{\url{https://en.wikipedia.org/wiki/Special_right_triangle}}%
\footnote{\url{https://en.wikipedia.org/wiki/Right_triangle}}.

\section*{Inverse trigonometric functions}

We can measure an angle by the ratio of the sides of the right triangle formed by the angle.
We use inverse trigonometric functions (inverse sine, inverse cosine, inverse tangent).

    \chapter{Vector}

\section{Vector space}

\index{vector space}%
\index{space!vector}%
A \emph{vector space} over a field \(F\) is a set \(V\) and the
\index{vector space axioms}%
\emph{vector space axioms} \cite{wpvectorspace}\cite{roman2005advanced}:
For all \(a, b \in F\) and \(x, y \in V\):
\begin{enumerate*}[label={(\arabic*)}]
    \item \(V\) forms an additive group,
    \item \(1 x = x\) where \(1\) is the multiplicative identity of \(F\),
    \item \((ab)x = a(bx)\),
    \item \(a(x+y) = ax+ay\),
    \item \((a+b)x = ax+by\).
\end{enumerate*}
Therefore, a vector space \(V\) over a field \(F\) is
an additive group \(V\) and a \emph{scalar multiplication} \(F \to V \to V\).

\index{vector}%
A \emph{vector} is an element of a vector space.

\index{vector!concatenation}%
\index{concatenation!vector}%
\index{vector concatenation}%
The \emph{vector concatenation} of \(a : \Real^m\) and \(b : \Real^n\)
is \(a|b : \Real^p\)
where \(p = m + n\),
\(a|b = (a_1 , \ldots , a_m , b_1 , \ldots , b_n)\),
and a scalar is treated as a vector of length 1.

\index{vector!column}%
\index{column vector}%
A \emph{column vector} of length \(n\) is a \(n \times 1\) matrix.

\index{vectors!orthogonal}%
\index{orthogonal vectors}%
Two vectors \(x\) and \(y\) are \emph{orthogonal} iff \(x \cdot y = 0\).
\index{vectors!parallel}%
\index{parallel vectors}%
Two vectors \(x\) and \(y\) are \emph{parallel} iff \(x \cdot y = \norm{x} \cdot \norm{y}\).

\index{dot product}%
\index{vectors!dot product}%
Relationship between length and dot product: \(\norm{x}^2 = x \cdot x\).
Dot product distributes addition: \(x \cdot (y+z) = x \cdot y + x \cdot z\).
Geometric interpretation of dot product: \(a \cdot b = \norm{a} \cdot \norm{b} \cdot \cos \theta\).

\index{unit vector}%
\index{vectors!unit}%
A \emph{unit vector} is a vector whose length is 1.

\index{vector projection}%
\index{vectors!projection}%
\index{projection!of a vector to another vector}%
The \emph{projection} of \(a\) to \(b\) is \((a \cdot b) \cdot b / |b|^2\).

\section{Matrix}

\index{scalar-matrix multiplication}%
\emph{Scalar-matrix multiplication} is \((ka)_{ij} = k \cdot a_{ij}\).
\index{matrix!addition}%
\index{addition!matrix}%
\emph{Matrix addition} is \((a + b)_{ij} = a_{ij} + b_{ij}\).
\index{matrix!multiplication}%
\index{multiplication!matrix}%
\emph{Matrix multiplication} is \((ab)_{ij} = \sum_{k=1}^n a_{ik} b_{kj}\) where
\(a : R^{m \times n}, b : R^{n \times p}, c : R^{m \times p}\).

\index{coefficient matrix}%
\index{matrix!coefficient}%
\index{system of linear equations}%
\index{unknown}%
A \emph{system of linear equations} is a matrix equation \(A x = b\)
where \(A\) is the \emph{coefficient matrix} and \(x\) is the \emph{unknown}.
\index{overdetermined system of linear equations}%
\index{system of linear equations!overdetermined}%
That system is \emph{overdetermined} iff \(A\) has more rows than columns.

\section{Least-squares solution}

\index{least-squares!solution of an overdetermined system of linear equations}%
\index{system of linear equations!overdetermined!least-squares solution}%
\index{system of linear equations!least-squares solution}%
If \(A : R^{m \times n}\) and \(b : R^m\),
then the \emph{least-squares solution} of \(A x = b\)
is the \(y\) that minimizes \(\norm{A y - b}^2\).
\index{normal equation}%
That \(y\) is also the solution of the corresponding \emph{normal equation}
\((A^T A) y = A^T b\).

\section{Hyperplane}

\index{hyperplane}%
\index{hyperplane!below}%
\index{hyperplane!below-or-on}%
\index{hyperplane!on}%
\index{hyperplane!above}%
\index{hyperplane!above-or-on}%
A \emph{hyperplane}
\(h : \Real^\infty \to \Real\)
is \(h~x = n \cdot (x - p)\)
where \(n\) is the \emph{normal} of \(h\)
and \(p\) is a point on \(h\).
The point \(x\)
is \emph{below} \(h\) iff \( h~x < 0 \),
is \emph{below-or-on} \(h\) iff \( h~x \le 0 \),
is \emph{on} \(h\) iff \( h~x = 0 \),
is \emph{above} \(h\) iff \( h~x > 0 \),
and
is \emph{above-or-on} \(h\) iff \( h~x \ge 0 \).

\index{hyperplane equation!matrix form}%
\index{matrix form of hyperplane equation}%
The \emph{matrix form} of the hyperplane equation \(f~x = a \cdot x + b\)
is \(f~x = (a|b)^T (x|1)\).

The \emph{distance} of a point \(x\) to hyperplane \(h = n \cdot (x - p)\)
is the length of the projection of \(x-p\) to \(n\).

\section{Matrix unary operations}

The
\index{transpose}%
\index{transpose!of matrix}%
\index{matrix!transpose}%
\emph{transpose} of \(M\) is \((M^T)_{ij} = M_{ji}\).

The
\index{conjugate transpose}%
\index{conjugate transpose!of matrix}%
\index{transpose!conjugate}%
\index{matrix!conjugate transpose}%
\emph{conjugate transpose} of \(M\) is \((M^*)_{ij} = (M_{ji})^*\).

\section{Special matrices}

\index{identity matrix}%
\index{matrix!identity}%
A matrix \(I : \Real^{n \times n}\) is \emph{identity} iff
\(\forall (A : \Real^{n \times n}) (IA = AI = A)\).
The \emph{\(n\times n\) identity matrix} is
\((I_n)_{ij} = \delta_{ij}\) where \(\delta\) is the
\index{Kronecker delta}%
Kronecker delta \(\delta_{ij} = [i=j]\).

A matrix \(M\) is
\index{unitary matrix}%
\index{matrix!unitary}%
\emph{unitary} iff \(M^*M = MM^* = I\)
where \(I\) is an identity matrix.

\section{Singular value decomposition}

The
\index{singular value decomposition}%
\index{matrix decomposition!singular value}%
\emph{singular value decomposition} of \(M\) is \(U S V^* = M\) where \(V^*\) is the conjugate transpose of \(V\).

\section{QR decomposition}

\(M\) is an
\index{triangular matrix}%
\index{matrix!triangular}%
\emph{upper triangular matrix} iff ...

\(M\) is an
\index{orthogonal matrix}%
\index{matrix!orthogonal}%
\emph{orthogonal matrix} iff ...

The
\index{QR decomposition}%
\index{matrix decomposition!QR}%
\emph{QR decomposition} of \(M\) is \(M = QR\) where
\(Q\) is an orthogonal matrix and
\(R\) is an upper triangular matrix.


    \part{Analytic geometry}

    Analytic geometry is geometry with real numbers.
    It marries mathematical analysis%
    \footnote{\url{https://en.wikipedia.org/wiki/Mathematical_analysis}}
    and geometry.

    \chapter{Real tuple spaces}

Earlier, we defined \enquote{space} as another word for \enquote{set} because we want
that definition to include every real tuple space.

The set \(\Real^n\) is the set of all real \(n\)-tuples.

The dimension of \(\Real^n\) is \(n\).

An example element of \(\Real^3\) is \((1,2,3)\).

\section{An example basis}

Imagine a flat sheet of paper.

Draw a point \(A\).

Draw a vector named \(i\), from \(A\), \SI{1}{cm} long, pointing right.

Draw another vector named \(j\), also from \(A\), \SI{1}{cm} long, but pointing up.

Thus, the vectors \(i\) and \(j\) are orthogonal.

Then, we declare the basis
\( e : \Real^2 \to E^2 \) as \( e(x,y) = xi + yj \).%

\section{Cartesian coordinate system}

Let \(E^n\) mean the \(n\)-dimensional Euclidean space.

The \(n\)-dimensional Cartesian coordinate system matches
a point in \(E^n\) and a point in \(\Real^n\).

\footnote{\url{https://en.wikipedia.org/wiki/Cartesian_coordinate_system}}

For example...

What mapping is drawn in \FigureRef{fig:a-2d-cartesian-coordinate-system}?

\begin{figure}[h]
    \centering
    \url{https://en.wikipedia.org/wiki/File:Cartesian-coordinate-system.svg}
    \caption{A two-dimensional Cartesian coordinate system}
    \label{fig:a-2d-cartesian-coordinate-system}
\end{figure}

\section{Example of geometry with numbers: describing a circle in \(\Real^2\)}

Every variable is a real number unless specified otherwise.

Assume a two-dimensional Cartesian coordinate system.

Consider a circle \(C\) with center \((0,0)\) and radius 1.
That circle can be described in two ways:
the algebraic description \eqref{eq:set-circle-algebraic},
and the parametric description \eqref{eq:set-circle-parametric}.
Both sets have the same members and describe the same circle.
\begin{align}
    \label{eq:set-circle-algebraic}
    C_{\text{algebraic}} &= \{ (x,y) ~|~ x^2 + y^2 = 1 \}
    \\
    \label{eq:set-circle-parametric}
    C_{\text{parametric}} &= \{ (r \cos t, r \sin t) ~|~ t \in [0,2\pi] \}
\end{align}

The parametric description \eqref{eq:set-circle-parametric}
tells us that the dimension of \(C\) is one
because the description uses one parameter \(t\).

\FIXME{Rename \enquote{parametric description} to \enquote{chart}?}

The derivative of the parametric function is the curve's velocity.

A shape's dimension is the number of parameters in its parametric description.

To see that both sets does describe a circle (and the same circle),
we can plot the graph of both sets.

\section*{Describing a curve parametrically or algebraically}

\paragraph{Using parametric equations}

We can describe a curve with a function of type \( \Real \to \Real^n \).
For example, we can write \(x(t) = (r \cos t, r \sin t)\) to describe a circle of radius \(r\).
We can also write the same equation as a set of parametric equations:
\(x(t) = r \cos t\) and \(y(t) = r \sin t\).

\paragraph{Using algebraic equations}

We can describe a circle as \( \{ (x,y) ~|~ (x,y) \in \Real^2, ~ x^2 + y^2 = r^2 \} \).
The equation describes a circle because the set of all points satisfying the equation forms a circle.

\paragraph{Comparing the approaches}

The parametric equation of a circle simplifies computing the derivative.

The algebraic equation of a circle simplifies computing the distance.

\section*{Calculating the length of a parametric curve}

Let the function \(x : \Real \to \Real^n\) describe a curve.

Define \(X(T) = \{ x(t) ~|~ t \in T \}\) as the \emph{image of \(T\) under \(x\)}.
\footnote{\url{https://en.wikipedia.org/wiki/Image_(mathematics)}}

The segment between \(a\) and \(b\) is \( X([a,b]) = \{ x(k) ~|~ a \le k \le b \} \).
What is the length of this segment?

Divide the curve into many segments.
Approximate each segment as a straight line segment.

The \emph{arc length}.

The length of the segment of \(x\) in \(T\) is
\begin{align*}
    L(x,T) &= \sum_k \norm{x(t_{k+1}) - x(t_k)}
    \\ &= \sum_k \frac{\norm{x(t_k + h_k) - x(t_k)}}{h_k} \cdot h_k
    \\ &= \int_T \abs{x'(t)} \dd{t}
\end{align*}

Wikipedia\footnote{\url{https://en.wikipedia.org/wiki/Arc_length\#Definition_for_a_smooth_curve}}
explains how to derive that equation.
The key is to multiply by \(h_k / h_k\).

\footnote{\url{https://en.wikipedia.org/wiki/Arc_length}}%
\footnote{\url{https://en.wikipedia.org/wiki/Curve\#Length_of_a_curve}}%
\footnote{\url{http://mathworld.wolfram.com/ArcLength.html}}

\section*{Describing a surface}

An example of a space is \( \{ (x,y,z) ~|~ x^2+y^2+z^2 = 1 \} \), the skin of the unit sphere in a 3-dimensional Euclidean space.

A \emph{smooth space} looks smooth (no discontinuities).
It has something to do with differentiability.

\section*{Describing spaces using vectors}

\section*{Describing a line, a plane, and a hyperplane}

A line is a two-dimensional hyperplane.
A plane is a three-dimensional hyperplane.

The \emph{line} that connects point \(A\) and point \(B\) is the set
\( \{ A + k \cdot AB ~|~ k \in \Real \} \).

To define a hyperplane, we need a \emph{normal vector} \(n\) and a point \(C\) on the plane.
The \emph{hyperplane} is then the set of every point \(P\) such that \(CP\) is orthogonal to \(n\).

Try convincing yourself that a line is a two-dimensional hyperplane.
In two-dimensional Euclidean space, fix a point \(C\),
and draw a vector \(n\) whose origin is \(C\).
Pick a \(P\) such that \(CP\) and \(n\) form a right angle.
Pick another such \(P\).
Pick yet another such \(P\).
Pick as many such \(P\) as you need to see that those points form a line.

\section*{Defining lines as cotangent spaces and tangent spaces?}

We have just shown that there are two ways to define a line:
by \emph{cotangent space} and by \emph{tangent space}?

\section*{Describing a circle, a sphere, and a hypersphere}

A circle has a center and a radius.
The circumference of a circle is the set of all points
whose distance from the center is the radius.

A circle with center \(C\) and radius \(r\)
is described by the set \( \{ P : \norm{CP} = r \} \).
Note that this set only describes the circumference.

With coordinates: \( \{ (x,y) : \norm{xi + yj} = r \} \).

With coordinates and orthonormal basis: \( \{ (x,y) : x^2 + y^2 = r^2 \} \).

The two-dimensional circle generalizes to the three-dimensional sphere,
which generalizes to the higher-dimensional hypersphere.

\section*{Describing a shape}

\paragraph{As a set of points}

A shape is a set of points.

\paragraph{With numbers}

Here we get used to using numbers to describe shapes.

Let's say we have a line that connects \((0,0)\) and \((1,1)\).
Some other points on that line are \((2,2)\) and \((3,3)\).
However, we want to describe \emph{all} points on that line.
The way we do it is: ``For every \((x,y)\) in \(\Real^2\), iff \(x=y\), then \((x,y)\) is on the line.''
We can write it in math notation as \( \{ (x,y) ~|~ (x,y) \in \Real^2, ~ x = y\} \).

We generalize.
A line that passes \((0,c/b)\) and \((c/a,0)\)
is described by the set \(\{ (x,y) ~|~ (x,y) \in \Real^2, ~ ax + by = c \}\).
If this is not obvious to you, try replacing \(x\) and \(y\) in the equation with some numbers
without violating the equation.

The set \(\{ (x,y) ~|~ (x,y) \in \Real^2, ~ x^2 + y^2 = r^2 \}\) describes a circle.

The set \(\{ (x,y,z) ~|~ (x,y,z) \in \Real^3, ~ x^2 + y^2 + z^2 = r^2 \}\) describes a sphere.

\index{definitions!dimension of a shape}%
\index{dimension of a shape}%
The number of the tuple component is the \emph{dimension} of the shape.

\paragraph{With functions}

A function \(\Real \to \Real^3\) describes a curve in a three-dimensional space.

\index{definitions!dimension of a shape}%
\index{dimension of a shape}%
The function's parameter count is the \emph{dimension} of the shape.

We can use a function \(\Real^2 \to \Real^3\) such as \( (u,v) \to (0,u,v^2) \).
This describes a \emph{surface} in a three-dimensional space.

We can use a vector equation such as \( m \cdot x + n = 0 \).

We can use an equation such as \( x^2 + y^2 + z^2 = 1 \).

The distance of two points on the surface is the length of the shortest one-dimensional submanifold
of that surface such that this submanifold connects those points.

What is the length of a one-dimensional submanifold?
The curve is described by \( x : \Real \to \Real^n \).
The length of a small segment around \(t\) is \(\norm{x(t+dt) - x(t)}\).

\section{We haven't understood yet}

\subsection{Using charts and atlases}

TODO
Motivate differential geometry:
How do we make a map of the Earth?
How do we project the surface of a sphere to a paper?
Curvilinear coordinates. Space curvature.
How do you describe a (curved) surface?
What is curvature?

Manifold generalizes surface.
A surface is a two-dimensional thing in a three-dimensional space.
There can be \(m\)-dimensional manifold in \(n\)-dimensional space, if \(m \le n\).

We describe the surface of the unit sphere as \( \{ (x,y,z) ~|~ x^2 + y^2 + z^2 = 1, ~ (x,y,z) \in \Real^3 \} \).
We can also in cylindrical coordinates \( \{ (1,a,b) ~|~ 0 \le a,b < 2\pi \} \).
The \emph{surface} of that sphere is a \emph{two}-dimensional manifold.

How do we describe a line on the sphere?
A great circle?

A \emph{chart} is a mapping (a function) between two manifolds?

An \emph{atlas} is a set of charts?

\subsection{Describing a cotangent bundle}

The cotangent bundle of a space \(S\) is the vector bundle of all the cotangent spaces at every point in \(S\).

The cotangent bundle of a space \(S\) is the dual bundle of the tangent bundle of \(S\).

\subsection{Treating a phase space as a cotangent bundle}

This system of two equations describes a system that consists of one free particle: \( x(t) = t \) and \( v(t) = 1 \).
The phase space of that system is \( \{ (t,mv) ~|~ t \in \Real \} \).
A set of \emph{canonical coordinates} describes a point in the phase space.
What physicists call \emph{phase space},
mathematicians call \emph{cotangent bundle of a manifold} (\enquote{Phase space}, Wikipedia).

\subsection{Understanding \enquote{locally}}

\enquote{locally}\footnote{\url{https://en.wikipedia.org/wiki/Local_property}}

\subsection{Understanding parallel transports}

To understand Riemann curvature tensor.
\cite{arnold1989mathematical}

\subsection{Describing smooth deformation}

A map from a sheet to a bent sheet.

\subsection{Describing strain using tensor}

\footnote{\url{https://en.wikipedia.org/wiki/Infinitesimal_strain_theory}}
\footnote{\url{https://en.wikipedia.org/wiki/Continuum_mechanics}}

\section{Generalizing two-dimensional shapes to higher dimensions}

\subsection{Generalizing line to plane and hyperplane}

\subsection{Generalizing circle to sphere and hypersphere}

\section{The word \enquote{manifold}}

You can skip this section.

Riemann defines manifold?
Poincar\'e's \emph{Analysis situs} defines \emph{manifold}%
\footnote{\url{http://www.maths.ed.ac.uk/~aar/papers/poincare2009.pdf}}

Manifolds are confusing
because the word \enquote{manifold} means \enquote{variety},
which doesn't help students guess anything about locally flat spaces.
History can explain this mess.%
\footnote{\url{https://en.wikipedia.org/wiki/Manifold\#History}}%
\footnote{\url{https://en.wikipedia.org/wiki/History_of_manifolds_and_varieties}}

\section*{Replacing \enquote{manifold} with \enquote{space}}

Sometimes we can replace \enquote{manifold} with \enquote{space}.
For example, if we are discussing about spaces that are smooth and locally flat,
then we can write \enquote{cotangent bundle of a space}
instead of \enquote{cotangent bundle of a manifold}.

\section*{Redefining \enquote{manifold} to mean \enquote{locally flat spaces}}

Every time we meet the word \enquote{manifold},
we should by reflex think \enquote{locally flat space}.

A manifold is a space that looks flat if we zoom close enough.
\enquote{Flat} means Euclidean, that is resembling an Euclidean space.

\ExerciseAnswer{Say to yourself until it becomes a reflex: \enquote{A manifold is a locally flat space.}}{A manifold is a locally flat space.}

\ShowAnswers

Some examples of one-dimensional manifolds are lines and circles.%
\footnote{\url{https://en.wikipedia.org/wiki/Manifold}}
Some examples of two-dimensional manifolds are flat sheets and bent seets.

\section{Raw thought}

A line is a set of points.
A sheet is a set of lines.
A cube is a set of sheets.
Is this thought useful?

We will study mappings of the form \(\Real^a \to \Real^b\).
For example, \( \Real \to \Real^n \) is the type of a curve,
and \( \Real^n \to \Real \) is the type of a scalar field.
Is there a deeper connection between curves and scalar fields?
Curves and embeddings?
Scalar fields and projections?

With the real tuple space and the Cartesian coordinate systems,
we can marry infinitesimal calculus and geometry into differential geometry.

With analytic geometry, we can describe shapes using real numbers.

\section{Embedding a space in another space}

An \(n+1\)-dimensional space is bigger than an \(n\)-dimensional space.

An \(n+1\)-dimensional space can contain an \(n\)-dimensional space.

An \(n\)-dimensional space can be embedded in an \(n+1\)-dimensional space.

\footnote{\url{https://en.wikipedia.org/wiki/Embedding}}

\section*{Ambient spaces}

An \(m\)-dimensional object in an \(n\)-dimensional ambient space is a function \(\Real^m \to \Real^n\).
It tells you how to embed the \(m\)-dimensional object into the \(n\)-dimensional ambient space.

    \chapter{One-dimensional spaces}

We can divide one-dimensional spaces into two categories:
\emph{lines} and \emph{curves}.
Lines are straight.
Curves may bend.
Thus every line is a curve.

\ResearchQuestion{%
Drawing a line is easy,
but how do we describe a line \emph{algebraically}?
}

\ResearchQuestion{%
A line is a straight one-dimensional space.
\emph{Straight} is defined by the ambient space that contains the line.
How do we define \emph{straight}?
}

\section{Defining a line}

If two points \(a\) and \(b\) are on a line,
then their midpoint \((a + b) / 2\) is also on the line.

(These aren't obvious to the uninitiated?)

If both \(a\) and \(b\) are on a line,
then the point \(k \cdot (b - a) + a\) is also on the line, for every \(k:\Real\).

If three points \(a,b,c\) are on a line,
then the displacements \(b-a\) and \(c-b\) are parallel.

\paragraph{Infinite extension of a line segment}
A \emph{line segment} is what is drawn using a straightedge.
A \emph{line} is obtained by infinitely extending a line segment in both directions.

\paragraph{Embedding of \(\Real\)}
\emph{A line is a straight embedding of \(\Real\).}
A line is something straight-shaped and isomorphic to \(\Real\).

But there is a problem with that definition.
That definition may include a pathological sheet, which should be a two-dimensional object.
This is a mapping from \(\Real^2\) to \(\Real\):
Let there be two real numbers \(a\) and \(b\).
Define the number \(c\) as \(\ldots a_1 b_1 a_0 b_0 . a_{-1} b_{-1} a_{-2} b_{-2} \ldots\).%
\footnote{\url{https://math.stackexchange.com/questions/75107/injective-map-from-mathbbr2-to-mathbbr}}%
\footnote{\url{https://math.stackexchange.com/questions/183361/examples-of-bijective-map-from-mathbbr3-rightarrow-mathbbr}}

We can define a line by a parametric equation: \( x(k) = k g + p \).

We can define a line by an algebraic equation: \( a \cdot x + b = 0 \).

\subsection{Defining a line as a curve with constant velocity}

The \emph{velocity} of the curve \( x : \Real \to \Real^n \) is the derivative of \(x\).
The velocity of \(x\) is the rate of change of \(x\).

\emph{A line is a curve whose velocity is constant.}

\subsection{Defining a line as a geodesic}

\section{Describing lines in a two-dimensional ambient space}

Every line can be described as the set
\( \{ (x,y) ~|~ (x,y) \in \Real^2, ~ a x + b y = c \} \).
(Why?)

\subsection{Describing a line that passes two points}

Describe a line that passes \((x_1,y_1)\) and \((x_2,y_2)\).

The description is
\begin{align*}
    a x_1 + b y_1 &= c
    \\
    a x_2 + b y_2 &= c
\end{align*}
Rearrange:
\begin{align*}
    x_1 a + y_1 b &= c
    \\
    x_2 a + y_2 b &= c
\end{align*}
Solve for \(a,b,c\).
\begin{align*}
    \Matrix{x_1 & y_1 \\ x_2 & y_2} \Matrix{a \\ b} = \Matrix{c \\ c}
\end{align*}
We can solve it using GNU Octave by typing \verb@[x1,y1;x2,y2] \ [c;c]@
but we have to substitute the variables with numbers first.

\subsection{Finding the angle formed by two lines}

\subsection{Translating lines and describing parallel lines}

Translating a line produces another line that is parallel to the original line.

Two lines \(ax+by=c\) and \(a'x+b'y=c'\) are parallel iff \(\abs{a/b} = \abs{a'/b'}\)?

\subsection{Describing orthogonal lines}

This is important for tangents, normals, and osculating circles.

\section{Describing higher-dimensional lines}

Every \(n\)-dimensional line can be described as
\( \{ k g + p ~|~ k \in \Real \} \)
if \(g, p : \Real^n\).

Describe a line that passes \(x_1\) and \(x_2\).
The description is
\begin{align*}
    x_1 &= k_1 g + p
    \\
    x_2 &= k_2 g + p
\end{align*}

\begin{align*}
    x_i - p &= k_i g
\end{align*}

How do we solve the equation \(a = kb\) if \(a,b\) are vectors and \(k\) is a scalar?

    \chapter{Bases}

The plural of \enquote{basis} is \enquote{bases}.
Don't confuse it with the plural of \enquote{base}.

\Table{
    \Columns{ll}
    \Caption{Basis-related notations}
    \Head{object & type}
    \Body{
        vector & \(V\)
        \\ tuple & \( \Real^n \)
        \\ parametric curve & \(\Real \to V\)
        \\ basis & \( \Real^n \to V \)
        \\ covector, scalar field & \(V \to \Real\)
        \\ cobasis & \( V \to \Real^n \)
        \\ vector change, vector field & \(V \to V\)
        \\ basis change & \( \Real^n \to \Real^n \)
    }
}

A basis invertibly maps a tuple to a vector.

Let \(V\) be a vector space.

A basis of \(V\) is an invertible function with type \( \Real^n \to V \).

With a basis, we can describe a vector by writing a tuple of numbers instead of drawing an arrow.

\section*{Example basis}

Let \(V\) be the space of all two-dimensional Euclidean vectors.

Thus the basis we describe here will have the type \( \Real^2 \to V \).

Let \(i\) be the unit vector pointing east (right).

Let \(j\) be the unit vector pointing north (up).

Note that \(i\) and \(j\) are orthogonal.

Then we can choose a basis \(e\) such that \(e(x,y) = xi+yj\).

In this basis, the tuple \((1,1)\) describes the vector
that has length \(\sqrt{2}\) and points northeast.

\section*{Example basis change}

Suppose that we rotate the basis \(e\)
(we rotate the coordinate axes) by 90 degrees counterclockwise.
Let the new basis be \(f\).
Then \(f(x,y) = -yi + xj\).

\section*{Don't confuse vectors and tuples}

The tuple is not the vector itself.
The tuple describes the vector.
The tuple and the vector are two different things.
They are related by the basis.

But we can indeed form an abstract-algebraic vector space \(\Real^n\) over the abstract-algebraic field \(\Real\),
so a real tuple is a vector.

\section{Linear basis}

A basis \(e : \Real^n \to V\) is linear iff \(e(a+b) = e(a) + e(b)\).
Note that we overload the plus sign.
The left plus sign is tuple addition.
The right plus sign is vector addition.

We can explode a linear basis \(e : \Real^n \to V\) to \(n\) vectors \(e_1,\ldots,e_n\),
each of the type \(V\), in this way:
\begin{align*}
    e(x_1,\ldots,x_n) &= e_1 x_1 + \ldots + e_n x_n
    \\ &= \Matrix{e_1 & \ldots & e_n}\Matrix{x_1 \\ \vdots \\ x_n}
    \\ &= EX
\end{align*}
and thus the linear basis \(e\) can be represented by the matrix \(E\) where
\[
    E = \Matrix{e_1 & \ldots & e_n}
\]

\section{Describing every vector as a linear combination of basis vectors}

We are talking about the two-dimensional Euclidean space here.
We can imagine it as an unbounded flat sheet of paper.

Let \(V^2\) be the set of all two-dimensional Euclidean vectors.

From \(V^2\), pick any two non-collinear vectors \(e_1\) and \(e_2\).

Let \(x_1,x_2\in\Real\).

The linear combination \(x_1 e_1 + x_2 e_2\) describes a vector in \(V^2\).

If we pick a basis,
we can represent every vector in \(V^2\) using two real numbers.
We can describe the entire \(V^2\) using \(\Real^2\)
as \( V^2 = \{ x_1 e_1 + x_2 e_2 ~|~ (x_1,x_2) \in \Real^2 \} \).

We say that \(E = \{e_1,e_2\}\) is a basis of the two-dimensional Euclidean space.

We say that \((x_1,x_2)\) is the coordinate tuple of vector \(v\) according to basis \(E\).
We can also say that \((x_1,x_2)\) is the \(E\)-coordinates of \(v\).

A basis of \(V^n\) is a set of \(n\) basis vectors
in which every pair of basis vectors are non-collinear.
With such basis, we can describe every vector in \(V^n\)
as a linear combination of those basis vectors.

\section{Representing a coordinate tuple by a column matrix}

We can write \((x,y,z)\) or we can write
\[
    \Matrix{x \\ y \\ z}
\]

\section{Scaling a vector}

If \(k\) is a number and \(v\) is a vector,
then \(kv\) is a vector that has the same direction as \(v\),
but the length of \(kv\) is \(k\) times the length of \(v\),
that is, \( \norm{k v} = k \norm{v} \).

We can think of \(-v\) (the negation of \(v\)) as scaling \(v\) by \(-1\).

If \(v = \sum_k x_k e_k\) then \(cv = \sum_k (c x_k) e_k \).

\section{The relationship between vectors and coordinates}

We have two choices

\(V \to \Real^n\)

\(\Real^n \to V\)

\section{Exploding a cobasis to covectors}

We can explode a cobasis \( e : V \to \Real^n \) to \(n\) covectors \( e_1, \ldots, e_n \),
each having type \( V \to \Real \), in this way:
\[
    e(v) = (e_1(v), \ldots, e_n(v))
\]

\section{Why are the basis and cobasis not the other way around?}

\enquote{A choice of an ordered basis for \(V\) is equivalent to a choice of a linear isomorphism \(\varphi\)
from the coordinate space \(F^n\) to \(V\).}%
\footnote{\url{https://en.wikipedia.org/wiki/Basis_(linear_algebra)\#Ordered_bases_and_coordinates}}

\section{Radial basis? Polar coordinates?}

Let \(e\) be the radial basis. Then \(e(v+w) \neq e(v) + e(w)\) in general.

\section{The cross product? The Levi-Civita symbol?}

The vector \( a \times b \) is the vector that is orthogonal to \(a\), orthogonal to \(b\).
Follow the right hand rule.
If \(a\) is represented by the thumb pointing right,
and \(b\) is represented by the index finger pointing forward,
then \(a \times b\) is represented by the middle finger pointing up.

\( a \times b = \sum_i\sum_j\sum_k \epsilon_{ijk} e_i a^j b^k \) ?

\footnote{\url{https://en.wikipedia.org/wiki/Levi-Civita_symbol\#Cross_product_(two_vectors)}}

\section{The Pythagorean theorem}

Let there be a right triangle.
Let \(c\) be its hypothenuse.
Let \(a\) and \(b\) be the other two sides.
Then, \( a^2 + b^2 = c^2 \).

Many proofs of this theorem are on the Internet.

\section{Closing}

With a basis, we can define vectors with numbers without drawing.
After we have a basis, we can use the calculus of infinitesimals on vectors.

    \chapter{Transformations}

\section{How transforming a vector transforms its coordinates}

We can imagine rotating a vector by 30 degrees,
but how do the coordinates change?

Let \(f : V \to V\) be a vector transformation.

Let \(T : V \to V\) be a linear function.

Let \(U : \Real^n \to \Real^n\) be a linear function.

Let \(v:V\) be a vector.

Let \(e:\Real^n \to V\) be a basis.

If \(v\) changes to \(T(v)\), then the \(e\)-coordinates change from \(x\) to \(U(x)\).

Let \(x = (x_1,\ldots,x_n)\).

We can explode \(U : \Real^n \to \Real^n\) into \(n\) functions
\(U_1,\ldots,U_n\), each of type \(\Real^n \to \Real\),
so that we can write \(U\) out as
\Formula{
    U(x) = (U_1(x), \ldots, U_n(x))
}

\(T\) is a vector transformation.
\(U\) is the coordinate transformation that corresponds to \(T\).

Let \(E\) be the basis.

Remember that we can write a vector as \(v = E(x)\).

Suppose that we transform a vector from \(v\) to \(T(v)\).

Thus \(T(v) = T(E(x)) = E(U(x))\).

Thus \(T \circ E = E \circ U\).

\section{Rotation in two-dimensional orthonormal basis}

The vector \(a\) rotated by \(t\) radians counterclockwise is the vector \(b\) having the same length
but such that the angle from \(a\) to \(b\) is \(t\).

Let \(\{i,j\}\) be an orthonormal basis.

It can be geometrically shown that
the result of rotating the vector \(xi+yj\) by \(a\) radians
is the vector \(x'i+y'j\) where
\begin{align}
    x' &= \cos(a) \cdot x - \sin(a) \cdot y
    \\ y' &= \sin(a) \cdot x + \cos(a) \cdot y
\end{align}
which we can write as the matrix equation
\Formula{
    \Matrix{x' \\ y'}
    = \Matrix{
        \cos(a) & -\sin(a)
        \\ \sin(a) & \cos(a)
    }
    \Matrix{x \\ y}
}
for which we can also write \( \text{rotate}(a, xi + yj) = x'i + y'j \).
We can also write \( \text{rotate}(e, a, (x,y)) = (x',y')\).
We can also write \( \text{rotate}(e, a, X) = R(a) \cdot X\).

If the basis is orthonormal, then we can define the rotation matrix
\Formula{
    R(a) = \Matrix{
        \cos a & -\sin a
        \\ \sin a & \cos a
    }
}
so that we can write the rotation as matrix multiplication
\( X' = R(a) \cdot X \).

Thus changing the vector \(v\) to \(\text{rotate}(a,v)\)
changes the \(e\)-coordinate tuple \(x\) to \(R(a) \cdot x\).

See also Wikipedia%
\footnote{\url{https://en.wikipedia.org/wiki/Rotation_(mathematics)\#Two_dimensions}}%
\footnote{\url{https://en.wikipedia.org/wiki/Rotation_matrix\#In_two_dimensions}}%
.

\section{Transforming the basis}

We have just rotated a vector.
What if we rotate the basis (the coordinate axes) instead?

See also Wikipedia%
\footnote{\url{https://en.wikipedia.org/wiki/Active_and_passive_transformation}}%
\footnote{\url{https://en.wikipedia.org/wiki/Change_of_basis}}%
\footnote{\url{https://en.wikipedia.org/wiki/Rotation_of_axes}}%
.

\section{Change of basis}

Let \( e : \Real^n \to V \) be a basis.

For example, if \(e(x,y) = xi + yj\).

Let \(t\) be rotation of 90 degrees counterclockwise.

We can think of a change of basis as an invertible function \( t : \Real^n \to \Real^n \).
Then, we can change the basis from \(e\) to \(e \circ t\).

We have a vector \(v : V\).
Its coordinate tuple under basis \(e\) is \(x\).
Its coordinate tuple under basis \(e \circ t\) is \(x'\).
\begin{align*}
    v &= v
    \\ e(x) &= (e \circ t)(x')
    \\ e(x) &= e(t(x'))
    \\ x &= t(x')
    \\ t^{-1}(x) &= x'
\end{align*}

Thus a vector is contravariant.

\section{Coordinateless and coordinateful description of scalar fields}

Let \(f : V \to \Real\) be a scalar field.
We call \(f\) coordinateless because \(f\) does not use any coordinates.

A coordinateful description of \(f\) under basis \(e : \Real^n \to V\) is
another function \(g : \Real^n \to \Real\)
such that \(g = f \circ e\).

How do we describe derivatives coordinatelessly?

\section{Keeping a function while changing basis: contravariance and covariance}

Let there be two functions \( p : \Real \to V \) and \( \phi : V \to \Real \).
Observe how their types mirror each other.
We will explain how \( p \) is contravariant and \( \phi \) is covariant.

An example of \(p\) is a parametric curve.

An example of \( \phi \) is a scalar field such as a temperature field,
which maps each point in space to the temperature at that point.
Another example is a height map,
which maps each point in space to the height of the terrain at that point.

Let \( p_e, p_f : \Real \to \Real^n \) be coordinateful descriptions of \( p \).
Let \(\phi_e : \Real^n \to \Real \) describe \(\phi\) using the basis \(e\).
Let \(\phi_f : \Real^n \to \Real \) describe \(\phi\) using the basis \(f\).
The subscript denotes the basis.
\begin{align*}
    e(p_e(t)) &= p(t)
    \\ f(p_f(t)) &= p(t)
    \\ \phi(v) &= \phi_e(e^{-1}(v))
    \\ \phi(v) &= \phi_f(f^{-1}(v))
\end{align*}

Even though \(p_e\) and \(p_f\) are different functions,
they describe the same coordinateless function \(p\), only with different bases.
We want to change the basis from \(e\) to \(f\), but we want \(p_f\) to describe \(p\).

We write \(f \equiv g\) to mean that \(f\) gives the same result as \(g\) for all parameters.
We have discussed this in \SectionRef{sec:function-equivalence}.
Thus, we can tidy up the equations as
\begin{align*}
    e \circ p_e &\equiv p
    \\ f \circ p_f &\equiv p
    \\ \phi &\equiv \phi_e \circ e^{-1}
    \\ \phi &\equiv \phi_f \circ f^{-1}
\end{align*}
Observe the inverses.

Let \( m : \Real^n \to \Real^n \) be the function that changes the basis from \( e \) to \( f \).
This means \( f \equiv e \circ m \).
(The other possibility \(m \circ e\) does not make sense because the types conflict.)

Then, observe how we have to change the coordinates to keep \(p\) and \(\phi\) the same:
\begin{align*}
    p &\equiv p
    &
    \phi &\equiv \phi
    \\
    e \circ p_e &\equiv f \circ p_f
    &
    \phi_e \circ e^{-1} &\equiv \phi_f \circ f^{-1}
    \\
    e \circ p_e &\equiv (e \circ m) \circ p_{e \circ m}
    &
    \phi_e \circ e^{-1} &\equiv \phi_{e \circ m} \circ (e \circ m)^{-1}
    \\
    e \circ p_e &\equiv e \circ m \circ p_{e \circ m}
    &
    \phi_e \circ e^{-1} &\equiv \phi_{e \circ m} \circ m^{-1} \circ e^{-1}
    \\
    p_e &\equiv m \circ p_{e \circ m}
    &
    \phi_e &\equiv \phi_{e \circ m} \circ m^{-1}
    \\
    m^{-1} \circ p_e &\equiv p_{e \circ m}
    &
    \phi_e \circ m &\equiv \phi_{e \circ m}
\end{align*}

Therefore, summarizing, we get
\begin{align}
    m^{-1} \circ p_e &\equiv p_{e \circ m}
    \\ \phi_e \circ m &\equiv \phi_{e \circ m}
\end{align}

We say that \(p\) is contravariant because of the \(m^{-1}\).

We say that \(\phi\) is covariant because of the \(m\).

A vector is contravariant.

Functions of type \( \Real \to V \) are contravariant.

Functions of type \( V \to \Real \) are covariant.

\section{Einstein notation?}

\paragraph{Subscripts and superscripts for coordinate tuple components?}

A vector is \(v = e(x^1,\ldots,x^n)\).

A parametric curve is \(p(t) = e((p_e(t))^1, \ldots, (p_e(t))^n)\).
We lift it to \(p = E(q^1, \ldots, q^n)\).

A covector is \(\phi(v) = \phi_e(x_1,\ldots,x_n)\).

We overload the superscript, which unfortunately has been used to mean raising a number to a power.
Don't confuse contravariant tuple component notation \(x^3\)
with power notation \(2^3 = 2 \times 2 \times 2\).

    \chapter{Frames}

(Wikipedia says frame (ordered basis) for what we say basis. Should we follow?)

\section{A frame is an origin point and a basis}

With a basis, we can describe every vector with tuples.

With a frame, we can describe every point with tuples.

A frame \(S\) is an origin point \( O \) and a basis \( E \).

We can describe a point \( P \) as \( O + OP \).
We then state \(OP\) as a linear combination of the vectors in \(E\).
Suppose that the linear combination is \(OP = x_1 e_1 + \ldots + x_n e_n\).
We call the tuple \( (x_1,\ldots,x_n) \) the \emph{\(S\)-coordinates} of \(P\).

\emph{Coordinate system} is another name for \emph{frame}.

\footnote{\url{https://en.wikipedia.org/wiki/Frame\#Mathematics}}

\section{Some coordinate systems}

A \emph{coordinate system} maps a coordinate tuple to a vector.

\subsection{The rectangular coordinate system}

Orthonormal basis.

\(R(x,y) = x e_1 + y e_2\).

\(R(x) = x_1 e_1 + x_2 e_2\).

In this system, the coordinates are the scalar coefficients in the linear combination of basis vectors.
The coordinates describe how the basis vectors should be linearly combined to form the described vector.

Let \(T : V^2 \to V^2\) be a linear transformation.
Then \(T(R(x)) = T(x_1 e_1 + x_2 e_2) = x_1 \cdot T(e_1) + x_2 \cdot T(e_2) = x_1 e_1' + x_2 e_2' = R'(x) \).

\subsection{The polar coordinate system}

\(P(r,t) = r e_1 \text{ rotated } t \text{ radians counterclockwise}\).

\section{Coordinate system transformations}

\subsection{Converting polar coordinate tuples to rectangular coordinate tuples}

Both the rectangular coordinate $(r\cos\theta, r\sin\theta)$ and the polar coordinate $(r,\theta)$
describe the same point in two-dimensional Euclidean space.
\[
R(r\cos\theta, r\sin\theta) = P(r,\theta)
\]

A point in a space can have different coordinates in different coordinate systems.

\section{Using coordinate systems to apply analysis to geometry}

Coordinate systems bridge synthetic geometry and analytic geometry.

A coordinate system transformation is a function taking a coordinate system and giving another coordinate system.

\section{Describing vectors with numbers without drawing}

Earlier in \ChapterRef{chp:vector},
we defined a vector as something telling us
how to go from an origin point to a destination point.

We can define three directions: right, forward, and up.
We can describe \emph{any} point in space by saying that the point is
\(x\) meters right, \(y\) meters forward, and \(z\) meters up from where we are standing.
We have just described a \emph{coordinate system}:
a mapping between a \emph{coordinate tuple} and a point in space.

A tuple and a coordinate system can represent a vector.

\( v = x_1 e_1 + x_2 e_2 \).

Describing a vector using bases:

\[
v = \sum_k x_k e_k
\]
where each \(x_k:\Real\) and each \(e_k:\Real^n\).

For example:

Let \( e_1 \) be a unit vector pointing right.

Let \( e_2 \) be a unit vector pointing up.

Then, \( 1 e_1 + 2 e_2 \) is a vector.

"The numbers in the list depend on the choice of coordinate system."%
\footnote{\url{https://en.wikipedia.org/wiki/Covariance_and_contravariance_of_vectors}}

\subsection{Basis vectors and coordinate axes}

An orthonormal basis makes good coordinate axes.%
\footnote{\url{https://en.wikipedia.org/wiki/Orthonormal_basis}}

The same point has different coordinates in different frames.

\section{Coordinate systems}

We can describe a point by a \emph{coordinate system} and a \emph{coordinate tuple}.

\index{definitions!coordinate}%
A \emph{coordinate} is a number.

\index{definitions!coordinate tuple}%
A \emph{coordinate tuple} is a tuple of \emph{coordinates}.
Example: \((1,2)\).

\index{definitions!coordinate system}%
A \emph{coordinate system} maps a coordinate tuple to a point in a space.

\index{definitions!\(n\)-dimensional real coordinate space}%
The \emph{n-dimensional real coordinate space} \( \Real^n \)
is the set of all \(n\)-tuples where every component is a real number.
It can \emph{represent} the \(n\)-dimensional Euclidean space.

The \(n\)-dimensional Cartesian coordinate system uses \(n\) basis vectors
to map \( \Real^n \) to \(n\)-dimensional Euclidean space.

\subsection{Drawing the standard two-dimensional Cartesian coordinate axes}

\subsection{Transforming the vector}

Also known as \emph{active transformation}.
This changes the vector.

We write \(v = E(x)\) to mean that \emph{the vector \(v\) is described by the coordinate tuple \(x\) under coordinate system \(E\)}.

Let \(v = E(x)\).
The vector \(v\) is described by the tuple \(x\) under coordinate system \(E\).

If we transform the vector to \(T(v)\),
then we will also transform the coordinate tuple to \(T(E(x))\),
which we can also write as \((T \circ E)(x)\) using function composition notation.

Thus, if \(v = E(x)\), then \( T(v) = (T \circ E)(x) \).

Therefore, transforming the vector by \(T\) has the same effect as
transforming the coordinate system (not the coordinate tuple) by \(T\).

\subsection{Transforming the basis}

\emph{Passive transformation}.
Changing the basis while still referring to the same point.
This does not change the vector.

Every vector can be stated as a linear combination of the basis vectors.

\index{definitions!coordinate transformation}%
A \emph{coordinate transformation} is a map from a coordinate system to a coordinate system.

There are many points in an Euclidean space, but none is special.
However, we can pick a point in Euclidean space, and call it \( O \), short for \emph{origin}.

We can describe the same vector using different coordinate tuples in different coordinate systems.
For example, the same vector \(v\) has coordinate tuples \(x\) in coordinate system \(E\)
and has coordinate tuples \(x'\) in coordinate system \(E'\).
\[
    v = E(x) = E'(x')
\]

\footnote{\url{https://en.wikipedia.org/wiki/Active_and_passive_transformation}}

Let \(T\) be a linear transformation.

If \(E'\) happens to be \(T \circ E\), then

\begin{align*}
    E(x) &= (T \circ E)(x') & \text{assumed}
    \\ E(x) &= T(E(x')) & \text{by definition of composition}
    \\ T^{-1}(E(x)) &= T^{-1}(T(E(x'))) & \text{applying \(T^{-1}\) to both sides}
    \\ (T^{-1}\circ E)(x) &= E(x')
\end{align*}

\subsection{Changing basis}

A basis transformation is a coordinate system transformation.
Such transformation has the type
$(\mathbb{R}^n \to \mathbb{R}^n) \to (\mathbb{R}^n \to \mathbb{R}^n)$.
Such basis transformation $E' = T~E$ can be written out
as $\vec{e}'_k = T_k~(\vec{e}_1, \ldots, \vec{e}_n)$,
which can be written out even more in the greatest detail as
\[
(\vec{e}'_k)_i = (T_k)_i~(\vec{e}_1, \ldots, \vec{e}_n)
\]
which are actually $n^2$ equations,
and each $(T_k)_i$ is a function that takes $n^2$ real numbers.
If the transformation is linear,
we can write the transformation as matrix multiplication:
\[
E' = ET
\]
which expands into $n$ equations each:
\[
\vec{e}_i' = \sum_{k=1}^n T_{ik} \vec{e}_k
\]
which, in turn, expands into $n$ equations again each:
\[
e_{ij}' = \sum_{k=1}^n T_{ik} \cdot e_{kj}
\]

Sometimes we want to use another basis to locate the same point.
That is, if we have a vector $x$ in basis $E$, and a vector $y$ in basis $TE$,
we want $Ex = TEy$.
We want to change the basis from $E$ to $TE$,
so old coordinate $x$ become $y$,
but we want $y$ in $TE$ such that $TEy$ is still $Ex$,
which we can state mathematically:
\begin{align*}
TEy &= Ex
\\ T^{-1}TEy &= T^{-1}Ex
\\ Ey &= T^{-1}Ex
\end{align*}
which says that the change of coordinate goes against the change of basis,
so we say that the vector $x$ is *contravariant* to the basis transformation $T$.
This means that if we move the origin of the coordinate system 3 units to the right,
then we will have to move the coordinates 3 units to the left
if we want the new coordinate to locate the same point.

What about covectors?
Covector $f$.
\[
f(TEy) = f(Ex)
\]
Due to the linearity of covectors:
\begin{align*}
f(Ex) &= f\left(\sum_i x_i \vec{e}_i\right) = \sum_i f(x_i \vec{e}_i)
\\ f(TEy) &= f\left(\sum_i y_i T\vec{e}_i \right) = \sum_i f(y_i T\vec{e}_i)
\\ \vec{e}'_k = \sum_i T_{ki} \vec{e}_i
\end{align*}

(todo show that covector is covariant)

\section{Coordinate transformation}

Let's say we have one space $S$,
and two coordinate systems $J : M \to S$ and $K : N \to S$.
A coordinate transformation from $J$ to $K$ (we name this transformation $T : M \to N$)
transforms a $J$-coordinate
to an $K$-coordinate describing the same point.
This transformation relates both coordinate systems as
$J~x = K~(T~x)$, which means that if $x$ is a coordinate in $J$, then $T~x$ is a coordinate in $K$ locating the same point.
We can also write the equation as $J = K \circ T$.

To consider each component of the coordinate separately,
we write
$T~(x_1,\ldots,x_m) = (y_1,\ldots,y_n)$
where
\begin{align*}
y_1 &= t_1~(x_1,\ldots,x_m)
\\ &\vdots
\\ y_n &= t_n~(x_1,\ldots,x_m).
\end{align*}
Thus we can think of $T$ as an $n$-tuple $(t_1,\ldots,t_n)$
whose each component $t_k$ takes an $m$-tuple.
We can extend the definition of function application so that it works on tuples:
\[
(t_1,\ldots,t_n)~\vec{x} = (t_1~\vec{x}, \ldots, t_n~\vec{x})
\]
but why would we?

If we define that, we can see how a change in a tuple component translates
to a change in the other tuple component.
\[
\frac{\partial y_i}{\partial x_j} = pd~j~t_i
\]

A space mapping maps the underlying space $M : R \to S$.

The transformation $T~x = 2\cdot x$ can be viewed as two things:
as a space transformation, it makes everything bigger;
as a coordinate transformation, it makes everything smaller.
Coordinate transformation works in reverse.
You can move the point, or you can change the coordinate.

Covariance.

Distance-preserving transformation.
$d~(T~x)~(T~y) = d~x~y$.

$T$-symmetry.
$f~(T~x) = T~(f~x)$?

Inverse transformation.
Composition of transformations.

We can use several coordinate systems on the same space.
Some coordinate systems are more convenient to work with.
To specify a place on Earth, we can use the
geographic coordinate system (latitudes and longitudes).
% https://en.wikipedia.org/wiki/Geographic_coordinate_system
A coordinate transformation does not move the point.

If a coordinate system is a bijection,
then it describes an isomorphism between
its coordinate space and the space it describes.
This means we can pick any of them we find most convenient,
and whatever works with it will work with the other.

Let's say we have two spaces $R$ and $S$.
$T : R \to S$.

Embedding and projection are mappings between spaces.
Embedding maps a lower-dimensional space to a higher-dimensional space;
projection maps a higher-dimensional space to a lower-dimensional space.

\subsection{Picking a frame: The standard basis vectors}

The standard basis vectors of $\mathbb{R}^3$
is $\{e_1,e_2,e_3\}$
where $e_1 = (1,0,0)$, $e_2 = (0,1,0)$, and $e_3 = (0,0,1)$,
which can also be written as the matrix
\[
    \Matrix{ e_1 & e_2 & e_3 }
    = \Matrix{ 1&0&0\\0&1&0\\0&0&1 }
\]
where each basis vector becomes a column in the matrix.

A coordinate system transformation multiplies the basis with the transformation matrix.

\subsection{Curvilinear coordinate system: polar coordinate system}

\section{Understanding a covector as a linear function}

Every linear endofunction of an $n$-dimensional vector space
can be written as a vector of $n$ of inner products.

\section{Understanding how cobasis is to covector as basis is to vector}

We can derive a basis for $V^*$ from a basis for $V$.
Let us write the basis for $V$ as \(E = \{ e_1, \ldots, e_n\}\).
We then define $G = \{ g_1, \ldots, g_n \}$ where \(g_k(x) = x \cdot e_k\).
If such $G$ also spans the entire $V^*$,
then $G$ is a basis for $V^*$,
and we call such $G$ the dual basis of $E$.

\section{Multiplying a covector and a vector}

The product between a covector $f : V^*$ and vector $\vec{x} : V$
is simply $f~\vec{x}$ (function application)?

\section{Eigenvalues, eigenvectors, and fixpoints of transformations? Why are we talking about this?}

We say that a point is a fixpoint (invariant)
of a transformation iff the transformation maps it to itself.
$f(x) = x$.
We can also say that $f$ is a fixtransform of $x$.

We can embed a plane on a sphere.

To describe an embedding,
we can use words, or we can use algebra:
we pick a coordinate system for each space,
and describe the embedding of the coordinates.
An embedding $E : R \to S$ is a mapping from a space $R$ to (a subset of) another space $S$.
Let $J : M \to R$ and $K : N \to S$ be coordinate systems for those spaces, respectively.
Let $T : M \to N$ be the coordinate transformation that corresponds to that embedding.
Let $x$ be a $J$-coordinate.
Then we have:
\[
K~(T~x) = E~(J~x)
\]
which can also be written
\[
K \circ T = E \circ J.
\]
Spherical coordinate system
use three: radius,: $(r,\theta,\phi)$. Here the radius is fixed so we can use $(\theta,\phi)$.

\section{Moving points around}

\subsection{Translating a point}

We can translate a point \(P\) by a vector \(v\).
The result is the point \(P + v\).

\subsection{Rotating a vector}

Rotation is always done with respect to the origin.
If you need otherwise, then translate, rotate, and inverse-translate.

Two-dimensional rotation is simple because the axis of rotation is a point.
Three-dimensional rotation is more complex because the axis of rotation is a line.

Let \(x\) be the vector that we want to rotate.

Let \(a\) be the axis of rotation.

Let \(n\) be orthogonal to \(a\).
Let \(a + n = x\).
Thus, \(a\), \(x\), and \(n\) form a right triangle, with \(x\) as the hypothenuse.

Let \(\theta\) be the angle of rotation.

Let \(A(x,y)\) be the angle from \(x\) to \(y\).

The result of rotating a vector \(x\) about the axis \(a\) by angle \(\theta\) is the vector \(y\).

\(\norm{x} = \norm{y}\)

\(A(a,x) = A(a,y)\).

\(A(n,x) + \theta \equiv A(n,y)\).

\subsection{Applying a rotation matrix to a vector}

This matrix%
\footnote{\url{https://en.wikipedia.org/wiki/Rotation_matrix}}
describes
two-dimensional rotation by a counterclockwise angle of \(\theta\):
\Formula{
    \Matrix{
        \cos\theta & -\sin\theta
        \\ \sin\theta & \cos\theta
    }.
}
Let that matrix be \(R\).
Let \(v\) be a vector.
Then \(Rv\) is \(v\) rotated by \(\theta\).

\section{Moving frames around without moving points}

\subsection{Locating the same point with different coordinate systems}

Example of coordinate transformation:
The same point in the same two-dimensional Euclidean space
is described by
both the polar coordinates \( (r,\theta) \)
and the rectangular coordinates \( (r \cos \theta, r \sin \theta) \).
The transformation is \( (r,\theta) \to (r \cos \theta, r \sin \theta) \).

% https://en.wikipedia.org/wiki/Real_coordinate_space

A \emph{coordinate system} $M : C \to S$ is a surjective mapping from
\emph{coordinate space} $C$ to \emph{target space} $S$.

A \emph{coordinate} is a point in \(C\).
The coordinate system tells us how to get to a point.

The \(n\)-dimensional real coordinate space is $\mathbb{R}^n$.
% https://en.wikipedia.org/wiki/Real_coordinate_space
It is also called the real $n$-space.
A point in the real $n$-space is an $n$-tuple of real numbers $(x_1,\ldots,x_n)$.

$(x,y)$ is the tuple of coordinates,
$x$ is the x-coordinate, and $y$ is the y-coordinate.

Coordinate systems unify geometry and
% https://en.wikipedia.org/wiki/Mathematical_analysis
mathematical analysis.
With coordinates,
we can solve geometric problems by
numbers, calculus, and algebra,
so that computers can
find the intersection of geometric objects
by solving the corresponding system of equations,
and find the size of a geometric object by solving the corresponding integral.

\section{Fields}

A field is a function that assigns something to each point in space.

A vector field is a function \( f : P \to V \).

We need two bases to coordinatefully describe a vector field.
One basis to turn \( P \) to \(\Real^m\).
Another basis to turn \(V\) to \(\Real^n\).

The coordinateful description of a field \(f\) under basis \(?,?\) is \(f_? : \Real^m \to \Real^n\).

A force field assigns a force vector to each point in space.

A scalar field assigns a scalar to each point in space.
An example of a scalar field is a temperature field.%
\footnote{\url{https://en.wikipedia.org/wiki/Temperature_gradient}}

Let \(T(P)\) be the temperature at point \(P\).
Because \(P = O + \sum_k x_k e_k\), we have \(T(O + \sum_k x_k e_k)\).

Mathematics and physics use the same term \enquote{field} to mean different things.

\footnote{\url{https://en.wikipedia.org/wiki/Field_(physics)}}

    \chapter{Matrices}

The plural of \emph{matrix} is \emph{matrices}.

Matrices were invented to abbreviate systems of linear equations?%
\footnote{\url{https://en.wikipedia.org/wiki/Matrix_(mathematics)\#History}}

A matrix is a rectangular array of numbers.

We write \(A_{ij}\) or \(A(i,j)\) to mean the cell at row \(i\) column \(j\).

\paragraph{Finding the dimension of a matrix}

We say that the dimension of a matrix \(A\) is \((m,n)\),
written \(\dim(A) = (m,n)\),
iff the matrix has \(m\) rows and \(n\) columns.

\paragraph{Example}
The following matrix has 2 rows and 3 columns, so its dimension is \((2,3)\).
\Formula{\NoNumber
    \Matrix{
        A_{11} & A_{12} & A_{13}
        \\
        A_{21} & A_{22} & A_{23}
    }
}

\section*{Don't confuse vectors and coordinate tuples}

Use the phrases \emph{column matrix} and \emph{row matrix}
instead of \emph{column vector} and \emph{row vector}.%
\footnote{\url{https://en.wikipedia.org/wiki/Row_and_column_vectors}}

The phrase \emph{column vector} confuses a geometric vector and
the coordinate tuple that describes the vector.

A vector is \emph{not} a bunch numbers.
A vector is a geometric object with length and direction.
A vector can be \emph{described} by a coordinate tuple under a coordinate system.

\section{Working with matrices}

\subsection{Multiplying matrices}

Matrix addition is simple:
add the elements at the same positions:
\Formula{
    (AB)_{ij} = A_{ij} + B_{ij}
}

The cell \((AB)_{ij}\) depends on row \(i\) of \(A\) and column \(j\) of \(B\).
\Formula{
    (AB)_{ij} = \sum_k A_{ik} B_{kj}
}

\subsection{Abbreviating a system of linear equations into a matrix equation}

A matrix \enquote{packs} a system of linear equations into one equation.

\[ AB = C \]

\subsection{Linear function? Using matrix to describe a linear transformation?}

A matrix is a linear map?

A matrix is a vector of covectors?

A linear function is called linear because it describes a line.

A function \(f\) is \emph{linear} iff \( f(x+y) = f(x) + f(y) \).

A \emph{transformation} is another word for function.

Transformation.
Linear transformation.
$T(ax+by) = aTx + bTy$.

Matrix arises naturally for describing linear transformations on a vector space.

\section{Generalizing vectors, matrices, and tensors to functions of indexes}

The notation \(A(i)\) suggests that a vector is a function \( I \to \Real \).
The notation \(A(i,j)\) suggests that a matrix is a function \( (I,J) \to \Real \).

\section{Thinking of a matrix as a linear operator}

\section{Tuple vs array vs matrix vs tensor}

An array is a bunch of numbers.

A matrix is a \emph{rectangular} array for which matrix operations are defined.

A tensor carries with it a coordinate system?

A tensor is \emph{not only} a multi-dimensional array of numbers.

A tuple is a one-dimensional (linear) array.

A matrix is a two-dimensional (rectangular) array.

A tensor is a multi-dimensional array.

    \chapter{Metrics}

A metric is a function that measures the distance between two points.

Let \(S\) be a space.

Let \(d : S \to S \to \Real\) be a metric of \(S\).

Thus \(d(x,y)\) is the distance from \(x\) to \(y\).

Thus \(d\) should satisfy \(d(x,y) = d(y,x)\).

\section*{Euclidean metrics of real tuple spaces}

Let \(d : \Real^n \to \Real^n \to \Real\).
Define \(d(x,y) = \sqrt{\sum_{k=1}^n (x_k - y_k)^2}\).
Then \(d\) we say that \(d\) an Euclidean metric of \(\Real^n\).

\section{Norms and metrics induce each other}

A norm induces a metric.
A metric induces a norm.%
\footnote{\url{https://en.wikipedia.org/wiki/Metric_(mathematics)\#Metrics_on_vector_spaces}}

\section{Geodesics}

The distance between two points is the length of the shortest curve that connects them.

\section{Measuring distances on curved spaces}

Example: circle

Draw a circle with center \(C\).
Draw two points \(P\) and \(Q\) on the circle's perimeter.
Then \(CPQ\) forms a sector whose arc is \(PQ\).
The distance from \(P\) to \(Q\) is that arc's length.

Every point the circle can be described by a real number \(a\).
This real number is the \enquote{angle} of the point.

\(d(a,b) = r \cdot (b-a)\).

TODO define a metric for a curve described by \(x : \Real \to \Real^2\).

A curve is in length-normal formulation (???)
iff \( L(t,u) = \abs{t-u} \).
Then the metric is \(d(x,y) = L(f^{-1}(x), f^{-1}(y))\).

We can also define.
\(d(a,b) = b-a\).

\footnote{\url{https://en.wikipedia.org/wiki/Intrinsic_metric}}

Draw a sphere with center \(C\).
Draw two points \(P\) and \(Q\) on the sphere.

\section{Generalizing shortest paths into geodesics}

\subsection{Understanding the shortest curve connecting two points}

The concept of geodesic arises naturally when we realize
that \emph{straightness} is defined in terms of \emph{distance}.
If we define a \emph{straight} line segment as the \emph{shortest path} connecting its endpoints,
then the concept of geodesics arises naturally by redefining what \emph{shortest} means.

In Euclidean geometry, the shortest path connecting two points is a line segment.
Remember that a line is a straight curve.

We can also write \emph{shortest} as \emph{having minimum distance},
to clarify the connection between straightness and distance.

A curve segment connecting two points is a \emph{geodesic}
iff the length of the curve segment is the distance between the two points.

If \(x\) is a point on a geodesic,
then every point \(x + h\) satisfying \(\norm{h} = d(x,x+h)\) is also on the geodesic.
Can this equation be used to derive the equation of a geodesic?
The definitions of the norm \(\norm{\cdot}\) and the distance \(d\) depend on the ambient space.

Thus geodesic is a generalization of straightness.
By \emph{straight}, we mean \emph{shortest} (having minimal length).
Every line is a geodesic.

A geodesic is a locally length-minimizing curve.%
\footnote{\url{http://mathworld.wolfram.com/Geodesic.html}}

In metric geometry, a geodesic is a curve which is everywhere locally a distance minimizer.%
\footnote{\url{https://en.wikipedia.org/wiki/Geodesic\#Metric_geometry}}

\subsection{Finding the equation of a geodesic}

If the curve is given by parametric equation,
then a geodesic equation can be obtained by minimizing the arc length of the curve?%
\footnote{\url{http://mathworld.wolfram.com/Geodesic.html}}

\section{Metric tensor}

We have always implicitly assumed that a space defines distance uniformly,
the same everywhere, that is, we have always assumed the isometry \(d(x+h,y+h) = d(x,y)\) for all \(h\).

\footnote{\url{https://en.wikipedia.org/wiki/Metric_tensor}}%
\footnote{\url{https://en.wikipedia.org/wiki/Arc_length\#Generalization_to_.28pseudo-.29Riemannian_manifolds}}

A metric tensor is a metric in tensor form?
A rank-\((a,b)\) tensor is a function with \(a+b\) parameters?

Dot product generalizes to inner product.

Typing rule:
if \( a, b : V \), then \( \langle a, b \rangle : \Real \).
An inner product is a function with type \( V \to V \to \Real \).

In Euclidean vector spaces,
\( \langle x, y \rangle = x \cdot y \).
The dot product is the inner product of Euclidean vector spaces.

An inner product defines orthogonality. Two vectors are orthogonal iff \( \langle x, y \rangle = 0 \).
An inner product also defines the length of a vector.
\[
\norm{x} = \sqrt{\langle x, x \rangle}
\]

Tissot's indicatrices visualize a metric tensor by scattering
many circles throughout a space
so that we can see how the space's metric tensor distorts them.
This method tells us how much our map is lying to us.%
\footnote{\url{https://en.wikipedia.org/wiki/Tissot\%27s_indicatrix}}
\disabled{}%vim indent bug

\( d(0,x) = \norm{x} \) only in Euclidean geometry?

Perhaps we should digress to mapmaking/cartography?

Motivate metric tensor using cartography?
The surface of the Earth is a sphere.
How do we project its surface to a flat paper?
How do we draw a map?

Inner product generalizes to metric tensor?

A metric tensor's type is also \( V \to V \to \Real \).

Metric on half circle \( \{ (r \cos t, r \sin t) ~|~ t \in [0,\pi] \} \).

\section{Wrong-but-useful is better than right-but-useless}

Ancient people believe that the Earth is a plate.

Most people in 2017 believe that the Earth is a ball.

The Earth is an oblate spheroid.
It's a slightly flattened sphere.
It's wider on the equator.

The Earth is irregular.

Note how the statements become more correct, less general, and less applicable.

Earth is a plate, ball, or spheroid, depending on what we are doing.
If we're driving a car, the Earth is a plate.
If we're trying to introduce differential geometry by analogy with the Earth, it's a ball.
If we are comparing the gravitational pull in many places,
then the Earth is a spheroid.
If we are trying to one-up someone else, then the Earth is irregular;
it has mountains and lakes.

We're abusing language.
What do we mean by \enquote{is}?

Wrong-but-useful is better than right-but-useless.

    \chapter{Kinematics, and the calculus of infinitesimals}
\label{sec:derivative}

Kinematics is the mathematics that describes motions.

A \emph{frame} defines \emph{where} and \emph{when}.

\emph{Motion} is change of position.%
\footnote{\url{https://en.wikipedia.org/wiki/Motion_\%28physics\%29}}

An object \emph{moves} iff its position changes.

The \emph{speed} of an object is how fast it moves:
how far it moves in how much time.
\emph{Fast} means high speed,
going far in little time,
traveling much distance in little time.

\emph{Average speed} is distance traveled divided by time required.

\emph{Velocity} is the rate of change of position.
Speed is the magnitude of velocity.
\emph{Rate of change} is defined by \emph{derivative} (\S\ref{sec:derivative}).

\section{Describing motions using equations relating position and time}

We can use a position function.
Its type is \( \Real \to V \).

An example of an equation of motion is \( x(t) = 2 t e_1 \).
It describes an object that moves with constant velocity \(2\) towards the positive x-axis.
An \emph{equation of motion} is an equation that describes
the motion of an object by relating time and position.

Each equation of motion corresponds to a moving object.

To describe more objects, use more equations.

Let \(e\) be a linear basis.
Suppose that the position of an object at time \(t\) is
\(x(t) = e(x_1(t), \ldots, x_n(t))\).
Then the velocity at time \(t\) is \(v(t) = \der(x,t) = e(v_1(t), \ldots, v_n(t)) \).
Can we say that \(v_k(t) = \der(x_k,t)\)?

Moral of the story:
If we have a linear basis,
then doing calculus on the coordinates
is doing calculus on the vectors.

\section{Notating derivatives}

Let \(f : \Real \to \Real\).

\paragraph{\der}

We can describe the derivative by a function \(\der : (\Real \to \Real) \to (\Real \to \Real)\).
\[
    \der(f,x) = \StandardPart\parenthesize{\frac{f^*(x+\delta)-f^*(x)}{\delta}}
\]
where \(f^*\) is the natural extension of \(f\) to the hyperreals.%
\footnote{\url{https://en.wikipedia.org/wiki/Non-standard_calculus\#Definition_of_derivative}}
Then \(\der(f,x)\) is the slope of the tangent line of \(f\) at \(x\).

Strictly, \(\der(f)(x)\), not \(\der(f,x)\).

Thus \(f' = \der(f)\).

\(\dd{x}\) means an infinitesimal change in \(x\)?
What does that mean?
What is \(x\)?

\paragraph{Euler\textendash{}Arbogast D-notation}

% https://en.wikipedia.org/wiki/Notation_for_differentiation#Euler.27s_notation

\( D_x E \) is the derivative of expression \(E\) with respect to variable \(x\).
The variable \(x\) should occur free in \(E\).

Typing rule:
If \(x:\Real\) and \(E:\Real\), then \(D_x E:\Real\).

The notation \(E[x:=y]\) means \(E\) but with each free occurrence of \(x\) replaced with \(y\).

Then \( D_x E = \lim_{h \to 0} \frac{E[x := x+h] - E}{h} \).

Then \( D_x E = d(x \to E) \).

Example: \( D_x (4x^2) = d(x \to 4x^2) = 8x \).

Example: \( D_y (x + y) = d(y \to x + y) = x + 1 \).

Advantage: With \(D_x\) notation, we can refer to an input by name;
With \(d_k\) notation, we can refer to an input by index only.

\(D\) is a custom syntax, not an ordinary function.

% Multivariate differential calculus
% Vector calculus

\section{Relating velocities, tangent lines, and derivatives}

There are several ways of understanding \(f'(x)\) (the derivative of \(f\) at \(x\)):
\UnorderedList{
    \item rate of change of \(f\) at \(x\); instantaneous velocity
    \item slope of the tangent line of \(f\) at \(x\)
    \item best linear approximation (not discussed here)
}

\paragraph{Average velocity and the secant line}

Let there be an object.

Let \(x(t) : V^2\) be a vector that describes its position at time \(t : \Real\).

The \emph{average velocity} of that object in the time interval \([t,t+\Delta t]\) is
\[ \frac{x(t+\Delta t) - x(t)}{\Delta t}. \]

If at time \(t_1\) its position is \(x_1\)
and at time \(t_2\) its position is \(x_2\),
then its \emph{average velocity} in the time interval between \(t_1\) and \(t_2\)
is \((x_2 - x_1) / (t_2 - t_1)\).

A \emph{secant line of \(f\)} is a line that passes \((x_1,f(x_1))\) and \((x_2,f(x_2))\).
Think of average velocity.

\paragraph{Instantaneous velocity and the tangent line}

If the position of an object at time \(t\) is \(x(t)\),
then its \emph{instantaneous velocity} at time \(t\) is \(v(t) = (d(x))(t)\).
The velocity function is the derivative of the position function.

The term \emph{instantaneous velocity} is often shortened to just \emph{velocity}.

The unqualified \emph{velocity} means \emph{instantaneous velocity}.

A car's speedometer measures its instantaneous speed.

Derivative is about \emph{rate of change}:
how fast a function changes value,
how big is the change in output compared to the change in input.

Consider a function \(f : \Real \to \Real\).
If the input is \(x\), then the output is \(f(x)\).
If you change the input by \(\dd{x}\), the output changes by \(\dd{y}\).
Formally, \(f(x+\dd{x}) = f(x)+\dd{y}\).

A \emph{tangent line of \(f\) at \(x\)} is what the secant line converges to
if both \(x_1\) and \(x_2\) converge to \(x\).
Think of instantaneous velocity.

\paragraph{Understanding the derivative as the slope of the tangent line}

The \emph{derivative of \(f\) at \(x\)} is the slope of the tangent line of \(f\) at \(x\).
Reminder: The line \(y = mx + c\) has slope \(m\).

\section{Describing motions using implicit equations}

An example of \emph{implicit} equation is \( x(t) = - (d(d(x)))(t) \).
This is also an example of a \emph{differential equation} because it contains the derivative operator \(d\).
Differential equations are discussed in \S\ref{sec:diff-eqn}.

\section{Integrals}

We can think of an integral in several ways:
\UnorderedList{
\item area under a curve
\item slicing and summing
}

\paragraph{Integrating by slicing and summing}

\section{Calculating derivatives and integrals quickly}

\paragraph{Calculating derivatives}

Symbolic calculation is enabled by
the \emph{constant rule} \eqref{der-constant-rule},
the \emph{power rule} \eqref{der-power-rule},
the \emph{product rule} \eqref{der-product-rule},
and the \emph{chain rule} \eqref{der-chain-rule}.
\begin{align}
    d (x \to y) &= 0 \text{ if \( y \) is a constant} \label{der-constant-rule}
    \\ d (x \to x^p) &= p \cdot x^{p-1} \text{ if \( p \) is a non-zero real number} \label{der-power-rule}
    \\ d(f \cdot g) &= d(f) \cdot d(g) \label{der-product-rule}
    \\ d(f \circ g) &= d(g) \cdot (d(f) \circ g) \label{der-chain-rule}
\end{align}

The \(d\) operator is linear:
If \(c\) is a constant, then \(d(c \cdot f) = c \cdot d(f)\).
Also, \(d(f+g) = d(f) + d(g)\).

\paragraph{Calculating integrals}

With the \emph{fundamental theorem of calculus},
we can compute integrals using antiderivatives.
With this shortcut, we can skip the slicing and summing.

\Formula{
    \int_{[a,b]} f(x) \dd{x} \dd{y} \dd{z} = F(b) - F(a)
}

\section{Exercise}

Compute the derivative of each of these functions:
\( x \to 1 \), \( x \to x \), \( x \to 2x \), \( x \to x^2 \),
\( x \to 3e^x \), \(x \to e^x \cdot x \).

Using the power rule, show that \( d(x \to x^2) = 2x \).

Show that \( d(x \to x^3 + x^2) = 3x^2 + 2x \).

    \chapter{Curvature}

Having defined \emph{spaces} \ParenRef{chp:manifold},
we now think about their \emph{curvature},
which we have to understand for general relativity%
\footnote{\url{https://en.wikipedia.org/wiki/Mathematics_of_general_relativity}}%
\footnote{\url{https://en.wikipedia.org/wiki/Introduction_to_the_mathematics_of_general_relativity}}%
.

There are several ways to express the curvature of a space.%
\footnote{\url{https://en.wikipedia.org/wiki/Curvature_of_Riemannian_manifolds\#Ways_to_express_the_curvature_of_a_Riemannian_manifold}}
What are they?
What are their benefits and drawbacks?
\enquote{the most standard one is the curvature tensor}

\section{Finding the normal bundle of a space}

\subsection{Finding the normal line at a point of a curve in \(\Real^2\)}

\subsection{Finding the normal plane at a point of a surface in \(\Real^3\)}

\section{The curvature of a curve at a point, and the circle that osculates the curve at that point}

\paragraph{Understanding curvature intuitively}

We have learned to describe shapes using equations and functions.
We have seen that some shapes are more sharply bent than others.
\enquote{Higher curvature} means \enquote{more sharply bent}.

\paragraph{Visualizing the osculating circle}

The \emph{center of curvature} at \(P\) is the point where two normal lines very close to \(P\) intersect.

Read the Wikipedia article about \emph{osculating circle}%
\footnote{\url{https://en.wikipedia.org/wiki/Osculating_circle}}.
\emph{Osculate} means kiss.

Read the Wikipedia article about \emph{curvature}\footnote{\url{https://en.wikipedia.org/wiki/Curvature}}.

The \emph{curvature} of a curve at point \(P\) is the reciprocal of the radius of the osculating circle of the curve at \(P\).
Formally, \( K = 1/R \).

\paragraph{Finding the center of curvature}

\paragraph{Finding the radius of curvature}

\paragraph{Defining the scalar curvature as the reciprocal of the radius of curvature}

\section{Measuring the curvatures of a parabola}

\section{Riemann tensors}

Other names: Riemann curvature tensor,
Riemann-Christoffel curvature tensor.%
\footnote{\url{http://mathworld.wolfram.com/RiemannTensor.html}}

Do we need to understand the Riemann tensor?
Is the Ricci tensor not enough?

\footnote{\url{https://en.wikipedia.org/wiki/Curvature_of_Riemannian_manifolds}}
\footnote{\url{https://en.wikipedia.org/wiki/Riemann_curvature_tensor}}

Levi-Civita connection, affine connection, metric connection

Riemann tensor generalizes Ricci tensor and scalar curvature.
We can contract a Riemann tensor into a Ricci tensor.%
\footnote{\url{http://mathworld.wolfram.com/RicciCurvatureTensor.html}}%
\footnote{\url{http://mathworld.wolfram.com/RiemannTensor.html}}

\footnote{\url{https://en.wikipedia.org/wiki/Ricci_curvature}}

\paragraph{Understanding the two-dimensional case}

\section{Why is the second derivative not enough for describing curvature?}

    \chapter{Geometry kitchen sink}
\label{chp:geometry}

\section{Combining synthetic and analytic geometry? Do we still need this distinction?}

Synthetic geometry uses axioms; it does not use formulas.%
\footnote{\url{https://en.wikipedia.org/wiki/Synthetic_geometry}}

Analytic geometry uses formulas from mathematical analysis.

Analytic geometry is a marriage between analysis and geometry.

When asked "What is a line in a 2-dimensional Euclidean space?",
a student of
synthetic geometry
may answer "It is the infinite extension (in both direction)
of the shortest geometric object connecting two points,"
while a student of
analytic geometry
may answer
``It is $\{ (x,y) ~|~ ax+by=c \text{ where } a,b,c\in\Real \}$.''
% https://en.wikipedia.org/wiki/Synthetic_geometry
% https://en.wikipedia.org/wiki/Analytic_geometry
% http://en.wikipedia.org/wiki/Foundations_of_geometry
% http://en.wikipedia.org/wiki/Timeline_of_geometry
% https://en.wikipedia.org/wiki/Algebraic_geometry
Algebraic geometry studies the solution of equations.

Topology studies topological spaces.
Analytic geometry uses coordinates to define spaces,
while topology uses neighborhood to define spaces.

\section{Understanding the advantages of synthetic geometry?}

Why do we begin with synthetic geometry?
Why don't we just go straight to analytic geometry?

\section{Rambling, too abstract}

What makes geometry interesting is not the spaces,
but the mapping between spaces?

(Why are we telling the reader this?)

There are many kinds of spaces we will talk about:

* algebraic structures (especially fields),
* coordinate space (a space of tuples),
* real coordinate space (a space of tuples of real numbers),
* vector space (a coordinate space with vector operations),
* inner product space (a space with an inner product),
* covector space (a space whose member is a covector, defined below).
* tensor space.

There are many ways to make a mapping between two spaces,
as we will see:

* physical field (mapping between two spaces, usually from a coordinate space),
* coordinate system (surjective mapping between two spaces),
* natural vectorization of a ring (mapping from a ring to one of its natural vector spaces),
* basis (linear coordinate system for a vector space),
* covector (linear mapping from a vector space to its ring space),
* cobasis (basis for a covector space).

Then, we will eat our own tail by stating that all those mappings are members of their respective function spaces
(a function space is a space whose members are functions):

* coordinate system transformation (a mapping between two mappings),

\section{Measuring distance using metrics}

With a metric, we can define
a (generalized) circle with radius $r$ and center $c$
to be the set of points
$\{x ~|~ d~c~x = r \}$.

An astronomer may need to tell another astronomer where the latter can find a celestial body.
He can give its coordinates in a coordinate system they have agreed on.
The astronomer can use a celestial coordinate system.
% https://en.wikipedia.org/wiki/Celestial_coordinate_system

\subsection{Measuring distance with Euclidean metrics}

\index{definitions!Euclidean space}%
The two-dimensional Euclidean space
may be imagined as an unbounded sheet of flat paper.

The \emph{Euclidean metric} for \(\Real^n\) is \( d(x,y) = \sqrt{\sum_{k=1}^n (y_k-x_k)^2} \).
It is the length of the shortest straight line segment from \(x\) to \(y\) in Euclidean geometry.
The two-dimensional Euclidean metric is the Pythagorean theorem.

Do not confuse Euclidean metric \(d\) and derivative operator \(d\).

\section{Not yet read}

\enquote{Two Approaches to Modelling the Universe: Synthetic Differential Geometry and Frame-Valued Sets},
John L. Bell\footnote{\url{http://citeseerx.ist.psu.edu/viewdoc/download?doi=10.1.1.114.1930&rep=rep1&type=pdf}}

\enquote{The power of analytic geometry derives very largely from the fact
that it permits the methods of the calculus, and, more generally, of
mathematical analysis, to be introduced into geometry.} (p.~1)

Foundations of geometry\footnote{\url{https://en.wikipedia.org/wiki/Foundations_of_geometry}}

Hilbert's axioms\footnote{\url{https://en.wikipedia.org/wiki/Hilbert\%27s_axioms}}

Birkhoff's axioms\footnote{\url{https://en.wikipedia.org/wiki/Birkhoff\%27s_axioms}}

\section{Mapping \(\Real^n\) to itself? Why?}

We have shown that we can use a real $n$-tuple as a coordinate to the real $n$-space.
In other words, we can map the real $n$-space to itself.
The identity coordinate system for the real $n$-space is
$I : \mathbb{R}^n \to \mathbb{R}^n$ where $I(x) = x$;
this $I$ is also the standard basis for the real $n$-space.

We also call a vector space a linear space.

There is another way to think of a coordinate in $\mathbb{R}^n$:
as a function $N \to \mathbb{R}$ where $N = \{1,2,3,\ldots,n\}$.
We can think of an $n$-dimensional $F$-vector as a function $N \to F$ where $N = \{1,2,3,\ldots,n\}$.
We can forget about dimensions and think of a $F$-vector as a function $\mathbb{N} \to F$.

We can also use vector spaces to talk conveniently about geometric objects.
With vector spaces, we can define a line by vector collinearity,
and define a plane by its normal vector.

Let there be two bases $J$ and $K$.
Let $T$ be coordinate transformation from $J$ to $K$
(that is $J~x = K~(T~x)$)
and $U$ be basis transformation from $J$ to $K$
(that is $K = U~J$).
\[
J~x = (U~J)~(T~x)
\]
\[
J = U \circ J \circ T
\]

Interesting things happen when we change the basis.
$T : E \to F$.

\section{Projective spaces (Why are we talking about this?)}

To grasp projective space, see real projective space.

Descartes's Cartesian coordinate system.

Klein's erlangen program.

Using algebra, we can describe a circle whose radius is $r$
and whose center is the origin in a 2-dimensional Euclidean space with Cartesian coordinate system
as $\{C_2~(x,y) ~|~ x^2+y^2 = r^2\}$.
We can find the intersection of two geometric objects
by solving the system of their equations.







% Synthetic differential geometry
% http://home.sandiego.edu/~shulman/papers/sdg-pizza-seminar.pdf
% Synthetic Differential Geometry: An application to Einstein’s Equivalence Principle
% http://www.math.ru.nl/~landsman/scriptieTim.pdf
% see the references in
% https://ncatlab.org/nlab/show/synthetic+differential+geometry
% https://mathoverflow.net/questions/186851/synthetic-vs-classical-differential-geometry

\section{Avoiding the tensor notation in Ricci calculus and the Einstein summation convention?}

% https://en.wikipedia.org/wiki/Ricci_calculus
We try not use the tensor notation in Ricci calculus,
of which the Einstein notation is a part.

Every text should be more readable than writable.

\section{Using equations to constrain a system, reducing its degree of freedom? Why are we talking about this?}

% https://en.m.wikipedia.org/wiki/Constraint_(classical_mechanics)

A \emph{constraint} is?
A constraint is modeled by an equation.
A system is modeled by a set of equations.
The system satisfies all equations in that set.

Example:
The position of an object moving in straight line is modeled by \(x(t) = k t\).

Constraint force is the cause of the constraint.

\subsection{Understanding a system's degree of freedom}

Bead sliding in a straight wire at an angle to gravitational field line.

Motivate degree of freedom with analytic geometry.
To describe a circle, we need two parameters c and r.

The degree of freedom of a system is the minimum number of parameters required to describe it.
Each parameter has type \(\Real\).

% https://en.wikipedia.org/wiki/Classical_central-force_problem
% https://en.wikipedia.org/wiki/Central_force

\section{Bilinear function? Why are we talking about this?}

A bilinear function \(f\) is left-linear and right-linear:
\(
f(x+z,y) = f(x,y) + f(z,y)
\)
and
\(
f(x,y+z) = f(x,y) + f(x,z)
\).

We can generalize that to multilinearity, but we won't.

* A linear function is multilinear.
* Otherwise a function is multilinear iff every partial application of it is also multilinear.

\section{Measuring the distance between two points on a sphere}

The distance from \(P\) to \(Q\) is \(\angle PCQ \cdot r\)

}
{
    \part{Basic physics}
    \chapter{Measurement}

\paragraph{Quantity, magnitude, and unit}

The quantity 1 kg has magnitude 1 and unit kg (kilogram).
The magnitude is a number.
The quantity 2 kg is twice 1 kg.
The quantity 10 kg is ten times 1 kg.

\section{SI units and prefixes}

SI stands for \emph{syst\`eme international d'unit\'es}
(international system of units).

The units are grouped into two kinds: base units and derived units.
Base units don't depend on other units.
Derived units depend on other units.
Example base units are kilogram (kg), meter (m), and second (s).
Example derived units are newton (N, which is \si{kg.m.s^{-2}})
and joule (J, which is \si{N.m}).

A prefix can be put before a unit to enlarge or shrink it.
Example prefixes are
\si{\micro} (micro, \(10^{-6}\)),
m (milli, \(10^{-3}\)),
k (kilo, \(10^3\)),
M (mega, \(10^6\)).
The quantities \SI{1}{kg}, \SI{1000}{g}, and \SI{1000000}{mg} are the same quantity.

Some units are rarely used.
An example rare unit is megagram (1 million grams).

For the complete list of SI units and prefixes, search the Internet%
\footnote{\url{https://en.wikipedia.org/wiki/International_System_of_Units\#Units_and_prefixes}}%
\footnote{\url{https://physics.nist.gov/cuu/Units/units.html}}%
.

\paragraph{A psychological effect}
Someone sounds heavier in grams than in kilograms.
For example, one friend of mine weighs one hundred \emph{thousand} grams.
We compare numbers more easily than we compare units.

\section{Writing numbers in scientific notation}

% https://en.wikipedia.org/wiki/Significant_figures
An example number in scientific notation is \( 1.23 \times 10^{50} \).
Without scientific notation, we would need to write out that number with its 47 trailing zeros.

The number \( 1.23 \times 10^{50} \) has three significant figures.

\section{Reporting a measurement and its uncertainty}

How do we measure quantities?
How do report a measurement?

Every measurement has an uncertainty.
Tools have limited precision.
We report a measurement by writing \SI{1.00(5)}{cm} or \SI[separate-uncertainty]{1.00+-0.05}{cm}
to mean that the actual quantity is somewhere between 0.95 cm and 1.05 cm.
We don't know the actual quantity.
We only know that it's between those.

\section{Sanity check with dimensional analysis}

Dimensional analysis can sanity-check a calculation.
If the unit is wrong, the calculation is wrong.
If the unit is right, the calculation may be right.

Dimensional analysis is a test.
It can detect falsehood.
It can't prove correctness.

}
\disabled{
    \part{Statics}
    \chapter{Weights and forces}

Pretend that the concept of \emph{mass} has not been invented.

\emph{Weight} is what a weight balance measures.

A weight balance has two arms.

Put a weight on an end of a weight balance.
Push the other end with your hand until the balance comes to rest.
When they reach equilibrium,
both of them exerts the same amount of \emph{force}.

\section{Law of the lever}

% https://en.wikipedia.org/wiki/Virtual_work#Law_of_the_lever
% https://en.wikipedia.org/wiki/Lever

\index{definitions!lever}%
\index{lever!definition}%
\index{simple machine!lever|see{lever}}%
A \emph{lever} has a fulcrum and two ends.

Let \(r_1\) be the distance between the first end to the fulcrum.

Let \(r_2\) be the distance between the second end to the fulcrum.

Let \(F_1\) be the weight placed at the first end.

Let \(F_2\) be the weight placed at the second end.

\index{Archimedes!law of the lever}%
\index{laws named after people!Archimedes's law of the lever}%
\index{laws!lever}%
\index{lever!law of the lever}%
\index{statics!Archimedes's law of the lever}%
\emph{Law of the lever}:
Such lever at equilibrium satisfies \(F_1 \cdot r_1 = F_2 \cdot r_2\).

We take this law as evident.
Doubt can be removed by a simple experiment.

Thus, a weight balance is a lever whose arms have equal length.

    \chapter{Linear elastostatics}

\section{Hooke's law of spring restoring force}
\label{sec:hooke-s-law}

Consider a spring at rest.
One end is fixed.
The other end is movable.

Let \(k\) be the spring's \emph{stiffness}.

Suppose that we pull the movable end,
displacing it by \(x\) from its resting position.

\index{laws named after people!Hooke's law of spring restoring force}%
\index{laws!spring restoring force}%
\index{Hooke!law of spring restoring force}%
\index{statics!Hooke's law of spring restoring force}%
\paragraph{Hooke's law}
For small displacements,
the spring's \emph{restoring force} at the movable end is \( F = - k x \).

It also works for rubbers and metals.%
\footnote{\url{http://www.continuummechanics.org/hookeslaw.html}}

\section{Seeing a rod as a stiff spring}

A solid rod behaves like a very stiff spring.

Let there be one such rod with length \(L\) and cross-section area \(A\).

Let one end be fixed.

If we push the other end,
that is if we apply a force \(F\) that is parallel
to the length of the rod (normal to the cross-section),
then we \emph{compress} the rod.
In this case, the \emph{stress} at the fixed end is \(F/A\).

The many definitions of \emph{strain}%
\footnote{\url{http://www.continuummechanics.org/strain.html}}

\footnote{\url{https://en.wikipedia.org/wiki/Saint-Venant\%27s_Principle}}

\footnote{\url{https://en.wikipedia.org/wiki/Linear_elasticity}}

\footnote{\url{https://en.wikipedia.org/wiki/Continuum_mechanics\#Car_traffic_as_an_introductory_example}}

\footnote{\url{https://en.wikipedia.org/wiki/Continuum_mechanics\#Major_areas}}

\section{Pressure, normal stress, shear stress, and stress}

Both pressure and stress are force per unit area.
But pressure is a scalar and stress is a tensor.%
\footnote{\url{https://physics.stackexchange.com/questions/107824/what-is-the-difference-between-stress-and-pressure}}

\section{Deformation}

A soft object is modeled by a subspace of a Euclidean space.
We call this subspace a \emph{material space}.

A \emph{deformation} assigns a vector to every point in the material space.

\section{Understanding the one-dimensional case}

\subsection{Understanding stress and strain}

Let \(\Delta x\) be the change in length.
Let \(x\) be the rest length.
The \emph{strain} is \( \Delta x / x \).

\subsection{Modeling the tension of a cord}

\subsection{Understanding Young's modulus (elastic modulus)}

Young's modulus (elastic modulus) is?

Relationship between length change and the exerted force.

Force that does not cause acceleration.

\subsection{Modeling continuous deformation using continuum mechanics}

Continuum mechanics?

How do we model stress in several dimensions?

\section{Understanding the three-dimensional case}

\subsection{Understanding Cauchy stress tensor}

The \emph{Cauchy stress tensor} is ...
Example: consider a cube:

% https://en.wikipedia.org/wiki/Stress_(mechanics)#General_stress
% https://en.wikipedia.org/wiki/Cauchy_stress_tensor

\section{Motivating tensor}

% https://en.wikipedia.org/wiki/Hooke%27s_law#General_tensor_form
Generalizing Hooke's law to continuous medium?

Generalizing Ohm's law to continuous medium?

    \chapter{Statics}

% https://en.wikipedia.org/wiki/Timeline_of_fundamental_physics_discoveries

\emph{Thermodynamics} began as a theory of steam engines.

\emph{Volume} is how much space something occupies.

\emph{Density} is weight per volume.

% https://en.wikipedia.org/wiki/Work_(physics)
\index{definitions!work}%
\index{work!definition}%
\emph{Work}: If one lifts a weight \(F\) so that its height increases by \(h\),
then he does a \emph{work} of \( W = F \cdot h \).
Coriolis defined this in 1826 \cite{coriolis1829calcul}
when steam engines lifted buckets of water out of flooded ore mines.
We shall generalize this definition later,
if the force and the displacement make an angle.

% https://en.wikipedia.org/wiki/Power_(physics)
\index{definitions!power}%
\index{power!definition}%
\emph{Power} is work done per unit time: \( P = W / t \).
This means that a steam engine with twice the power
will clean the same mine in half the time.

% https://hsm.stackexchange.com/questions/414/when-were-the-modern-notions-of-work-and-energy-created
% Helmholtz 1847?

\section{Archimedes's principle of buoyancy}

% https://en.wikipedia.org/wiki/Archimedes%27_principle
% https://en.wikipedia.org/wiki/On_Floating_Bodies

Put a solid into a container full of liquid.

The volume of the spilled part of the liquid is equal to
the volume of the submerged part of the solid.

\index{Archimedes!principle of buoyancy}%
\index{laws named after people!Archimedes's principle of buoyancy}%
\index{laws!buoyancy}%
\paragraph{Archimedes's principle of buoyancy}
Equal are the weight of the object and the buoyant force on the object.
(???)

\section{Pascal's law of fluid pressure transmission}

Blaise Pascal 1647

Pascal's law: Incompressible fluid spreads pressure evenly.

\index{Pascal!law of fluid pressure transmission}%
\index{laws named after people!Pascal's law of fluid pressure transmission}%
\index{laws!fluid pressure transmission}%
\index{statics!Pascal's law of fluid pressure transmission}%
\( P = \rho g h \)

\paragraph{Appreciating Pascal's barrel demonstration}

Counterintuitive: The hydrostatic pressure
does not depend on \emph{how much} fluid.
It depends on \emph{how deep}.
\footnote{\url{https://www.youtube.com/watch?v=EJHrr21UvY8}}

\section{Understanding the zeroth law of thermodynamics}

Put hot iron into cold water.
Eventually both become equally warm.

\index{laws!thermodynamics, zeroth}%
\emph{Zeroth law of thermodynamics}:
Heat never spontaneously flows from cold to hot.

\section{Unstructured content}

TODO Pendulum

\index{definitions!pendulum}%
\index{pendulum!definition}%
A pendulum is a bob hung on a string.

\emph{Conservation of mechanical energy}:
A released pendulum comes back to the same height.

TODO
Interplay between potential and kinetic energy:
Galileo's interrupted pendulum

TODO Vacuum

Boyle showed that objects of different masses fall with the same acceleration.

TODO Toricelli manometer

TODO von Guericke, Magdeburg

TODO Boyle

TODO Pascal

Boyle's experiments

\index{laws named after people!Lavoisier's law of conservation of mass}%
TODO Lavoisier's law of conservation of mass

\section{Understanding energy}

Conservation of energy

Kinetic energy

\emph{Kinetic energy} is \( \frac{1}{2} m |v|^2 \) which can also be written as \( |p|^2 / (2m) \).
This is explained by energy conservation and work by a constant force \(F\) that accelerates an initially resting mass.
\(F = ma\) and \(s = \frac{1}{2}at^2\) and \( W = Fs \) and \( v = at \) therefore \( W = E_k = \frac{1}{2} m(at)^2 = \frac{1}{2}mv^2 \).

\section{Understanding gases}

% https://en.wikipedia.org/wiki/Perfect_gas
% https://en.wikipedia.org/wiki/Gas#Historical_synthesis

A \emph{gas} is ...

\emph{Pressure} is measured by a manometer.

In statics, the \emph{volume} of a gas is the volume of its container.
Statics assumes that a gas fills its container evenly.

\emph{Temperature} is measured by a thermometer.
The unit of temperature is \emph{kelvin} (K).

% ?
Gas and piston at equilibrium:
Gas and a piston with weight \(F\).

\section{Using gas laws}

Let there be a container of gas with pressure \(P_1\) and volume \(V_1\).
Let this gas expand or shrink without changing its temperature
so that its pressure becomes \(P_2\) and its volume becomes \(V_2\).

\index{laws!gas pressure and volume}%
\index{laws named after people!Boyle's law of gas pressure and volume}%
\index{Boyle!Boyle's law of gas pressure and volume}%
\emph{Boyle's law}: \( P_1 V_1 = P_2 V_2 \).

Other gas laws

\emph{Charles's law}?
\emph{Dalton's law}?

% https://en.wikipedia.org/wiki/Dalton%27s_law
% https://en.wikipedia.org/wiki/Combined_gas_law
% https://en.wikipedia.org/wiki/Gay-Lussac%27s_law#Pressure-temperature_law
% https://en.wikipedia.org/wiki/Avogadro%27s_law

\index{laws!ideal gas}
\emph{Ideal gas law}: \( PV = nRT \).

Kinetic energy of one mole of gas is \( \frac{3}{2} RT \).

Statistical thermodynamics: kinetic theory of gases?

\section{Understanding Boltzmann's constant}

% https://en.wikipedia.org/wiki/Boltzmann_constant
\emph{Boltzmann's constant} relates the average kinetic energy of particles in a gas and the temperature of the gas?

% https://en.wikipedia.org/wiki/Gas_constant
The \emph{gas constant} (molar gas constant, universal gas constant, ideal gas constant)?

\section{Understanding Avogadro's number}

\emph{Avogadro's number} is?

Terms?

System and environment

Thermodynamic equilibrium

\section{Understanding heat}

Heat capacity

\emph{Black's principle}:
When two liquids are mixed, the heat released by one equals the heat absorbed by the other.
???

???
If \(m_1\) amount of water at temperature \(T_1\) is mixed with \(m_2\) amount of water at temperature \(T_2\),
then the result, after equilibrium, is \(m_1+m_2\) amount of water at temperature \(\frac{m_1 T_1 + m_2 T_2}{m_1+m_2}\).

Specific heat

Latent heat

\section{Understanding thermodynamic process and cycle?}

Isobaric?
Isochoric?
Adiabatic?
Expansion of gas?
Work done by a gas?

Carnot engine?

Thermodynamic efficiency?

\section{Understanding the laws of thermodynamics}

% https://en.wikipedia.org/wiki/Laws_of_thermodynamics
% https://en.wikipedia.org/wiki/History_of_entropy

\section{Working with simple machines}

% https://en.wikipedia.org/wiki/Simple_machine

\UnorderedList{
\item Lever
\item Wheel and axle
\item{Pulley}
\item{Tilted plane}
\item{Wedge}
\item{Screw}
}

TODO:
Modern machine theory: Kinematic chains

\section{On ignorance}

In the 18th century, occasionally, steam boilers and coal mines exploded, killing tens of people.

Then nuclear power plants exploded.

What if a Dyson sphere exploded...

% Chemistry
% Thermostatics
% Heat and fluids

}
\disabled{
    \part{Other}
    \chapter{Relativity}

\paragraph{Needed before reading this chapter}
Make sure that you have understood Chapter~\ref{chp:geometry}~(p.~\pageref{chp:geometry}).

\UnorderedList{
\item We see every moving clock ticking slower than our clock.
}

When reality and theory disagree, reality always wins.
We celebrate every time we falsify a theory.
It means that our knowledge is building.

To talk about relativity, we need \emph{analytic geometry},
which allows us to use numbers to describe locations, shapes, and other geometrical concepts.

After reading this chapter, you should be able to:
\UnorderedList{
\item craft a coordinate system
\item use vector to describe relative positions/displacements
\item calculate the time dilation, length contraction, momentum, and kinetic energy of an object moving in spacetime
\item describe the curvature of spacetime due to a point mass
\item visualize the Schwarzschild metric
}

\section{Calculating the Lorentz boost}

\index{Lorentz transformation}%
\index{Lorentz boost}%
Let \(c\) be the speed of light.
The \emph{Lorentz boost} of an object moving with constant speed \(v\) is \(\gamma(v)\) where
\Formula{
    \gamma(v) = \frac{1}{\sqrt{1 - \frac{v^2}{c^2}}}.
}
What does it mean?\footnote{\url{https://en.wikipedia.org/wiki/Lorentz_transformation\#Physical_implications}}

If you are interested in the derivation of the Lorentz transformation,
see \S\ref{sec:derive-lorentz-transform} (p. \pageref{sec:derive-lorentz-transform}) \emph{later}.
Don't do it now.

% https://en.wikipedia.org/wiki/Lorentz_transformation#boost
\ExerciseAnswer{Compute the slowing down of a clock that is moving at half the speed of light.
How many tick does the moving clock make for each tick that a stationary clock makes?
What if the moving clock's speed is \(c/4\)?
What if the moving clock's speed is \SI{100}{km/h}?
Hint: A handy conversion factor to remember is \(\SI{18}{km/h} = \SI{5}{m/s}\).
}{TODO}

\ExerciseAnswer{Exercise? Which clock is slower?
\(A\) is resting and \(B\) is moving.
\(A\) sees that \(B\)'s clock ticks slower.
But from \(B\)'s point of view,
\(A\) is moving and \(B\) is resting,
so \(B\) sees that \(A\)'s clock ticks slower.
Thus, which clock ticks slower?
}{Is this question even valid?}

\ShowAnswers

Time dilation: the faster a clock moves, the slower it ticks.

% this has a point?
% http://alternativephysics.org/book/index.htm

% Disproved by Schutz:
% Why Time Dilation must be Impossible:
% because it's impossible to tell which is moving faster than which,
% because the frame of reference itself could be moving.
% http://alternativephysics.org/book/TimeDilation.htm

% https://physics.stackexchange.com/questions/18867/how-does-time-dilation-work-without-a-privileged-reference-frame
% https://www.reddit.com/r/askscience/comments/1oej1u/how_do_we_pick_the_right_frame_of_reference_for/
% https://en.wikipedia.org/wiki/Twin_paradox#Resolution_of_the_paradox_in_special_relativity

\emph{Minkowski spacetime}?

\emph{World line} is ...

Proper time?

\subsection{Calculating time dilation}

\subsection{Calculating length contraction}

\subsection{Calculating the vector Lorentz transformation}

\section{Understanding rotation in Minkowski space}

\section{???}

\section{Understanding the stress-energy tensor}

\section{???}

\section{Understanding Einstein field equations}

\paragraph{Key idea}
Mass bends spacetime.

As energy at a point increases, spacetime bends more there.

% https://en.wikiversity.org/wiki/General_relativity
% https://en.wikipedia.org/wiki/Riemann_curvature_tensor
% Notation
% https://en.wikipedia.org/wiki/Ricci_calculus

A \emph{tensor} is ...

% FIXME copied from wikipedia
The tensor form of the Einstein field equations is%
\footnote{\url{https://en.wikipedia.org/wiki/Einstein_field_equations}}
\Formula{
    R_{ij} - \frac{1}{2} R g_{ij} + \Lambda g_{ij} = \frac{8\pi G}{c^4} T_{ij}
}
where \(R_{ij}\) is the Ricci curvature tensor, \(R\) is the scalar curvature, \(g_{ij}\) is the metric tensor,
\(\Lambda\) is the cosmological constant, \(G\) is Newton's gravitational constant, \(c\) is the speed of light in vacuum,
and \(T_{ij}\) is the stress–energy tensor.

\paragraph{Exercise}
Describe the bending of space due to a point mass.

\subsection{Calculating black hole event horizon radius?}

% https://en.wikipedia.org/wiki/John_Michell_(natural_philosopher)#Black_holes

In 1783, John Michell wrote about \emph{dark stars},
an object so heavy that light can't escape it?

\section{Calculating gravitational lensing}

% \section{Hawking radiation}

% \section{Black hole thermodynamics}

\section{Deriving Einstein field equations}

(To be done.)

\section{Relativity}

\subsection{Some basic terms}

% Read Landau & Lifshitz relativity volume?

A \emph{spacetime} \(M\) is a four-dimensional space.

An \emph{\(M\)-event} is a point in \(M\).

An \emph{\(M\)-frame} is an invertible function \(\Real^4 \to M\).
Such frame assigns a unique coordinate tuple to every event.

Let \(A : \Real^4 \to M\) be an \(M\)-frame.

Let \(B : \Real^4 \to M\) be an \(M\)-frame.

Let \(E : M\) be an \(M\)-event.

Let \(t \in \Real\).

Let \(x \in \Real^3\).

The notation \(E = A(t,x)\) means that
the event \(E\) \emph{happens} at \emph{\(A\)-time} \(t\) and \emph{\(A\)-position} \(x\).
It means that, \emph{from \(A\)'s point of view},
the event happens at time \(t\) and position \(x\).
We also say that such \(t\) is the \emph{\(A\)-time of \(E\)}
and such \(x\) is the \emph{\(A\)-position of \(E\)}.

We can write unqualified \emph{event} and \emph{frame} to mean \emph{\(M\)-event} and \emph{\(M\)-frame}
if it's clear from context that the spacetime being discussed is \(M\).
Similarly, we can write unqualified \emph{time} and \emph{position} to mean \emph{\(A\)-time} and \emph{\(A\)-position}
if it's clear from context that the frame being discussed is \(A\).

\emph{Observer} is another word for \emph{frame}.

\emph{Time} is measured by a \emph{clock}.

Time orders (sequences) events.

The \emph{position} of an object is where it is in a frame.

\emph{Distance} is measured by a \emph{ruler}.

% https://en.wikipedia.org/wiki/Inertial_frame_of_reference

\subsection{Motion and trajectory}

How do we describe motion?

Let \(A\) be an \(M\)-frame.

Let \(B\) be an \(M\)-frame.

Let \(P\) be a moving object with negligible size.

We describe \emph{A's account} of the motion of \(P\) as
\enquote{The \emph{\(A\)-position} of \(P\) at \emph{\(A\)-time} \(t\) is \(x(t)\).}

% TODO clear up

The \emph{\(M\)-trajectory} of the object \(P\)
is the set \( \{ A(t,x(t)) ~|~ t \in \Real \} \subseteq M \).
It means that at \(A\)-time \(t\), the \(A\)-position of \(P\) is \(x(t)\).
The \emph{\(A\)-velocity} of \(P\) is the derivative of that \(x\).
For example, if \(x(t) = vt + x_0\),
then \(A\) sees that \(P\) is moving with constant velocity \(v\).
The function \(x\) is \(A\)'s description of \(P\)'s motion.
Thus, velocity is also relative, because time and position are relative,
and velocity is defined in terms of time and position.

\index{definitions!stationary}%
\index{stationary!definition}%
An object is \emph{\(A\)-stationary} iff its \(A\)-velocity is constantly zero.

\(A\) and \(B\) may disagree about the motion of \(P\), but \(P\)'s \(M\)-trajectory is the same:
\( \{ A(t,x(t)) ~|~ t \in \Real \} = \{ B(u,y(u)) ~|~ u \in \Real \} \).
The problem is to state \(u\) and \(y\) in terms of \(t\) and \(x\).
It can be thought that the trajectory is the unknowable objective existence of the object?

\emph{\(M\)-world-line} is synonym of \emph{\(M\)-trajectory}.

\section{Deciding whether two events are simultaneous}

What happens at the same time according to an observer?

Two \(M\)-events are \emph{\(A\)-simultaneous} iff they have the same \(A\)-time.
Formally, $A(t,x)$ and $A(u,y)$ are \emph{\(A\)-simultaneous} iff $t = u$.

The set of all events that happen at \(A\)-time \(t\) is \(S_A(t) = \{ A(t,x) ~|~ x \in \Real^3 \}\).

The simultaneity relationship forms an equivalence class?
Isochrone?

Let the \(A\)-velocity of \(B\) be constant \(b\).

From \(A\)'s point of view, at time \(t\),
the light has traveled a distance \(ct\).
Thus, \(S_A(t) = \{ (t,x) ~|~ \norm{x} = ct \}\).
\(X_A(t) = \{ x ~|~ \norm{x} = ct \}\).

From \(B\)'s point of view, at time \(t\),
the light has traveled a distance \(ct\),
but every point in space has moved by \(-vt\).
Thus, the set of all \(B\)-simultaneous events at \(B\)-time \(t\) is
\( S_B(t) = \{ (t,x) ~|~ \norm{x - vt} = ct \} \).
\(X_B(t) = \{ x ~|~ \norm{x - vt} = ct \}\).

Space homogeneity principle:
\(d(x+h,y+h) = d(x,y)\).
translating a rod preserves its length.
The distance between two points depends only on the relative position of the points.

Time homogeneity principle is similar to space homogeneity principle.

Plot \((t,x)\) satisfying \(\norm{x-vt} = ct\).
In two-dimensional space, it is a pair of lines.
In three-dimensional space, it is a cone.
\begin{align*}
    \norm{x-vt}^2 &= \sum_{k \in \{1,2,3\}} (x_k - v_kt)^2
\end{align*}

From \(B\)'s point of view, at time \(t\),
after duration \(dt\),
the light has traveled a distance \(c~dt\),
but every point in space has moved by \(-v(t)~dt\).
Thus, the set of all \(B\)-simultaneous events at \(B\)-time \(t + dt\) is
\( S_B(t + dt) = \{ (t,x) ~|~ \norm{x - v(t)~dt} = c~(t + dt) \} \).

How do we measure the distance to somewhere?
We fire a light at it.
The time light takes to get there is proportional to the distance.

A stationary observer \(A\) fires light everywhere.
According to \(A\), everything in \(\{ x ~|~ \norm{x} = ct \}\)
happens at the same \(A\)-time \(t\).

Two events are \emph{\(A\)-simultaneous} iff their \(A\)-times match.

TODO
What does `synchronized' mean?
What does it mean that two clocks are synchronized?

% from TED history of the world: Matter is congealed energy.

The water waves we see propagate in water,
the sound waves we hear propagate in air,
so it's tempting to guess that the light (the optical waves) we see propagate in
\emph{luminiferous aether} (``fresh air that carries light''), who coined that name?
The wave in moving water?
The \emph{Michelson\textendash{}Morley aether interferometry experiment} tried to measure
the velocity of the Earth with respect to the aether,
but they found no significant differences.
Who threw away the aether idea?
Light does not need a medium to propagate.

\subsection{Inertial frame}

What is a \emph{moving frame}?
How do we formalize it?
Given what \(A\) sees, how do we compute what \(B\) sees?

Let \(A\) and \(B\) agree on the origin: \(A(0,0) = B(0,0)\).

Let the \(A\)-velocity of \(B\) be constant \(b\).

Let the \emph{target} \(T\) be an \(A\)-stationary object at \(A\)-position \(x\).

Let \(E = A(0,0) = B(0,0)\) be the event \enquote{the light fired from the origin}.

Let \(F = A(t,x) = B(t',x')\) be the event \enquote{the light arrived at the target}.

\(A(0,x-ct) = A(0,0)\)

\(B(0,x'-c't') = B(0,0)\)

Let \(c\) be the \(A\)-velocity of the light fired from \(E\) to \(F\).

Thus \(x = ct\). Thus \(c\) and \(x\) have the same direction.

Then \(A(0,x) = B(0,x')\) is the initial event of \(T\).

Then \(F = A(t,x) = B(t',x' - vt')\).

Then \(F = A(t,ct) = B(t',x' - vt')\).

Every \emph{object} has a position in space.

\emph{Law of the velocity of two observers}:
If \(A\) sees \(B\) moving with velocity \(v\),
then \(B\) sees \(A\) moving with velocity \(-v\).
(The \(A\)-velocity of \(B\) is the negation of the \(B\)-velocity of \(A\).)

How do we measure distance?

To measure distance, we use light.

Let \(P\) be an \(A\)-position.

Let \(Q\) be an \(A\)-position.

In frame \(A\),
if light traverses \(PQ\) in duration \(t\),
then the \(A\)-length of \(PQ\) is \(d_A(P,Q) = c t\).

If light takes the same \(A\)-duration to traverse \(OA\) and \(OB\),
then \(OA\) and \(OB\) have equal \(A\)-length: \(d(O,A) = d(O,B)\).

Define \(S(O,q)\) as the set of every point whose distance from \(O\) is \(q\).
Formally, \(S(O,q) = \{ P ~|~ d(O,P) = q \}\).

Law: \(v_{AB} = -v_{BA}\).

To derive Lorentz transformation, we use three observers \(A,B,C\)
where \(A\) sees \(B\) moving but \(A\) sees \(C\) stationary.

An \emph{event} is a point \((t,x)\) in spacetime \(\Real \times \Real^3\).
Let \(A\) and \(B\) agree at \((0,0)\).
An observer is a frame is a chart is a coordinate system.
Let \(C_A(e)\) be the \(A\)-coordinates of \(e\).

Let \(E_A(t)\) be the set of all events that happen in \(A\)-time \(t\).
Then it is \(E_A(t) = \{ e ~|~ \exists x ~ C_A(e) = (t,x) \}\).

Let the position of \(M\) at \(O\)-time \(t\) be \(x(t)\).
If \(O\) emits light, and object \(M\) is moving with velocity \(v\),
then the light and \(M\) will meet at \(d(O,x_\text{meet})\)
where there exists \(t\) such that \(x(t) = x_\text{meet}\).
It is the intersection of a sphere and a line.

The distance takes two objects, not two points.
Let there be objects \(S, M, T\) (stationary, moving, and target).
According to \(S\), \(M\) is moving with velocity \(v\), and \(T\) is stationary.
Suppose that when \(S\) and \(M\) both coincided at \((t_0,x_0)\),
both of them fired a light at \(T\) (target).
\(S\) sees that \(d(S,T) = d(x_0,x_T)\).
\(M\) sees that \(d(M,T) = d(x_0,x_T - vu)\).

\subsection{Old text}

All lights that originate at \(A(t,x)\) reach \(A\)-position 0 simultaneously at \(A\)-time \(t + x/c\).
The set of all $\{ A(t,x) ~|~ \norm{x} = ct \}$ is the \(A\)-simultaneity surface at \(A\)-time $t$.
Two events in the same \(A\)-simultaneity surface happens at the same \(A\)-time.

Let O's firing light in all directions happens at $A=O(0,0)$
and the detection of that light by Q happens at $B = O(\vec{x},t)$.
If $\vec{x}$ is always the O-position of Q,
then
\[
t = x/c.
\]

Now suppose that $Q$ has constant O-velocity $\vec{v}$,
and the O-position of $Q$ at O-time 0 is $\vec{x}$.
Q's detection of light now happens at $B'=O(\vec{x}',t')$ instead, where
\begin{align*}
\vec{x}' = \vec{x} + \vec{v} t'
\\ c t' = \norm{\vec{x}'}
\end{align*}
due to the light speed constancy principle.

Suppose $\vec{v}$ and $\vec{x}$ are perpendicular to each other.
\begin{align*}
(x')^2 = x^2 + (vt')^2 = (ct')^2
\\ x^2 = (c^2-v^2)(t')^2
\\ t' = \frac{x}{\sqrt{c^2-v^2}}
\\ t' = \frac{x}{c \sqrt{1-(v/c)^2}} = \alpha x/c = \alpha t
\\ x' = \alpha x
\end{align*}

To help visualize this, we can use the spacetime diagram.
The x-axis is the distance from O-position zero.
The y-axis is the interval from O-time zero.

Suppose that they weren't perpendicular.
\begin{align*}
\vec{x}' = \vec{x} + \vec{v} t'
\\ c t' = x'
\\ x_k' = x_k + v_k t'
\\ c^2 (t')^2 = (x')^2
\\ = \norm{\vec{x} + \vec{v} t'}^2
\\ = x^2 + 2 t' \vec{x} \cdot \vec{v} + v^2(t')^2
\\ = x^2 + 2 t' \vec{x} \cdot \vec{v} + (v^2 - c^2)(t')^2
\\ t' = \frac{- 2 \vec{x} \cdot \vec{v} \pm \sqrt{(2 \vec{x}\cdot\vec{v})^2 - 4 (v^2 - c^2) x^2}}{2 (v^2 - c^2)}
\\ t' = \frac{\vec{x}\cdot\vec{v} + \sqrt{(\vec{x}\cdot\vec{v})^2 + (c^2 - v^2) x^2}}{c^2 - v^2}
\\ t' = \frac{\vec{x}/c\cdot\vec{v}/c + \sqrt{(\vec{x}/c\cdot\vec{v}/c)^2 + (1 - (v/c)^2) (x/c)^2}}{1 - (v/c)^2}
\\ t' = \frac{(x/c)(v/c) \cos \theta + \sqrt{((x/c)(v/c)\cos\theta)^2 + (1 - (v/c)^2) (x/c)^2}}{1 - (v/c)^2}
\\ t' = \frac{t(v/c) \cos \theta + \sqrt{(t(v/c)\cos\theta)^2 + (1 - (v/c)^2) t^2}}{1 - (v/c)^2}
\end{align*}

A light fired at $O(\vec{x},t)$ can be at any $O(\vec{x}+\vec{dx},t+dt)$
where $dx = c~dt$.

A light that O detects at $O(\vec{x},t)$ could come from any $O(\vec{x}-\vec{dx},t-dt)$ where
$dx = c~dt$.

Let M has constant O-velocity $\vec{v}$.
Let M pass $O(0,0)$.

A light that M detected at $O(\vec{v}t,t)$ could come from any $O(\vec{v}t-\vec{dx},t-dt)$ where $dx = c~dt$.
Therefore O observes that M must detect all such lights at the same M-time that corresponds to O-time $t$.
Therefore O observes that all such things must be M-simultaneous.

A light that O detected at $O(0,t)$ and M detected at $O(\vec{v}\tau,\tau)$
could come from any $O(-\vec{dx},t-dt)$ and $O(\vec{v}\tau-\vec{d\xi},d\tau)$ where $dx = c~dt$ and $d\xi = c~d\tau$.

A light that O detected at $O(0,0)$ and M detected at $O(0,\tau)$
could come from any $O(-\vec{dx},t-dt)$ and $O(\vec{v}\tau-\vec{d\xi},d\tau)$ where $dx = c~dt$ and $d\xi = c~d\tau$.

A light fired from $O(\vec{x},t)$ will be detected by O at $O(0,t)$ and by M at $O(\vec{v}\tau, \tau)$
where
\[
x = ct
\\
\norm{\vec{x} - \vec{v}\tau} = c\tau 
\]

A light that M detected at $M(0,\tau)$ could come from any $M(\vec{d\xi},\tau-d\tau)$ where $d\xi = c~d\tau$.
A light that O detected at $M(-\vec{v}\tau,\tau)$ could come from any $M(-\vec{v}\tau+\vec{d\xi},\tau-d\tau)$ where $d\xi = c~d\tau$.

\[
O(0,0) = M(0,0)
\\ O(\vec{v}t,t) = M(0,\tau)
\\ M(-\vec{v}\tau,\tau) = O(0,t)
\]

This is the same case where everything else has constant M-velocity $-\vec{v}$.

A light fired at $M(\vec{x},t)$ can be at any $M(\vec{x}+\vec{dx},t+dt)$ where
\[
dx = c~dt
\]
according to light speed constancy principle.

However, $M(\vec{x},t)$ and $M(\vec{x}+\vec{v}u,t+u)$ are the same point in space for all $u$.
If, at $M(0,t+u)$, M detects a light fired from $M(\vec{x},t)$ (thus $u=x/c$),
then the firer actually was $M(\vec{x}+\vec{v}u,t)$.

Let $O(0,0) = M(0,0)$.
Let $O(\vec{v}t,t) = M(0,\tau)$.

Suppose that $O$ observes $M$ to be moving according to $\vec{p}$
where $\vec{p}(t) = \vec{a} + \vec{v}t$ is the O-position of $M$ at O-time $t$.
A light that $M$ detects at $O(\vec{p}(t),t)$ could come from any $O(\vec{p}(t)-\vec{dx},t-dt)$
where $dx = c~dt$.
A light from $O(\vec{x},t)$ will be detected by O at $O(0,t+x/c)$
and will be detected by M at $O(\vec{a}+\vec{v}(t+u)-\vec{x},t+u)$
where $\norm{\vec{a}+\vec{v}(t+u)-\vec{x}} = cu$.
If $\vec{x}=0$ and $t=0$, then it will be detected by M at $O(\vec{a}+\vec{v}u,u)$ where $\norm{\vec{a}+\vec{v}u}=cu$.

A light from $O(0,t)$ will be detected by M at $O(\vec{a}+\vec{v}u,u)$ where $\norm{\vec{a}+\vec{v}u} = c(u-t)$
A light from $O(\vec{x},0)$ will be detected by M at $O(\vec{a}+\vec{v}t-\vec{x},t)$ where $\norm{\vec{a}+\vec{v}t-\vec{x}} = ct$

\subsection{Frame transformation}

We \emph{describe frame transformation}.
We can write the equation \(A(t,x) = B(t',x')\) as \(A(t,x) = B(T(t,x), X(t,x))\)
with unknown functions \(T\) and \(X\).
We are trying to state \(t'\) in terms of \(t\) and \(x\).
We write \(A(t,x) = B(t',x')\) to mean that the \(A\)-coordinates \((t,x)\)
and the \(B\)-coordinates \((t',x')\) refer to the same \(M\)-event.
The same \(M\)-event can have different \(A\)-coordinates and \(B\)-coordinates.

\subsection{(being written)}

What is the \(A\)-distance traveled in \(A\)-duration \(dt\) by an object moving with constant \(A\)-velocity \(v\)?
It is \(\norm{v ~ dt} = \norm{v} ~ dt\).

How do we formalize the constancy of the speed of light?

\section{Special relativity}

Special relativity can be derived from Galileo's principle of relativity
and Einstein's principle of the universality of the speed of light.
\cite[p.~1]{schutz2009first}

\index{laws!relativity}%
\index{laws named after people!Galileo's principle of relativity}%
\emph{Galileo's principle of relativity}: No experiment can measure the absolute velocity of an observer.
There is no preferred inertial frame.
There is no special inertial frame.
There is no absolute motion.

\index{laws!universality of the speed of light}%
\index{laws named after people!Einstein's principle of the universality of the speed of light}%
\emph{Einstein's principle of the universality of the speed of light}:
The speed of light is the same for all inertial (non-accelerating) observers.
The origin of this postulate is \enquote{the law of electromagnetics is the same at all reference frames}
or \enquote{the law of physics is the same in all reference frames}.
Everyone sees the same Maxwell's equations.
Everyone measures the same speed of light,
regardless of how fast they are moving.
Did Einstein know the Michelson\textendash{}Morley experiment?

So what?
What are the consequences?

Lorentz transformation can be derived from the principle of the universality of the speed of light,
with the aid of a spacetime diagram?
What is a spacetime diagram?
Minkowski diagram.
% https://en.wikipedia.org/wiki/Minkowski_diagram
Hermann Minkowski introduced in 1908.

\section{Ramble}

Spacetime metric

% http://en.wikibooks.org/wiki/Special_Relativity/Simultaneity,_time_dilation_and_length_contraction

% http://en.wikisource.org/wiki/On_the_Electrodynamics_of_Moving_Bodies_(1920_edition)

Let $O$ be the type of frames and $M$ be the type of motions.
Define $m~x~y$ as the motion of $y$ as observed by $x$:
\[
m : O \to O \to M.
\]
Define $a~x~y$ as adding the motion $x$ to the frame $y$:
\[
a : M \to O \to O.
\]
A motion has spacetime distance elapsed as measured by a frame:
\[
s : O \to M \to S.
\]

A frame always looks stationary to itself.
\[
m~x~x = 0
\]
where $0$ is the null motion.

Inverse motion:
\[
m~x~y = -m~y~x
\]

$m$ looks like subtraction.

Transformation of frames.
\begin{align*}
m~(T~x)~y &= T^{-1}~(m~x~y)
\\ m~x~(T~y) &= T~(m~x~y)
\end{align*}

If the same transform is applied to both frames:
\[
m~x~y = m~(T~x)~(T~y)
\]

The order of transformation does not matter:
\begin{align*}
m~(T~x)~(U~y) &= T^{-1}~(U~(m~x~y)) = U~(T^{-1}~(m~x~y))
\end{align*}

Frame seeing another frame that sees yet another frame.
\[
m~x~z = m~x~y + m~y~z
\]
where $+$ is motion addition.

We see that motions form a monoid.

This motion is straightforward to imagine: $x~t = 3 t$.

A motion seen from another reference frame.

The space picture:
We can think of motion as a path in spacetime.
\[
\begin{bmatrix}t\\x\end{bmatrix}
\]

Force-view of pendulum.

\begin{align*}
F_x &= 0
\\
F_y &= -mg
\end{align*}

It is easier if we change the coordinate system.
We look from the bob's point of view
so that $(0,0)$ is always the bob
and $(0,1)$ is always the direction from the bob to the pivot.

$\theta=0$ represents negative y-axis.
Positive increment is counterclockwise.

\begin{align*}
F_{x'} &= -m g \sin \theta
\\ a_{x'} &= -g \sin \theta
\\ \frac{\partial^2 \theta}{\partial t^2} &= -g \sin \theta
\end{align*}

Field-view of pendulum.

\begin{align*}
P &= m g h
\\
K &= \frac{1}{2} m v^2
\end{align*}

We can describe a manifold by its tangent space.

% https://physics.stackexchange.com/questions/305857/lorentz-transformation-without-constant-speed-of-light-in-vacuum-reasonable/305861#305861

% Cite this paper: Nothing but Relativity
% https://arxiv.org/pdf/physics/0302045v1.pdf
% This paper is already cited by the above paper
% "One more derivation of the Lorentz transformation"
% https://fenix.tecnico.ulisboa.pt/downloadFile/3779571248372/Levy-Leblond_(76).pdf

% ramble

OpenStax University Physics Volume 3 \cite{openstaxphysics3}

TODO
Should we move this chapter earlier to introduce coordinates?
But this chapter should come after electromagnetism?

TODO
How do we describe the bending of space?
How do we describe an almost-Euclidean curved space?
The curvature of space?
By the metric?

(Rewrite: simultaneity hypersurface)
An event is a point in spacetime.
Let \(A,B,O\) be spatial points.
Two points \(A\) and \(B\) are \emph{\(O\)-simultaneous} iff a light fired from \(O\) reaches \(A\) and \(B\) at the same \(O\)-time.
If we are stationary, then our simultaneity hypersurface is a hypersphere.

A world line is a one-dimensional manifold in \(M\).

If \(A\) measures that a light has moved by spatial distance \(dx\), then \(A\)'s clock will have advanced by \(dt = dx / c\).

If \(A\) emits a light, and waits until a time interval of \(dt\) elapsed in his clock, then he will measure that the light has traveled a distance of \(c~dt\).

If \(A\)'s clock measures that a time interval of \(dt\) has elapsed, then the light will have moved by spatial distance \(dx = c ~ dt\).

%https://en.wikipedia.org/wiki/Postulates_of_special_relativity#Mathematical_formulation_of_the_postulates
Pseudo-Riemannian manifold
%https://en.wikipedia.org/wiki/Relativity_of_simultaneity#Accelerated_observers
radar-time/distance definition

The distance of an event \(A\) from another event \(B\) is the time required by light to reach \(B\) from \(A\).

From the point of view of the moving observer, he is stationary, and it is the world (everything else) that is moving towards him with velocity \(-v\). If he fire a ball of light, then after he measures time \(t\), then...

%https://en.wikipedia.org/wiki/Bondi_k-calculus
%https://en.wikipedia.org/wiki/Rapidity

%https://physics.stackexchange.com/questions/305857/lorentz-transformation-without-constant-speed-of-light-in-vacuum-reasonable/305861#305861

\section{Spacetime diagram and world-lines}

What is the coordinate system of an observer moving with constant velocity \(v\)?

A spacetime diagram represents the worldview of an observer?
Consider a spacetime diagram with one time dimension and one space dimension.
A point \((t,x)\) represents an event that occurs at time \(t\) and position \(x\) according to \(O\).
The horizontal is space coordinate \(x\).
The vertical is time coordinate \(t\).
Each event has a spacetime coordinate \((t,x,y,z)\).

Let \(A\) be at 0.
A light coming from a distance \(x\) from \(A\) will reach \(A\) after \(x/c\) time.

If \(A\) fires a light, let the light \(A\)-velocity be \(c\)
(a vector whose magnitude is the speed of light),
then after \(A\)-time-interval \(dt\),
the light will have traveled an \(A\)-distance of \(\norm{c ~ dt}\).
But \(B\) sees \(A\)-velocity \(c'\) and the light has traveled an \(A\)-distance of \(\norm{c' ~ dt - v ~ dt}\) from \(B\).

Speed is distance divided by time.

We can measure distance and time:
fire a light, and wait for its reflection to come back.
If \(t\) time elapsed until we measure the reflection, then the round-trip distance traveled by the light is \(ct\).

\emph{The same point in spacetime is assigned different coordinates.}

How do we know that two different coordinate tuples refer to the same point in spacetime?

Let \(M\) be the event space.
An event is a 4-tuple \((t,x,y,z) \in \Real^4\).
A coordinate system is a bijective mapping \(\Real^4 \to M\).
Let \(C_A\) and \(C_B\) be coordinate systems.
The same event \(e = C_A^{-1}(p_A) = C_B^{-1}(p_B)\).
How can we know that \(p_A\) and \(p_B\) map to the same event?

If \(A\) measures a constant-velocity object at \(C_A^{-1}(t_1,x_1)\) and \(C_A^{-1}(t_2,x_2)\),
then the velocity of the object is \(C_A^{-1}\left(\frac{x_2-x_1}{t_2-t_1}\right)\).

Two observers \(B\) and \(C\) move with the same \(A\)-velocity \(v\).
\(B\) fires a light to \(C\).

Let \(e_k\) be \(k\)'s coordinate system.
Thus \(e_A p = e_B q\) means that \(p\) and \(q\) happen at the same point in spacetime.

An observer is a coordinate system?

The \emph{\(A\)-world-line} of an object \(P\) moving with constant \(A\)-velocity \(v\)
is \( wlPA = \{ (t, x_0 + v t) ~|~ t \in \Real \} \)
where \(x_0\) is where the object is at \(A\)-time 0.
That line is the trajectory of \(P\) in \(A\)'s coordinate system.
Iff \(P\) is a light, then \(\norm{v} = c\).

TODO How do we add velocities in special relativity?

\section{What?}

An observer brings a clock and a light detector with him.
He determines the location
of another object by measuring how long his clock elapses
between his firing a light and his detecting its reflection.
If he fires a light with direction $\hat{x}$
when his clock says $t$ and
detects its reflection with direction $-\hat{x}$ when his clock says $u$,
then he infers that his light hit the object when his clock said $(t+u)/2$,
and he infers that the object was at $\hat{x} c(u-t)/2$ when his light hit the object;
he assumes that the speed of light is constant.

An observer measures the velocity of another object by firing two lights:
the first with direction $\hat{x}$ when his clock says $t$,
and the second with direction $D\hat{x}$ ($\hat{x}$ rotated a little) when his clock says $t + dt$.
If he detects the first reflection when his clock says $u$
and he detects the second reflection when his clock says $u + du$,
then he infers that his first light hits the object at $\hat{x} c(u-t)/2$ when his clock says $(t+u)/2$
and his second light hits the object at $D\hat{x} c(u-t+du-dt)/2$ when his clock says $(t+dt+u+du)/2$,
and therefore he infers that the object has displaced by
$D\hat{x} c(u-t+du-dt)/2 - \hat{x} c(u-t)/2$
while his clock has elapsed by $(dt+du)/2$.
We rearrange the displacement:
\begin{align*}
& D\hat{x} c(u-t+du-dt)/2 - \hat{x} c(u-t)/2
\\ &= \frac{c\hat{x}}{2} ((u-t+du-dt)D - (u-t)I)
\\ &= \frac{c\hat{x}}{2} ((u-t)(D-I) + (du-dt)D)
\end{align*}
therefore he measures that the object's average velocity between $t$ and $t+dt$ was:
\begin{align*}
&\frac{\frac{c\hat{x}}{2} ((u-t)(D-I) + (du-dt)D)}{(dt+du)/2}
\\ &= \frac{c\hat{x} ((u-t)(D-I) + (du-dt)D)}{dt+du}
\end{align*}
which becomes this if $D=I$ ($\vec{v} \cdot \hat{x} = 0$):
\begin{align*}
&\frac{\frac{c\hat{x}}{2} ((u-t)(D-I) + (du-dt)D)}{(dt+du)/2}
\\ \vec{v} &= \frac{du-dt}{du+dt} c \hat{x}
\\ -v/c &= \frac{du-dt}{du+dt}
\\ (du+dt) v/c &= dt-du
\\ (1+v/c)du &= (1-v/c)dt
\\ \frac{du}{dt} &= \frac{1-v/c}{1+v/c} = \frac{c-v}{c+v}
\end{align*}

When he detects a light when his clock says $t$,
he knows the direction where the light is going ($\hat{x}$),
and thus he knows the direction the light came from ($-\hat{x}$),
but he does not know where exactly in his spacetime whence the light originates;
the light could come from any $\hat{x}u$ when his clock said $t-u$, for all real $u$.

Now we introduce coordinate systems.
O can describe an event in his O-spacetime coordinate $(t,x,y,z)$,
and P can describe the same event in his P-spacetime coordinate $(t',x',y',z')$.

An event happens at O-spacetime point P iff
O would instantaneously observe the event if he were at P.
If a light is fired from $(t,x,y,z)$ towards O,
then he will detect it at $(t+\sqrt{x^2+y^2+z^2}/c,0,0,0)$.
The set of points that are O-simultaneous at O-time $t$ is $\{(t,x,y,z) ~|~ x,y,z\in\mathbb{R}\}$.

Suppose that M has a ruler.
One tip, A, is at M.
The other tip, B, is at $\vec{x}$.
Event C: At M-time $t_0$, M fires a light from A.
Event D: At M-time $t_1$, the light hits B and reflects.
Event E: At M-time $t_2$, the reflection arrives at A.
The M-length of the ruler can be found from $\norm{\vec{x}} = c~(t_2-t_0)/2$.

Suppose that O sees M with constant velocity $\vec{v}$.

% http://www.pbs.org/wgbh/nova/blogs/physics/2014/04/how-many-dimensions-does-the-universe-really-have/

If something travels in O-spacetime from $(t,x,y,z)$ to $(t+dt,x+dx,y+dy,z+dz)$,
then its velocity can be found as follows:
\[
(dx)^2 + (dy)^2 + (dz)^2 = v^2~(dt)^2.
\]
If that thing is light then $v=c$.

The light-distance between two points in spacetime:
\[
(ds)^2 = (dx)^2 + (dy)^2 + (dz)^2 - c^2~(dt)^2.
\]

An observer exists as a point in space.
Given a point, he can measure his distance to it.
Given two points, he can measure his distance to each;
he can measure the distance between them;
he can measure the angle formed by the points through him;
he can determine if they lie in a line;
he can calculate the area enclosed by those points and him.

% European Journal of Physics
% Relativity, H. Bondi
% http://iopscience.iop.org/article/10.1088/0034-4885/22/1/304
% Nothing but relativity, redux
% http://iopscience.iop.org/article/10.1088/0143-0807/28/6/011/meta
% Lorentz transformations with arbitrary line of motion
% http://iopscience.iop.org/article/10.1088/0143-0807/28/2/004

% The principle of relativity and the indeterminacy of special relativity
% http://iopscience.iop.org/article/10.1088/0143-0807/29/1/004

\footnote{\url{https://en.wikipedia.org/wiki/Orbital_decay}}%
\footnote{\url{https://en.wikipedia.org/wiki/Gravitational_wave}}%
\footnote{\url{https://en.wikipedia.org/wiki/Two-body_problem_in_general_relativity}}

\footnote{\url{https://en.wikipedia.org/wiki/Newtonian_motivations_for_general_relativity}}

Somatogravic illusion
\footnote{\url{http://aviationknowledge.wikidot.com/aviation:somatogravic-illusion}}

Brans-Dicke theory

\footnote{\url{https://en.wikipedia.org/wiki/Hafele\%E2\%80\%93Keating_experiment}}%
\footnote{\url{https://en.wikipedia.org/wiki/Ives\%E2\%80\%93Stilwell_experiment}}%
\footnote{\url{https://en.wikipedia.org/wiki/Kennedy\%E2\%80\%93Thorndike_experiment}}

    \part{Mechanics}
    \chapter{Mechanics}

\emph{Mechanics} is a theory of motion.

Reading:
\emph{The science of mechanics} by Ernst Mach.
Historical evolution.
The principles of statics.
The principles of dynamics.

\section{Understanding mass}

% http://www.ag-physics.org/rmass/
% https://en.wikipedia.org/wiki/Mass
The \emph{mass} of an object is the difficulty of changing its velocity.

Mass is resistance to force.

The mass of an object is the amount of matter in that object.

The \emph{rest mass} of an object is its mass measured if it is at rest.

\section{Understanding force}

\emph{Force} is the rate of change of momentum.

A force \emph{acts} on an object.

\section{Using vectors to model forces and others}

Position, momentum, velocity, acceleration, and force are modeled by \emph{vectors} (\S\ref{sec:vector}).
The position of \(B\) as measured from \(A\) is modeled by a \emph{vector} \(AB\).

\section{Superposing forces}

Forces acting on an object obey the \emph{superposition principle}:
the result of two forces \(F_1\) and \(F_2\) acting on the same object
is the same as the result of one force \(F_1+F_2\) acting on that object.

The \emph{net force} acting on an object is the sum of all other forces acting on that object.

\emph{Resultant force} is another term for \emph{net force}.

\section{Understanding moving frames}

A frame of reference may be \emph{moving},
for example when you look outside from a moving car.

\section{Understanding inertial frames}

An \emph{inertial frame of reference} \(R\) is a frame of reference such that
for each each object \( M \), if the net force acting on \( M \) is zero, then \(R\) sees that the acceleration of \(M\) is zero.

\section{Appreciating Galileo's ramps}

Galileo put a ramp (inclined plane),
rolled a ball from its top,
and measured the time required by the ball to reach the bottom.

% https://en.wikipedia.org/wiki/Inclined_plane
A narrow ramp.
To measure time, he put bells along the ramp.
The rolling ball hits different bells at different times.

% https://en.wikipedia.org/wiki/Equations_for_a_falling_body
Galileo's law of falling body

In year? Galileo \( h = k t^2 \).

\section{Newton's second law of motion}

If an object has constant mass \( m \) and a constant force \( F \) is acting on it,
then \( a = F/m \) is that object's constant acceleration.

\section{Mechanical system}

A \emph{mechanical system} is a set of objects \( \{ M_1,\ldots,M_n \} \) and forces \( \{ F_1,\ldots,F_n \} \).
Each \(F_k\) is an expression.
With Newton's laws, we can turn such mechanical system into \(n\) equations,
each of the form \( F_k = m_k \cdot d(d(x_k)) \) for \(k\) from 1 to \(n\).

One way of describing the motion of an object is by modeling time as a real number \( t \),
and modeling the position as a function of time \( x : \Real \to \Real^n \).
Thus, at time \( t \), the object is at \( x(t) \).

\section{Field}

A \emph{field} assigns something to each point in space.
The \emph{gravitational field} assigns to each point a \emph{gravitational force per unit mass}.

A field is modeled by a \emph{multivariate function} (a function that takes several variables).
The variables can be grouped into a vector.
This gives the impression that the function takes one big vector instead of several scattered real numbers.

A \emph{scalar field} is a field that gives a scalar.

A \emph{vector field} is a field that gives a vector.

A field \(f\) is \emph{uniform} iff \(f(x)\) is the same for all \(x\).

\section{Weight}

After Newton's law of universal gravitation,
\emph{weight} means gravitational force.
The weight of an object on Earth is the gravitational force exerted by Earth on that object.
\emph{Work} generalizes to \( W = F \cdot x \).

\emph{Work} was defined as weight times height.

\section{Path of an object in a field}

\emph{Path} of an object moving in a field.
A \emph{conservative force} is a force whose work depends only on the difference between the beginning and ending position,
and not in the path?
A force whose work is the same for every path from \(A\) to \(B\)?
The \emph{action} of a path?
Principle of stationary action?

\section{Conservative force}

% https://en.m.wikipedia.org/wiki/Conservative_force

Conservative force \emph{conserves} mechanical energy.

\section{Potential energy}

% https://en.wikipedia.org/wiki/Potential_energy

Wikipedia "potential energy":
Potential energy is associated with forces that act on a body in a way that the total work done by these forces on the body depends only on the initial and final positions of the body in space. These forces, that are called conservative forces, can be represented at every point in space by vectors expressed as gradients of a certain scalar function called potential.

\section{Field as gradient of potential}

(This requires multivariate calculus.)

\section{Galilean invariance?}

% https://en.wikipedia.org/wiki/Galilean_invariance
% https://en.wikipedia.org/wiki/Galileo%27s_ship
% Galilean boost
% https://en.wikipedia.org/wiki/Galilean_transformation
% https://en.wikipedia.org/wiki/Galilean_transformation#Galilean_group

Also known as \emph{Galilean relativity}.
The \emph{Galilean invariance} is the statement
that Newton's laws of motion is the same in all inertial frame of references.

% https://en.wikipedia.org/wiki/Galilean_invariance
% Einstein's cabin

\section{Confirming experiments}

The experiment of dropping a feather and a ball in vacuum confirms classical mechanics.

\section{Disagreeing experiments}

Problem in atomic theory?

Double-slit electron experiment?

\section{Branches of mechanics}

\emph{Statics} is?
\emph{Dynamics} is?
\emph{Kinematics} describes motion without considering its cause.

\section{Michell\textendash{}Cavendish torsion balance experiment}

This experiment finds out \(G\), the gravitational constant.

\section{Moving from force-based thinking to energy-based thinking}

We thought \( F(t) = -g \) and thus \( a(t) = -g/m \) and \( v(t) = - gt / m \) and \( x(t) = - gt^2 / (2m) \).
We think about the forces,
figure out the accelerations,
integrate them to get the velocities,
and integrate them to get the positions.

How does Hamiltonian mechanics explain a ball falling near the ground?
\( P = mgh \).
\( K = \frac{1}{2}mv^2 \).
The state of the system is \( (h, mv) \).
The operator is \( P(h, mv) = mgh \) and \( K(h, mv) = \frac{1}{2}mv^2 \).
\( P + K = \text{constant} \) which means that \( \pdv{P}{h} = 0 \) and \( \pdv{K}{v} = 0 \).

\section{Generalization}

Weight is gravitational force.

\section{More complex cases?}

So far everything has been constant.
Now we shall consider the case where they change with time.

Let \(g\) be a vector.
For understanding phase space, we will consider
the motion of a point mass \(M\) influenced by a uniform gravitational field \( G(x) = g \).

The acceleration will be \( a(t) = g \).
The velocity can be obtained by integrating \( a \).
The position and acceleration are related by the equation \( a = d(d(x)) \).
In Newtonian dynamics, if we know \( x(0) \), \( v(0) \),
and all the forces acting on a body,
then we can calculate the trajectory (all past and future position and velocity) of that body.

Let \( F(t) \) be the \emph{force acting on \( M \)} (that is, the sum of all forces acting on \(M\)) at time \(t\).
Let \( x(t) \) be the position of \( M \) at time \(t\).
Let \( v(t) \) be the velocity of \( M \) at time \(t\).
Let \( a(t) \) be the acceleration of \( M \) at time \(t\).
Then \( a = d(v) \) and \( v = d(x) \).
Let \( p : \Real \to \Real^n \).
Let \( p(t) \) be the momentum of \( M \) at time \( t \).
Then \( F = d(p) \).

Newton's laws of motion:%
\footnote{\url{https://en.wikipedia.org/wiki/Newton\%27s_laws_of_motion}}

First law:
In an inertial frame of reference, an object either remains at rest or continues to move at a constant velocity, unless acted upon by a force.
Second law:
In an inertial reference frame, the vector sum of the forces F on an object is equal to the mass m of that object multiplied by the acceleration a of the object: \( F = d \ p \).
Let \( p : T \to M \cdot V \).
Third law:
When one body exerts a force on a second body, the second body simultaneously exerts a force equal in magnitude and opposite in direction on the first body.

Andrew Motte's 1729 English translation of Newton's 1726 third edition of
\emph{Philosophiae naturalis principia mathematica} uses English words and geometry;
the modern statement uses algebra.

Newton's law of universal gravitation:%
\footnote{\url{https://en.wikipedia.org/wiki/Newton\%27s_law_of_universal_gravitation\#Modern_form}}

Force carrier\footnote{\url{https://en.wikipedia.org/wiki/Force_carrier}}

% https://en.m.wikipedia.org/wiki/Kinetic_theory_of_gases

% https://en.m.wikipedia.org/wiki/Philosophiæ_Naturalis_Principia_Mathematica

Shell theorem

Newton's laws of motion imply Kepler's laws of planetary motion.

    \chapter{Analytical mechanics}

% https://halshs.archives-ouvertes.fr/halshs-00116768/file/chap5.pdf
\emph{The origins of analytical mechanics in 18th century}, Marco Panza
\cite{panza2003origins}

About the name:
Why is it ``analytical mechanics'' and not ``analytic mechanics''?

\emph{Analytical mechanics} is mathematical analysis applied to mechanics.
Calculus is a part of mathematical analysis.
Calculus is about limit, derivative, and integral.
Several \emph{formulations} of mechanics are
Newtonian, Lagrangian, Hamiltonian, Routhian,
one due to Appell,
and one due to Udwadia\textendash{}Kalaba.

% https://en.wikipedia.org/wiki/Analytical_mechanics
% https://en.wikipedia.org/wiki/Mathematical_analysis
% https://en.wikipedia.org/wiki/Gauss%27s_principle_of_least_constraint

\section{Principle of economy}

\section{Variational principles}

\emph{D'Alembert's principle of virtual work}?
Virtual work?
Virtual displacement?
% https://en.wikipedia.org/wiki/D%27Alembert%27s_principle
% https://en.wikipedia.org/wiki/Virtual_displacement
% http://fy.chalmers.se/~tfemc/mekanikkompendium.pdf
% https://en.wikipedia.org/wiki/Hamilton%27s_principle
% https://en.wikipedia.org/wiki/Principle_of_least_action
% https://en.wikipedia.org/wiki/Routhian_mechanics
% https://en.wikipedia.org/wiki/Appell%27s_equation_of_motion
% https://en.m.wikipedia.org/wiki/Udwadia–Kalaba_equation

\section{Lagrangian mechanics}

Introduction to analytical mechanics \cite[p.~43]{varvoglis2014history}

% https://en.wikipedia.org/wiki/Lagrangian_mechanics#From_Newtonian_to_Lagrangian_mechanics

% https://archive.org/details/springer_10.1007-978-94-015-8903-1
% https://en.m.wikipedia.org/wiki/Mécanique_analytique
Lagrange's \emph{M\'ecanique analytique} was published in 1788 \cite{lagrange1997analytical}.

Lagrangian mechanics only works for conservative forces?

The \emph{Lagrangian} of a mechanical system is?

% https://en.m.wikipedia.org/wiki/Lagrangian_mechanics

Let \( e_k \in \Real^n \) be the \emph{\(k\)th basis vector of \( \Real^n \)};
it is \( e_k = [\delta_{ik}]_{i=1}^n \);
the \(k\)th component of \( e_k \) is one; every other component of \( e_k \) is zero;
\( (e_k)_k = 1 \) and \( (e_k)_i = 0 \) if \( i \neq k \).

For example, assume \( \Real^3 \), and let there be a particle \( M \)
whose mass is \( m \)
and whose position is \( (0,0,0) \).
The \emph{gravitational field} of \( M \) at \( x \) is \( g(x) = - G m x / |x|^3 \).
Such \( g \) is a vector field.
The \emph{gravitational potential} of \( M \) is the \( \phi \) such that \( \nabla \phi = g \).
Such \( \phi \) is a scalar field.
The \emph{potential energy of \( N \) due to \( M \)} is \( K_{MN} = m_N \cdot \phi(x_N) \).

Every particle in a system translates to two things in the \emph{phase space}: a position and a momentum.
We can describe a system without explicit reference to time.
We describe each particle by a set of \emph{position-momentum pairs}.
This set is the \emph{phase space} of the system.

Why bother using phase space if systems of equations work just fine?

We can \emph{describe} the trajectory of a particle using a function
whose input is a real number representing relative time
and output is a three-dimensional real vector representing relative position.
This is straightforward to imagine.

However, we can also describe the same thing by a set of ordered pairs
\( \{ (t,x) ~|~ \text{the object is at \(x\) at time \(t\)} \} \).

Newton's law of gravity describes the force that a
\emph{point mass} exerts on another point mass.
It still applies to planets even if when we assume that a planet is a point mass.
\[
    F_{ab} = \frac{G m_a m_b}{|r_{ab}|^2} \hat{r}_{ab}
\]

% https://en.wikipedia.org/wiki/Gauss%27s_law_for_gravity

Newton's second law:
\(F = dp\) where \(F(t)\) is force acting on the point mass at time \(t\)
and \(p(t)\) is the momentum of the point mass at time \(t\).

The \emph{degree of freedom} of a system is the minimum number of parameters required to describe that system.

A system of \emph{equations of motion} has the form:
\begin{align*}
    x_1(t) &= \ldots
    \\
    & \vdots
    \\
    x_n(t) &= \ldots
\end{align*}

Imagine that in front of you there is a \emph{pendulum} hanging on a thread attached to the roof.
To model that system, we could pick the XYZ coordinate system
where, from your point of view,
the positive X axis is rightward, the positive Y axis is forward, and the positive Z axis is upward.
Thus, at all times, the force acting on the pendulum is \( F = (0,0,-mg) \).
This should be straightforward to imagine.
But you have to determine the tension of the thread that constrains the pendulum's motion.

But we can pick another coordinate system where a point is described by \( (\theta) \).
Let \(\theta\) be the angle from the vertical axis to the thread:
\( \theta = 0 \) means that the thread is vertical,
and positive \( \theta \) means that the pendulum is to your right.
Let \( K \) be the line length.
\(h = K - (1 - \cos \theta) K = K \cos \theta\).
\(P = m g h\).
\(K = \frac{1}{2} m v^2\).
Generalized coordinates: \((h,\theta)\) instead of \((x,y)\).
Translation rules: \(x = K \sin \theta\) and \(y = K \cos \theta\).
The point \(x,y=0,0\) is the lowest point of the pendulum.

\section{Example: Two rigid bodies}

Assume constant mass.
\begin{align*}
    m_1 \cdot (d^2 x_1)(t) &= G m_1 m_2 \cdot (x_2(t) - x_1(t)) / \norm{x_1(t) - x_2(t)}^3
    \\
    m_2 \cdot (d^2 x_2)(t) &= G m_1 m_2 \cdot (x_1(t) - x_2(t)) / \norm{x_1(t) - x_2(t)}^3
\end{align*}
Matrix form:
\begin{align*}
    \bmat{
        m_1 \cdot (d^2 x_1)(t)
        \\
        m_2 \cdot (d^2 x_2)(t)
    }
    &=
    \frac{G m_1 m_2}{\norm{x_1(t) - x_2(t)}^3}
    \bmat{
        x_2(t) - x_1(t)
        \\
        x_1(t) - x_2(t)
    }
\end{align*}

\paragraph{Example}
Uniform gravitation field.
Phase space coordinate \(h\) where \(h\) is height.
\(P(h) = m g h\).
\(K(h) = m (v(h))^2 / 2\).
Conservation of energy: \(P + K = E\).
\(d_h E = 0 = m g + \frac{1}{2} m \cdot (d_h v)(h) \cdot 2 v(h)\).
\(0 = g + (d_h v)(h) \cdot v(h)\).
\(- g = d_h v \cdot v\).

\section{Canonical coordinates}

\section{Poisson bracket}

The \emph{Poisson bracket} is ...

\section{Hamiltonian mechanics}

With physical laws, we can predict the state of physical systems.

A
\index{configuration space}%
\emph{configuration space} is a vector space where each vector is a generalized coordinate tuple.

Example of Hamiltonian mechanics:
In two-body problem,
the state space is ...,
the configuration space is ...,

\section{Noether's theorem}

\subsection{Conservation of energy}

\subsection{Newton's third law of motion}

    \chapter{Quantum mechanics}

\cite{manousakis2016practical}

\section{Understanding the need for quantum mechanics}

Classical mechanics works for slow big things.

Relativity works for fast big things.

Non-relativistic quantum mechanics works for slow small.

Relativistic quantum mechanics works for fast small things.

We abbreviate \emph{quantum mechanics} to QM.

\begin{table}[h]
    \centering
    \begin{tabular}{l|ll}
        & slow & fast
        \\
        \hline
        small & non-relativistic QM & relativistic QM
        \\
        big & Galilean relativity & Einsteinian relativity
    \end{tabular}
    \caption{Four theories}
\end{table}

% https://en.wikipedia.org/wiki/History_of_quantum_mechanics

% http://theorie2.physik.uni-erlangen.de/index.php/Papers_from_the_beginning_of_quantum_mechanics

% Schrodinger's paper
% https://journals.aps.org/pr/abstract/10.1103/PhysRev.28.1049 (paywall)
% http://web.archive.org/web/20081217040121/http://home.tiscali.nl/physis/HistoricPaper/Schroedinger/Schroedinger1926c.pdf

Dirac's 1978 4th edition \emph{Principles of quantum mechanics} \cite{dirac1978principles}

Self-interference? Counterintuitive.

% https://en.wikipedia.org/wiki/Mathematical_formulation_of_quantum_mechanics#Postulates_of_quantum_mechanics

% https://en.wikipedia.org/wiki/Matter_wave
% https://en.wikipedia.org/wiki/Matter_wave#de_Broglie_relations
% https://en.wikipedia.org/wiki/Pilot_wave
% causal interpretation of quantum mechanics
% https://en.wikipedia.org/wiki/De_Broglie%E2%80%93Bohm_theory
% https://en.wikipedia.org/wiki/Louis_de_Broglie

% https://en.wikipedia.org/wiki/Bra%E2%80%93ket_notation

% https://en.wikipedia.org/wiki/Uncertainty_principle
The \emph{Heisenberg uncertainty principle}?

% https://en.wikipedia.org/wiki/Wave_function#Definition_.28one_spinless_particle_in_1d.29
Example wave function?

Dirac's relativistic quantum mechanics

Positron

Quantum tunneling.
It boosts nuclear decay.
With it, a particle can escape a trap.
But, it prevents smaller computers.

\section{Knowing the history of quantum mechanics? Not the best way to learn?}

Bohr model of hydrogen atom

From Wikipedia\footnote{\url{https://en.wikipedia.org/wiki/Photoelectric_effect\#History}}.
1887: Heinrich Hertz discovered the photoelectric effect.
1902: Philipp Lenard observed that the energy of an emitted electron increased with the frequency of the light.
% https://en.wikipedia.org/wiki/Annus_Mirabilis_papers#Photoelectric_effect
1905: Albert Einstein explained that light is a particle, and the energy of a light particle is \(E = hf\).
Bohr inferred that because hydrogen emission spectrum line is discrete,
then hydrogen electron can only occupy certain orbitals with discrete spacing between orbitals.

% https://en.wikipedia.org/wiki/Photoelectric_effect
\index{effect!photoelectric}
\emph{Photoelectric effect}:
High-frequency light shone onto metal dislodges electron.
Low-frequency light has no effect.
The voltage required to stop the dislodged electron
depends on the frequency of the light.
The current depends on the intensity of the light.
The \emph{work function} of the metal is \( \phi = h f_0 \),
the energy required to dislodge its electron.
The rest of the energy becomes the kinetic energy of the electron.
The kinetic energy of an ejected electron is \( E = h f - \phi \).

The energy of a photon with frequency \(f\) is \( E = hf \).

\emph{Stern\textendash{}Gerlach experiment}

% https://en.wikipedia.org/wiki/Stern%E2%80%93Gerlach_experiment

\emph{Zeeman effect: electron spin splits spectral lines}

\emph{Stark effect: electric field splits spectral lines}

The 1914 \emph{Franck\textendash{}Hertz experiment}\footnote{\url{https://en.wikipedia.org/wiki/Franck\%E2\%80\%93Hertz_experiment}}
hurled electrons at mercury vapor.
No electron lost less than \SI{4.9}{eV} of kinetic energy.
This experiment supports Bohr's idea of quantized orbitals?
\cite{franck1967zusammenstosse}

% https://en.wikipedia.org/wiki/Spectral_line
Every atom has a unique spectral line, an emission pattern.
To identify the atom, we put it in a \emph{spectrometer},
and match the emitted spectral line
to known patterns\footnote{We can find some known patterns in \url{https://en.wikipedia.org/wiki/Spectral_line}.}.
Such activity is called \emph{spectrometry}.
Spectral lines also tell us what stars are made of.

It was 1925.
Heisenberg was trying to explain hydrogen \emph{spectral lines}.
While the \emph{old quantum theory} was trying to fix classical mechanics,
Heisenberg suggested that we start afresh instead \cite{heisenberg1925quantum}.
\footnote{\url{http://www.vub.ac.be/CLEA/IQSA/history.html}}

\Hyperlink{https://en.wikipedia.org/wiki/Matrix_mechanics}{Understanding matrix mechanics}

Born\textendash{}Heisenberg\textendash{}Jordan formulation:
\footnote{\url{http://fisica.ciens.ucv.ve/~svincenz/SQM333.pdf}}
infinite matrices.
What?

% https://en.wikipedia.org/wiki/Davisson%E2%80%93Germer_experiment
1923\textendash{}1927:
Davisson\textendash{}Germer experiment of de Broglie wavelength:

de Broglie wavelength is about the relationship between the momentum and the wave vector (and thus the wave number) of free particles.
\( p = \hbar k \).
\cite{okun2012abc}

1927: Davisson and Germer's electron diffraction experiment:

% https://en.wikipedia.org/wiki/History_of_quantum_field_theory
Meanwhile, Paul Dirac was trying to quantize the electromagnetic field.

% https://en.wikipedia.org/wiki/Quantum_electrodynamics
% https://en.wikipedia.org/wiki/Precision_tests_of_QED

% quantum mechanics in one dimension
% http://www.tcm.phy.cam.ac.uk/~bds10/aqp/handout_1d.pdf

% QM exercises
% https://www.math.temple.edu/~prisebor/qm1.pdf

% https://plato.stanford.edu/entries/qt-nvd/#3
% In a letter to Birkhoff from 1935, von Neumann says: “I would like to make a confession which may seem immoral: I do not believe in Hilbert space anymore”

%Heisenberg 1925 paper
%scan of English translation
%http://www.mat.unimi.it/users/galgani/arch/heis25ajp.pdf
%backup link
%http://fisica.ciens.ucv.ve/~svincenz/SQM261.pdf
%scan of German original?
%http://www.chemie.unibas.ch/~steinhauser/documents/Heisenberg_1925_33_879-893.pdf

%on that paper
%The 1925 Born and Jordan paper “On quantum mechanics”
%http://people.isy.liu.se/icg/jalar/kurser/QF/references/onBornJordan1925.pdf
% Heisenberg's 1924 paper
% Heisenberg\textendash{}Born\textendash{}Jordan "On quantum mechanics"
% http://people.isy.liu.se/icg/jalar/kurser/QF/references/onBornJordan1925.pdf


%2004 paper
%Understanding Heisenberg's 'Magical' Paper of July 1925: a New Look at the Calculational Details
%https://arxiv.org/pdf/quant-ph/0404009.pdf
%https://arxiv.org/abs/quant-ph/0404009

% https://en.wikipedia.org/wiki/Heisenberg_picture

1961: Clauss J\"onsson electron double-slit experiment:

% https://en.wikipedia.org/wiki/Interaction-free_measurement
% https://en.wikipedia.org/wiki/Elitzur%E2%80%93Vaidman_bomb_tester
% https://en.wikipedia.org/wiki/Mach%E2%80%93Zehnder_interferometer

Quantum physics and nuclear physics\footnote{\url{https://www.patana.ac.th/secondary/science/anrophysics/ntopic13/commentary.htm}}

\section{Working with complex numbers}

The \emph{imaginary number}\footnote{There is nothing imaginary about the imaginary number.
There is nothing real about the real number either.
Descartes coined these names, and they have stuck.}
is \( i = \sqrt{-1} \).
Thus, \(i^2 = -1\).

The set of \emph{complex numbers} is \( \Complex = \{ a + bi ~|~ a, b \in \Real \} \).

Example of complex number: \( 1 + 2i \).

Every real number is a complex number:
a real number is a complex number with zero imaginary part.
Thus, \( \Real \subset \Complex \).

\section{Working with complex vectors}

A \emph{complex vector} is a vector where every element is a complex number.
The set of \(n\)-dimensional complex vectors is \(\Complex^n\).

The \emph{dual} of a vector space \(V\) is \(V^*\)
which is obtained by transposing every element of \(V\).
Thus, every column vector in \(V\) becomes a row vector in \(V^*\).
Also, \((V^*)^* = V\).

The \emph{complex Hilbert space} is a subset of \(\Complex^\infty\).

\emph{Dirac bra-ket notation}:

\(\bra{A}\) (read ``bra-\(A\)'') is a row vector in the dual space of \(\Complex^\infty\).

\(\ket{B}\) (read ``ket-\(B\)'') is a column vector in \(\Complex^\infty\).

\(\braket{A}{B}\) is matrix multiplication \(\bra{A} \cdot \ket{B}\).
The result of \(\bra{A} \cdot \ket{B}\) is a complex number.

A ket represents a \emph{quantum state}.

\section{Working with operators}

% FIXME COIK
% https://en.wikipedia.org/wiki/Projective_Hilbert_space
% https://en.wikipedia.org/wiki/Quantum_state
A \emph{pure quantum state} is a ray in complex Hilbert space.

% https://en.wikipedia.org/wiki/Configuration_space_(physics)
A \emph{configuration} of a \(d\)-dimensional system with \(n\) particles consists of \(d \times n\) real numbers.
The \emph{configuration space} of the system is the set of all such points.

An \emph{operator} is a higher-order function?

\footnote{\url{https://en.wikipedia.org/wiki/Operator_(physics)\#Examples_of_applying_quantum_operators}}

Linear operator, adjoint, Hermitian, matrix

% https://en.wikipedia.org/wiki/Wave%E2%80%93particle_duality

For example, consider a ball falling near the ground.
How do we say it in the language of operators?

\section{Working with wave functions}

The \emph{wave function} of a system is a function that takes a configuration of the system and gives a complex number.

Example: the wave function of a system with \(n\) particles has the shape \(\psi(x_1,\ldots,x_n,t)\)
where every \(x_k : \Real^3\) and \(t:\Real\).

\emph{Born's statistical interpretation}:
The value of \( |\psi(c)|^2 \) is the \emph{probability density}
of measuring/finding/observing the configuration \(c\)?

Sobolev space?
\(L^2\) space?

The momentum of the object?

The position of the object?

% FIXME COIK
An \emph{observable} is a linear self-adjoint operator on a Hilbert space.

% https://en.wikipedia.org/wiki/Schr%C3%B6dinger_picture
The \emph{Schr\"odinger equation} constrains the wave function of a system?

% E.T. Jaynes's book
% http://bayes.wustl.edu/etj/prob/book.pdf

% https://en.m.wikipedia.org/wiki/Old_quantum_theory

% https://en.m.wikipedia.org/wiki/Transformation_theory_(quantum_mechanics)

% https://physics.stackexchange.com/questions/1417/general-relativity-gravitation-in-time-and-one-spatial-dimension

% Schutz's book?


An Introduction to Relativistic Quantum
Mechanics
I. From Relativity to Dirac Equation

% https://arxiv.org/pdf/0708.0052.pdf


% https://ocw.mit.edu/courses/physics/8-323-relativistic-quantum-field-theory-i-spring-2008/lecture-notes/


% https://en.wikipedia.org/wiki/Quantum_field_theory

Introduction to
Relativistic Quantum Field Theory
% http://www.tep.physik.uni-freiburg.de/lectures/QFT14/qft.pdf

% quantum mechanics made simple, weng cho CHEW
% http://wcchew.ece.illinois.edu/chew/course/QMALL20121005.pdf

% rather unstructured
% http://studylib.net/

\subsection{Modeling a lone one-dimensional free particle}

The wave function of one free (not influenced by any external fields) spinless one-dimensional particle is
\( \psi(x,t) = A \exp(k x - \omega t) \).

Momentum? Energy? Position?

\subsection{Modeling two one-dimensional particles}

Simulation hypothesis.

Assume the simulation hypothesis: the Universe is a simulation.

How do we overload it?
How do we crash it?
How do we glitch it?
How do we hack it?
How do we debug it?

A ket is a column vector. A bra is a row vector. The dual of a ket is its \emph{conjugate transpose}.
A ket represents a \emph{quantum state}?

quantum locking\footnote{\url{https://www.ted.com/talks/boaz_almog_levitates_a_superconductor/transcript}}

    \part{Particles}
    \chapter{Statistical physics}

\section{Appreciating the relevance of statistics}

Consider a box of gas with 1 billion particles.
It is impractical to model that by 1 billion equations of motion.
However, we can still say something useful,
because \emph{statistics} allows us to \emph{summarize} the gas.
With statistics, we can talk about \emph{macroscopic behavior},
but we can't talk about individual particles;
we get the summary and we sacrifice the details.

Now we can talk about the \emph{distribution} of the velocity of the particles,
such as \emph{50\% of the particles are slower than something}.

Statistical physics is macro-physics.
The idea is we consider a statistics of the system.
We look at the big picture instead of looking at each particle.
There are many particles.
We cannot say anything about one particle.
What is an example of \emph{statistical ensemble}?

\section{Getting used to probability and statistics}

\ExerciseAnswer{(Discrete probability) Roll a fair six-faced die once. What is the probability of getting the three-dotted face?}{\(3/6\).}
\ExerciseAnswer{(Joint probability of independent events) Roll a fair six-faced die three times. What is the probability of getting the three-dotted face three times?}{%
\((3/6) \times (3/6) \times (3/6)\).}

\ShowAnswers

A \emph{distribution} of a set \(\Omega\) describes how members of \(\Omega\) are distributed.
Let \(f\) be the density of that distribution.
Then \(f(x)\) describes the tendency of values to gather around \(x\)?
Values tend to gather near the peaks of \(f\).

\section{Using the Maxwell\textendash{}Boltzmann distribution of speed (for what?)}

An example question that statistical physics (statistical mechanics) can answer is
``What is the probability of finding a particle with a given speed?''
For example, see the probability density function of the Maxwell\textendash{}Boltzmann distribution.

Maxwell distribution is a chi-distribution with 3 degrees of freedom.

Don't remember the equation.
To be a physicist, you don't need to remember this; you can always go to Wikipedia or open a book.
The important thing is that you know \emph{what it means} and \emph{what it's useful for}.
The density of \emph{Maxwell\textendash{}Boltzmann distribution} is \(f(v)\).
The number \(\int_A f\) describes the \emph{probability of finding a particle
whose speed is in the set (the range) \(A\)}.
Let that sink for a moment, especially if you aren't yet comfortable with probability theory.
The density of \emph{Maxwell\textendash{}Boltzmann distribution} is
???
\Formula{
    \NoNumber
    f(v) = \parenthesize{ \frac{m}{2\pi k T} }^{3/2} 4 \pi v^2 \exp \parenthesize{ - \frac{mv^2}{2kT} }
}
Who got that? How?

% https://en.wikipedia.org/wiki/Temperature
TODO paraphrase this Wikipedia text:
Based on the historical development of the kinetic theory of gases, temperature is proportional to the average kinetic energy of the random motions of the constituent microscopic particles

% https://en.wikipedia.org/wiki/Maxwell–Boltzmann_distribution

Statistical mechanics explains thermodynamics.

% https://en.wikipedia.org/wiki/Thermodynamics

\emph{mole} is

Chemistry?

Entropy?

Canonical ensemble?

Statistical ensemble?

% http://demonstrations.wolfram.com/BoseEinsteinFermiDiracAndMaxwellBoltzmannStatistics/

    \chapter{Particles}

\section{Black body, temperature, and radiation}

\index{black body!definition}%
\index{definitions!black body}%
A \emph{black body} absorbs all radiation that hits it.

A black body radiates.
The spectrum depends on the body's temperature.

Every temperatured object radiates
because it consists of electrons and
accelerating electron emits electromagnetic radiation?
Because heat is vibration of particles?

% https://en.wikipedia.org/wiki/Bremsstrahlung
\index{braking radiation}%
\index{bremsstrahlung}%
Braking radiation (\emph{bremsstrahlung}) happens for the same reason:
deceleration of charged particle.
Deceleration is negative acceleration.

% https://www.quora.com/Why-does-accelerating-charge-radiate-electromagnetic-radiation
% https://en.wikipedia.org/wiki/Li%C3%A9nard%E2%80%93Wiechert_potential
An accelerating charge radiates because of Li\'enard\textendash{}Wiechert potential?

Does an accelerating charge always radiate?

\section{de Broglie wavelength}

Every particle of momentum \(p\) is also a wave of wavelength \(\lambda = h / p\),
and vice versa: every wave is also a particle?

The \emph{Planck constant} is \( h \approx \SI{6E-34}{J.s} \).

\section{Condensed matter physics}

Bose\textendash{}Einstein condensate

Spontaneous emission

\section{Laser}

Laser cooling: using laser to cool atoms.

Thought experiment: Schr\"odinger's cat

Cosmology

Big bang

Cosmic microwave background radiation

Tunneling

Josephson effect

% https://en.wikipedia.org/wiki/Josephson_effect

Macroscopic quantum phenomena

% https://en.wikipedia.org/wiki/Macroscopic_quantum_phenomena

Superfluidity

% https://en.wikipedia.org/wiki/Superfluidity

Photon energy and momentum

% how are photons produced?
% https://science.howstuffworks.com/light7.htm
% > A photon is produced whenever an electron in a higher-than-normal orbit falls back to its normal orbit. (?)
% https://en.wikipedia.org/wiki/Laser
% https://en.wikipedia.org/wiki/Spontaneous_emission
% https://en.wikipedia.org/wiki/Stimulated_emission


% https://physics.stackexchange.com/questions/2229/if-photons-have-no-mass-how-can-they-have-momentum
% Energy of photon E = hf = pc (because photon rest mass is zero)
% Momentum of photon p = E/c = hf/c

\section{Mirror}

% mirror
% https://en.wikipedia.org/wiki/Radiation_pressure
% https://en.wikipedia.org/wiki/Optical_amplifier

Conservation of momentum in light reflected by mirror? Quantum explanation of how mirror works?

% https://en.wikipedia.org/wiki/Franck%E2%80%93Hertz_experiment
% https://en.wikipedia.org/wiki/Thermionic_emission

Half-silvered mirror?

How optical diode works

% "optical diode"
% https://en.wikipedia.org/wiki/Optical_isolator
% http://physicsworld.com/cws/article/multimedia/2015/jul/08/how-do-you-produce-a-single-photon
% https://www.scientificamerican.com/article/how-do-mirrors-reflect-ph/

Hardy's paradox

% https://en.wikipedia.org/wiki/Hardy%27s_paradox

Argument: the universe is computer simulation

% http://www.bottomlayer.com/bottom/argument/Argument4.html

\section{Radiant exitance}

\index{radiant power}%
\index{power!radiant}%
\index{units!radiant power (\si{W})}%
The unit of \emph{radiant power} is watt.

\index{definitions!radiant exitance}%
\index{radiant exitance!definition}%
The \emph{radiant exitance} (radiant emittance)
of a surface is the radiant power
emitted by that surface
per unit area of that surface.
\index{radiant exitance!unit (\si{W/m^2})}%
\index{units!radiant exitance (\si{W/m^2})}%
The unit of radiant exitance is \si{W/m^2}.

% https://en.wikipedia.org/wiki/Larmor_formula
\emph{Larmor formula}?

Rayleigh\textendash{}Jeans law

Wien approximation

% https://en.wikipedia.org/wiki/Stefan%E2%80%93Boltzmann_law

\index{black body!Stefan\textendash{}Boltzmann law of radiant exitance}%
\index{laws!black body radiant exitance}%
\index{laws named after people!Stefan\textendash{}Boltzmann law of black body radiant exitance}%
\emph{Stefan\textendash{}Boltzmann law:}
The total radiant exitance of a black body at temperature \(T\) is \( j^\star = \sigma T^4 \).

% https://en.wikipedia.org/wiki/Josef_Stefan#Work
1879, Stefan derived from Dulong and Petit.
1884, Boltzmann: heat engine with light as working matter (opposed to the usual steam).
Stefan\textendash{}Boltzmann law can be derived from Planck's law?

\section{Planck's law of black body radiation}

% https://en.wikipedia.org/wiki/Photon_gas
% Stefan–Boltzmann law - the total flux emitted by a black body

% https://en.wikipedia.org/wiki/Rayleigh%E2%80%93Jeans_law
% https://en.wikipedia.org/wiki/Wien_approximation
% https://en.wikipedia.org/wiki/Planck%27s_law

Let \(h\) be Planck's constant.

Let \(k\) be Boltzmann's constant.

\index{laws!black body radiation}%
\index{laws named after people!Planck's law of black body radiation}%
\emph{Planck's law of black body radiation}:
The \emph{radiance} of a black body for frequency \(f\) at temperature \(T\) is
\begin{equation}
    B(f,T) = \frac{2hf^3}{c^2} \cdot \frac{1}{\exp\left(\frac{hf}{kT}\right) - 1}.
\end{equation}

    \chapter{Nuclear power}

\Hyperlink{https://en.wikipedia.org/wiki/Nuclear_physics}{Nuclear physics, history, modern}

\cite{harms2000principles}

\paragraph{Why study nuclear power?}
We want to build a nuclear fusion power plant.

% Summary
% Question prioritization

% has some references
% http://www.nuclear-power.net/laws-of-conservation/law-of-conservation-of-momentum/
% from google search: nuclear product momentum

% http://www.ppe.gla.ac.uk/~protopop/teaching/NPP/P4-NPP.pdf
% http://www.ppe.gla.ac.uk/~protopop/teaching/NPP/

% https://en.wikipedia.org/wiki/Fusion_power
% https://en.wikipedia.org/wiki/Particle_accelerator
% https://en.wikipedia.org/wiki/Electrostatic_nuclear_accelerator

% https://en.wikipedia.org/wiki/Ab_initio_methods_(nuclear_physics)

% Chemistry 418/518 Nuclear Chemistry Winter 2015
% http://oregonstate.edu/instruct/ch374/ch418518/
% http://oregonstate.edu/instruct/ch374/ch418518/Chapter%2010%20NUCLEAR%20REACTIONS.pdf

% http://www.academia.edu/23818742/Blue_Sky_Research_Fusion_Power_via_Piezoelectric_Crystals

\ResearchQuestion{How practical is maintaining a 5-megavolt electric field gradient to slow down a typical alpha particle?}

\ResearchQuestion{Vacuum breakdown voltage? Why is electrical arc in vacuum even possible?}

What determines the momentum of the fusion products?

Ideas that have been tried?

Who coined the term \emph{radioactive}?

What is a nuclide?

What is an isotope?

What is a nuclear decay?

What is half life?

What is plasma physics?

Heavy atoms split.
Why?

\section{Summary}

We want to have \emph{practically unlimited} energy.

Nuclear power source is the densest known power source in 2017.
% https://en.wikipedia.org/wiki/Space-based_solar_power

A \emph{nucleon} is a proton or a neutron.

\section{Understanding current approaches of building fusion plants}

% https://en.wikipedia.org/wiki/Nuclear_fusion#Methods_for_achieving_fusion

Current projects:
ITER fusion reactor.
% https://www.iter.org/

% https://en.wikipedia.org/wiki/Lawson_criterion
% https://en.wikipedia.org/wiki/Fusion_energy_gain_factor

Crowdfunded nuclear reactor?

\section{Dreaming of personal nuclear power plants}

%https://whatisnuclear.com/
%https://en.wikipedia.org/wiki/Table_of_nuclides
%https://www-nds.iaea.org/relnsd/vcharthtml/VChartHTML.html

Let's build a personal nuclear power plant at home.
We dream of a tool for extracting energy out of the air we breathe.
We dream of something like Mr. Fusion from \enquote{Back to the future} (1985 film).
Put anything in, as long as it is lighter than iron, and the mass transforms into energy.
This may be too na\"ive.

% https://en.wikipedia.org/wiki/Proton%E2%80%93proton_chain_reaction

% https://en.wikipedia.org/wiki/Small,_sealed,_transportable,_autonomous_reactor

Goal:
portable safe personal nuclear reactor.

\section{Understanding the atomic theory and the nucleus}

An \emph{atom} has a \emph{nucleus}.%
\footnote{The word \emph{atom} was borrowed from a Greek word meaning \emph{indivisible}.
The word \emph{nucleus} was borrowed from a Latin word meaning \emph{kernel}.}

A \emph{nucleus} has \emph{protons} and \emph{neutrons}.

The \emph{proton number} is the number of protons.
The \emph{neutron number} is the number of neutrons.
The \emph{mass number} is the sum of the proton number and the neutron number.

An \emph{isotope} has the same proton number but different neutron number.

% https://en.wikipedia.org/wiki/Uranium
A uranium atom (U) has 92 protons.
The isotope U-238 has 146 neutrons because \(92 + 143 = 235\).
\enquote{U-238} means the uranium isotope with mass number 238.

\ExercisePage{
\section{Exercise: Atomic theory}
\ExerciseAnswer{What is the mass number of a U-235 atom?}{235. This is by definition: the mass number of A-xyz is xyz.}
\ExerciseAnswer{How many protons are in a U-235 atom?}{92. This is by definition, so the only way to know it is to remember it.}
\ExerciseAnswer{How many neutrons are in a U-235 atom?}{\(235 - 92 = 143\).
If you remember the proton number of U, you can compute the neutron number of U-235, U-238, and so on.}
\ExerciseAnswer{Is an atom with 100 protons and 120 neutrons an isotope of uranium?}{No. Every uranium atom has 92 protons by definition.}
}

Democritus...

Dalton's theory of atoms...

Farnsworth's fusor is an electrostatic confinement fusion device, but its net power is negative.

\section{Getting used to electronvolt, a unit of energy}

\index{electronvolt}%
\emph{Electronvolt} is a unit of energy:
\begin{equation}
    \SI{1}{eV} \approx \SI{1.6E-19}{J}
\end{equation}

One \emph{electronvolt} is the kinetic energy that an \emph{electron}
gains by climbing one \emph{volt} of electric potential difference.
One electronvolt is the absolute charge of an electron multiplied by one volt.

One electronvolt is the difference of the kinetic energy of an electron
due to moving across one volt of potential difference.

% https://en.wikipedia.org/wiki/Orders_of_magnitude_(energy)
% What? The food energy of a snicker bar exceeds the kinetic energy of a 2-ton vehicle traveling at 115 km/h?
Let's build our intuition of electronvolts.
These are approximate.

\SI{2}{eV} is the energy of a red photon (600 nm wavelength)?

\SI{10}{eV} is ultraviolet-A?

\SI{5}{MeV} is the typical energy of an alpha particle?

\section{Understanding radioactivity}

% https://en.wikipedia.org/wiki/Henri_Becquerel
% https://en.wikipedia.org/wiki/Abel_Ni%C3%A9pce_de_Saint-Victor#Near-discovery_of_radioactivity

What is a photographic plate?

% FIXME copied from Wikipedia
1857:
\enquote{Ni\'epce de Saint-Victor observed that,
even in complete darkness,
certain salts could expose photographic emulsions.}

1896:
Becquerel found that uranium darkens photographic plates.

% Rutherford

% https://en.wikipedia.org/wiki/Radio#Etymology
% animation of an antenna transmitting a radio wave
% https://en.wikipedia.org/wiki/Radio_wave

How do we measure radioactivity?

\section{Nuclear fusion plan?}

Quantum tunneling and fusion?
%https://en.wikipedia.org/wiki/Gamow_factor

Quantum Tunneling in Nuclear Fusion, 1997
%https://arxiv.org/pdf/nucl-th/9708036.pdf

How do we craft the wave function to facilitate fusion?

The sun is massive enough for fusion to just happen.
What's the smallest fusion power plant we can make?

\section{Understanding plasma confinement challenges}

Does fusion power plant require confinement?
Can we build one without confinement?
Can we build one without vacuum possible?

\section{Understanding piezoelectricity}

How does piezoelectricity work?
How does electret work?

Can piezoelectric achieve at least \SI{50}{kV}?
% https://en.wikipedia.org/wiki/Pyroelectric_fusion
% https://en.wikipedia.org/wiki/Cockcroft%E2%80%93Walton_generator
% https://en.wikipedia.org/wiki/Inertial_electrostatic_confinement

What is the electric field strength required to stop a \SI{5}{MeV} alpha particle in distance \(d\)?
Assume that the mass is \SI{6.6E-27}{kg} and the charge is \(2e\).
% https://en.wikipedia.org/wiki/Alpha_particle

\section{Understanding aneutronic fusion}

% https://en.wikipedia.org/wiki/Aneutronic_fusion
Aneutronic fusion is fusion that doesn't produce neutrons.
Neutron is lost energy.

Neutron has zero charge but non-zero magnetic moment.
We can steer a neutron with a magnetic field.

%is this legitimate?
%https://www.indiegogo.com/projects/focus-fusion-empowertheworld--3#/

% https://en.wikipedia.org/wiki/Stellar_nucleosynthesis

Where does the produced energy go?
Some escapes.
Some helps other atoms fuse.

\section{Understanding tokamaks and why they are popular}

What is a tokamak?
Is it popular?
Why?

The beam loses energy.
It radiates them.
Because it is accelerated to move in circles.

Impulsing a pendulum
Back and forth laser oscillation, add power

\section{Understanding how quantum tunneling affects nuclear fusion}

% HOW LOW - ENERGY FUSION CAN OCCUR
% https://arxiv.org/pdf/1211.1243.pdf

Probability of nuclear fusion due to quantum tunneling?

% https://en.wikipedia.org/wiki/Nonlinear_optics#Frequency_doubling
% https://en.wikipedia.org/wiki/Two-photon_absorption
% https://en.wikipedia.org/wiki/Two-photon_physics
There may be a device that absorbs 2 photons with energy \(hf\) each and emits 1 photon with energy \(2hf\).
Frequency doubling.
Optical capacitor?
Optical amplifier?

\section{Research question: Can we focus matter-wave as we focus light waves?}

Quantum parabolic reflector/focuser?
And use that to increase the probability of fusion?

How can we use our knowledge of particle physics to solve the world's energy problem?
Almost everything we can see can be harnessed.
Even the air.

Hybrid fission-fusion reactor, like hybrid fission-fusion bomb?

Quantum electrodynamics, quantum chromodynamics, quantum field theory

% quantum physics
% https://en.m.wikipedia.org/wiki/Matter_wave

\section{Research question: How do we build nuclear fusion reactors?}

stable open-air plasma ring:
piezoelectric plasma generator?
\url{https://futurism.com/stable-plasma-ring-created-open-air-first-time-ever/}

% use electret for plasma confinement for nuclear fusion?
% https://en.wikipedia.org/wiki/Electret

\section{Research question: How do we build self-sustaining plasma?}

% https://www.quora.com/What-are-the-cheapest-metals
% Materials Selection in Mechanical Design, Fifth Edition (9780081005996): Michael F. Ashby: Books

% kurikulum teknik nuklir UGM
% https://tf.ugm.ac.id/images/akademik/Kurikulum_TN_2016.pdf

%Pengantar Teknik Nuklir
%Deteksi dan Pengukuran Radiasi 
%Elektronika Nuklir
%Praktikum Elektronika Nuklir
%Fisika Reaktor Nuklir
%Praktikum Deteksi dan Pengukuran Radiasi
%Radiokimia
%Praktikum Radiokimia
%Proteksi Radiasi
%Komputasi Nuklir
%Prakt ikum Fisika Reaktor Nuklir
%Pengelolaan dan Pengolahan Limbah Radioaktif
%Sistem Keselamatan, Keamanan dan Safeguard Nuklir 

%TKN 4501 Termal Hidraulika Reaktor Nuklir (*) 3 2
%TKN 4502 Teknologi Pembangkit Daya Nuklir (*) 2 3
%TKN 4503 Instrumentasi Nuklir (*#$) 2 4
%TKN 4504 Material Nuklir (#$) 2 5
%TKN 4505 Analisis Reaktor Nuklir (*$)
%Pengelolaan dan Pengolahan BBN (*#$)
%Kimia Radiasi
%Perancangan Sistem Nuklir (*#$)
%Teknologi Reaktor Maju (*#$)
%Sistem Komponen Pendukung Reaktor Nuklir
%Manajemen BBN dalam Teras Reaktor (*#$)
%Sistem Kogenerasi Nuklir
%Teknologi Reaktor Fusi Nuklir (*#$)
%Teknologi Akselerator
%Teknologi Pengendalian Reaktor Nuklir
%Penerapan Radioisotop (*#)
%Penerapan Radiasi (*#)
%Teknik Pemisahan Isotop (*#)

%[1] R.L.  Murray, 2014 .  Nuclear Energy: A n Introduction to The Concepts, Systems, and Application of Nuclear Processes, 7 th Ed .  Butterworth - Heinnem an

Fermi's golden rule:
probability of quantum state transition
% https://en.wikipedia.org/wiki/Fermi's_golden_rule

% https://en.wikipedia.org/wiki/Fusion_power
% A reaction's cross section, denoted σ, is the measure of the probability that a fusion reaction will happen.
% In a plasma, particle velocity can be characterized using a probability distribution.

% https://en.wikipedia.org/wiki/Townsend_discharge
% https://en.wikipedia.org/wiki/Paschen%27s_law
% https://en.wikipedia.org/wiki/Two-photon_physics

\section{Converting other forms of energy to electrical energy}

% https://en.wikipedia.org/wiki/Direct_energy_conversion

Photon-matter interaction

% The interaction of photons with the matter.
% http://meroli.web.cern.ch/meroli/Lecture_photon_interaction.html
% https://physics.stackexchange.com/questions/83105/explain-reflection-laws-at-the-atomic-level
% https://physics.stackexchange.com/questions/132515/does-a-reflection-still-transfer-momentum-to-an-mirror

Photon-photon interaction

\section{Understanding weak and strong interaction}

Flavor change happens by weak decay.

Weak force, beta decay

% https://en.wikipedia.org/wiki/Weak_interaction

Other names: Weak force, weak interaction, weak nuclear force.

Field names:
Quantum flavordynamics / electroweak theory studies weak interaction.
Quantum chromodynamics studies strong interaction.

Strong force binds nucleus together / keeps nucleus intact.
Strong force confines quark into hadron.

\section{Navigating the particle zoo}

The Standard Model continues atomic theory.

% https://en.m.wikipedia.org/wiki/Nuclear_force

A \emph{meson} is 1 quark and 1 antiquark.

A \emph{baryon} is 3 quarks.

Gluon

Baryon

A \emph{fermion} is a particle that follows Fermi\textendash{}Dirac statistics.
A \emph{boson} is a particle that follows Bose\textendash{}Einstein statistics.

Quark

Lepton

Pion

Kaon

Muon

\section{Mass defect and nuclear binding energy: Where does the mass go?}

Revisit Lavoisier's law of conservation of mass.

\( E = mc^2 \).

An extremely tiny mass is \emph{lost} (converted into energy that escapes the weight scale).

Nuclear binding energy

\section{Nuclear fission}

\section{Nuclear fusion}

\section{Finding current research}

IAEA (International Atomic Energy Agency) holds a fusion energy conference every two years.
% https://www.iaea.org/
% http://www-pub.iaea.org/iaeameetings/48315/26th-IAEA-Fusion-Energy-Conference
% https://nucleus.iaea.org/sites/fusionportal/Pages/FEC/FEC.aspx


% Photonuclear reactions triggered by lightning discharge
% https://www.nature.com/articles/nature24630

% Converting chemical energy to electrical energy
% https://en.wikipedia.org/wiki/Fuel_cell

% https://en.wikipedia.org/wiki/Physics_beyond_the_Standard_Model

Criteria and candidates for terrestrial reactions
\footnote{\url{https://en.wikipedia.org/wiki/Nuclear_fusion\#Criteria_and_candidates_for_terrestrial_reactions}}

A common method for detecting neutrons involves converting the energy released from neutron capture reactions into electrical signals.%
\footnote{\url{https://en.wikipedia.org/wiki/Neutron\#Detection}}

\section{Energy}

Energy is a quantity that happens to be conserved?

\section{Wrongish}

Why is there no electron-electron fusion?
Because the strong nuclear force doesn't work on electrons.
Electrons are leptons?
Proton and neutron are baryons?
Hadrons?

\footnote{\url{https://en.wikipedia.org/wiki/Neutron_transport}}

\section{Unclear websites}

\footnote{\url{http://reactor-core.org/}}

    \chapter{Wave and optics}

Differential equation? Diffusion? Oscillation? Wave behavior?

This chapter has two purposes:
introduce differential equations to the reader,
and prepare the reader for quantum mechanics.

\section{Wave}

% https://en.wikipedia.org/wiki/Wave
("Wave", Wikipedia):
A \emph{wave} is an oscillation accompanied by a transfer of energy?
A wave is a disturbance that transfers energy through matter or space?

\section{Detecting waves}

We can detect a wave by a diffraction slit.
If it's a wave, it diffracts.

We assume the converse: if it diffracts, it's very likely a wave.

\emph{Undulation} is an old term for \emph{wave}.

\section{Oscillation of a loaded spring}

If a spring is loaded, pulled, and released, then it will oscillate.

The equation of motion can be derived from Hooke's law of spring restoring force (\S\ref{sec:hooke-s-law}).

Wave equation

Second-order differential equation

Water wave

D'Alembert's waves?

How do we describe oscillation?
Periodic motion?
Harmonic motion?

How do we describe a diffusion?
How do we derive the wave equation from the diffusion equation?

% https://en.wikipedia.org/wiki/Fick%27s_laws_of_diffusion
% https://en.wikipedia.org/wiki/Continuity_equation
% https://en.wikipedia.org/wiki/Diffusion_equation#Derivation
Diffusion equation is derived from continuity equation and Fick's laws of diffusion.
Assume homogeneous (made of the same thing everywhere)
isotropic (behaving the same everywhere) medium?
Diffusion:
The rate of diffusion is proportional to the gradient?
\[
    f(x,t+h) - f(x,t) = c \cdot \frac{[f(x - h, t) - f(x,t)] + [f(x + h, t) - f(x,t)]}{2}
\]
Divide both sides by \(h\)
\[
    D_t f(x,t) = c \cdot D_x f(x,t)
\]
\[
    \frac{\partial f}{\partial t} = c \cdot \frac{\partial f}{\partial x}
\]
\[
    \frac{\partial f}{\partial t} = \vec{c} \cdot \nabla f
\]
???

How do we describe waves?

How do we describe waves on a string?
Pulse on a string?
Pulse on a chain of springs?
Replace the springs with more smaller springs?

% https://en.wikipedia.org/wiki/D%27Alembert%27s_formula

A function \(f\) has \emph{period} \(p\) iff \(f(x+p) = f(x)\) for all \(x\).

Let \(f(x,t)\) be the \emph{amplitude} of the wave at position \(x\) and time \(t\).

Let the oscillator be at position \(0\).

Let \(g\) be an unknown function.

Flow:
\(f(x,t + dt) - f(x,t) = g(c,f(x,t),f(x-dx,t),f(x+dx,t))\)

\(f(x,t + dt) - f(x,t) = [f(x-dx,t)-f(x,t)] + [f(x+dx,t)-f(x,t)]\)

\section{Velocities}

Propagation velocity

Phase velocity

Group velocity

\section{Light wave}

\section{Fermat's principle of least time}

Light takes the path that takes the least time.

\section{Snell\textendash{}Descartes law of refraction}

% https://en.wikipedia.org/wiki/Snell%27s_law
% https://en.wikipedia.org/wiki/Snell%27s_law#History
\begin{equation}
    \frac{\sin \theta_1}{\sin \theta_2} = \frac{v_1}{v_2} = \frac{\lambda_1}{\lambda_2} = \frac{n_2}{n_1}
\end{equation}

Descartes 1637 \emph{Dioptrics},

Huygens 1678: Huygens\textendash{}Fresnel principle.

Snell's law can be derived from Fermat's principle?

Snell's law can be derived from Huygens\textendash{}Fresnel principle?

% https://en.wikipedia.org/wiki/Huygens%E2%80%93Fresnel_principle

\section{Optics}

\emph{Wavenumber} is?
\emph{Wavelength} is?
\emph{Frequency} is?
\emph{Phase speed} is?
\emph{Group velocity} is?

\emph{Dispersion relation} is?

\emph{Doppler effect} is

Expanding universe?

\section{Reflection}
\section{Diffraction}
\section{Diffusion}
\section{Dispersion}
\section{Interference}
\section{Superposition}
\section{Fresnel spot}
\section{Transversal wave}
\section{Longitudinal wave}
\section{Young's double-slit experiment}
\section{Isochronic oscillation of a pendulum}

\section{Camera obscura}

\section{Newton's 1672 prism splits white light into colors?}

\section{Young's 1803 double-slit experiment}

    \chapter{Electromagnetism}

The important things:
\UnorderedList{
\item Magnetic field induced by an electric current.
\item Lorentz force on a charge moving in magnetic field.
}

\emph{Electric current} is a flow of electric charges.
The unit of current is \emph{ampere}.
One ampere is one coulomb per second.

\section{William Gilbert's research of magnets}

\section{Electrical objects}

\emph{Electric charge} is similar to mass.

\emph{Point charge} is similar to point mass.

A point charge exists in a \emph{medium}.
An example medium is vacuum.
A \emph{homogeneous medium} is a medium that is the same everywhere.
Every part of a homogeneous medium has the same property as every other part of the same medium.

An \emph{electron} has negative electric charge.
A \emph{proton} has positive electric charge.

The \emph{oil drop experiment} measures the charge of an electron.

\section{Coulomb's law of electrostatics}

Let there be two objects \(C_1\) and \(C_2\).
Let both of them exist in the same homogeneous medium.
Let \( k \) be a \emph{constant} that depends on the medium.

Let \( q_1 \) be the charge of \(C_1\).

Let \( q_2 \) be the charge of \(C_2\).

Let \( x_{12} \) be the position of \(C_2\) as seen from \(C_1\).
In other words, let \( x_{12} \) be the vector from \(C_1\) to \(C_2\).

Let \( F_2 \) be the force exerted by \(C_1\) on \(C_2\).

Let both \(C_1\) and \(C_2\) be \emph{stationary} (let their velocities be zero).

\emph{Coulomb's law} states that \( F_2 = k q_1 q_2 x_{12} / |x_{12}|^3 \),
and, by symmetry, \( F_1 = k q_2 q_1 x_{21} / |x_{21}|^3 = -F_2 \) because \( x_{12} = -x_{21} \).
This is the \emph{electrostatic force}.
Coulomb's law is similar to Newton's law of universal gravitation,
but Coulomb's can repel or attract, depending on the sign of the charges.
Newton's has only been found to attract because negative mass has not been found.

% https://en.wikipedia.org/wiki/Dielectric
A \emph{dielectric} is ...

% https://en.wikipedia.org/wiki/Force_between_magnets#Gilbert_Model
% https://en.wikipedia.org/wiki/Magnetic_dipole%E2%80%93dipole_interaction
% https://en.wikipedia.org/wiki/Magnetism#History
% https://en.wikipedia.org/wiki/Faraday%27s_law_of_induction

% Where does magnetism come from?
% Magnetic moment? Electric current? Spin magnetic moment?
% https://en.wikipedia.org/wiki/Magnetism#Sources_of_magnetism
% https://en.wikipedia.org/wiki/Magnetism#Quantum-mechanical_origin_of_magnetism

\emph{Faraday's law of induction}:
A changing magnetic field through a closed loop causes an electric current?

Force between magnets, Gilbert model, magnetic pole?

\section{Pieces of electromagnetism}

% https://en.wikipedia.org/wiki/Lorentz_force

In 1820, Hans Christian {\O}rsted found that electric current deflects magnetic needle.

Thomson: \(F = q v \times B\).

\emph{Lorentz force law}:
A particle of charge \(q\) moving with velocity \(v\)
in an electric field \(E\) and a magnetic field \(B\)
experiences a force \( F = q E + q v \times B \).
This force is called \emph{Lorentz force}.
("Lorentz force", Wikipedia)
The equation can be factored to \( F/q = E + v \times B \).
If \(E = 0\) and \(q > 0\), then the Lorentz force can be visualized as follows:
Use your right hand, point your thumb right (this is the direction of \(v\)),
your index finger forward (this is the direction of \(B\)),
and your middle finger up (this is the direction of \(F\)).

\emph{Biot-Savart law}:
A constant electric current causes a magnetic field that does not vary with time.
Right-hand rule.
Use your right hand.
Raise your thumb.
Curl the other four fingers.
The thumb is the direction of the current.
The other four fingers is the direction of the magnetic field lines.

\emph{Ampere's law}?

\emph{Gauss's law of magnetism}?

\section{Maxwell's equations}

\section{Heaviside's vector-calculus formulation}

\section{Electromagnetic radiation, electromagnetic wave}

\section{Lorentz invariance}

\section{Electricity, electrochemistry}

An \emph{electrochemical cell} has chemical reaction and voltage.
A \emph{battery} is a collection of electrochemical cells.

% https://en.wikipedia.org/wiki/Electricity
\emph{Electricity} is the set of physical phenomena associated with the presence of electric charge ("Electricity", Wikipedia, unhelpful?).

Why can't battery be more efficient?

Why can't solar cells be cheaper and more efficient?

% https://en.wikipedia.org/wiki/Solar_cell
% https://en.wikipedia.org/wiki/Photovoltaic_effect
The \emph{photovoltaic effect} is ...
A \emph{photovoltaic cell}, also called \emph{solar cell}, is ...

Energy storage: flywheel, battery, reserve hydroelectric reverse water pumping back upstream, thermal energy storage, etc.

\section{Units}

1 ampere is ...

1 farad is ...

1 tesla is ...

1 weber is ...

\section{Electric circuits}

\section{Resistance}

\section{Ohm's law}

\emph{Ohm's law}: \( V = IR \).

\section{Hertz's antenna experiment}

\section{Electromagnetic wave}

\section{Light is electromagnetic wave}

\section{Electromagnetic radiation}

\section{Magnetism}

Magnetism is caused by electron spin?%
\footnote{\url{https://en.wikipedia.org/wiki/Magnetism\#Sources_of_magnetism}}

Tesla is the unit of magnetic flux density (magnetic field strength).
1 tesla is 1 weber per \si{m^2}.
\enquote{A particle, carrying a charge of one coulomb, and moving perpendicularly through a magnetic field of one tesla,
at a speed of one metre per second, experiences a force with magnitude one newton, according to the Lorentz force law.}%
\footnote{\url{https://en.wikipedia.org/wiki/Tesla_(unit)\#Definition}}%
\footnote{\url{https://en.wikipedia.org/wiki/Orders_of_magnitude_(magnetic_field)}}

    \chapter{Astronomy}

\section{Reading sky map to find celestial objects}

% https://en.wikipedia.org/wiki/Celestial_sphere

Celestial sphere.
The sky looks as if it were projected to a spherical screen.
If a star is far enough, it will look as if it were fixed in the sky.

You find a star in the sky.
You write a letter to your friend.
How do you write where that star is?
How do you explain to him which direction he should look at?

You use the \emph{equatorial coordinate system}.

\emph{Right ascension}, \emph{declination}, and \emph{epoch}.

Example: Alpha Centauri A.
Right ascension 14 h 39 m 35.06311 s.
Declination \(-60\deg\) 50' 15.0992".
Epoch J2000.

% https://en.wikipedia.org/wiki/Epoch_(astronomy)#Julian_years_and_J2000
J2000 is the Gregorian date 2000-01-01 12:00 TT (terrestrial time).

Star chart, star map, sky map

% https://en.wikipedia.org/wiki/Celestial_coordinate_system

Celestial coordinate system

% https://en.wikipedia.org/wiki/Star_chart

% https://en.wikipedia.org/wiki/Celestial_coordinate_system

Equatorial coordinate system

% https://en.wikipedia.org/wiki/Alpha_Centauri

\section{Distance}

% https://en.wikipedia.org/wiki/Parsec

1 au (astronomical unit) is roughly the distance between the Sun and the Earth.
It is about 150 million km.

Parsec is a unit of length.
\( 648000/\pi \).
1 pc is about 3.26 ly.

A \emph{light year} is the distance traveled by light in one year.
\emph{Light year} (ly) is a unit of \emph{distance}, not time.
1 au is about 6 light minutes.

\section{Objects}

A \emph{planet} is?

% https://en.wikipedia.org/wiki/Stellar_evolution
A \emph{star} is a luminous sphere of plasma held together by its own gravity.
("Star", Wikipedia)
Every star begins from collapsing clouds of gas and dust.
A \emph{protostar} is ...
A \emph{main-sequence star} is ...

A \emph{solar system} is?

A \emph{galaxy} is?

A \emph{nebula} is?

A \emph{constellation} is?

A \emph{satellite} is?

A \emph{moon} is?

A \emph{comet} is?

An \emph{asteroid} is?

A \emph{supernova} is?

A \emph{brown dwarf} is?

A \emph{white dwarf} is?

A \emph{black hole} is?

\section{Cosmology}

\section{Cosmogony}

    \part{Ending}
    \chapter{Computers}

We recommend that you skip this chapter
until you use a computer to do physics calculations.

\section{Working with limited precision}

Computers can only store finitely many digits.
Computers can only represent a \emph{finite} subset of \(\Real\).
Computers can't represent most real numbers.

The take-home message is:
\emph{Mixing numbers with varying exponents increases errors}.

To understand these phenomena, we have to understand how computers represent numbers.

\section{Understanding how computers represent numbers}

There are many ways:
\UnorderedList{
\item floating-point integers
\item fixed-point integers
\item arbitrary-precision integers
\item symbolic representations
\item one's-complement signed integers
\item two's-complement signed integers
\item unsigned integers
\item sign-magnitude
}

The representation depends on the program.

Each representation has its benefits and drawbacks.

If you are doing physics with computers,
then there's a high chance that the numbers you see are \emph{IEEE 754 double-precision floating-point integers},
which are often shortened to \emph{doubles}.%
\footnote{\url{https://en.wikipedia.org/wiki/Double-precision_floating-point_format}}

\footnote{\url{https://en.wikipedia.org/wiki/Binary_number}}%
\footnote{\url{https://en.wikipedia.org/wiki/Signed_number_representations}}

\subsection{Understanding \emph{doubles}}

A \emph{double} represents a number as the base-2 scientific notation
\(
(-1)^s \times 1 . d_{51} d_{50} d_{49} \ldots d_2 d_1 d_0 \times 2^{p - 1023}
\)
Note that the first significant digit is always one and is not stored in memory.
Each \(d_k\) is a binary digit (zero or one).

Bit 63 is the sign bit \(s\):
0 means positive;
1 means negative.

Bits 62\textendash{}52 (11 bits) are the \emph{biased exponent} \(p\).

Bits 51\textendash{}0 (52 bits) are called the \emph{significand} or \emph{mantissa}.
Bit 51 is \(d_{51}\), and so on; bit 0 is \(d_0\).

\subsection{Understanding the problem with doubles}

If a fraction has a denominator that is not a power of two,
then a double can't represent the fraction exactly.
For example, a double can't represent even a simple fraction such as \(1/3\) exactly
because the base-2 expansion of \(1/3\) doesn't terminate,
in the same way that the decimal expansion of \(1/3\) (\(0.333\ldots\)) doesn't terminate.

A double has only 53 significant binary digits.
A multiplication of two 53-digit numbers may produce up to a 106-digit number.
The result is rounded to 53 digits; thus at most 53 digits are lost.

There is also a website\footnote{\url{http://floating-point-gui.de/}} that explains the issue.

\subsection{Understanding computer algebra systems}

Computer algebra systems can represent numbers like \(\sqrt{2}\) and \(1/3\) exactly because
it does not represent numbers as strings of digits.
It stores \(\sqrt{2}\) as something like \verb@(sqrt 2)@.
It does not evaluate \(\sqrt{2}\) to \(1.4142\ldots\) before storing it.

\subsection{Understanding ulp: units of least precision}

The standard is IEEE 754.

For example, in IEEE 754 double-precision floating-point integers,
\( 2^{53} + 1 = 2^{53} \).

In some browsers, you can verify this. Press Ctrl+Shift+J to open its Console,
and then enter \verb@Math.pow(2,53)@, and then enter \verb@Math.pow(2,53)+1@,
and see that they give the same number.

\section{Solving a system of linear equations}

You can use GNU Octave.

Use left division:
\begin{verbatim}
A \ C
\end{verbatim}

This also works for Matlab.

\section{Related fields of study}

Related fields of study are
\emph{scientific computing} (also known as \emph{computational science})
and \emph{numerical analysis}.%
\footnote{\url{https://en.wikipedia.org/wiki/Computational_science}}%
\footnote{\url{https://en.wikipedia.org/wiki/Numerical_analysis}}

\section{Using an equation to program a computer}

    \chapter{Mathematics 2}

This chapter contains skipped mathematical details.
Theoretical physicists may have to care.
Engineers don't have to.

\section{Deriving the Lorentz transformation}

\label{sec:derive-lorentz-transform}

\index{Lorentz transformation!derivation}%
We can derive the Lorentz transformation in many ways,
but I think the most elegant derivation is \cite{pal2003nothing}.
It does so without assuming the constancy of the speed of light.
You \emph{should} read that paper to understand special relativity.
It's very readable.
I think every physics text should be at least as readable as that.
Next, read \cite{gannett2007nothing} which improves that paper.

Other derivations%
\footnote{\url{https://en.wikipedia.org/wiki/Lorentz_transformation\#Derivation_of_the_group_of_Lorentz_transformations}}.

% From E = mc2 to the Lorentz transformations via the law of addition of relativistic % velocities
% http://iopscience.iop.org/article/10.1088/0143-0807/26/4/006

% Unified derivation of the Galileo and the Lorentz transformations
% http://iopscience.iop.org/article/10.1088/0143-0807/3/2/008

% Yet another unified derivation of galilean and Lorentz transformations
% http://iopscience.iop.org/article/10.1088/0143-0807/4/4/110

\section{Differential 1-form}

% practical introduction to differential forms
% https://www.cefns.nau.edu/~schulz/diff.pdf

% primer on differential forms
% https://arxiv.org/pdf/1206.3323.pdf

% mathematics for physics
% http://www.goldbart.gatech.edu/PostScript/MS_PG_book/bookmaster.pdf

In \( dx(t) = a(t) \cdot dt(t) \),
each of \(dx\), \(a\), and \(dt\) is a \emph{function of \(t\)}.
We lift multiplication from real numbers to function space \( f \cdot g = t \to f(t) \cdot g(t) \),
and we write \( dx = a \cdot dt \).

\section{Limit}

(Delete this. Use non-standard (infinitesimal) analysis instead.)

\index{definitions!limit}%
Practically,
the \emph{limit of \(E\) as \(h\) approaches \(a\)}, written \( \lim_{h \to a} E \),
is what the expression \( E \) converges to
if \(h\) is taken from a sequence that converges to \(a\).

\section{Integral}

\subsection{Measure of real interval}

Let \( a, b: \Real \).
Let \(a \le b\).

The notation \( [a,b] \) describes the set of all real numbers between \(a\) and \(b\), including both.
Formally, \( [a,b] = \{ x ~|~ a \le x ~\wedge~ x \le b \} \).

\index{definitions!measure}%
Its \emph{measure} is \( \mu([a,b]) = b - a \).

The measure of a set describes how big that set is.

Do not confuse \emph{measure} with \emph{cardinality} (how many elements a set has).

\subsection{Partitioning a set}

\index{definitions!partition (to partition a set)}%
To \emph{partition} a set is to divide it into disjoint subsets.
A partitioning of a set \(S\) into \(n\) partitions is \((S_1,\ldots,S_n)\) such that
every pair of partitions is disjoint and the union of all of partitions is \(S\).

\subsection{Riemann integral}

% FIXME: COIK (clear only if known)
\index{definitions!integral}%
Practically, the \emph{integral} of a function \(f\) in \(A\)
is the area between the curve of the function and the x-axis that coincides with \(A\).
Let \( f : \Real \to \Real \).
Partition \(A\) into \(n\) partitions \(A_1,\ldots,A_n\).
For each \(k\), pick one point \(a_k \in A_k\).
\index{definitions!Riemann sum}%
A \emph{Riemann sum} is \( S_n = \sum_{k=1}^n f(a_k) \cdot A_k \).
\index{definitions!Riemann integral}%
The \emph{Riemann integral of \(f\) in \(A\)} is
\( \int_A f = \lim_{n \to \infty} S_n \).

\section{Calculus}

\subsection{Antiderivative}

% https://en.wikipedia.org/wiki/Antiderivative
Let \( f = d(F) \).

Then \( f \) is \emph{the derivative} of \( F \).

\index{definitions!antiderivative}%
Then \( F \) is \emph{an antiderivative} of \(f\).

A function has many antiderivatives.
Let \(c\) be a constant.
If \(F\) is an antiderivative of \(f\), then so is \(F + c\).

Interpretation:
\( (d(f))(x) \) is the rate of change of \(f\) at \(x\).
\( (d(f))(x) \) is the slope of the tangent line of \(f\) at \(x\).

\section{Differential equation}
\label{sec:diff-eqn}

We overload the notation 0 to mean not only the number zero but also the constant function \( x \to 0 \).

\index{definitions!differential equation}%
A \emph{differential equation} is an equation that has the derivative operator \(d\).
Example: \( f + d(d(f)) = 0 \).
Let \(e\) be the base of natural logarithm.
Example: \( f(x) = e^x \) is one of the solutions of \( f = d(f) \).

\section{Calculus of variations?}

\subsection{Path, functional of a path, calculus of variations, Euler\textendash{}Lagrange equation}

\index{definitions!path}%
A \emph{path} in \( \Real^n \) can be described by a function \( f : \Real \to \Real^n \).
The path is then the set \( \{ f(a) ~|~ a \in \Real \} \), that is the \emph{image} of \(f\).
We overload notation.
\( (f+g)(x) = f(x) + g(x) \).

A \emph{functional} has type \( (\Real \to \Real^n) \to \Real \).

\section{Differential form, 1-form, 2-form}

% https://en.wikipedia.org/wiki/Differential_form
% prerequisite for:
% - sympletic manifold
% - K\"ahler manifold
% - Calabi\textendash{}Yau manifold

\section{Relating derivatives, tangent lines, best linear approximations}

\section{Generalizing vector spaces}

(We don't seem to need vector space over a field beyond \(\Complex\) in this book.
We don't need this level of generality in this book.
We should remove this paragraph.)
Let \(F\) be a field (in abstract algebra).
A \emph{vector space over \(F\)} is a space \( V \), a \emph{vector addition} operator \( + : V \to V \to V \),
a \emph{scalar multiplication} operator \( \cdot : F \to V \to V \), and the \emph{vector space axioms},
which can be seen in the \enquote{Definition} section in the Wikipedia article for \enquote{vector space}.
A \emph{vector} is a point in \( V \).
If \( F = \Real \), then \( V \) is called a \emph{real vector space}.
If \( V = \Real^n \), then \( V \) is called the \emph{\(n\)-dimensional real vector space}.
Do not confuse the field in physics (multivariate function) with the field in abstract algebra.

\section{Defining the derivative using infinitesimals}

\footnote{\url{https://en.wikipedia.org/wiki/Non-standard_calculus}}

\index{definitions!derivative (by infinitesimal)}%
\( d(f) = x \to st\left(\frac{f(x+\delta)-f(x)}{\delta}\right) \)

\section{Defining derivative using limit}

Let \( f : \Real \to \Real \).

\index{definitions!derivative}%
The \emph{derivative of \( f \) at \(x\)} is \( \dd(f,x) = \lim_{h \to 0} \frac{f(x+h) - f(x)}{h} \).

The type of \(\dd\) is \( (\Real \to \Real) \to (\Real \to \Real) \),
which means that \(\dd\) takes a function and gives a function.

\section{Understanding the derivative as the best linear approximation at a point}

\( f(x+h) \approx f(x) + h \cdot f'(x) \)

Using this, we can approximate \(\sqrt{x}\) near a point we know,
we can approximate \(\sin(x)\) and \(\exp(x)\) for small \(x\).

\section{Understanding partial derivatives}

% https://en.wikipedia.org/wiki/Automatic_differentiation
% Symbolic differentiation: https://en.wikipedia.org/wiki/Computer_algebra
% https://en.wikipedia.org/wiki/Numerical_differentiation

Let \( f : \Real^n \to \Real \).

\index{definitions!partial derivative}%
The \emph{partial derivative of \(f\) at \( x \) with respect to the \(k\)th input}
is \( d_k(f,x) = \lim_{h \to 0} \frac{f(x + h e_k) - f(x)}{h} \).

The type of \(d_k\) is \((\Real^n \to \Real) \to (\Real^n \to \Real)\),
which means that \( d_k \) takes a function and gives another function.

\section{Understanding the gradient}

Let \(f : \Real^n \to \Real\).

\index{definitions!gradient}%
The \emph{gradient of \(f\) at \(x\)} is the vector \( (\nabla f)(x) \) whose \(k\)th component is \( ((\nabla f)(x))_k = d_k(f,x) \).
Note that \( (\nabla f)(x) \) is a vector in \( \Real^n \) and \( (\nabla f) : \Real^n \to \Real^n \).

The type of \(\nabla f\) is \(\Real^n \to \Real^n\).

\section{Understanding total derivative}

\section{(Does not belong in this chapter)}

Let \( M \) be a \emph{point mass} with \emph{mass} \( m : \Real \) and \emph{velocity} \( v : \Real^n \).
Its \emph{momentum} is \( p = m \cdot v \).

% https://en.wikipedia.org/wiki/Mechanical_equilibrium
In statics, a system is in \emph{static equilibrium} iff it is at rest and it stays at rest.

The \emph{momentum} of an object is the difficulty of stopping it.

An object is \emph{at rest} iff its velocity is zero.
To \emph{stop} an object (to put it to \emph{rest}) is to make its velocity zero.
\emph{Stationary} is another word for \emph{at rest}.

    \chapter{Ending}

\section{Finding things to do after finishing this book}

\begin{itemize}
    \item Be an engineer. Build things.
    \item Be a teacher. Teach physics.
    \item Be a writer. Improve this book. Write about the latest advances in physics.
    \item Be a theoretical physicist. Solve unsolved physics problems.\footnote{\url{https://en.wikipedia.org/wiki/List_of_unsolved_problems_in_physics}}
\end{itemize}

Congratulations on making it this far.
We hope this book helps you reach your goals.
We wish you all the best.

    \appendix
    \part{Appendices}
    \chapter{Ordering of materials}

You don't have to read this chapter which
explains the reason for the ordering of materials in this book.

\section{Appetizer}

High school math.

Kinematics is discussed before the rest of mechanics.
Kinematics and dynamics motivate infinitesimal calculus and systems of differential equations.
It is more productive to think in infinitesimals because infinitesimals motivate differential geometry.

The historical ordering of materials is not always optimal for learning.

\section{Tensor}

Tensors have to be introduced early because all modern theories seem to use it.
Example theories are quantum field theories (quantum electrodynamics, quantum chromodynamics).

Thus we begin the book with relativity.

There are several routes to tensors:

\UnorderedList{
\item from differential geometry, via Riemann curvature tensor
\item from vectors and matrices by generalization as multi-dimensional array of numbers
\item from statics, via Cauchy stress tensor
}

\section{Electromagnetism}

Perhaps the easiest to electromagnetics is
from gravitostatics to electrostatics by analogy between mass and electric charge,
then to magnetostatics,
then to electrodynamics,
then to electromagnetism.

Another route is from hydrostatics to electrostatics by analogy flow-current pressure-voltage.

However, all such analogies break down when we have to explain electromagnetic radiation.

\section{Quantum mechanics}

There are several routes to quantum mechanics.

The historical way:
old quantum theory, de Broglie wavelength, Bohr hydrogen atom model, Heisenberg matrix mechanics.
This isn't good for teaching.
The old quantum theory isn't coherent.

From state spaces, quantum states, spins, qubits \cite{susskind2014quantum}.
The student can skip classical mechanics.

From wave functions, Schr\"odinger equation.
From studying the wave function of one-dimensional spinless particle.

From Hamiltonian mechanics, operator theory.

From Poisson brackets? Commutators?

From photoelectric effect, from black-body radiation.

The Wikipedia way\footnote{\url{https://en.m.wikipedia.org/wiki/Introduction_to_quantum_mechanics}}

Tensors are introduced before QFT because QFT is formulated in tensors.

    \chapter{Inventory}

This tries to standardize the physics skill set.

Every skill is a boolean.
A person either has the skill or does not have the skill.

\section{Measuring things}

\UnorderedList{
\item Reporting measurement with the proper significant digits and uncertainty.
\item Determining the uncertainty of a measuring instrument.
\item Using a ruler (graduated straightedge) to measure the distance between two points.
\item Using a protractor to measure the angle between two lines.
\item Using a thermometer to measure the temperature of a body.
}

\section{Working with numbers}

\UnorderedList{
\item Estimating the result, correct to one significant digit, of an arithmetic operation involving arbitrarily large numbers.
}

\section{Working with triangles}

\UnorderedList{
\item Using the Pythagorean theorem.
\item Using a computer to compute the value of a trigonometric function at a given input.
    The trigonometric functions are sine, cosine, tangent, inverse sine, inverse cosine, and inverse tangent.
}

    \chapter{Teaching physics}

CERN physics skill inventory\footnote{\url{http://ais.web.cern.ch/ais/apps/sti/examples_e.html}}

\section{Cognitive load}

It is important to minimize cognitive load.%
\footnote{\url{http://billblondeau.com/techwriter/good_technical_writing.htm}}
The cognitive load is minimum if it's something the student already knows.

Extraneous and germane cognitive load in psychology.
\footnote{\url{https://hobbolog.wordpress.com/2016/06/09/cognitive-load-theory-the-right-kind-of-load/}}
Similar to Fred Brooks' accidental and essential complexity in software.

\section{Exercises}

Exercises are vital for understanding.
Students must think for themselves.
They don't get anything by passive listening.

Exercises are checkpoints.

The teacher must make the student ask the question the teacher wants in the sequence the teacher wants.
The teacher must predict the student's question.
The book must predict the student's questions in the most intuitive sequence possible.
The book must cause the students to ask a question and answer it themselves.

The job of the teacher is
to plan a journey,
to stimulate the students,
to guide the students.

The readers can't ask the book if there is something they don't understand.
Therefore the book must handle all possibilities of the readers's failing to understand.

Study is doing the optimal sequence of \emph{exercises}.
The essence of teaching is \emph{crafting} such sequence.

\section{Introduce mathematics as we go}

It is easier for the student if the teacher
presents a concrete instance before the abstraction.

Avoid unused mathematics; don't just generalize a concept just because you can.
For example, in physics, vector spaces are mostly over \(\Real\) or \(\Complex\),
so we don't need to define vector space over a field (the algebraic structure).
We don't want the readers to take a long detour to abstract algebra.
We don't want to teach the readers anything that they are not likely to use.
We want to be as practical as possible, don't dumb down the material;
if the physics unavoidably requires mathematics in all its generality, then so be it.
We assume that the readers don't know mathematics, but want to learn mathematics,
but only want to learn mathematics that they will use.
We should not introduce mathematics that is not used anywhere else in this book.
We should not teach the wrong thing.
Where we diverge from mathematics, we shall indicate.

\section{How do we measure the student's mastery?}

Know? Able to use?

\section{What should we avoid?}

Avoid \enquote{popular} science books.
You are not their target audience.
The material is watered down.
They exist to entertain laypeople, not to teach researchers.

Ignore all books that have \emph{no} mathematics.

Ignore all books that have \emph{only} mathematics.

\section{How to teach yourself}

Learn how to learn.

Learn how to teach.

Alternate between being a teacher and being a student.

\section{People's ways of teaching}

\cite{kusse2010mathematical} motivates tensor by generalizing Ohm's law of electrical resistance
from discrete element to continuous medium.

\cite{scott2015student} is an exercise book for general relativity.
It's designed as a companion to \cite{schutz2009first}.

% Teaching general relativity to undergraduates
%http://physicstoday.scitation.org/doi/10.1063/PT.3.1605
%http://web.mit.edu/edbert/GR/gr1.pdf
%https://www.space.com/17661-theory-general-relativity.html

% Teaching special relativity
%https://manyworldstheory.com/2012/11/16/get-off-that-train-a-different-way-to-teach-special-relativity/

\section{Book reviews}

\subsection{Susskind\textendash{}Hrabovsky\textendash{}Friedman 2014 books}

Susskind and Friedman published \cite{susskind2014quantum}.
Susskind and Hrabovsky published \cite{susskind2014theoretical}.
Both books were published in 2014.

Leonard Susskind and George Hrabovsky's 2014 book
\emph{The theoretical minimum: what you need to know to start doing physics}
\cite{susskind2014theoretical}
begins by defining \emph{state spaces} and \emph{dynamical laws}.
Then it wastes at least five pages mathematicizing Aristotle.\footnote{\url{https://en.wikipedia.org/wiki/Chekhov\%27s_gun}}

It also teaches differential calculus, integral calculus, partial derivatives.

At page 90, it begins talking about \emph{phase spaces}.

What it does right is that it spreads lots of exercises throughout the book.

\section{What you need for what?}

If you want to build a nuclear fusion power plant, study nuclear physics.
What is nuclear physics?

What is energy? Ability to do work?

\section{Mathematical notation}

If \(f : \Real \to \Real\), then don't write \enquote{the function \(f(x)\)}
The function is \(f\).
The result of applying \(f\) to \(x\) is \(f(x)\).

\section{Physics curriculum around the world}

% https://en.wikipedia.org/wiki/Advanced_Placement
% https://en.wikipedia.org/wiki/AP_Physics_1
% https://en.wikipedia.org/wiki/AP_Physics_2

The USA have \emph{Advanced Placement} (AP).
High-school students can take AP exams to earn college credits while they are still in high school.

MIT? Harvard? Stanford?

\section{Indonesian}

\footnote{\url{https://www.itb.ac.id/news/read/5164/home/dosen-itb-luncurkan-buku-fisika-dasar-siap-unduh}}%
\footnote{\url{https://drive.google.com/file/d/0B3b8pBt2LxtWSkhCeC1nWmNXNFE/view}}

\section{Opinions}

On teaching mathematics, V. I. Arnold,
translated by A. V. Goryunov.%
\footnote{\url{https://www.uni-muenster.de/Physik.TP/~munsteg/arnold.html}}

The evolution of mathematics in the 20th century,
Michael Atiyah.%
\footnote{\url{https://web.math.rochester.edu/people/faculty/cmlr/Advice-Files/Atiyah-Mathematics.pdf}}

\section{Theory of learning}

\footnote{\url{https://en.wikipedia.org/wiki/Instructional_scaffolding}}%
\footnote{\url{https://en.wikipedia.org/wiki/Zone_of_proximal_development}}

    \chapter{Ramble}

\section{Books}

Study books:

Lev Okun's 2012 \emph{ABC of physics: a very brief guide} \cite{okun2012abc} has less than 200 pages.
The \enquote{formulas selected for this book are so simple that the knowledge of
elementary mathematics taught at high schools is sufficient
for understanding them} \cite[p.~vi]{okun2012abc}.
However, some parts are clear only if known.

\emph{History and evolution of concepts in physics}, 2014, Harry Varvoglis \cite{varvoglis2014history}.
With history, we can appreciate the work of others.
History explains why things are the way they are.
% It also has slides.
% http://www.tat.physik.uni-tuebingen.de/~varvoglis/

Thick books:

% http://www.lightandmatter.com/area1sn.html
% http://www.lightandmatter.com/books.html
\emph{Simple nature}, by Benjamin Cromwell.
This is a free book.

\emph{Fundamentals of physics extended}
(10th edition, by Halliday, Resnick, and Walker)
\cite{halliday2013fundamentals}

Book about physics experiments and experimental physics?

Landau\textendash{}Lifshitz books?

Reference books:

% http://202.38.64.11/~jmy/documents/ebooks/Hassani%20Mathematical%20Physics%20A%20Modem%20Introduction%20to%20Its%20Foundations%20-%20S.%20Hassani%20%5B0-387-98579-4%5D.pdf
The book \emph{Mathematical physics: a modern introduction to its foundations} (2013, Sadri Hassani)
\cite{hassani2013mathematical}
is for refreshing yourself about mathematical concepts that you have understood but you have forgotten,
not for studying physics from scratch.

\section{Unprocessed online resources}

Introduction to theoretical physics%
\footnote{\url{https://en.wikibooks.org/wiki/Introduction_to_Theoretical_Physics}}

Timeline of fundamental physics discoveries.%
\footnote{\url{https://en.wikipedia.org/wiki/Timeline_of_fundamental_physics_discoveries}}

Gerard 't Hooft has some advices on what to study to be a good theoretical physicist.%
\footnote{\url{http://www.staff.science.uu.nl/~gadda001/goodtheorist/index.html}}

% Landau\textendash{}Lifshitz book?
% https://www.reddit.com/r/Physics/comments/1dmxq7/our_beloved_landaulifshitz_books_are_available/
% https://archive.org/details/QuantumMechanics_104

% https://en.wikipedia.org/wiki/List_of_experiments#Physics

Quantum field theory%
\footnote{\url{http://www.thphys.uni-heidelberg.de/~weigand/QFT2-14/SkriptQFT2.pdf}}%
\footnote{\url{ftp://ftp.theorie.physik.uni-goettingen.de/pub/papers/rehren/qft13.pdf}}

Table of contents for College physics : a strategic approach / Randall D. Knight, Brian Jones, Stuart Field.
almost optimal order of materials
% http://catdir.loc.gov/catdir/toc/ecip072/2006032583.html

'big ideas'
% http://assets.pearsonschoolapps.com/asset_mgr/current/201412/pdf_145323.pdf

Wikipedia physics portal%
\footnote{\url{https://en.wikipedia.org/wiki/Portal:Physics/Navigation}}









Are the books recommended here concise enough?%
\footnote{\url{https://www.susanjfowler.com/blog/2016/8/13/so-you-want-to-learn-physics}}

philosophy of learning
amount learned is proportional to time put in
best way to learn is to figure out ideas yourself or teach them to someone else
the object of  a lecture is not so much to inform you of important facts, but rather to stimulate you to try to learn about some concept
\footnote{\url{https://ocw.mit.edu/ans7870/18/18.013a/textbook/chapter01/section01.html}}


Biochemistry: Wohler synthesized urea.

% https://en.wikipedia.org/wiki/Timeline_of_physical_chemistry
1869 Mendeleev periodic table

Alternating current flowing in wire emits EM radiation.






Constraint force acting on a ball rolling down a tilted plane.

Does powder/grain (such as sand) behave like an incompressible fluid? I guess they should. If they do, then liquid is a collection of tiny solids.

U-tubes

Stevin's law

% https://en.m.wikipedia.org/wiki/Communicating_vessels






Strong force keeps the atomic nucleus together. Electrostatic repulsion.


Constraint force

Every constraint force does zero work?

Example: The tension of pendulum string. Door hinge. Normal force on a box sliding down a tilted plane.

Related are the law of the lever and the conservation of energy.

The work done by \(F_k\) is \(W_k = F_k s_k\) where \(s_k = r_k da_k\).
W1 = W2
F1 r1 da1 = F2 r2 da2
F1 r1 da1 = F2 r2 da2



% https://physics.stackexchange.com/questions/12435/einsteins-postulates-leftrightarrow-minkowski-space-for-a-layman/13621#13621
\cite{dyson1972missed}



A Radically Modern Approach to Introductory Physics
% http://kestrel.nmt.edu/~raymond/books/radphys/book1/book1.html#x1-390004.1


physics compendium
Compendium of theoretical physics
% http://202.38.64.11/~jmy/documents/ebooks/Compendium%20of%20Theoretical%20Physics,Springer2005,529p,0387257993.pdf

concepts of modern physics
% http://web.pdx.edu/~pmoeck/lectures/beiser%206.pdf

Introduction to the Basic Concepts of Modern Physics
% http://www.springer.com/la/book/9788847006072


1910 Rutherford scattering:
Atom is mostly empty space?

hacking the quantum:
how anyone can become an amateur quantum physicist
% https://blogs.scientificamerican.com/critical-opalescence/hacking-the-quantum-a-new-book-explains-how-anyone-can-become-a-amateur-quantum-physicist/

% https://www.quora.com/What-is-the-best-way-to-self-teach-physics

% https://en.wikipedia.org/wiki/Double-slit_experiment

modern physics
% https://cnx.org/contents/rydUIGBQ@5.1:_dS0E2kQ@2/Introduction

openstax university physics 1, 2, 3%
\footnote{\url{https://cnx.org/contents/1Q9uMg_a@6.3:Gofkr9Oy@9/Preface}}%
\footnote{\url{https://cnx.org/contents/eg-XcBxE@4.1:Gofkr9Oy@9/Preface}}%
\footnote{\url{https://cnx.org/contents/rydUIGBQ@5.1:Gofkr9Oy@9/Preface}}

thermostatics%
\footnote{\url{http://www.ueltschi.org/teaching/chapthermostatics.pdf}}

\section{Discarded sources}

History of physics (abandoned work?)%
\footnote{\url{http://www.historyworld.net/wrldhis/PlainTextHistories.asp?groupid=2506&HistoryID=ac25&gtrack=pthc}}

Not useful, too wordy, no mathematics:%
\footnote{\url{http://lesswrong.com/lw/r5/the_quantum_physics_sequence/}}

\section{Ramble}

History of concepts%
\footnote{\url{http://www.springer.com/us/book/9783319042916}}

Understanding energy%
\footnote{\url{https://en.wikipedia.org/wiki/History_of_energy}}%
\footnote{\url{https://en.wikipedia.org/wiki/Conservation_of_Energy\#historical_information}}%
\footnote{\url{https://hsm.stackexchange.com/questions/414/when-were-the-modern-notions-of-work-and-energy-created}}%
\footnote{\url{https://en.wikipedia.org/wiki/Theory_of_heat}}

Synthetic physics?
Physics from first principles?

("What If Quantum Theory Violates All Mathematics?")%
\footnote{\url{https://www.degruyter.com/view/j/phys.2017.15.issue-1/phys-2017-0069/phys-2017-0069.xml?format=INT}}
What are the Bell inequalities?
Why are they famous?

The principle of relative locality%
\footnote{\url{https://arxiv.org/PS_cache/arxiv/pdf/1101/1101.0931v2.pdf}}
What is this?

\section{Irrelevant information}

% https://en.wikipedia.org/wiki/Mousetrap_car
% https://en.wikipedia.org/wiki/Torsion_spring
Torsion spring, angular form of Hooke's law

% https://en.wikipedia.org/wiki/Screw_theory

\section{Professions and their concerns}

\subsection{Mathematicians}

\subsection{Physicists}

Physicists aim to find out the truth.

\subsection{Engineers}

Engineers cares about models that allow them to do their work with acceptable error.

Engineers apply knowledge to solve practical problems.

\section{Civil construction}

\section{Burying water pipes and electrical cables}

Why do we bury pipes?
To avoid freezing.

However, in Jakarta, why do we bury pipes?
There is no reason.
Thus pipes should, for easier maintenance.
My house had an unknown pipe leak.
To trace that, one would have to dig 2 meters below the ground.
One would have to demolish all the floor.

Thus, in tropical climates, pipes should be above the ground.

The same logic goes for cables.
Cables should go below the ceiling, above the ground, and on the wall, not in the wall.
Don't bury cable in the wall.

If you mind the sight, use a portable temporary wall.

\section{Consciousness}

Consciousness requires memory.

Consciousness requires feedback.

I move my hand, and I see my hand move,
therefore I infer that I can control my hand,
and therefore I infer that my hand is part of my self.

My \emph{self} is \emph{everything} I can control.

If another person \(B\) absolutely obeys my orders,
then \(B\) is a part of my self.

I know the extent of my self by experimenting.
I find out what I can control and what I can't.
The part I can control, I call my \emph{self}.

The self of \(A\) is everything that \(A\) can directly control.
If \(A\) is driving a car, the car becomes part of \(A\)'s self, until he stops driving.

\section{Digression: law of physics}

A \emph{law of physics} is a mathematical statement.
\enquote{The laws of physics are the same in all frames} means that the statement has the same forms.
For example, \(F = m a\) has the shape \(F' = m' a'\) in another inertial frame.

Two people can see the same thing from different point of view.
What does the other person see?
Where and when are things, from another point of view?

\section{Structure}

Each section title must be the goal of that section.

Bloom's taxonomy of learning\footnote{\url{http://edutechwiki.unige.ch/en/Learning_level\#Blooms_taxonomy}}

\section{Other resources on the Internet}

If an Internet resource (such as Wikipedia)
happens to teaching something well,
then we should refer to it instead of duplicating the effort.

outline of physics\footnote{\url{https://en.m.wikipedia.org/wiki/Outline_of_physics}}

Theoretical physics\footnote{\url{https://en.wikipedia.org/wiki/Theoretical_physics}}
\footnote{\url{https://en.m.wikipedia.org/wiki/Conceptual_physics}}

\Hyperlink{https://en.wikipedia.org/wiki/Physicist}{Physicist, career, award}

\Hyperlink{https://en.wikipedia.org/wiki/Branches_of_physics}{Branches of physics}

Forget about prizes.
Science is a collaboration, not a competition.
The only way to win the prize is to win the opinion of the judges,
and you don't know the judges.

What do physicists think about nature in 2017?

How can we contribute to physics?

    % shared:
% mathematics
% physics

\chapter{Writing}

\epigraph{It seems that perfection is attained not when there is nothing more to add, but when there is nothing more to remove.}{Antoine de Saint-Exup\'ery (1900\textendash{}1944), \emph{L'Avion}}

Writing is a scalable way to spread your ideas.

\section{Reasons for reading fiction}

In fiction, you can pretend that you are someone else.

Fiction helps you imagine.

Fiction helps you escape reality.

\section{Deletion}

Every word must be important.
Use the fewest words to influence the reader.

\section{Respect the reader}

The highest honor for a writer is the reader's attention.

\section{Internet}

Once it's on the Internet, it's there forever.
You can't delete it.

Don't write things that can be used against you.

\section{How to write a book}

Write things.
Order things.
Delete as many as possible.
Repeat.

Alternate between focusedly-dip-deep-into-a-section and be-a-generalist-and-read-a-chapter.

A book begins as a bunch of small isolated islands of text containing main ideas and structural drafts.

Write.
Make mistakes.
Rewrite.

\section{Form}

The first sentence of a paragraph makes claim.
The rest of the paragraph supports the claim.
Thus, the reader can skim by reading the first sentence of every paragraph.

\emph{Ordering}:
If understanding \(B\) depends on understanding \(A\),
then \(A\) must come before \(B\) in the text.
Don't make the reader jump around.

Define uncommon term before using it.

\emph{Grouping}:
Related text should be near.

\emph{Non-redundancy}:
Never waste what the reader has read.

\emph{Clarity}:
If the reader reads the book linearly,
then he/she should understand a sentence just by reading it once.
The reader should parse the sentence in one pass.
Avoid ambiguous syntax.

% https://en.wikipedia.org/wiki/Garden_path_sentence
Avoid \emph{garden path sentences}.
Readers must parse a sentence before they understand it.
Understanding is hard.
Parsing adds unnecessary difficulty.

\section{Why a book, not a website?}

A book has pages.
Pages help readers estimate the amount of material.
Pages help readers estimate the amount of time they have to invest.
Pages are significant visual cue.
Online text seems endless.
Links distract.

\section{Academic publishing reform}

Do not let Aaron Swartz die in vain.

% https://en.wikipedia.org/wiki/Aaron_Swartz#Open_Access
% https://en.wikisource.org/wiki/Guerilla_Open_Access_Manifesto
% https://archive.org/stream/GuerillaOpenAccessManifesto/Goamjuly2008_djvu.txt

% https://en.wikipedia.org/wiki/Library_Genesis

% https://en.wikipedia.org/wiki/ICanHazPDF

% https://en.wikipedia.org/wiki/Sci-Hub
% https://vk.com/sci_hub

% https://en.wikipedia.org/wiki/Academic_journal_publishing_reform

% https://en.wikipedia.org/wiki/Library.nu

\section{Open-access journals}

% http://www.mdpi.com/journal/universe

% https://benthamopen.com/PHY/home/

After reading \cite{gopen1990science}, we conclude:
The subjects of the sentences in a paragraph should match the paragraph's topic.
IF passivating a sentence would match its subject to the topic of the containing paragraph, then passivate it.
Don't mindlessly activate all sentences.

There is a summary%
\footnote{\url{https://www.crowl.org/Lawrence/writing/GopenSwan90.html}}.

    \chapter{Philosophy}

\section{Thinking as a physicist (What are we trying to say here?)}

Knowledge is a tower.
It is built from the bottom upward.
Concepts stand on other concepts.
The bottom of that tower is everything we know from our senses.

All models are wrong, but some are useful. (George E. P. Box)

% http://chem.tufts.edu/answersinscience/relativityofwrong.htm
% https://en.wikipedia.org/wiki/The_Relativity_of_Wrong
% https://en.wikipedia.org/wiki/Wronger_than_wrong
Asimov's relativity of wrong:
Both round earth theory and flat earth theory are wrong,
but believing that they are equally wrong
is wronger than both of them combined.

Popper's falsifiability (testability)?

\emph{Abductive reasoning}:
We know from experiment that \(B\), \(C\), and \(D\).
Find a theory \(A\) that explains them.

% https://en.wikipedia.org/wiki/Problem_of_induction
% https://en.wikipedia.org/wiki/Uniformitarianism
\index{laws!uniformity of nature}%
We assume the \emph{principle of the uniformity of nature}:
The laws of nature is the same everywhere everytime \cite{hume1793inquiry}.

\section{Ontology and epistemology (What are we trying to say here?)}

\emph{Ontology} is about what exists in a domain of discourse.
Ontology defines objects, their properties, and their relationships.
\emph{Object} is another word for \emph{thing}.
\emph{Entity} is distinguishable object.
An entity has identity.
An entity can be distinguished from other entities.

\emph{Epistemology} is about how we know something.
In physics, we know things from experiments and inference.
% http://www2.phy.ilstu.edu/pte/publications/scientific_epistemology.pdf
\cite{wenning2009scientific} explains scientific epistemology (about 15 pages).
% https://en.wikipedia.org/wiki/Scientific_method
The \emph{scientific method}?

How can something be said to exist?

% An Ontology for Engineering Mathematics
% http://www-ksl.stanford.edu/knowledge-sharing/papers/engmath.html

% https://www.technologyreview.com/s/429561/the-measurement-that-would-reveal-the-universe-as-a-computer-simulation/
% ontological problem
If we write a computer simulation, then there \emph{is} an absolute position of space.
If we believe that we are a simulation, then there should be a special origin.

\section{Question, epistemology, ramble}

How do we answer ``What is \(A\)?''?

What is a question?
What are its answers?%
\footnote{\url{https://plato.stanford.edu/entries/questions/}}

``\(A\) is \(B\)'' means that \(A\) is another word for \(B\).

``\(A\) is \(B\) that \(C\)''
means that every \(A\) is an instance of \(B\) which satisfies \(C\).
Example: ``a city car is a car designed for traveling inside a city''.
But ``designed for'' is intensional.
How do we know?
If I'm a bad designer, and I make a car I call a ``city car'',
and it fares poorly in a city, then can it be called a ``city car''?

But we can then ask, ``What is \(B\)?'',
but we can't go on forever.
We have to end at something that has to be accepted as is, as a definition, as an axiom.

\(A\) has \(B\) (example: a car has doors).
A big bus is a bus that is big (a bus whose size is great).

A description of reality is not reality.
A theory is a description of reality, a model of reality, a representation of reality.
If reality and theory disagree, then reality wins and theory must change.

To \emph{know} an object is to be familiar with that object.
To \emph{know} an object is to be able to \emph{use} that object.
To \emph{understand} is to \emph{know}.
To \emph{comprehend}.

If \(A\) is tangible (such as a thing described by a concrete noun), it is understandable by sensing (perception).
\emph{Tangible} does not mean \emph{exist}.
A ``red swan'' is tangible, but it doesn't have to exist.

If \(A\) is intangible, it is understandable by thinking (inference).

If \(A\) is a tangible verb (a verb that can be done by actuator (as opposed to sensor)), it is understandable by moving one's body.

If \(A\) is understandable, then ``\(B\) is \(A\)'' is understandable.

If \(A\) is understandable, then ``\(B\) is an \(A\)'' is understandable.

If \(A\) and \(C\) are understandable,
then ``\(B\) is an \(A\) that \(C\)'' (``\(B\) is a \(C\) \(A\)'') is understandable.
Not always the case, for example: \(B\) is an invisible color.
Both ``color'' and ``invisible'' are understandable,
but ``invisible color'' is not understandable due to contradiction in terms.
Thus, in order to be understandable, a statement must be logically satisfiable.

``Why did you do that?''
If the answer is \(A\), then if \(A\) were false, then he would not do that.

``How did you do that?''
If the answer is \(A\), then if we do \(A\), we will accomplish the same.

    \backmatter
    \clearpage
    \index{postulates|see{laws}}%
    \index{principles|see{laws}}%
    \index{law|seealso{law named after people}}%
    \index{law named after people|seealso{law}}%
    \index{radiant emittance!see{radiant exitance}}%
    \index{radiant flux!see{radiant power}}%
    \printindex
    \clearpage
    \bibliography{../../bib}{}
    \bibliographystyle{plain}
}

% http://www.damtp.cam.ac.uk/user/tong/teaching.html
% https://www.symmetrymagazine.org/article/january-2015/how-to-build-your-own-particle-detector
% http://www.fisika.ui.ac.id/en/akademik/kurikulum_
% http://www.fi.itb.ac.id/struktur-kurikulum-dan-silabus/

\end{document}
