\chapter{Mathematics 2}

This chapter contains skipped mathematical details.
Theoretical physicists may have to care.
Engineers don't have to.

\section{Deriving the Lorentz transformation}

\label{sec:derive-lorentz-transform}

\index{Lorentz transformation!derivation}%
We can derive the Lorentz transformation in many ways,
but I think the most elegant derivation is \cite{pal2003nothing}.
It does so without assuming the constancy of the speed of light.
You \emph{should} read that paper to understand special relativity.
It's very readable.
I think every physics text should be at least as readable as that.
Next, read \cite{gannett2007nothing} which improves that paper.

Other derivations%
\footnote{\url{https://en.wikipedia.org/wiki/Lorentz_transformation\#Derivation_of_the_group_of_Lorentz_transformations}}.

% From E = mc2 to the Lorentz transformations via the law of addition of relativistic % velocities
% http://iopscience.iop.org/article/10.1088/0143-0807/26/4/006

% Unified derivation of the Galileo and the Lorentz transformations
% http://iopscience.iop.org/article/10.1088/0143-0807/3/2/008

% Yet another unified derivation of galilean and Lorentz transformations
% http://iopscience.iop.org/article/10.1088/0143-0807/4/4/110

\section{Differential 1-form}

% practical introduction to differential forms
% https://www.cefns.nau.edu/~schulz/diff.pdf

% primer on differential forms
% https://arxiv.org/pdf/1206.3323.pdf

% mathematics for physics
% http://www.goldbart.gatech.edu/PostScript/MS_PG_book/bookmaster.pdf

In \( dx(t) = a(t) \cdot dt(t) \),
each of \(dx\), \(a\), and \(dt\) is a \emph{function of \(t\)}.
We lift multiplication from real numbers to function space \( f \cdot g = t \to f(t) \cdot g(t) \),
and we write \( dx = a \cdot dt \).

\section{Limit}

(Delete this. Use non-standard (infinitesimal) analysis instead.)

\index{definitions!limit}%
Practically,
the \emph{limit of \(E\) as \(h\) approaches \(a\)}, written \( \lim_{h \to a} E \),
is what the expression \( E \) converges to
if \(h\) is taken from a sequence that converges to \(a\).

\section{Integral}

\subsection{Measure of real interval}

Let \( a, b: \Real \).
Let \(a \le b\).

The notation \( [a,b] \) describes the set of all real numbers between \(a\) and \(b\), including both.
Formally, \( [a,b] = \{ x ~|~ a \le x ~\wedge~ x \le b \} \).

\index{definitions!measure}%
Its \emph{measure} is \( \mu([a,b]) = b - a \).

The measure of a set describes how big that set is.

Do not confuse \emph{measure} with \emph{cardinality} (how many elements a set has).

\subsection{Partitioning a set}

\index{definitions!partition (to partition a set)}%
To \emph{partition} a set is to divide it into disjoint subsets.
A partitioning of a set \(S\) into \(n\) partitions is \((S_1,\ldots,S_n)\) such that
every pair of partitions is disjoint and the union of all of partitions is \(S\).

\subsection{Riemann integral}

% FIXME: COIK (clear only if known)
\index{definitions!integral}%
Practically, the \emph{integral} of a function \(f\) in \(A\)
is the area between the curve of the function and the x-axis that coincides with \(A\).
Let \( f : \Real \to \Real \).
Partition \(A\) into \(n\) partitions \(A_1,\ldots,A_n\).
For each \(k\), pick one point \(a_k \in A_k\).
\index{definitions!Riemann sum}%
A \emph{Riemann sum} is \( S_n = \sum_{k=1}^n f(a_k) \cdot A_k \).
\index{definitions!Riemann integral}%
The \emph{Riemann integral of \(f\) in \(A\)} is
\( \int_A f = \lim_{n \to \infty} S_n \).

\section{Calculus}

\subsection{Antiderivative}

% https://en.wikipedia.org/wiki/Antiderivative
Let \( f = d(F) \).

Then \( f \) is \emph{the derivative} of \( F \).

\index{definitions!antiderivative}%
Then \( F \) is \emph{an antiderivative} of \(f\).

A function has many antiderivatives.
Let \(c\) be a constant.
If \(F\) is an antiderivative of \(f\), then so is \(F + c\).

Interpretation:
\( (d(f))(x) \) is the rate of change of \(f\) at \(x\).
\( (d(f))(x) \) is the slope of the tangent line of \(f\) at \(x\).

\section{Differential equation}
\label{sec:diff-eqn}

We overload the notation 0 to mean not only the number zero but also the constant function \( x \to 0 \).

\index{definitions!differential equation}%
A \emph{differential equation} is an equation that has the derivative operator \(d\).
Example: \( f + d(d(f)) = 0 \).
Let \(e\) be the base of natural logarithm.
Example: \( f(x) = e^x \) is one of the solutions of \( f = d(f) \).

\section{Calculus of variations?}

\subsection{Path, functional of a path, calculus of variations, Euler\textendash{}Lagrange equation}

\index{definitions!path}%
A \emph{path} in \( \Real^n \) can be described by a function \( f : \Real \to \Real^n \).
The path is then the set \( \{ f(a) ~|~ a \in \Real \} \), that is the \emph{image} of \(f\).
We overload notation.
\( (f+g)(x) = f(x) + g(x) \).

A \emph{functional} has type \( (\Real \to \Real^n) \to \Real \).

\section{Differential form, 1-form, 2-form}

% https://en.wikipedia.org/wiki/Differential_form
% prerequisite for:
% - sympletic manifold
% - K\"ahler manifold
% - Calabi\textendash{}Yau manifold

\section{Relating derivatives, tangent lines, best linear approximations}

\section{Generalizing vector spaces}

(We don't seem to need vector space over a field beyond \(\Complex\) in this book.
We don't need this level of generality in this book.
We should remove this paragraph.)
Let \(F\) be a field (in abstract algebra).
A \emph{vector space over \(F\)} is a space \( V \), a \emph{vector addition} operator \( + : V \to V \to V \),
a \emph{scalar multiplication} operator \( \cdot : F \to V \to V \), and the \emph{vector space axioms},
which can be seen in the \enquote{Definition} section in the Wikipedia article for \enquote{vector space}.
A \emph{vector} is a point in \( V \).
If \( F = \Real \), then \( V \) is called a \emph{real vector space}.
If \( V = \Real^n \), then \( V \) is called the \emph{\(n\)-dimensional real vector space}.
Do not confuse the field in physics (multivariate function) with the field in abstract algebra.

\section{Defining the derivative using infinitesimals}

\footnote{\url{https://en.wikipedia.org/wiki/Non-standard_calculus}}

\index{definitions!derivative (by infinitesimal)}%
\( d(f) = x \to st\left(\frac{f(x+\delta)-f(x)}{\delta}\right) \)

\section{Defining derivative using limit}

Let \( f : \Real \to \Real \).

\index{definitions!derivative}%
The \emph{derivative of \( f \) at \(x\)} is \( \dd(f,x) = \lim_{h \to 0} \frac{f(x+h) - f(x)}{h} \).

The type of \(\dd\) is \( (\Real \to \Real) \to (\Real \to \Real) \),
which means that \(\dd\) takes a function and gives a function.

\section{Understanding the derivative as the best linear approximation at a point}

\( f(x+h) \approx f(x) + h \cdot f'(x) \)

Using this, we can approximate \(\sqrt{x}\) near a point we know,
we can approximate \(\sin(x)\) and \(\exp(x)\) for small \(x\).

\section{Understanding partial derivatives}

% https://en.wikipedia.org/wiki/Automatic_differentiation
% Symbolic differentiation: https://en.wikipedia.org/wiki/Computer_algebra
% https://en.wikipedia.org/wiki/Numerical_differentiation

Let \( f : \Real^n \to \Real \).

\index{definitions!partial derivative}%
The \emph{partial derivative of \(f\) at \( x \) with respect to the \(k\)th input}
is \( d_k(f,x) = \lim_{h \to 0} \frac{f(x + h e_k) - f(x)}{h} \).

The type of \(d_k\) is \((\Real^n \to \Real) \to (\Real^n \to \Real)\),
which means that \( d_k \) takes a function and gives another function.

\section{Understanding the gradient}

Let \(f : \Real^n \to \Real\).

\index{definitions!gradient}%
The \emph{gradient of \(f\) at \(x\)} is the vector \( (\nabla f)(x) \) whose \(k\)th component is \( ((\nabla f)(x))_k = d_k(f,x) \).
Note that \( (\nabla f)(x) \) is a vector in \( \Real^n \) and \( (\nabla f) : \Real^n \to \Real^n \).

The type of \(\nabla f\) is \(\Real^n \to \Real^n\).

\section{Understanding total derivative}

\section{(Does not belong in this chapter)}

Let \( M \) be a \emph{point mass} with \emph{mass} \( m : \Real \) and \emph{velocity} \( v : \Real^n \).
Its \emph{momentum} is \( p = m \cdot v \).

% https://en.wikipedia.org/wiki/Mechanical_equilibrium
In statics, a system is in \emph{static equilibrium} iff it is at rest and it stays at rest.

The \emph{momentum} of an object is the difficulty of stopping it.

An object is \emph{at rest} iff its velocity is zero.
To \emph{stop} an object (to put it to \emph{rest}) is to make its velocity zero.
\emph{Stationary} is another word for \emph{at rest}.
