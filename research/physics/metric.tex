\chapter{Metrics}

A metric is a function that measures the distance between two points.

Let \(S\) be a space.

Let \(d : S \to S \to \Real\) be a metric of \(S\).

Thus \(d(x,y)\) is the distance from \(x\) to \(y\).

Thus \(d\) should satisfy \(d(x,y) = d(y,x)\).

\section*{Euclidean metrics of real tuple spaces}

Let \(d : \Real^n \to \Real^n \to \Real\).
Define \(d(x,y) = \sqrt{\sum_{k=1}^n (x_k - y_k)^2}\).
Then \(d\) we say that \(d\) an Euclidean metric of \(\Real^n\).

\section{Norms and metrics induce each other}

A norm induces a metric.
A metric induces a norm.%
\footnote{\url{https://en.wikipedia.org/wiki/Metric_(mathematics)\#Metrics_on_vector_spaces}}

\section{Geodesics}

The distance between two points is the length of the shortest curve that connects them.

\section{Measuring distances on curved spaces}

Example: circle

Draw a circle with center \(C\).
Draw two points \(P\) and \(Q\) on the circle's perimeter.
Then \(CPQ\) forms a sector whose arc is \(PQ\).
The distance from \(P\) to \(Q\) is that arc's length.

Every point the circle can be described by a real number \(a\).
This real number is the \enquote{angle} of the point.

\(d(a,b) = r \cdot (b-a)\).

TODO define a metric for a curve described by \(x : \Real \to \Real^2\).

A curve is in length-normal formulation (???)
iff \( L(t,u) = \abs{t-u} \).
Then the metric is \(d(x,y) = L(f^{-1}(x), f^{-1}(y))\).

We can also define.
\(d(a,b) = b-a\).

\footnote{\url{https://en.wikipedia.org/wiki/Intrinsic_metric}}

Draw a sphere with center \(C\).
Draw two points \(P\) and \(Q\) on the sphere.

\section{Generalizing shortest paths into geodesics}

\subsection{Understanding the shortest curve connecting two points}

The concept of geodesic arises naturally when we realize
that \emph{straightness} is defined in terms of \emph{distance}.
If we define a \emph{straight} line segment as the \emph{shortest path} connecting its endpoints,
then the concept of geodesics arises naturally by redefining what \emph{shortest} means.

In Euclidean geometry, the shortest path connecting two points is a line segment.
Remember that a line is a straight curve.

We can also write \emph{shortest} as \emph{having minimum distance},
to clarify the connection between straightness and distance.

A curve segment connecting two points is a \emph{geodesic}
iff the length of the curve segment is the distance between the two points.

If \(x\) is a point on a geodesic,
then every point \(x + h\) satisfying \(\norm{h} = d(x,x+h)\) is also on the geodesic.
Can this equation be used to derive the equation of a geodesic?
The definitions of the norm \(\norm{\cdot}\) and the distance \(d\) depend on the ambient space.

Thus geodesic is a generalization of straightness.
By \emph{straight}, we mean \emph{shortest} (having minimal length).
Every line is a geodesic.

A geodesic is a locally length-minimizing curve.%
\footnote{\url{http://mathworld.wolfram.com/Geodesic.html}}

In metric geometry, a geodesic is a curve which is everywhere locally a distance minimizer.%
\footnote{\url{https://en.wikipedia.org/wiki/Geodesic\#Metric_geometry}}

\subsection{Finding the equation of a geodesic}

If the curve is given by parametric equation,
then a geodesic equation can be obtained by minimizing the arc length of the curve?%
\footnote{\url{http://mathworld.wolfram.com/Geodesic.html}}

\section{Metric tensor}

We have always implicitly assumed that a space defines distance uniformly,
the same everywhere, that is, we have always assumed the isometry \(d(x+h,y+h) = d(x,y)\) for all \(h\).

\footnote{\url{https://en.wikipedia.org/wiki/Metric_tensor}}%
\footnote{\url{https://en.wikipedia.org/wiki/Arc_length\#Generalization_to_.28pseudo-.29Riemannian_manifolds}}

A metric tensor is a metric in tensor form?
A rank-\((a,b)\) tensor is a function with \(a+b\) parameters?

Dot product generalizes to inner product.

Typing rule:
if \( a, b : V \), then \( \langle a, b \rangle : \Real \).
An inner product is a function with type \( V \to V \to \Real \).

In Euclidean vector spaces,
\( \langle x, y \rangle = x \cdot y \).
The dot product is the inner product of Euclidean vector spaces.

An inner product defines orthogonality. Two vectors are orthogonal iff \( \langle x, y \rangle = 0 \).
An inner product also defines the length of a vector.
\[
\norm{x} = \sqrt{\langle x, x \rangle}
\]

Tissot's indicatrices visualize a metric tensor by scattering
many circles throughout a space
so that we can see how the space's metric tensor distorts them.
This method tells us how much our map is lying to us.%
\footnote{\url{https://en.wikipedia.org/wiki/Tissot\%27s_indicatrix}}
\disabled{}%vim indent bug

\( d(0,x) = \norm{x} \) only in Euclidean geometry?

Perhaps we should digress to mapmaking/cartography?

Motivate metric tensor using cartography?
The surface of the Earth is a sphere.
How do we project its surface to a flat paper?
How do we draw a map?

Inner product generalizes to metric tensor?

A metric tensor's type is also \( V \to V \to \Real \).

Metric on half circle \( \{ (r \cos t, r \sin t) ~|~ t \in [0,\pi] \} \).

\section{Wrong-but-useful is better than right-but-useless}

Ancient people believe that the Earth is a plate.

Most people in 2017 believe that the Earth is a ball.

The Earth is an oblate spheroid.
It's a slightly flattened sphere.
It's wider on the equator.

The Earth is irregular.

Note how the statements become more correct, less general, and less applicable.

Earth is a plate, ball, or spheroid, depending on what we are doing.
If we're driving a car, the Earth is a plate.
If we're trying to introduce differential geometry by analogy with the Earth, it's a ball.
If we are comparing the gravitational pull in many places,
then the Earth is a spheroid.
If we are trying to one-up someone else, then the Earth is irregular;
it has mountains and lakes.

We're abusing language.
What do we mean by \enquote{is}?

Wrong-but-useful is better than right-but-useless.
