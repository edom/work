\chapter{Frames}

(Wikipedia says frame (ordered basis) for what we say basis. Should we follow?)

\section{A frame is an origin point and a basis}

With a basis, we can describe every vector with tuples.

With a frame, we can describe every point with tuples.

A frame \(S\) is an origin point \( O \) and a basis \( E \).

We can describe a point \( P \) as \( O + OP \).
We then state \(OP\) as a linear combination of the vectors in \(E\).
Suppose that the linear combination is \(OP = x_1 e_1 + \ldots + x_n e_n\).
We call the tuple \( (x_1,\ldots,x_n) \) the \emph{\(S\)-coordinates} of \(P\).

\emph{Coordinate system} is another name for \emph{frame}.

\footnote{\url{https://en.wikipedia.org/wiki/Frame\#Mathematics}}

\section{Some coordinate systems}

A \emph{coordinate system} maps a coordinate tuple to a vector.

\subsection{The rectangular coordinate system}

Orthonormal basis.

\(R(x,y) = x e_1 + y e_2\).

\(R(x) = x_1 e_1 + x_2 e_2\).

In this system, the coordinates are the scalar coefficients in the linear combination of basis vectors.
The coordinates describe how the basis vectors should be linearly combined to form the described vector.

Let \(T : V^2 \to V^2\) be a linear transformation.
Then \(T(R(x)) = T(x_1 e_1 + x_2 e_2) = x_1 \cdot T(e_1) + x_2 \cdot T(e_2) = x_1 e_1' + x_2 e_2' = R'(x) \).

\subsection{The polar coordinate system}

\(P(r,t) = r e_1 \text{ rotated } t \text{ radians counterclockwise}\).

\section{Coordinate system transformations}

\subsection{Converting polar coordinate tuples to rectangular coordinate tuples}

Both the rectangular coordinate $(r\cos\theta, r\sin\theta)$ and the polar coordinate $(r,\theta)$
describe the same point in two-dimensional Euclidean space.
\[
R(r\cos\theta, r\sin\theta) = P(r,\theta)
\]

A point in a space can have different coordinates in different coordinate systems.

\section{Using coordinate systems to apply analysis to geometry}

Coordinate systems bridge synthetic geometry and analytic geometry.

A coordinate system transformation is a function taking a coordinate system and giving another coordinate system.

\section{Describing vectors with numbers without drawing}

Earlier in \ChapterRef{chp:vector},
we defined a vector as something telling us
how to go from an origin point to a destination point.

We can define three directions: right, forward, and up.
We can describe \emph{any} point in space by saying that the point is
\(x\) meters right, \(y\) meters forward, and \(z\) meters up from where we are standing.
We have just described a \emph{coordinate system}:
a mapping between a \emph{coordinate tuple} and a point in space.

A tuple and a coordinate system can represent a vector.

\( v = x_1 e_1 + x_2 e_2 \).

Describing a vector using bases:

\[
v = \sum_k x_k e_k
\]
where each \(x_k:\Real\) and each \(e_k:\Real^n\).

For example:

Let \( e_1 \) be a unit vector pointing right.

Let \( e_2 \) be a unit vector pointing up.

Then, \( 1 e_1 + 2 e_2 \) is a vector.

"The numbers in the list depend on the choice of coordinate system."%
\footnote{\url{https://en.wikipedia.org/wiki/Covariance_and_contravariance_of_vectors}}

\subsection{Basis vectors and coordinate axes}

An orthonormal basis makes good coordinate axes.%
\footnote{\url{https://en.wikipedia.org/wiki/Orthonormal_basis}}

The same point has different coordinates in different frames.

\section{Coordinate systems}

We can describe a point by a \emph{coordinate system} and a \emph{coordinate tuple}.

\index{definitions!coordinate}%
A \emph{coordinate} is a number.

\index{definitions!coordinate tuple}%
A \emph{coordinate tuple} is a tuple of \emph{coordinates}.
Example: \((1,2)\).

\index{definitions!coordinate system}%
A \emph{coordinate system} maps a coordinate tuple to a point in a space.

\index{definitions!\(n\)-dimensional real coordinate space}%
The \emph{n-dimensional real coordinate space} \( \Real^n \)
is the set of all \(n\)-tuples where every component is a real number.
It can \emph{represent} the \(n\)-dimensional Euclidean space.

The \(n\)-dimensional Cartesian coordinate system uses \(n\) basis vectors
to map \( \Real^n \) to \(n\)-dimensional Euclidean space.

\subsection{Drawing the standard two-dimensional Cartesian coordinate axes}

\subsection{Transforming the vector}

Also known as \emph{active transformation}.
This changes the vector.

We write \(v = E(x)\) to mean that \emph{the vector \(v\) is described by the coordinate tuple \(x\) under coordinate system \(E\)}.

Let \(v = E(x)\).
The vector \(v\) is described by the tuple \(x\) under coordinate system \(E\).

If we transform the vector to \(T(v)\),
then we will also transform the coordinate tuple to \(T(E(x))\),
which we can also write as \((T \circ E)(x)\) using function composition notation.

Thus, if \(v = E(x)\), then \( T(v) = (T \circ E)(x) \).

Therefore, transforming the vector by \(T\) has the same effect as
transforming the coordinate system (not the coordinate tuple) by \(T\).

\subsection{Transforming the basis}

\emph{Passive transformation}.
Changing the basis while still referring to the same point.
This does not change the vector.

Every vector can be stated as a linear combination of the basis vectors.

\index{definitions!coordinate transformation}%
A \emph{coordinate transformation} is a map from a coordinate system to a coordinate system.

There are many points in an Euclidean space, but none is special.
However, we can pick a point in Euclidean space, and call it \( O \), short for \emph{origin}.

We can describe the same vector using different coordinate tuples in different coordinate systems.
For example, the same vector \(v\) has coordinate tuples \(x\) in coordinate system \(E\)
and has coordinate tuples \(x'\) in coordinate system \(E'\).
\[
    v = E(x) = E'(x')
\]

\footnote{\url{https://en.wikipedia.org/wiki/Active_and_passive_transformation}}

Let \(T\) be a linear transformation.

If \(E'\) happens to be \(T \circ E\), then

\begin{align*}
    E(x) &= (T \circ E)(x') & \text{assumed}
    \\ E(x) &= T(E(x')) & \text{by definition of composition}
    \\ T^{-1}(E(x)) &= T^{-1}(T(E(x'))) & \text{applying \(T^{-1}\) to both sides}
    \\ (T^{-1}\circ E)(x) &= E(x')
\end{align*}

\subsection{Changing basis}

A basis transformation is a coordinate system transformation.
Such transformation has the type
$(\mathbb{R}^n \to \mathbb{R}^n) \to (\mathbb{R}^n \to \mathbb{R}^n)$.
Such basis transformation $E' = T~E$ can be written out
as $\vec{e}'_k = T_k~(\vec{e}_1, \ldots, \vec{e}_n)$,
which can be written out even more in the greatest detail as
\[
(\vec{e}'_k)_i = (T_k)_i~(\vec{e}_1, \ldots, \vec{e}_n)
\]
which are actually $n^2$ equations,
and each $(T_k)_i$ is a function that takes $n^2$ real numbers.
If the transformation is linear,
we can write the transformation as matrix multiplication:
\[
E' = ET
\]
which expands into $n$ equations each:
\[
\vec{e}_i' = \sum_{k=1}^n T_{ik} \vec{e}_k
\]
which, in turn, expands into $n$ equations again each:
\[
e_{ij}' = \sum_{k=1}^n T_{ik} \cdot e_{kj}
\]

Sometimes we want to use another basis to locate the same point.
That is, if we have a vector $x$ in basis $E$, and a vector $y$ in basis $TE$,
we want $Ex = TEy$.
We want to change the basis from $E$ to $TE$,
so old coordinate $x$ become $y$,
but we want $y$ in $TE$ such that $TEy$ is still $Ex$,
which we can state mathematically:
\begin{align*}
TEy &= Ex
\\ T^{-1}TEy &= T^{-1}Ex
\\ Ey &= T^{-1}Ex
\end{align*}
which says that the change of coordinate goes against the change of basis,
so we say that the vector $x$ is *contravariant* to the basis transformation $T$.
This means that if we move the origin of the coordinate system 3 units to the right,
then we will have to move the coordinates 3 units to the left
if we want the new coordinate to locate the same point.

What about covectors?
Covector $f$.
\[
f(TEy) = f(Ex)
\]
Due to the linearity of covectors:
\begin{align*}
f(Ex) &= f\left(\sum_i x_i \vec{e}_i\right) = \sum_i f(x_i \vec{e}_i)
\\ f(TEy) &= f\left(\sum_i y_i T\vec{e}_i \right) = \sum_i f(y_i T\vec{e}_i)
\\ \vec{e}'_k = \sum_i T_{ki} \vec{e}_i
\end{align*}

(todo show that covector is covariant)

\section{Coordinate transformation}

Let's say we have one space $S$,
and two coordinate systems $J : M \to S$ and $K : N \to S$.
A coordinate transformation from $J$ to $K$ (we name this transformation $T : M \to N$)
transforms a $J$-coordinate
to an $K$-coordinate describing the same point.
This transformation relates both coordinate systems as
$J~x = K~(T~x)$, which means that if $x$ is a coordinate in $J$, then $T~x$ is a coordinate in $K$ locating the same point.
We can also write the equation as $J = K \circ T$.

To consider each component of the coordinate separately,
we write
$T~(x_1,\ldots,x_m) = (y_1,\ldots,y_n)$
where
\begin{align*}
y_1 &= t_1~(x_1,\ldots,x_m)
\\ &\vdots
\\ y_n &= t_n~(x_1,\ldots,x_m).
\end{align*}
Thus we can think of $T$ as an $n$-tuple $(t_1,\ldots,t_n)$
whose each component $t_k$ takes an $m$-tuple.
We can extend the definition of function application so that it works on tuples:
\[
(t_1,\ldots,t_n)~\vec{x} = (t_1~\vec{x}, \ldots, t_n~\vec{x})
\]
but why would we?

If we define that, we can see how a change in a tuple component translates
to a change in the other tuple component.
\[
\frac{\partial y_i}{\partial x_j} = pd~j~t_i
\]

A space mapping maps the underlying space $M : R \to S$.

The transformation $T~x = 2\cdot x$ can be viewed as two things:
as a space transformation, it makes everything bigger;
as a coordinate transformation, it makes everything smaller.
Coordinate transformation works in reverse.
You can move the point, or you can change the coordinate.

Covariance.

Distance-preserving transformation.
$d~(T~x)~(T~y) = d~x~y$.

$T$-symmetry.
$f~(T~x) = T~(f~x)$?

Inverse transformation.
Composition of transformations.

We can use several coordinate systems on the same space.
Some coordinate systems are more convenient to work with.
To specify a place on Earth, we can use the
geographic coordinate system (latitudes and longitudes).
% https://en.wikipedia.org/wiki/Geographic_coordinate_system
A coordinate transformation does not move the point.

If a coordinate system is a bijection,
then it describes an isomorphism between
its coordinate space and the space it describes.
This means we can pick any of them we find most convenient,
and whatever works with it will work with the other.

Let's say we have two spaces $R$ and $S$.
$T : R \to S$.

Embedding and projection are mappings between spaces.
Embedding maps a lower-dimensional space to a higher-dimensional space;
projection maps a higher-dimensional space to a lower-dimensional space.

\subsection{Picking a frame: The standard basis vectors}

The standard basis vectors of $\mathbb{R}^3$
is $\{e_1,e_2,e_3\}$
where $e_1 = (1,0,0)$, $e_2 = (0,1,0)$, and $e_3 = (0,0,1)$,
which can also be written as the matrix
\[
    \Matrix{ e_1 & e_2 & e_3 }
    = \Matrix{ 1&0&0\\0&1&0\\0&0&1 }
\]
where each basis vector becomes a column in the matrix.

A coordinate system transformation multiplies the basis with the transformation matrix.

\subsection{Curvilinear coordinate system: polar coordinate system}

\section{Understanding a covector as a linear function}

Every linear endofunction of an $n$-dimensional vector space
can be written as a vector of $n$ of inner products.

\section{Understanding how cobasis is to covector as basis is to vector}

We can derive a basis for $V^*$ from a basis for $V$.
Let us write the basis for $V$ as \(E = \{ e_1, \ldots, e_n\}\).
We then define $G = \{ g_1, \ldots, g_n \}$ where \(g_k(x) = x \cdot e_k\).
If such $G$ also spans the entire $V^*$,
then $G$ is a basis for $V^*$,
and we call such $G$ the dual basis of $E$.

\section{Multiplying a covector and a vector}

The product between a covector $f : V^*$ and vector $\vec{x} : V$
is simply $f~\vec{x}$ (function application)?

\section{Eigenvalues, eigenvectors, and fixpoints of transformations? Why are we talking about this?}

We say that a point is a fixpoint (invariant)
of a transformation iff the transformation maps it to itself.
$f(x) = x$.
We can also say that $f$ is a fixtransform of $x$.

We can embed a plane on a sphere.

To describe an embedding,
we can use words, or we can use algebra:
we pick a coordinate system for each space,
and describe the embedding of the coordinates.
An embedding $E : R \to S$ is a mapping from a space $R$ to (a subset of) another space $S$.
Let $J : M \to R$ and $K : N \to S$ be coordinate systems for those spaces, respectively.
Let $T : M \to N$ be the coordinate transformation that corresponds to that embedding.
Let $x$ be a $J$-coordinate.
Then we have:
\[
K~(T~x) = E~(J~x)
\]
which can also be written
\[
K \circ T = E \circ J.
\]
Spherical coordinate system
use three: radius,: $(r,\theta,\phi)$. Here the radius is fixed so we can use $(\theta,\phi)$.

\section{Moving points around}

\subsection{Translating a point}

We can translate a point \(P\) by a vector \(v\).
The result is the point \(P + v\).

\subsection{Rotating a vector}

Rotation is always done with respect to the origin.
If you need otherwise, then translate, rotate, and inverse-translate.

Two-dimensional rotation is simple because the axis of rotation is a point.
Three-dimensional rotation is more complex because the axis of rotation is a line.

Let \(x\) be the vector that we want to rotate.

Let \(a\) be the axis of rotation.

Let \(n\) be orthogonal to \(a\).
Let \(a + n = x\).
Thus, \(a\), \(x\), and \(n\) form a right triangle, with \(x\) as the hypothenuse.

Let \(\theta\) be the angle of rotation.

Let \(A(x,y)\) be the angle from \(x\) to \(y\).

The result of rotating a vector \(x\) about the axis \(a\) by angle \(\theta\) is the vector \(y\).

\(\norm{x} = \norm{y}\)

\(A(a,x) = A(a,y)\).

\(A(n,x) + \theta \equiv A(n,y)\).

\subsection{Applying a rotation matrix to a vector}

This matrix%
\footnote{\url{https://en.wikipedia.org/wiki/Rotation_matrix}}
describes
two-dimensional rotation by a counterclockwise angle of \(\theta\):
\Formula{
    \Matrix{
        \cos\theta & -\sin\theta
        \\ \sin\theta & \cos\theta
    }.
}
Let that matrix be \(R\).
Let \(v\) be a vector.
Then \(Rv\) is \(v\) rotated by \(\theta\).

\section{Moving frames around without moving points}

\subsection{Locating the same point with different coordinate systems}

Example of coordinate transformation:
The same point in the same two-dimensional Euclidean space
is described by
both the polar coordinates \( (r,\theta) \)
and the rectangular coordinates \( (r \cos \theta, r \sin \theta) \).
The transformation is \( (r,\theta) \to (r \cos \theta, r \sin \theta) \).

% https://en.wikipedia.org/wiki/Real_coordinate_space

A \emph{coordinate system} $M : C \to S$ is a surjective mapping from
\emph{coordinate space} $C$ to \emph{target space} $S$.

A \emph{coordinate} is a point in \(C\).
The coordinate system tells us how to get to a point.

The \(n\)-dimensional real coordinate space is $\mathbb{R}^n$.
% https://en.wikipedia.org/wiki/Real_coordinate_space
It is also called the real $n$-space.
A point in the real $n$-space is an $n$-tuple of real numbers $(x_1,\ldots,x_n)$.

$(x,y)$ is the tuple of coordinates,
$x$ is the x-coordinate, and $y$ is the y-coordinate.

Coordinate systems unify geometry and
% https://en.wikipedia.org/wiki/Mathematical_analysis
mathematical analysis.
With coordinates,
we can solve geometric problems by
numbers, calculus, and algebra,
so that computers can
find the intersection of geometric objects
by solving the corresponding system of equations,
and find the size of a geometric object by solving the corresponding integral.

\section{Fields}

A field is a function that assigns something to each point in space.

A vector field is a function \( f : P \to V \).

We need two bases to coordinatefully describe a vector field.
One basis to turn \( P \) to \(\Real^m\).
Another basis to turn \(V\) to \(\Real^n\).

The coordinateful description of a field \(f\) under basis \(?,?\) is \(f_? : \Real^m \to \Real^n\).

A force field assigns a force vector to each point in space.

A scalar field assigns a scalar to each point in space.
An example of a scalar field is a temperature field.%
\footnote{\url{https://en.wikipedia.org/wiki/Temperature_gradient}}

Let \(T(P)\) be the temperature at point \(P\).
Because \(P = O + \sum_k x_k e_k\), we have \(T(O + \sum_k x_k e_k)\).

Mathematics and physics use the same term \enquote{field} to mean different things.

\footnote{\url{https://en.wikipedia.org/wiki/Field_(physics)}}
