\chapter{Functions}

\footnote{\url{https://en.wikipedia.org/wiki/Lambda_calculus}}%
\footnote{\url{https://en.wikipedia.org/wiki/Lambda_calculus_definition}}

\index{definitions!function}%
We can think of a function as a computer.

The type \( a \to b \) is inhabited by functions that take an inhabitant of \(a\) and give an inhabitant of \(b\).

A function \(f : a \to b\) maps an \(a\) to a \(b\).

The expression \(f(x)\) means the result of applying \(f\) to \(x\).

If \(x : a\), then \( f(x) : b \).

For example, if \(f(x) = x^2 + x + 1\), then \(f(2+3) = (2+3)^2 + (2+3) + 1\) by substituting \(x\) with \(2+3\).

Don't write \(f(x)\) to mean the function \(f\).
The function is \(f\).
The expression \(f(x)\) means the result of applying \(f\) to \(x\).

\paragraph{Defining functions}

An \emph{unnamed function expression} is written like \( x \to x^2 + 1 \).

Example application: the expression \( (x \to x + 1)(5) \) reduces to \( 5 + 1 \).

\paragraph{Composing functions}

We write \(f \circ g\) to mean \( x \to f(g(x)) \).

\paragraph{Applying functions}

\paragraph{The inverse of an invertible function}

If \(f(x) = y\), then \(f^{-1}(y) = x\),
but only if \(f\) does not map anything else to \(y\).
We say that \(f^{-1}\) is the inverse of \(f\).

An invertible function is a function whose inverse is also a function.

\((f \circ g)^{-1} \equiv g^{-1} \circ f^{-1}\)

\section{Function-returning functions}

\(x \to (y \to x+y)\).

\section{Currying: one parameter is enough}

A two-input function has a type like \( (a,b) \to c \).

\index{definitions!currying}%
\emph{Currying} is the transformation from \( f(x,y) \) to \( (f'(x))(y) \) (from \(f\) to \(x \to y \to f(x,y)\)).
\emph{Uncurrying} is the inverse of currying.

We conflate \( (f'(x))(y) \) and \( f(x,y) \).

We assume that every function takes one input.

Do not confuse:
\begin{itemize}
    \item a function that takes \(n\) inputs, and
    \item a function that takes \emph{one} input that is an \(n\)-tuple.
\end{itemize}

\section{Plotting the graph of a function}

If you type \verb@sin(x)@ into Google\footnote{\url{https://www.google.com/}},
it will plot the graph of \(y = \sin(x)\) for you.

You can also type \verb@y=sin(x)@ into Wolfram Alpha\footnote{\url{https://wolframalpha.com/}}.

You can use GNU Octave.

You can use Gnuplot.

\paragraph{Thinking of an operator as a function}

An infix operator is an operator that is placed between two things.
For example, the dot in \(a \cdot b\) is an infix operator.

An operator does not have to be a symbol.
It can be a letter.
It can be a complicated notation.
For example, \(\pdv{f}{x}\).

We can treat an infix operator as another way of writing a function application.
We can think of \(a \cdot b\) as a convenient way of writing \((\cdot)(a,b)\).

\footnote{\url{https://en.wikipedia.org/wiki/Operator_(mathematics)}}%
\footnote{\url{https://en.wikipedia.org/wiki/Operator_(physics)}}

\section{Equivalence of functions, eta-reduction}
\label{sec:function-equivalence}

We write \(f \equiv g\) to mean that \(f\) gives the same result as \(g\) for all parameters.

\section{Exercises}

To evaluate something is to reduce it to normal form.

\ExerciseAnswer{Evaluate the expression \((x \to x + 1)(5)\), assuming the usual arithmetic.}{Substitute: \(5 + 1\). Reduce: \(6\).}

\ShowAnswers
