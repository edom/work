\chapter{Lies}

\epigraph{
    [\textellipsis] The author feels that this technique of deliberate lying will actually make it easier for you to learn the ideas.
Once you understand a simple but false rule, it will not be hard to supplement that rule with its exceptions.
}{Donald Ervin Knuth, \emph{The \TeX{}book} (1984)}

This chapter describes the differences between this book and mainstream mathematics.

\section{Set theory}

\index{set theory}%
There are several set theories:
na\"ive set theory (which is practical but inconsistent),
Zermelo-Fraenkel (ZF),
Zermelo-Fraenkel with axiom of choice (ZFC),
Von Neumann-Bernays-G\"odel (NBG),
Morse-Kelley (MK),
Tarski-Grothendieck,
and others,
but all of them are built on logic.

\section{Probability space}

\index{sigma-algebra}%
Elsewhere, a probability space is \((\Omega,F,P)\) where \(F\) is a \emph{\(\sigma\)-algebra},
but here we simplify it to \((\Omega,P)\) with implied \(F = 2^\Omega\).

\section{Integral}

\index{integral}%
\index{measurable function}%
\index{function!measurable}%
An integral requires a \emph{measurable} function, not just any function.

\section{Turing machine}

There are as many definitions of Turing machines as there are textbooks.
