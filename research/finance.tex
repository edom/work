\chapter{Finance}

\section{Asset}

If \(x \in \Real\) then \(cash(x)\) is an asset.

If \(t \in \Real\) and \(x\) is an asset,
then \(delay(t,x)\) is an asset.

If \(p \in [0,1]\) and \(x\) is an asset,
then \(chance(p,x)\) is an asset.

See also \emph{Composing contracts: an adventure in financial engineering} \cite{SpjContract}.

\section{Value function}

Let \(V\) be the value function.

\(V(cash(x)) = x\).

\(V(chance(p,x)) = p \cdot V(x)\).

\(V(delay(t,x)) = V(x)/(1+r)^t\).

\section{Time preference}

Let \(r\) be the interest rate.
Let \(x\) be an amount.

The \emph{future value} of \(x\) at time \(t\) is \((1+r)^t x\).

The \emph{present value} of \(delay(t,x)\) is \(x / (1+r)^t\).

\section{Risk-free interest rate}

\emph{Your} risk-free interest rate is
the highest government-guaranteed interest you can get.

\section{Perpetuity}

A perpetuity is \(\sum_{t \in \Nat} delay(t,p) \).

Let \(r\) be the risk-free interest rate.
Let \(p\) be an installment.
Let all installments be the same.

The present value of such perpetuity is \(\frac{p}{r}\).

\section{No-arbitrage}
