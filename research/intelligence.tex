\chapter{Intelligence}

We can hide indexes by using the delay operator:
\[
    d~f~k = f~(k-1)
\]
We can rewrite
\( x_k + x_{k-1} = 0 \)
to
\(x + d~x = 0\).

\section{Relative intelligence measure}

Given a system, measure its intelligence relative to something.

We can measure with respect to a function.
The result of the function when applied to the system history
is the intelligence measure of the system.

\section{Intelligent system as a special case of second-order endofunctions on an infinite-dimensional real space}

Consider infinite-dimensional space \(A^\infty\).
It can be thought as a type whose inhabitants are infinite sequences.
Each element of each sequence is an inhabitant of \(A\).
An endofunction \(A^\infty \to A^\infty\) relates the input and output signals at certain time.
A second-order endofunction \((A^\infty \to A^\infty) \to (A^\infty \to A^\infty)\) is the architecture of the intelligent system;
it determines how the system changes itself.

\begin{center}
    \begin{tabular}{ll}
        thing & type
        \\
        \hline
        signal & \(A\)
        \\
        signals & \(A^\infty\)
        \\
        behavior & \(A^\infty \to A^\infty\)
    \end{tabular}
\end{center}

A vector in \(A^\infty\) represents all signals at a certain time.
There are input, output, and internal signals.
All of them are real numbers.

Alternative summation
\[
    N \left( \prod_i k_i \cdot f_i \right)
\]

\section{Other raw ideas}

Evolution of a system?

Relative measure of intelligence.
\(m\)-intelligence.
\(m : S \to \mathbb{R}\).

What about competitive learning? Selforganizing maps?

Multiplication is like and, addition is like or.

\subsection{Feedback}

An example discrete function with feedback is
\( f~(x+1) = x + \frac{1}{8} \cdot f~x \).
An example continuous function with feedback is
\( f~(x+h) = x + \frac{1}{8} \cdot f~x \cdot h \).
The discrete case is a difference equation.
The continuous case is a differential equation.

Example: linear interpolating latch,
generalization of level-triggered D latch:
\[
    f~w~x~(t+1) = (1 - w) \cdot f~0~x~t + w \cdot x
\]
\[
    f~w~x~0 = w \cdot x
\]
where \(w\) is the write enable signal
(a real number in the unit interval),
\(x\) is the current input,
and \(t\) is the discrete time.
\[
    y' = (1 - w) \cdot y + w \cdot x
\]

A function \(f\) with natural number argument
is an infinite sequence \(f(0), f(1), \ldots, f(n), \ldots\).

\section{Reward system and reinforcement learning}

In mathematical optimization there is something called
\emph{loss function}, also known as \emph{cost function}.

Learning must be driven by something;
otherwise a system won't know what to try.
Intelligence is function optimization.

Let \( r : A^\infty \to A \) be the \emph{reward function}.
This is intrinsic to the system.
This drives the system's behavior.
The system always behaves with the goal of maximizing the reward.
This function explains why a system behaves the way it does.
The reward function depends only on inputs and the internal state.

Consider the system at a given time.
We have input \(x\).
We have behavior function \(f\).
We have reward \(r = R~x\).
We hope that the machine
replaces the behavior function with \(f'\)
to increase the next reward.
How are \(f'\) and \(r\) related?

We make the machine assume that there is a correlation
between its output and its reward.
But isn't this gratuitous?
But if there isn't a causal relationship between the output and the input,
how can something learn at all?
Feedback is necessary but not sufficient for learning.

Learning is a special case of control systems in electrical engineering.

\section{A concrete architecture for growable neural networks on year-2016 machines}

The signal buffer is \(S\) and \(T\), an array, the buffered output of neurons.
Each neuron accepts an arbitrary number of scalars but outputs exactly one scalar.

Is a constant function intelligent? Is a complicated function intelligent?
Is a program that has one-billion lines of nontrivial code intelligent?

The weight buffer is \(W\).

A neuron has \(w\) (an array of weight indexes), \(x\) (an array of input indexes), and \(y\) (an output index).
Both \(w\) and \(x\) must have the same length.

In each iteration, for each neuron \((w,x,y)\),
we do \(T[y] := N \left( \sum_i W[w_i] \cdot S[x_i] \right) \),
and then swap the pointers \(S\) and \(T\).
The function \(N\) is the clamping function.

\section{Examples}

You can skip this section if you already understand what I'm talking about.

It seems that neurons encode weights using pulse frequency modulation? Citation?

Example of an infinite-dimensional vector:
\[
\begin{bmatrix}
1/1^2 & 1/2^2 & 1/3^2 & \ldots & 1/n^2 & \ldots
\end{bmatrix}
\]

Example of an endofunction:
\( f~x = y \) where \( y_k = x_k / k \).
\[
    f
    \begin{bmatrix}
        x_1 & x_2 & \ldots & x_n & \ldots
    \end{bmatrix}
    =
    \begin{bmatrix}
        x_1 / 1 & x_2 / 2 & \ldots & x_n / n & \ldots
    \end{bmatrix}
\]

Example of a second-order endofunction:
\[
    a~f = \sum_i k_i \cdot f_i
\]
where each \(f_i : A^\infty \to A^\infty\) is a function,
each \(k_i\) is a scalar, and \( \sum |k_i| < \infty \).

Another convolutional example of a second-order endofunction:
\[
    (a~f)_n = \sum_{i=0}^n f_{n-i} \cdot f_i
\]
where \(((f \cdot g)~x)_n = (f~x)_n \cdot (g~x)_n\).
We can define a vector elementwisely.

Consider \(A = \mathbb{R}\).

If we have a vector space \(A\),
we can make the function space \(A \to A\) a vector space for free,
like \(f + g = x \to f(x) + g(x)\).
If \(A\) is a vector space, then \(A \to A\) is also a vector space,
and then \((A \to A) \to (A \to A)\) is also a vector space.

Projectors are functions whose type is like \(A^\infty \to A\).
They are everywhere in machine learning.
They are topological maps.
All linear classifiers are projectors.
Perceptrons and support vector machines are linear classifiers.

In this chapter, we show how intelligence arise
from projectors that take other projectors.
\begin{align*}
    A^\infty &\to A^\infty
    \\
    (A^\infty \to A^\infty)^\infty &\to (A^\infty \to A^\infty)
\end{align*}

\section{Comparison between machine and human}

Perceptrons work by updating weight, but the structure doesn't change;
there are no new neurons.
Brain works by updating weight \emph{and} the number of signals;
the structure changes.
We have to make an artificial neural network that can grow.

Let a behavior function be an inhabitant of \(I^n \to I\).
In a human, \(n\) changes with time;
humans grow and have more senses as they age.
However, we can model this growth
using infinite-dimensional space whose
members have only finitely many nonzero components.
Instead of using a dependently-typed function \(f~t : I^{n~t} \to I\),
we use \(f~t : I^\infty \to I\),
and require that the input of \(f~t\) have at most \(n~t\) nonzero components.

The human sensory system can be thought as a subspace
of \([0,1]^\infty\) where each vector
is required to have at most \(n\) nonzero components,
where \(n\) is the number of sensors.
We don't know \(n\) exactly,
but it's a finite natural number,
although it may be rather large.

The primary difference between an artificial neural network and a human brain
is the artificial neural network does not change after time,
whereas the human brain always changes.
Once the weights of such network is set by training,
it never changes again until it's retrained.
The human brain has a \emph{hardwired reward function}
that changes the brain to maximize the reward.
This reward function is the dopamine level.
Therefore if we can formulate a suitable reward function,
we can make a seed AI.
But there is a branch that does this.
It's called \emph{lifelong machine learning}.

\section{Fractal appearance of intelligence}

Intelligence looks fractal.
The smallest unit of processing, the neuron, is a projector.
The whole behavior of a human is a projector.
Those neurons compose to form a bigger projector.

We can form a vector space from a function space \(A^\infty \to A\),
provided that we also define some operations on \(A\).
If you can define distance \(d_A\) on \(A\),
you can define distance on \(d_F\) like this:
\[
    d_F~f~g = \sum_{X \in P(A)} d_A~(f~x)~(g~x) \cdot |X|
\]
where \(x \in X\), \(P\) divides \(A\) into partitions,
and \(|X|\) is the size of the infinitesimal subspace \(X\).
In other words, \(P(A)\) is the set of all disjoint infinitesimal subsets of \(A\).
But isn't that just a Riemann sum?

\section{Architecture function}

Let \(A = I^\infty \to I\).
We can state the fractal nature of projectors explicitly:
\[
    f~x = a~(f_1,f_2,f_3,\ldots)~x
\]
where \(a : A^\infty \to A\) is the \emph{architecture function}
and each \(f_k : A\) is a component function,
and there are only finitely many nonzero components.
Observe that \emph{the architecture function itself is also a projector}.
It's a higher-order projector.

There are two fundamental subtypes:
bell (like normal distribution function) and sigmoid (like asinh),
but because we restrict ourself to unit interval,
we can use bells only.
A \emph{bell} is a function with exactly one maximum.
A unit interval projector can be approximated as a weighted combination of bells.
\[
    f~x = N \left( \sum_k w_k \cdot f_k~x \right)
\]
Each \(f_k : I^n \to I\) is a component.
The function \(N : \mathbb{R} \to I\) is the normalization function,
usually a sigmoid, usually asinh,
or even this ramp:
\[
    N~x =
    \begin{cases}
        -1 & : x \le -1
        \\
        x & : -1 \le x \le 1
        \\
        1 & : 1 \le x
    \end{cases}
\]

The problem of machine learning architecture
is \emph{restating \(f\) as a combination of simpler functions}
that is general enough so that it can navigate the real-world.
Indeed every \(f_k\) could be an unit interval projector,
and we have a hierarchy of projectors
like multilayer perceptrons.

We want it to be smooth.
If \(x\) and \(y\) are near,
then \(f~x\) and \(f~y\) should also be near.
As \(x\) gets closer to \(y\), so should \(f~x\) get closer to \(f~y\).

The function \(f\) can have any shape,
but most learning happens in something like a discrete Hilbert space, if there is such a thing.
For example, an image binary classifier is a function \(f : P^n \to B\)
where \(P\) is the pixel type and \(n\) is the number of pixels,
and the boundary between two classes is blurred but smooth
(not clear-cut, but not blotchy either).

Every binary classifier is a function \(I^n \to I\).
Both perceptron and support vector machine have the same type,
but different implementation details.
Truncating the output of a projector produces a binary classifier.

To build our intuition we can consider the case \(A \to A\) and \(A^2 \to A\).
The function \(A \to A\) can be graphed on paper.
We can also draw a heightmap for \(A^2 \to A\).
The graph of \(A\) can have many humps.

Intelligence is what is exhibited by a function of type \(A^\infty \to A\).
Summarization of information.

\section{The sigmoid dilemma}

An undesirable property of sigmoid is \(f~(f~|x|) < |x|\).
Every continuous sigmoid function has that problem.
The ramp function doesn't have this problem,
but its derivative has discontinuities.
