\chapter{Human as a feedback system}

\section{Human behavior as a special case of the general feedback equation}

Let \(x ~ t\) be the input vector at time \(t\);
this vector has at least some billions of elements.
The function \(x\) represents the state of all sensors at a given time.

Let \(y ~ t\) be the control vector at time \(t\);
this vector is also big.

Let \(z~t\) be the output vector at time \(t\).

The environment feeds back a part of the output to the input.
Can the agent determine the response function?

The feedback forms memory, but see ``Memory without feedback in a neural network''.
https://www.ncbi.nlm.nih.gov/pubmed/19249281

\section{Hardwiring the concept of time}

We can transform a non-temporal behavior \(f~x = y\) into a temporal behavior \(f'~t = y'\)?

\section{Life of one neuron?}

% http://biology.stackexchange.com/questions/5306/how-do-neurons-form-new-connections-in-brain-plasticity
