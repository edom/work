\chapter{Human as a feedback system}

\section{Human behavior as a special case of the general feedback equation}

Let \(x ~ t\) be the input vector at time \(t\);
this vector has at least some billions of elements.
The function \(x\) represents the state of all sensors at a given time.

Let \(y ~ t\) be the control vector at time \(t\);
this vector is also big.

Let \(z~t\) be the output vector at time \(t\).

The environment feeds back a part of the output to the input.
Can the agent determine the response function?

The feedback forms memory, but see ``Memory without feedback in a neural network''.
https://www.ncbi.nlm.nih.gov/pubmed/19249281

\section{Hardwiring the concept of time}

We can transform a non-temporal behavior \(f~x = y\) into a temporal behavior \(f'~t = y'\)?

\section{Life of one neuron?}

% http://biology.stackexchange.com/questions/5306/how-do-neurons-form-new-connections-in-brain-plasticity

\section{A brain at a given time is an array function.}

A brain at a given time is an array function
having type \(\Real^\infty \to \Real^\infty\).
Each component of the input array is a signal from a sensor.
Each component of the output array goes to an actuator.

Since the brain is finite,
there must be infinitely many zeros in the input and output arrays.

\section{An array iself is also a function.}

An \(E\)-array is a function having type \(\Nat \to E\).
The input is an index.
The output is the value of the component at that index.
Subscripting denotes function application.

\section{Each brain has a maximand.}

Such maximand is a hidden function.
The brain always tries to maximize the maximand.

A differential change in brain tries to increase the maximand.
The brain follows gradient.

\section{Consider functions of length-one arrays.}

Let \(h\) be a differential change in brain.

\section{Draft}

The only way to know whether the system has learning something
is by testing it with samples the system has never seen.

Practically all machine learning cases deal with functions
that is continuous enough to form a Hilbert space.

Every classification problem in the real world can be written as \(f : I^n \to I\) for an \(n : \Nat\).
Usually \(I\) is discrete.

Consider the case where \(I = [0,1]\).
Continuous map from \(I^n\) to \(I\).
Continuous map from a hyperplane to a line.

\section{How do we relate vector functions and intelligence?}

\section{How does feedback happen in the brain?}

Feedback is due to environment and the physical laws.
When we move our hand, we see it, because the light
reflected by our hand now reaches our eyes.

The next input depends on the previous input.
\begin{align*}
    y_k &= b~x_k
    \\
    x_{k+1} &= f~x_k~y_k
\end{align*}

\section{The brain is a recurrence relation.}

This pictures the brain as a parallel dataflow computer
with clock period of a few microseconds.

% https://en.wikipedia.org/wiki/Dataflow_architecture

Let \(m\) be memory, \(x\) be senses, and \(y\) be actuators.
\begin{align*}
    m_{t+1} &= f~x_t~m_t
    \\
    y_{t+1} &= g~x_t~m_t
\end{align*}

There is also a version with implicit time.
\begin{align*}
    m' &= f~x~m
    \\
    y' &= g~x~m
\end{align*}

There is also a continuous version.
\begin{align*}
    m_{t+h} &= h \cdot f~x_t~m_t
    \\
    y_{t+h} &= h \cdot g~x_t~m_t
\end{align*}

\section{The brain evolved from simpler nervous systems.}

Nervous systems are control systems.

Nervous systems must have provided some evolutionary benefit;
otherwise natural selection would have phased them out.

Bacterial chemotaxis detects chemical concentration difference.

Nematode.
Caenorhabditis elegans.
