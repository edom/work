\chapter{Temporal systems}

A temporal system, a time-dependent system, or a time-variant system is a system that depends on time.
With time, we can define more interesting systems.

\begin{m:def}[Temporal system]
    A temporal system is a function whose type is \((T \to X) \to T \to Y\).
\[
    \SysTmp~T~X~Y = (T \to X) \to (T \to Y)
\]
\end{m:def}

We can see a temporal system as a transformation of time functions.
\((T \to X) \to (T \to Y)\).

Example:
\(f~x~t = (x~t)^2\).

Example:
\[f~x~t = x~t + \int_0^t (s - f~x~t) \cdot dt\].

%wrong
\begin{m:def}[Temporal system]
    A temporal system is \((x,y,f,T)\) that satisfies
    \(\forall t \in T : f~y~x~t\).
\end{m:def}

\section{First-order system}

The previous section talks about second-order system.

This is a first-order system:
\(\SysTmp~T~X~Y = X \to T \to Y \).

\(\SysTmp~T~X~Y = T \to X \to Y \).

There are two points of view: \(d_x~f\) and \(d_t~f\).

First-order system should be more analyzable.

Continuous-time and discrete-time system?

In the above definition, \(T\) is the time type.
If \(T = \R\) we call the system continuous-time.
If \(T = \N\) we call the system discrete-time.

\section{Chaining temporal systems}

We can feed the output temporal system \(f\)
to the input of the temporal system \(g\)
this produces the temporal system \(h\) where
\(h~x~t = g~(f~x)~t\),
or \(h~x = g~(f~x)\) after eta-conversion,
or \(h = g \circ f\).
It turns out that system composition is just plain function composition.

\section{Stateless and stateful systems}

A system is stateless iff the same input always gives the same output.
There is no way to tell apart a system that has state
but doesn't use it and a system that really has no state.

\begin{m:def}[Stateless system]
    A stateless system is a temporal system that satisfies \(\forall t : \forall u: x~t = x~u \implies y~t = y~u\).
\end{m:def}

In a stateful system, the same input can give different outputs, depending on time.

Why do we define those?

\section{Property}

If \(p\) is a predicate that is always true for a system,
then \(p\) is a property of that system.
\( \SysTmp~T~X~Y \to \{0,1\} \)

\section{Constraint}

A constraint of \(S\) is a property of \(S\) that is always true.

\section{Parameter/family}

Parameterized system.

\( P \to \SysTmp~T~X~Y \)

System parameter.

Family of systems.

Indexed family of systems.

\section{Measure}

Categorical inverse of parameter. (Whatever categorical inverse means.)

From type theory point of view, parametrization is the inverse of measurement.

\( \SysTmp~T~X~Y \to M \)

\section{Temporal measure}

\( m : \SysTmp~T~X~Y \to (T \to M) \)

Find \(s\) that minimizes \(m~s~t\) as \(t\) grows.

\section{System space}

Like function space.
Metric space.

\section{System endofunction}

\( \SysTmp~T~X~Y \to \SysTmp~T~X~Y \).

\section{Output-input gradient}

\( f : \SysTmp~T~X~Y \)

\( f~(x+h) - f~x = h \cdot d~f~x \) but \(h\) is a function.

\(m\)-adaptivity

\( (m~f~(x+h) - m~f~x) / h \)

Reversal:
\( \SysTmp~T~X~Y \to \SysTmp~T~Y~X \)

Time-reversible/Time-symmetric:
\( f~x~t = f~x~(-t) \)

\section{Minimand}

The minimand is the thing that is to be minimized.
It's an English word.
The minimand of a temporal system is a function that is minimized as time goes by.

Recall that a temporal system has type \((T \to X) \to T \to Y\).
A minimand is a function that has type \((T \to X) \to T \to M\).

\begin{m:def}[A minimand of a temporal system]
    The function \(g\) is a minimand of a temporal system \(s\) iff \(g~s~t \xrightarrow[t \to \infty]{} 0\).
\end{m:def}

There's always a trivial minimand: \(g~s~t = 0\).

Does every system have a non-trivial minimand?

\section{Constrained system}

Constrained system: a system whose equation is subject to constraints (which can be inequalities).
Every system is constrained; the definition requires it. So why bother defining this?

\section{Optimizing system}

A system is \emph{optimizing} iff it optimizes a function.
We call this function a \emph{goal function}.
The purpose of the system is to minimize the goal function.

A goal is something that a system wants to reach.
This implies that the definition of goal involves time.
The goal function is usually hidden.

\section{Purposeful system}

We also call a purposeful system an optimizing system.

\emph{Purpose requires time.}

Let \(x\) be a function of time.
Let the equation \(f~x~t = y\) govern the system.
Let \(g~x\) be a function of time.
The system is \emph{purposeful} iff \(g~x~t\) approaches zero as \(t\) grows,
for some non-trivial \(g\).
We say that \(g\) is a \emph{purpose} or a \emph{goal} of the system.
The goal function may represent the sensed error
with respect to a setpoint.

A purposeful system doesn't have to be adaptive.
A simple thermostat is purposeful but not adaptive.

\section{How do we measure how well a system serves its purpose?}

is like measuring the rate of convergence of an approximation scheme.

\section{What is an intelligent system?}

Stable system:
See stability theory.
Lyapunov.

How do we measure how adaptive a system is?

An adaptive system is a system that adapts.

Adaptation implies change.

Adapt means ``fit, adjust''.
% http://www.dictionary.com/browse/adapt

% https://en.wikipedia.org/wiki/Adaptive_system
% http://link.springer.com/article/10.1007%2Fs11047-008-9096-6
% Adaptation, anticipation and rationality in natural and artificial systems: computational paradigms mimicking nature

Adaptive with respect to what?

Chaotic system:
Small change in input causes large change in output.
See chaos theory.
