\chapter{Nominal}

A \emph{nominal} is a construction that functions like a noun.

Every noun is a nominal.

\section{Conjunction: と}

Unlike English ``and'', Japanese と conjoins \emph{nominals} only.
English ``and'' can conjoin noun phrases or sentences.
田中さんと中川さんは東京に行く。Tanaka-san and Nakagawa-san goes to Toukyou.

[〈鳥を食べる人〉と〈魚を食べる人〉]が良いです。
People who eat chicken and people who eat fish are okay.

彼と私は同じ夢を見ました。
He and I had the same dream.

\section{Disjunction: か}

鳥か魚は良いです。
Chicken or fish are okay.
(Whichever of chicken or fish you give me, I will eat it.)

\section{Possession: の}

カラスの羽feather of crow

父の車father's car

彼の友達の娘 his friend's daughter

彼の妹の娘の友達his younger sister's daughter's friend

父の黒い車father's black car

細い俳優の車skinny actor's car

私の家の右the house at the right side of my house

私の家の右の右the house at the right side of the house at the right side of my house

\section{Modification by i-adjectives}

Every nominal can be modified by prepending an adjectival.

Every i-adjective is an adjectival.

黒い車black car

悲しい人sad person

細い悲しい人thin sad person

Every dictionary-form verb is an adjectival.

Every perfect-form verb is an adjectival.

黒い is an i-adjective.
車 is a noun.

車car.

黒いblack

黒い車 is an i-adjectival-modified nominal. It means ``black car''.

高い人tall person

黒くない車non-black car

黒かった車formerly-black car (a car that was black)

黒くなかった車formerly-nonblack car (a car that was not black)

Start with 黒い. Negate it to 黒くない
and then perfect it to 黒くなかった.

高い黒い車expensive black car

高い黒くない車expensive non-black car

高くない黒い車non-expensive black car

高くない黒くない車non-expensive non-black car

高かった黒かった車formerly-expensive formerly-black car

\section{More modifications by adjectivals}

Japanese subordinate clauses do not use relative pronouns.
魚を食べる is both a complete statement and a subordinate adjective clause.
As a complete statement, it means ``The implied entity eats fish.''
As a subordinate clause, it means ``who eats fish'' or ``which eats fish.''
Thus, 魚を食べる人 is a nominal that means ``the person who eats fish,''
and parses as 魚を食べる・人.

魚を食べない人person who does not eat fish

魚を食べなかった人person who did not eat fish

結婚出来ない男a man who cannot marry

妹が居ない人
people who do not have younger sisters

お金がない人
people who do not have money

夜で魚を食べる人
people who eat fish at night

石を投げる子供stone-throwing child

愛されなかった人a person who was not loved

笑う人laughing person

笑わない人non-laughing person; person who does not laugh; person who is not laughing

笑わなかった人formerly-non-laughing person; person who did not laugh

笑った人person who laughed

愛される人loved person

愛されない人unloved person

愛された人formerly-loved person

愛されなかった人formerly-nonloved person; a person who was not loved

Begin with 愛 (love).
Append する, producing 愛する (to love; who loves).
Passivate する to される by u-to-areru, producing 愛される (to be loved; who is loved).
Negate される to されない by v1-ru-to-nai, producing 愛されない (who is not loved).
Perfect されない to されなかった by i-to-katta, producing 愛されなかった (who was not loved).

話されなかった言葉
words that were not spoken

\section{In a relative clause, が can change to の}

In a relative clause, が can change to の.
For example, both 海が見え街 and 海の見え街 is to be taken to mean
``seaside town'' (a town where the sea is visible)
where the adjectival 海が見える(where the sea is visible) explains the noun 街.
Both 日本人が知らない日本語
and 日本人の知らない日本語
means ``the Japanese language that the Japanese people do not know.''
Because の was が, the nominal 日本人の知らない日本語 parses as
〈日本人・の・知らない〉日本語 (the Japanese 〈that the Japanese people do not know〉)
and not as
[日本人・の]〈知らない・日本語〉
([The Japanese people's]〈unknown・Japanese language〉).

\section{In doubt}

愛される田中さんの女性Beloved Tanaka-san's lady (or ladies)

田中さんの愛される女性Tanaka-san's beloved lady (or ladies)
