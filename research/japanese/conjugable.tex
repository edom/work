\chapter{Inflectables}

\section{Perfection of dictionary-form verb: た、だ}

``Perfect'' means ``complete'', as in ``past perfect'', not flawless.
Every dictionary-form verb can be perfected into an perfect-form verb.
The result is still a verb.

笑う。
The implied entity laughs.
The implied entity will laugh.

笑った。
The implied entity laughed (and is no longer laughing now).
The implied entity has laughed.
The implied entity had laughed.

\section{Perfection of i-adjective: i-to-katta}

速い becomes 速かった

\section{Negation of i-adjective: i-to-kunai}

速い becomes 速くない

\section{i-to-kunakatta = i-to-kunai + i-to-katta}

\section{Volitionalization: u-to-itai}

笑う becomes 笑いたい

話す becomes 話したい

\section{Conjunction: i-form}

食べる becomes 食べ

話す becomes 話し

立つ becomes 立ち

\section{Progressive form}

Examples:
the 歩き in 歩きタバコ(smoking while walking).
歩きタバコするto smoke while walking.

\section{Concurrency: X-while-Y: i-form of Y + nagara + X}

Japanese is consistently head-final,
and while-Y explains X,
so while-Y comes before X.
The modifier always comes before the head.

食べながら while eating

立ちながら while standing

食べながら話すto speak while eating

立ちながら食べるto eat while standing

\section{In doubt}

A verbal can be modified by adverbials.
A nominal can be modified by adjectivals.
