\chapter{Kanji 4}

\section{3 (roof radical) 11 窓}

窓(ソウ、まど)window.

\section{3 女 10 姫 13 嫌}

姫(ひめ)princess.

嫌い(きらい)hate.
嫌悪感(ケンオカン)aversion.

\section{4 火 15 熱 16 燃}

熱い(あつい)(adj)hot (temperature)

燃える(もえる)(v1,vi) to burn; to get fired up

\section{4 手 15 撃 16 操}

撃(ゲキ)beat.
電撃(デンゲキ)electric shock.
衝撃(ショウゲキ)shock; crash; impact.

操る(あやつる)(vt) to be fluent in (a language)

体操(タイソウ)physical exercise; gymnastics; calisthenics.

\section{4 心 13 愛 17 優}

愛(アイ)(n) love.
愛す(あいす)(vt) to love.
同性愛(ドウセイアイ)homosexuality; same-sex love.

優しい(やさしい)tender; kind; gentle; affectionate; suave

\section{5 目 17 瞳}

瞳(ひとみ)pupil (of the eye).

\section{5 用 12 備}

備(ビ)provision.
備考(ビコウ)note; remarks; nota bene; NB.
備える(そなえる)to provide; to equip; to install.

\section{6 糸 11 終率 13 続}

最終(サイシュウ)last; final; closing.
終わる(おわる)to finish; to end; to close.

率(リツ)(suf) rate; ratio; proportion.
識字率(シキジリツ)literacy rate.

続(ゾク、つづ)continue.
相続(ソウゾク)succession; inheritance.
存続(ソンゾク)duration; continuance.
続く(つづく)to continue.

\section{7 言 20 議}

議(ギ)deliberation; consultation; debate; consideration.
議論(ギロン)argument; discussion; dispute; controversy.

\section{7 貝 12 貿 15 遺}

貿(ボウ)trade.
貿易(ボウエキ)trade (foreign)

遺(イ、ユイ、のこ)bequeath; leave behind.
遺児(イジ)orphan.
遺物(イブツ)relic; memento.
遺品(イヒン)articles left by the deceased.
遺書(イショ)will; testament.
遺体(イタイ)corpse; remains.
遺言(ユイゴン)will; testament.
遺す(のこす)to bequeath; to leave behind; to save; to reserve.

\section{9 頁 11 頃}

頃(ころ)time of year; season.

\section{10 鬼 14 魂 21 魔}

鬼(キ、おに)
ogre; demon.
spirit of a deceased person.
吸血鬼(キュウケツキ)vampire; blood-sucking demon.

魂(たましい)soul.
魂 is made of 云 (cloud) and 鬼 (ghost; demon; spirit).

魔(マ)witch; demon; evil spirit.
魔女(マジョ)witch.
魔法(マホウ)magic (magick, not the magical tricks); witchcraft; sorcery.
悪魔(アクマ)evil spirit.
邪魔(ジャマ)intrusion.
邪魔するto intrude.

\section{13 違}

相違(ソウイ)difference; discrepancy; variation.
違う(ちがう)(vi) to differ; to not match the correct answer.
間違う(まちがう)to make a mistake; to be incorrect; to be mistaken.
違反(イハン)violation of law.

\section{14 暮}

暮(ボ、くら)livelihood.
暮らす(くらす)to live; to get along.

\section{15 暴 19 爆}

暴 depicts the antler of a buck, representing a savage attack, a violence.
暴動(ボウドウ)insurrection; rebellion; revolt; riot; uprising.
暴風(ボウフウ)storm; windstorm; gale.
暴れる(あばれる)(v1,vi) to rage; to act violently.

爆(バク)burst; explode; bomb.
自爆(ジバク)suicide bombing; self-destruct.
水爆(スイバク)hydrogen bomb.
原爆(ゲンバク)atomic bomb; nuclear bomb.
空爆(クウバク)aerial bombing; air raid.
爆殺(バクサツ)killing by bombing.
爆死(バクシ)death by explosion.
爆音(バクオン)sound of explosion or detonation.

\section{15 趣}

趣味(シュミ)hobby; taste, preference.

\section{17 覧}

覧(ラン)perusal.
回覧(カイラン)circulation.
回覧板(カイランバン)circular notice
(especially those distributed to households within a neighborhood association).

\section{Confusing characters}

\subsection{Spirit and cloth: 4 礻 5 衤}

\subsection{Arrow and loss: 矢失}

\subsection{3 夂夊 4 攵}

In the Japanese language,
these characters become parts of other characters
instead of being used on their own.

夂 depicts two legs followed by something from behind.

夊 depicts a footprint.

攵 is a variant of 攴 depicting a branch and a hand.

\subsection{Samurai and earth: 3 士土}

士(samurai) has longer upper horizontal stroke.
土(earth) has shorter upper horizontal stroke.

\subsection{Hat, sun, moon, meat, inner}

冃(hat)

日(sun)

月(moon)

\subsection{石 (stone) and 右 (right)}

\subsection{人 (person) and 入 (enter)}

\subsection{王 (king) and 生 (sprout)}

\subsection{業菐美}

13 業 contains 业未.
14 菐 contains 业土人.

\section{Calendar}

These kanji readings for today, yesterday, and tomorrow are irregular.

今日(きょう)today

昨日(きのう)yesterday

明日(あした)tomorrow

Names of weekdays.

日曜日(ニチヨウび)Sunday

月曜日(ゲツヨウび)Monday

火曜日(カヨウび)Tuesday

水曜日(スイヨウび)Wednesday

木曜日(モクヨウび)Thursday

金曜日(キンヨウび)Friday

土曜日(ドヨウび)Saturday

毎日(マイニチ)everyday

Expressions.

また明日(あした)see you again tomorrow; means 'again' and 'tomorrow'

\section{Homophones}

\subsection{おさめる: 4 収 8 治 10 納}

納める(おさめる)(v1,vt)
to dedicate; to make an offering to; to pay (fees) to.
収納(シュウノウ)crop; harvest.
未納(ミノウ)default; failure to pay; overdue payment.

治(おさむ)(name) Osamu.
治る(なおる)(vi) to heal.
治める(おさめる)(v1,vt)
to dominate; to rule; to govern; to manage.
to tranquilize; to pacify; to subdue.
to suppress.
政治(セイジ)politics.

\subsection{うつ:}

\section{つくる: 7 作 10 造}

\section{つかえる}

\section{かわる}
