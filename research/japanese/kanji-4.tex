\chapter{Kanji 4}

\section{4 火水 5 氷永 8 泳}

火(カ、ひ)fire, flame.

火事(カジ)fire (disaster).

大火(タイカ)big fire.

水(スイ、みず)water.

井 depicts a square well.

井(い)well (water reservoir).

氷(こおり)ice

永(エイ、なが)eternity.

永い(ながい)(adj-i) very long (time).

泳ぐ(およぐ)(vi) to swim

\subsection{(4 火) 6 灰 7 災 8 炎}

灰(カイ、はい)ashes.

火災(カサイ)fire (disaster).

災い(わざわい)(n) calamity; catastrophe.

炎(ほのお)flame, blaze.

炎天(エンテン)scorching sun.

\subsection{(4 火) 12 焼 15 熱 16 燃}

焼(ショウ)bake; burn.

焼き鳥(やきとり)grilled chicken meat.

熱い(あつい)(adj)hot (temperature)

燃える(もえる)(v1,vi) to burn; to get fired up.

再燃(サイネン)recurrence; revival; resuscitation;
getting spirited/interested once again.

\section{4 心 5 必}

心 is involved in a lot of feeling-related characters.

心配(シンパイ)(adj-na,n,vs) worry, concern, anxiety.

心配(シンパイ)(n,vs) care, help.

必 is unrelated to 心. They only look similar.

必(ヒツ、かなら)inevitable.

必ず(かならず)(adv) always, invariably, certainly.

必要(ヒツヨウ)(adj-na,n) necessity, need.

\subsection{Thoughts: 7 忘 9 思}

忘(ボウ、わす)forget.

忘れる(わすれる)(v1) to forget.

忘年会(ボウネンカイ)year-end party
(lit. forget-year meeting, a meeting to forget the year).

思(シ)think.
思う(おもう)to think

\subsection{10 息}

息(ソク、いき)breath.

息子(むすこ)son

\subsection{13 感}

感じる(カンじる)(v1) to feel.

感心(カンシン)(adj-na,n,vs) admiration. Well done!

\subsection{Other: 9 急 11 悪}

急(キュウ)urgent, sudden, abrupt.

急ぐ(いそぐ)to hurry.

悪(アク)evil, wickedness.

悪人(アクニン)bad person, villain.

悪い(わるい)bad, poor; evil; unprofitable; at fault.

\subsection{(4 心) 7 応 13 愛 17 優}

応え(こたえ)response; reply; answer; solution.

応える(こたえる)(v1) to respond; to reply; to answer.

愛(アイ)(n) love.

愛す(あいす)(vt) to love.

同性愛(ドウセイアイ)homosexuality; same-sex love.

優しい(やさしい)tender; kind; gentle; affectionate; suave

\ruby{愛}{まな}\ruby{弟}{デ}\ruby{子}{シ}favorite pupil/student.

\section{4 水 16 激}

激(ゲキ)violent.

激しい(はげしい)violent.

激しい雨(はげしいあめ)violent rain.

\section{4 中夬 5 央 6 仲 7 決}

中(チュウ、なか)middle.

中学(チュウガク)middle school; junior high school.

…の中(…のなか)middle of something.

田中(たなか)(name) Tanaka.

中山(なかやま)(name) Nakayama.

中川(なかがわ)(name) Nakagawa.

夬Kangxi radical 37.

央(オウ)center.
中央(チュウオウ)center; central.

仲(なか)relation; relationship.

仲間(なかま)company; fellow; colleague; associate; comrade; partner.

決(ケツ)decide.

決まる(きまる)to be decided; to be settled. to look good in clothes.

\subsection{9 映}

映(エイ)reflect.

\section{4 耂 6 考老}

耂 depicts an old man, a bent-over figure with long hair.

考(コウ)consider.

考える(かんがえる)(v1,vt) to consider; to think about.

老い(おい)old age; old (of person); at a late time in life.

老人(ロウジン)old person.

老若(ロウニャク)old and young; all ages.

\subsection{7 孝 11 教}

孝(コウ)filial piety.

教(キョウ)teach.

教育(キョウイク)training; education.

教会(キョウカイ)church.

教え(おしえ)teaching; doctrine.

教える(おしえる)(v1,vt) to teach.

\subsection{8 者 11 著都 14 緒}

者(シャ)(n,suf) someone of that nature; someone doing that work.

者(もの)(n) person (rarely used without a qualifier).

学者(ガクシャ)scholar.

作者(サクシャ)author.

業者(ギョウシャ)trader; merchant.

研究者(ケンキュウシャ)researcher.

著(チョ)renowned.

著書(チョショ)literary work; book; textbook.

著名人(チョメイジン)celebrity.

都(ト、ツ、みやこ)metropolis.

緒(ショ)thong.

一緒(イッショ)together.

\section{4 王 8 国 9 美皇}

王(オウ)king

国(コク、くに) country.

国王(コクオウ)king.

国内(コクナイ)internal; domestic.

共和国(キョウワコク)republic.

インドネシア共和国Republic of Indonesia.

美(ビ)beauty.

美人(ビジン)beautiful woman.

美しい(うつくしい)beautiful.

美味しい(おいしい)delicious (idiosyncratic reading).

皇(コウ)imperial.

皇居(コウキョ)imperial palace.

\section{4 元之文}

元(ゲン、ガン,もと)origin; beginning; source.
元(もと)origin; source.
元々(もともと)
originally; by nature; from the start; since the beginning.

之(これ)this

文(ブン)sentence (literature).
文字(モジ)letter (of alphabet); character (a Han character)
文書(ブンショ)sentence.
文化(ブンカ)culture.
作文(サクブン)writing.
小学生の作文(ショウガクセイのサクブン)elementary-schoolchild writing.

\section{4 今 6 合}

今(コン、いま)now.
今回(コンカイ)this time; this occasion; this occurrence.

合(ゴウ、あ)fit.
合う(あう)to fit.

\section{4 殳 7 投没 10 殺 11 設 15 撃}

殳 depicts a tool or a weapon.

投げる(なげる)to throw

没(ボツ)drowning

殺す(ころす)to kill.

殺害(サツガイ)murder.

殺人(サツジン)murder.

設ける(もうける)to establish.

撃(ゲキ)beat.

電撃(デンゲキ)electric shock.

衝撃(ショウゲキ)shock; crash; impact.

\section{3 尸 4 戸 5 冊 8 雨}

尸 Kangxi radical 44 (``corpse'').

戸(コ、と、べ)door.

神戸(こうべ)Koube (an area in Japan).

雨(あ\め)rain

\subsection{5 冊 9 柵}

冊(サツ)counter for books.

冊子(サッシ)booklet.

小冊子(ショウサッシ)booklet; pamphlet.

一冊(イッサツ)one book.

柵(サク)fence.

\subsection{6 尺 11 訳}

尺(シャク)shaku (a unit of length).

訳(ヤク)translate.

訳す(ヤクす)(vt) to translate.

\subsection{7 戻 所 10 涙 14 漏 15 編}

戻る(もどる)(vi) to turn back; to return; to go back;
to come back to a previously visited place

所(ところ)place.

近所(キンジョ)neighborhood.

名所(メイショ)famous place.

涙(なみだ)tear (eyewater)

漏れる(もれる)to leak (liquid)

編(ヘン)editing; compilation.

編集(ヘンシュウ)Edit (menu item in computer user interface).

編む(あむ)(vt)
to knit; to plait; to braid.
to compile (an anthology); to edit.

\section{4 勿 8 易 12 陽場}

勿(ブツ)

易(エキ)easy; simple.

易しい(やさしい)(adj-i) easy; plain; simple.

交易(コウエキ)trade; commerce.

易い(やすい)(adj-i) easy (not difficult).

陽(ヨウ)the yang in yin and yang.

太陽(タイヨウ)sun.

場(ジョウ、ば)place.

場所(ばショ)place.

場合(ばあい)case; situation.

\section{4 日月 9 韋}

日(ひ)sun.

日々(ひび)daily; days; old days.

月(つき)moon.

韋 is Kangxi radical 178 (``tanned leather'').

\subsection{5 白 7 伯 8 拍泊迫}

白(ハク)white.

白い(しろい)white.

伯(ハク)chief.

拍(ハク)clap.

泊(ハク)overnight.

迫(ハク)urge.

\subsection{8 昔 10 借}

昔(セキ、むかし)long ago.

借(シャク)borrow.

\subsection{12 散}

散(サン)scatter.

\subsection{12 間温}

間(カン、ケン)interval; space.

間(あいだ)gap; interval; distance; span; stretch (space or time).

間(ま)space; room; time; pause.

人間(ニンゲン)human being.

世間(セケン)world; society.

温かい(あたたかい)(adj-i) warm (of a tangible object; to the touch)

\subsection{9 昼}

昼(チュウ、ひる)daytime.

\subsection{6 早 9 草 12 朝}

早い(はやい)(adj-i) early.

草(くさ)grass

朝(チョウ、あさ)morning.

今朝(けさ)this morning.

早朝(ソウチョウ)early morning.

\subsection{8 明 9 春星}

明(メイ)bright.
明るい(あかるい)bright.

春(シュン、はる)spring (season)

星(セイ、ほし)star

\subsection{(6 早 9 韋) 8 𠦝卓 13 違 16 衛 18 韓}

𠦝 is an alternative form of 卓.

卓(タク)eminent.

違(イ)differ.

相違(ソウイ)difference; discrepancy; variation.

違う(ちがう)(vi) to differ; to not match the correct answer.

間違う(まちがう)to make a mistake; to be incorrect; to be mistaken.

違反(イハン)violation of law.

衛(エイ)defense.

衛生(エイセイ)satellite.

人工衛星(ジンコウエイセイ)human-made satellite.

韓(カン)Korea.

\section{4 公 5 広払 7 私 8 拡}

公(コウ、おおやけ)public; communal; official; governmental.

公安(コウアン)public safety; public welfare.

広(コウ、ひろ)wide.

広い(ひろい)(adj-i) spacious; vast; wide.

広告(コウコク)advertisement.

払(フツ)pay.

私(シ、わたくし、わたし)I; me.

拡(カク)broaden.

\section{4 幻 5 幼}

幻(まぼろし)phantom; vision; illusion; dream

幼(ヨウ、おさな)infancy.
幼い(おさない)very young; immature. childish.

\section{4 心 7 志快}

志(シ、こころざし)intention

快(カイ)cheerful.
快い(こころよい)cheerful.

\section{4 不 7 否}

不(フ)(prefix) not; bad; poor.

不安(フアン)anxiety; insecurity.

不明(フメイ)unknown; obscure; anonymous; unidentified.

否(イナ、いや)negate.

\section{(4 木) 7 村枚}

村(むら)village

枚(マイ)(counter) sheet; thin flat object.
一枚(イチマイ)one sheet.

\section{(4 戈) 6 伐成 7 戒}

伐(バツ)fell; strike; attack; punish.

成(セイ)become.

作成(サクセイ)(n,vs) writing; creation.

コンピュータプログラムを作成するto write a computer program.

戒(カイ)commandment.

十戒(ジッカイ)
(Buddhist) 10 precepts.
(Christian) 10 commandments.

戒める(いましめる)(vt) to admonish.

\section{(4 god radical) 7 社 10 神}

社(シャ)company; firm; association; shrine

社(やしろ)shrine (usually Shinto).

神(かみ)god; spirit; thunder

\section{4 水 7 沈 10 浮}

沈下(チンカ)sinking; subsidence.

沈む(しずむ)(vi) to sink (descend into liquid).
This was simplified from 18 瀋.

浮かぶ(うかぶ)to float (be supported by liquid)

\section{4 水 9 津}

津(つ)seaport; harbor.

津波(つなみ)(n) tsunami; tidal wave.

\section{4 壬 6 任 13 賃}

壬(ニン、ジン、みずのえ)depicts a carrying pole.

任(ニン)responsibility.

賃(チン)fare.

\section{4 天夫 5 矢失}

天(テン)sky; heaven

夫(フ、おっと)husband

矢(シ、や)arrow

失 depicts something falling from a hand.
失う(うしなう)(vt) to lose; to part with.
失明(シツメイ)loss of eyesight.
失血(シッケツ)loss of blood.

\section{4 止 8 歩}

止 depicts a footprint.

止(シ)stop.

止まる(とまる)(vi) to stop (moving); to come to a halt.

止める(やめる)(v1) to stop (doing something).

止める(とめる)(v1) to stop moving (walking, etc.); to park (a car).

歩(ホ、フ、ブ)walk.

歩く(あるく)to walk.

\subsection{6 企}

企(キ)plan.

\subsection{7 足⻊ 14 踊}

足(ソク、あし)foot.
Kangxi radical 157.

⻊ is the component form of 足 (foot).

踊(ヨウ)jump.

\subsection{9 是 12 提 18 題}

是(ゼ)just so.

提(テイ)present.

提供(テイキョウ)offer.

題(ダイ)topic.

問題(モンダイ)problem.

\subsection{5 正 10 症}

正(セイ、ショウ)correct.

不正(フセイ)(adj-na,n) injustice; illegality; fraud.

不正なソフトウェアmalicious software.

正しい(ただしい)right; correct.

症(ショウ)(n,suf) illness.

\subsection{7 走 8 定}

走(ソウ)run.

走る(はしる)(v5r,vi) to run

定(テイ、ジョウ)fix; determine; establish; settle; decide.

安定(アンテイ)stability; equilibrium.

予定(ヨテイ)(n,vs) plan; arrangement; schedule; program.

定住(テイジュウ)settlement; permanent residence.

定める(さだめる)(v1,vt) to decide; to establish; to determine.

未定(ミテイ)not yet fixed; undecided; pending.

\section{4 斤 7 辛}

\subsection{7 近 8 析}

斤 depicts an axe.

辛 depicts a tool used to mark slaves and criminals;
this sometimes also depicts a tree.

辛い(からい)(adj-i) spicy, salty, harsh, hot, acrid.

辛い(つらい)(adj-i) bitter; painful; heartbreaking; difficult.
Suffix づらい(adj-i) means ``difficult to do''.

読みづらい(adj-i) difficult to read.

書きづらい(adj-i) difficult to write.

読みづらい漢字difficult-to-read Han character.

近い(ちかい)(adj-i) near (spatial distance).

近々(ちかぢか)soon.

近作(キンサク)recent work.

最近(サイキン)most recent; recently; these days; nowadays.

析(セキ)chop.

分析(ブンセキ)analysis.

\subsection{13 新 16 親}

新 depicts cutting tree down with axe.

新(シン)new.

新聞(シンブン)news.

新車(シンシャ)new car.

最新(サイシン)newest.

新しい(あたらしい)(adj-i) new.

親(シン、おや)parent.

両親(リョウシン)both parents.

親友(シニュウ)close friend.

母親(ははおや)mother.
