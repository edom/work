\chapter{Kanji 4}

\section{Confusing characters}

\subsection{Spirit and cloth: 4 礻 5 衤}

\subsection{3 夂夊 4 攵 8 㑒}

In the Japanese language,
these characters become parts of other characters
instead of being used on their own.

夂 depicts two legs followed by something from behind.

夊 depicts a footprint.

攵 is a variant of 攴 depicting a branch and a hand.

㑒 is simplified from the 13-stroke 僉
meaning ``all, together, unanimous''.

\subsection{Samurai and earth: 3 士土}

士(samurai) has longer upper horizontal stroke.
土(earth) has shorter upper horizontal stroke.

\subsection{Hat, sun, moon, meat, inner}

冃(hat)

日(sun)

月(moon)

\subsection{石 (stone) and 右 (right)}

\subsection{人 (person) and 入 (enter)}

\subsection{王 (king) and 生 (sprout)}

\subsection{業菐美}

13 業 contains 业未.
14 菐 contains 业土人.
