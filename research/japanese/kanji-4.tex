\chapter{Kanji 4}

\section{15 暴熱趣}

暴 depicts the antler of a buck, representing a savage attack, a violence.
暴動(ボウドウ)insurrection; rebellion; revolt; riot; uprising.
暴風(ボウフウ)storm; windstorm; gale.
暴れる(あばれる)(v1,vi) to rage; to act violently.

熱い(あつい)(adj)hot (temperature)

趣味(シュミ)hobby; taste, preference.

\section{16 燃}

燃える(もえる)(v1,vi) to burn; to get fired up

\section{19 爆}

爆(バク)burst; explode; bomb.
自爆(ジバク)suicide bombing; self-destruct.
水爆(スイバク)hydrogen bomb.
原爆(ゲンバク)atomic bomb; nuclear bomb.
空爆(クウバク)aerial bombing; air raid.
爆殺(バクサツ)killing by bombing.
爆死(バクシ)death by explosion.
爆音(バクオン)sound of explosion or detonation.

\section{20 議}

議(ギ)deliberation; consultation; debate; consideration.
議論(ギロン)argument; discussion; dispute; controversy.

\section{21 魔}

邪魔(ジャマ)intrusion.
邪魔するto intrude.

\section{Confusing characters}

\subsection{Spirit and cloth: 4 礻 5 衤}

\subsection{3 夂夊 4 攵 8 㑒}

In the Japanese language,
these characters become parts of other characters
instead of being used on their own.

夂 depicts two legs followed by something from behind.

夊 depicts a footprint.

攵 is a variant of 攴 depicting a branch and a hand.

㑒 is simplified from the 13-stroke 僉
meaning ``all, together, unanimous''.

\subsection{Samurai and earth: 3 士土}

士(samurai) has longer upper horizontal stroke.
土(earth) has shorter upper horizontal stroke.

\subsection{Hat, sun, moon, meat, inner}

冃(hat)

日(sun)

月(moon)

\subsection{石 (stone) and 右 (right)}

\subsection{人 (person) and 入 (enter)}

\subsection{王 (king) and 生 (sprout)}

\subsection{業菐美}

13 業 contains 业未.
14 菐 contains 业土人.
