\chapter{Kanji 4}

\section{4 火 15 熱 16 燃}

熱い(あつい)(adj)hot (temperature)

燃える(もえる)(v1,vi) to burn; to get fired up

火事(カジ)fire (disaster).
ラメン店で火事fire at a ramen shop.

\section{4 水 14 滴漏}

滴(しずく)a drop of water; a drip.
滴る(したたる)to drip (fall one drop at a time).

漏れる(もれる)to leak (liquid)

\section{4 心 13 意愛 17 優}

意(イ)feelings; thoughts.
小生意気(こなまイキ)cheekiness; impudence.

愛(アイ)love

愛(アイ)(n) love

優しい(やさしい)tender; kind; gentle; affectionate; suave

\section{4 手 15 撃}

電撃(デンゲキ)electric shock

\section{7 言 9 計訂 10 記 13 話 14 読語 15 誰}

言葉(ことば)word; dialect

計(ケイ)plan.
計画(ケイカク)plan; project; schedule; scheme; program; programme.

訂正(テイセイ)correction; revision; amendment

記(キ) record

記す(しるす)to record, to write down

記事(キジ)article (writing).
選り抜き記事(よりぬきキジ)selected articles.
新しい記事(あたらしいキジ)new articles.

記録(キロク)record

話 talk

話す(はなす)to talk

読 read

読む(よむ)to read

語(ゴ) language

…語(…ゴ)... language

日本語(ニホンゴ)Japanese language

英語(エイゴ)English language

15 誰(だれ)who

\section{14 態}

態(ざま)mess; sorry state; plight; sad sight

変態(ヘンタイ)sexual perversion

\section{14 罰}

罰金(バッキン)fine; penalty

\section{15 暴 19 爆}

15 暴(バク)(antler of a buck; savage attack)

暴れる(あばれる)(v1,vi) to rage; to act violently

19 爆(バク)burst; explode; bomb

自爆(ジバク)suicide bombing; self-destruct

水爆(スイバク)hydrogen bomb

原爆(ゲンバク)atomic bomb; nuclear bomb

空爆(クウバク)aerial bombing; air raid

爆殺(バクサツ)killing by bombing

爆死(バクシ)death by explosion

\section{18 類}

類(ルイ)kind; sort; type

人類(ジンルイ)mankind

\section{19 鏡}

8 金 + 5 立 + 7 見 - 1

鏡(かがみ)mirror

眼鏡(めがね)eyeglasses
