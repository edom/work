\chapter{Text}

a.k.a. Learn Kanji by Reading a Long Incoherent Story

This text tries to use each kanji.

Learn reading, grammar, writing, and vocabulary at once.
Kill four birds with one stone.
But one who chases two rabbits catches neither.

\newcommand\hlp[1]{%
    {%
        \parindent0em%
        \par%
        \vspace{\myspacing}%
        {\fontsize{10pt}{10pt}\selectfont#1}%
        \vspace{\myspacing}%
        \par%
    }%
}

\section{女子供好}

\sbs{
    \ruby{女}{おんな}は\ruby{子}{こ}\ruby{供}{ども}が\ruby{好}{す}き。
}{
    The lady likes children.
}

\sbs{
    \ruby{和}{ワ}\ruby{食}{ショク}を\ruby{食}{た}べる。
}{
    I eat Japanese food.
}

\section{門閉開}

\sbs{\ruby{門}{かど}を\ruby{開}{ひら}いて!}{Open the gate!}

\sbs{門を\ruby{閉}{し}めて!}{Close the gate!}

\section{止足歩走}

\sbs{止め!}{Halt! (Stop walking!)}

\sbs{\ruby{足}{あし}で\ruby{歩}{ある}く。}{I walk with legs/feet.}

\sbs{\ruby{馬}{うま}は\ruby{走}{はし}っている。}{The horse is running.}

\section{昇登}

\sbs{\ruby{枝}{えだ}\ruby{豆}{まめ}}{green pea}

\sbs{
    \ruby{登}{ト}\ruby{山}{ザン}は\ruby{山}{やま}に\ruby{登}{のぼ}る\ruby{事}{こと}。
}{
    \foreign{Bergsteigen} is mountain climbing.
}

\sbs{
    クライミングは\ruby{登}{トウ}\ruby{攀}{ハン}。
}{
    \foreign{Klettern} is climbing.
}

\sbs{
    \ruby{太}{タイ}\ruby{陽}{ヨウ}は\ruby{昇}{のぼ}る。
}{
    The sun rises.
}

\section{6 毎}

\sbs{
    \ruby{毎}{マイ}\ruby{日}{ニチ}
}{
    everyday
}

\section{Unsorted}

\hlp{
人(ひと)
土(つち)
立つ(たつ)
木(き)
下(した)
休む(やすむ)
上げる(あげる)
青空(あおぞら)
翌朝(ヨクあさ)
}

\sbs{
彼は、「あの犬は、昨日・来たけど、今日・来ない。明日・来るか?」と想った。
}{
He wonders, ``That dog came yesterday, but it didn't come today.
Will it come tomorrow?''
}
\sbs{
    翌朝、田に道を歩きながら、吠え声を聞いた。
    犬を見た。
    「どこに行っていた~?」と言った。
}{
    The next morning, as he walks on the street
    to the rice field, he hears a bark.
    He sees the dog.
    ``Where have you been?'' he says.
}

\sbs{
}{
}

\sbs{
    朝が来た。
    太陽が昇った。
}{
    Morning has come.
    The sun has risen.
}

\hlp{
犬(いぬ)
来る(くる)
来た(きた)
来ない(こない)
昨日(きのう)
今日(きょう)
明日(あした)
想う(おもう)
}

\hlp{
    朝(あさ)
    太陽(タイヨウ)
    昇る(のぼる)
}

\hlp{
    吠え声(ほえごえ)
    聞く(きく)
    行く(いく)
    言う(いう)
    道(みち)
    歩く(あるく)
}

\sbs{
    人はいる。
}{
    There is a person.
}

\newcommand\myfootnote{I don't know why it's に instead of で,
but Google returned more results for に than で.}

\sbs{
    土に\footnote{\myfootnote}立っている。
    木の下で休んでいる。
    青空を見上げている。
}{
    He is standing on the ground.
    He is resting beneath a tree.
    He is looking up at the blue sky.
}

\hlp{
日(ひ)
月(つき)
明るい(あかるい)
光る(ひかる)
星(ほし)
風(かぜ)
雲(くも)
}
\sbs{
日は光っている。
月は明るい。
星も。
風は雲を動かしている。
}{
The sun is shining.
The moon is bright.
The stars are bright too.
The wind is moving the clouds.
}

\section{4 水 7 冷体良 12 飲温}

\hlp{
良い(いい)
良くない(よくない)
水(みず)
冷たい(つめたい)
温かい(あたたかい)
体(からだ)
飲む(のむ)
}
\sbs{
冷たい水を飲むのって体に良くないの?

冷たいお水と温かいお水、どちらが良い?
}{
Is drinking cold water not good for the body?

Which is good, cold water or warm water?
}

\section{14 聞読 13 新}

新聞(シンブン)

読む(よむ)

新聞を読む。
The implied entity reads newspaper.

\section{6 名 9 前}

名前(なまえ)

お名前は…Your name is...

\section{5 田生世 6 行 9 界}

\ruby{世}{セ}\ruby{界}{カイ}で\ruby{生}{い}きる。The implied entity lives in the world.

\section{3 子 4 反今 n 私対食供}

私(わたし)

金(かね)

子供(こども)

反対する子供rebelling child.

\hlp{
    食べる(たべる)
    飯(めし)
}

\sbs{
今日も、飯を食べる。
昨日も。
明日も。
}{
Today, he also eats rice.
So did he yesterday.
So will he tomorrow.
}

\hlp{
    捕まえる(つかまえる)
    包丁(ホウチョウ)
    魚(さかな)
    切る(きる)
    料理(リョウリ)
    調理(チョウリ)
    古里(ふるさと)
    違う(ちがう)
}

\sbs{
    古里を想う。
    「『料理』と『調理』は、どう違うの?ウィキペディアで『調理』だ。」
}{
    He thinks about his hometown.
    ``How do `cooking' and `cooking' differ?
    It's `cooking' on Wikipedia.''
}

\sbs{
    魚を捕まえる。
    包丁で切る。
    料理する。
    調理する。
}{
He catches a fish.
He cuts it with a kitchen knife.
He cooks it.
He cooks it.
}

\sbs{
    食べ物は、食べられる物。
    飲み物は、飲まれる物。
}{
    A food is a thing you can eat.
    A drink is a thing you can drink.
}

\section{8 金 n 良有}

有る(ある)

私は、お金が有る。
I have money.

\ruby{良}{い}い。
Good.

\section{12 間 2 人}

人間(ニンゲン)

人間じゃない。The implied entity is not a human.

\section{9 音 14 聞 n 何}

\sbs{
あの\ruby{音}{ね}は、\ruby{何}{なに}?
}{
What is that sound?
}

\sbs{
あれは何の音?
}{
What sound is that?
}

\sbs{
\ruby{聞}{き}かないよ。
}{
The implied entity does not hear the implied sound.
}

\sbs{
あれは何?
}{
What is that?
}

\hlp{
未来(ミライ)
暗い(くらい)
}
\sbs{
未来は暗い。
}{
The future is dark.
}

\hlp{
花(はな)
見る(みる)
}
\sbs{
    花を見たい。
}{
    The implied entity wants to see flowers.
}

\hlp{
今(いま)
両親(リョウシン)
部屋(へや)
本当(ホントウ)
人生(ジンセイ)
死ぬ(しぬ)
始める(はじめる)
}

\sbs{
今、両親は、部屋で、子供を作っている。
}{
Both parents are making a child in the room now.
}

\sbs{
本当の人生は、両親が死んでから始める。
}{
Real life begins after both parents die.
}

\sbs{
    人は、

    実を言う人が嫌い。

    実を言うのが嫌い。

    実が嫌い。
}{
    People hate people who tell the truth.

    They hate telling the truth.

    They hate the truth.
}

\section{10 書 n 昨作文}

作文(サクブン)

作る(つくる)

書く(かく)

昨日、作文を書いた。
The implied entity wrote an essay yesterday.

\section{2 力}

力がない。
The implied entity does not have the strength.

\section{9 重}

重い(おもい)

重いだ。
It's heavy.

\section{13 働 11 動 6 自 7 車}

自動車(ジドウシャ)automobile

働く(はたらく)

死ぬまで働け!
Work until you die!

\section{8 車 11 転 12 運}

車(くるま)

運転(ウンテン)

車を運転する。The implied entity drives a car.

\section{何故}

何故(なぜ)

何故働く?
Why work?

\section{6 好}

ラメンが好き。The implied entity likes ramen (a kind of Japanese noodle).

\section{10 家}

家(いえ)

親は家に居る。
Parents are at home.

\section{What}

\sbs{
    君は強いね。
}{
    You're strong huh.
}

\sbs{
    \ruby{誰}{だれ}の\ruby{支}{ささ}えも\ruby{要}{い}らない。
}{
    The implied entity does not need anyone's help/support.
}

\hlp{
飛ぶ(とぶ)
飛ばない(とばない)
飛べない(とべない)
鳥(とり)
良く(よく)
知る(しる)
一例(イチレイ)
}

\sbs{
鳥は飛ぶ。
ペンギンは飛ばない。
ペンギンは良く知られている飛べない鳥の一例である。
}{
Birds fly.
Penguins don't fly.
Penguin is an example of well-known non-flying birds.
}

\hlp{
    毒(ドク)
    殺す(ころす)
}
\sbs{
    ヒトラーのナチスは、毒ガスで、ユダヤ\ruby{人}{ひと}を殺した。
}{
    Hitler's Nazi killed the Jews using poison gas.
}
