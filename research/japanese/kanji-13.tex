\chapter{Kanji 13}

\section{13 数禁}

数(スウ、かず)number; amount; count.
数える(かぞえる)(v1,vt) to count; to enumerate.
算数(サンスウ)arithmetics.
数万(スウマン)tens of thousands.

禁じる(キンじる)(v1,vt) to prohibit.

\section{14 概}

概要(ガイヨウ)outline; summary

\section{14 際}

際(サイ)occasion; circumstances.

際限(サイゲン)limits; bounds.

学祭(ガクサイ)interdisciplinary.

国際(コクサイ)international.

\section{15 質}

質(シツ)(suffix) substance; quality; matter.
質問(シツモン)question.

\section{15 膝}

膝(ひざ)knee; lap

\section{15 監}

監(カン)government official; rule; administer.
監禁(カンキン)confinement; bondage.

\section{16 諦頭}

諦める(あきらめる)(v1,vt)
to give up; to abandon.

頭(トウ、あたま)head

\section{18 顔類}

顔(ガン、かお)face.

顔面(ガンメン)face (of a person).

類(ルイ)kind; sort; type.

人類(ジンルイ)mankind.

\section{15 暴 19 爆}

暴 depicts the antler of a buck, representing a savage attack, a violence.
暴動(ボウドウ)insurrection; rebellion; revolt; riot; uprising.
暴風(ボウフウ)storm; windstorm; gale.
暴れる(あばれる)(v1,vi) to rage; to act violently.

爆(バク)burst; explode; bomb.
自爆(ジバク)suicide bombing; self-destruct.
水爆(スイバク)hydrogen bomb.
原爆(ゲンバク)atomic bomb; nuclear bomb.
空爆(クウバク)aerial bombing; air raid.
爆殺(バクサツ)killing by bombing.
爆死(バクシ)death by explosion.
爆音(バクオン)sound of explosion or detonation.

\section{15 趣}

趣味(シュミ)hobby; taste, preference.
