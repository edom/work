\chapter{Kanji 2}

\section{Pictures: 5 写 7 図 8 画}

写生(シャセイ)sketch.
写真(シャシン)multimedia; photograph; movie
写す(うつす)(vt) to photograph.

図(ズ)map; drawing; picture; plan; illustration; diagram; figure; chart

画 means picture.
画(カク)counter for kanji strokes.
字画(ジカク)number of strokes in a character.
画家(ガカ)painter.

\section{Cardinal directions: 5 北 6 西 8 東 9 南}

北(ホク、きた)north

西(セイ、にし)west

東(トン、ひがし)east

南(ナン、みなみ)south

They combine as in English.

北西(ホクセイ)northwest

北東(ホクトウ)northeast

東北(トウホク)Touhoku (a prefecture)

南西(ナンセイ)southwest

南東(ナントウ)southeast

\section{Calendar}

These kanji readings for today, yesterday, and tomorrow are irregular.

今日(きょう)today

昨日(きのう)yesterday

明日(あした)tomorrow

Names of weekdays.

日曜日(ニチヨウび)Sunday

月曜日(ゲツヨウび)Monday

火曜日(カヨウび)Tuesday

水曜日(スイヨウび)Wednesday

木曜日(モクヨウび)Thursday

金曜日(キンヨウび)Friday

土曜日(ドヨウび)Saturday

毎日(マイニチ)everyday

Expressions.

また明日(あした)see you again tomorrow; means 'again' and 'tomorrow'

\section{Shops: 8 店}

店(テン)(n) store; shop.
ラメン店ramen shop (ramen is a kind of Japanese noodle).

\section{Myself: 2 厶 4 公 7 私}

厶 means I, myself, private.

公(おおやけ)public; communal; official; governmental.
公安(コウアン)public safety; public welfare.

私(わたし)I; me

\section{Inches: 3 寸 5 付 7 村 10 時}

% https://en.wiktionary.org/wiki/%E5%AF%B8#Japanese
寸depicts a position on the forearm
where the pulse can be palpated by compressing the radial artery.
寸(スン)an ancient unit of length, approximately 3 cm.

付ける(つける)(v1,vt) to attach, join, stick, glue, fasten.

村(むら)village

時(とき)time.
時代(ジダイ)era.
三国時代(サンゴクジダイ)The Three Kingdoms period.
戦国時代(センゴクジダイ)The Warring States period.

\section{Threads}

\subsection{4 幻 9 紅 10 紙}

幻(まぼろし)phantom; vision; illusion; dream

紅(くれない)deep red; crimson

紙(かみ)paper

手紙(てがみ)letter (the document, not the alphabet)

\section{Passable: 5 可 10 哥 14 歌}

可 passable
許可(キョカ)permission; authorization; approval.

哥 means older brother.
This character is not used on its own in Japan.

歌う(うたう)to sing

\section{Escape: 11 脱}

The left side is 肉.
脱出(ダッシュツ)escape.

\section{Foresight: 6 先 8 洗}

先生(せんせい)teacher; master; doctor

洗う(あらう)(vt) to wash

\section{Who: 15 誰}

誰(だれ)who

\subsection{Wide-narrow: 5 広 9 狭}

広い(ひろい)(adj-i) spacious; vast; wide.
広告(コウコク)advertisement.

狭める(せばめる)(v1,vt) to narrow.
狭い(せまい)(adj-i) narrow; confined; small.

\section{Metals: 8 金 13 鉄 14 銅銀 16 鋼}

金has a lot to do with metals.
金(キン、かね)gold; money.

金属(キンゾク)metal.
重金属(ジュウキンゾク)heavy metal.
These are also the chemistry terms.

金色(キンいろ、コンジキ)golden (color)

鉄(テツ、くろがね)iron (lit. 黒金 black metal).
鉄人(てつジン)iron man; strong man.

鉄道(テツドウ)railroad; railway

銅(ドウ、あかがね)copper (lit. 赤金 red metal)

銀(ギン、しろがね)silver (lit. 白金 white metal)

鋼(コウ、はがね)steel

青銅(セイドウ)bronze

鋼鉄(コウテツ)steel

\subsection{Firearms: 14 銃}

銃(ジュウ)gun; small firearms

\subsection{Mirror: 19 鏡}

8 金 + 5 立 + 7 見 - 1

鏡(かがみ)mirror

眼鏡(めがね)eyeglasses

\section{Trouble: 7 困}

困る(こまる)(vi) to be troubled; to be embarrassed

\section{Grass: 6 艸}

\subsection{7 花 9 草}

花(はな)flower

草(くさ)grass

\subsection{Tea: 9 茶}

茶(チャ) tea

\subsection{11 菌}

菌(キン)fungus; germ; bacterium

\section{Fire: 7 災 8 炎}

災い(わざわい)(n) calamity; catastrophe.
火災(カサイ)fire (disaster).

炎(ほのお)flame, blaze.
炎天(エンテン)scorching sun.

\subsection{15 熱 16 燃}

熱い(あつい)(adj)hot (temperature)

燃える(もえる)(v1,vi) to burn; to get fired up

火事(カジ)fire (disaster).
ラメン店で火事fire at a ramen shop.

\section{Gate: 12 間 12 開 14 関}

間 interval

開く(ひらく)to open (door, business, eye, mouth, ...)

関(カン、せき)barrier; gate
