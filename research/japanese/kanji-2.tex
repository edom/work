\chapter{Kanji 2}

\section{2 人 6 因仲 7 作戻 10 涙}

因(イン)cause; factor

仲間(なかま)friend (how does this differ from 友達(ともだち)?)

7 作(サク) work; harvest

作る(つくる)to make

7 戻る(もどる)(vi) to turn back; to return; to go back;
to come back to a previously visited place

10 涙(なみだ)tear (eyewater)

\section{2 又 8 取受}

取る(とる)(vt) take; fetch; take up

受ける(うける)(v1,vt) to receive

\section{2 了 3 子 5 字 8 学}

男子(ダンシ)young man, boy

女子(ジョシ)young woman, girl

太子(タイシ)crown prince

赤字(あかジ)deficit; red letter; red Han-character

漢字(カンジ)Han character

科学(カガク)science

\section{2 刀 4 切方}

4 切 (spoon and sword)

切る(きる)(v5r) cut

切(サイ、セツ)

一切(イッサイ)absolutely; (when used with negative) at all

大切(タイセツ)(adj-na,n) important

4 方(かた)(tip of a sword)

\section{3 口 5 古兄 7 告 8 味}

5 古 old

古consists of 十(ten) and 口(mouth, generation).

5 古い(ふるい)old

古株(ふるかぶ)veteran; senior; old-timer

5 兄(あに)older brother

告げる(つげる)to inform; to tell

告白(コクハク)confess (usually of love)

8 味(あじ)flavor; taste

美味しい(おいしい)delicious

\section{3 女 7 姉 8 妹}

姉(あね)older sister

妹(いもうと)younger sister.
妹consists of 女(lady) and 未(not yet).

\section{3 幺 4 幻 9 紅 10 紙 11 細}

4 幻(まぼろし)phantom; vision; illusion; dream

紅(くれない)deep red; crimson

紙(かみ)paper

手紙(てがみ)letter (the document, not the alphabet)

細い(ほそい)thin; slender
The left-side part of 細is the left-side form of 糸.

\section{3 寸 5 付 7 村}

付ける(つける)(v1,vt) to attach, join, stick, glue, fasten

村(むら)village

口付け(くちづけ)、口づけ(くちづけ)(v1) to kiss

\section{4 日 9 星}

日付(ひづけ)date.日付別(ひづけベツ)separate by date.(context? usage?)

星(ほし)star

\section{4 王 5 主 9 美皇}

5 主(おも)chief; main; principal; important

主人公(シュジンコウ)hero; main character

9 美 beauty

美味しい(おいしい)delicious (idiosyncratic reading)

美しい(うつくしい)beautiful

9 皇(コウ)imperial

皇居(コウキョ)imperial palace

\section{4 木 5 禾 6 米休 7 体困来 9 秋茶}

5 禾 depicts a plant stalk.

9 秋(あき)autumn; fall season.
The character depicts the burning of plant stalks (after harvest).

米(ベイ、こめ)rice

米国(ベイコク)United States of America

6 休 rest

休む(やすむ)to rest

7 体(タイ、からだ)body

肉体(ニクタイ)body; flesh.

9 茶(チャ) tea

7 困 trouble

困る(こまる)(vi) to be troubled; to be embarrassed
(example?)

\section{4 文}

文字(モジ)letter (of alphabet); character (a Han character)

文書(ブンショ)sentence

文化(ブンカ)culture

\section{4 火 7 災}

火事(カジ)fire (disaster)

火災(カサイ)fire (disaster)

\section{4 手 8 押 9 持}

手洗い(てあらい)restroom; lavatory; toilet; a place for washing hands

持つ(もつ)to hold; to carry; to possess

押し(おし)(n) push

\section{4 牛 6 件}

件(ケン)matter; case; item

\section{4 止 7 走 8 歩}

止める(やめる)(v1) to stop

止める(とめる)(v1) to stop

歩く(あるく)to walk

走る(はしる)(v5r,vi) to run

\section{4 心 5 必 7 応}

必 is unrelated to 心. They only look similar.

必ず(かならず)(adv) always, invariably, certainly.
必要(ヒツヨウ)(adj-na,n) necessity, need.

応え(こたえ)response; reply; answer; solution

応える(こたえる)(v1) to respond; to reply; to answer

\section{4 水 6 江汗 9 海洗 10 酒}

冷たい(つめたい)(adj) cold (to the touch)

沈む(しずむ)to sink (descend into liquid)

江(え)inlet; bay

汗(あせ)(n) sweat

汗をかく(exp,v5k) to sweat

汗を流す(exp,v5s) to work hard; to sweat

酒(さけ)(water west) sake (a Japanese liquor)

9 海(うみ)sea; beach.
The original character has 10 strokes.
Shinjitai replaces the two dots in the middle
with one vertical stroke.

洗う(あらう)(vt) to wash

泳ぐ(およぐ)to swim

洪水(コウズイ)(n) flood (of liquid)

大水(おおみず)(n) flood (of liquid)

\section{4 斤 7 近 13 新}

斤(キン) (axe)

近い(ちかい)(adj-i) near

新(シン) new

(cut tree down with an axe)

新しい(あたらしい)new

新聞(シンブン)newspaper (lit. new hear)

新車(シンシャ)new car

\section{5 且 7 助 8 狙}

助(すけ)assistance

助(ジョ)(pref) help; rescue; assistant

狙う(ねらう)(vt) to aim at

\section{5 母 7 毎 8 海}

7 毎(マイ)every

毎日(マイニチ)everyday

毎月(マイゲツ、マイつき)every month

毎時(マイジ)every hour

毎回(マイカイ)every time

毎年(マイネン、マイとし)every year

8 海(うみ)sea; beach

\section{5 田 7 男 9 界}

7 男(おとこ)man

男is田(rice field) and 力(strength).

9 界(カイ)world

学界(ガッカイ)academic world

世界(セカイ)world

業界(ギョウカイ)industry world, business world

\section{5 目 6 自}

6 自(ジ) self

自ら(みずから)(adv) personally

自在(ジザイ)freely (at will)

自分(ジブン)self (context? example usage?)

\section{5 生 8 性}

性(セイ)nature; sex; gender

男性(だんせい)male

女性(じょせい)female

\section{5 矢 7 医 8 知}

矢(や)arrow

医(イ)medicine; healing; curing; doctor (medical)

日本人の知らない日本語the Japanese language that the Japanese people don't know

\section{6 全}

全 depicts a whole piece of jade.

全(ゼン) whole

全部(ゼンブ)altogether; everything

全く(まったく)(adv) completely, entirely, wholly, totally

\section{6 艸 7 花 9 草}

7 花(はな)flower

9 草(くさ)grass

\section{6 血}

止血(シケツ)stop bleeding; hemostasis

\section{6 西}

西(にし)west

\section{6 共 8 供}

共(とも)companion; follower; attendant; retinue

子供(こども)child

\section{6 肉 11 脱}

脱出(ダッシュツ)escape

\section{6 会}

会う(あう)to meet (?)

会社(カイシャ)company; corporation

会社員(カイシャイン)company employee

年会(ネンカイ)yearly meeting; annual convention

社会(シャカイ)society

会見(カイケン)interview

\section{6 安}

安い(やすい)cheap; inexpensive

安全(アンゼン)safety; security

安心(アンシン)relief; peace of mind

\section{6 宅}

住宅(ジュウタク)residence; housing; residential building

自宅(ジタク)one's home

自宅火災(ジタクカサイ)house fire; home fire (disaster)

\section{6 在}

存在(ソンザイ)

\section{6 先 8 洗}

先生(せんせい)teacher; master; doctor

洗う(あらう)(vt) to wash

\section{7 貝 9 則}

7 貝(かい)cowry

(Cowry is a kind of seashell used as money in ancient China.)

則(のり)law; rule; regulation

\section{7 呆 9 保}

呆depicts a child.

呆れる(あきれる)(v1,vi) to be amazed, astonished, astounded.

保する(ほする)to guarantee

\section{7 辛}

辛depicts a tool used to mark slaves and criminals.

辛い(からい)(adj-i) spicy, salty, harsh, hot, acrid

辛い(つらい)(adj-i) painful; heartbreaking; difficult.
Suffix づらい(adj-i) means ``difficult to do''.
読みづらい(adj-i) difficult to read.
書きづらい(adj-i) difficult to write.
読みづらい漢字difficult-to-read Han character.

\section{7 弟}

弟(おとうと)(hum) younger brother.
弟さん(おとうとさん)(hon) younger brother.

兄弟(キョウダイ)siblings;
brothers and sisters
(although the characters mean older brother and younger brother).

\section{7 言 9 信}

言contains口(mouth).

言(こと) say

言う(いう)to say

信(シン)faith; trust

信じる(シンじる)(v1,vt) to believe; to have faith in

\section{8 門 12 間 12 開 14 関}

12 間 interval

12 開 open

開く(ひらく)to open (door, business, eye, mouth, ...)

14 関(カン、せき)barrier; gate

\section{8 雨 11 雪 12 雲 13 雷電}

雪(ゆき)snow

雲(くも)cloud

雷(かみなり)thunder

電(デン) lightning

電光(デンコウ)lightning

電気(デンキ)electricity (lit. lightning spirit)

電話(デンワ)telephone (lit. lightning talk)

電車(デンシャ)electric train (lit. lightning carriage)

電撃(デンゲキ)electric shock

電気自動車(デンキジドウシャ)electric car

\section{11 魚}

魚(さかな)fish

魚介(ギョカイ)seafood; marine products

\section{気}

天気(テンキ)weather

気持ち(きもち)

\section{6 名}

名前(なまえ)
name; full name.
given name; first name.

\section{6 死}

死(シ)death.
死ぬ(しぬ)to die.

死去(シキョ)death

死亡(シボウ)death; mortality

\section{8 使(シ)use}

使う(つかう)to use

使用(シヨウ)(n) use

\section{4 反(ハン)anti-}

反する(ハンする)to oppose; to rebel; to revolt

反体制(ハンタイセイ)anti-establishment

\section{8 長}

長(チョウ)
long (distance or time).
leader.
eldest.

長い(ながい)long (distance); long (time)

長女(チョウジョ)eldest daughter; first-born daughter

市長(シチョウ)mayor (a government official)

身長(シンチョウ)height (of body)

最長(サイチョウ)longest, tallest

社長(シャチョウ)company president

\section{8 兒 7 児}

兒 is a simplification of 兒 depicting an infant
with an imperfect cranium (fontanelles).

乳児(ニュウジ)infant; suckling baby

男児(ダンジ)boy; son

\section{10 症}

症(ショウ)(n,suf) illness

\section{6 羽 10 弱 11 習}

6 羽(はね)feather

10 弱

弱い(よわい)weak

11 習 (the bottom character is自not日).

習う(ならう)(vt) to learn

練習(レンシュウ)training; practice

\section{6 曲}

曲(キョク)music

作曲(サッキョク)musical composition

\section{6 外}

6 外(ガイ)foreign

外人(ガイジン)foreigner, foreign person

外国(ガイコク)foreign country

外界(ガイカイ)outside world

海外(カイガイ)foreign; abroad; overseas

\section{7 形(かたち) shape; form}

\section{8 隹}

隹 depicts a short-tailed bird.

誰(だれ)who

\section{9 食(ショク) food, eat}

食べ物(たべもの)food

食物(ショクもの)food

食べる(たべる)(v1) to eat

飲む(のむ)to drink (any liquid, not just liquor)

\section{9 省}

4 少 + 5 目

省く(はぶく)(vt)
to omit; to leave out; to exclude.
to curtail; to save; to cut down; to economize.

省(ショウ)(suf) ministry; department

国交省(コッコウショウ)(abbr)
Ministry of Land, Infrastructure, Transport, and Tourism

\section{6 有}

有 depicts a hand holding a piece of 肉(meat).

有る(ある)to exist

\section{6 耂 8 者}

耂 depicts a bent-over figure with long hair, an old man.

者(シャ)(n,suf) someone of that nature; someone doing that work.
者(もの)(n) person (rarely used without a qualifier).
