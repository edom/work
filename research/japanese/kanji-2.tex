\chapter{Kanji 2}

7 言 say

\section{2 人 6 仲 7 作戻 10 涙}

仲間(なかま)friend (how does this differ from 友達(ともだち)?)

7 作(サク) work; harvest

作る(つくる)to make

7 戻る(もどる)(vi) to turn back; to return; to go back;
to come back to a previously visited place

10 涙(なみだ)tear (eyewater)

\section{2 了 5 字 8 学}

赤字(あかジ)deficit; red letter; red Han-character

科学(カガク)science

\section{4 日 9 星}

日付(ひづけ)date.日付別(ひづけベツ)separate by date.(context? usage?)

星(ほし)star

\section{4 文}

文書(ブンショ)sentence

文化(ブンカ)culture

\section{4 殳 (a hand holding a tool; weapon)}

This appears in
殺す(ころす)(to kill),
電撃(デンゲキ)(electric shock),
投げる(なげる)(to throw),
and 設ける(もうける)(to establish).

殺害(サツガイ)murder

殺人(サツジン)murder (legal term?)

\section{4 水 9 洗}

洗う(あらう)(vt) to wash

\section{5 出}

出来上がる(できあがる)(vi) to be finished; to be completed; to be ready

\section{5 石}

5 石(いし)stone

\section{6 会}

会う(あう)to meet (?)

会社(カイシャ)company; corporation

会社員(カイシャイン)company employee

年会(ネンカイ)yearly meeting; annual convention

社会(シャカイ)society

\section{6 安}

安い(やすい)cheap; inexpensive

安全(アンゼン)safety; security

\section{6 在}

存在(ソンザイ)

\section{6 先 8 洗}

洗う(あらう)(vt) to wash

先生(せんせい)teacher; master; doctor

\section{7 貝(かい)cowry}

(Cowry is a kind of seashell used as money in ancient China.)

買う(かう)to buy; to purchase

賣る(うる)to sell.
This kanji has been simplified to 売る.

\section{7 辛}

辛depicts a tool used to mark slaves and criminals.

辛い(からい)(adj-i) spicy, salty, harsh, hot, acrid

辛い(つらい)(adj-i) painful; heartbreaking; difficult.
Suffix づらい(adj-i) means ``difficult to do''.
読みづらい(adj-i) difficult to read.
書きづらい(adj-i) difficult to write.
読みづらい漢字difficult-to-read Han character.

\section{3 口 7 言 9 信 10 記 13 話 14 読語 15 誰}

7 言(こと) say

言contains口(mouth).

言う(いう)to say

言葉(ことば)word; dialect

9 信(シン)faith; trust

信じる(シンじる)(v1,vt) to believe; to have faith in

10 記(キ) record

記す(しるす)to record, to write down

記事(キジ)article (writing).
選り抜き記事(よりぬきキジ)selected articles.
新しい記事(あたらしいキジ)new articles.

記録(キロク)record

13 話 talk

話す(はなす)to talk

14 読 read

読む(よむ)to read

14 語(ゴ) language

…語(…ゴ)... language

日本語(ニホンゴ)Japanese language

英語(エイゴ)English language

15 誰(だれ)who

\section{8 押}

押し(おし)(n) push

\section{8 門 12 間 12 開 14 関}

8 門(かど) gate

12 間 interval

12 開 open

開く(ひらく)to open (door, business, eye, mouth, ...)

14 関(カン、せき)barrier; gate

\section{8 知}

日本人の知らない日本語the Japanese language that the Japanese people don't know

\section{8 金 13 鉄 14 銀銅 16 鋼}

8 金(キン、かね)gold; money

金has a lot to do with metals.

金属(キンゾク)metal. This is also the chemistry term.

重金属(ジュウキンゾク)heavy metal. This is also the chemistry term.

金色(キンいろ)golden (color)

金色(コンジキ)golden (color)

13 鉄(テツ、くろがね)iron (lit. 黒金 black metal)

鉄道(テツドウ)railroad; railway

14 銀(ギン、しろがね)silver (lit. 白金 white metal)

14 銅(ドウ、あかがね)copper (lit. 赤金 red metal)

16 鋼(コウ、はがね)steel

青銅(セイドウ)bronze

鋼鉄(コウテツ)steel
