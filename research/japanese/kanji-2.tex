\chapter{Kanji 2}

\section{4 凶引公幻}

凶(キョウ)evil; villain; bad luck; disaster.
凶悪(キョウアク)(adj-na) atrocious; fiendish; brutal; villainous.

引(イン)pull.
引用(インヨウ)quotation; citation; reference.
引く(ひく)(vi,vt) to pull.

公(おおやけ)public; communal; official; governmental.
公安(コウアン)public safety; public welfare.

幻(まぼろし)phantom; vision; illusion; dream

\section{5 冬}

冬(ふゆ)winter

\section{5 犯打圧可刊付幼込広写}

犯(ハン)crime.
犯す(おかす)to commit (a crime); to perpetrate (a crime).

打(ダ)strike; hit; knock; pound.
安打(アンダ)safe hit (baseball).
打つ(うつ)to hit; to strike; to knock; to beat; to punch; to slap.

圧(アツ)pressure.
気圧(キアツ)air pressure.
指圧(シアツ)finger pressure massage.

可(カ)passable; acceptable; tolerable.
許可(キョカ)permission; authorization; approval.

刊行(カンコウ)publication; issue.
月刊(ゲッカン)monthly publication; monthly issue.
夕刊(ユウカン)evening newspaper.

付ける(つける)(v1,vt) to attach, join, stick, glue, fasten.
一寸(ちょっと)just a minute; short time; just a little.
「一寸待って下さい。」``Please wait a moment.''
「一寸!」``Hey!''.
口付け(くちづけ)(n) kiss.
口付ける(くちづける)(v1) to kiss.
日付(ひづけ)date.

幼い(おさない)very young; immature. childish.

込む(こむ)to be crowded.

広い(ひろい)(adj-i) spacious; vast; wide.
広告(コウコク)advertisement.

写生(シャセイ)sketch.
写真(シャシン)multimedia; photograph; movie
写す(うつす)(vt) to photograph.

\section{6 早仲名伐式}

早い(はやい)(adj-i) early.

仲間(なかま)friend (how does this differ from 友達(ともだち)?)

名前(なまえ)
name; full name.
given name; first name.

伐(バツ)fell; strike; attack; punish.

式(シキ)style; ceremony; numerical formula.
公式(コウシキ)formal; official.
公式ブログ official blog.

\section{6 伝安宅次行先地巡考吸気}

伝(デン)transmit.
自伝(ジデン)autobiography.
手伝う(てつだう)to help; to assist; to take part in.
伝言(デンゴン)verbal message.
伝記(デンキ)life story.
伝道(デンドウ)proselytizing; evangelism; missionary work.
伝える(つたえる)to convey; to report; to transmit; to communicate.

安い(やすい)cheap; inexpensive.
安全(アンゼン)safety; security.
安心(アンシン)relief; peace of mind.

住宅(ジュウタク)residence; housing; residential building.
自宅(ジタク)one's home.
自宅火災(ジタクカサイ)house fire; home fire (disaster).

次(つぎ)next (in sequence)

行く(いく)(vi) to go.
行う(おこなう)(vt) to do.

先(さき)before; ahead; previous; future.
先生(せんせい)teacher; master; doctor

地(チ)land (that is being used for an activity).
空き地(あきチ)vacant land.
耕地(コウチ)arable land.
団地(ダンチ)multi-unit apartments.

巡る(めぐる)to go around.
お巡り(おまわり)policeman.

考える(かんがえる)(v1,vt) to consider; to think about.

吸う(すう)to suck with mouth

気(キ)spirit; mind; air; atmosphere
元気(ゲンキ)health(y); vigor; vitality; spirit.
天気(テンキ)weather.
気持ち(きもち)feeling.

\section{7 君防対}

君(きみ)you.
…君(…クン)(suffix) Mr. (junior).
君主(クンシュ)ruler; monarch; sovereign.

防(ボウ)defense; resistance.
防ぐ(ふせぐ)to resist; to defend against.

対(タイ)versus...
対する(タイする)to face each other.

\section{7 言告走見利私助災}

言contains口(mouth).
言(こと)saying.
言う(いう)to say.
言葉(ことば)word; dialect.

告げる(つげる)to inform; to tell.
告白(コクハク)confess (usually of love).

走る(はしる)(v5r,vi) to run

見る(みる)(v1,vt) to see.
見える(みえる)(v1,vi) to appear.

利(リ)
advantage; benefit; profit;
interest (the amount added on top of the principal that is paid back).

助(すけ)assistance.
助(ジョ)(pref) help; rescue; assistant.
助ける(たすける)(v1,vt) to help.

私(わたし)I; me

災い(わざわい)(n) calamity; catastrophe.
火災(カサイ)fire (disaster).

\section{7 姉困村社戻応医図}

姉(あね)(humble) older sister.
お姉さん(おねえさん)(honorific) older sister.

困る(こまる)(vi) to be troubled; to be embarrassed

村(むら)village

社(シャ)company; firm; association; shrine

戻る(もどる)(vi) to turn back; to return; to go back;
to come back to a previously visited place

応え(こたえ)response; reply; answer; solution.
応える(こたえる)(v1) to respond; to reply; to answer.

医(イ)medicine; healing; curing; doctor (medical)

図(ズ)map; drawing; picture; plan; illustration; diagram; figure; chart

\section{8 明所官易表供}

明(メイ)bright.
明るい(あかるい)bright.

所(ところ)place.
近所(キンジョ)neighborhood.
名所(メイショ)famous place.

官(カン)government official.
官界(カンカイ)bureaucracy.
長官(チョウカン)secretary; director; chief; director general.

易(エキ)easy; simple.
易しい(やさしい)(adj-i) easy; plain; simple.
交易(コウエキ)trade; commerce.
易い(やすい)(adj-i) easy (not difficult).

表示(ヒョウジ)manifestation; demonstration. display. representation.

供える(そなえる)(v1,vt) to offer; to sacrifice; to dedicate.
子供(こども)child.

\section{8 和知直使昂学乳空妻狙}

和(ワ)peace; Japan.
平和(ヘイワ)peace.
和む(なごむ)(vi) to be softened; to calm down.
和らげる(やわらげる)(v1,vt) to soften; to moderate; to relieve.

知る(しる)to know.
日本人の知らない日本語the Japanese language that the Japanese people don't know

直(ジキ)soon.
直(チョク)direct; in person; frankness; honesty.
直す(なおす)to heal; to cure.
Chinese 直 has 7 strokes.
Japan adds the lower-left corner stroke.

使用(シヨウ)(n) use.
使う(つかう)to use.

昂る(たかぶる)(vi) to get excited; to get worked up

学(ガク)learning, scholarship, erudition, knowledge.
中学(チュウガク)middle school; junior high school.
大学(ダイガク)university.
学界(ガッカイ)academic world.
科学(カガク)science.

乳(ちち)breasts.
授乳(ジュニュウ)breast-feeding.

空(そら)sky.
空港(クウコウ)airport.
空く(すく)(vi) to become less crowded; to get empty.
空く(あく)(vi) to be open; to be empty.

夫妻(フサイ)married couple; husband and wife

狙う(ねらう)(vt) to aim at

\section{8 定妹性青毒画炎味歩国}

定(テイ、ジョウ)fix; determine; establish; settle; decide.
安定(アンテイ)stability; equilibrium.
未定(ミテイ)not yet fixed; undecided; pending.
定住(テイジュウ)settlement; permanent residence.
定める(さだめる)(v1,vt) to decide; to establish; to determine.

妹(いもうと)(humble) younger sister.
妹さん(いもうとさん)(honorific) younger sister.

性(セイ)nature; sex; gender.
男性(だんせい)male.
女性(じょせい)female.

青(あお)(n) blue; green.
青い(あおい)(adj-i) blue; green.
青ざめる(あおざめる)(v1,vi) to become pale.

毒(ドク)poison.
毒ガス(ドクガス)poison gas.

画 means picture.
画(カク)counter for kanji strokes.
字画(ジカク)number of strokes in a character.
画家(ガカ)painter.

炎(ほのお)flame, blaze.
炎天(エンテン)scorching sun.

味(あじ)flavor; taste.
美味しい(おいしい)delicious.

歩く(あるく)to walk

国(コク、くに) country.
国王(コクオウ)king.
国内(コクナイ)internal; domestic.

\section{9 送音専負臭査変保郎飛室狭紅}

送る(おくる)to send.
見送る(みおくる)to see off; to escort (for parting); to farewell.
先送る(さきおくる)to postpone.

音(オン、おと、ね)sound.

専門家(センモンカ)expert; specialist

負(フ)negative; minus.
負う(おう)to bear; to carry on one's back.
負かす(まかす)(vt) to defeat.
負ける(まける)(v1,vi)
to lose; to be defeated.
to succumb; to give in; to surrender; to yield.
to be inferior to.

臭: In China 10 strokes, in Japan 9 strokes.
The lower character is 犬 in China and 大 in Japan.
臭い(くさい)(adj-i) stinking; malodorous; ill-smelling.

査(サ)investigate.
巡査(ジュンサ)policeperson.
主査(シュサ)chief examiner; chief investigator.
査問(サモン)enquiry; hearing.

変(ヘン)strange.
変わる(かわる)to change; to transform.
変身(ヘンシン)metamorphosis; transformation.
大変(タイヘン)(adv,adj-na,n)very.

保する(ホする)to guarantee.
保つ(たもつ)to preserve.
保安(ホアン)peace preservation; security.

郎(ロウ)son.
野郎(ヤロウ)rascal.

飛(ヒ)fly; jump.
飛ぶ(とぶ)to jump.

室(むろ)room.

狭める(せばめる)(v1,vt) to narrow.
狭い(せまい)(adj-i) narrow; confined; small.

紅(くれない)deep red; crimson

\section{9 待叙星相省美皇界}

待つ(まつ)(vt,vi) to wait; to wait for; to await.

叙(ジョ)confer; relate; narrate; describe.
自叙伝(ジジョデン)autobiography.

星(ほし)star

首相(シュショウ)prime minister

省(ショウ)(suf) ministry; department.
国交省(コッコウショウ)(abbr)
Ministry of Land, Infrastructure, Transport, and Tourism.
省く(はぶく)(vt)
to omit; to leave out; to exclude.
to curtail; to save; to cut down; to economize.

美 beauty.
美味しい(おいしい)delicious (idiosyncratic reading).
美しい(うつくしい)beautiful.

皇(コウ)imperial.
皇居(コウキョ)imperial palace.

界(カイ)world.
世界(セカイ)world.
学界(ガッカイ)academic world.
業界(ギョウカイ)industry world, business world.

\section{9 冒春秋政}

冒 depicts a hat obstructing the sight, implying rashness
(acting without enough thought).
冒す(おかす)(vt) to risk.
冒険(ボウケン)adventure.

春(はる)spring (season)

秋 depicts the burning of plant stalks (after harvest).
秋(あき)autumn; fall season.

政(セイ、まつりごと)rule; government.

\section{10 夏}

夏(なつ)summer

\section{10 配党娘息舐能連笑料家神紙症弱}

配(ハイ)distribute.
配本(ハイホン)distribution of books.
配送センター(ハイソウセンター)distribution center.
配る(くばる)to distribute; to deliver.

党(トウ)political party; faction.
党首(トウシュ)political party leader.

娘(むすめ)daughter

息子(むすこ)son

舐める(なめる)(v1,vt) to lick

能(のう)talent; gift; function.

連(レン)connection; sequence; chaining.
連休(レンキュウ)consecutive holidays.
常連(ジョウレン)regular customer; regular patron.
連れる(つれる)(v1) to lead or take a person.
関連ニュース(カンレンニュース)related news.

笑う(わらう)to laugh.
笑む(えむ)to smile.

料(リョウ)(suffix) material; charge; rate; fee

家 has at least 3 meanings, depending on how it is read.

家(カ)-er; -ist; someone who does something.
書家(ショカ)calligrapher.
画家(ガカ)painter.
漫画家(マンガカ)Japanese-comic-book-drawing artist.
活動家(カツドウカ)activist.
研究科(ケンキュウカ)researcher.
作家(サッカ)author; creator; writer; artist.
小説家(ショウセツカ)novelist; fiction writer.
政治家(セイジカ)politician; statesman.
作曲家(サッキョクカ)music composer.
史家(シカ)historian.

家(うち)house.
「今夜私の家(うち)に来てください。」Please come to my house tonight.

家(ケ)family.
中川家(なかがわケ)the Nakagawa family.
田中家(たなかケ)the Tanaka family.
マッカーサー家(マッカーサーケ)the MacArthur family; the MacArthurs.

神(かみ)god; spirit; thunder

紙(かみ)paper.
手紙(てがみ)letter (the document, not the alphabet).

症(ショウ)(n,suf) illness

弱い(よわい)weak

\section{11 唾細現問部眼脱動習得雪}

唾(つば)spit

細い(サイ、ほそい)(adj-i) thin; slender.
細る(ほそる)to become thin.
細かい(こまかい)(adj-i) small; trivial.

現す(あらわす)(vt) to reveal; to show; to display.
現れる(あらわれる)(v1,vi) to appear; to become visible; to materialize.

問(モン)(suffix, counter) counter for questions.
問う(とう)to ask (a question).
質問(シツモン)question; inquiry; enquiry.
問題(モンダイ)problem.

部(ブ)section; department; part.
市部(シブ)urban areas.
学部(ガクブ)a department in a faculty in a university.
東部(トウブ)eastern part.
一部.

眼(まなこ)eyeball.
両眼(リョウガン)both eyes.

脱出(ダッシュツ)escape.
The left component is 肉.

動(ドウ)motion.
動く(うごく)(vi) to move.
動画(ドウガ)animation, motion picture.
自動車(ジドウシャ)automobile.
動力(ドウリョク)power; motive power.

習う(ならう)(vt) to learn.
練習(レンシュウ)training; practice.
Note that the bottom character is 自, not 日.

得る(える)(v1,vt) to get; to acquire; to obtain; to earn; to win; to gain; to secure; to attain.
納得(ナットク)understanding, agreement.
納得するto consent, agree; to understand the reason for something.

雪(ゆき)snow

\section{12 覚集道陽間開飲}

覚める(さめる)(v1,vi) to wake up

集(シュウ)gather; meet; congregate; swarm; flock.
集める(あつめる)(v1,vt) to collect; to assemble; to gather.

道(みち)street; road.
鉄道(テツドウ)railway.

陽(ヨウ)the yang in yin and yang.
太陽(タイヨウ)sun.

間(カン、ケン)interval; space.
人間(ニンゲン)human being.

開く(ひらく)to open (door, business, eye, mouth, ...)

飲む(のむ)to drink (any liquid, not just liquor)

\section{12 雲番朝塔}

雲(くも)cloud

番組(バンぐみ)television program

朝(あさ)morning.
今朝(けさ)this morning.
早朝(ソウチョウ)early morning.

塔(トウ)tower.
管制塔(カンセイトウ)control tower.

\section{13 罪雷電稚}

罪(ザイ)guilt.
罪(つみ)crime; fault; sin.
犯罪(ハンザイ)crime.

雷(かみなり)thunder.
電(デン)lightning.
電光(デンコウ)lightning.
電気(デンキ)electricity (lit. lightning spirit).
電話(デンワ)telephone (lit. lightning talk).
電車(デンシャ)electric train (lit. lightning carriage).
電気自動車(デンキジドウシャ)electric car.

稚(チ)immature; young.
幼稚(ヨウチ)infancy; childish; infantile.

\section{14 聞歌駅関}

聞く(きく)to hear

歌う(うたう)to sing

駅 is simplified from 23 驛.
駅(エキ)train station.

関(カン、せき)barrier; gate

\section{Cardinal directions: 5 北 6 西 8 東 9 南}

北(ホク、きた)north

西(セイ、にし)west

東(トン、ひがし)east

南(ナン、みなみ)south

They combine as in English.

北西(ホクセイ)northwest

北東(ホクトウ)northeast

東北(トウホク)Touhoku (a prefecture)

南西(ナンセイ)southwest

南東(ナントウ)southeast

\section{Thing: 6 件 8 物事}

件(ケン)matter; case; item

物(ブツ、モツ、もの)thing; object; matter.
物語る(ものがたる)(vt) to tell; to indicate.
書物(ショモツ)books.
食べ物(たべもの)food.

仕事(シごと)(n) work; job; business; occupation; employment.
火事(カジ)fire (as a disaster).
ラメン店で火事fire at a ramen shop.
有事(ユウジ)emergency.
無事(ブジ)safety; peace; quietness.

\section{Money: 10 員 11 側 12 買 15 賞賣(売)}

員(イン)(suffix) member.
工員(コウイン)factory worker.
会社員(カイシャイン)company employee.

側(がわ、かわ)side

側(そば)vicinity; near; beside

買う(かう)to buy; to purchase

賞(ショウ)prize; award
受賞(ジュショウ)winning (a prize).

賣る(うる)to sell.
This kanji has been simplified to 7 売る.

\section{Metals: 8 金 13 鉄 14 銅銀 16 鋼}

金has a lot to do with metals.
金(キン、かね)gold; money.

金属(キンゾク)metal.
重金属(ジュウキンゾク)heavy metal.
These are also the chemistry terms.

金色(キンいろ、コンジキ)golden (color)

鉄(テツ、くろがね)iron (lit. 黒金 black metal).
鉄人(てつジン)iron man; strong man.

鉄道(テツドウ)railroad; railway

銅(ドウ、あかがね)copper (lit. 赤金 red metal)

銀(ギン、しろがね)silver (lit. 白金 white metal)

鋼(コウ、はがね)steel

青銅(セイドウ)bronze

鋼鉄(コウテツ)steel

\section{Firearms and mirror: 14 銃 19 鏡}

銃(ジュウ)gun; small firearms

鏡(かがみ)mirror.
眼鏡(めがね)eyeglasses.

\section{People}

\subsection{Feelings: 6 好 8 委 13 嫌}

好き(すき)like; love; prefer

委員(イイン)committee member.
委ねる(ゆだねる)(v1,vt) to entrust to.

嫌い(きらい)hate

\section{Occupation: 5 仕 8 者}

仕事(シごと)work.
仕方(シかた)way; method; manner.
仕方ないit can't be helped; there's no other way.

者(シャ)(n,suf) someone of that nature; someone doing that work.
者(もの)(n) person (rarely used without a qualifier).
学者(ガクシャ)scholar.
業者(ギョウシャ)trader; merchant.
研究者(ケンキュウシャ)researcher.
作者(サクシャ)author.

Difference between 者 and 家(カ): ???

\section{Water}

\subsection{Feelings: 7 冷 12 温}

These adjectives describe the feeling when
touching something, not of weather or wind.

冷たい(つめたい)(adj-i) cold (of a tangible object)

温かい(あたたかい)(adj-i) warm (of a tangible object)

\subsection{State: 9 洪}

洪水(コウズイ)(n) flood (of liquid)

大水(おおみず)(n) flood (of liquid)

\subsection{Places: 6 江 8 沼 9 海津}

江(え)inlet; bay

沼(ぬま)swamp; bog

海(うみ)sea; beach.
The original character has 10 strokes.
Shinjitai replaces the two dots in the middle
with one vertical stroke.

津(つ)seaport; harbor.
津波(つなみ)(n) tsunami; tidal wave.

\subsection{Bodily fluids: 6 汗 10 涙}

汗(あせ)(n) sweat.
汗をかく(exp,v5k) to sweat.
汗を流す(exp,v5s) to work hard; to sweat.

涙(なみだ)tear (eyewater)

\subsection{Action done by the liquid: 8 波 10 凍流}

波(なみ)(n) wave (of liquid)

凍る(こおる)to freeze.
But the kanji for for ice is 氷(こおり).

流す(ながす)to flow (liquid)

\subsection{Action done to or with the liquid: 7 没沈 8 泳 9 洗 10 浮}

没(ボツ)drowning

沈下(チンカ)sinking; subsidence.
沈む(しずむ)(vi) to sink (descend into liquid).
This was simplified from 18 瀋.

泳ぐ(およぐ)(vi) to swim

洗う(あらう)(vt) to wash

浮かぶ(うかぶ)to float (be supported by liquid)

\subsection{Drinks: 10 酒}

酒(さけ)sake (a Japanese liquor)

\subsection{14 滴漏}

滴(しずく)a drop of water; a drip.
滴る(したたる)to drip (fall one drop at a time).

漏れる(もれる)to leak (liquid)

\section{Heart: 4 心 5 必}

心 is involved in a lot of feeling-related characters.
心(シン、こころ)heart.
心配(シンパイ)(adj-na,n,vs) worry, concern, anxiety.
心配(シンパイ)(n,vs) care, help.

必 is unrelated to 心. They only look similar.
必ず(かならず)(adv) always, invariably, certainly.
必要(ヒツヨウ)(adj-na,n) necessity, need.

\subsection{Thoughts: 7 忘 9 思}

忘れる(わすれる)(v1) to forget.
忘年会(ボウネンカイ)year-end party
(lit. forget-year meeting, a meeting to forget the year).

思う(おもう)to think

\subsection{Feelings: 12 悲 13 意感}

悲しい(かなしい)sad.
悲恋(ヒレン)disappointed love

意(イ)feelings; thoughts.
意欲(イヨク)motivation; will.
意味合い(イミあい)implication; nuance
小生意気(こなまイキ)cheekiness; impudence.

感じる(カンじる)(v1) to feel.

\subsection{Love: 10 恋 13 愛 17 優}

恋(レン、こい)romance; love; tender passion.
恋人(こいびと)lover; sweetheart.
恋文(こいぶみ)love letter.

愛(アイ)(n) love

優しい(やさしい)tender; kind; gentle; affectionate; suave

\subsection{Other: 9 急 11 悪 14 態}

急(キュウ)urgent, sudden, abrupt.
急ぐ(いそぐ)to hurry.

悪(アク)evil, wickedness.
悪人(アクニン)bad person, villain.
悪い(わるい)bad, poor; evil; unprofitable; at fault.

態(ざま)mess; sorry state; plight; sad sight.
変態(ヘンタイ)sexual perversion.

\section{Hand}

\subsection{Arm or hand movements: 7 投 9 指}

投げる(なげる)to throw

指(ゆび)finger.
指す(さす)(vt) to point.

\subsection{Ambiguously figurative: 11 探}

探 depicts a hand groping in a deep cave.
手探り(てさぐり)groping; fumbling.
探す(さがす)(vt)
to search for something lost.
to search for something desired.
探る(さぐる)to feel around for; to fumble for; to grope for.

\subsection{Figurative: 8 押 9 持 10 殺 11 設}

押す(おす)to push; to press; to cram into; to force.
to stamp.
to overwhelm.
押し(おし)(n) push.

持つ(もつ)to hold; to carry; to possess

殺す(ころす)to kill.
殺害(サツガイ)murder.
殺人(サツジン)murder.

設ける(もうける)to establish.

\subsection{Giving and taking: 8 取受 11 授}

取る(とる)(vt) take; fetch; take up.
買い取り(かいとり)purchase; sale. purchase on a non-return policy.

受ける(うける)(v1,vt) to receive

授ける(さずける)(v1,vt) to grant; to award.
授受(ジュジュ)give-and-receive.

\section{Blades}

\subsection{Tangible cutting: 4 切}

切 (spoon and sword).
切る(きる)(v5r) cut.
切(サイ、セツ).
一切(イッサイ)absolutely; (when used with negative) at all.
大切(タイセツ)(adj-na,n) important.

\subsection{Intangible cutting: 4 分}

分 depicts something separated by a blade.
分ける(わける)(v1,vt) to divide; to split; to share; to distribute.
1分(イップン)one minute.
12時34分(ジュウニジサンジュウヨンプン)12:34 (time).

\subsection{Swordtip: 4 方}

方 depicts the tip of a sword.

方(かた)(honorific) person.
あの方(あのかた)that person.

\subsection{Dissection: 10 剖}

剖(ボウ)dissection.

\subsection{Law: 6 刑 8 法 9 則 14 罰}

刑(ケイ)(n,n-suf) penalty; sentence; punishment

法(ホウ)law; rule; method; principle.

則(のり)law; rule; regulation.
法則(ホウソク)law; rule.

罰(バツ)punishment; penalty.
罰する(ばっする)to punish; to penalize.
罰金(バッキン)fine; monetary penalty.

\subsection{Separation: 7 別}

別 depicts sword cutting bone.
別な(ベツな)(adj-na) different; separate; another
日付別(ひづけベツ)separate by date.

\subsection{Reduction: 12 減}

減(ゲン)reduction; 10\%減 ten percent reduction.

\subsection{Burglar: 13 賊}

賊(ゾク)burglar; robber.
海賊(カイゾク)pirate; sea robber.

\subsection{War: 13 戦}

戦(いくさ)war.
内戦(ナイセン)civil war.
世界大戦(セカイタイセン)World War.

\section{Creation: つくる: 7 作 10 造}

作(サク)work; harvest.
作る(つくる)to make.
作る(つくる).

造る(つくる).
木造(モクゾウ)wooden; made of wood.
