\chapter{Kanji 2}

\section{2 厶 4 公 7 私}

厶 means I, myself, private.

公(おおやけ)public; communal; official; governmental.
公安(コウアン)public safety; public welfare.

私(わたし)I; me

\section{3 寸 5 付 7 村 10 時}

% https://en.wiktionary.org/wiki/%E5%AF%B8#Japanese
寸depicts a position on the forearm
where the pulse can be palpated by compressing the radial artery.
寸(スン)an ancient unit of length, approximately 3 cm.

付ける(つける)(v1,vt) to attach, join, stick, glue, fasten.

村(むら)village

時(とき)time.
時代(ジダイ)era.
三国時代(サンゴクジダイ)The Three Kingdoms period.
戦国時代(センゴクジダイ)The Warring States period.

\section{3 幺 6 糸}

幺 depicts a tiny/small/short thread.

糸 combines 幺 and 小.
糸(いと)thread.

The left-side part of 細is the left-side form of 糸.

\subsection{4 幻 9 紅 10 紙}

幻(まぼろし)phantom; vision; illusion; dream

紅(くれない)deep red; crimson

紙(かみ)paper

手紙(てがみ)letter (the document, not the alphabet)

\subsection{15 線}

打線(ダセン)baseball lineup

\section{4 牛 6 件}

件(ケン)matter; case; item

\section{4 斤 7 近 13 新}

The character 斤(キン)depicts (but does not mean) an axe.

新(シン) new

(cut tree down with an axe)

新しい(あたらしい)new

新聞(シンブン)newspaper (lit. new hear)

新車(シンシャ)new car

\section{5 且 7 助 8 狙}

且つ(かつ)and.
且又(かつまた)besides; furthermore; moreover

助(すけ)assistance

助(ジョ)(pref) help; rescue; assistant

狙う(ねらう)(vt) to aim at

\section{5 可 10 哥 14 歌}

可 passable
許可(キョカ)permission; authorization; approval.

哥 means older brother.
This character is not used on its own in Japan.

歌う(うたう)to sing

\section{5 生 8 青毒性 11 清}

性(セイ)nature; sex; gender

男性(だんせい)male

女性(じょせい)female

青(あお)(n) blue; green.
青い(あおい)(adj-i) blue; green.
青ざめる(あおざめる)(v1,vi) to become pale.

毒(ドク)poison.
毒ガス(ドクガス)poison gas.

清い(きよい)clear; pure; noble

\section{5 史 8 使}

史家(シカ)historian.

使用(シヨウ)(n) use.
使う(つかう)to use.

\section{5 矢 7 医 8 知}

矢(や)arrow

医(イ)medicine; healing; curing; doctor (medical)

日本人の知らない日本語the Japanese language that the Japanese people don't know

\section{6 合会}

What is the difference between 合 and 会?

合う(あう)to meet

会う(あう)to meet (?)

会社(カイシャ)company; corporation

会社員(カイシャイン)company employee

年会(ネンカイ)yearly meeting; annual convention

社会(シャカイ)society

会見(カイケン)interview

\section{6 共 8 供}

共(とも)companion; follower; attendant; retinue

子供(こども)child

\section{6 全}

全 depicts a whole piece of jade.

全(ゼン) whole

全部(ゼンブ)altogether; everything

全く(まったく)(adv) completely, entirely, wholly, totally

\section{6 肉 11 脱}

脱出(ダッシュツ)escape

\section{6 先 8 洗}

先生(せんせい)teacher; master; doctor

洗う(あらう)(vt) to wash

\section{6 在存有}

有 depicts a hand holding a piece of 肉(meat).
有る(ある)to exist.

存じる(ゾンじる)(v1,humble) to think, feel, consider, know.
存在(ソンザイ)existence; being.
共存(キョウゾン)coexistence.
存亡(ソンボウ)life-or-death; existence; destiny.

\section{7 辛}

辛depicts a tool used to mark slaves and criminals.

辛い(からい)(adj-i) spicy, salty, harsh, hot, acrid

辛い(つらい)(adj-i) painful; heartbreaking; difficult.
Suffix づらい(adj-i) means ``difficult to do''.
読みづらい(adj-i) difficult to read.
書きづらい(adj-i) difficult to write.
読みづらい漢字difficult-to-read Han character.

\section{8 門 12 間 12 開 14 関}

門(かど) gate

間 interval

開く(ひらく)to open (door, business, eye, mouth, ...)

関(カン、せき)barrier; gate

\section{8 長}

長(チョウ)
long (distance or time).
leader.
eldest.

長い(ながい)long (distance); long (time)

長女(チョウジョ)eldest daughter; first-born daughter

市長(シチョウ)mayor (a government official)

身長(シンチョウ)height (of body)

最長(サイチョウ)longest, tallest

社長(シャチョウ)company president

\section{10 症}

症(ショウ)(n,suf) illness

\section{7 形(かたち) shape; form}

\section{8 隹}

隹 depicts a short-tailed bird.

誰(だれ)who

\section{9 食(ショク) food, eat}

食べ物(たべもの)food

食物(ショクもの)food

食べる(たべる)(v1) to eat

飲む(のむ)to drink (any liquid, not just liquor)
