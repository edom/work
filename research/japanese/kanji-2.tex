\chapter{Kanji 2}

\section{2 入ヒ七匕 4 区凶 5 兄}

入(ニュウ)enter.

入る(いる)(v5i,vt)
to enter; to go into; to get into.

入力(ニュウリョク)input. data entry.

ヒ is the katakana ``hi''.

七(シチ、なな)seven.

匕(ヒ、さじ)spoon.

区(ク)ward; district (an administrative division).

凶(キョウ)evil; villain; bad luck; disaster.

凶悪(キョウアク)(adj-na) atrocious; fiendish; brutal; villainous.

兄(ケイ、キョウ、あに)older brother.

兄(あに)(humble) older brother.

お兄さん(おにいさん)(honorific) older brother.

\subsection{(2 入) 4 内 5 込}

内(ナイ、うち)inside.

込む(こむ)to be crowded.

\subsection{(2 入) 6 肉}

肉 depicts the ribs of an animal's torso.

肉(ニク)meat; flesh; body (as opposed to spirit).

肉 is sometimes corrupted to 月.

\subsection{(2 匕) 5 北}

北(ホク、きた)north

\subsection{(2 匕) 4 化 7 花}

化(カ、ばけ)change.

化(カ)(suffix) -ization, -ification.

化学(カガク)chemistry.

化石(カセキ)fossilization.

分化(ブンカ)specialization.

化ける(ばける)(v1,vi) to take the form of.

花(はな)flower

\subsection{(2 匕) 4 比 9 皆}

比 depicts two men.

比(ヒ)compare.

皆(カイ、みな)all.

\subsection{(5 兄) 10 党}

党(トウ)political party; faction.

党首(トウシュ)political party leader.

\subsection{(6 肉 4 心 2 匕) 10 能 14 態}

能(のう)talent; gift; function.

態(ざま)mess; sorry state; plight; sad sight.

変態(ヘンタイ)sexual perversion.

\subsection{(7 兑 6 肉) 11 脱 14 説 15 閲}

兑 is not used on its own in Japanese.

脱出(ダッシュツ)escape.

説(セツ)theory.

説明(セツメイ)explanation.

閲(エツ)review.

\subsection{(6 肉 4 凶) 10 悩 11 脳}

悩(ノウ)trouble.

脳(ノウ)brain.

The left radical of 脳 is 肉.

脳内(ノウナイ)intracranial; inside the brain

\section{2 了}

了 depicts a wrapped baby with only head visible.

了(リョウ)finish.

\section{2 人 4 仁 5 代}

仁(ジン)kindness.

グローバル化(グローバルカ)globalization.

代(ダイ、タイ)substitute.

代わる(かわる)(vi) to substitute for.

\section{2 ト卜 6 外}

ト is the katakana ``to''.

卜 is the ``divination'' radical.

For the computer, those are different characters.

外(ガイ、ゲ、そと、ほか、はず)outside; foreign.

外(そと)outside; exterior. open air.

外(ほか)other (places and things); the rest.

外す(はずす)to unfasten. to remove. to leave. to miss (a target).

外人(ガイジン)foreigner, foreign person.

外国(ガイコク)foreign country.

外界(ガイカイ)outside world.

海外(カイガイ)foreign; abroad; overseas.

外見(ガイケン)outward apperarance.

\section{2 刀 5 刊 7 利判}

刊行(カンコウ)publication; issue.
月刊(ゲッカン)monthly publication; monthly issue.
夕刊(ユウカン)evening newspaper.

利(リ)
advantage; benefit; profit;
interest (the amount added on top of the principal that is paid back).

判(ハン、わか)judge.
判る(わかる)

\section{2 力 4 木介 5 田 9 重}

力(リョク、リキ、ちから)strength.

百人力(ヒャクジンリキ)tremendous strength.

木(モク、き)tree.

介(カイ)can mean mediation or shellfish.

仲介(チュウカイ)agency; intermediation.

介入(カイニュウ)intervention.

介助(カイジョ)help; assistance; aid.

魚介(ギョカイ)marine products; seafood.

田(た)rice field.

重 depicts a man carrying a bag.

重い(おもい)heavy (of weight).

\subsection{(2 力) 5 功 8 協}

功(コウ)achievement

協(キョウ)cooperation.

\subsection{(4 木) 8 林 12 森}

林(リン、はやし)woods; forest; copse; thicket.

森(シン、もり)forest.

森林(シンリン)forest woods.

\subsection{(4 木) 8 果 11 菓}

果(カ)fruit.

菓(カ)confectionery.

菓子(カシ)pastry; confectionery.

\subsection{(4 木) 6 米}

米(ベイ、マイ、こめ)rice.

\subsection{(4 木) 5 本 6 休 7 体}

本(ホン)book, counter for long cylindrical objects.
(In Ancient China, a book is a scroll.)

一本(イッポン)one long cylindrical object, one point.

日本(ニホン)Japan.

休(キュウ)rest.

休日(キュウジツ)day off; holiday.

休む(やすむ)to rest.

体(タイ、からだ)body.

肉体(ニクタイ)body; flesh.

重体(ジュウタイ)seriously ill; critically ill.

\subsection{(4 木) 5 禾 9 秋 11 菌}

禾 depicts a plant stalk.

秋 depicts the burning of plant stalks (after harvest).

秋(シュウ、あき)autumn; fall season.

菌(キン)fungus; germ; bacterium.

\subsection{(4 木) 9 茶}

茶(チャ、サ)tea.

\subsection{(4 木) 5 未末 7 来}

未 depicts a tree that has not fruited.

未(ミ)not yet.

来 depicts fruits hanging on a tree.

未来(ミライ)future (lit. not yet come).

末(マツ、すえ)end.

来(ライ)come; next.

来月(ライゲツ)next month.

来年(ライネン)next year.

「来年田中さんが日本に行きます。」Next year, Mr. Tanaka will go to Japan.
(``Next year'' means one year after the moment the speaker says it.)

来る(くる)to come.
This is an irregular verb.
The past form is 来た(きた).

\subsection{(4 木) 13 楽}

楽 depicts a wooden stringed musical instrument.

楽(ガク、ラク)pleasure.

楽天(ラクテン)optimism.

音楽(オンガク)music.

楽しい(たのしい)happy.

\subsection{(4 田) 7 町}

町(チョウ、まち)town.

小町(こまち)belle; town beauty.

\subsection{(4 田) 7 男 9 勇}

男(おとこ)man.

勇(ユウ)courage.

勇む(いさむ)to be in high spirits.

通(ツウ)pass through.

通る(とおる)to go by.

\subsection{(4 田) 9 界 12 堺}

界(カイ)world.

世界(セカイ)world.

学界(ガッカイ)academic world.

業界(ギョウカイ)industry world, business world.

堺(カイ、さかい)world.

\subsection{(4 田) 7 里 11 理野 12 量}

里(リ、さと)hometown.

理(リ、ことわり)reason.

心理(シンリ)state of mind; mentality; psychology.

料理(リョウリ)cooking; cookery; cuisine.

野(ヤ、の)field; plains; rustic.

量(リョウ)quantity.

大量(タイリョウ)large quantity.

\subsection{5 甲 8 押}

甲(コウ)armor.

押(オウ)push.

押し(おし)(n) push.

押す(おす)to push; to press; to cram into; to force.
to stamp.
to overwhelm.

\subsection{5 由 6 因 7 困}

由(ユ、ユウ、よし)cause; reason.

自由(ジユウ)(n,adj-na) freedom; liberty.
(exp) as it pleases you.

自由なソフトウェアlibre software.
\ruby{無}{ム}\ruby{料}{リョウ}ソフトウェアgratis (no-fee) software.

因(イン)cause; factor

因る(よる)(vi) to be caused by.

因みに(ちなみに)(conj) by the way.
Usually written ちなみに.

困(コン)quandary.

困る(こまる)(vi) to be troubled; to be embarrassed.

\subsection{(5 由) 8 抽届}

抽(チュウ)pluck.

抽出(チュウシュツ)selection (from a group); sampling.

届(カイ)deliver.

届ける(とどける)(v1,vt) to make sure of.

\subsection{11 動}

動(ドウ)motion.

不動(フドウ)immobile.

動画(ドウガ)animation, motion picture.

自動車(ジドウシャ)automobile.

動力(ドウリョク)power; motive power.

動く(うごく)(vi) to move.

\section{2 又 5 求}

又 depicts the right hand.

又(また)again; also.

求(キュウ)request.

求める(もとめる)to want

\subsection{4 収反 8 取}

収(シュウ)obtain; income (monetary).

年収(ネンシュウ)annual income.

収める(おさめる)to obtain.

反(ハン)anti-.

反する(ハンする)to oppose; to rebel; to revolt.

反体制(ハンタイセイ)anti-establishment.

取る(とる)(vt) take; fetch; take up.
買い取り(かいとり)purchase; sale. purchase on a non-return policy.

\subsection{4 攴支 9 度}

攴(ぼくづくり)represents folding chair.

支(シ)branch.

度(ド、たび)degrees.

\ruby{毎}{マイ}\ruby{回}{カイ}\ruby{来}{く}る\ruby{度}{たび}
every time the implied entity comes.

今度(コンド)
now; this time; this occurrence.
next time; another time.

もう一度(もうイチド)once more.

\subsection{(4 攴 5 求) 11 救}

救(キュウ)salvation.

\subsection{4 友 7 抜 13 暖}

友(ユウ、とも)friend.

抜(バツ、ぬ)slip out.
抜ける(ぬける)(v1,vi) to escape.

暖かい(あたたかい)(adj-i) warm; genial

\subsection{8 受 11 授}

受ける(うける)(v1,vt) to receive

授ける(さずける)(v1,vt) to grant; to award.
授受(ジュジュ)give-and-receive.

\section{2 勹 3 勺 5 包 8 抱}

勹 is Kangxi radical 20 meaning ``wrap''.

勺 depicts something in the spoon.

包(ホウ)wrap.

包む(つつむ)(vt) to wrap up; to pack.

抱(ホウ)embrace; hug.

抱く(だく)to embrace; to hug.

\subsection{(3 勺) 8 的 9 約}

的(テキ)(suffix) -like; typical.

男性的(ダンセイテキ)manly.

的(テキ、まと)mark; target.

目的(モクテキ)purpose; goal; aim; objective; intention.

約(ヤク)promise.

約束(ヤクソク)arrangement; promise; pact; engagement.

\subsection{6 危}

危(キ、あや)dangerous.

危急(キキュウ)emergency.

危険(キケン)danger.

危うい(あやうい)dangerous.

危ない(あぶない)dangerous.

\section{Cardinal directions: 9 南}

南(ナン、みなみ)south

They combine as in English.

北西(ホクセイ)northwest

北東(ホクトウ)northeast

東北(トウホク)Touhoku (a prefecture)

南西(ナンセイ)southwest

南東(ナントウ)southeast

\section{Thing: 6 件 8 物事}

件(ケン)matter; case; item

物(ブツ、モツ、もの)thing; object; matter.

書物(ショモツ)books.

食べ物(たべもの)food.

物語(ものがたり)tale; story; legend.

物語る(ものがたる)(vt) to tell; to indicate.

火事(カジ)fire (as a disaster).

ラメン店で火事fire at a ramen shop.

有事(ユウジ)emergency.

無事(ブジ)safety; peace; quietness.

\section{Blades}

\subsection{Tangible cutting: 4 切}

切 depicts spoon and sword.
切(セツ、サイ、き)cut.
切る(きる)(v5r) cut.
大切(タイセツ)(adj-na,n) important.
一切(イッサイ)absolutely; (when used with negative) at all.

\subsection{Intangible cutting: 4 分}

分 depicts something separated by a blade.
分(フン、ブン)minute.
1分(イップン)one minute.
12時34分(ジュウニジサンジュウヨンプン)12:34 (time).

分(わ)understand.
分かる(わかる)to be understood.

分ける(わける)(v1,vt) to divide; to split; to share; to distribute.

\subsection{Swordtip: 4 方}

方 depicts the tip of a sword.
方(ホウ、かた)direction.
方(かた)(honorific) person.
あの方(あのかた)that person.

\subsection{Separation: 7 別}

別 depicts sword cutting bone.
別な(ベツな)(adj-na) different; separate; another
日付別(ひづけベツ)separate by date.

\section{2 丁 5 庁打}

丁(ひのと)depicts a nail.
丁(チョウ)(suffix)
counter for long and narrow (relatively flat) thing
such as paper, guns, scissors, spades, hoes, guitars.
一丁(イッチョウ)one sheet; one page.
one serving (in a restaurant).
丁重(テイチョウ)polite; courteous; hospitable.
包丁(ホウチョウ)kitchen knife.

丁(チョウ)town section.

庁(チョウ)government office.

打(ダ)strike; hit; knock; pound.

安打(アンダ)safe hit (baseball).

打つ(うつ)to hit; to strike; to knock; to beat; to punch; to slap.

\section{2 刀 9 削前契 13 解}

削(サク)shave; sharpen; delete.

削る(けずる)to shave; to sharpen; to erase; to delete.

前(ゼン、まえ)before (time), in front of.

午前(ゴゼン)morning; before noon; a.m. (ante meridien).

契(ケイ)pledge.

契約(ケイヤク)contract.

契る(ちぎる)(v5r,vt) to pledge; to promise; to swear.

解(ゲ、カイ)untie.

了解(リョウカイ)understanding; roger that.

見解(ケンカイ)opinion; point of view.

専門家見解(センモンカケンカイ)expert opinion.

解熱(ゲネツ)alleviation of fever.

解約(カイヤク)cancellation of contract.

解禁(カイキン)lifting of a ban.

解く(とく)to solve; to answer. to untangle (hair).

\section{2 力 7 努 12 筋}

努める(つとめる)(v1,vt) to endeavor; to try; to strive.
努力(ドリョク)great effort; exertion; endeavor

筋肉(キンニク)muscle.

\section{2 人 7 伴}

伴(ハン、バン)consort.

伴う(ともなう)to accompany.

\section{Blades: 2 刀刂 3 弋 4 戈 5 戊 6 戌 9 咸}

刀(トン、かたな)sword.

刂 depicts a knife.

弋 depicts shooting with a bow and an arrow.

戈 depicts a spear-axe (a halberd).

戊 depicts a dagger-axe (an ancient Chinese weapon).

戌 depicts an axe.

咸 depicts an axe and a mouth.

\section{(4 戈) 12 減 13 戦賊}

減(ゲン)reduction; 10\%減 ten percent reduction.

戦(いくさ)war.
内戦(ナイセン)civil war.
世界大戦(セカイタイセン)World War.

賊(ゾク)burglar; robber.
海賊(カイゾク)pirate; sea robber.
