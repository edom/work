\chapter{Kanji 2}

\section{2 刀 4 切方}

4 切 (spoon and sword)

切る(きる)(v5r) cut

切(サイ、セツ)

一切(イッサイ)absolutely; (when used with negative) at all

大切(タイセツ)(adj-na,n) important

4 方(かた)(tip of a sword)

\section{2 人 6 仲 7 作戻 10 涙}

仲間(なかま)friend (how does this differ from 友達(ともだち)?)

7 作(サク) work; harvest

作る(つくる)to make

7 戻る(もどる)(vi) to turn back; to return; to go back;
to come back to a previously visited place

10 涙(なみだ)tear (eyewater)

\section{2 了 3 子 5 字 8 学}

男子(ダンシ)young man, boy

女子(ジョシ)young woman, girl

太子(タイシ)crown prince

赤字(あかジ)deficit; red letter; red Han-character

漢字(カンジ)Han character

科学(カガク)science

\section{2 力 9 重 11 動}

11 動(ドウ) motion

動く(うごく)(vi) to move

動画(ドウガ)animation, motion picture

自動車(ジドウシャ)automobile

動力(ドウリョク)power; motive power

\section{2 又 8 受}

受ける(うける)(v1,vt) to receive

\section{3 女 6 好 8 妹妻 13 嫌}

6 好き(すき)like; love; prefer

8 妹(いもうと)younger sister.
妹consists of 女(lady) and 未(not yet).

夫妻(フサイ)married couple; husband and wife

13 嫌い(きらい)hate

\section{3 幺 4 幻 9 紅 10 紙 11 細}

4 幻(まぼろし)phantom; vision; illusion; dream

紅(くれない)deep red; crimson

紙(かみ)paper

手紙(てがみ)letter (the document, not the alphabet)

細い(ほそい)thin; slender
The left-side part of 細is the left-side form of 糸.

\section{3 工}

工業(コウギョウ)manufacturing industry

工作(コウサク)work; construction; handicraft

\section{3 口 5 古兄 7 告 8 味}

5 古 old

古consists of 十(ten) and 口(mouth, generation).

5 古い(ふるい)old

古株(ふるかぶ)veteran; senior; old-timer

5 兄(あに)older brother

告げる(つげる)

告白(コクハク)confess (usually of love)

8 味(あじ)flavor; taste

美味しい(おいしい)delicious

\section{3 寸 5 付 7 村}

付ける(つける)(v1,vt) to attach, join, stick, glue, fasten

村(むら)village

口付け(くちづけ)、口づけ(くちづけ)(v1) to kiss

\section{4 水 6 汗 9 海 10 酒}

冷たい(つめたい)(adj) cold (to the touch)

沈む(しずむ)to sink (descend into liquid)

6 汗(あせ)(n) sweat

汗をかく(exp,v5k) to sweat

汗を流す(exp,v5s) to work hard; to sweat

酒(さけ)(water west) sake (a Japanese liquor)

9 海(うみ)sea; beach.
The original character has 10 strokes.
Shinjitai replaces the two dots in the middle
with one vertical stroke.

泳ぐ(およぐ)to swim

洪水(コウズイ)(n) flood (of liquid)

大水(おおみず)(n) flood (of liquid)

\section{4 日 9 星}

日付(ひづけ)date.日付別(ひづけベツ)separate by date.(context? usage?)

星(ほし)star

\section{4 文}

文書(ブンショ)sentence

文化(ブンカ)culture

\section{4 水 9 洗}

洗う(あらう)(vt) to wash

\section{4 王 5 主 9 美皇}

5 主(おも)chief; main; principal; important

主人公(シュジンコウ)hero; main character

9 美 beauty

美味しい(おいしい)delicious (idiosyncratic reading)

美しい(うつくしい)beautiful

9 皇(コウ)imperial

皇居(コウキョ)imperial palace

\section{4 手 9 持}

持つ(もつ)to hold; to carry; to possess

手洗い(てあらい)restroom; lavatory; toilet; a place for washing hands

\section{4 木 5 禾 6 米休 7 体困来 9 秋茶}

5 禾 depicts a plant stalk.

9 秋(あき)autumn; fall season.
The character depicts the burning of plant stalks (after harvest).

米(ベイ、こめ)rice

米国(ベイコク)United States of America

6 休 rest

休む(やすむ)to rest

7 体(タイ、からだ)body

肉体(ニクタイ)body; flesh.

9 茶(チャ) tea

7 困 trouble

困る(こまる)(vi) to be troubled; to be embarrassed
(example?)

\section{4 止 7 走 8 歩}

止める(やめる)(v1) to stop

止める(とめる)(v1) to stop

歩く(あるく)to walk

走る(はしる)(v5r,vi) to run

\section{4 心 7 応}

応え(こたえ)response; reply; answer; solution

応える(こたえる)(v1) to respond; to reply; to answer

\section{5 必}

5 必(ヒツ)

必 is unrelated to 心. They only look similar.

必ず(かならず)(adv) always, invariably, certainly

必要(ヒツヨウ)(adj-na); (n) necessity, need

\section{5 母 7 毎 8 海}

7 毎(マイ)every

毎日(マイニチ)everyday

毎月(マイゲツ、マイつき)every month

毎時(マイジ)every hour

毎回(マイカイ)every time

毎年(マイネン、マイとし)every year

8 海(うみ)sea; beach

\section{5 出}

出来上がる(できあがる)(vi) to be finished; to be completed; to be ready

\section{5 田 7 男 9 界}

7 男(おとこ)man

男is田(rice field) and 力(strength).

9 界(カイ)world

学界(ガッカイ)academic world

世界(セカイ)world

業界(ギョウカイ)industry world, business world

\section{5 目 6 自}

6 自(ジ) self

自ら(みずから)(adv) personally

自在(ジザイ)freely (at will)

自分(ジブン)self (context? example usage?)

\section{6 会}

会う(あう)to meet (?)

会社(カイシャ)company; corporation

会社員(カイシャイン)company employee

年会(ネンカイ)yearly meeting; annual convention

社会(シャカイ)society

会見(カイケン)interview

\section{6 安}

安い(やすい)cheap; inexpensive

安全(アンゼン)safety; security

安心(アンシン)relief; peace of mind

\section{6 血}

止血(シケツ)stop bleeding; hemostasis

\section{6 在}

存在(ソンザイ)

\section{6 共 8 供}

共(とも)companion; follower; attendant; retinue

子供(こども)child

\section{6 先 8 洗}

洗う(あらう)(vt) to wash

先生(せんせい)teacher; master; doctor

\section{7 貝 9 則}

7 貝(かい)cowry

(Cowry is a kind of seashell used as money in ancient China.)

則(のり)law; rule; regulation

\section{7 呆 9 保}

呆depicts a child.

呆れる(あきれる)(v1,vi) to be amazed, astonished, astounded.

保する(ほする)to guarantee

\section{7 辛}

辛depicts a tool used to mark slaves and criminals.

辛い(からい)(adj-i) spicy, salty, harsh, hot, acrid

辛い(つらい)(adj-i) painful; heartbreaking; difficult.
Suffix づらい(adj-i) means ``difficult to do''.
読みづらい(adj-i) difficult to read.
書きづらい(adj-i) difficult to write.
読みづらい漢字difficult-to-read Han character.

\section{3 口 7 言 9 信}

7 言(こと) say

言contains口(mouth).

言う(いう)to say

9 信(シン)faith; trust

信じる(シンじる)(v1,vt) to believe; to have faith in

\section{8 押}

押し(おし)(n) push

\section{8 門 12 間 12 開 14 関}

12 間 interval

12 開 open

開く(ひらく)to open (door, business, eye, mouth, ...)

14 関(カン、せき)barrier; gate

\section{8 知}

日本人の知らない日本語the Japanese language that the Japanese people don't know

\section{8 金 13 鉄 14 銀銅 16 鋼}

8 金(キン、かね)gold; money

金has a lot to do with metals.

金属(キンゾク)metal. This is also the chemistry term.

重金属(ジュウキンゾク)heavy metal. This is also the chemistry term.

金色(キンいろ)golden (color)

金色(コンジキ)golden (color)

13 鉄(テツ、くろがね)iron (lit. 黒金 black metal)

鉄道(テツドウ)railroad; railway

14 銀(ギン、しろがね)silver (lit. 白金 white metal)

14 銅(ドウ、あかがね)copper (lit. 赤金 red metal)

16 鋼(コウ、はがね)steel

青銅(セイドウ)bronze

鋼鉄(コウテツ)steel

鉄人(てつジン)iron man; strong man

\section{8 雨 11 雪 12 雲 13 雷電}

11 雪(ゆき)snow

12 雲(くも)cloud

13 雷(かみなり)thunder

13 電(デン) lightning

電光(デンコウ)lightning

電気(デンキ)electricity (lit. lightning spirit)

電話(デンワ)telephone (lit. lightning talk)

電車(デンシャ)electric train (lit. lightning carriage)

電撃(デンゲキ)electric shock

電気自動車(デンキジドウシャ)electric car

\section{9 変}

変(ヘン)strange

変身(ヘンシン)metamorphosis; transformation

大変(タイヘン)(adv,adj-na,n) very

\section{10 馬 14 駅}

馬(うま)horse

駅(エキ)train station. 駅is新字体of驛.

\section{6 肉 11 脱}

脱出(ダッシュツ)escape

\section{11 魚}

魚(さかな)fish

魚介(ギョカイ)seafood; marine products

\section{気}

天気(テンキ)weather

気持ち(きもち)

\section{6 死(シ)death}

死亡(シボウ)death; mortality

\section{8 使(シ)use}

使う(つかう)to use

使用(シヨウ)(n) use

\section{4 反(ハン)anti-}

反する(ハンする)to oppose; to rebel; to revolt

反体制(ハンタイセイ)anti-establishment

\section{8 長(チョウ)long (distance or time); leader}

長い(ながい)long (distance); long (time)

市長(シチョウ)mayor (a government official)

身長(シンチョウ)height (of body)

最長(サイチョウ)longest, tallest

社長(シャチョウ)company president

\section{6 羽(はね)feather 10 弱 11 習}

6 羽(はね)feather

10 弱

弱い(よわい)weak

11 習 (the bottom character is自not日).

習う(ならう)(vt) to learn

練習(レンシュウ)training; practice

\section{6 曲(キョク)music}

作曲(さっきょく)music

音楽(おんがく)music

楽しい(たのしい)happy

了解(りょうかい)understanding; roger that

\section{6 外}

6 外(ガイ)foreign

外人(ガイジン)foreigner, foreign person

外国(ガイコク)foreign country

外界(ガイカイ)outside world

海外(カイガイ)foreign; abroad; overseas

\section{7 形(かたち) shape; form}

\section{9 食(ショク) food, eat}

食べ物(たべもの)food

食物(ショクもの)food

食べる(たべる)(v1) to eat

飲む(のむ)to drink (any liquid, not just liquor)

\section{8 隹 (short-tailed bird)}

誰 (だれ)who

\section{10 書(ショ)write}

It's a hand holding a pen writing on paper.

Is this related to 事(ごと)?

書く(かく)to write
