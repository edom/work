\chapter{Kanji 2}

\section{2 人 4 仁化 5 代 7 作}

仁(ジン)kindness.

化(カ、ばけ)change.

化(カ)(suffix) -ization, -ification.

グローバル化(グローバルカ)globalization.

化学(カガク)chemistry.

化石(カセキ)fossilization.

分化(ブンカ)specialization.

化ける(ばける)(v1,vi) to take the form of.

代(ダイ、タイ)substitute.

代わる(かわる)(vi) to substitute for.

作(サク)work; harvest.

作る(つくる)to make.

\section{7 呆 9 保}

呆 depicts a child.

呆れる(あきれる)(v1,vi) to be amazed, astonished, astounded.

保する(ホする)to guarantee.

保つ(たもつ)to preserve.

保安(ホアン)peace preservation; security.

\section{5 史 6 吏 7 更 8 使 9 便}

史(シ)history.

史家(シカ)historian.

吏(リ)officer.

更(コウ)grow late.

更ける(ふける)(vi) to get late; to advance; to wear on.

使(シ)use.

使用(シヨウ)(n) use.

使う(つかう)to use.

便(ベン、ビン)convenience.

便利(ベンリ)convenient; handy; useful.

不便(フベン)inconvenience.

便所(ベンジョ)lavatory.

大便(ダイベン)feces; excrement; shit.

便り(たより)news; tidings; information; letter.

\section{5 召 8 沼 11 紹}

召す(めす)(honorific) to invite; to eat

沼(ぬま)swamp; bog

紹(ショウ)introduce.

紹介(ショウカイ)introduction; referral.

\section{5 皮 8 波 10 疲}

皮(かわ)skin; hide.

波(なみ)(n) wave (of liquid)

疲(ヒ)exhausted.

疲れる(つかれる)(vi) to get tired.

\section{8 非 10 俳 13 罪}

非(ヒ)un-; non-; negative; mistake; wrong.

非常(ヒジョウ)extraordinary; unusual

非ず(あらず)(exp) no; never mind.

俳(ハイ)haiku.

俳優(ハイユウ)actor; actress.

罪(ザイ)sin; guilt.
罪(つみ)sin; crime; fault.
犯罪(ハンザイ)crime.
七つの大罪(ななつのダイザイ)seven deadly sins.

\section{2 卜 6 外}

外(ガイ、ゲ、そと、ほか、はず)outside; foreign.

外(そと)outside; exterior. open air.

外(ほか)other (places and things); the rest.

外す(はずす)to unfasten. to remove. to leave. to miss (a target).

外人(ガイジン)foreigner, foreign person.

外国(ガイコク)foreign country.

外界(ガイカイ)outside world.

海外(カイガイ)foreign; abroad; overseas.

\section{2 刀 5 刊 7 利判}

刊行(カンコウ)publication; issue.
月刊(ゲッカン)monthly publication; monthly issue.
夕刊(ユウカン)evening newspaper.

利(リ)
advantage; benefit; profit;
interest (the amount added on top of the principal that is paid back).

判(ハン、わか)judge.
判る(わかる)

\section{2 力 5 功}

功(コウ)achievement

\section{2 又 4 収反友 7 抜}

収(シュウ)obtain; income (monetary).

年収(ネンシュウ)annual income.

収める(おさめる)to obtain.

反(ハン)anti-.

反する(ハンする)to oppose; to rebel; to revolt.

反体制(ハンタイセイ)anti-establishment.

友(ユウ、とも)friend.

抜(バツ、ぬ)slip out.
抜ける(ぬける)(v1,vi) to escape.

\section{(3 walking person radical) 6 行}

行(コウ、ギョウ、い、ゆ、おこな)go.
行く(いく)(vi) to go.
行く(ゆく)(vi) to go.
行う(おこなう)(vt) to do.

\section{(3 roof radical) 6 守安宅 8 空定官 9 室 10 家}

守(シュ)protect.

守る(まもる)to protect.

安全(アンゼン)safety; security.
安心(アンシン)relief; peace of mind.

安い(やすい)cheap; inexpensive; secure.

宅(タク)home; house; residence.
住宅(ジュウタク)residence; housing; residential building.
自宅(ジタク)one's home.
自宅火災(ジタクカサイ)house fire; home fire (disaster).

宅(いえ)home.

空(そら)sky.
空港(クウコウ)airport.
空く(すく)(vi) to become less crowded; to get empty.
空く(あく)(vi) to be open; to be empty.

定(テイ、ジョウ)fix; determine; establish; settle; decide.
安定(アンテイ)stability; equilibrium.
予定(ヨテイ)(n,vs) plan; arrangement; schedule; program.
定住(テイジュウ)settlement; permanent residence.
定める(さだめる)(v1,vt) to decide; to establish; to determine.
未定(ミテイ)not yet fixed; undecided; pending.

官 depicts many rooms in a building.
官(カン)government official.
官界(カンカイ)bureaucracy.
長官(チョウカン)secretary; director; chief; director general.

室(シツ、むろ)room.

家 has at least 3 meanings, depending on how it is read.

家(カ)-er; -ist; someone who does something.
書家(ショカ)calligrapher.
画家(ガカ)painter.
漫画家(マンガカ)Japanese-comic-book-drawing artist.
活動家(カツドウカ)activist.
研究科(ケンキュウカ)researcher.
作家(サッカ)author; creator; writer; artist.
小説家(ショウセツカ)novelist; fiction writer.
政治家(セイジカ)politician; statesman.
作曲家(サッキョクカ)music composer.
史家(シカ)historian.

家(ケ)family.
中川家(なかがわケ)the Nakagawa family.
田中家(たなかケ)the Tanaka family.
マッカーサー家(マッカーサーケ)the MacArthur family; the MacArthurs.

家(うち)house.
「今夜私の家(うち)に来てください。」Please come to my house tonight.

\section{3 也 5 他 6 地}

也 means ``also''.

他(タ、ほか)other.

他人(タニン)another person; other people; stranger.

地(チ、ジ)land (that is being used for an activity).

空き地(あきチ)vacant land.

耕地(コウチ)arable land.

団地(ダンチ)multi-unit apartments.

\section{3 女 6 好 7 姉 8 妹委妻 10 娘}

好き(すき)like; love; prefer

姉(シ、あね)older sister.
姉(あね)(humble) older sister.
お姉さん(おねえさん)(honorific) older sister.

妹(マイ)younger sister.
妹(いもうと)(humble) younger sister.
妹さん(いもうとさん)(honorific) younger sister.

委員(イイン)committee member.
委ねる(ゆだねる)(v1,vt) to entrust to.

夫妻(フサイ)married couple; husband and wife

娘(むすめ)daughter

\section{3 口 5 古 6 名吸 7 言告 8 味 11 唾}

古 consists of 十(ten) and 口(mouth, generation).
古(コ、ふる)old.
古い(ふるい)old (not of person); ancient; obsolete.

名前(なまえ)
name; full name.
given name; first name.
名手(メイシュ)expert.

吸(キュウ、す)suck.
吸う(すう)to suck with mouth

言contains口(mouth).
言(ゲン、ゴン).
言(こと)saying.
言う(いう)to say.
言葉(ことば)word; dialect.

告げる(つげる)to inform; to tell.
告白(コクハク)confess (usually of love).

味(あじ)flavor; taste.
美味しい(おいしい)delicious.

唾(つば)spit

\section{(3 walking radical) 6 巡 7 近 9 送 10 速 12 道達}

巡る(めぐる)to go around.
お巡り(おまわり)policeman.

近い(ちかい)(adj-i) near (spatial distance).
近々(ちかぢか)soon.
近作(キンサク)recent work.
最近(サイキン)most recent; recently; these days; nowadays.

送る(おくる)to send.
見送る(みおくる)to see off; to escort (for parting); to farewell.
先送る(さきおくる)to postpone.

速(ソク)fast.

早速(サッソク)(adv) immediately.

急速(キュウソク)rapid (progress).

速い(はやい)fast.

道(ドウ、みち)street; road.
鉄道(テツドウ)railway.

達(タツ)attain.

\section{(4 god radical) 7 社 10 神}

社(シャ)company; firm; association; shrine

社(やしろ)shrine (usually Shinto).

神(かみ)god; spirit; thunder

\section{4 殳 7 投没 10 殺 11 設}

殳 depicts a tool or a weapon.

投げる(なげる)to throw

没(ボツ)drowning

殺す(ころす)to kill.
殺害(サツガイ)murder.
殺人(サツジン)murder.

設ける(もうける)to establish.

\section{4 戸 5 冊 8 雨 : 7 戻 所 10 涙 14 漏 15 編}

戸(コ、と、べ)door.

神戸(こうべ)Koube (an area in Japan).

冊(サツ)counter for books.

冊子(サッシ)booklet.

小冊子(ショウサッシ)booklet; pamphlet.

一冊(イッサツ)one book.

雨(あ\め)rain

戻る(もどる)(vi) to turn back; to return; to go back;
to come back to a previously visited place

所(ところ)place.

近所(キンジョ)neighborhood.

名所(メイショ)famous place.

涙(なみだ)tear (eyewater)

漏れる(もれる)to leak (liquid)

編(ヘン)editing; compilation.

編集(ヘンシュウ)Edit (menu item in computer user interface).

編む(あむ)(vt)
to knit; to plait; to braid.
to compile (an anthology); to edit.

\section{4 勿 8 易 12 陽}

勿(ブツ)

易(エキ)easy; simple.

易しい(やさしい)(adj-i) easy; plain; simple.

交易(コウエキ)trade; commerce.

易い(やすい)(adj-i) easy (not difficult).

陽(ヨウ)the yang in yin and yang.

太陽(タイヨウ)sun.

\section{4 日 6 早 8 明 9 春星 12 朝}

早い(はやい)(adj-i) early.

明(メイ)bright.
明るい(あかるい)bright.

春(シュン、はる)spring (season)

星(セイ、ほし)star

朝(チョウ、あさ)morning.
今朝(けさ)this morning.
早朝(ソウチョウ)early morning.

\section{4 公 5 広 7 私}

公(コウ、おおやけ)public; communal; official; governmental.
公安(コウアン)public safety; public welfare.

広(コウ、ひろ)wide.
広い(ひろい)(adj-i) spacious; vast; wide.
広告(コウコク)advertisement.

私(シ、わたくし、わたし)I; me

\section{4 凶}

凶(キョウ)evil; villain; bad luck; disaster.

凶悪(キョウアク)(adj-na) atrocious; fiendish; brutal; villainous.

\section{4 幻 5 幼}

幻(まぼろし)phantom; vision; illusion; dream

幼(ヨウ、おさな)infancy.
幼い(おさない)very young; immature. childish.

\section{4 手 8 抽 11 探}

抽(チュウ)pluck.
抽出(チュウシュツ)selection (from a group); sampling.

探 depicts a hand groping in a deep cave.
手探り(てさぐり)groping; fumbling.
探す(さがす)(vt)
to search for something lost.
to search for something desired.
探る(さぐる)to feel around for; to fumble for; to grope for.

\section{4 水 Places: 6 江池}

江(え)inlet; bay

池(チ、いけ)pond

\section{4 水 Drinks: 10 酒}

酒(さけ)sake (a Japanese liquor)

\section{5 包 8 抱}

包(ホウ)wrap.

包む(つつむ)(vt) to wrap up; to pack.

抱(ホウ)embrace; hug.

抱く(だく)to embrace; to hug.

\section{6 危}

危(キ、あや)dangerous.

危急(キキュウ)emergency.

危険(キケン)danger.

危うい(あやうい)dangerous.

危ない(あぶない)dangerous.

\section{5 冬 10 夏 11 終}

冬(トウ、ふゆ)winter

夏(カ、なつ)summer

終(シュウ)end.

最終(サイシュウ)last; final; closing.

終了(シュウリョウ)end; close; termination.

終わる(おわる)to finish; to end; to close.

\section{7 応}

応え(こたえ)response; reply; answer; solution.
応える(こたえる)(v1) to respond; to reply; to answer.

\section{5 犯 8 狙 9 狭}

犯(ハン、おか)crime.
犯す(おかす)to commit (a crime); to perpetrate (a crime).

狙う(ねらう)(vt) to aim at

狭める(せばめる)(v1,vt) to narrow.
狭い(せまい)(adj-i) narrow; confined; small.

\section{5 写}

写生(シャセイ)sketch.
写真(シャシン)multimedia; photograph; movie
写す(うつす)(vt) to photograph.

\section{(5 目) 9 相省冒 11 眼}

相(ソウ、あい)mutual.
相手(あいて)companion; partner; company.

相(ショウ)minister.
首相(シュショウ)prime minister; chancellor; premier.

省(ショウ)(suf) ministry; department.
国交省(コッコウショウ)(abbr)
Ministry of Land, Infrastructure, Transport, and Tourism.
省く(はぶく)(vt)
to omit; to leave out; to exclude.
to curtail; to save; to cut down; to economize.

冒 depicts a hat obstructing the sight, implying rashness
(acting without enough thought).
冒す(おかす)(vt) to risk.
冒険(ボウケン)adventure.

眼(まなこ)eyeball.
両眼(リョウガン)both eyes.

\section{7 見 11 現 12 覚}

見る(みる)(v1,vt) to see.
見える(みえる)(v1,vi) to appear.

現す(あらわす)(vt) to reveal; to show; to display.
現れる(あらわれる)(v1,vi) to appear; to become visible; to materialize.

覚める(さめる)(v1,vi) to wake up

\section{6 舌 9 活 10 舐 13 話}

舌(ゼツ、した)tongue.

活(カツ)active.

活きる(いきる)

舐める(なめる)(v1,vt) to lick

話(ワ、はなし)talk.

会話(カイワ)conversation.

話す(はなす)to talk.

\section{6 寺 9 待持 10 時特}

寺(ジ、てら)Buddhist temple.

待つ(まつ)(vt,vi) to wait; to wait for; to await.

持つ(もつ)to hold; to carry; to possess

時(ジ、とき)time.

時代(ジダイ)era.

三国時代(サンゴクジダイ)The Three Kingdoms period.

戦国時代(センゴクジダイ)The Warring States period.

特(トク)special.

特に(トクに)particularly; especially.

特技(トクギ)special skill.

特待(トクタイ)special treatment.

特待生(トクタイセイ)scholarship student.

\section{6 交 10 校}

交(コウ、まじ)intersect.
交わる(まじわる)cross; intersect; join; meet.

国交(コッコウ)diplomatic relations

学校(ガッコウ)school

\section{6 式}

式(シキ)ceremony.

式(シキ)style.
公式(コウシキ)formal; official.
公式ブログ official blog.

式(シキ)numerical formula.

\section{6 次}

次(つぎ)next (in sequence)

\section{6 先 9 洗}

先(さき)before; ahead; previous; future.

先生(センセイ)teacher; master; doctor.

先日(センジツ)a few days ago; the other day.

洗う(あらう)(vt) to wash

\section{6 気 7 汽}

気(キ)spirit; mind; air; atmosphere

元気(ゲンキ)health(y); vigor; vitality; spirit.

天気(テンキ)weather.

気持ち(きもち)feeling.

汽(キ)steam.

汽車(キシャ)steam train.

\section{7 攻}

攻(コウ、せめ)aggression.
攻める(せめる)(v1,vt) to attack; to assault; to assail.
攻防(コウボウ)attack and defense.

\section{7 君}

君(きみ)you.
…君(…クン)(suffix) Mr. (junior).
君主(クンシュ)ruler; monarch; sovereign.

\section{7 防}

防(ボウ)defense; resistance.
防ぐ(ふせぐ)to resist; to defend against.

\section{7 走 8 歩}

走(ソウ).
走る(はしる)(v5r,vi) to run

歩(ホ、フ、ブ)walk.
歩く(あるく)to walk.

\section{7 助 9 査}

助(すけ)assistance.
助(ジョ)(pref) help; rescue; assistant.
助ける(たすける)(v1,vt) to help.

査(サ)investigate.
巡査(ジュンサ)policeperson.
主査(シュサ)chief examiner; chief investigator.
査問(サモン)enquiry; hearing.

\section{4 火 6 灰 7 災 8 炎}

灰(カイ、はい)ashes.

災い(わざわい)(n) calamity; catastrophe.
火災(カサイ)fire (disaster).

炎(ほのお)flame, blaze.
炎天(エンテン)scorching sun.

\section{4 心 7 志快}

志(シ、こころざし)intention

快(カイ)cheerful.
快い(こころよい)cheerful.

\section{4 不 7 否}

否(イナ、いや)negate.

\section{7 医 8 知}

医(イ)medicine; healing; curing; doctor (medical)

知(チ)know.
知る(しる)to know.
日本人の知らない日本語the Japanese language that the Japanese people don't know

\section{7 身}

身(シン)somebody; person.
自身(ジシン)self.
私自身(わたしジシン)I myself; me myself.
出身(シュッシン)person's origin (town, city, country, etc.).
出身地(シュッシンチ)birthplace.
身長(シンチョウ)height (of body).

\section{(4 木) 7 村枚}

村(むら)village

枚(マイ)(counter) sheet; thin flat object.
一枚(イチマイ)one sheet.

\section{8 的 9 約}

的(テキ)(suffix) -like; typical.

男性的(ダンセイテキ)manly.

的(テキ、まと)mark; target.

目的(モクテキ)purpose; goal; aim; objective; intention.

約(ヤク)promise.

約束(ヤクソク)arrangement; promise; pact; engagement.

\section{(3 子) 8 学乳}

学(ガク)learning, scholarship, erudition, knowledge.
中学(チュウガク)middle school; junior high school.
大学(ダイガク)university.
学界(ガッカイ)academic world.
科学(カガク)science.

乳(ちち)breasts.
授乳(ジュニュウ)breast-feeding.

\section{8 和}

和(ワ)peace; Japan.
平和(ヘイワ)peace.
和む(なごむ)(vi) to be softened; to calm down.
和らげる(やわらげる)(v1,vt) to soften; to moderate; to relieve.

\section{8 昂}

昂る(たかぶる)(vi) to get excited; to get worked up

\section{(5 生) 8 性青毒 11 清情 12 晴}

性(セイ)nature; sex; gender.
男性(だんせい)male.
女性(じょせい)female.

青(あお)(n) blue; green.
青い(あおい)(adj-i) blue; green.
青ざめる(あおざめる)(v1,vi) to become pale.

毒(ドク)poison.
毒ガス(ドクガス)poison gas.

清(セイ)clear.

清い(きよい)clear; pure; noble.

情(ジョウ)feelings; emotion; passion; sympathy.

晴(セイ、はれ)clear.

\section{8 画}

画 means picture.
画(カク)counter for kanji strokes.
字画(ジカク)number of strokes in a character.
画家(ガカ)painter.
企画(キカク)plan.

\section{9 音 13 暗}

音(オン、おと、ね)sound.

本音(ホンね)real intention; motive.

暗(アン)dark.

暗い(くらい)dark.

\section{9 専臭変郎}

専門家(センモンカ)expert; specialist

負(フ)negative; minus.
負う(おう)to bear; to carry on one's back.
負かす(まかす)(vt) to defeat.
負ける(まける)(v1,vi)
to lose; to be defeated.
to succumb; to give in; to surrender; to yield.
to be inferior to.

臭: In China 10 strokes, in Japan 9 strokes.
The lower character is 犬 in China and 大 in Japan.
臭い(くさい)(adj-i) stinking; malodorous; ill-smelling.

変(ヘン)strange.
変わる(かわる)to change; to transform.
変身(ヘンシン)metamorphosis; transformation.
大変(タイヘン)(adv,adj-na,n)very.

郎(ロウ)son.
野郎(ヤロウ)rascal.

\section{9 紅}

紅(くれない)deep red; crimson

\section{9 叙}

叙(ジョ)confer; relate; narrate; describe.
自叙伝(ジジョデン)autobiography.

\section{9 界}

界(カイ)world.
世界(セカイ)world.
学界(ガッカイ)academic world.
業界(ギョウカイ)industry world, business world.

\section{9 胃政}

胃(イ)stomach.

政(セイ、まつりごと)rule; government.

\section{(7 貝) 7 売 9 負 10 員 11 側 12 買 15 賞賣}

売 is simplified from 15 賣.

負(フ)negative; minus.
負う(おう)to bear; to carry on one's back.
負かす(まかす)(vt) to defeat.
負ける(まける)(v1,vi)
to lose; to be defeated.
to succumb; to give in; to surrender; to yield.
to be inferior to.

員(イン)(suffix) member.
工員(コウイン)factory worker.
会社員(カイシャイン)company employee.

側(がわ、かわ)side

側(そば)vicinity; near; beside

買(バイ)buy.
買う(かう)to buy; to purchase

賞(ショウ)prize; award
受賞(ジュショウ)winning (a prize).

賣(バイ).
賣る(うる)to sell.

\section{10 原 13 源}

原(ゲン、はら)meadow; field; plain; prairie; tundra; moor; wilderness.

原(ゲン)original; source; raw; origin.
原文(ゲンブン)original text.
原油(ゲンユ)crude oil.
原料(ゲンリョウ)raw materials.

起源(キゲン)origin; beginning; rise.

\section{10 配息}

配(ハイ)distribute.
配本(ハイホン)distribution of books.
配送センター(ハイソウセンター)distribution center.
配る(くばる)to distribute; to deliver.

息子(むすこ)son

\section{10 能笑料}

能(のう)talent; gift; function.

笑う(わらう)to laugh.
笑む(えむ)to smile.

料(リョウ)(suffix) material; charge; rate; fee

\section{11 部}

部(ブ)section; department; part.
市部(シブ)urban areas.
学部(ガクブ)a department in a faculty in a university.
東部(トウブ)eastern part.
部門(ブモン)division (of a larger group).
一部.

\section{(6 肉) 11 脱}

脱出(ダッシュツ)escape.
The left component is 肉.

\section{11 動}

動(ドウ)motion.
動く(うごく)(vi) to move.
動画(ドウガ)animation, motion picture.
自動車(ジドウシャ)automobile.
動力(ドウリョク)power; motive power.

\section{11 得}

得る(える)(v1,vt) to get; to acquire; to obtain; to earn; to win; to gain; to secure; to attain.
納得(ナットク)understanding, agreement.
納得するto consent, agree; to understand the reason for something.

\section{(8 雨) 11 雪 12 雲 13 雷電}

雪(セツ、ゆき)snow

雲(ウン、くも)cloud

雷(かみなり)thunder.

電(デン)lightning.
電光(デンコウ)lightning.
電気(デンキ)electricity (lit. lightning spirit).
電話(デンワ)telephone (lit. lightning talk).
電車(デンシャ)electric train (lit. lightning carriage).
電気自動車(デンキジドウシャ)electric car.

\section{11 問 12 間開 14 聞関}

問(モン)(suffix, counter) counter for questions.

問う(とう)to ask (a question).

質問(シツモン)question; inquiry; enquiry.

間(カン、ケン)interval; space.

間(あいだ)gap; interval; distance; span; stretch (space or time).

間(ま)space; room; time; pause.

人間(ニンゲン)human being.

世間(セケン)world; society.

開く(ひらく)to open (door, business, eye, mouth, ...)

聞(ブン、モン)hear.

聞く(きく)to hear.

関(カン、せき)barrier; gate

\section{12 番}

番(バン)number.
番組(バンぐみ)television program.

\section{14 歌}

歌(カ、うた)song.
歌声(うたごえ)singing voice.
歌う(うたう)to sing.

\section{Cardinal directions: 5 北 6 西 8 東 9 南}

北(ホク、きた)north

西(セイ、サイ、にし)west

東(トン、ひがし、あずま)east

南(ナン、みなみ)south

They combine as in English.

北西(ホクセイ)northwest

北東(ホクトウ)northeast

東北(トウホク)Touhoku (a prefecture)

南西(ナンセイ)southwest

南東(ナントウ)southeast

\section{Thing: 6 件 8 物事}

件(ケン)matter; case; item

物(ブツ、モツ、もの)thing; object; matter.

書物(ショモツ)books.

食べ物(たべもの)food.

物語(ものがたり)tale; story; legend.

物語る(ものがたる)(vt) to tell; to indicate.

火事(カジ)fire (as a disaster).

ラメン店で火事fire at a ramen shop.

有事(ユウジ)emergency.

無事(ブジ)safety; peace; quietness.

\section{8 金 13 鉄 14 銅銀 16 鋼}

金has a lot to do with metals.
金(キン、かね)gold; money.

金属(キンゾク)metal.
重金属(ジュウキンゾク)heavy metal.
These are also the chemistry terms.

金色(キンいろ、コンジキ)golden (color)

鉄(テツ、くろがね)iron (lit. 黒金 black metal).
鉄人(てつジン)iron man; strong man.

鉄道(テツドウ)railroad; railway

銅(ドウ、あかがね)copper (lit. 赤金 red metal)

銀(ギン、しろがね)silver (lit. 白金 white metal)

鋼(コウ、はがね)steel

青銅(セイドウ)bronze

鋼鉄(コウテツ)steel

\section{Heart: 4 心 5 必}

心 is involved in a lot of feeling-related characters.

心配(シンパイ)(adj-na,n,vs) worry, concern, anxiety.
心配(シンパイ)(n,vs) care, help.

必 is unrelated to 心. They only look similar.
必(ヒツ、かなら)inevitable.
必ず(かならず)(adv) always, invariably, certainly.
必要(ヒツヨウ)(adj-na,n) necessity, need.

\subsection{Thoughts: 7 忘 9 思}

忘(ボウ、わす)forget.
忘れる(わすれる)(v1) to forget.
忘年会(ボウネンカイ)year-end party
(lit. forget-year meeting, a meeting to forget the year).

思(シ)think.
思う(おもう)to think

\subsection{Feelings: 12 悲 13 意感}

悲しい(かなしい)sad.
悲恋(ヒレン)disappointed love

意(イ)feelings; thoughts.
意欲(イヨク)motivation; will.
意味合い(イミあい)implication; nuance
小生意気(こなまイキ)cheekiness; impudence.

感じる(カンじる)(v1) to feel.

\subsection{Love: 10 恋}

恋(レン、こい)romance; love; tender passion.
恋人(こいびと)lover; sweetheart.
恋文(こいぶみ)love letter.

\subsection{Other: 9 急 11 悪 14 態}

急(キュウ)urgent, sudden, abrupt.
急ぐ(いそぐ)to hurry.

悪(アク)evil, wickedness.
悪人(アクニン)bad person, villain.
悪い(わるい)bad, poor; evil; unprofitable; at fault.

態(ざま)mess; sorry state; plight; sad sight.
変態(ヘンタイ)sexual perversion.

\section{Hand}

\subsection{Figurative: 8 押}

押す(おす)to push; to press; to cram into; to force.
to stamp.
to overwhelm.
押し(おし)(n) push.

\subsection{Giving and taking: 8 取受 11 授}

取る(とる)(vt) take; fetch; take up.
買い取り(かいとり)purchase; sale. purchase on a non-return policy.

受ける(うける)(v1,vt) to receive

授ける(さずける)(v1,vt) to grant; to award.
授受(ジュジュ)give-and-receive.

\section{Blades}

\subsection{Tangible cutting: 4 切}

切 depicts spoon and sword.
切(セツ、サイ、き)cut.
切る(きる)(v5r) cut.
大切(タイセツ)(adj-na,n) important.
一切(イッサイ)absolutely; (when used with negative) at all.

\subsection{Intangible cutting: 4 分}

分 depicts something separated by a blade.
分(フン、ブン)minute.
1分(イップン)one minute.
12時34分(ジュウニジサンジュウヨンプン)12:34 (time).

分(わ)understand.
分かる(わかる)to be understood.

分ける(わける)(v1,vt) to divide; to split; to share; to distribute.

\subsection{Swordtip: 4 方}

方 depicts the tip of a sword.
方(ホウ、かた)direction.
方(かた)(honorific) person.
あの方(あのかた)that person.

\subsection{Dissection: 10 剖}

剖(ボウ)dissection.

\subsection{Law: 6 刑 9 則 14 罰}

刑(ケイ)(n,n-suf) penalty; sentence; punishment

則(のり)law; rule; regulation.
法則(ホウソク)law; rule.

罰(バツ)punishment; penalty.
罰する(ばっする)to punish; to penalize.
罰金(バッキン)fine; monetary penalty.

\subsection{Separation: 7 別}

別 depicts sword cutting bone.
別な(ベツな)(adj-na) different; separate; another
日付別(ひづけベツ)separate by date.

\subsection{Reduction: 12 減}

減(ゲン)reduction; 10\%減 ten percent reduction.

\subsection{Burglar: 13 賊}

賊(ゾク)burglar; robber.
海賊(カイゾク)pirate; sea robber.

\subsection{War: 13 戦}

戦(いくさ)war.
内戦(ナイセン)civil war.
世界大戦(セカイタイセン)World War.

\section{(4 戈) 6 伐成 7 戒}

伐(バツ)fell; strike; attack; punish.

成(セイ)become.

作成(サクセイ)(n,vs) writing; creation.

コンピュータプログラムを作成するto write a computer program.

戒(カイ)commandment.

十戒(ジッカイ)
(Buddhist) 10 precepts.
(Christian) 10 commandments.

戒める(いましめる)(vt) to admonish.

\section{10 造}

造る(つくる).
木造(モクゾウ)wooden; made of wood.
