\chapter{Kanji 2}

\section{2 人 6 因仲 7 作戻}

因(イン)cause; factor

仲間(なかま)friend (how does this differ from 友達(ともだち)?)

7 作(サク) work; harvest

作る(つくる)to make

7 戻る(もどる)(vi) to turn back; to return; to go back;
to come back to a previously visited place

\section{2 又 4 文反 8 取受}

又 depicts the right hand.
又(また)again; also.

文字(モジ)letter (of alphabet); character (a Han character)
文書(ブンショ)sentence.
文化(ブンカ)culture.

反(ハン)anti-.
反する(ハンする)to oppose; to rebel; to revolt.
反体制(ハンタイセイ)anti-establishment.

取る(とる)(vt) take; fetch; take up.
買い取り(かいとり)purchase; sale. purchase on a non-return policy.


受ける(うける)(v1,vt) to receive

\section{2 了 3 子 5 字 8 学}

男子(ダンシ)young man, boy

女子(ジョシ)young woman, girl

太子(タイシ)crown prince

赤字(あかジ)deficit; red letter; red Han-character

漢字(カンジ)Han character

科学(カガク)science

\section{2 厶 4 公 5 広 7 私}

厶 means I, myself, private.

公(おおやけ)public; communal; official; governmental.
公安(コウアン)public safety; public welfare.

広い(ひろい)(adj-i) spacious; vast; wide.
広告(コウコク)advertisement.

私(わたし)I; me

\section{2 刀刂 4 分方切 6 刑 7 別 8 法 9 則}

刂 depicts a knife.

分 depicts something separated by a knife.
分ける(わける)(v1,vt) to divide; to split; to share; to distribute.
1分(イップン)one minute.
12時34分(ジュウニジサンジュウヨンプン)12:34 (time).

方(かた)(tip of a sword)

切 (spoon and sword).
切る(きる)(v5r) cut.
切(サイ、セツ).
一切(イッサイ)absolutely; (when used with negative) at all.
大切(タイセツ)(adj-na,n) important.

刑(ケイ)(n,n-suf) penalty; sentence; punishment

別 depicts sword cutting bone.
別な(ベツな)(adj-na) different; separate; another

法(ホウ)law; rule; method; principle

則(のり)law; rule; regulation

法則(ホウソク)law; rule

\section{2 力 4 介 7 助 9 界重 11 動}

介 can mean mediation.
紹介(ショウカイ)introduction; referral.
仲介(チュウカイ)agency; intermediation.
介入(カイニュウ)intervention.
介助(カイジョ)help; assistance; aid.

介 can mean shellfish.
魚介(ギョカイ)marine products; seafood.

助ける(たすける)(v1,vt) to help

界(カイ)world.
世界(セカイ)world.
学界(ガッカイ)academic world.
業界(ギョウカイ)industry world, business world.

動(ドウ)motion.
動く(うごく)(vi) to move.
動画(ドウガ)animation, motion picture.
自動車(ジドウシャ)automobile.
動力(ドウリョク)power; motive power.

\section{3 口 5 古兄 7 告 8 味}

5 古 old

古consists of 十(ten) and 口(mouth, generation).

5 古い(ふるい)old

古株(ふるかぶ)veteran; senior; old-timer

5 兄(あに)older brother

告げる(つげる)to inform; to tell

告白(コクハク)confess (usually of love)

8 味(あじ)flavor; taste

美味しい(おいしい)delicious

\section{3 土 6 地}

地(チ)earth

団地(ダンチ)multi-unit apartments

\section{3 宀 6 安宅 9 室}

宀 depicts a roof.

安い(やすい)cheap; inexpensive

安全(アンゼン)safety; security

安心(アンシン)relief; peace of mind

住宅(ジュウタク)residence; housing; residential building

自宅(ジタク)one's home

自宅火災(ジタクカサイ)house fire; home fire (disaster)

室(むろ)room.

\section{3 幺 4 幻 9 紅 10 紙 11 細}

4 幻(まぼろし)phantom; vision; illusion; dream

紅(くれない)deep red; crimson

紙(かみ)paper

手紙(てがみ)letter (the document, not the alphabet)

細い(ほそい)thin; slender
The left-side part of 細is the left-side form of 糸.

\section{3 女 7 姉 8 妹}

姉(あね)older sister

妹(いもうと)younger sister.
妹consists of 女(lady) and 未(not yet).

\section{3 寸 5 付 7 村 10 時}

% https://en.wiktionary.org/wiki/%E5%AF%B8#Japanese
寸depicts a position on the forearm
where the pulse can be palpated by compressing the radial artery.
寸(スン)an ancient unit of length, approximately 3 cm.

付ける(つける)(v1,vt) to attach, join, stick, glue, fasten.
口付け(くちづけ)、口づけ(くちづけ)(v1) to kiss.

村(むら)village

時(とき)time.
時代(ジダイ)era.
三国時代(サンゴクジダイ)The Three Kingdoms period.
戦国時代(センゴクジダイ)The Warring States period.

\section{4 日 9 星 12 朝}

日付(ひづけ)date.日付別(ひづけベツ)separate by date.(context? usage?)

星(ほし)star

朝(あさ)morning.
今朝(けさ)this morning.
早朝(ソウチョウ)early morning.

\section{4 王 5 主 9 美皇}

5 主(おも)chief; main; principal; important

主人公(シュジンコウ)hero; main character

9 美 beauty

美味しい(おいしい)delicious (idiosyncratic reading)

美しい(うつくしい)beautiful

9 皇(コウ)imperial

皇居(コウキョ)imperial palace

\section{4 木 1: 4 木 8 林 12 森}

木(き)tree

林(はやし)

森(もり)

\section{4 木 2: 5 本未 7 来 8 東}

5 本(ホン)book, counter for long cylindrical objects

一本(イッポン)one long cylindrical object, one point

日本(ニホン)Japan

5 未(ミ) not yet

未来(ミライ)future (lit. not yet come)

7 来(ライ)come; next

来depicts fruits hanging on a tree.
It means that the time to harvest has come.

来る(くる)to come. This is an irregular verb. The past form is 来た(きた).

来月(ライゲツ)next month.
来年(ライネン)next year.
「来年田中さんが日本に行きます。」Next year, Mr. Tanaka will go to Japan.
(``Next year'' means one year after the moment the speaker says it.)

8 東(トン、ひがし)east

\section{4 木 3: 5 禾 6 米休 7 困来体 9 秋茶}

禾 depicts a plant stalk.

秋 depicts the burning of plant stalks (after harvest).
秋(あき)autumn; fall season.

休む(やすむ)to rest

困る(こまる)(vi) to be troubled; to be embarrassed

米(ベイ、こめ)rice.
米国(ベイコク)United States of America.

体(タイ、からだ)body.
肉体(ニクタイ)body; flesh.

茶(チャ) tea

\section{4 火 7 災 8 炎}

As the bottom part of another character,
the fire character is written as four dot strokes.

火(カ、ひ)fire, flame.
火事(カジ)fire (disaster).
大火(タイカ)big fire.

災い(わざわい)(n) calamity; catastrophe.
火災(カサイ)fire (disaster).

炎(ほのお)flame, blaze.
炎天(エンテン)scorching sun.

\section{4 手 8 押 9 持}

手洗い(てあらい)restroom; lavatory; toilet; a place for washing hands

持つ(もつ)to hold; to carry; to possess

押し(おし)(n) push

\section{4 牛 6 件}

件(ケン)matter; case; item

\section{4 止 7 走 8 歩}

止める(やめる)(v1) to stop

止める(とめる)(v1) to stop

歩く(あるく)to walk

走る(はしる)(v5r,vi) to run

\section{4 心 5 必 7 応}

必 is unrelated to 心. They only look similar.

必ず(かならず)(adv) always, invariably, certainly.
必要(ヒツヨウ)(adj-na,n) necessity, need.

応え(こたえ)response; reply; answer; solution

応える(こたえる)(v1) to respond; to reply; to answer

\section{4 斤 7 近 13 新}

斤(キン) (axe)

近い(ちかい)(adj-i) near

新(シン) new

(cut tree down with an axe)

新しい(あたらしい)new

新聞(シンブン)newspaper (lit. new hear)

新車(シンシャ)new car

\section{5 田 7 男 9 界}

7 男(おとこ)man

男is田(rice field) and 力(strength).

\section{5 目 6 自 9 冒 11 現}

自(ジ) self

自ら(みずから)(adv) personally

自在(ジザイ)freely (at will)

自分(ジブン)self (context? example usage?)

冒 depicts a hat obstructing the sight, implying rashness
(acting without enough thought).
Don't confuse the hat 冃 with 日(sun) and 月(moon).
冒す(おかす)(vt) to risk.
冒険(ボウケン)adventure.

現す(あらわす)(vt) to reveal; to show; to display.
現れる(あらわれる)(v1,vi) to appear; to become visible; to materialize.

\section{5 且 7 助 8 狙}

助(すけ)assistance

助(ジョ)(pref) help; rescue; assistant

狙う(ねらう)(vt) to aim at

\section{5 母 7 毎 8 海}

7 毎(マイ)every

毎日(マイニチ)everyday

毎月(マイゲツ、マイつき)every month

毎時(マイジ)every hour

毎回(マイカイ)every time

毎年(マイネン、マイとし)every year

8 海(うみ)sea; beach

\section{5 生 8 性}

性(セイ)nature; sex; gender

男性(だんせい)male

女性(じょせい)female

\section{5 矢 7 医 8 知}

矢(や)arrow

医(イ)medicine; healing; curing; doctor (medical)

日本人の知らない日本語the Japanese language that the Japanese people don't know

\section{6 曲}

曲(キョク)music

作曲(サッキョク)musical composition

\section{6 再}

再(サイ)again, re-

再生(サイセイ)playback; rebirth

再開(サイカイ)reopening

再来(サイライ)return, comeback

\section{6 全}

全 depicts a whole piece of jade.

全(ゼン) whole

全部(ゼンブ)altogether; everything

全く(まったく)(adv) completely, entirely, wholly, totally

\section{6 艸 7 花 9 草}

7 花(はな)flower

9 草(くさ)grass

\section{6 血}

止血(シケツ)stop bleeding; hemostasis

\section{6 耳 14 聞}

聞く(きく)to hear

\section{6 西}

西(にし)west

\section{6 共 8 供}

共(とも)companion; follower; attendant; retinue

子供(こども)child

\section{6 肉 11 脱}

脱出(ダッシュツ)escape

\section{6 羽 10 弱 11 習}

6 羽(はね)feather

10 弱

弱い(よわい)weak

11 習 (the bottom character is自not日).

習う(ならう)(vt) to learn

練習(レンシュウ)training; practice

\section{6 合会}

What is the difference between 合 and 会?

合う(あう)to meet

会う(あう)to meet (?)

会社(カイシャ)company; corporation

会社員(カイシャイン)company employee

年会(ネンカイ)yearly meeting; annual convention

社会(シャカイ)society

会見(カイケン)interview

\section{6 名}

名前(なまえ)
name; full name.
given name; first name.

\section{6 先 8 洗}

先生(せんせい)teacher; master; doctor

洗う(あらう)(vt) to wash

\section{6 死}

死(シ)death.
死ぬ(しぬ)to die.

死去(シキョ)death

死亡(シボウ)death; mortality

\section{6 外}

6 外(ガイ)foreign

外人(ガイジン)foreigner, foreign person

外国(ガイコク)foreign country

外界(ガイカイ)outside world

海外(カイガイ)foreign; abroad; overseas

\section{6 気}

気(キ)spirit; mind; air; atmosphere

元気(ゲンキ)

天気(テンキ)weather

気持ち(きもち)

\section{6 在存耂有 8 者}

有 depicts a hand holding a piece of 肉(meat).
有る(ある)to exist.

存在(ソンザイ)

耂 depicts a bent-over figure with long hair, an old man.

者(シャ)(n,suf) someone of that nature; someone doing that work.
者(もの)(n) person (rarely used without a qualifier).

\section{7 呆 9 保}

呆depicts a child.

呆れる(あきれる)(v1,vi) to be amazed, astonished, astounded.

保する(ほする)to guarantee

\section{7 辛}

辛depicts a tool used to mark slaves and criminals.

辛い(からい)(adj-i) spicy, salty, harsh, hot, acrid

辛い(つらい)(adj-i) painful; heartbreaking; difficult.
Suffix づらい(adj-i) means ``difficult to do''.
読みづらい(adj-i) difficult to read.
書きづらい(adj-i) difficult to write.
読みづらい漢字difficult-to-read Han character.

\section{7 弟}

弟(おとうと)(hum) younger brother.
弟さん(おとうとさん)(hon) younger brother.

兄弟(キョウダイ)siblings;
brothers and sisters
(although the characters mean older brother and younger brother).

\section{7 言 9 信}

言contains口(mouth).

言(こと) say

言う(いう)to say

信(シン)faith; trust

信じる(シンじる)(v1,vt) to believe; to have faith in

\section{8 門 12 間 12 開 14 関}

門(かど) gate

間 interval

開く(ひらく)to open (door, business, eye, mouth, ...)

関(カン、せき)barrier; gate

\section{8 雨 11 雪 12 雲 13 雷電}

雪(ゆき)snow

雲(くも)cloud

雷(かみなり)thunder

電(デン) lightning

電光(デンコウ)lightning

電気(デンキ)electricity (lit. lightning spirit)

電話(デンワ)telephone (lit. lightning talk)

電車(デンシャ)electric train (lit. lightning carriage)

電撃(デンゲキ)electric shock

電気自動車(デンキジドウシャ)electric car

\section{11 魚}

魚(さかな)fish

魚介(ギョカイ)seafood; marine products

\section{8 使(シ)use}

使う(つかう)to use

使用(シヨウ)(n) use

\section{8 長}

長(チョウ)
long (distance or time).
leader.
eldest.

長い(ながい)long (distance); long (time)

長女(チョウジョ)eldest daughter; first-born daughter

市長(シチョウ)mayor (a government official)

身長(シンチョウ)height (of body)

最長(サイチョウ)longest, tallest

社長(シャチョウ)company president

\section{8 兒 7 児}

兒 is a simplification of 兒 depicting an infant
with an imperfect cranium (fontanelles).

乳児(ニュウジ)infant; suckling baby

男児(ダンジ)boy; son

\section{10 症}

症(ショウ)(n,suf) illness

\section{7 形(かたち) shape; form}

\section{8 隹}

隹 depicts a short-tailed bird.

誰(だれ)who

\section{9 食(ショク) food, eat}

食べ物(たべもの)food

食物(ショクもの)food

食べる(たべる)(v1) to eat

飲む(のむ)to drink (any liquid, not just liquor)

\section{9 省}

4 少 + 5 目

省く(はぶく)(vt)
to omit; to leave out; to exclude.
to curtail; to save; to cut down; to economize.

省(ショウ)(suf) ministry; department

国交省(コッコウショウ)(abbr)
Ministry of Land, Infrastructure, Transport, and Tourism
