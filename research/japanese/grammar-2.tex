\chapter{Grammar 2}

All truth values are to be interpreted probabilistically.
The statement ``everybody has a chicken'' is neither true nor false;
it has truth value somewhere between 0 and 1.

Probabilistic temporal modal logic?

I-phrase

I-clause

U-phrase

U-clause

A nominal is a thing that acts like a noun.

An adjectival is a thing that modifies a nominal.

\section{Predication and is-a}

車は高い。
Cars are expensive.
\[
    車(x) \vdash 高い(x)
\]

車は高いか?
Are cars expensive?
\[
    ? : 車(x) \vdash 高い(x)
\]

黒い車は良いです。
Black cars are good.
\[
    車(x), 黒い(x) \vdash 良い(x)
\]

犬は動物。
Dog is a mammal.
\[
    犬(x) \vdash 動物(x)
\]

中村さんは俳優です。Nakamura-san is an actor.
\[
    \vdash 俳優(中村さん)
\]
or
\[
    中村さん(x) \vdash 俳優(x)
\]

\section{Equation}

あの人は田中さんです。

\[
    あの人 = 田中さん
\]

\section{Relative clauses}

妹が居ない人
people who do not have younger sisters
\[
    人(x), \neg \exists y (y \text{ is an 妹 of } x)
\]

お金がない人
people who do not have money
\[
    人(x), \neg \text{possess}(x,お金)
\]

お金がない人がいる。
There are people who do not have money.
\[
    \exists x (人(x) \wedge \neg \text{possess}(x,お金))
\]

お金がない人が悲しい。
People who do not have money are sad (feel sad).
(This is just an example sentence.
It has nothing to do with the real world.)
\[
    人(x), \neg \text{possess}(x,お金) \vdash 悲しい(x)
\]

\section{Example constructions: が、は}

お金(おかね):お金がありません。The implied entity does not have money.

山(やま)、高い(たかい):
この山は高いです。This mountain is tall.

川(かわ)、広い(ひろい):
この川は広いです。This river is wide.

石(いし)、子供(こども)、投げる(なげる):
石を投げる子供stone-throwing child

名前(なまえ):お名前は?
And your name is...?

\section{Example constructions: て下さい}

戸(と)、開く(ひらく)、下さい(ください):
戸を開いてく下さい。Please open the door.

\section{Nouns made by conjoining nouns: と}

Unlike English ``and'', Japanese と conjoins \emph{nouns} only.
English ``and'' can conjoin noun phrases or sentences.
田中さんと中川さんは東京に行く。Tanaka-san and Nakagawa-san goes to Toukyou.
\[
    \vdash \text{destination}(東京),行く(田中さん),行く(中川さん)
\]

彼(かれ)、私(わたし)、同じ(おなじ)、見る(みる):
彼と私は同じ夢を見ました。
He and I had the same dream.
\[
    夢(彼,x), 夢(私,y) \vdash 同じ(x,y)
\]

\section{Example constructions: Genitive: の}

羽(はね)、黒(くろ):カラスの羽は黒い。The color of a crow's feathers is black.

Crowのfeatherのcolorはblack。

\[
    羽(x), カラス(y), \text{belongs-to}(x,y) \vdash 黒い(x)
\]

\section{Example constructions: Nominalization: の}

\subsection{In doubt}

犬(いぬ)、来る(くる)、来た(きた):

田中さんは来た。Tanaka-san came.
\[
    \vdash 来た(田中さん)
\]
About Tanaka-san, he came.
What Tanaka-san did was coming.

来たのは田中さんです。
About having come,
it is Tanaka-san who did that (and not somebody else).
\[
    来た(x) \vdash x = 田中さん
\]
If anyone came, then it was Tanaka-san.
Who came is Tanaka-san.
Tanaka-san is the person who came (and not somebody else).

勉強(ベンキョウ)、貴方(あなた):
勉強するのは貴方にいい。
Studyingはyouにgood。
Studying is good for you.
Parse tree:
貴方にいいmodifiesの,
貴方にいいのis the topic,
貴方にいいのはmodifiesいい.
Logic: good-for(studying,you).

貴方にいいのは勉強するのです。
What is good for youはstudyingです。
What is good for you is studying.
Parse tree:((貴方)に・いい・の)は・勉強・する・の・です。
Another possible parse tree:
(貴方)に・(いい・の)は・勉強・する・の・です。
For you, what is good is studying.

勉強するのは何のため?
What is studying for?
What is the purpose of studying?

\section{Example constructions: たい}

魚(さかな)、食べる(たべる):魚を食べたくありません。The implied entity does not want to eat fish.

\section{Questionable}

良い(いい)、良くなかった(よくなかった):
良くなかった良い事things that was good that was not good.

Does AのBとC parse as Aの(BとC) or (AのB)とC?
