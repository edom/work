\chapter{3 口 mouth}

\section{3 口(くち) mouth}

\section{5 古 old}

古consists of 十(ten) and 口(mouth, generation).

5 古い(ふるい)old

\section{7 言(こと) say}

This character is quite productive.

言う(いう)to say

言葉(ことば)word; dialect

\subsection{10 記 record}

記す(しるす)to record, to write down

記事(キジ)article (writing).
選り抜き記事(よりぬきキジ)selected articles.
新しい記事(あたらしいキジ)new articles.

記録(キロク)record

\subsection{13 話 talk}

話す(はなす)to talk

\subsection{14 読 read}

読む(よむ)to read

\subsection{14 語(ゴ) language}

…語(…ゴ)... language

日本語(ニホンゴ)Japanese language

英語(エイゴ)English language

\subsection{15 誰(だれ)who}

\section{8 味(あじ)flavor; taste}
