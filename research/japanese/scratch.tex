\chapter{Scratch space}

感じる(カンじる)to feel

9 前(まえ)before (time), in front of

5 石(いし)stone

左(ひだり)left

右(みぎ)right

頭(あたま)head

名詞(メイシ)noun

文書(ブンショ)sentence

戻る(もどる)to go back

歌う(うたう)to sing

笑う(わらう)to laugh; to smile

若い(わかい)

仕事(シごと)(n) work; job; business; occupation; employment

TODO

the right-side radical of 使う(つかう)to use

喧嘩(ケンカ)fight; brawl

博打(バクチ)gambling

北東(ホクトウ)northeast

星(ほし)star

空(そら)sky

空港(クウコウ)airport

違う(ちがう)

間違う(まちがう)

遠い(とおい)far

近い(ちかい)near

百人力(ヒャクジンリキ)tremendous strength (lit. one-hundred-people strength)

辞書(ジショ)dictionary

交わる(まじわる)cross; intersect; join; meet (?)

危ない(あぶない)dangerous

社会(シャカイ)society

政治(セイジ)politics

経済(ケイザイ)economy

国際(コクサイ)international

文化(ブンカ)culture

科学(カガク)science

写真(シャシン)multimedia; photograph; movie (?)

日付(ひづけ)date.日付別(ひづけベツ)separate by date.(context? usage?)

毒ガス(ドクガス)poison gas

内戦(ナイセン)civil war

仲間(なかま)friend (how does this differ from 友達(ともだち)?)

年会(ネンカイ)yearly meeting; annual convention

連携(レンケイ)collaboration; cooperation

意欲(イヨク)motivation; will

貿易(ボウエキ)trade (foreign)

赤字(あかジ)deficit; red letter; red Han-character

対(タイ)versus... (usage?)

お絵描き(おエかき)oekaki; painting; drawing

ネタバレspoiler (of a movie, a story, etc.); something that spoils the end of a movie, a story, etc.

沼(ぬま)swamp; bog

堕ちる(おちる)(v1,vi)to fall down; to drop (?)

脳内(ノウナイ)intracranial; inside the brain

授乳(ジュニュウ)breast-feeding

出来上がる(できあがる)(vi) to be finished; to be completed; to be ready

質問(シツモン)question; inquiry; enquiry

問題(モンダイ)problem

吸う(すう)to suck with mouth

仕方(シかた)way.仕方ないit can't be helped; there's no other way.

平和(ヘイワ)peace.平和だpeace (used in a post that may easily anger the reader) (?)

垢(あか)dirt; filth

一切(イッサイ)absolutely; (when used with negative) at all

関係(カンケイ)relation; connection

肉体関係(ニクタイカンケイ)sexual relations

垢と一切関係ないIt has absolutely nothing to do with dirt.

趣味(シュミ)hobby; taste, preference.

邪魔(ジャマ)intrusion
邪魔するto intrude

接吻(セップン)kiss

待つ(まつ)(vt,vi) to wait; to wait for; to await.

春(はる)spring (season)

夏(なつ)summer

冬(ふゆ)winter

時代(ジダイ)era.
三国時代(サンゴクジダイ)The Three Kingdoms period.
戦国時代(センゴクジダイ)The Warring States period.

一寸(ちょっと)just a minute; short time; just a little.
「一寸待って下さい。」``Please wait a moment.''
「一寸!」``Hey!''

呆れる(あきれる)(v1,vi)to be amazed, astonished, astounded.

諦める(あきらめる)(v1,vt)to give up; to abandon

一部
方
需要
編
一般
閲覧
お気

今日も
元気
な膝小僧ときれいな
お目眼
多少画質

お巡り(おまわり)
policeman

ぐるぐる

警察(ケイサツ)police

安全(アンゼン)safety; security

専門家(センモンカ)expert; specialist

見解(ケンカイ)opinion; point of view

専門家見解(センモンカケンカイ)expert opinion

納得(ナットク)understanding, agreement.
納得するto consent, agree; to understand the reason for something.

番組(バンぐみ)television program

洗う(あらう)(vt) to wash

会社(カイシャ)company; corporation

会社員(カイシャイン)company employee

魚介(ギョカイ)seafood; marine products

駅(エキ)train station. 駅is新字体of驛.

人間(ニンゲン)humanity?

人類(ジンルイ)humankind?

日本人(にホンジン)Japanese person

主人公(シュジンコウ)hero, main character

禁じる(キンじる)(v1,vt) to prohibit

監禁(カンキン)confinement; bondage

% https://en.wiktionary.org/wiki/%E8%BE%9B%E3%81%84#Japanese
% https://en.wiktionary.org/wiki/%E8%BE%9B#Japanese

7 辛

辛depicts a tool used to mark slaves and criminals.

辛い(からい)(adj-i) spicy, salty, harsh, hot, acrid

辛い(つらい)(adj-i) painful; heartbreaking; difficult.
Suffix づらい(adj-i) means ``difficult to do''.
読みづらい(adj-i) difficult to read.
書きづらい(adj-i) difficult to write.
読みづらい漢字difficult-to-read Han character.

大火(タイカ)big fire (lit. big fire)

大切(タイセツ)(adj-na, n) important (lit. big cut)

大変(タイヘン)(adv, adj-na, n) very (lit. big unusual/strange)

飛ぶ(とぶ)

存在(ソンザイ)

持つ(もつ)

日本人の知らない日本語the Japanese language that the Japanese people don't know
