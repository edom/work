\chapter{Scratch space}

1013	状	狀	犬	7	5		form	ジョウ

402	況		水	8	S		condition	キョウ

状況(ジョウキョウ)state of affairs (around you).

後ろ(うしろ)back; behind; rear.

1709	膝		肉	15	S	2010	knee	ひざ

1566	曇		日	16	S		cloudy weather	ドン、くも-る

883	収	收	攴	4	6		take in	シュウ、おさ-める、おさ-まる

収まる(おさまる)to be in one's (supposed/intended) place.

1054	振		手	10	S		shake	シン、ふ-る、ふ-るう、ふ-れる

1809	返		辵	7	3		return	ヘン、かえ-す、かえ-る

振り返る(ふりかえる)(v5r,vi)to turn head; to look over one's shoulder. to turn around. to look back.

1359	恥		心	10	S		shame	チ、は-じる、はじ、は-じらう、は-ずかしい

43	逸	逸 [4]	辵	11	S		deviate	イツ

逸らす(そらす)(vt) to turn away; to avert.

問い(とい)(n) question.

全く(まったく)(adv) really; truly; entirely; completely; wholly; indeed.

2004	腰		肉	13	S		loins	ヨウ、こし

腰(こし)back; lower back; loins; hip; waist; lumbar region.

28	移		禾	11	5		shift	イ、うつ-る、うつ-す

乱暴(ランボウ)rude; lawless.

抓る(つねる)(v5r,vt) to pinch.

1067	震		雨	15	S		quake	シン、ふる-う、ふる-える

1807	片		片	4	6		one-sided	ヘン、かた

1043	伸		人	7	S		lengthen	シン、の-びる、の-ばす、の-べる

掴まる(つかまる)(v5r,vi)to be caught; to be arrested. to hold on to; to grasp.

巡る(めぐる)(v5r,vi) to go around.

1213	訴		言	12	S		sue	ソ、うった-える

訴える(うったえる)(v1,vt) to raise; to bring to (someone's attention).

1917	妙		女	7	S		exquisite	ミョウ

どんどんdrumming (noise). rapidly; steadily.

1035	触	觸	角	13	S		contact	ショク、ふ-れる、さわ-る

触れる(ふれる)(v1,vi) to touch; to feel.

1400	張		弓	11	5		stretch	チョウ、は-る

637	絞		糸	12	S		strangle	コウ、しぼ-る、し-める、し-まる

絞る(しぼる)to wring; to squeeze; to press; to extract.

730	搾		手	13	S		squeeze	サク、しぼ-る

しぼる:
絞る vs 搾る

1727	描		手	11	S		sketch	ビョウ、えが-く、か-く

565	弧		弓	9	S		arc	コ

1412	嘲 [7]		口	15	S	2010	ridicule	チョウ、あざけ-る

741	擦		手	17	S		grate	サツ、す-る、す-れる

1491	塗		土	13	S		paint	ト、ぬ-る

布団(ふとん)(ateji) futon (Japanese mattress)

404	挟	挾	手	9	S	1981	pinch	キョウ、はさ-む、はさ-まる

立てる(たてる)(v1,vt) to stand up.

2026	落		艸	12	3		fall	ラク、お-ちる、お-とす

落ちる(おちる)

落とす(おとす)

2117	弄		廾	7	S	2010	tamper with	ロウ、もてあそ-ぶ

弄る(いじる)(v5r,vt) to tamper with.

1603	濃		水	16	S		concentrated	ノウ、こ-い

756	残	殘	歹	10	4		remainder	ザン、のこ-る、のこ-す

煽る(あおる)(v5r) to fan; to agitate; to stir up.

1136	醒		酉	16	S	2010	be disillusioned	セイ

醒める(さめる)(v1,vi) to be disillusioned.

317	起		走	10	3		wake up	キ、お-きる、お-こる、お-こす

121	憶		心	16	S		recollection	オク

記憶(キオク)memory; recollection; remembrance. storage.

愛撫(アイブ)caress

1330	奪		大	14	S		rob	ダツ、うば-う

奪う(うばう)(v5u,vt) to snatch away.

すらすらsmoothly

426	極		木	12	4		poles	キョク、ゴク、きわ-める、きわ-まる、きわ-み

極まる(きわまる)(v5r,vi) to terminate; to reach an extreme.

1457	締		糸	15	S		tighten	テイ、し-まる、し-める

反応(ハンノウ)reaction; response.

顔(かお)face (person).

固い(かたい)(adj-i) hard; solid; tough.

1638	薄		艸	16	S		dilute	ハク、うす-い、うす-める、うす-まる、うす-らぐ、うす-れる

薄い(うすい)(adj-i) thin. pale; light. watery; dilute; sparse.

ごりごりscraping; scratching. hard (to the bite, to the touch).

抉る(えぐる)(v5r,vt) to gouge; to hollow out.

29	萎		艸	11	S	2010	wither	イ、な-える

萎える(なえる)(v1,vi) to wither. to droop. to be lame.

1421	沈		水	7	S		sink	チン、しず-む、しず-める

沈む(しずむ)(v5m,vi) to sink; to feel depressed.

408	胸		肉	10	6		bosom	キョウ、むね、(むな)

1916	脈		肉	10	4		vein	ミャク

2025	絡		糸	12	S		entwine	ラク、から-む、から-まる、から-める

深々(シンシン)(adj-t,adv-to)silent (especially of the passing of the night). piercing.

深々(フカブカ)(adv-to) very deeply.

1229	挿	插	手	10	S	1981	insert	ソウ、さ-す

挿入(ソウニュウ)insertion.

2046	律		彳	9	6		law	リツ、(リチ)

脈絡(ミャクラク)chain of reasoning; logical connection; coherence.

310	奇		大	8	S		strange	キ

1130	精		米	14	5		refined	セイ、(ショウ)

落書き(ラクがき)scrawl.

搾乳(サクニュウ)milking.

記念(キネン)commemoration.

2 乃

1513	透		辵	10	S		transparent	トウ、す-く、す-かす、す-ける

556	厳	嚴	口	17	6		strict	ゲン、(ゴン)、おごそ-か、きび-しい

1256	増	增	土	14	5		increase	ゾウ、ま-す、ふ-える、ふ-やす

増す(ます)to increase.

厳しさ(きびしさ)strictness.

含む(ふくむ)(v5m,vt) to contain.

1211	組		糸	11	2		association	ソ、く-む、くみ

取り組む(とりくむ)(v5m,vi) to tackle; to deal with.

1427	追		辵	9	3		follow	ツイ、お-う

追い付く(おいつく)(v5,vi) to catch up with.

1256	増	增	土	14	5		increase	ゾウ、ま-す、ふ-える、ふ-やす

増やす(ふやす)(v5,vt) to increase.

対応(タイオウ)interaction; correspondence.

実態(ジッタイ)true state; actual condition; reality.

653	拷		手	9	S		torture	ゴウ

拷問(ゴウモン)torture.

公開(コウカイ)open to the public; exhibit.

聞き取る(ききとる)to catch (a person's words); to follow; to understand.

拘束(コウソク)restriction.

806	示		示	5	5		indicate	ジ、シ、しめ-す

示す(しめす)(v5,vt) to denote; to show; to indicate.

明らか(あきらか)obvious.

告発(コクハツ)indictment; prosecution; complaint.

判明(ハンメイ)establishing; proving; identifying; confirming.

新た(あらた)new; fresh.

1529	踏		足	15	S		step	トウ、ふ-む、ふ-まえる

踏む(ふむ)(v5,vt) to step on.

1060	進		辵	11	3		advance	シン、すす-む、すす-める

前進(ゼンシン)advance; drive; progress.

待合室(まちあいシツ)waiting room.

低速(テイソク)low gear; slow speed.

走行(ソウコウ)running a wheeled vehicle.

1726	病		疒	10	3		sick	ビョウ、(ヘイ)、や-む、やまい

52	院		阜	10	3		institution	イン

病院(ビョウイン)hospital.

突入(トツニュウ)rushing; breaking into.

566	故		攴	9	5		circumstances	コ、ゆえ

事故(ジコ)accident; incident; trouble.

1437	低		人	7	4		low	テイ、ひく-い、ひく-める、ひく-まる

大分(ダイブン)considerably; greatly; a lot.

261	患		心	11	S		afflicted	カン、わずら-う

1002	障		阜	14	6		hurt	ショウ、さわ-る

507	穴		穴	5	6		hole	ケツ、あな

2104	裂		衣	12	S		split	レツ、さ-く、さ-ける

1607	破		石	10	5		rend	ハ、やぶ-る、やぶ-れる

破る(やぶる)to tear (such as paper).

1282	損		手	13	5		loss	ソン、そこ-なう、そこ-ねる

破損(ハソン)damage.

1415	調		言	15	3		investigate	チョウ、しら-べる、ととの-う、ととの-える

110	押		手	8	S		push	オウ、お-す、お-さえる

押す(おす)(v5,vt) to push.

押し退ける(おしのける)(v1,vt) to push aside.

調べる(しらべる)(v1,vt) to investigate.

1451	停		人	11	4		halt	テイ

喰う(くう)(male,vulgar) to eat

1430	通		辵	10	2		pass through	ツウ、(ツ)、とお-る、とお-す、かよ-う

妊娠(ニンシン)conception; pregnancy.

伝える(つたえる)(vt) to convey.

ふくよかplump; well-rounded.

聞く(きく)to ask; to enquire; to query.

無神経(ムシンケイ)thick-skinned; insensitive to criticism or insults.

質問(シツモン)question.

妊娠していないふくよかな女性
a plump woman who is not pregnant

無神経な質問
insensible question

妊娠していないふくよかな女性に「予定日はいつ?」と聞く無神経な質問。

関連企業(カンレンキギョウ)associated company; affiliated business.

387	挙	擧	手	10	4		raise	キョ、あ-げる、あ-がる

\ruby{4}{ヨ}\ruby{人}{ニン}に\ruby{1}{ひと}\ruby{人}{り}がX 1 in 4 people X.

新社会人(シンシャカイジン)new members of society (especially after turning 20 or joining a company); new working adults

997	傷		人	13	6		wound	ショウ、きず、いた-む、いた-める

傷付ける(きずつける)to hurt someone's feelings or pride

多過ぎる(おおすぎる)to be too many; to be too much

1497	怒		心	9	S		angry	ド、いか-る、おこ-る

怒り(いかり)anger.

挙げる(あげる)(vt)

上げる(あげる)

予定日(ヨテイび)scheduled date; expected date.

660	刻		刀	8	6		engrave	コク、きざ-む

深刻(シンコク)serious.

面倒臭い(メンドウくさい)bothersome (to do); troublesome.

愛情(アイジョウ)love; affection

安心感(アンシンカン)sense of security

1550	得		彳	11	4		acquire	トク、え-る、う-る

得る(える)(v1,vt) to get; to obtain.

560	呼		口	8	6		call	コ、よ-ぶ

悩む(なやむ)to be worried.

多い(おおい)many; numerous.

物言う(ものいう)to talk; to carry meaning.

気分(キブン)feeling; mood.

気持ち(きもち)feeling; sensation; mood.

対等(タイトウ)equality (especially of status or terms).

清楚(セイソ)neat and clean; tidy; trim.

生返事(なまヘンジ)half-hearted reply; vague answer; reluctant answer.

645	興		臼	16	5		entertain	コウ、キョウ、おこ-る、おこ-す

興味(キョウミ)interest (in something).

私に興味ないの?Aren't you interested in me?

533	献	獻	犬	13	S		offering	ケン、(コン)

1766	服		月	8	3		clothes	フク

1241	装	裝	衣	12	6		attire	ソウ、ショウ、よそお-う

1850	褒	襃	衣	15	S	1981	praise	ホウ、ほ-める

理想的(リソウテキ)ideal.

219	較		車	13	S		contrast	カク

比較(ヒカク)comparison.

共通(キョウツウ)

語らう(かたらう)(vt) to talk; to tell

155	暇		日	13	S		spare time	カ、ひま

あまりremainder; rest; residue; remnant

792	視	視 [4]	見	11	6		look at	シ

視点(シテン)opinion; point of view.

よりvsから?

293	含		口	7	S		include	ガン、ふく-む、ふく-める

含む(ふくむ)(vt) to contain

含める(ふくめる)(vt) to include

1836	放		攴	8	3		release	ホウ、はな-す、はな-つ、はな-れる、ほう-る

457	屈		尸	8	S		yield	クツ

1039	辱		辰	10	S		embarrass	ジョク、はずかし-める

屈辱的(クツジョクテキ)humiliating.

???
前で後で
前に後に
???

行動(コウドウ)action

感性(カンセイ)sensitivity

存分(ゾンブン)to one's heart's content; as much as one wants.

無意味(ムイミ)meaningless; nonsense.

料金(リョウキン)fee; price.

偏見(ヘンケン)prejudice.

外人は外国の方。

迷惑(メイワク)trouble; bother; annoyance.

自身(ジシン)self-confidence.

支配(シハイ)control.

太め(ふとめ)chubby; plump.

1191	遷		辵	15	S		transition	セン

1192	選		辵	15	4		choose	セン、えら-ぶ

1619	敗		攴	11	4		failure	ハイ、やぶ-れる

447	苦		艸	8	3		suffer	ク、くる-しい、くる-しむ、くる-しめる、にが-い、にが-る

閉じる(とじる) vs 閉める(しめる)?

窓を閉めるclose window (of a building)

タブを閉じるclose tab

1802	壁		土	16	S		wall	ヘキ、かべ

27	異		田	11	6		uncommon	イ、こと

481	掲	揭	手	11	S		put up (a notice)	ケイ、かか-げる

「小説を読もう!」は約471,337作品の小説が無料で読める小説サイトです。
``Shousetsu-wo yomou!''is a site of about 471,337 freely-readable literary-work novels/short-stories.

読める(よめる)(v1,vi)to be legible; to be readable.
to be pronounceable.
to be predictable.

9 耶(ヤ)question mark

有耶無耶(ウヤムヤ)indefinite; hazy; vague; unsettled; undecided.

1468	哲		口	10	S		philosophy	テツ

哲学者(テツガクシャ)philosopher

1192	選		辵	15	4		choose	セン、えら-ぶ

1273	属	屬	尸	12	5		belong	ゾク

無所属(ムショゾク)independent (in politics); non-partisan.

1770	福	福 [4]	示	13	3		luck	フク

幸福(コウフク)happiness.

先発(センパツ)forerunner

987	勝		力	12	3		win	ショウ、か-つ、まさ-る

決勝(ケッショウ)decision of a contest; finals (in sports).

大会(タイカイ)convention; tournament; mass meeting; rally

375	宮		宀	10	3		Shinto shrine	キュウ、グウ、(ク)、みや

23	為	爲	爪	9	S		do	イ

為(ため)sake; purpose; benefit.

お前の為に!
It's for your own good!

僕はお前の為にこんな事をしている。
I'm doing things like this for your sake.

What is the difference between 者 and 家(カ)?

掛ける、さえ

1704	眉		目	9	S	2010	eyebrow	ビ、(ミ)、まゆ

眉間(ミケン)glabella; middle forehead; area between the eyebrows.

13 睨

睨み(にらみ)glare; sharp look.

睨み付ける(にらみつける)(v1,vt) to glare at; to scowl at.

2133	惑		心	12	S		beguile	ワク、まど-う

231	掛		手	11	S		hang	か-ける、か-かる、かかり

874	種		禾	14	4		kind	シュ、たね

流石(さすが)(ateji)as one would expect.

布団(フトン)(ateji)futon (Japanese mattress).

「X」と「Y」は、どう違う?
How does X and Y differ?

「X」と「Y」の違いは?
What is the difference between X and Y?
Technically, we can parse it as
``What can you say about X and 「Y」の違い (the difference of Y)?''

素  糸 10 5  elementary ソ、ス
遊  辵 12 3  play ユウ、(ユ)、あそ-ぶ
織  糸 18 5  weave ショク、シキ、お-る
状 狀 犬 7 5  form ジョウ

現状

このページを翻訳しますか?Translate this page?

すべてall.

すべてのタブをブックマークに\ruby{追}{つい}\ruby{加}{か}Add all tabs to bookmarks.

次回(ジカイ)next time.

上書き(うわがき)(n,vs) overwrite.

ぴりりtingling; stinging; pungently.

報告(ホウコク)report; information.

1846 報  土 12 5  report ホウ、むく-いる

1343 端  立 14 S  edge タン、はし、は、はた
端末(タンマツ)(comp) (abbr) terminal; computer terminal.

為 爲 爪 9 S  do イ

行為(コウイ)act; deed; conduct.

指示(シジ)(n,vs) instruction

Xように(exp) in order to X

在日(ザイニチ)in Japan; people in Japan.

現状(ゲンジョウ)present condition; existing state; status quo.

そもそもin the first place

解決(カイケツ)(n,vs) solution.

取得(シュトク)acquisition.
資格(シカク)qualifications.
取得資格vocational certificate.

979	称	稱	禾	10	S		appellation	ショウ
愛称(アイショウ)pet name.

舞台(ブタイ)
stage (theater).

写真集(シャシンシュウ)photo album.

1153	籍		竹	20	S		enroll	セキ
書籍(ショセキ)
book; publication.

Xきっかけに(exp) with X as a start

291	鑑		金	23	S		specimen	カン、かんが-みる
鑑賞(カンショウ)appreciation (of art).

人口(ジンコウ)population.
人口10,000人(ジインコウイチマンニン)
Population: 10,000 people.

1720	票		示	11	4		ballot	ヒョウ
投票(トウヒョウ)voting; poll.

944	初		刀	7	4		first	ショ、はじ-め、はじ-めて、はつ、うい、そ-める
初出(ショシュツ)first appearance.
初め(はじめ)first doing of something.
Compare: 始め(はじめ)beginning of something.

努める(つとめる)(v1,vt)to endeavor

2048	略		田	11	5		abbreviation	リャク
2100	歴	歷	止	14	4		curriculum	レキ

略歴(リャクレキ)
brief personal record;
short curriculum vitae.

1038	職		耳	18	5		employment	ショク
就職(シュウショク)finding employment.

801	誌		言	14	6		document	シ
雑誌(ザッシ)magazine (a periodical publication).

387	挙	擧	手	10	4		raise	キョ、あ-げる、あ-がる

153	過		辵	12	5		go beyond	カ、す-ぎる、す-ごす、あやま-つ、あやま-ち

過去 (カコ) (n-adv,n)
past; bygone days.
a past (a personal history one would prefer remained secret).

過程(カテイ)progress; mechanism.

過ごす(すごす)(v5,vt) to pass time. to spend. to overdo.

1507	倒		人	10	S		overthrow	トウ、たお-れる、たお-す

1945	猛		犬	11	S		fierce	モウ

627	降		阜	10	6		descend	コウ、お-りる、お-ろす、ふ-る
降る(ふる)to come down.
雨降り(あめふり)rainfall; rainy weather.
雨が降る。It's raining.
雪が降る。It's snowing.
雨だ。It's raining.

事実上(ジジツジョウ)(n,adj-no)
as a matter of fact; actually; in reality.

長編小説(チョウヘンショウセツ)novel.

需 demand

要 need

一般(イッパン)general; universal; ordinary; average.

ぐるぐる

すてきlovely; beautiful; dreamy; great; superb; cool

% https://en.wiktionary.org/wiki/%E8%BE%9B%E3%81%84#Japanese
% https://en.wiktionary.org/wiki/%E8%BE%9B#Japanese

子供を作るto make a child

建築家(ケンチクカ)architect.

制作(セイサク)
work (film, book).
production; creation.

注目(チュウモク)attention.

活動(カツドウ)activity.

1594	念		心	8	4		thought	ネン

1700	碑	碑 [4]	石	14	S		tombstone	ヒ

懺悔(ザンゲ)repentance; confession; penitence

値打ち(ねうち)value; worth; price; dignity

寒い(さむい)cold (weather; wind); chilly

警察(ケイサツ)police

素晴らしい(すばらしい)

卒業(ソツギョウ)

擦り傷(すりきず)(n) scratch; graze; abrasion

…階建て(…カイだて)(suf) ...-story building.
7階建て 7-story building.

14 増える(ふえる)(v1,vi) to increase; to multiply

15 選 elect; select

13 連携(レンケイ)collaboration; cooperation

喧嘩(ケンカ)fight; brawl

博打(バクチ)gambling

お絵描き(おエかき)oekaki; painting; drawing

ネタバレspoiler (of a movie, a story, etc.); something that spoils the end of a movie, a story, etc.

% https://en.wikipedia.org/wiki/Administrative_divisions_of_Japan
