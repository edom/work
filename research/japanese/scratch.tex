\chapter{Scratch space}

What is the difference between 者 and 家(カ)?

\section{Not yet categorized}

「X」と「Y」は、どう違う?
How does X and Y differ?

「X」と「Y」の違いは?
What is the difference between X and Y?
Technically, we can parse it as
``What can you say about X and 「Y」の違い (the difference of Y)?''

1929 迷 辵 9 5 astray メイ、まよ-う mei, mayo-u
732 咲 口 9 S blossom さ-く sa-ku
2029 乱 亂 乙 7 6 riot ラン、みだ-れる、みだ-す ran, mida-reru, mida-su
560 呼 口 8 6 call コ、よ-ぶ ko, yo-bu
1809 返 辵 7 3 return ヘン、かえ-す、かえ-る hen, kae-su, kae-ru
1490	渡		水	12	S		transit	ト、わた-る、わた-す
渡す(わたす)to give.
62	唄		口	10	S	2010	songs with samisen	(うた)
2007 踊 足 14 S jump ヨウ、おど-る、おど-り yō, odo-ru, odo-ri
借
恥
呂
素
閃
熊
遊
織
理

吠

すべてall.

すべてのタブをブックマークに\ruby{追}{つい}\ruby{加}{か}Add all tabs to bookmarks.

次回(ジカイ)next time.

上書き(うわがき)(n,vs) overwrite.

ぴりりtingling; stinging; pungently.

報告(ホウコク)report; information.

1846	報		土	12	5		report	ホウ、むく-いる

1343	端		立	14	S		edge	タン、はし、は、はた
端末(タンマツ)(comp) (abbr) terminal; computer terminal.

23	為	爲	爪	9	S		do	イ
行為(コウイ)act; deed; conduct.

処理(ショリ)(n,vs) processing.

指示(シジ)(n,vs) instruction

Xように(exp) in order to X

在日(ザイニチ)in Japan; people in Japan.

現状(ゲンジョウ)present condition; existing state; status quo.

そもそもin the first place

1316	題		頁	18	3		topic	ダイ
問題(モンダイ)problem.

解決(カイケツ)(n,vs) solution.

取得(シュトク)acquisition.
資格(シカク)qualifications.
取得資格vocational certificate.

979	称	稱	禾	10	S		appellation	ショウ
愛称(アイショウ)pet name.

舞台(ブタイ)
stage (theater).

写真集(シャシンシュウ)photo album.

1153	籍		竹	20	S		enroll	セキ
書籍(ショセキ)
book; publication.

親友(シニュウ)close friend

母親(ははおや)mother.

Xきっかけに(exp) with X as a start

291	鑑		金	23	S		specimen	カン、かんが-みる
鑑賞(カンショウ)appreciation (of art).

人口(ジンコウ)population.
人口10,000人(ジインコウイチマンニン)
Population: 10,000 people.

1720	票		示	11	4		ballot	ヒョウ
投票(トウヒョウ)voting; poll.

944	初		刀	7	4		first	ショ、はじ-め、はじ-めて、はつ、うい、そ-める
初出(ショシュツ)first appearance.
初め(はじめ)first doing of something.
Compare: 始め(はじめ)beginning of something.

101	演		水	14	5		perform	エン
主演(シュエン)leading (starring) role.
共演(キョウエン)(vs) costarring.

1921	務		力	11	5		duty	ム、つと-める、つと-まる
努める(つとめる)(v1,vt)to endeavor

略歴(リャクレキ)
brief personal record;
short curriculum vitae.

899	就		尢	12	6		concerning	シュウ、(ジュ)、つ-く、つ-ける
1038	職		耳	18	5		employment	ショク
就職(シュウショク)finding employment.

801	誌		言	14	6		document	シ
雑誌(ザッシ)magazine (a periodical publication).

387	挙	擧	手	10	4		raise	キョ、あ-げる、あ-がる

711	催		人	13	S		sponsor	サイ、もよお-す
開催(カイサイ)holding a meeting; opening an exhibition.
主催(シュサイ)sponsorship.

509	決		水	7	3		decide	ケツ、き-める、き-まる
決まる(きまる)to be decided; to be settled. to look good in clothes.

1003	憧		心	15	S	2010	desire	ショウ、あこが-れる
憧れる(あこがれる)(v1,vi) to admire.
アイドルに憧れた。
The implied entity admired idols.

153	過		辵	12	5		go beyond	カ、す-ぎる、す-ごす、あやま-つ、あやま-ち

過去 (カコ) (n-adv,n)
past; bygone days.
a past (a personal history one would prefer remained secret).

過程(カテイ)progress; mechanism.

過ごす(すごす)(v5,vt) to pass time. to spend. to overdo.

1507	倒		人	10	S		overthrow	トウ、たお-れる、たお-す

97	猿		犬	13	S	1981	monkey	エン、さる

1945	猛		犬	11	S		fierce	モウ

2048	略		田	11	5		abbreviation	リャク

2100	歴	歷	止	14	4		curriculum	レキ

627	降		阜	10	6		descend	コウ、お-りる、お-ろす、ふ-る
降る(ふる)to come down.
雨降り(あめふり)rainfall; rainy weather.
雨が降る。It's raining.
雪が降る。It's snowing.
雨だ。It's raining.

事実上(ジジツジョウ)(n,adj-no)
as a matter of fact; actually; in reality.

事務所(ジムショ)office.

長編小説(チョウヘンショウセツ)novel.

需 demand

要 need

一般(イッパン)general; universal; ordinary; average.

ぐるぐる

すてきlovely; beautiful; dreamy; great; superb; cool

% https://en.wiktionary.org/wiki/%E8%BE%9B%E3%81%84#Japanese
% https://en.wiktionary.org/wiki/%E8%BE%9B#Japanese

子供を作るto make a child

演 performance; act; play; stage

部屋(へや)room.

建築家(ケンチクカ)architect.

制作(セイサク)
work (film, book).
production; creation.

784	姿		女	9	6		figure	シ、すがた

注目(チュウモク)attention.

活動(カツドウ)activity.

1594	念		心	8	4		thought	ネン

1700	碑	碑 [4]	石	14	S		tombstone	ヒ

懺悔(ザンゲ)repentance; confession; penitence

値打ち(ねうち)value; worth; price; dignity

寒い(さむい)cold (weather; wind); chilly

警察(ケイサツ)police

素晴らしい(すばらしい)

卒業(ソツギョウ)

擦り傷(すりきず)(n) scratch; graze; abrasion

…階建て(…カイだて)(suf) ...-story building.
7階建て 7-story building.

14 増える(ふえる)(v1,vi) to increase; to multiply

15 選 elect; select

13 連携(レンケイ)collaboration; cooperation

喧嘩(ケンカ)fight; brawl

博打(バクチ)gambling

お絵描き(おエかき)oekaki; painting; drawing

ネタバレspoiler (of a movie, a story, etc.); something that spoils the end of a movie, a story, etc.
