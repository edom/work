\chapter{Scratch space}

27	異		田	11	6		uncommon	イ、こと

481	掲	揭	手	11	S		put up (a notice)	ケイ、かか-げる

「小説を読もう!」は約471,337作品の小説が無料で読める小説サイトです。
``Shousetsu-wo yomou!''is a site of about 471,337 freely-readable literary-work novels/short-stories.

読める(よめる)(v1,vi)to be legible; to be readable.
to be pronounceable.
to be predictable.

9 耶(ヤ)question mark

有耶無耶(ウヤムヤ)indefinite; hazy; vague; unsettled; undecided.

1468	哲		口	10	S		philosophy	テツ

哲学者(テツガクシャ)philosopher

1192	選		辵	15	4		choose	セン、えら-ぶ

1273	属	屬	尸	12	5		belong	ゾク

無所属(ムショゾク)independent (in politics); non-partisan.

1770	福	福 [4]	示	13	3		luck	フク

幸福(コウフク)happiness.

先発(センパツ)forerunner

987	勝		力	12	3		win	ショウ、か-つ、まさ-る

決勝(ケッショウ)decision of a contest; finals (in sports).

大会(タイカイ)convention; tournament; mass meeting; rally

375	宮		宀	10	3		Shinto shrine	キュウ、グウ、(ク)、みや

23	為	爲	爪	9	S		do	イ

為(ため)sake; purpose; benefit.

お前の為に!
It's for your own good!

僕はお前の為にこんな事をしている。
I'm doing things like this for your sake.

What is the difference between 者 and 家(カ)?

掛ける、さえ

1704	眉		目	9	S	2010	eyebrow	ビ、(ミ)、まゆ

眉間(ミケン)glabella; middle forehead; area between the eyebrows.

13 睨

睨み(にらみ)glare; sharp look.

睨み付ける(にらみつける)(v1,vt) to glare at; to scowl at.

2133	惑		心	12	S		beguile	ワク、まど-う

231	掛		手	11	S		hang	か-ける、か-かる、かかり

874	種		禾	14	4		kind	シュ、たね

流石(さすが)(ateji)as one would expect.

布団(フトン)(ateji)futon (Japanese mattress).

「X」と「Y」は、どう違う?
How does X and Y differ?

「X」と「Y」の違いは?
What is the difference between X and Y?
Technically, we can parse it as
``What can you say about X and 「Y」の違い (the difference of Y)?''

素  糸 10 5  elementary ソ、ス
遊  辵 12 3  play ユウ、(ユ)、あそ-ぶ
織  糸 18 5  weave ショク、シキ、お-る
状 狀 犬 7 5  form ジョウ

現状

このページを翻訳しますか?Translate this page?

すべてall.

すべてのタブをブックマークに\ruby{追}{つい}\ruby{加}{か}Add all tabs to bookmarks.

次回(ジカイ)next time.

上書き(うわがき)(n,vs) overwrite.

ぴりりtingling; stinging; pungently.

報告(ホウコク)report; information.

1846 報  土 12 5  report ホウ、むく-いる

1343 端  立 14 S  edge タン、はし、は、はた
端末(タンマツ)(comp) (abbr) terminal; computer terminal.

為 爲 爪 9 S  do イ

行為(コウイ)act; deed; conduct.

指示(シジ)(n,vs) instruction

Xように(exp) in order to X

在日(ザイニチ)in Japan; people in Japan.

現状(ゲンジョウ)present condition; existing state; status quo.

そもそもin the first place

解決(カイケツ)(n,vs) solution.

取得(シュトク)acquisition.
資格(シカク)qualifications.
取得資格vocational certificate.

979	称	稱	禾	10	S		appellation	ショウ
愛称(アイショウ)pet name.

舞台(ブタイ)
stage (theater).

写真集(シャシンシュウ)photo album.

1153	籍		竹	20	S		enroll	セキ
書籍(ショセキ)
book; publication.

Xきっかけに(exp) with X as a start

291	鑑		金	23	S		specimen	カン、かんが-みる
鑑賞(カンショウ)appreciation (of art).

人口(ジンコウ)population.
人口10,000人(ジインコウイチマンニン)
Population: 10,000 people.

1720	票		示	11	4		ballot	ヒョウ
投票(トウヒョウ)voting; poll.

944	初		刀	7	4		first	ショ、はじ-め、はじ-めて、はつ、うい、そ-める
初出(ショシュツ)first appearance.
初め(はじめ)first doing of something.
Compare: 始め(はじめ)beginning of something.

努める(つとめる)(v1,vt)to endeavor

2048	略		田	11	5		abbreviation	リャク
2100	歴	歷	止	14	4		curriculum	レキ

略歴(リャクレキ)
brief personal record;
short curriculum vitae.

1038	職		耳	18	5		employment	ショク
就職(シュウショク)finding employment.

801	誌		言	14	6		document	シ
雑誌(ザッシ)magazine (a periodical publication).

387	挙	擧	手	10	4		raise	キョ、あ-げる、あ-がる

153	過		辵	12	5		go beyond	カ、す-ぎる、す-ごす、あやま-つ、あやま-ち

過去 (カコ) (n-adv,n)
past; bygone days.
a past (a personal history one would prefer remained secret).

過程(カテイ)progress; mechanism.

過ごす(すごす)(v5,vt) to pass time. to spend. to overdo.

1507	倒		人	10	S		overthrow	トウ、たお-れる、たお-す

1945	猛		犬	11	S		fierce	モウ

627	降		阜	10	6		descend	コウ、お-りる、お-ろす、ふ-る
降る(ふる)to come down.
雨降り(あめふり)rainfall; rainy weather.
雨が降る。It's raining.
雪が降る。It's snowing.
雨だ。It's raining.

事実上(ジジツジョウ)(n,adj-no)
as a matter of fact; actually; in reality.

長編小説(チョウヘンショウセツ)novel.

需 demand

要 need

一般(イッパン)general; universal; ordinary; average.

ぐるぐる

すてきlovely; beautiful; dreamy; great; superb; cool

% https://en.wiktionary.org/wiki/%E8%BE%9B%E3%81%84#Japanese
% https://en.wiktionary.org/wiki/%E8%BE%9B#Japanese

子供を作るto make a child

建築家(ケンチクカ)architect.

制作(セイサク)
work (film, book).
production; creation.

注目(チュウモク)attention.

活動(カツドウ)activity.

1594	念		心	8	4		thought	ネン

1700	碑	碑 [4]	石	14	S		tombstone	ヒ

懺悔(ザンゲ)repentance; confession; penitence

値打ち(ねうち)value; worth; price; dignity

寒い(さむい)cold (weather; wind); chilly

警察(ケイサツ)police

素晴らしい(すばらしい)

卒業(ソツギョウ)

擦り傷(すりきず)(n) scratch; graze; abrasion

…階建て(…カイだて)(suf) ...-story building.
7階建て 7-story building.

14 増える(ふえる)(v1,vi) to increase; to multiply

15 選 elect; select

13 連携(レンケイ)collaboration; cooperation

喧嘩(ケンカ)fight; brawl

博打(バクチ)gambling

お絵描き(おエかき)oekaki; painting; drawing

ネタバレspoiler (of a movie, a story, etc.); something that spoils the end of a movie, a story, etc.

% https://en.wikipedia.org/wiki/Administrative_divisions_of_Japan
