\chapter{Grammar 1}

Phrases.

(Actually, not only grammar, but also semantics, in logic.)

\section{Existence: ある、いる}

有る(ある)possess

在る(ある)exist (inanimate)

居る(いる)exist (animate)

リンゴが有る。
The implied entity has an apple.

リンゴが在る。
There is an apple.
An apple exists.

犬は居る。
There is a dog.

妹が居る。
The implied entity has a younger sister.

\section{Negation: ない}

無い(ない)not; not exist

リンゴが無い。
The implied entity does not have any apples,
or there are no apples.
(It depends on context.)

妹が居ない。
The implied entity does not have a younger sister.

\section{Perfection: た}

(``Perfect'' means ``complete'', as in ``past perfect'', not flawless.)

笑う。
The implied entity laughs.
The implied entity will laugh.

笑った。
The implied entity laughed.
The implied entity has laughed.
The implied entity had laughed.

\section{Politification: u-to-imasu}

Irregular:

する becomes します

来る(くる) becomes 来ます(きます)

Regular:

在る becomes 在ります

居る becomes 居ます

笑う becomes 笑います

\section{Polite negation: masu-to-masen}

In polite speech, to negate a verb,
politely-negate the polite form.

笑わない becomes 笑いません.
Politify 笑う to 笑います.
Politely-negate 笑います to 笑いません.

Do not politify the negative form.
Doing so would turn 笑わない to 笑わないです,
which is not the polite counterpart of 笑わない.

Do not casually-negate the polite form.
Doing so would turn 笑います to 笑いましない,
which is not Japanese.

\section{Polite perfection: masu-to-mashita}

Politely-perfect the polite form.

笑います becomes 笑いました

Politify, politely-negate, and then politely-perfect: imasen-to-imasendeshita.

笑わなかった corresponds to 笑いませんでした

\section{I-modified noun phrase}

車car

黒いblack

黒い車black car

高い人tall person

黒くない車non-black car

黒かった車formerly-black car (a car that was black)

黒くなかった車formerly-nonblack car (a car that was not black)

Start with 黒い. Negate it to 黒くない
and then perfect it to 黒くなかった.

高い黒い車expensive black car

高い黒くない車expensive non-black car

高くない黒い車non-expensive black car

高くない黒くない車non-expensive non-black car

高かった黒かった車formerly-expensive formerly-black car

\section{U-modified noun phrase}

人person

笑う人laughing person

笑わない人non-laughing person; person who does not laugh; person who is not laughing

笑わなかった人formerly-non-laughing person; person who did not laugh

笑った人person who laughed

愛される人loved person

愛されない人unloved person

愛された人formerly-loved person

愛されなかった人formerly-nonloved person; a person who was not loved

Begin with 愛 (love).
Append する, producing 愛する (to love; who loves).
Passivate する to される by u-to-areru, producing 愛される (to be loved; who is loved).
Negate される to されない by v1-ru-to-nai, producing 愛されない (who is not loved).
Perfect されない to されなかった by i-to-katta, producing 愛されなかった (who was not loved).

話されなかった言葉
words that were not spoken

\subsection{In doubt}

愛される田中さんの女性Beloved Tanaka-san's lady (or ladies)

田中さんの愛される女性Tanaka-san's beloved lady (or ladies)
