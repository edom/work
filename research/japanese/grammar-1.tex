\chapter{Grammar 1}

Throughout the series, we will build and understand increasingly complex constructions.

\section{Noun}

The adjective of ``noun'' is ``nominal''.

Nouns do not change form.

人(ひと) person

\section{Dictionary-form verb}

辞書形(ジショケイ)dictionary form

Every dictionary-form verb ends with a u-sound (u, ru, su, ku, etc.),
but not every word ending with u-sound is a dictionary-form verb.
Example: 夜(よる)is a noun, not a verb.

食べる(たべる)to eat

歩く(あるく)to walk

笑う(わらう)to laugh

\section{Verbal noun}

A verbal noun is a noun that can be turned into a dictionary-form verb
by appending suru, zuru, jiru, or something similar:

感(カン) becomes 感じる which is a dictionary-form verb.

信(シン) becomes 信じる.

安心(アンシン) becomes 安心する (but 心 on its own does not become 心する)

勉強(ベンキョウ) becomes 勉強する.

A (verbal noun + jiru/suru/zuru/etc.) is a dictionary-form verb.
