\chapter{People}

\section{People}

\subsection{Followers and friends: 6 仲共 8 供}

仲間(なかま)friend (how does this differ from 友達(ともだち)?)

共(とも)companion; follower; attendant; retinue

子供(こども)child

\subsection{Returning: 7 戻}

戻る(もどる)(vi) to turn back; to return; to go back;
to come back to a previously visited place

\subsection{Children: 9 保}

保する(ほする)to guarantee

\subsection{Child-rearing: 8 学乳}

学(ガク)learning, scholarship, erudition, knowledge.
中学(チュウガク)middle school; junior high school.
大学(ダイガク)university.
学界(ガッカイ)academic world.
科学(カガク)science.

乳(ちち)breasts.
授乳(ジュニュウ)breast-feeding.

\subsection{Spouses: 4 夫 8 妻}

夫(フ、おっと)husband

夫妻(フサイ)married couple; husband and wife

\subsection{Feelings: 6 好 8 委 13 嫌}

好き(すき)like; love; prefer

委員(イイン)committee member.
委ねる(ゆだねる)(v1,vt) to entrust to.

嫌い(きらい)hate

\subsection{4 氏}

氏 depicts a man bowing to the left.
氏(シ)(suffix,honorific) Mr.; Mrs.. family. clan.
氏(うじ)family name; birth; lineage.

\section{Body parts}

\subsection{Externals: 10 唇 16 頭}

口付け(くちづけ)(n) kiss.
口付ける(くちづける)(v1) to kiss.

唇(くちびる)(n) lips.

頭(あたま)head

\subsection{Eye-related: 7 見 9 冒 11 現 12 覚}

見る(みる)(v1,vt) to see.
見える(みえる)(v1,vi) to appear.

冒 depicts a hat obstructing the sight, implying rashness
(acting without enough thought).
冒す(おかす)(vt) to risk.
冒険(ボウケン)adventure.

現す(あらわす)(vt) to reveal; to show; to display.
現れる(あらわれる)(v1,vi) to appear; to become visible; to materialize.

覚める(さめる)(v1,vi) to wake up

\subsection{Sound-related: 6 曲 7 声 14 聞}

曲(キョク)music.
作曲(サッキョク)musical composition.

声(こえ)voice.
This character was simplified from 17 聲.

聞く(きく)to hear

\subsection{Mouth-related: 8 味 11 問}

味(あじ)flavor; taste.
美味しい(おいしい)delicious.

問(モン)(suffix, counter) counter for questions.
問う(とう)to ask (a question).
質問(シツモン)question; inquiry; enquiry.
問題(モンダイ)problem.

\subsection{Mouth-related: 6 名吸 7 告 10 舐 11 唾}

名前(なまえ)
name; full name.
given name; first name.

吸う(すう)to suck with mouth

告げる(つげる)to inform; to tell.
告白(コクハク)confess (usually of love).

舐める(なめる)(v1,vt) to lick

唾(つば)spit

\subsection{Locomotion: 7 走 8 歩}

走る(はしる)(v5r,vi) to run

歩く(あるく)to walk

\subsection{Internals: 12 筋}

筋肉(キンニク)muscle

\section{Work, craftsmanship, and family}

\subsection{Craftmanship: 3 工 7 作}

工(コウ、ク)craft.
工学(コウガク)engineering.
大工(ダイク)carpenter.
木工(モッコウ)carpenter.
工(たくみ)(name) Takumi.

作(サク)work; harvest.
作る(つくる)to make.

\subsection{Occupation: 5 仕 8 者 10 家}

仕事(シごと)work.
仕方(シかた)way; method; manner.
仕方ないit can't be helped; there's no other way.

者(シャ)(n,suf) someone of that nature; someone doing that work.
者(もの)(n) person (rarely used without a qualifier).
学者(ガクシャ)scholar.
業者(ギョウシャ)trader; merchant.
研究者(ケンキュウシャ)researcher.
作者(サクシャ)author.

家(カ)-er; -ist; someone who does something.
書家(ショカ)calligrapher.
画家(ガカ)painter.
漫画家(マンガカ)Japanese-comic-book-drawing artist.
活動家(カツドウカ)activist.
研究科(ケンキュウカ)researcher.
作家(サッカ)author; creator; writer; artist.
小説家(ショウセツカ)novelist; fiction writer.
政治家(セイジカ)politician; statesman.
作曲家(サッキョクカ)music composer.
史家(シカ)historian.

Difference between 者 and 家(カ): ???

\subsection{Domicile: 10 家}

家(うち)house.
「今夜私の家(うち)に来てください。」Please come to my house tonight.

\subsection{Family: 10 家}

家(ケ)family.
中川家(なかがわケ)the Nakagawa family.
田中家(たなかケ)the Tanaka family.
マッカーサー家(マッカーサーケ)the MacArthur family; the MacArthurs.

\section{Celestials, skies, gods, and flying: 4 日月 5 白 6 早 8 明}

日(ひ)sun.
日々(ひび)daily; days; old days.

月(つき)moon.

白(ハク)white.
白い(しろい)white.

早い(はやい)(adj-i) early.

明(メイ)bright.
明るい(あかるい)bright.

\subsection{9 星 12 朝}

日付(ひづけ)date.日付別(ひづけベツ)separate by date.(context? usage?)

星(ほし)star

朝(あさ)morning.
今朝(けさ)this morning.
早朝(ソウチョウ)early morning.

\subsection{Skies: 4 天 8 空}

天(テン) sky; heaven

空(そら)sky.
空港(クウコウ)airport.
空く(すく)(vi) to become less crowded; to get empty.
空く(あく)(vi) to be open; to be empty.

\subsection{Gods: 5 申 10 神}

申 depicts a bolt of lightning.
申す(もうす)(humble,vt) to say; to speak.

神(かみ)god; spirit; thunder

\subsection{Atmosphere: 6 気}

気(キ)spirit; mind; air; atmosphere

元気(ゲンキ)

天気(テンキ)weather

気持ち(きもち)feeling

\subsection{Seasons: 5 冬 9 春秋 10 夏}

春(はる)spring (season)

夏(なつ)summer

秋 depicts the burning of plant stalks (after harvest).
秋(あき)autumn; fall season.

冬(ふゆ)winter

\subsection{Atmospheric conditions: 8 雨 11 雪 12 雲 13 雷電}

雨(あめ)rain

雪(ゆき)snow

雲(くも)cloud

雷(かみなり)thunder

電(デン) lightning

電光(デンコウ)lightning

電気(デンキ)electricity (lit. lightning spirit)

電話(デンワ)telephone (lit. lightning talk)

電車(デンシャ)electric train (lit. lightning carriage)

電撃(デンゲキ)electric shock

電気自動車(デンキジドウシャ)electric car

\subsection{Feathers: 6 羽 10 弱 11 習}

6 羽(はね)feather

10 弱

弱い(よわい)weak

11 習 (the bottom character is自not日).

習う(ならう)(vt) to learn

練習(レンシュウ)training; practice

\section{Government}

\subsection{King: 4 王 5 玉生 8 国}

4 王(オウ)king

5 玉(ギョク、たま) ball

生depicts a sprout, something sprouting from the ground.
生(セイ)nature; sex; gender.
学生(ガクセイ)student.
生まれる(うまれる)(v1,vi) to be born.

8 国(コク、くに) country.
国王(コクオウ)king.
国内(コクナイ)internal; domestic.

\subsection{5 主 9 美皇}

5 主(おも)chief; main; principal; important

主人公(シュジンコウ)hero; main character

9 美 beauty

美味しい(おいしい)delicious (idiosyncratic reading)

美しい(うつくしい)beautiful

9 皇(コウ)imperial

皇居(コウキョ)imperial palace

\subsection{8 治}

政治(セイジ)politics.
治める(おさめる)(v1,vt)
to dominate; to rule; to govern; to manage.
to tranquilize; to pacify; to subdue.
to suppress.

治る(なおる)(vi) to heal

治(おさむ)(name) Osamu

仙台(センダイ)(city name) Sendai

\subsection{12 塔}

塔(トウ)tower

管制塔(カンセイトウ)control tower

\section{Strength: 7 助 11 動}

助ける(たすける)(v1,vt) to help

動(ドウ)motion.
動く(うごく)(vi) to move.
動画(ドウガ)animation, motion picture.
自動車(ジドウシャ)automobile.
動力(ドウリョク)power; motive power.

\section{Earth}

\subsection{Utilized land: 6 地}

地(チ)land (that is being used for an activity).
空き地(あきチ)vacant land.
耕地(コウチ)arable land.
団地(ダンチ)multi-unit apartments.

\subsection{Dirt: 9 垢}

垢(あか)dirt; filth; grime

\section{Blades}

\subsection{Tangible cutting: 4 切}

切 (spoon and sword).
切る(きる)(v5r) cut.
切(サイ、セツ).
一切(イッサイ)absolutely; (when used with negative) at all.
大切(タイセツ)(adj-na,n) important.

\subsection{Intangible cutting: 4 分}

分 depicts something separated by a blade.
分ける(わける)(v1,vt) to divide; to split; to share; to distribute.
1分(イップン)one minute.
12時34分(ジュウニジサンジュウヨンプン)12:34 (time).

\subsection{Swordtip: 4 方}

方 depicts the tip of a sword.

方(かた)(honorific) person.
あの方(あのかた)that person.

\subsection{Dissection: 10 剖}

剖(ボウ)dissection.

\subsection{Law: 6 刑 8 法 9 則 14 罰}

刑(ケイ)(n,n-suf) penalty; sentence; punishment

法(ホウ)law; rule; method; principle.

則(のり)law; rule; regulation.
法則(ホウソク)law; rule.

罰(バツ)punishment; penalty.
罰する(ばっする)to punish; to penalize.
罰金(バッキン)fine; monetary penalty.

\subsection{Separation: 7 別}

別 depicts sword cutting bone.
別な(ベツな)(adj-na) different; separate; another

\subsection{Reduction: 12 減}

減(ゲン)reduction; 10\%減 ten percent reduction.

\subsection{Burglar: 13 賊}

賊(ゾク)burglar; robber.
海賊(カイゾク)pirate; sea robber.

\subsection{War, savagery, and violence: 13 戦 15 暴 19 爆}

戦(いくさ)war.
内戦(ナイセン)civil war.
世界大戦(セカイタイセン)World War.

暴 depicts the antler of a buck, representing a savage attack, a violence.
暴動(ボウドウ)insurrection; rebellion; revolt; riot; uprising.
暴風(ボウフウ)storm; windstorm; gale.
暴れる(あばれる)(v1,vi) to rage; to act violently.

爆(バク)burst; explode; bomb.
自爆(ジバク)suicide bombing; self-destruct.
水爆(スイバク)hydrogen bomb.
原爆(ゲンバク)atomic bomb; nuclear bomb.
空爆(クウバク)aerial bombing; air raid.
爆殺(バクサツ)killing by bombing.
爆死(バクシ)death by explosion.

\section{Hand}

\subsection{Arm or hand movements: 7 投 9 指}

投げる(なげる)to throw

指(ゆび)finger.
指す(さす)(vt) to point.

\subsection{Ambiguously figurative: 11 探}

探 depicts a hand groping in a deep cave.
手探り(てさぐり)groping; fumbling.
探す(さがす)(vt)
to search for something lost.
to search for something desired.
探る(さぐる)to feel around for; to fumble for; to grope for.

\subsection{Figurative: 8 押 9 持 10 殺 11 設}

押す(おす)to push; to press; to cram into; to force.
to stamp.
to overwhelm.
押し(おし)(n) push.

持つ(もつ)to hold; to carry; to possess

殺す(ころす)to kill.
殺害(サツガイ)murder.
殺人(サツジン)murder.

設ける(もうける)to establish.

\subsection{Giving and taking: 8 取受 11 授}

取る(とる)(vt) take; fetch; take up.
買い取り(かいとり)purchase; sale. purchase on a non-return policy.

受ける(うける)(v1,vt) to receive

授ける(さずける)(v1,vt) to grant; to award.
授受(ジュジュ)give-and-receive.

\subsection{15 撃}

電撃(デンゲキ)electric shock

\section{Water}

\subsection{Feelings: 7 冷 12 温}

These adjectives describe the feeling when
touching something, not of weather or wind.

冷たい(つめたい)(adj-i) cold (of a tangible object)

温かい(あたたかい)(adj-i) warm (of a tangible object)

\subsection{State: 9 洪}

洪水(コウズイ)(n) flood (of liquid)

大水(おおみず)(n) flood (of liquid)

\subsection{Places: 6 江 8 沼 9 海津}

江(え)inlet; bay

沼(ぬま)swamp; bog

海(うみ)sea; beach.
The original character has 10 strokes.
Shinjitai replaces the two dots in the middle
with one vertical stroke.

津(つ)seaport; harbor.
津波(つなみ)(n) tsunami; tidal wave.

\subsection{Bodily fluids: 6 汗 10 涙}

汗(あせ)(n) sweat.
汗をかく(exp,v5k) to sweat.
汗を流す(exp,v5s) to work hard; to sweat.

涙(なみだ)tear (eyewater)

\subsection{Action done by the liquid: 8 波 10 凍流}

波(なみ)(n) wave (of liquid)

凍る(こおる)to freeze.
But the kanji for for ice is 氷(こおり).

流す(ながす)to flow (liquid)

\subsection{Action done to or with the liquid: 7 没沈 8 泳 9 洗 10 浮}

没(ボツ)drowning

沈む(しずむ)(vi) to sink (descend into liquid)

泳ぐ(およぐ)(vi) to swim

洗う(あらう)(vt) to wash

浮かぶ(うかぶ)to float (be supported by liquid)

\subsection{Drinks: 10 酒}

酒(さけ)sake (a Japanese liquor)

\subsection{14 滴漏}

滴(しずく)a drop of water; a drip.
滴る(したたる)to drip (fall one drop at a time).

漏れる(もれる)to leak (liquid)

\section{Heart: 4 心 5 必}

心 is involved in a lot of feeling-related characters.
心(シン、こころ)heart.
心配(シンパイ)(adj-na,n,vs) worry, concern, anxiety.
心配(シンパイ)(n,vs) care, help.

必 is unrelated to 心. They only look similar.
必ず(かならず)(adv) always, invariably, certainly.
必要(ヒツヨウ)(adj-na,n) necessity, need.

\subsection{Response: 7 応}

応え(こたえ)response; reply; answer; solution.
応える(こたえる)(v1) to respond; to reply; to answer.

\subsection{Thoughts: 7 忘 9 思}

忘れる(わすれる)(v1) to forget.
忘年会(ボウネンカイ)year-end party
(lit. forget-year meeting, a meeting to forget the year).

思う(おもう)to think

\subsection{Feelings: 12 悲 13 意感}

悲しい(かなしい)sad.
悲恋(ヒレン)disappointed love

意(イ)feelings; thoughts.
意欲(イヨク)motivation; will.
意味合い(イミあい)implication; nuance
小生意気(こなまイキ)cheekiness; impudence.

感じる(カンじる)(v1) to feel.

\subsection{Love: 10 恋 13 愛 17 優}

恋(レン、こい)romance; love; tender passion.
恋人(こいびと)lover; sweetheart.
恋文(こいぶみ)love letter.

愛(アイ)(n) love

優しい(やさしい)tender; kind; gentle; affectionate; suave

\subsection{Other: 9 急 11 悪 14 態}

急(キュウ)urgent, sudden, abrupt.
急ぐ(いそぐ)to hurry.

悪(アク)evil, wickedness.
悪人(アクニン)bad person, villain.
悪い(わるい)bad, poor; evil; unprofitable; at fault.

態(ざま)mess; sorry state; plight; sad sight.
変態(ヘンタイ)sexual perversion.

\section{Money: 10 員 11 側 12 買 15 賞賣(売)}

員(イン)(suffix) member.
工員(コウイン)factory worker.
会社員(カイシャイン)company employee.

側(がわ、かわ)side

側(そば)vicinity; near; beside

買う(かう)to buy; to purchase

賞(ショウ)prize; award

賣る(うる)to sell.
This kanji has been simplified to 7 売る.

\section{Communication: 7 言}

言contains口(mouth).
言(こと)saying.
言う(いう)to say.
言葉(ことば)word; dialect.

\subsection{Faith: 9 信}

信(シン)faith; trust.
信じる(シンじる)(v1,vt) to believe; to have faith in.

\subsection{Schemes: 9 計訂}

計(ケイ)plan.
計画(ケイカク)plan; project; schedule; scheme; program; programme.

訂正(テイセイ)correction; revision; amendment

\subsection{Mouth: 13 話 15 談}

話す(はなす)to talk.

談(ダン)discuss.
相談(ソウダン)consultation.
示談(ジダン)out-of-court settlement.
座談会(ザダンカイ)symposium; round-table discussion.

\subsection{Written: 10 記 14 読語}

記(キ)record.
記す(しるす)to record, to write down.
記録(キロク)record.
記事(キジ)article (writing).
選り抜き記事(よりぬきキジ)selected articles.
新しい記事(あたらしいキジ)new articles.

読む(よむ)to read.

語(ゴ)language.
日本語(ニホンゴ)Japanese language.
英語(エイゴ)English language.

\section{Roofs: 6 安宅 9 室}

安い(やすい)cheap; inexpensive

安全(アンゼン)safety; security

安心(アンシン)relief; peace of mind

住宅(ジュウタク)residence; housing; residential building

自宅(ジタク)one's home

自宅火災(ジタクカサイ)house fire; home fire (disaster)

室(むろ)room.

\section{Roads: 12 道 13 違}

道(みち)street; road.
鉄道(テツドウ)railway.

違う(ちがう)(vi) to differ; to not match the correct answer.

\section{Nourishment: 9 食 12 飲}

食べ物(たべもの)food

食物(ショクもの)food

食べる(たべる)(v1) to eat

飲む(のむ)to drink (any liquid, not just liquor)

\section{Illness: 10 症}

症(ショウ)(n,suf) illness

\section{Arrow, medicine, and knowledge: 5 矢 7 医 8 知}

矢(や)arrow

医(イ)medicine; healing; curing; doctor (medical)

日本人の知らない日本語the Japanese language that the Japanese people don't know

\section{Usage: 8 使}

使用(シヨウ)(n) use.
使う(つかう)to use.

\section{Sprout: 8 青毒性 11 清}

性(セイ)nature; sex; gender

男性(だんせい)male

女性(じょせい)female

青(あお)(n) blue; green.
青い(あおい)(adj-i) blue; green.
青ざめる(あおざめる)(v1,vi) to become pale.

毒(ドク)poison.
毒ガス(ドクガス)poison gas.

清い(きよい)clear; pure; noble

\section{Conjunction: 5 且 7 助 8 狙}

且つ(かつ)and.
且又(かつまた)besides; furthermore; moreover

助(すけ)assistance

助(ジョ)(pref) help; rescue; assistant

狙う(ねらう)(vt) to aim at

\section{Thing: 6 件 8 物事}

件(ケン)matter; case; item

物(ブツ、モツ、もの)thing; object; matter.
物語る(ものがたる)(vt) to tell; to indicate.
書物(ショモツ)books.
食べ物(たべもの)food.

仕事(シごと)(n) work; job; business; occupation; employment.
火事(カジ)fire (as a disaster).
有事(ユウジ)emergency.
無事(ブジ)safety; peace; quietness.

\section{Yin-yang: 12 陽}

陽(ヨウ)the yang in yin and yang

太陽(タイヨウ)sun
