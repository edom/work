\chapter{Kanji 7}

\section{7 弟}

弟(テイ、ダイ、おとうと)younger brother.

弟(おとうと)(humble) younger brother.

弟さん(おとうとさん)(honorific) younger brother.

兄弟(キョウダイ)siblings;
brothers and sisters
(although the characters mean older brother and younger brother).

\section{7 児}

児 depicts an infant with imperfect cranium (fontanelles).

児 is simplified from 兒.

乳児(ニュウジ)infant; suckling baby.

男児(ダンジ)boy; son.

\section{7 臣 10 姫 17 覧}

臣(シン、ジン)retainer.
Kangxi radical 131.

姫(ひめ)princess.

覧(ラン)perusal.

回覧(カイラン)circulation.

回覧板(カイランバン)circular notice
(especially those distributed to households within a neighborhood association).

\section{7 旲 10 莫 12 募 13 墓 14 慕暮}

旲 is an uncommon kanji.

莫大(バクダイ)enormous; vast.

募(ボ)recruit.

募集(ボシュウ)recruiting; taking applications.

応募(オウボ)application (registration).

墓(ボ、はか)grave; tomb.

慕(ボ)pining.

慕う(したう)(vt) to yearn for; to love dearly.

Xを兄のように慕っている。
The implied entity loves X as if
X were the implied entity's own older brother.

暮(ボ、くら)livelihood.

暮らす(くらす)to live; to get along.

\section{7 赤 9 変 10 恋}

赤い(あかい)(adj-i) red.

変(ヘン)strange.

変わる(かわる)to change; to transform.

変身(ヘンシン)metamorphosis; transformation.

大変(タイヘン)(adv,adj-na,n)very.

恋(レン、こい)romance; love; tender passion.

恋人(こいびと)lover; sweetheart.

恋文(こいぶみ)love letter.

\section{7 車 11 転 15 輪}

車(くるま)car; vehicle. wheel.

水車(スイシャ)water wheel.

転(テン)revolve; turn around; change.

反転(ハンテン)rolling over; turning around.

転ぶ(ころぶ)(vi) to fall down; to fall over.

輪(リン、わ)ring; circle; hoop; wheel.

指輪(ゆびわ)ring (finger accessory).

車輪(シャリン)car wheel.

五輪(ゴリン)the Olympics.

\subsection{9 軍 10 連 12 運}

軍(グン)army.

連(レン)connection; sequence; chaining.

連休(レンキュウ)consecutive holidays.

常連(ジョウレン)regular customer; regular patron.

連れる(つれる)(v1) to lead or take a person.

関連ニュース(カンレンニュース)related news.

運(ウン、はこ)carry; transport.

運ぶ(はこぶ)to carry; to transport; to move.

自動運転(ジドウウンテン)
automatic operation (machine); automatic driving (vehicle).

\section{7 改}

改める(あらためる)(v1,vt) to revise.

\section{7 我 13 義 20 議}

我(ガ、われ)ego; I; me; oneself.

自我(ジガ)self.

無我(ムガ)selflessness.

義(ギ)in-law.

義兄(ギケイ)elder-brother-in-law; elder stepbrother.

義姉(ギシ)elder-sister-in-law; elder stepsister.

義父(ギフ)father-in-law; foster father; stepfather.

定義(テイギ)definition (of terms).

原義(ゲンギ)original meaning.

同義語(ドウギゴ)synonym.

議(ギ)deliberation; consultation; debate; consideration.

議論(ギロン)argument; discussion; dispute; controversy.

\section{(7 言) 9 信計訂 10 記討 11 訪 12 詞 13 詳 14 語 15 談}

信(シン)faith; trust.

信じる(シンじる)(v1,vt) to believe; to have faith in.

計(ケイ)measure; plan.

計画(ケイカク)plan; project; schedule; scheme; program; programme.

計る(はかる)(vt) to measure.

訂正(テイセイ)correction; revision; amendment

記(キ)record.

記録(キロク)record.

記録するto set a record (such as in sports); to break a record.

記録をもつto hold such record.

記事(キジ)article (writing).

選り抜き記事(よりぬきキジ)selected articles.

新しい記事(あたらしいキジ)new articles.

記す(しるす)to record, to write down.

討(トウ)chastise.

討伐(トウバツ)subjugation; suppression.

討つ(うつ)to shoot at; to attack, defeat, destroy, avenge.

訪問(ホウモン)call; visit.

訪中(ホウチュウ)a visit to China.

訪日(ホウニチ)a visit to Japan.

訪米(ホウベイ)a visit to America.

訪れる(おとずれる)(v1,vt) to visit.

訪ねる(たずねる)to visit.

詞(シ、ことば)part of speech; words; poetry.

歌詞(カシ)song lyrics.

作詞(サクシ)song lyrics.

名詞(メイシ)noun.

詳(ショウ、くわ)detailed.

詳細(ショウサイ)details; particulars.

不詳(フショウ)unknown; unidentified; unspecified.

詳しい(くわしい)detailed; full; accurate.

語(ゴ)language.

日本語(ニホンゴ)Japanese language.

英語(エイゴ)English language.

用語(ヨウゴ)term.

言語(ゲンゴ)(linguistic) language.

プログラミング言語programming language.

語る(かたる)(vt) to talk; to tell; to recite.

談(ダン)discuss.

相談(ソウダン)consultation.

示談(ジダン)out-of-court settlement.

座談会(ザダンカイ)symposium; round-table discussion.

\section{7 攻}

攻(コウ、せめ)aggression.
攻める(せめる)(v1,vt) to attack; to assault; to assail.
攻防(コウボウ)attack and defense.

\section{7 君}

君(きみ)you.
…君(…クン)(suffix) Mr. (junior).
君主(クンシュ)ruler; monarch; sovereign.

\section{7 防}

防(ボウ)defense; resistance.

防ぐ(ふせぐ)to resist; to defend against.

防止(ボウシ)prevention; check.

\section{7 助 9 査}

助(すけ)assistance.
助(ジョ)(pref) help; rescue; assistant.
助ける(たすける)(v1,vt) to help.

査(サ)investigate.
巡査(ジュンサ)policeperson.
主査(シュサ)chief examiner; chief investigator.
査問(サモン)enquiry; hearing.

\section{7 医 8 知}

医(イ)medicine; healing; curing; doctor (medical)

知(チ)know.
知る(しる)to know.
日本人の知らない日本語the Japanese language that the Japanese people don't know

\section{7 身}

身(シン)somebody; person.

自身(ジシン)self.

私自身(わたしジシン)I myself; me myself.

出身(シュッシン)person's origin (town, city, country, etc.).

出身地(シュッシンチ)birthplace.

身長(シンチョウ)height (of body).

\section{7 売 13 続 14 読}

売 is simplified from 15 賣.

賣(バイ)sell.

賣る(うる)to sell.

続(ゾク、つづ)continue.

相続(ソウゾク)succession; inheritance.

存続(ソンゾク)duration; continuance.

続く(つづく)to continue.

読(ドク)read.

読者(ドクシャ)reader.

読む(よむ)to read.

\section{7 呆 9 保}

呆 depicts a child.

呆れる(あきれる)(v1,vi) to be amazed, astonished, astounded.

保する(ホする)to guarantee.

保つ(たもつ)to preserve.

保安(ホアン)peace preservation; security.

\section{7 豕 10 家 11 豚 12 象}

豕(いのこ)(pig radical).

家 has at least 3 meanings, depending on how it is read.

家(カ)-er; -ist; someone who does something.

書家(ショカ)calligrapher.

画家(ガカ)painter.

漫画家(マンガカ)Japanese-comic-book-drawing artist.

活動家(カツドウカ)activist.

研究科(ケンキュウカ)researcher.

作家(サッカ)author; creator; writer; artist.

小説家(ショウセツカ)novelist; fiction writer.

政治家(セイジカ)politician; statesman.

作曲家(サッキョクカ)music composer.

史家(シカ)historian.

家(ケ)family.

中川家(なかがわケ)the Nakagawa family.

田中家(たなかケ)the Tanaka family.

マッカーサー家(マッカーサーケ)the MacArthur family; the MacArthurs.

家(うち)house.

「今夜私の家(うち)に来てください。」Please come to my house tonight.

豚(ぶた)pig.

象(ショウ、ゾウ)elephant.

対象(タイショウ)target; object (of worship, study, etc.).

\section{7 谷 10 浴 11 欲}

谷(コク、たに)valley.

浴(ヨク)bathe.

欲(ヨク)longing.

欲しい(ほしい)(adj-i) want.

...欲しい(ほしい)(aux-adj) I want you to ...

食べて欲しいI want you to eat.

黙って欲しいI want you to shut up.
