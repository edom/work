\chapter{Grammar}

% https://en.wikipedia.org/wiki/Japanese_verb_conjugation

\section{Noun clause}

「文字を読み書き出来ない子供たちの未来」means
``the future of children who cannot read and write Chinese characters''.
The primary noun is 「未来」 (``future''),
which is modified by the adjective
「文字を読み書き出来ない子供たちの」
(``of children who cannot read and write Chinese characters'').
「文字を読み書き出来ない」(``unable to read and write Chinese characters'')
is an adjective that modifies 「子供たち」 (children).

\section{Vocabulary}

本(ほん)book

魚(さかな)fish

食べる(たべる)eat

読む(よむ)read

(S) is the implied subject.

\section{Present imperfect}

魚を食べる。(S) eats fish.

魚を食べない。(S) doesn't eat fish.

\section{Past perfect}

魚を食べなかった。(S) didn't eat fish.

本を読んだ。(S) read a book.

\section{Command}

魚を食べなさい。Eat fish.

魚を食べな。(childspeak?) Eat fish.

魚を食べて。Eat fish.

魚を食べないで。Don't eat fish.

\section{Passive}

食べられた。(S) was eaten.

読まれた。(S) was read.

\section{I want to ...}

食べたい。I want to eat.

読みたい。I want to read.

\section{Polite}

You have to learn to say the same thing all over again in different registers.

食べます(polite) to eat

食べません(polite) not eat

食べました(polite) ate

魚を食べました。(polite) (S) ate fish.

魚が食べられました。(polite)Fish was eaten. (???)

魚を食べません。(polite) (S) doesn't eat fish.

魚を食べないでください。(polite command) Please do not eat fish.

\section{But, although, despite, in spite of}

でも

けど

けれど

\section{Then}

そして

\section{So, thus, therefore}

だから

\section{To, for}

\section{From, to (place)}

から

より

\section{From, until (time)}

\section{In order to}

\section{Nevertheless}

\section{Whereas}

\section{On the other hand}

\section{Too bad ...}

\section{Because, because of, due to}

から

たら

ば

\section{About, regarding}

ついて

\section{About, approximately}

くらい

\section{Hmm...}

あの

えと

\section{Huh?}

え?

おれ?

あれ?

\section{Let's ...}

\section{Don't ...}

\section{Please ...}

\section{Are you sure?} 

\section{Why not ...}

\section{I think ...}

\section{Perhaps ...}

たぶん

もしかして

\section{I mean..., What I'm trying to say is..., How do I say this...}

\section{Absolutely}

\section{Precisely}

\section{Actually}

\section{Frankly}

\section{Indeed}

\section{Anyway}

\section{I'm sorry}

すま

すみません

ごめん

ごめんなさい

\section{Excuse me}

しつれいします

おじゃまします

\section{Long time no see}

おひさしぶり

\section{How dare you...}

\section{Like it or not}

\section{Or else}

\section{Questions: 7 何 15 誰}

何(なに)what

誰(だれ)who

\section{Uncategorized}

として

さすが

まじで
