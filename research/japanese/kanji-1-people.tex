\section{People-related}

\subsection{2 人 3 己大 4 犬太父 5 立}

人(ジン、レン、ニン、ひと)person; people; human

己(コ、おのれ)self

大 depicts a person with outstretched arms.
大(タイ)big.
大きい(おおきい)big.

犬(いぬ)dog.

太い(タイ、タ、ふとい)(adj-i) fat; thick.
太る(ふとる)to grow fat; to become fat; to gain weight.

父(フ、ちち、とう)father.
父(ちち)(humble) father.
お父さん(おとうさん)(honorific) father.

立 depicts a person standing on ground.
立つ(たつ)(vi) to stand

\subsection{7 弟児}

弟(テイ、ダイ、おとうと)younger brother.
弟(おとうと)(humble) younger brother.
弟さん(おとうとさん)(honorific) younger brother.
兄弟(キョウダイ)siblings;
brothers and sisters
(although the characters mean older brother and younger brother).

児 depicts an infant with imperfect cranium (fontanelles).
児 is simplified from 兒.
乳児(ニュウジ)infant; suckling baby.
男児(ダンジ)boy; son.

\subsection{Body parts: 4 心 6 自血}

心(シン、こころ)heart.

自(ジ、シ、みずか)self; oneself.
自ら(みずから)(adv) personally.
自在(ジザイ)freely (at will).
自分(ジブン)self.
自身(ジシン)self.

血(ち)blood.
止血(シケツ)stop bleeding; hemostasis.

\subsection{Administrative divisions: 5 市 9 県}

% https://en.wikipedia.org/wiki/Administrative_divisions_of_Japan

市(シ)city (an administrative division).

市(いち)market; fair (trade show).

県(ケン)prefecture.
