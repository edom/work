\chapter{Kanji}

Information is compiled from Wiktionary,
edict, compdic, kanjidic, gjiten,
and Japanese-to-English Google-Translate.

\section{History}

Japan borrowed Han characters.

To understand a character, see how it combines with other characters.

To keep you motivated, the characters have been sorted by number of strokes.

A kanji can have several readings:
go'on, kan'on, kun-yomi, name reading, and stylistic/idiosyncratic/ad-hoc reading.

新字体(シンジタイ)(lit. new Han-character body)

旧字体(キュウジタイ)(lit. old Han-character body)

A Latin alphabet letter encodes a \emph{sound}.
You can pronounce the word without understanding it.
A Han character encodes a \emph{concept}.
You can understand the meaning without pronouncing it.

Katakana shows Chinese reading and hiragana shows Japanese reading.

\section{Organization}

The characters are sorted ascending by ``scratchiness''.
Boxes are less scratchy than lines.
Lines are less scratchy than dots.
Symmetry is less scratchy than asymmetry.
Fewer strokes are less scratchy than more strokes.
However, this is subjective, approximate, and inconsistent.
I don't always follow my own rules.
The goal is to optimize the learning rate of the reader.
The reader is expected to read the character list
sequentially and repeatedly.

Level 1 looks like
an aspiring artist's minimalist abstract paintings.
Level 2 is a combinatorial explosion
of two juxtaposed abstract paintings.
Level 3 ramps up the difficulty even more with more strokes and less symmetry.
Finally, Level 4 looks like total chicken scratch to the uninitiated,
is barely legible on screen even in a 12-point sans-serif font,
and may raise questions such as
``You have a character for that?''
and
``What drug were the sages smoking?''
