\chapter{Kanji}

Information is compiled from Wiktionary,
edict, compdic, kanjidic, gjiten,
and Japanese-to-English Google-Translate.

Japanese schoolgoers take 12 years to study 2,000 characters.

\section{History}

Japan borrowed Han characters.

To understand a character, see how it combines with other characters.

To keep you motivated, the characters have been sorted by number of strokes.

A kanji can have several readings:
go'on, kan'on, kun-yomi, name reading, and stylistic/idiosyncratic/ad-hoc reading.

新字体(シンジタイ)(lit. new Han-character body)

旧字体(キュウジタイ)(lit. old Han-character body)

A Latin alphabet letter encodes a \emph{sound}.
You can pronounce the word without understanding it.
A Han character encodes a \emph{concept}.
You can understand the meaning without pronouncing it.

Katakana shows Chinese reading and hiragana shows Japanese reading.

\section{Grouping}

The goal of grouping is to allow the brain to \emph{chunk}.
When the brain recalls any of the member of the group,
it automatically recalls other members of the group.

Use recursive grouping that is 4 levels deep because \(7^4 = 2401\).

\section{Grouping by reducibility}

Chapter \(n\) should not define a kanji
that has more than \(n\) components.
Each section in a chapter groups kanji by semantic closeness
(not by visual closeness).

Example of irreducibility:
台 (tower) is treated as one component;
for pattern recognition purposes,
it is not treated as 厶 stacked on 口.

\section{A possible grouping: scratchiness}

The characters are sorted ascending by ``scratchiness''.
Boxes are less scratchy than lines.
Lines are less scratchy than dots.
Symmetry is less scratchy than asymmetry.
Fewer strokes are less scratchy than more strokes.
However, this is subjective, approximate, and inconsistent.
I don't always follow my own rules.
The goal is to optimize the learning rate of the reader.
The reader is expected to read the character list
sequentially and repeatedly.

Level 1 looks like
an aspiring artist's minimalist abstract paintings.
Level 2 is a combinatorial explosion
of two juxtaposed abstract paintings.
Level 3 ramps up the difficulty even more with more strokes and less symmetry.
Finally, Level 4 looks like total chicken scratch to the uninitiated,
is barely legible on screen even in a 12-point sans-serif font,
and may raise questions such as
``You have a character for that?''
and
``What drug were the sages smoking?''

\section{Another grouping: recency}

Group characters invented in the same period together.
Ancient Chinese concepts,
Old Chinese concepts,
Middle Chinese concepts,
Contemporary Chinese concepts.

\section{Another grouping}

Use visual similarity to group characters by concept.

Visual similarity does not always imply conceptual similarity.

\section{Historical context}

All dates are approximate.

In 12,000 BCE, dogs had been domesticated.

In 4,000 BCE, Egypt had had papyrus.

In 2,500 BCE, Egypt had hieroglyphs.

In 1,000 BCE, China had had oracle bone writing.

Around 495 BCE, Pythagoras died.

In 479 BCE, Confucius died.

Around 400 BCE, Gautama Buddha died.

In 399 BCE, Socrates died.

Around 348 BCE, Plato died.

In 323 BCE, Alexander the Great died.

In 322 BCE, Aristotle died.

In 206 BCE, China began the Han dynasty.

Around 100 BCE, China had paper.

In 44 BCE, Julius Caesar died.

Around 30, Jesus died.

In 220, China ended the Han dynasty.

In 632, Muhammad died.

In 1000, China had had gunpowder.

Around 1014 and 1031, Murasaki Shikibu, writer of The Tale of Genji, died.

Around 1056, Benedict IX died.

In 1095, the First Crusade began.

In 1185, Japan ended the Heian period, and began the Kamakura period.
Japan also had imported Han characters from China.

In 1192, Minamoto no Yoritomo officially established the Kamakura shogunate.

In 1227, Genghis Khan died.

In 1293, Raden Wijaya founded what would later be known as the Majapahit Empire.

In 1353, the Black Death ended.

In 1431, Joan of Arc died.

In 1468, Johannes Gutenberg died.

In 1506, Christopher Columbus died.

In 1519, Leonardo da Vinci died.

In 1521, Ferdinand Magellan died.

In 1546, Martin Luther died.

In 1564, Michelangelo died.

In 1582, Oda Nobunaga died.

In October 1582, Gregory XIII introduced
what would later be known as the Gregorian calendar.

In 1603, Japan began the Edo period (Tokugawa period).

In 1657, the Great fire of Meireki happened.

In 1710, The Kangxi Emperor ordered the compilation of the Kangxi Dictionary.
In 1716, it was published.

Around 1726, Isaac Newton died.

In 1750, Johann Sebastian Bach died.

In 1783, Leonhard Euler died.

In 1791, Wolfgang Amadeus Mozart died.

In 1793, Marie Antoinette died.

In 1819, James Watt died.

In 1821, Napoleon Bonaparte died.

In 1827, Ludwig van Beethoven died.

In 1823, David Ricardo died.

In 1846, Udagawa Youan died.

In 1858, Matthew Calbraith Perry died.

In 1865, Abraham Lincoln died.

In 1868, the Edo period (Tokugawa period) ended.
1868-05-03: Meiji restoration marks the end of the Edo period.

In 1886, TEPCO
(Tokyo Electric Power Company Holdings, Incorporated; 東京電力ホールディングス株式会社)
began operation as 東京電燈 (Tokyo Electric Lighting).

In 1897, Henry George died.

In 1901, Queen Victoria died.

In 1917, the Balfour Declaration supporting Zionism was made.

In 1924, Vladimir Ilyich Ulyanov (a.k.a. Lenin) died.

In 1942, the World War II began.

In 1943, Nikola Tesla died.

In 1945, Adolf Hitler died.
World War II ended in the same year.

In 1946, John Maynard Keynes died.

In 1947, the Truman Doctrine was announced,
marking the beginning of the Cold War.

1950--1953: Korean War

In 1954, Alan Turing died.

In 1955, the Vietnam War began.

In 1961, Yuri Gagarin in Vostok 1
became the first man to travel into space.

In 1963, John Fitzgerald Kennedy died.

In 1968, Martin Luther King Jr. died.

In 1970, Sukarno died.

In 1975, the Vietnam War ended.

In 1976, Mao Zedong died.

In 1980, Mohammad Hatta died.

In 1979--1989, the United States Central Intelligence Agency
carried out Operation Cyclone,
arming and financing Afghanistan mujahideen.

In 1986, Lafayette Ronald Hubbard died.

In 1991, the Soviet union ended, marking the end of the Cold War.

In 2001, the World Trade Center was destroyed.

In 2006, Milton Friedman died.
