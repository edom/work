\chapter{Kanji}

Japan borrowed Han characters.

To understand a character, see how it combines with other characters.

To keep you motivated, the characters have been sorted by number of strokes.

TODO: use katakana for Chinese reading and hiragana for Japanese reading.

A kanji can have several readings:
go'on, kan'on, kun-yomi, name reading, and stylistic/idiosyncratic/ad-hoc reading.

The characters are grouped by closeness of meaning:
聞(hear) is grouped under 耳(ear) and not 門(gate).

Information is compiled from Wiktionary
and the Japanese dictionary that comes with gjiten.

新字体(シンジタイ)(lit. new Han-character body)

旧字体(キュウジタイ)(lit. old Han-character body)
