\chapter{Kanji}

Information is compiled from Wiktionary,
edict, compdic, kanjidic, gjiten,
and Japanese-to-English Google-Translate.

Japanese schoolgoers take 12 years to study 2,000 characters.

\section{History}

Japan borrowed Han characters.

To understand a character, see how it combines with other characters.

To keep you motivated, the characters have been sorted by number of strokes.

A kanji can have several readings:
go'on, kan'on, kun-yomi, name reading, and stylistic/idiosyncratic/ad-hoc reading.

新字体(シンジタイ)(lit. new Han-character body)

旧字体(キュウジタイ)(lit. old Han-character body)

A Latin alphabet letter encodes a \emph{sound}.
You can pronounce the word without understanding it.
A Han character encodes a \emph{concept}.
You can understand the meaning without pronouncing it.

Katakana shows Chinese reading and hiragana shows Japanese reading.

\section{Grouping}

The goal of grouping is to allow the brain to \emph{chunk}.
When the brain recalls any of the member of the group,
it automatically recalls other members of the group.

Use recursive grouping that is 4 levels deep because \(7^4 = 2401\).

\section{A possible grouping: scratchiness}

The characters are sorted ascending by ``scratchiness''.
Boxes are less scratchy than lines.
Lines are less scratchy than dots.
Symmetry is less scratchy than asymmetry.
Fewer strokes are less scratchy than more strokes.
However, this is subjective, approximate, and inconsistent.
I don't always follow my own rules.
The goal is to optimize the learning rate of the reader.
The reader is expected to read the character list
sequentially and repeatedly.

Level 1 looks like
an aspiring artist's minimalist abstract paintings.
Level 2 is a combinatorial explosion
of two juxtaposed abstract paintings.
Level 3 ramps up the difficulty even more with more strokes and less symmetry.
Finally, Level 4 looks like total chicken scratch to the uninitiated,
is barely legible on screen even in a 12-point sans-serif font,
and may raise questions such as
``You have a character for that?''
and
``What drug were the sages smoking?''

\section{Another grouping: recency}

Group characters invented in the same period together.
Ancient Chinese concepts,
Old Chinese concepts,
Middle Chinese concepts,
Contemporary Chinese concepts.

\section{Another grouping}

Use visual similarity to group characters by concept.

Visual similarity does not always imply conceptual similarity.

\section{Numbers: 1 一 2 二 3 三 ... 四五六七八九十 ... 百千万}

一(イチ、ひと)one

二(ニ、ふた)two

三(サン、み)three

四 four

五 five

六 six

七 seven

八 eight

九 nine

十 ten

百(ヒャク)hundred

千(セン)thousand

万(マン)ten thousand

\section{People: 2 人 2 了 3 士女 6 耂 7 男呆}

人 depicts a person.
人(ジン、レン、ニン、ひと)person; people; human

了 depicts a wrapped baby with only head visible.

士(さむらい)gentleman; samurai.
士(シ)(suffix)
a qualified person;
a person with a qualified profession.
学士(ガクシ)university graduate (who has obtained a degree).
工学士(コウガクシ)Bachelor of Engineering.

女(おんな)woman

耂 depicts a bent-over figure with long hair, an old man.

呆 depicts a child.
呆れる(あきれる)(v1,vi) to be amazed, astonished, astounded.

男 is 田 (rice field) and 力 (strength).
男(おとこ)man.

\subsection{5 立}

立 depicts a person standing on ground.
立つ(たつ)(vi) to stand

\subsection{6 因仲 7 作戻}

因(イン)cause; factor

仲間(なかま)friend (how does this differ from 友達(ともだち)?)

7 作(サク) work; harvest

作る(つくる)to make

7 戻る(もどる)(vi) to turn back; to return; to go back;
to come back to a previously visited place

\subsection{Babies and children: 3 子 6 字 7 児 8 兒 9 保}

子is了 with arms visible.
子(シ、こ) child.
男子(ダンシ)young man, boy.
女子(ジョシ)young woman, girl.
太子(タイシ)crown prince.

字(ジ)character, letter (of an alphabet, not the letter that is a document)
赤字(あかジ)deficit; red letter; red Han-character.
漢字(カンジ)Han character.

児 is a simplification of 兒 depicting an infant
with an imperfect cranium (fontanelles).
乳児(ニュウジ)infant; suckling baby.
男児(ダンジ)boy; son.

保する(ほする)to guarantee

\subsection{Child-rearing: 8 学乳}

学(ガク)learning, scholarship, erudition, knowledge.
中学(チュウガク)middle school; junior high school.
大学(ダイガク)university.
学界(ガッカイ)academic world.
科学(カガク)science.

授乳(ジュニュウ)breast-feeding.

\subsection{Family members: 4 父 5 母兄 7 弟姉 8 妹 10 娘}

父(ちち)(humble) father.
お父さん(おとうさん)(honorific) father.

母 depicts a pair of breasts.
母(はは)mother.
お母さん(おかあさん)(honorific)

兄(あに)(humble) older brother.
お兄さん(おにいさん)(honorific) older brother.

弟(おとうと)(humble) younger brother.
弟さん(おとうとさん)(honorific) younger brother.

兄弟(キョウダイ)siblings;
brothers and sisters
(although the characters mean older brother and younger brother).

姉(あね)(humble) older sister.
お姉さん(おねえさん)(honorific) older sister.

妹(いもうと)(humble) younger sister.
妹さん(いもうとさん)(honorific) younger sister.

息子(むすこ)son

娘(むすめ)daughter

\subsection{Spouses: 4 夫 8 妻}

夫(フ、おっと)husband

夫妻(フサイ)married couple; husband and wife

\subsection{Feelings: 6 好 8 委 13 嫌}

好き(すき)like; love; prefer

委員(イイン)committee member.
委ねる(ゆだねる)(v1,vt) to entrust to.

嫌い(きらい)hate

\subsection{4 氏}

氏 depicts a man bowing to the left.
氏(シ)(suffix,honorific) Mr.; Mrs.. family. clan.
氏(うじ)family name; birth; lineage.

\subsection{Occupation: 5 仕 8 者 10 家}

仕事(シごと)work.
仕方(シかた)way; method; manner.
仕方ないit can't be helped; there's no other way.

者(シャ)(n,suf) someone of that nature; someone doing that work.
者(もの)(n) person (rarely used without a qualifier).
学者(ガクシャ)scholar.
業者(ギョウシャ)trader; merchant.
研究者(ケンキュウシャ)researcher.
作者(サクシャ)author.

家(カ)-er; -ist; someone who does something.
書家(ショカ)calligrapher.
画家(ガカ)painter.
漫画家(マンガカ)Japanese-comic-book-drawing artist.
活動家(カツドウカ)activist.
研究科(ケンキュウカ)researcher.
作家(サッカ)author; creator; writer; artist.
小説家(ショウセツカ)novelist; fiction writer.
政治家(セイジカ)politician; statesman.
作曲家(サッキョクカ)music composer.
史家(シカ)historian.

Difference between 者 and 家(カ): ???

\subsection{Domicile: 10 家}

家(うち)house.
「今夜私の家(うち)に来てください。」Please come to my house tonight.

\subsection{Family: 10 家}

家(ケ)family.
中川家(なかがわケ)the Nakagawa family.
田中家(たなかケ)the Tanaka family.
マッカーサー家(マッカーサーケ)the MacArthur family; the MacArthurs.

\section{Body parts: 5 目 6 血}

目(め)eye

血(ち)blood.
止血(シケツ)stop bleeding; hemostasis.

\subsection{Eye-related: 7 見 9 冒 11 現 12 覚}

見る(みる)(v1,vt) to see.
見える(みえる)(v1,vi) to appear.

冒 depicts a hat obstructing the sight, implying rashness
(acting without enough thought).
冒す(おかす)(vt) to risk.
冒険(ボウケン)adventure.

現す(あらわす)(vt) to reveal; to show; to display.
現れる(あらわれる)(v1,vi) to appear; to become visible; to materialize.

覚める(さめる)(v1,vi) to wake up

\subsection{Ears: 6 曲耳 7 声 14 聞}

耳(みみ)ear

曲(キョク)music.
作曲(サッキョク)musical composition.

声(こえ)voice.
This character was simplified from 17 聲.

聞く(きく)to hear

\subsection{Mouth: 3 口 6 舌 8 味 10 唇 11 問}

口(くち)mouth.
口付け(くちづけ)(n) kiss.
口付ける(くちづける)(v1) to kiss.

舌(した)tongue.

味(あじ)flavor; taste.
美味しい(おいしい)delicious.

唇(くちびる)(n) lips.

問(モン)(suffix, counter) counter for questions.
問う(とう)to ask (a question).
質問(シツモン)question; inquiry; enquiry.
問題(モンダイ)problem.

\subsection{Recency: 5 古}

古 consists of 十(ten) and 口(mouth, generation).
古い(ふるい)old.
古株(ふるかぶ)veteran; senior; old-timer.

\subsection{Mouth-related: 6 名吸 7 告}

名前(なまえ)
name; full name.
given name; first name.

吸う(すう)to suck with mouth

告げる(つげる)to inform; to tell.
告白(コクハク)confess (usually of love).

\subsection{Foot-related: 4 止 7 走 8 歩}

止める(やめる)(v1) to stop

止める(とめる)(v1) to stop

歩く(あるく)to walk

走る(はしる)(v5r,vi) to run

\subsection{Head: 9 首 16 頭}

首(くび)neck

頭(あたま)head

\subsection{Muscle: 12 筋}

筋肉(キンニク)muscle

\section{Administration}

\subsection{Regions: 4 区 5 市 8 京}

区(ク)ward; district (an administrative area).

市(シ)city (an administrative area).
市(いち)market; fair (trade show).

京(キョウ)imperial capital

\subsection{9 相省}

首相(シュショウ)prime minister

4 少 + 5 目

省く(はぶく)(vt)
to omit; to leave out; to exclude.
to curtail; to save; to cut down; to economize.

省(ショウ)(suf) ministry; department

国交省(コッコウショウ)(abbr)
Ministry of Land, Infrastructure, Transport, and Tourism

\section{Strength: 2 力 9 重}

力(リョク、リキ、ちから)strength.
百人力(ヒャクジンリキ)tremendous strength.

重 depicts a man carrying a bag.
重い(おもい)heavy (of weight).

\subsection{7 助 11 動}

助ける(たすける)(v1,vt) to help

動(ドウ)motion.
動く(うごく)(vi) to move.
動画(ドウガ)animation, motion picture.
自動車(ジドウシャ)automobile.
動力(ドウリョク)power; motive power.

\section{Earth: 3 土 4 介 5 田 9 界}

土(ド、つち)
soil; earth; ground; dirt; ground (as opposed to the heavens).

介 can mean mediation.
紹介(ショウカイ)introduction; referral.
仲介(チュウカイ)agency; intermediation.
介入(カイニュウ)intervention.
介助(カイジョ)help; assistance; aid.

介 can mean shellfish.
魚介(ギョカイ)marine products; seafood.

田(た)rice field.

界(カイ)world.
世界(セカイ)world.
学界(ガッカイ)academic world.
業界(ギョウカイ)industry world, business world.

\subsection{Land: 6 地}

地(チ)land (that is being used for an activity).
空き地(あきチ)vacant land.
耕地(コウチ)arable land.
団地(ダンチ)multi-unit apartments.

\subsection{Dirt: 9 垢}

垢(あか)dirt; filth; grime

\section{Loss: 3 亡 5 失 6 死}

亡 means death; destruction; perishment; the deceased.
亡くす(なくす)(vt) to lose something (because he/she/it dies).

失 depicts something falling from a hand.
失う(うしなう)(vt) to lose; to part with.
失明(シツメイ)loss of eyesight.
失血(シッケツ)loss of blood.

死(シ)death.
死ぬ(しぬ)to die.
死亡(シボウ)death; mortality.
死去(シキョ)death.

\section{Productive abstract concepts}

\subsection{Negation: 4 不反 12 無}

不(フ)(prefix) not; bad; poor.
不安(フアン)anxiety; insecurity.
不明(フメイ)unknown; obscure; anonymous; unidentified.

反(ハン)anti-.
反する(ハンする)to oppose; to rebel; to revolt.
反体制(ハンタイセイ)anti-establishment.

無(ム) no, -less, without.
無駄(ムダ)uselessness.
無用(ムヨウ)uselessness.
無敵(ムテキ)invincible, unrivaled (lit. no-enemy).
無茶(ムチャ)absurd, unreasonable (lit. no-tea).
無人(ムジン)unmanned (lit. no-human).
無言(ムゴン)silence (lit. no-say).

\subsection{Turning-into: 4 化}

化(カ)(suffix) -ization, -ification.
グローバル化(グローバルカ)globalization.
化ける(ばける)(v1,vi) to take the form of.
化学(カガク)chemistry.
化石(カセキ)fossilization.
分化(ブンカ)specialization.

\subsection{Not-yet and next: 5 未 7 来}

5 未(ミ)not yet.
未来(ミライ)future (lit. not yet come).
未 depicts a tree that has not fruited.
来 depicts fruits hanging on a tree.
They mean that the time to harvest has not or has come.

来(ライ)come; next.
来月(ライゲツ)next month.
来年(ライネン)next year.
「来年田中さんが日本に行きます。」Next year, Mr. Tanaka will go to Japan.
(``Next year'' means one year after the moment the speaker says it.)

来る(くる)to come.
This is an irregular verb.
The past form is 来た(きた).

\subsection{Self: 6 自}

自(ジ)self.
自ら(みずから)(adv) personally.
自在(ジザイ)freely (at will).
自分(ジブン)self (context? example usage?).

\subsection{Repetition: 6 再}

再(サイ)again, re-

再生(サイセイ)playback; rebirth

再開(サイカイ)reopening

再来(サイライ)return, comeback

\subsection{Same: 6 同}

同じ(おなじ)same.
同性愛(ドウセイアイ)same-sex love.

\subsection{Foreign: 6 外}

6 外(ガイ)foreign (not from somewhere nearby).
外人(ガイジン)foreigner, foreign person.
外国(ガイコク)foreign country.
外界(ガイカイ)outside world.
海外(カイガイ)foreign; abroad; overseas.

\subsection{Every: 7 毎}

毎(マイ)every.
毎日(マイニチ)everyday.
毎月(マイゲツ、マイつき)every month.
毎時(マイジ)every hour.
毎回(マイカイ)every time (every time it happens); every occurrence.
毎年(マイネン、マイとし)every year.

\subsection{Most: 12 最}

最(サイ)most.
最も(もっとも)most.
日本の最も高い山(ニホンのもっともたかいやま)Japan's highest mountain.
世界で最も太い人(セカイでもっともふといひと)The fattest person in the world.
最小(サイショウ)smallest.
最大(サイダイ)biggest.
最初(サイショ)first.
最後(サイゴ)last.
最新(サイシン)newest.
最高(サイコウ)best, highest, tallest.

\section{Size and amount}

\subsection{Big-small: 3 大小}

大 depicts a person with outstretched arms.
大(タイ)big.
大きい(おおきい)big.

小 depicts three sand granules.
小さい(ちいさい)small.
小(ショウ)small.

\subsection{Thick-thin: 4 太 11 細}

太い(タイ、タ、ふとい)(adj-i) fat; thick.
太る(ふとる)to grow fat; to become fat; to gain weight.

細い(サイ、ほそい)(adj-i) thin; slender.
細る(ほそる)to become thin.
細かい(こまかい)(adj-i) small; trivial.

\subsection{Few-many: 4 少 6 多}

少し(すこし)(adv, n) small quantity; few; a little

多 depicts two pieces of meat.
多 means ``many''.
多(タ)(prefix) multi-.

\subsection{Wide-narrow: 5 広 9 狭}

広い(ひろい)(adj-i) spacious; vast; wide.
広告(コウコク)advertisement.

狭める(せばめる)(v1,vt) to narrow.
狭い(せまい)(adj-i) narrow; confined; small.

\section{Blades: 2 刀刂 3 弋 4 戈 5 戊 6 戌}

刀(トン、かたな)sword.

刂 depicts a knife.

弋 depicts shooting with a bow and an arrow.

戈 depicts a spear-axe (a halberd).

戊 depicts a dagger-axe (an ancient Chinese weapon).

戌 depicts an axe.

\subsection{Tangible cutting: 4 切}

切 (spoon and sword).
切る(きる)(v5r) cut.
切(サイ、セツ).
一切(イッサイ)absolutely; (when used with negative) at all.
大切(タイセツ)(adj-na,n) important.

\subsection{Intangible cutting: 4 分}

分 depicts something separated by a blade.
分ける(わける)(v1,vt) to divide; to split; to share; to distribute.
1分(イップン)one minute.
12時34分(ジュウニジサンジュウヨンプン)12:34 (time).

\subsection{4 方}

方 depicts the tip of a sword.

方(かた)(honorific) person.
あの方(あのかた)that person.

\subsection{10 剖}

剖(ボウ)dissection.

\subsection{Law: 6 刑 8 法 9 則 14 罰}

刑(ケイ)(n,n-suf) penalty; sentence; punishment

法(ホウ)law; rule; method; principle.

則(のり)law; rule; regulation.
法則(ホウソク)law; rule.

罰(バツ)punishment; penalty.
罰する(ばっする)to punish; to penalize.
罰金(バッキン)fine; monetary penalty.

\subsection{Separation: 7 別}

別 depicts sword cutting bone.
別な(ベツな)(adj-na) different; separate; another

\subsection{9 咸 12 減}

咸 depicts a mouth and an axe.

減(ゲン)reduction; 10\%減 ten percent reduction.

\subsection{Burglar: 13 賊}

賊(ゾク)burglar; robber.
海賊(カイゾク)pirate; sea robber.

\subsection{War: 13 戦}

戦(いくさ)war

内戦(ナイセン)civil war

世界大戦(セカイタイセン)World War

\subsection{Savagery and violence: 15 暴 19 爆}

暴 depicts the antler of a buck, representing a savage attack, a violence.
暴動(ボウドウ)insurrection; rebellion; revolt; riot; uprising.
暴風(ボウフウ)storm; windstorm; gale.
暴れる(あばれる)(v1,vi) to rage; to act violently.

爆(バク)burst; explode; bomb.
自爆(ジバク)suicide bombing; self-destruct.
水爆(スイバク)hydrogen bomb.
原爆(ゲンバク)atomic bomb; nuclear bomb.
空爆(クウバク)aerial bombing; air raid.
爆殺(バクサツ)killing by bombing.
爆死(バクシ)death by explosion.

\section{Trees: 4 木}

\subsection{More trees: 4 木 8 林 12 森}

木(き)tree

林(はやし)woods; forest; copse; thicket

森(もり)forest

森林(シンリン)forest woods

\subsection{Long cylindrical object: 5 本}

5 本(ホン)book, counter for long cylindrical objects.
(In Ancient China, a book is a scroll.)
一本(イッポン)one long cylindrical object, one point.
日本(ニホン)Japan.

\subsection{Plant stalk and rice: 5 禾 6 米}

禾 depicts a plant stalk.

米(ベイ、こめ)rice.

\subsection{Rest and body: 6 休 7 体}

休む(やすむ)to rest

体(タイ、からだ)body.
肉体(ニクタイ)body; flesh.

\subsection{Trouble: 7 困}

困る(こまる)(vi) to be troubled; to be embarrassed

\section{Grass: 6 艸}

\subsection{7 花 9 草}

花(はな)flower

草(くさ)grass

\subsection{Tea: 9 茶}

茶(チャ) tea

\subsection{11 菌}

菌(キン)fungus; germ; bacterium

\section{Fire: 4 火 7 災 8 炎}

As the bottom part of another character,
the fire character is written as four dot strokes.

火(カ、ひ)fire, flame.
火事(カジ)fire (disaster).
大火(タイカ)big fire.

災い(わざわい)(n) calamity; catastrophe.
火災(カサイ)fire (disaster).

炎(ほのお)flame, blaze.
炎天(エンテン)scorching sun.

\subsection{15 熱 16 燃}

熱い(あつい)(adj)hot (temperature)

燃える(もえる)(v1,vi) to burn; to get fired up

火事(カジ)fire (disaster).
ラメン店で火事fire at a ramen shop.

\section{Cardinal directions: 5 北 6 西 8 東 9 南}

北(ホク、きた)north

西(セイ、にし)west

東(トン、ひがし)east

南(ナン、みなみ)south

They combine as in English.

北西(ホクセイ)northwest

北東(ホクトウ)northeast

東北(トウホク)Touhoku (a prefecture)

南西(ナンセイ)southwest

南東(ナントウ)southeast

\section{Countries}

日本(ニホン)Japan

中国(チュウゴク)People's Republic of China

英国(エイコク)United Kingdom

米国(ベイコク)United States of America

\section{Calendar}

These kanji readings for today, yesterday, and tomorrow are irregular.

今日(きょう)today

昨日(きのう)yesterday

明日(あした)tomorrow

Names of weekdays.

日曜日(ニチヨウび)Sunday

月曜日(ゲツヨウび)Monday

火曜日(カヨウび)Tuesday

水曜日(スイヨウび)Wednesday

木曜日(モクヨウび)Thursday

金曜日(キンヨウび)Friday

土曜日(ドヨウび)Saturday

毎日(マイニチ)everyday

Expressions.

また明日(あした)see you again tomorrow; means 'again' and 'tomorrow'

\subsection{Seasons: 5 冬 9 春秋 10 夏}

春(はる)spring (season)

夏(なつ)summer

秋 depicts the burning of plant stalks (after harvest).
秋(あき)autumn; fall season.

冬(ふゆ)winter

\section{Personal data}

出身(シュッシン)person's origin (town, city, country, etc.)

出身地(シュッシンチ)birthplace

誕生日(タンジョウビ)birthday; birth date; day of birth.

身長(シンチョウ)height (of body)

体重(タイジュウ)body weight

血液型(ケツエキガタ)blood type.「A型」type a.

好きなもの(すきなもの)likes

嫌いなもの(きらいなもの)dislikes

\section{Hand: 2 又 4 手殳}

又 depicts the right hand.
又(また)again; also.

手(て)hand.
手がける(てがける)(v1) to make; to do; to produce; to work on.
手洗い(てあらい)restroom; lavatory; toilet; a place for washing hands

殳 depicts a hand holding a tool or a weapon.

\subsection{Arm or hand movements: 7 投 9 指}

投げる(なげる)to throw

指(ゆび)finger.
指す(さす)(vt) to point.

\subsection{Ambiguously figurative: 11 探}

探 depicts a hand groping in a deep cave.
手探り(てさぐり)groping; fumbling.
探す(さがす)(vt)
to search for something lost.
to search for something desired.
探る(さぐる)to feel around for; to fumble for; to grope for.

\subsection{Figurative: 8 押 9 持 10 殺 11 設}

押す(おす)to push; to press; to cram into; to force.
to stamp.
to overwhelm.
押し(おし)(n) push.

持つ(もつ)to hold; to carry; to possess

殺す(ころす)to kill.
殺害(サツガイ)murder.
殺人(サツジン)murder.

設ける(もうける)to establish.

\subsection{Giving and taking: 8 取受 11 授}

取る(とる)(vt) take; fetch; take up.
買い取り(かいとり)purchase; sale. purchase on a non-return policy.

受ける(うける)(v1,vt) to receive

授ける(さずける)(v1,vt) to grant; to award.
授受(ジュジュ)give-and-receive.

\subsection{15 撃}

電撃(デンゲキ)electric shock

\section{4 文}

文(ブン)sentence (literature).
文字(モジ)letter (of alphabet); character (a Han character)
文書(ブンショ)sentence.
文化(ブンカ)culture.
作文(サクブン)writing.
小学生の作文(ショウガクセイのサクブン)elementary-schoolchild writing.

\section{Water: 2 冫 3 氵 4 水 5 氷}

冫ice radical

氵water radical

水(スイ、みず)water.
水車(スイシャ)water wheel.

氷(こおり)ice

\subsection{Feelings: 7 冷 12 温}

These adjectives describe the feeling when
touching something, not of weather or wind.

冷たい(つめたい)(adj-i) cold (of a tangible object)

温かい(あたたかい)(adj-i) warm (of a tangible object)

\subsection{State: 9 洪}

洪水(コウズイ)(n) flood (of liquid)

大水(おおみず)(n) flood (of liquid)

\subsection{Places: 6 江 8 沼 9 海津}

江(え)inlet; bay

沼(ぬま)swamp; bog

海(うみ)sea; beach.
The original character has 10 strokes.
Shinjitai replaces the two dots in the middle
with one vertical stroke.

津(つ)seaport; harbor.
津波(つなみ)(n) tsunami; tidal wave.

\subsection{Bodily fluids: 6 汗 10 涙}

汗(あせ)(n) sweat.
汗をかく(exp,v5k) to sweat.
汗を流す(exp,v5s) to work hard; to sweat.

涙(なみだ)tear (eyewater)

\subsection{Action done by the liquid: 8 波 10 凍流}

波(なみ)(n) wave (of liquid)

凍る(こおる)to freeze.
But the kanji for for ice is 氷(こおり).

流す(ながす)to flow (liquid)

\subsection{Action done to or with the liquid: 7 没沈 8 泳 9 洗 10 浮}

没(ボツ)drowning

沈む(しずむ)(vi) to sink (descend into liquid)

泳ぐ(およぐ)(vi) to swim

洗う(あらう)(vt) to wash

浮かぶ(うかぶ)to float (be supported by liquid)

\subsection{Drinks: 10 酒}

酒(さけ)sake (a Japanese liquor)

\subsection{14 滴漏}

滴(しずく)a drop of water; a drip.
滴る(したたる)to drip (fall one drop at a time).

漏れる(もれる)to leak (liquid)

\section{Heart: 4 心 5 必}

心 is involved in a lot of feeling-related characters.
心(シン、こころ)heart.
心配(シンパイ)(adj-na,n,vs) worry, concern, anxiety.
心配(シンパイ)(n,vs) care, help.

必 is unrelated to 心. They only look similar.
必ず(かならず)(adv) always, invariably, certainly.
必要(ヒツヨウ)(adj-na,n) necessity, need.

\subsection{What: 7 応}

応え(こたえ)response; reply; answer; solution.
応える(こたえる)(v1) to respond; to reply; to answer.

\subsection{Thoughts: 7 忘 9 思}

忘れる(わすれる)(v1) to forget.
忘年会(ボウネンカイ)year-end party
(lit. forget-year meeting, a meeting to forget the year).

思う(おもう)to think

\subsection{Feelings: 12 悲 13 意感}

悲しい(かなしい)sad.
悲恋(ヒレン)disappointed love

意(イ)feelings; thoughts.
意欲(イヨク)motivation; will.
意味合い(イミあい)implication; nuance
小生意気(こなまイキ)cheekiness; impudence.

感じる(カンじる)(v1) to feel.

\subsection{Love: 10 恋 13 愛 17 優}

恋(レン、こい)romance; love; tender passion.
恋人(こいびと)lover; sweetheart.
恋文(こいぶみ)love letter.

愛(アイ)(n) love

優しい(やさしい)tender; kind; gentle; affectionate; suave

\subsection{Other: 9 急 11 悪 14 態}

急(キュウ)urgent, sudden, abrupt.
急ぐ(いそぐ)to hurry.

悪(アク)evil, wickedness.
悪人(アクニン)bad person, villain.
悪い(わるい)bad, poor; evil; unprofitable; at fault.

態(ざま)mess; sorry state; plight; sad sight.
変態(ヘンタイ)sexual perversion.

\section{Tower: 5 台 14 臺}

台(うてな)tower; stand; pedestal.
台 is simplified form of 14 臺.

\subsection{8 治}

政治(セイジ)politics.
治める(おさめる)(v1,vt)
to dominate; to rule; to govern; to manage.
to tranquilize; to pacify; to subdue.
to suppress.

治る(なおる)(vi) to heal

治(おさむ)(name) Osamu

仙台(センダイ)(city name) Sendai

\subsection{12 塔}

塔(トウ)tower

管制塔(カンセイトウ)control tower

\section{Pictures: 5 写 7 図 8 画}

写生(シャセイ)sketch.
写真(シャシン)multimedia; photograph; movie
写す(うつす)(vt) to photograph.

図(ズ)map; drawing; picture; plan; illustration; diagram; figure; chart

画 means picture.
画(カク)counter for kanji strokes.
字画(ジカク)number of strokes in a character.
画家(ガカ)painter.

\section{Money: 7 貝 10 員 11 側 12 買 15 賞賣(売)}

貝 depicts a cowry (a kind of seashell used as money in ancient China).
貝 is not related to 目.
貝(かい)shell; shellfish.

員(イン)(suffix) member.
工員(コウイン)factory worker.
会社員(カイシャイン)company employee.

側(がわ、かわ)side

側(そば)vicinity; near; beside

買う(かう)to buy; to purchase

賞(ショウ)prize; award

賣る(うる)to sell.
This kanji has been simplified to 7 売る.

\section{Communication: 7 言}

言contains口(mouth).
言(こと)saying.
言う(いう)to say.
言葉(ことば)word; dialect.

\subsection{Faith: 9 信}

信(シン)faith; trust.
信じる(シンじる)(v1,vt) to believe; to have faith in.

\subsection{Schemes: 9 計訂}

計(ケイ)plan.
計画(ケイカク)plan; project; schedule; scheme; program; programme.

訂正(テイセイ)correction; revision; amendment

\subsection{Mouth: 13 話 15 談}

話す(はなす)to talk.

談(ダン)discuss.
相談(ソウダン)consultation.
示談(ジダン)out-of-court settlement.
座談会(ザダンカイ)symposium; round-table discussion.

\subsection{Written: 10 記 14 読語}

記(キ)record.
記す(しるす)to record, to write down.
記録(キロク)record.
記事(キジ)article (writing).
選り抜き記事(よりぬきキジ)selected articles.
新しい記事(あたらしいキジ)new articles.

読む(よむ)to read.

語(ゴ)language.
日本語(ニホンゴ)Japanese language.
英語(エイゴ)English language.

\section{Animals: 4 犬牛 6 虫 10 馬 11 魚猫豚鳥}

魚介(ギョカイ)seafood; marine products

動物(ドウブツ)animals.

犬(いぬ)dog.

牛(ギュウ、うし)
cow; bull; ox; buffalo.
beef.

虫(むし)insect; bug; cricket; moth

物(ブツ、モツ、もの)thing; object; matter.
物語る(ものがたる)(vt) to tell; to indicate.
書物(ショモツ)books.
食べ物(たべもの)food.

動物園(ドウブツエン)zoo (lit. animal garden)

馬(うま)horse.

鳥(とり)bird; chicken; chicken meat.
焼き鳥(やきとり)grilled chicken meat.

魚(さかな)fish.

猫(ねこ)cat.

豚(ぶた)pig.

\subsection{9 臭}

臭: In China 10 strokes, in Japan 9 strokes.
The lower character is 犬 in China and 大 in Japan.
臭い(くさい)(adj-i) stinking; malodorous; ill-smelling.

\section{Skies, gods, and flying}

\subsection{Skies: 4 天 8 空}

天(テン) sky; heaven

空(そら)sky.
空港(クウコウ)airport.
空く(すく)(vi) to become less crowded; to get empty.
空く(あく)(vi) to be open; to be empty.

\subsection{Gods: 5 申 10 神}

申 depicts a bolt of lightning.
申す(もうす)(humble,vt) to say; to speak.

神(かみ)god; spirit; thunder

\subsection{Atmosphere: 6 気}

気(キ)spirit; mind; air; atmosphere

元気(ゲンキ)

天気(テンキ)weather

気持ち(きもち)feeling

\subsection{Atmospheric conditions: 8 雨 11 雪 12 雲 13 雷電}

雨(あめ)rain

雪(ゆき)snow

雲(くも)cloud

雷(かみなり)thunder

電(デン) lightning

電光(デンコウ)lightning

電気(デンキ)electricity (lit. lightning spirit)

電話(デンワ)telephone (lit. lightning talk)

電車(デンシャ)electric train (lit. lightning carriage)

電撃(デンゲキ)electric shock

電気自動車(デンキジドウシャ)electric car

\subsection{Feathers: 6 羽 10 弱 11 習}

6 羽(はね)feather

10 弱

弱い(よわい)weak

11 習 (the bottom character is自not日).

習う(ならう)(vt) to learn

練習(レンシュウ)training; practice

\section{Metals: 8 金 13 鉄 14 銅銀 16 鋼}

金has a lot to do with metals.
金(キン、かね)gold; money.

金属(キンゾク)metal.
重金属(ジュウキンゾク)heavy metal.
These are also the chemistry terms.

金色(キンいろ、コンジキ)golden (color)

鉄(テツ、くろがね)iron (lit. 黒金 black metal).
鉄人(てつジン)iron man; strong man.

鉄道(テツドウ)railroad; railway

銅(ドウ、あかがね)copper (lit. 赤金 red metal)

銀(ギン、しろがね)silver (lit. 白金 white metal)

鋼(コウ、はがね)steel

青銅(セイドウ)bronze

鋼鉄(コウテツ)steel

\subsection{14 銃}

銃(ジュウ)gun; small firearms

\subsection{19 鏡}

8 金 + 5 立 + 7 見 - 1

鏡(かがみ)mirror

眼鏡(めがね)eyeglasses

\section{Roofs: 3 宀 6 安宅 9 室}

宀 depicts a roof.

安い(やすい)cheap; inexpensive

安全(アンゼン)safety; security

安心(アンシン)relief; peace of mind

住宅(ジュウタク)residence; housing; residential building

自宅(ジタク)one's home

自宅火災(ジタクカサイ)house fire; home fire (disaster)

室(むろ)room.

\section{Roads: 3 辶 12 道 13 違}

辶 is the combining form of 辵 meaning ``walking''.
Chinese dictionaries say this has 3 strokes.
Japanese dictionaries say this has 4 strokes,
but also say that 近 has 7 strokes.
If the Japanese dictionaries are to be followed,
then 近 should have 8 strokes.

道(みち)street; road.
鉄道(テツドウ)railway.

違う(ちがう)(vi) to differ; to not match the correct answer.

\subsection{Distance: 7 近 13 遠}

近い(ちかい)(adj-i) near.
最近(サイキン)most recent; recently; these days; nowadays.

遠い(とおい)(adj-i) far
