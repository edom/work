\chapter{Kanji 12}

\section{12 最}

最(サイ)most.

最も(もっとも)most.

日本の最も高い山(ニホンのもっともたかいやま)Japan's highest mountain.

世界で最も太い人(セカイでもっともふといひと)The fattest person in the world.

最小(サイショウ)smallest.

最大(サイダイ)biggest.

最初(サイショ)first.

最後(サイゴ)last.

最新(サイシン)newest.

最高(サイコウ)best, highest, tallest.

\section{12 着}

着(チャク、き、ぎ)wear; clothing.
古着(ふるぎ)old clothes; second-hand clothes.
下着(シタぎ)underwear.
上着(うわぎ)coat; jacket; outer garment.
水着(みずぎ)bathing suit; swimsuit.
着用(チャクヨウ)wearing (uniform; seat belt).
着る(きる)(v1,vt) to wear (in modern Japanese, from the shoulders down).

着(チャク、つ)arrival.
先着(センチャク)first arrival.
新着(シンチャク)new arrival.
着く(つく)to arrive; to reach.

\section{12 報}

報(ホウ)report; information; news.
報じる(ホウじる)(v1,vt) to report; to inform.
報いる(むくいる)(v1,vt) to reward; to recompense; to repay.

\section{12 無}

無(ム) no, -less, without.
無駄(ムダ)uselessness.
無用(ムヨウ)uselessness.
無敵(ムテキ)invincible, unrivaled (lit. no-enemy).
無茶(ムチャ)absurd, unreasonable (lit. no-tea).
無人(ムジン)unmanned (lit. no-human).
無言(ムゴン)silence (lit. no-say).

\section{12 番 18 翻}

番(バン)number.
番組(バンぐみ)television program.

翻 飜 飛 18 S  flip ホン、ひるがえ-る、ひるがえ-す

翻訳(ホンヤク)translation.
