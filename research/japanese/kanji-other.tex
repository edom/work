\chapter{Other characters}

\section{Counting}

5 半(ハン)half

6 百(ヒャク)hundred

3 千(セン)thousand

3 万(マン)ten thousand

\section{動物 Animals}

動物(ドウブツ)animals.

物(ブツ).
Semantic 牛(cow) and phonetic 勿(ブツ).

動物園(ドウブツエン)zoo (lit. animal garden)

10 馬(うま)horse.

11 鳥(とり)bird; chicken; chicken meat.
焼き鳥(やきとり)grilled chicken meat.

11 猫(ねこ)cat.

11 豚(ぶた)pig.

\section{Personal data}

出身(シュッシン)person's origin (town, city, country, etc.)

出身地(シュッシンチ)birthplace

誕生日(タンジョウビ)birthday; birth date; day of birth.

身長(シンチョウ)height (of body)

体重(タイジュウ)body weight

血液型(ケツエキガタ)blood type.「A型」type a.

好きなもの(すきなもの)likes

嫌いなもの(きらいなもの)dislikes

\section{Hard-to-translate}

さすが

\section{Axe radical}

axe

感じる(カンじる)to feel

海賊(カイゾク)pirate; sea robber

咸 (mouth and axe)

減(ゲン)reduction; 10\%減 ten percent reduction.

\section{Country names}

米穀(ベイコク)United States of America

\section{Weekdays and calendars}

These kanji readings for today, yesterday, and tomorrow are irregular.

今日(きょう)today

昨日(きのう)yesterday

明日(あした)tomorrow

Names of weekdays.

日曜日(ニチヨウび)Sunday

月曜日(ゲツヨウび)Monday

火曜日(カヨウび)Tuesday

水曜日(スイヨウび)Wednesday

木曜日(モクヨウび)Thursday

金曜日(キンヨウび)Friday

土曜日(ドヨウび)Saturday

毎日(マイニチ)everyday

Expressions.

また明日(あした)see you again tomorrow; means 'again' and 'tomorrow'

\section{Family}

TODO inner-outer

父(ちち)father

母(はは)mother
