\chapter{Other characters}

\section{2 刀(トウ)sword}

刀(かたな)katana (Japanese sword)

\subsection{4 切 (spoon and sword)}

切る(きる)(v5r) cut

\subsection{4 方(かた)(tip of a sword)}

\section{3 女(おんな)woman}

6 好き(すき)like

13 嫌い(きらい)hate

\section{3 山(サン、やま) mountain}

山(やま)mount, mountain

\section{3 工(コウ、ク) craft}

工学(コウガク)engineering

工業(コウギョウ)manufacturing industry

工作(コウサク)work; construction; handicraft

大工(ダイク)carpenter

工(たくみ)(name) Takumi

\section{3 小(ショウ)small}

(three sand granules)

小さい(ちいさい)small

\subsection{4 少 few}

少し(すこし)(adv, n) small quantity; few; a little

止める(やめる)to stop

止める(とめる)to stop

歩く(あるく)to walk

\section{3 土(つち) earth; soil}

9 室(むろ)room (the character consists of roof, private, and earth)

\section{4 中(チュウ、なか) middle}

中学(チュウガク) middle school; junior high school

…の中(…のなか)middle of something

田中(たなか)(name) Tanaka

中山(なかやま)(name) Nakayama

中川(なかがわ)(name) Nakagawa

\section{4 王 king}

\subsection{5 玉(ギョク、たま) ball}

\subsection{8 国(コク、くに) country}

\subsection{9 美 beauty}

美味しい(おいしい)delicious (idiosyncratic reading)

美しい(うつくしい)beautiful

\section{4 天(テン) sky; heaven}

天気(テンキ)weather

\section{4 火(カ、ひ)fire}

The fire radical becomes four dot strokes at the bottom side.

4 火(ひ)fire, flame

火事(カジ)fire (disaster).
ラメン店で火事fire at a ramen shop.

\subsection{8 炎(ほのお)flame, blaze}

炎天(エンテン)scorching sun

\subsection{熱い(あつい)(adj)hot (temperature)}

\subsection{燃える(もえる)(v1,vi) to burn}

\section{4 斤(キン) (axe)}

\subsection{13 新(シン) new}

(cut tree down with an axe)

新しい(あたらしい)new

新聞(シンブン)newspaper (lit. new hear)

新車(シンシャ)new car

\section{5 母 (breasts)}

母depicts a pair of breasts.

母(はは)mother

\subsection{7 毎(マイ) every}

毎日(マイニチ)everyday

毎月(マイゲツ、マイつき)every month

毎時(マイジ)every hour

毎回(マイカイ)every time

毎年(マイネン、マイとし)every year

\subsection{8 海(うみ)(water breast)sea}

\section{5 生 (sprout)}

生depicts something sprouting from the ground.

5 生(セイ)nature; sex; gender

生まれる(うまれる)(v1,vi) to be born

8 性(セイ)nature; sex; gender

男性(だんせい)male

女性(じょせい)female

\section{5 目 eye}

目(め)eye

見る(みる)(v1,vt) to see

見える(みえる)(v1,vi) to appear

\section{5 田 rice field}

\subsection{7 男(おとこ)man}

男is田(rice field) and 力(strength).

\subsection{9 界(カイ)world}

学界(ガッカイ)academic world

世界(セカイ)world

業界(ギョウカイ)industry world, business world

\subsection{6 自(ジ) self}

自分(ジブン)self (context? example usage?)

自爆(ジバク)suicide bombing, self-destruct (lit. self explode)

自動車(ジドウシャ)automobile (lit. self-moving carriage)

自ら(みずから)(adv) personally

\section{6 手(て) hand}

\section{6 回(カイ) (spiral)}

回す(まわす)to turn; to rotate

一回(イッカイ)once; one time

一回目(イッカイめ)first

\section{6 全(ゼン) whole}

(a whole piece of jade)

全部(ゼンブ)altogether; everything

全く(まったく)(adv) completely, entirely, wholly, totally

\section{6 外(ガイ) foreign}

外人(ガイジン)foreigner, foreign person

外国(ガイコク)foreign country

外界(ガイカイ)outside world

次(つぎ)next (in sequence) (?)

\section{6 糸(いと) thread}

\section{6 西(にし) west}

\section{7 形(かたち) shape; form}

\section{8 金(キン、かね)gold; money}

金has a lot to do with metals.

金属(キンゾク)metal. This is also the chemistry term.

重金属(ジュウキンゾク)heavy metal. This is also the chemistry term.

金色(キンいろ)golden (color)

金色(コンジキ)golden (color)

13 鉄(テツ、くろがね)iron (lit. 黒金 black metal)

14 銀(ギン、しろがね)silver (lit. 白金 white metal)

14 銅(ドウ、あかがね)copper (lit. 赤金 red metal)

16 鋼(コウ、はがね)steel

青銅(セイドウ)bronze

鋼鉄(コウテツ)steel

\section{8 門(かど) gate}

\subsection{12 開 open}

開く(ひらく)to open (door, business, eye, mouth, ...)

\subsection{12 間 interval}

\section{9 食(ショク) food, eat}

食べ物(たべもの)food

食物(ショクもの)food

食べる(たべる)(v1) to eat

飲む(のむ)to drink (any liquid, not just liquor)

\section{12 無(ム) no, -less, without}

無駄(ムダ)uselessness

無用(ムヨウ)uselessness

無敵(ムテキ)invincible, unrivaled (lit. no-enemy)

無茶(ムチャ)absurd, unreasonable (lit. no-tea)

無人(ムジン)unmanned (lit. no-human)

無言(ムゴン)silence (lit. no-say)

\section{6 再(サイ) again, re-}

再生(サイセイ)playback; rebirth

再開(サイカイ)reopening

再来(サイライ)return, comeback

\section{6 曲(キョク)music}

作曲(さっきょく)music

音楽(おんがく)music

楽しい(たのしい)happy

了解(りょうかい)understanding; roger that

\section{6 耳(みみ) ear}

14 聞 hear

聞く(きく)to hear

\section{8 隹 (short-tailed bird)}

誰 (だれ)who

\section{10 書(ショ)write}

It's a hand holding a pen writing on paper.

Is this related to 事(ごと)?

書く(かく)to write

\section{12 最(サイ) most}

最も(もっとも)most.
日本の最も高い山(ニホンのもっともたかいやま)Japan's highest mountain.
世界で最も太い人(セカイでもっともふといひと)The fattest person in the world.

最小(サイショウ)smallest

最大(サイダイ)biggest

最初(サイショ)first

最後(サイゴ)last

最新(サイシン)newest

最高(サイコウ)best, highest, tallest

\section{7 貝(かい)cowry}

(Cowry is a kind of seashell used as money in ancient China.)

買う(かう)to buy; to purchase

賣る(うる)to sell.
This kanji has been simplified to 売る.

\section{4 殳 (a hand holding a tool; weapon)}

This appears in
殺す(ころす)(to kill),
電撃(デンゲキ)(electric shock),
投げる(なげる)(to throw),
and 設ける(もうける)(to establish).

\section{Counting}

5 半(ハン)half

6 百(ヒャク)hundred

3 千(セン)thousand

3 万(マン)ten thousand

\section{Not yet grouped}

9 前(まえ)before (time), in front of

身長(しんちょう)height (of body)

最長(さいちょう)longest, tallest

5 石(いし)stone

左(ひだり)left

右(みぎ)right

頭(あたま)head

名詞(メイシ)noun

文書(ブンショ)sentence

戻る(もどる)to go back

歌う(うたう)to sing

笑う(わらう)to laugh; to smile

若い(わかい)

仕事(シごと)(n) work; job; business; occupation; employment

TODO

the right-side radical of 使う(つかう)to use

8 店(テン) (n) store; shop

ラメン店ramen shop (ramen is a kind of Japanese noodle)

\section{Animals}

動物(ドウブツ)animals.
The character says ``moving product''.

牛(ギュウ、うし)
cow; bull; ox; buffalo.
beef.

物(ブツ).
Semantic 牛(cow) and phonetic 勿(ブツ).

動物園(ドウブツエン)zoo (lit. animal garden)

4 犬(いぬ)dog.

10 馬(うま)horse.

11 鳥(とり)bird; chicken; chicken meat.
焼き鳥(やきとり)grilled chicken meat.

11 猫(ねこ)cat.

11 豚(ぶた)pig.

\section{Grass}

7 花(はな)flower

9 草(くさ)grass
