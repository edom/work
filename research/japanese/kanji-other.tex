\chapter{Other characters}

\section{3 山(サン、やま) mountain}

山(やま)mount, mountain

\section{3 工(コウ、ク) craft}

工学(コウガク)engineering

工業(コウギョウ)manufacturing industry

工作(コウサク)work; construction; handicraft

大工(ダイク)carpenter

工(たくみ)(name) Takumi

\section{4 止}

止める(やめる)to stop

止める(とめる)to stop

歩く(あるく)to walk

走る(はしる)(v5r,vi) to run

\section{3 土(つち) earth; soil}

9 室(むろ)room (the character consists of roof, private, and earth)

\section{4 中(チュウ、なか) middle}

中学(チュウガク) middle school; junior high school

…の中(…のなか)middle of something

田中(たなか)(name) Tanaka

中山(なかやま)(name) Nakayama

中川(なかがわ)(name) Nakagawa

\section{4 斤 13 新}

4 斤(キン) (axe)

13 新(シン) new

(cut tree down with an axe)

新しい(あたらしい)new

新聞(シンブン)newspaper (lit. new hear)

新車(シンシャ)new car

\section{5 母 7 毎 8 海}

母depicts a pair of breasts.

母(はは)mother

7 毎(マイ) every

毎日(マイニチ)everyday

毎月(マイゲツ、マイつき)every month

毎時(マイジ)every hour

毎回(マイカイ)every time

毎年(マイネン、マイとし)every year

8 海(うみ)(water breast)sea

\section{5 生 (sprout)}

生depicts something sprouting from the ground.

5 生(セイ)nature; sex; gender

生まれる(うまれる)(v1,vi) to be born

8 性(セイ)nature; sex; gender

男性(だんせい)male

女性(じょせい)female

\section{6 全(ゼン) whole}

(a whole piece of jade)

全部(ゼンブ)altogether; everything

全く(まったく)(adv) completely, entirely, wholly, totally

\section{6 外(ガイ) foreign}

外人(ガイジン)foreigner, foreign person

外国(ガイコク)foreign country

外界(ガイカイ)outside world

次(つぎ)next (in sequence) (?)

\section{6 西(にし) west}

\section{7 形(かたち) shape; form}

\section{9 食(ショク) food, eat}

食べ物(たべもの)food

食物(ショクもの)food

食べる(たべる)(v1) to eat

飲む(のむ)to drink (any liquid, not just liquor)

\section{12 無(ム) no, -less, without}

無駄(ムダ)uselessness

無用(ムヨウ)uselessness

無敵(ムテキ)invincible, unrivaled (lit. no-enemy)

無茶(ムチャ)absurd, unreasonable (lit. no-tea)

無人(ムジン)unmanned (lit. no-human)

無言(ムゴン)silence (lit. no-say)

\section{6 再(サイ) again, re-}

再生(サイセイ)playback; rebirth

再開(サイカイ)reopening

再来(サイライ)return, comeback

\section{6 曲(キョク)music}

作曲(さっきょく)music

音楽(おんがく)music

楽しい(たのしい)happy

了解(りょうかい)understanding; roger that

\section{6 耳(みみ) ear}

14 聞 hear

聞く(きく)to hear

\section{8 隹 (short-tailed bird)}

誰 (だれ)who

\section{10 書(ショ)write}

It's a hand holding a pen writing on paper.

Is this related to 事(ごと)?

書く(かく)to write

\section{12 最(サイ) most}

最も(もっとも)most.
日本の最も高い山(ニホンのもっともたかいやま)Japan's highest mountain.
世界で最も太い人(セカイでもっともふといひと)The fattest person in the world.

最小(サイショウ)smallest

最大(サイダイ)biggest

最初(サイショ)first

最後(サイゴ)last

最新(サイシン)newest

最高(サイコウ)best, highest, tallest

\section{6 死(シ)death}

死亡(シボウ)death; mortality

\section{8 使(シ)use}

使う(つかう)to use

使用(シヨウ)(n) use

\section{8 店(テン) (n) store; shop}

ラメン店ramen shop (ramen is a kind of Japanese noodle)

\section{15 暴(バク)(antler of a buck; savage attack)}

暴れる(あばれる)(v1,vi) to rage; to act violently

\subsection{19 爆(バク)burst; explode; bomb}

自爆(ジバク)suicide bombing

水爆(スイバク)hydrogen bomb

原爆(ゲンバク)atomic bomb; nuclear bomb

空爆(クウバク)aerial bombing; air raid

爆殺(バクサツ)killing by bombing

爆死(バクシ)death by explosion

\section{Counting}

5 半(ハン)half

6 百(ヒャク)hundred

3 千(セン)thousand

3 万(マン)ten thousand

\section{Animals}

動物(ドウブツ)animals.
The character says ``moving product''.

牛(ギュウ、うし)
cow; bull; ox; buffalo.
beef.

物(ブツ).
Semantic 牛(cow) and phonetic 勿(ブツ).

動物園(ドウブツエン)zoo (lit. animal garden)

4 犬(いぬ)dog.

10 馬(うま)horse.

11 鳥(とり)bird; chicken; chicken meat.
焼き鳥(やきとり)grilled chicken meat.

11 猫(ねこ)cat.

11 豚(ぶた)pig.

\section{Grass}

7 花(はな)flower

9 草(くさ)grass

\section{Personal data}

出身(シュッシン)person's origin (town, city, country, etc.)

出身地(シュッシンチ)birthplace

誕生日(タンジョウビ)birthday; birth date; day of birth.

身長(シンチョウ)height (of body)

体重(タイジュウ)body weight

血液型(ケツエキガタ)blood type.「A型」type a.

好きなもの(すきなもの)likes

嫌いなもの(きらいなもの)dislikes

\section{Hard-to-translate}

さすが

\section{4 反(ハン)anti-}

反する(ハンする)to oppose; to rebel; to revolt

反体制(ハンタイセイ)anti-establishment

\section{8 長(チョウ)long (distance or time); leader}

長い(ながい)long (distance); long (time)

身長(シンチョウ)height (of body)

最長(サイチョウ)longest, tallest

社長(シャチョウ)company president

\section{変(ヘン)strange}

変態(ヘンタイ)sexual perversion

変身(ヘンシン)metamorphosis; transformation

\section{Axe radical}

axe

感じる(カンじる)to feel

海賊(カイゾク)pirate; sea robber

咸 (mouth and axe)

減(ゲン)reduction; 10\%減 ten percent reduction.

\section{Country names}

米穀(ベイコク)United States of America

\section{Weekdays and calendars}

These kanji readings for today, yesterday, and tomorrow are irregular.

今日(きょう)today

昨日(きのう)yesterday

明日(あした)tomorrow

Names of weekdays.

日曜日(ニチヨウび)Sunday

月曜日(ゲツヨウび)Monday

火曜日(カヨウび)Tuesday

水曜日(スイヨウび)Wednesday

木曜日(モクヨウび)Thursday

金曜日(キンヨウび)Friday

土曜日(ドヨウび)Saturday

毎日(マイニチ)everyday

Expressions.

また明日(あした)see you again tomorrow; means 'again' and 'tomorrow'

\section{6 羽(はね)feather}

10 弱

弱い(よわい)weak

11 習 (the bottom character is自not日).

習う(ならう)(vt) to learn

練習(レンシュウ)training; practice

\section{Family}

TODO inner-outer

父(ちち)father

母(はは)mother

兄(あに)older brother

姉(あね)older sister

兄弟(キョウダイ)siblings
