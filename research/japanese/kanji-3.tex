\chapter{Kanji 3}

\section{Components: 14 菐}

菐 is the 14-stroke ``thicket'' radical.

\section{13 遠園}

遠(エン)far.

遠い(とおい)(adj-i) far (spatial distance).

園(エン、その)park; garden; yard; farm.

動物園(ドウブツエン)zoo; animal park; zoological garden.

幼稚園(ヨウチエン)kindergarten.

\section{2 日 9 昼}

昼(チュウ、ひる)daytime.

\section{2 刀 9 削前契 13 解}

削(サク)shave; sharpen; delete.

削る(けずる)to shave; to sharpen; to erase; to delete.

前(ゼン、まえ)before (time), in front of.

午前(ゴゼン)morning; before noon; a.m. (ante meridien).

契(ケイ)pledge.

契約(ケイヤク)contract.

契る(ちぎる)(v5r,vt) to pledge; to promise; to swear.

解(ゲ、カイ)untie.

了解(リョウカイ)understanding; roger that.

見解(ケンカイ)opinion; point of view.

専門家見解(センモンカケンカイ)expert opinion.

解熱(ゲネツ)alleviation of fever.

解約(カイヤク)cancellation of contract.

解禁(カイキン)lifting of a ban.

解く(とく)to solve; to answer. to untangle (hair).

\section{2 力 7 努 12 筋}

努める(つとめる)(v1,vt) to endeavor; to try; to strive.
努力(ドリョク)great effort; exertion; endeavor

筋肉(キンニク)muscle.

\section{2 人 7 伴 13 僧}

伴(ハン、バン)consort.

伴う(ともなう)to accompany.

僧(ソウ)monk; priest.

\section{3 彳 9 律後 15 衝}

律(リツ、リチ)law.
黄金律(オウゴンリツ)Golden Rule.

後(ゴ、コウ、のち、あと)after.
後ろ(うしろ)back; behind; rear.
食後(ショクゴ)after a meal; postprandial.
今後(コンゴ)from now on; hereafter.

衝(ショウ)collide.

\section{3 土 9 垢 12 場塔}

垢(あか)dirt; filth; grime

場(ジョウ、ば)place.

塔(トウ)tower.
管制塔(カンセイトウ)control tower.

\section{3 夕 8 夜}

夜(ヤ、よる)night.

\section{4 水 12 温}

温かい(あたたかい)(adj-i) warm (of a tangible object; to the touch)

\section{4 水 7 沈 8 泳 10 浮}

沈下(チンカ)sinking; subsidence.

沈む(しずむ)(vi) to sink (descend into liquid).
This was simplified from 18 瀋.

泳ぐ(およぐ)(vi) to swim

浮かぶ(うかぶ)to float (be supported by liquid)

\section{4 水 10 凍}

凍る(こおる)to freeze.
But the kanji for for ice is 氷(こおり).

\section{4 水 9 津}

津(つ)seaport; harbor.

津波(つなみ)(n) tsunami; tidal wave.

\section{4 水 13 準 14 滴}

準(シュン)level; standard.

水準(スイジュン)water level. level; standard.

滴(しずく)a drop of water; a drip.

滴る(したたる)to drip (fall one drop at a time).

\section{4 手 11 接}

接(セツ)touch; contact.

接吻(セップン)kiss.

接ぐ(つぐ)to join (two things into one); to piece together; to graft.

接する(セッする)to come in contact with; to touch.

\section{4 木 10 格}

格(カク)personality; character.

性格(セイカク)
character; personality; disposition; nature.

\section{5 癶 9 発}

癶 depicts footsteps, dotted tent, or legs.

発(ハツ、ホツ)departure.

発車(ハッシャ)(n,vs) departure of a vehicle.

発達(ハッタツ)development; growth.

\section{9 勇 10 通 12 痛}

勇(ユウ)courage.

勇む(いさむ)to be in high spirits.

通(ツウ)pass through.

通る(とおる)to go by.

痛(ツウ、いた)pain; hurt; damage; bruise.

苦痛(クツウ)pain; agony; bitterness.

痛む(いたむ)to hurt; to feel pain. to be injured.

痛い(いたい)painful; sore.

胃が痛い(イがいたい)The stomach is aching.

\section{5 田 11 野}

野(ヤ、の)field; plains; rustic.

\section{(6 糸) 7 系 15 線}

系(ケイ)lineage.

線(セン、すじ)line; stripe.
line (telephone line).
line (of a railroad).
打線(ダセン)baseball lineup.
前線(ゼンセン)front line; weather front.

\section{(6 糸) 10 紙 11 細}

紙(シ、かみ)paper.

手紙(てがみ)letter (the document, not the alphabet).

細(サイ)thin.

細い(ほそい)(adj-i) thin; slender.

細る(ほそる)to become thin.

細かい(こまかい)(adj-i) small; trivial.

\section{(6 糸) 11 経}

経(ケイ、キョウ)manage.

経つ(たつ)(vi) to pass; to lapse.

経る(へる)to pass; to elapse; to experience.

\section{7 車 10 連 11 転 15 輪}

車(くるま)car; vehicle. wheel.

水車(スイシャ)water wheel.

連(レン)connection; sequence; chaining.

連休(レンキュウ)consecutive holidays.

常連(ジョウレン)regular customer; regular patron.

連れる(つれる)(v1) to lead or take a person.

関連ニュース(カンレンニュース)related news.

転(テン)revolve; turn around; change.

反転(ハンテン)rolling over; turning around.

転ぶ(ころぶ)(vi) to fall down; to fall over.

輪(リン、わ)ring; circle; hoop; wheel.

指輪(ゆびわ)ring (finger accessory).

車輪(シャリン)car wheel.

五輪(ゴリン)the Olympics.

\section{7 花 9 草茶 11 菌}

花(はな)flower

草(くさ)grass

茶(チャ、サ) tea

菌(キン)fungus; germ; bacterium

\section{7 改}

改める(あらためる)(v1,vt) to revise.

\section{7 我 13 義}

我(ガ、われ)ego; I; me; oneself.

自我(ジガ)self.

無我(ムガ)selflessness.

義(ギ)in-law.

義兄(ギケイ)elder-brother-in-law; elder stepbrother.

義姉(ギシ)elder-sister-in-law; elder stepsister.

義父(ギフ)father-in-law; foster father; stepfather.

定義(テイギ)definition (of terms).

原義(ゲンギ)original meaning.

同義語(ドウギゴ)synonym.

\section{(7 言) 9 信計訂 10 記討 11 訪 12 詞 13 詳 14 語 15 談}

信(シン)faith; trust.

信じる(シンじる)(v1,vt) to believe; to have faith in.

計(ケイ)measure; plan.

計画(ケイカク)plan; project; schedule; scheme; program; programme.

計る(はかる)(vt) to measure.

訂正(テイセイ)correction; revision; amendment

記(キ)record.

記録(キロク)record.

記録するto set a record (such as in sports); to break a record.

記録をもつto hold such record.

記事(キジ)article (writing).

選り抜き記事(よりぬきキジ)selected articles.

新しい記事(あたらしいキジ)new articles.

記す(しるす)to record, to write down.

討(トウ)chastise.

討伐(トウバツ)subjugation; suppression.

討つ(うつ)to shoot at; to attack, defeat, destroy, avenge.

訪問(ホウモン)call; visit.

訪中(ホウチュウ)a visit to China.

訪日(ホウニチ)a visit to Japan.

訪米(ホウベイ)a visit to America.

訪れる(おとずれる)(v1,vt) to visit.

訪ねる(たずねる)to visit.

詞(シ、ことば)part of speech; words; poetry.

歌詞(カシ)song lyrics.

作詞(サクシ)song lyrics.

名詞(メイシ)noun.

詳(ショウ、くわ)detailed.

詳細(ショウサイ)details; particulars.

不詳(フショウ)unknown; unidentified; unspecified.

詳しい(くわしい)detailed; full; accurate.

語(ゴ)language.

日本語(ニホンゴ)Japanese language.

英語(エイゴ)English language.

用語(ヨウゴ)term.

言語(ゲンゴ)(linguistic) language.

プログラミング言語programming language.

語る(かたる)(vt) to talk; to tell; to recite.

談(ダン)discuss.

相談(ソウダン)consultation.

示談(ジダン)out-of-court settlement.

座談会(ザダンカイ)symposium; round-table discussion.

\section{8 隹 12 集雇 13 稚}

隹(ふるとり)depicts a short-tailed bird or an old bird.

集(シュウ)gather; meet; congregate; swarm; flock.

集める(あつめる)(v1,vt) to collect; to assemble; to gather.

雇(コ)employ.

解雇(カイコ)dismissal; firing; layoff.

雇う(やとう)to employ. to hire; to charter.

稚(チ)immature; young.

幼稚(ヨウチ)infancy; childish; infantile.

\section{8 直 10 真}

直(ジキ)soon.

直(チョク)direct; in person; frankness; honesty.

直す(なおす)to heal; to cure.
Chinese 直 has 7 strokes.
Japan adds the lower-left corner stroke.

直(ただち)

真(シン、ま)true.
真意(シンイ)real intention; true motive; true meaning.
真っ黒(まっくろ)pitch black.
真っ先(まっさき)the foremost; the beginning.

\section{8 金 14 銃 19 鏡}

銃(ジュウ)gun; small firearms

鏡(かがみ)mirror.

眼鏡(めがね)eyeglasses.

\section{8 実店例}

実(ジツ)truth; reality.
実(み)fruit.
実る(みのる)to bear fruit; to ripen.

店(テン)(n) store; shop.
ラメン店ramen shop (ramen is a kind of Japanese noodle).

店(みせ).

例えば(たとえば)for example.
例える(たとえる)(v1,vt)
to compare; to liken; to illustrate.
用例(ヨウレイ)example; illustration.

\section{9 追 12 運}

追(ツイ、お)chase; follow; pursue.
追う(おう)to chase; to run after; to follow.
追い風(おいかぜ)tailwind; favorable wind.

運(ウン、はこ)carry; transport.
運ぶ(はこぶ)to carry; to transport; to move.
自動運転(ジドウウンテン)
automatic operation (machine); automatic driving (vehicle).

\section{9 品}

品(しな)article; item; thing; goods; stock.
品(ヒン)quality.
品物(しなもの)goods.
作品(サクヒン)work (book; film; composition; etc.). opus.
日用品(ニチヨウヒン)daily necessities.
一品(イッピン)one item; one article; one course of meal.
上品(ジョウヒン)elegant; refined; polished.
品性(ヒンセイ)character (elegant attitude; elegant behavior).

\section{9 面 11 異}

異(イ)uncommon; different; unusual.
異国(イコク)foreign country.
異性(イセイ)different sex; opposite sex.
異なる(ことなる)to differ; to vary; to disagree.

\section{10 帰}

帰(キ)homecoming.
帰る(かえる)(v5r,vi)
to return; to come home; to go home; to go back.

\section{10 除}

除(ジョ、ジ、のぞ)exclude.
除く(のぞく)to exclude.
削除(サクジョ)elimination; cancellation; deletion.

\section{10 唇}

唇(くちびる)(n) lips.

\section{11 鳥 14 鳴}

鳴(メイ)chirp.

鳴く(なく)to chirp; to sing (bird).

\section{11 理}

理(リ、ことわり)reason.
料理(リョウリ)cooking; cookery; cuisine.

\section{11 済脳啓祭教}

済ませる(すませる)(v1,vt) to finish; to end.
経済(ケイザイ)economy.

脳内(ノウナイ)intracranial; inside the brain

啓(ケイ).
拝啓(ハイケイ):拝啓… Dear ...

祭り(まつり)feast; festival

教(キョウ)teach.
教育(キョウイク)training; education.
教会(キョウカイ)church.

教え(おしえ)teaching; doctrine.
教える(おしえる)(v1,vt) to teach.

\section{12 着}

着(チャク、き、ぎ)wear; clothing.
古着(ふるぎ)old clothes; second-hand clothes.
下着(シタぎ)underwear.
上着(うわぎ)coat; jacket; outer garment.
水着(みずぎ)bathing suit; swimsuit.
着用(チャクヨウ)wearing (uniform; seat belt).
着る(きる)(v1,vt) to wear (in modern Japanese, from the shoulders down).

着(チャク、つ)arrival.
先着(センチャク)first arrival.
新着(シンチャク)new arrival.
着く(つく)to arrive; to reach.

\section{12 堺堕奥報}

堺(カイ、さかい)world.

堕(ダ)degeneration; degradation.
堕胎(ダタイ)abortion; feticide; babykilling.
堕落(ダラク)depravity; corruption; degradation.

奥(おく)interior.
奥山(おくやま)remote mountain.

報(ホウ)report; information; news.
報じる(ホウじる)(v1,vt) to report; to inform.
報いる(むくいる)(v1,vt) to reward; to recompense; to repay.

\section{13 新}

新(シン)new.
最新(サイシン)newest.

新しい(あたらしい)(adj-i) new.

\section{13 数禁楽暖業}

数(スウ、かず)number; amount; count.
数える(かぞえる)(v1,vt) to count; to enumerate.
算数(サンスウ)arithmetics.
数万(スウマン)tens of thousands.

禁じる(キンじる)(v1,vt) to prohibit.

楽(ガク、ラク)music; pleasure.
音楽(オンガク)music.

楽しい(たのしい)happy.

暖かい(あたたかい)(adj-i) warm; genial

業(ギョウ)industry.
工業(コウギョウ)manufacturing industry.

\section{14 算際僕概}

算(サン)calculation.
算出(サンシュツ)calculation; computation.
加算(カサン)addition.
引き算(ひきザン)subtraction.
公算(コウサン)probability; likelihood.

際(サイ)occasion; circumstances.
際限(サイゲン)limits; bounds.
学祭(ガクサイ)interdisciplinary.
国際(コクサイ)international.

僕(ボク)I; me (male).

概要(ガイヨウ)outline; summary

\section{15 質}

質(シツ)(suffix) substance; quality; matter.
質問(シツモン)question.

\section{15 膝}

膝(ひざ)knee; lap

\section{15 監稿誰}

監(カン)government official; rule; administer.
監禁(カンキン)confinement; bondage.

稿(コウ)draft; copy; manuscript

誰(だれ)who

\section{16 親諦頭}

親(シン、おや)parent.
両親(リョウシン)both parents.

諦める(あきらめる)(v1,vt)
to give up; to abandon.

頭(トウ、あたま)head

\section{18 験顔曜類}

験(ケン)verify.
実験(ジッケン)experiment.
治験(チケン)clinical trial.
受験(ジュケン)taking an examination (such as school and university entrance).

顔(ガン、かお)face.
顔面(ガンメン)face (of a person).

曜(ヨウ)(weekday name).
日曜日(ニチヨウビ)Sunday.

類(ルイ)kind; sort; type.
人類(ジンルイ)mankind.

\section{Craft: 3 工}

工作(コウサク)work; construction; handicraft.

\section{Countries}

日本(ニホン)Japan

中国(チュウゴク)People's Republic of China

英国(エイコク)United Kingdom

米国(ベイコク)United States of America

\section{Personal data}

誕生日(タンジョウビ)birthday; birth date; day of birth.

体重(タイジュウ)body weight

血液型(ケツエキガタ)blood type.「A型」type a.

好きなもの(すきなもの)likes

嫌いなもの(きらいなもの)dislikes

\section{Ungrouped}

警察(ケイサツ)police

素晴らしい(すばらしい)

卒業(ソツギョウ)

擦り傷(すりきず)(n) scratch; graze; abrasion

今度(コンド)
now; this time; this occurrence.
next time; another time.

…階建て(…カイだて)(suf) ...-story building.
7階建て 7-story building.

14 増える(ふえる)(v1,vi) to increase; to multiply

15 選 elect; select

13 連携(レンケイ)collaboration; cooperation

喧嘩(ケンカ)fight; brawl

博打(バクチ)gambling

お絵描き(おエかき)oekaki; painting; drawing

ネタバレspoiler (of a movie, a story, etc.); something that spoils the end of a movie, a story, etc.

関係(カンケイ)relation; connection.
肉体関係(ニクタイカンケイ)sexual relations.
垢と一切関係ないIt has absolutely nothing to do with dirt.
