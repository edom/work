\chapter{Kanji 3}

\section{Danger: 6 危}

危ない(あぶない)dangerous

\section{Countries}

日本(ニホン)Japan

中国(チュウゴク)People's Republic of China

英国(エイコク)United Kingdom

米国(ベイコク)United States of America

\section{Personal data}

出身(シュッシン)person's origin (town, city, country, etc.)

出身地(シュッシンチ)birthplace

誕生日(タンジョウビ)birthday; birth date; day of birth.

身長(シンチョウ)height (of body)

体重(タイジュウ)body weight

血液型(ケツエキガタ)blood type.「A型」type a.

好きなもの(すきなもの)likes

嫌いなもの(きらいなもの)dislikes

\section{Industry: 5 业 13 業 14 菐}

13 業 contains 业未.
14 菐 contains 业土人.

業(ギョウ)industry

工業(コウギョウ)manufacturing industry

工作(コウサク)work; construction; handicraft

\section{Evil: 4 凶 11 脳}

凶(キョウ)evil; villain; bad luck; disaster.
凶悪(キョウアク)(adj-na) atrocious; fiendish; brutal; villainous.

脳内(ノウナイ)intracranial; inside the brain

\section{Strange: 9 変}

変(ヘン)strange.
変わる(かわる)to change; to transform.
変身(ヘンシン)metamorphosis; transformation.
大変(タイヘン)(adv,adj-na,n)very.

\section{Train station: 14 駅}

駅 is simplified from 23 驛.
駅(エキ)train station.

\section{Doctrine: 11 教}

教会(キョウカイ)church

\section{Happiness}

\subsection{Excitement: 8 昂}

昂る(たかぶる)(vi) to get excited; to get worked up

\subsection{Happiness: 13 楽}

音楽(オンガク)music

楽しい(たのしい)happy

\section{Kind: 18 類}

類(ルイ)kind; sort; type

人類(ジンルイ)mankind

\section{Baseball: 15 線}

打線(ダセン)baseball lineup

\section{Interior: 12 奥}

奥(おく)interior.
奥山(おくやま)remote mountain.

\section{Confusing characters}

\subsection{Spirit and cloth: 4 礻 5 衤}

\subsection{3 夂夊 4 攵 8 㑒}

In the Japanese language,
these characters become parts of other characters
instead of being used on their own.

夂 depicts two legs followed by something from behind.

夊 depicts a footprint.

攵 is a variant of 攴 depicting a branch and a hand.

㑒 is simplified from the 13-stroke 僉
meaning ``all, together, unanimous''.

\subsection{Samurai and earth: 3 士土}

士(samurai) has longer upper horizontal stroke.
土(earth) has shorter upper horizontal stroke.

\subsection{Hat, sun, moon, meat, inner}

冃(hat)

日(sun)

月(moon)

\subsection{石 (stone) and 右 (right)}

\subsection{人 (person) and 入 (enter)}

\subsection{王 (king) and 生 (sprout)}

\subsection{業菐美}
