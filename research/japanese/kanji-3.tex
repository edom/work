\chapter{Kanji 3}

\section{6 危 7 努 8 店 9 垢 10 時唇}

危ない(あぶない)dangerous

努める(つとめる)(v1,vt) to endeavor; to try; to strive.
努力(ドリョク)great effort; exertion; endeavor

店(テン)(n) store; shop.
ラメン店ramen shop (ramen is a kind of Japanese noodle).

垢(あか)dirt; filth; grime

時(とき)time.
時代(ジダイ)era.
三国時代(サンゴクジダイ)The Three Kingdoms period.
戦国時代(センゴクジダイ)The Warring States period.

唇(くちびる)(n) lips.

\section{11 啓祭教清}

啓(ケイ).
拝啓(ハイケイ):拝啓… Dear ...

祭り(まつり)feast; festival

教え(おしえ)teaching; doctrine.
教育(キョウイク)training; education.
教会(キョウカイ)church.

清い(きよい)clear; pure; noble

\section{12 奥詞筋}

奥(おく)interior.
奥山(おくやま)remote mountain.

詞(シ).
名詞(メイシ)noun.

筋肉(キンニク)muscle.

\section{13 禁準楽暖業}

禁じる(キンじる)(v1,vt) to prohibit.

準(シュン).
水準(スイジュン)water level. level; standard.

楽しい(たのしい)happy.
音楽(オンガク)music.

暖かい(あたたかい)(adj-i) warm; genial

業(ギョウ)industry.
工業(コウギョウ)manufacturing industry.

\section{14 菐僕概}

菐 the 14-stroke ``thicket'' radical.

僕(ボク)I; me (male).

概要(ガイヨウ)outline; summary

\section{15 線監編 16 諦頭 18 類}

線(セン)line; stripe.
line (telephone line).
line (of a railroad).
打線(ダセン)baseball lineup.

監(カン)government official; rule; administer.
監禁(カンキン)confinement; bondage.

編む(あむ)(vt)
to knit; to plait; to braid.
to compile (an anthology); to edit.

諦める(あきらめる)(v1,vt)
to give up; to abandon.

頭(あたま)head

類(ルイ)kind; sort; type.
人類(ジンルイ)mankind.

\section{15 稿誰暴}

稿(コウ)draft; copy; manuscript

誰(だれ)who

暴 depicts the antler of a buck, representing a savage attack, a violence.
暴動(ボウドウ)insurrection; rebellion; revolt; riot; uprising.
暴風(ボウフウ)storm; windstorm; gale.
暴れる(あばれる)(v1,vi) to rage; to act violently.

\section{19 爆}

爆(バク)burst; explode; bomb.
自爆(ジバク)suicide bombing; self-destruct.
水爆(スイバク)hydrogen bomb.
原爆(ゲンバク)atomic bomb; nuclear bomb.
空爆(クウバク)aerial bombing; air raid.
爆殺(バクサツ)killing by bombing.
爆死(バクシ)death by explosion.

\section{9 信}

信(シン)faith; trust.
信じる(シンじる)(v1,vt) to believe; to have faith in.

\section{9 計}

計(ケイ)plan.
計画(ケイカク)plan; project; schedule; scheme; program; programme.

\section{Revision: 7 改 9 訂}

7 改める(あらためる)(v1,vt) to revise.

訂正(テイセイ)correction; revision; amendment

\section{Mouth: 13 話 15 談}

話す(はなす)to talk.

談(ダン)discuss.
相談(ソウダン)consultation.
示談(ジダン)out-of-court settlement.
座談会(ザダンカイ)symposium; round-table discussion.

\section{Written: 10 記 14 読語}

記(キ)record.
記す(しるす)to record, to write down.
記録(キロク)record.
記事(キジ)article (writing).
選り抜き記事(よりぬきキジ)selected articles.
新しい記事(あたらしいキジ)new articles.

読む(よむ)to read.

語(ゴ)language.
日本語(ニホンゴ)Japanese language.
英語(エイゴ)English language.

\section{Calendar: 18 曜}

These kanji readings for today, yesterday, and tomorrow are irregular.

今日(きょう)today

昨日(きのう)yesterday

明日(あした)tomorrow

Names of weekdays.

日曜日(ニチヨウび)Sunday

月曜日(ゲツヨウび)Monday

火曜日(カヨウび)Tuesday

水曜日(スイヨウび)Wednesday

木曜日(モクヨウび)Thursday

金曜日(キンヨウび)Friday

土曜日(ドヨウび)Saturday

毎日(マイニチ)everyday

Expressions.

また明日(あした)see you again tomorrow; means 'again' and 'tomorrow'

\section{4 凶}

凶(キョウ)evil; villain; bad luck; disaster.
凶悪(キョウアク)(adj-na) atrocious; fiendish; brutal; villainous.

\section{11 脳}

脳内(ノウナイ)intracranial; inside the brain

\section{Grass: 7 花 9 草茶 11 菌}

花(はな)flower

草(くさ)grass

茶(チャ) tea

菌(キン)fungus; germ; bacterium

\section{Economy}

経済(ケイザイ)economy

貿易(ボウエキ)trade (foreign)

\section{Fire: 15 熱 16 燃}

熱い(あつい)(adj)hot (temperature)

燃える(もえる)(v1,vi) to burn; to get fired up

火事(カジ)fire (disaster).
ラメン店で火事fire at a ramen shop.

\section{Craft: 3 工}

工作(コウサク)work; construction; handicraft.

\section{Countries}

日本(ニホン)Japan

中国(チュウゴク)People's Republic of China

英国(エイコク)United Kingdom

米国(ベイコク)United States of America

\section{Personal data}

出身(シュッシン)person's origin (town, city, country, etc.)

出身地(シュッシンチ)birthplace

誕生日(タンジョウビ)birthday; birth date; day of birth.

身長(シンチョウ)height (of body)

体重(タイジュウ)body weight

血液型(ケツエキガタ)blood type.「A型」type a.

好きなもの(すきなもの)likes

嫌いなもの(きらいなもの)dislikes

普通

\section{Ungrouped}

警察(ケイサツ)police

素晴らしい(すばらしい)

卒業(ソツギョウ)

議論(ギロン)

擦り傷(すりきず)(n) scratch; graze; abrasion

今度(コンド)
now; this time; this occurrence.
next time; another time.

率(リツ)(suf) rate; ratio; proportion.
識字率(シキジリツ)literacy rate.

…階建て(…カイだて)(suf) ...-story building.
7階建て 7-story building.

了解(りょうかい)understanding; roger that.
見解(ケンカイ)opinion; point of view.
専門家見解(センモンカケンカイ)expert opinion.

操る(あやつる)(vt) to be fluent in (a language)

増える(ふえる)(v1,vi) to increase; to multiply

覧 perusal

報 report; news

選 elect; select

野 plains; rustic

転 revolve

連携(レンケイ)collaboration; cooperation

接吻(セップン)kiss

喧嘩(ケンカ)fight; brawl

博打(バクチ)gambling

間違う(まちがう)

国際(コクサイ)international

お絵描き(おエかき)oekaki; painting; drawing

ネタバレspoiler (of a movie, a story, etc.); something that spoils the end of a movie, a story, etc.

堕ちる(おちる)(v1,vi)to fall down; to drop (?)

関係(カンケイ)relation; connection

肉体関係(ニクタイカンケイ)sexual relations

垢と一切関係ないIt has absolutely nothing to do with dirt.

趣味(シュミ)hobby; taste, preference.

邪魔(ジャマ)intrusion
邪魔するto intrude

膝 knee; lap

僧 monk

質 substance

特 special

情 feelings; emotion; passion; sympathy
