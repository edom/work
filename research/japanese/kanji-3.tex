\chapter{Kanji 3}

\section{4 水 8 沼波 9 津 10 浮涙凍流 11 清}

浮かぶ(うかぶ)to float (be supported by liquid)

涙(なみだ)tear (eyewater)

凍る(こおる)to freeze.
But the kanji for for ice is 氷(こおり).

流す(ながす)to flow (liquid)

清い(きよい)clear (the character consists of water and blue 青)

沼(ぬま)swamp; bog

波(なみ)(n) wave (of liquid)

津(つ)seaport; harbor.
津波(つなみ)(n) tsunami; tidal wave.

\section{4 心 7 忘 9 急 9 思 10 恋 11 悪 12 悲}

心 is involved in a lot of feeling-related characters.

心配(シンパイ)(adj-na,n,vs) worry, concern, anxiety

心配(シンパイ)(n,vs) care, help

7 忘 forget

忘れる(わすれる)(v1) to forget

忘年会(ボウネンカイ)year-end party (lit. forget-year meeting, a meeting to forget the year)

9 急(キュウ) urgent, sudden, abrupt

急ぐ(いそぐ)to hurry

9 思 think

思is田(rice field) and 心(heart).

思う(おもう)to think

10 恋(レン、こい) romance, love

恋(こい)(n) love, tender passion

恋人(こいびと)lover; sweetheart

恋文(こいぶみ)love letter

11 悪(アク) bad, evil, wicked

悪(アク)evil, wickedness

悪人(アクニン)bad person, villain

悪い(わるい)bad, poor; evil; unprofitable; at fault

12 悲(ヒ) sad

悲しい(かなしい)sad

悲恋(ヒレン)disappointed love

\section{4 殳 10 殺}

殳 depicts a hand holding a tool or a weapon.

殺す(ころす)to kill

投げる(なげる)to throw

設ける(もうける)to establish

殺害(サツガイ)murder

殺人(サツジン)murder

\section{Road radical}

違う(ちがう)

遠い(とおい)far

近い(ちかい)near

\section{6 交 10 校}

交わる(まじわる)cross; intersect; join; meet (?)

学校(ガッコウ)school

\section{8 狙}

狙う(ねらう)(vt) to aim at

\section{12 陽}

陽(ヨウ)the yang in yin and yang

太陽(タイヨウ)sun

\section{12 筋}

筋肉(キンニク)muscle

\section{10 時}

時(とき)time

時代(ジダイ)era.
三国時代(サンゴクジダイ)The Three Kingdoms period.
戦国時代(センゴクジダイ)The Warring States period.

\section{11 教}

教会(キョウカイ)church

\section{13 戦}

戦(いくさ)war

内戦(ナイセン)civil war

世界大戦(セカイタイセン)World War

\section{9 室}

9 室(むろ)room

\section{6 再}

再(サイ)again, re-

再生(サイセイ)playback; rebirth

再開(サイカイ)reopening

再来(サイライ)return, comeback

\section{12 最}

最(サイ)most

最も(もっとも)most.
日本の最も高い山(ニホンのもっともたかいやま)Japan's highest mountain.
世界で最も太い人(セカイでもっともふといひと)The fattest person in the world.

最小(サイショウ)smallest

最大(サイダイ)biggest

最初(サイショ)first

最後(サイゴ)last

最新(サイシン)newest

最高(サイコウ)best, highest, tallest

\section{12 無}

無(ム) no, -less, without

無駄(ムダ)uselessness

無用(ムヨウ)uselessness

無敵(ムテキ)invincible, unrivaled (lit. no-enemy)

無茶(ムチャ)absurd, unreasonable (lit. no-tea)

無人(ムジン)unmanned (lit. no-human)

無言(ムゴン)silence (lit. no-say)

\section{6 当}

当たり(あたり)

本当(ホントウ)
