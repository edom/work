\chapter{Kanji 3}

\section{3 廾 4 开升}

廾 depicts two hands.

开 is a simplification of 幵 (raising both hands).

升(ショウ)measuring box.

\subsection{(3 廾) 14 算}

算(サン)calculation.

算出(サンシュツ)calculation; computation.

加算(カサン)addition.

引き算(ひきザン)subtraction.

公算(コウサン)probability; likelihood.

\subsection{(4 升) 8 昇}

昇(ショウ)rise up.

\subsection{(4 开) 6 刑 9 型 12 開}

刑(ケイ)(n,n-suf) penalty; sentence; punishment

型(ケイ、かた)model.

血液型(ケツエキがた)blood type.「A型」type a.

開く(ひらく)to open (door, business, eye, mouth, ...)

\section{3 彡 4 手毛加}

彡 is hair; Kangxi radical 59.

手(て)hand.

手がける(てがける)(v1) to make; to do; to produce; to work on.

手洗い(てあらい)restroom; lavatory; toilet; a place for washing hands

毛(モウ、け)hair.

加(カ)add.

\subsection{(3 彡) 7 形 8 参}

形(ケイ、ギョウ、かたち)shape; form.

参(サン)participate.

参加者(サンカシャ)participant; entrant.

参考(サンコウ)reference; consultation.

\section{3 习 6 羽 7 卵 9 飛 10 弱 11 習}

习 is simplified from 習.

羽(ウ、は、はね)feather

卵(ラン、たまご)egg.

飛(ヒ)fly; jump.

飛ぶ(とぶ)to jump.

弱(ジャク)weak.

弱い(よわい)weak.

練習(レンシュウ)training; practice.

習う(ならう)(vt) to learn.

Note that the bottom component of 習 is 自, not 日.

\section{3 土 9 垢 12 塔}

垢(あか)dirt; filth; grime

塔(トウ)tower.
管制塔(カンセイトウ)control tower.

\section{3 夕 8 夜}

夕(ゆう)evening

夜(ヤ、よる)night.

\section{3 彳 9 律後 15 衝}

律(リツ、リチ)law.
黄金律(オウゴンリツ)Golden Rule.

後(ゴ、コウ、のち、あと)after.
後ろ(うしろ)back; behind; rear.
食後(ショクゴ)after a meal; postprandial.
今後(コンゴ)from now on; hereafter.

衝(ショウ)collide.

\section{3 士女 4 氏 5 仕民}

士(さむらい)gentleman; samurai.

士(シ)(suffix)
a qualified person;
a person with a qualified profession.

学士(ガクシ)university graduate (who has obtained a degree).

工学士(コウガクシ)Bachelor of Engineering.

女(おんな)woman

氏 depicts a man bowing to the left.

氏(シ)(suffix,honorific) Mr.; Mrs.. family. clan.

氏(うじ)family name; birth; lineage.

仕(シ)serve.

仕事(シごと)(n) work; job; business; occupation; employment.

仕方(シかた)way; method; manner.

仕方ない。It can't be helped. There's no other way.

仕える(つかえる)(v1,vi) to serve; to work; to attend.

民(ミン、たみ)people

\subsection{8 妾 11 接}

妾 concubine.

接(セツ)touch; contact.

接吻(セップン)kiss.

接ぐ(つぐ)to join (two things into one); to piece together; to graft.

接する(セッする)to come in contact with; to touch.

\subsection{11 婚}

婚(コン)marriage.

新婚(シンコン)newlywed.

\section{3 上下 4 出}

上(うえ)up

下(した)down

出 depicts something coming out of an open box.
出る(でる)(v1,vi) to go out; to exit; to leave.
出来る(できる)(v1,vi) to be able to do.
出来上がる(できあがる)(vi) to be finished; to be completed; to be ready

\section{3 干 4 午牛 6 汗}

干(カン、ほ)dry.
干す(ほす)(vt) to dry (to make something dry).

午(ゴ、うま)noon.

牛(ギュウ、うし)
cow; bull; ox; buffalo.
beef.

汗(あせ)(n) sweat.

汗をかく(exp,v5k) to sweat.

汗を流す(exp,v5s) to work hard; to sweat.

\section{3 土 5 圧}

土(ド、つち)
soil; earth; ground; dirt; ground (as opposed to the heavens).

圧(アツ)pressure.

気圧(キアツ)air pressure.

指圧(シアツ)finger pressure massage.

\section{3 山小 5 石 6 虫 11 鳥}

山(サン、やま)mount; mountain.

小 depicts three sand granules.

小(ショウ)small.

小さい(ちいさい)small.

石(セキ、コク、いし)stone.

虫(むし)insect; bug; cricket; moth

鳥(チョウ、とり)bird; chicken; chicken meat.

\subsection{4 少}

少(ショウ、すく、すこ)a few; a little.

少ない(すくない)(adj-i) few; little; scarce; limited; insufficient.

少し(すこし)(adv,n) small quantity; few; a little.

\subsection{8 岩 9 砂}

岩(ガン、いわ)rock.

砂(サ、シャ、すな)sand.

\subsection{8 𧈧 11 強}

強(キョウ、ゴウ)strong.

強い(つよい)(adj-i) strong.

\subsection{9 風 11 嵐}

風(フウ、かぜ)wind; -like.

嵐(あらし)storm.

\subsection{10 島 14 鳴}

島(トウ、しま)island.

鳴(メイ)chirp.

鳴く(なく)to chirp; to sing (bird).

\section{3 工 6 江 9 紅}

工(コウ、ク)craft.

工作(コウサク)work; construction; handicraft.

工学(コウガク)engineering.

大工(ダイク)carpenter.

木工(モッコウ)carpenter.

工(たくみ)(name) Takumi.

江(え)inlet; bay

紅(くれない)deep red; crimson

\section{3 亡才幺纟}

亡(ボウ)death; destruction; perishment; the deceased.

亡くす(なくす)(vt) to lose something (because he/she/it dies).

マンション\ruby{火}{カ}\ruby{災}{サイ}、%
4\ruby{歳}{サイ}\ruby{女}{ジョ}\ruby{児}{ジ}\ruby{死}{シ}\ruby{亡}{ボウ}%
 きょうだい2人\ruby{重}{ジュウ}\ruby{体}{タイ}
Fire in a large apartment: 4-year girl dead, 2 siblings in critical state.

才(サイ)ability; talent.
In China 3 strokes, in Japan 4 strokes.

天才(テンサイ)genius; prodigy.

幺 depicts a tiny/small/short thread.

纟 is the simplified form of 糹
(left form of 6 糸 (thread)).

\section{3 弓 4 引}

弓(キュウ、ゆみ)bow (archery, violin).

引(イン、ひ)pull.
引用(インヨウ)quotation; citation; reference.
引く(ひく)(vi,vt) to pull.

\section{3 寸}

% https://en.wiktionary.org/wiki/%E5%AF%B8#Japanese
寸depicts a position on the forearm
where the pulse can be palpated by compressing the radial artery.
寸(スン)an ancient unit of length, approximately 3 cm.

一寸(ちょっと)just a minute; short time; just a little.

一寸待って下さい。Please wait a moment.

一寸!Hey!

\subsection{5 付 6 守 7 対 11 得}

付ける(つける)(v1,vt) to attach, join, stick, glue, fasten.

口付け(くちづけ)(n) kiss.

口付ける(くちづける)(v1) to kiss.

日付(ひづけ)date.

守(シュ)protect.

守る(まもる)to protect.

対(タイ)versus...

対する(タイする)to face each other.

得る(える)(v1,vt) to get; to acquire; to obtain; to earn; to win; to gain; to secure; to attain.
納得(ナットク)understanding, agreement.
納得するto consent, agree; to understand the reason for something.

\subsection{6 寺 9 待持 10 時特}

寺(ジ、てら)Buddhist temple.

待つ(まつ)(vt,vi) to wait; to wait for; to await.

持つ(もつ)to hold; to carry; to possess

時(ジ、とき)time.

時代(ジダイ)era.

三国時代(サンゴクジダイ)The Three Kingdoms period.

戦国時代(センゴクジダイ)The Warring States period.

特(トク)special.

特に(トクに)particularly; especially.

特技(トクギ)special skill.

特待(トクタイ)special treatment.

特待生(トクタイセイ)scholarship student.

\section{3 丸 4 円 6 色}

丸(ガン、まる)circle

円(エン)yen (Japanese currency).
円(まる)circle.

色(ショク、いろ)color.

色々(いろいろ)(adj-na) various.

\section{3 川 4 卬 5 母 5 生 7 貝 9 頁}

川(かわ)river

卬 is an uncommon kanji depicting a man standing and a man kneeling.

母 depicts a pair of breasts.

母(ボ、はは、かあ)mother.

母(はは)(humble) mother.

お母さん(おかあさん)(honorific) mother.

生 depicts a sprout, something sprouting from the ground.

生(セイ)nature; sex; gender.

学生(ガクセイ)student.

生まれる(うまれる)(v1,vi) to be born.

貝 depicts a cowry (a kind of seashell used as money in ancient China).

貝(かい)shell; shellfish.

頁(おおがい)originally depicts a head,
but comes to mean ``leaf'' (an organ in plants).

\subsection{(3 川) 10 流}

流す(ながす)to flow (liquid)

\subsection{(3 川 9 頁) 12 順}

順(ジュン)obey.

\subsection{(5 母) 7 毎 9 海}

毎(マイ、ごと)every.

毎日(マイニチ)everyday.

毎月(マイゲツ、マイつき)every month.

毎時(マイジ)every hour.

毎回(マイカイ)every time (every time it happens); every occurrence.

毎年(マイネン、マイとし)every year.

海(カイ、うみ)sea; beach.
The original character has 10 strokes.
Shinjitai replaces the two dots in the middle
with one vertical stroke.

\subsection{(5 生) 7 麦 8 性}

麦(バク、むぎ)wheat.

性(セイ)nature; sex; gender.

男性(だんせい)male.

女性(じょせい)female.

\subsection{(5 生) 11 産}

産(サン)give birth.

不動産(フドウサン)real estate.

\subsection{(4 卬) 7 迎 8 昂 10 留 12 貿}

迎(ゲイ)welcome.

昂る(たかぶる)(vi) to get excited; to get worked up

留(リュウ)detain.

留学(リュウガク)studying abroad (usually at university level).

貿(ボウ)trade.

貿易(ボウエキ)trade (foreign)

\subsection{(7 貝 5 𠀐) 12 貴 15 遺}

貴(キ)precious; honorable.

貴方(あなた)you (referring to someone of equal or lower status).

遺(イ、ユイ)bequeath; leave behind.

遺児(イジ)orphan.

遺物(イブツ)relic; memento.

遺品(イヒン)articles left by the deceased.

遺書(イショ)will; testament.

遺体(イタイ)corpse; remains.

遺言(ユイゴン)will; testament.

遺す(のこす)to bequeath; to leave behind; to save; to reserve.

\subsection{(7 貝 2 刀) 9 則 11 側}

則(のり)law; rule; regulation.

法則(ホウソク)law; rule.

側(がわ、かわ)side

側(そば)vicinity; near; beside

\subsection{(7 貝) 10 唄員 13 損}

唄(バイ)songs with samisen.

員(イン)(suffix) member.

工員(コウイン)factory worker.

会社員(カイシャイン)company employee.

損(ソン)loss.

\subsection{(7 貝) 9 負 12 買 15 賞賣}

負(フ)negative; minus.

負う(おう)to bear; to carry on one's back.

負かす(まかす)(vt) to defeat.

負ける(まける)(v1,vi)
to lose; to be defeated.
to succumb; to give in; to surrender; to yield.
to be inferior to.

買(バイ)buy.

買う(かう)to buy; to purchase

賞(ショウ)prize; award

受賞(ジュショウ)winning (a prize).

賣 is the old form of 7 売.

\subsection{(7 貝 2 匕) 11 頃}

頃(ころ)time of year; season.

\subsection{8 毒}

毒(ドク)poison.

毒ガス(ドクガス)poison gas.

\subsection{8 青 11 清情 12 晴}

青(セイ、あお)(n) blue; green.

青い(あおい)(adj-i) blue; green.

青ざめる(あおざめる)(v1,vi) to become pale.

清(セイ)clear.

清い(きよい)clear; pure; noble.

情(ジョウ)feelings; emotion; passion; sympathy.

晴(セイ、はれ)clear.

\subsection{11 責 16 積 17 績}

責(セキ)blame.

積(セキ)accumulate.

積雪(セキセツ)fallen snow;
snow cover;
snow that has fallen onto the ground.

面積(メンセキ)area (mathematics).

績(セキ)exploits.

\section{3 也 5 他 6 地池}

也 means ``also''.

他(タ、ほか)other.

他人(タニン)another person; other people; stranger.

地(チ、ジ)land (that is being used for an activity).

空き地(あきチ)vacant land.

耕地(コウチ)arable land.

団地(ダンチ)multi-unit apartments.

池(チ、いけ)pond

\section{3 女 6 好 7 姉 8 妹委妻 10 娘}

好き(すき)like; love; prefer

姉(シ、あね)older sister.
姉(あね)(humble) older sister.
お姉さん(おねえさん)(honorific) older sister.

妹(マイ)younger sister.
妹(いもうと)(humble) younger sister.
妹さん(いもうとさん)(honorific) younger sister.

委員(イイン)committee member.
委ねる(ゆだねる)(v1,vt) to entrust to.

夫妻(フサイ)married couple; husband and wife

娘(むすめ)daughter

\section{3 口 4 𧘇 6 衣}

口(くち)mouth.

𧘇 is an unknown radical.

衣 depicts a cloth.

衣(ころも)
clothes;
garment. gown;
robe. coating (glaze; batter; icing).

\subsection{5 古 6 吉𠮷 7 告}

古 consists of 十(ten) and 口(mouth, generation).

古(コ、ふる)old.

古い(ふるい)old (not of person); ancient; obsolete.

吉(キチ、キツ)good luck.

𠮷 is an alternative form of 吉.

告(コク)tell.

告げる(つげる)to inform; to tell.

告白(コクハク)confess (usually of love).

\subsection{(5 古) 8 固 12 結}

固(コ)hard.

結(ケツ、むす、ゆ)bind; tie (not the wearable).

結う(ゆう)to do up hair.

結ぶ(むすぶ)to bind; to tie; to link.

結合(ケツゴウ)combination; union; binding; catenation; coupling; joining; bond.

結婚(ケッコン)marriage.

結婚式(ケッコンシキ)marriage ceremony.

\subsection{(5 古) 8 周 11 週 14 滴 15 調}

周(シュウ)circumference.

週(シュウ)week.

滴(しずく)a drop of water; a drip.

滴る(したたる)to drip (fall one drop at a time).

調(チョウ)investigate.

\subsection{6 舌 7 乱 9 活 10 舐 13 話}

舌(ゼツ、した)tongue.

乱(ラン)riot.

活(カツ)active.

活きる(いきる)

舐める(なめる)(v1,vt) to lick

話(ワ、はなし)talk.

会話(カイワ)conversation.

話す(はなす)to talk.

\subsection{(4 𧘇) 8 表}

表(ヒョウ)express.

表示(ヒョウジ)manifestation; demonstration. display. representation.

\subsection{(6 衣) 16 壊}

壊(カイ)demolition.

壊す(こわす)(vt) to demolish.

\subsection{(4 𧘇 6 𠮷) 10 袁 13 猿遠園}

袁(エン、オン)robe; long kimono.

猿(エン、さる)monkey.

遠(エン)far.

遠い(とおい)(adj-i) far (spatial distance).

園(エン、その)park; garden; yard; farm.

動物園(ドウブツエン)zoo; animal park; zoological garden.

幼稚園(ヨウチエン)kindergarten.

\subsection{(7 告) 10 造 14 酷}

造る(つくる)to make; to construct.

木造(モクゾウ)wooden; made of wood.

酷(コク)cruel.

\subsection{7 言 14 罰}

言(ゲン、ゴン)say.

言(こと)saying.

言う(いう)to say.

文言(ブンゲン)wording.

言葉(ことば)word; dialect.

罰(バツ)punishment; penalty.

罰金(バッキン)fine; monetary penalty.

罰する(ばっする)to punish; to penalize.

\subsection{10 訓}

訓(クン)instruction.

教訓(キョウクン)lesson; precept; moral instruction.

\subsection{6 名吸 8 味 11 唾問}

名手(メイシュ)expert.

名前(なまえ)
name; full name.
given name; first name.

有名(ユウメイ)(adj-na) fame; famous.

吸(キュウ、す)suck.

吸う(すう)to suck with mouth

味(あじ)flavor; taste.

美味しい(おいしい)delicious.

唾(つば)spit

問(モン)(suffix, counter) counter for questions.

問う(とう)to ask (a question).

質問(シツモン)question; inquiry; enquiry.

\section{(3 walking radical) 6 巡}

巡る(めぐる)to go around.
お巡り(おまわり)policeman.

\section{7 束 10 速 16 頼}

束(ソク、たば)bundle.

速(ソク)fast.

早速(サッソク)(adv) immediately.

急速(キュウソク)rapid (progress).

速い(はやい)fast.

頼(ライ)trust.

頼む(たのむ)to request. to entrust to. to rely on.

頼 was simplified from 賴.

\section{3 子 6 字 8 学乳}

子 depicts a child with arms visible.

子(シ、こ) child.

男子(ダンシ)young man, boy.

女子(ジョシ)young woman, girl.

太子(タイシ)crown prince.

字(ジ)character, letter (of an alphabet, not the letter that is a document)

赤字(あかジ)deficit; red letter; red Han-character.

漢字(カンジ)Han character.

学(ガク)learning, scholarship, erudition, knowledge.

中学(チュウガク)middle school; junior high school.

大学(ダイガク)university.

学界(ガッカイ)academic world.

科学(カガク)science.

乳(ちち)breasts.
授乳(ジュニュウ)breast-feeding.

\section{3 亍 6 行 11 術 12 街}

行(コウ、ギョウ)go.

行く(いく)(vi) to go.

行く(ゆく)(vi) to go.

行う(おこなう)(vt) to do.

術(ジュツ)art.

術者(ジュツシャ)practitioner (of medicine, art, etc.).

街(ガイ、まち)street.

\section{(3 roof radical) 6 安宅 8 空 11 窓}

安全(アンゼン)safety; security.
安心(アンシン)relief; peace of mind.

安い(やすい)cheap; inexpensive; secure.

宅(タク)home; house; residence.
住宅(ジュウタク)residence; housing; residential building.
自宅(ジタク)one's home.
自宅火災(ジタクカサイ)house fire; home fire (disaster).

宅(いえ)home.

空(そら)sky.
空港(クウコウ)airport.
空く(すく)(vi) to become less crowded; to get empty.
空く(あく)(vi) to be open; to be empty.

窓(ソウ、まど)window.

\section{3 女 13 嫌}

嫌い(きらい)hate.

嫌悪感(ケンオカン)aversion.

\section{3 𠫔}

𠫔 (U+20AD4) depicts forearm.

\subsection{4 云 6 会伝 7 芸}

云 depicts clouds.

云う(いう)to say; to call; to name.

会(カイ、エ、あ)meet.

会う(あう)to meet (face to face).

会社(カイシャ)company; corporation.

会社員(カイシャイン)company employee.

年会(ネンカイ)yearly meeting; annual convention.

社会(シャカイ)society.

会見(カイケン)interview.

伝(デン)transmit.

自伝(ジデン)autobiography.

手伝う(てつだう)to help; to assist; to take part in.

伝言(デンゴン)verbal message.

伝記(デンキ)life story.

伝道(デンドウ)proselytizing; evangelism; missionary work.

伝える(つたえる)to convey; to report; to transmit; to communicate.

芸(ゲイ)art.

芸能界(ゲイノウカイ)world of show business.

芸能人(ゲイノウジン)performer; a talent in show business.

\subsection{6 至 9 室}

至(シ)climax.

至言(シゲン)wise saying.

室(シツ、むろ)room.
