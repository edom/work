\chapter{Kanji 3}

\section{2 冂 3 大 6 米 12 奥}

冂 depicts an inverted box.

奥(おく)interior

奥山(おくやま)remote mountain

\section{3 工 5 业 13 業 14 菐}

13 業 contains 业未.
14 菐 contains 业土人.

業(ギョウ)industry

工業(コウギョウ)manufacturing industry

工作(コウサク)work; construction; handicraft

\section{3 女 6 好 8 妻 13 嫌}

好き(すき)like; love; prefer

夫妻(フサイ)married couple; husband and wife

嫌い(きらい)hate

\section{4 水 1: 2 冫 3 氵 4 水 5 氷 6 江汗 7 冷沈 8 泳 9 洪海洗 10 酒}

Water-related things.

冫ice radical

氵water radical

水(スイ、みず)water.
水車(スイシャ)water wheel.

氷(こおり)ice

江(え)inlet; bay

汗(あせ)(n) sweat.
汗をかく(exp,v5k) to sweat.
汗を流す(exp,v5s) to work hard; to sweat.

冷たい(つめたい)(adj-i) cold (to the touch)

沈む(しずむ)to sink (descend into liquid)

泳ぐ(およぐ)to swim

洪水(コウズイ)(n) flood (of liquid)

大水(おおみず)(n) flood (of liquid)

海(うみ)sea; beach.
The original character has 10 strokes.
Shinjitai replaces the two dots in the middle
with one vertical stroke.

洗う(あらう)(vt) to wash

酒(さけ)sake (a Japanese liquor)

\section{4 水 2: 8 沼波 9 津 10 凍流涙浮 11 清}

沼(ぬま)swamp; bog

波(なみ)(n) wave (of liquid)

津(つ)seaport; harbor.
津波(つなみ)(n) tsunami; tidal wave.

凍る(こおる)to freeze.
But the kanji for for ice is 氷(こおり).

流す(ながす)to flow (liquid)

涙(なみだ)tear (eyewater)

浮かぶ(うかぶ)to float (be supported by liquid)

清い(きよい)clear (the character consists of water and blue 青)

\section{4 殳 7 投没 10 殺 11 設}

殳 depicts a hand holding a tool or a weapon.

投げる(なげる)to throw

没(ボツ)death

殺す(ころす)to kill

殺害(サツガイ)murder

殺人(サツジン)murder

設ける(もうける)to establish

\section{4 心 7 忘 9 急 9 思 10 恋 11 悪 12 悲}

心 is involved in a lot of feeling-related characters.

心配(シンパイ)(adj-na,n,vs) worry, concern, anxiety

心配(シンパイ)(n,vs) care, help

7 忘 forget

忘れる(わすれる)(v1) to forget

忘年会(ボウネンカイ)year-end party (lit. forget-year meeting, a meeting to forget the year)

意味合い(イミあい)implication; nuance

9 急(キュウ) urgent, sudden, abrupt

急ぐ(いそぐ)to hurry

9 思 think

思is田(rice field) and 心(heart).

思う(おもう)to think

10 恋(レン、こい) romance, love

恋(こい)(n) love, tender passion

恋人(こいびと)lover; sweetheart

恋文(こいぶみ)love letter

11 悪(アク) bad, evil, wicked

悪(アク)evil, wickedness

悪人(アクニン)bad person, villain

悪い(わるい)bad, poor; evil; unprofitable; at fault

12 悲(ヒ) sad

悲しい(かなしい)sad

悲恋(ヒレン)disappointed love

\section{4 辶 8 近 9 首 13 道遠違}

辶 is the combining form of 辵 meaning ``walking''.
Chinese dictionaries say this has 3 strokes.
Japanese dictionaries say this has 4 strokes.

首(くび)neck

首相(シュショウ)prime minister

道(みち)street; road

近い(ちかい)near

遠い(とおい)far

違う(ちがう)

\section{5 台 8 治 14 臺}

台(うてな)tower; stand; pedestal.

台 is simplified form of 14 臺.

治める(おさめる)(v1,vt)
to dominate; to rule; to govern; to manage.
to tranquilize; to pacify; to subdue.
to suppress.

治る(なおる)(vi) to heal

治(おさむ)(name) Osamu

仙台(センダイ)(city name) Sendai

\section{5 写}

写生(シャセイ)sketch

写す(うつす)(vt) to photograph

\section{6 当}

当たり(あたり)

本当(ホントウ)

\section{6 交 10 校}

交わる(まじわる)cross; intersect; join; meet (?)

国交(コッコウ)diplomatic relations

学校(ガッコウ)school

\section{6 艸 11 菌}

菌(キン)fungus; germ; bacterium

\section{6 糸 15 線}

打線(ダセン)baseball lineup

\section{7 貝 10 員 11 側 12 買 15 賞賣(売)}

員(イン)(suffix) member.
工員(コウイン)factory worker.
会社員(カイシャイン)company employee.

側(がわ、かわ)side

側(そば)vicinity; near; beside

買う(かう)to buy; to purchase

賞(ショウ)prize; award

賣る(うる)to sell.
This kanji has been simplified to 7 売る.

\section{7 言 9 計訂 10 記 13 話 14 読語 15 誰}

言葉(ことば)word; dialect

計(ケイ)plan.
計画(ケイカク)plan; project; schedule; scheme; program; programme.

訂正(テイセイ)correction; revision; amendment

記(キ) record

記す(しるす)to record, to write down

記事(キジ)article (writing).
選り抜き記事(よりぬきキジ)selected articles.
新しい記事(あたらしいキジ)new articles.

記録(キロク)record

話 talk

話す(はなす)to talk

読 read

読む(よむ)to read

語(ゴ) language

…語(…ゴ)... language

日本語(ニホンゴ)Japanese language

英語(エイゴ)English language

15 誰(だれ)who

\section{7 図}

図(ズ)map; drawing; picture; plan; illustration; diagram; figure; chart

\section{8 事 10 書}

仕事(シごと)(n) work; job; business; occupation; employment.
火事(カジ)fire (as a disaster).

書 depicts a hand holding a pen writing on paper.
書く(かく)to write.
書(ショ)writing.

\section{9 前}

9 前(まえ)before (time), in front of

午前(ゴゼン)morning; before noon; a.m. (ante meridien)

\section{9 変}

変(ヘン)strange.
変わる(かわる)to change; to transform.
変身(ヘンシン)metamorphosis; transformation.
大変(タイヘン)(adv,adj-na,n)very.

\section{10 真}

真(シン)truth; reality

\section{10 馬 14 駅}

馬(うま)horse

駅(エキ)train station. 駅is新字体of驛.

\section{11 教}

教会(キョウカイ)church

\section{12 陽}

陽(ヨウ)the yang in yin and yang

太陽(タイヨウ)sun

\section{12 筋}

筋肉(キンニク)muscle

\section{12 最}

最(サイ)most

最も(もっとも)most.
日本の最も高い山(ニホンのもっともたかいやま)Japan's highest mountain.
世界で最も太い人(セカイでもっともふといひと)The fattest person in the world.

最小(サイショウ)smallest

最大(サイダイ)biggest

最初(サイショ)first

最後(サイゴ)last

最新(サイシン)newest

最高(サイコウ)best, highest, tallest

\section{12 無}

無(ム) no, -less, without

無駄(ムダ)uselessness

無用(ムヨウ)uselessness

無敵(ムテキ)invincible, unrivaled (lit. no-enemy)

無茶(ムチャ)absurd, unreasonable (lit. no-tea)

無人(ムジン)unmanned (lit. no-human)

無言(ムゴン)silence (lit. no-say)

\section{13 戦}

戦(いくさ)war

内戦(ナイセン)civil war

世界大戦(セカイタイセン)World War

\section{8 昂}

昂る(たかぶる)(vi) to get excited; to get worked up

\section{8 金 13 鉄 14 銅銀銃 16 鋼}

金has a lot to do with metals.
金(キン、かね)gold; money.

金属(キンゾク)metal.
重金属(ジュウキンゾク)heavy metal.
These are also the chemistry terms.

金色(キンいろ、コンジキ)golden (color)

鉄(テツ、くろがね)iron (lit. 黒金 black metal).
鉄人(てつジン)iron man; strong man.

鉄道(テツドウ)railroad; railway

銅(ドウ、あかがね)copper (lit. 赤金 red metal)

銀(ギン、しろがね)silver (lit. 白金 white metal)

銃(ジュウ)gun; small firearms

鋼(コウ、はがね)steel

青銅(セイドウ)bronze

鋼鉄(コウテツ)steel

\section{10 神}

神(かみ)god; spirit; thunder

\section{13 楽}

音楽(オンガク)music

楽しい(たのしい)happy
