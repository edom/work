\chapter{Kanji 3}

\section{Components: 14 菐}

菐 is the 14-stroke ``thicket'' radical.

\section{6 危}

危ない(あぶない)dangerous

\section{7 改花努}

改める(あらためる)(v1,vt) to revise.

花(はな)flower

努める(つとめる)(v1,vt) to endeavor; to try; to strive.
努力(ドリョク)great effort; exertion; endeavor

\section{8 店河例}

店(テン)(n) store; shop.
ラメン店ramen shop (ramen is a kind of Japanese noodle).

河(かわ)river; stream.
河川(カセン)rivers.
大河(タイガ)large river.

例えば(たとえば)for example.
例える(たとえる)(v1,vt)
to compare; to liken; to illustrate.
用例(ヨウレイ)example; illustration.

\section{9 品草茶垢信計訂}

品(しな)article; item; thing; goods; stock.
品(ヒン)quality.
品物(しなもの)goods.
作品(サクヒン)work (book; film; composition; etc.). opus.
日用品(ニチヨウヒン)daily necessities.
一品(イッピン)one item; one article; one course of meal.
上品(ジョウヒン)elegant; refined; polished.
品性(ヒンセイ)character (elegant attitude; elegant behavior).

草(くさ)grass

茶(チャ) tea

垢(あか)dirt; filth; grime

信(シン)faith; trust.
信じる(シンじる)(v1,vt) to believe; to have faith in.

計(ケイ)plan.
計画(ケイカク)plan; project; schedule; scheme; program; programme.

訂正(テイセイ)correction; revision; amendment

\section{10 帰討通時唇特}

帰る(かえる)(v5r,vi)
to return; to come home; to go home; to go back.

討(トウ)chastise.
討伐(トウバツ)subjugation; suppression.
討つ(うつ)to shoot at; to attack, defeat, destroy, avenge.

通(ツウ)pass through.
通る(とおる)to go by.

時(とき)time.
時代(ジダイ)era.
三国時代(サンゴクジダイ)The Three Kingdoms period.
戦国時代(センゴクジダイ)The Warring States period.

唇(くちびる)(n) lips.

特(トク)special.
特に(トクに)particularly; especially.
特技(トクギ)special skill.

\section{11 異野転接菌}

異(イ)uncommon; different; unusual.
異国(イコク)foreign country.
異性(イセイ)different sex; opposite sex.
異なる(ことなる)to differ; to vary; to disagree.

野(ヤ、の)field; plains; rustic.

転(テン)revolve; turn around; change.
反転(ハンテン)rolling over; turning around.
転ぶ(ころぶ)(vi) to fall down; to fall over.

接(セツ)touch; contact.
接ぐ(つぐ)to join (two things into one); to piece together; to graft.
接吻(セップン)kiss.
接する(セッする)to come in contact with; to touch.

菌(キン)fungus; germ; bacterium

\section{11 経済脳情啓祭教清}

経(キョウ)sutra; Buddhist scripture.
経つ(たつ)(vi) to pass; to lapse.
経る(へる)to pass; to elapse; to experience.

済ませる(すませる)(v1,vt) to finish; to end.
経済(ケイザイ)economy.

脳内(ノウナイ)intracranial; inside the brain

情(ジョウ)feelings; emotion; passion; sympathy.

啓(ケイ).
拝啓(ハイケイ):拝啓… Dear ...

祭り(まつり)feast; festival

教え(おしえ)teaching; doctrine.
教育(キョウイク)training; education.
教会(キョウカイ)church.

清い(きよい)clear; pure; noble

\section{12 堺堕貿普奥詞筋報}

堺(カイ、さかい)world.

堕(ダ)degeneration; degradation.
堕胎(ダタイ)abortion; feticide; babykilling.
堕落(ダラク)depravity; corruption; degradation.

貿(ボウ)trade.
貿易(ボウエキ)trade (foreign)

普(フ)universal; wide; general.
普通(フツウ)general; ordinary; usual.
普通の人間(フツウのニンゲン)ordinary human.

奥(おく)interior.
奥山(おくやま)remote mountain.

詞(シ).
名詞(メイシ)noun.

筋肉(キンニク)muscle.

報(ホウ)report; information; news.
報じる(ホウじる)(v1,vt) to report; to inform.
報いる(むくいる)(v1,vt) to reward; to recompense; to repay.

\section{13 違僧話詳園数禁準楽暖業}

相違(ソウイ)difference; discrepancy; variation.
違う(ちがう)(vi) to differ; to not match the correct answer.
間違う(まちがう)to make a mistake; to be incorrect; to be mistaken.

僧(ソウ)monk; priest.

話す(はなす)to talk.
会話(カイワ)conversation.

詳

園(エン)park; garden; yard; farm.
動物園(ドウブツエン)zoo; animal park; zoological garden.
幼稚園(ヨウチエン)kindergarten.

数(かず)number; amount.
数える(かぞえる)(v1,vt) to count; to enumerate.
算数(サンスウ)arithmetics.
数万(スウマン)tens of thousands.

禁じる(キンじる)(v1,vt) to prohibit.

準(シュン).
水準(スイジュン)water level. level; standard.

楽しい(たのしい)happy.
音楽(オンガク)music.

暖かい(あたたかい)(adj-i) warm; genial

業(ギョウ)industry.
工業(コウギョウ)manufacturing industry.

\section{14 際僕概}

算(サン)calculation.
算出(サンシュツ)calculation; computation.
加算(カサン)addition.
引き算(ひきザン)subtraction.
公算(コウサン)probability; likelihood.

際(サイ)occasion; circumstances.
際限(サイゲン)limits; bounds.
学祭(ガクサイ)interdisciplinary.
国際(コクサイ)international.

僕(ボク)I; me (male).

概要(ガイヨウ)outline; summary

\section{15 膝質談線監編稿誰}

膝(ひざ)knee; lap

質(シツ)(suffix) substance; quality; matter.
質問(シツモン)question.

談(ダン)discuss.
相談(ソウダン)consultation.
示談(ジダン)out-of-court settlement.
座談会(ザダンカイ)symposium; round-table discussion.

線(セン)line; stripe.
line (telephone line).
line (of a railroad).
打線(ダセン)baseball lineup.

監(カン)government official; rule; administer.
監禁(カンキン)confinement; bondage.

編む(あむ)(vt)
to knit; to plait; to braid.
to compile (an anthology); to edit.

稿(コウ)draft; copy; manuscript

誰(だれ)who

\section{16 親諦頭}

親(おや)parent.
両親(リョウシン)both parents.

諦める(あきらめる)(v1,vt)
to give up; to abandon.

頭(あたま)head

\section{18 顔曜類}

顔(かお)face.
顔面(ガンメン)face (of a person).

曜(ヨウ)(weekday name).
日曜日(ニチヨウビ)Sunday.

類(ルイ)kind; sort; type.
人類(ジンルイ)mankind.

\section{Written: 10 記 14 読語}

記(キ)record.
記す(しるす)to record, to write down.
記録(キロク)record.
記事(キジ)article (writing).
選り抜き記事(よりぬきキジ)selected articles.
新しい記事(あたらしいキジ)new articles.

読む(よむ)to read.

語(ゴ)language.
日本語(ニホンゴ)Japanese language.
英語(エイゴ)English language.

\section{Craft: 3 工}

工作(コウサク)work; construction; handicraft.

\section{Countries}

日本(ニホン)Japan

中国(チュウゴク)People's Republic of China

英国(エイコク)United Kingdom

米国(ベイコク)United States of America

\section{Personal data}

出身(シュッシン)person's origin (town, city, country, etc.)

出身地(シュッシンチ)birthplace

誕生日(タンジョウビ)birthday; birth date; day of birth.

身長(シンチョウ)height (of body)

体重(タイジュウ)body weight

血液型(ケツエキガタ)blood type.「A型」type a.

好きなもの(すきなもの)likes

嫌いなもの(きらいなもの)dislikes

\section{Ungrouped}

警察(ケイサツ)police

素晴らしい(すばらしい)

卒業(ソツギョウ)

擦り傷(すりきず)(n) scratch; graze; abrasion

今度(コンド)
now; this time; this occurrence.
next time; another time.

率(リツ)(suf) rate; ratio; proportion.
識字率(シキジリツ)literacy rate.

…階建て(…カイだて)(suf) ...-story building.
7階建て 7-story building.

13
了解(りょうかい)understanding; roger that.
見解(ケンカイ)opinion; point of view.
専門家見解(センモンカケンカイ)expert opinion.

16 操る(あやつる)(vt) to be fluent in (a language)

14 増える(ふえる)(v1,vi) to increase; to multiply

17 覧 perusal

15 選 elect; select

13 連携(レンケイ)collaboration; cooperation

喧嘩(ケンカ)fight; brawl

博打(バクチ)gambling

お絵描き(おエかき)oekaki; painting; drawing

ネタバレspoiler (of a movie, a story, etc.); something that spoils the end of a movie, a story, etc.

関係(カンケイ)relation; connection

肉体関係(ニクタイカンケイ)sexual relations

垢と一切関係ないIt has absolutely nothing to do with dirt.
