\chapter{Kanji 3}

\section{Knit: 15 編}

編む(あむ)(vt)
to knit; to plait; to braid.
to compile (an anthology); to edit.

\section{Dirt: 9 垢}

垢(あか)dirt; filth; grime

\section{Feast: 11 祭}

祭り(まつり)feast; festival

\section{Danger: 6 危}

危ない(あぶない)dangerous

\section{Countries}

日本(ニホン)Japan

中国(チュウゴク)People's Republic of China

英国(エイコク)United Kingdom

米国(ベイコク)United States of America

\section{Personal data}

出身(シュッシン)person's origin (town, city, country, etc.)

出身地(シュッシンチ)birthplace

誕生日(タンジョウビ)birthday; birth date; day of birth.

身長(シンチョウ)height (of body)

体重(タイジュウ)body weight

血液型(ケツエキガタ)blood type.「A型」type a.

好きなもの(すきなもの)likes

嫌いなもの(きらいなもの)dislikes

\section{Industry: 5 业 13 業 14 菐}

13 業 contains 业未.
14 菐 contains 业土人.

業(ギョウ)industry

工業(コウギョウ)manufacturing industry

工作(コウサク)work; construction; handicraft

\section{Evil: 4 凶 11 脳}

凶(キョウ)evil; villain; bad luck; disaster.
凶悪(キョウアク)(adj-na) atrocious; fiendish; brutal; villainous.

脳内(ノウナイ)intracranial; inside the brain

\section{Doctrine: 11 教}

教会(キョウカイ)church

\section{Happiness}

\subsection{Excitement: 8 昂}

昂る(たかぶる)(vi) to get excited; to get worked up

\subsection{Happiness: 13 楽}

音楽(オンガク)music

楽しい(たのしい)happy

\section{Kind: 18 類}

類(ルイ)kind; sort; type

人類(ジンルイ)mankind

\section{Figurative line: 15 線}

線(セン)line; stripe.
line (telephone line).
line (of a railroad).

打線(ダセン)baseball lineup

\section{Interior: 12 奥}

奥(おく)interior.
奥山(おくやま)remote mountain.

\section{Confusing characters}

\subsection{Spirit and cloth: 4 礻 5 衤}

\subsection{3 夂夊 4 攵 8 㑒}

In the Japanese language,
these characters become parts of other characters
instead of being used on their own.

夂 depicts two legs followed by something from behind.

夊 depicts a footprint.

攵 is a variant of 攴 depicting a branch and a hand.

㑒 is simplified from the 13-stroke 僉
meaning ``all, together, unanimous''.

\subsection{Samurai and earth: 3 士土}

士(samurai) has longer upper horizontal stroke.
土(earth) has shorter upper horizontal stroke.

\subsection{Hat, sun, moon, meat, inner}

冃(hat)

日(sun)

月(moon)

\subsection{石 (stone) and 右 (right)}

\subsection{人 (person) and 入 (enter)}

\subsection{王 (king) and 生 (sprout)}

\subsection{業菐美}

\section{Hand}

暖かい(あたたかい)(adj-i) warm; genial

\section{Economy}

経済(ケイザイ)economy

貿易(ボウエキ)trade (foreign)

\section{Outline: 14 概}

概要(ガイヨウ)outline; summary

\section{Effort: 7 努}

努める(つとめる)(v1,vt) to endeavor; to try; to strive.
努力(ドリョク)great effort; exertion; endeavor

\section{Ungrouped}

諦める(あきらめる)(v1,vt)to give up; to abandon

警察(ケイサツ)police

素晴らしい(すばらしい)

卒業(ソツギョウ)

議論(ギロン)

擦り傷(すりきず)(n) scratch; graze; abrasion

今度(コンド)
now; this time; this occurrence.
next time; another time.

拝啓(ハイケイ):拝啓… Dear ...

水準(スイジュン)water level. level; standard.

率(リツ)(suf) rate; ratio; proportion.
識字率(シキジリツ)literacy rate.

…階建て(…カイだて)(suf) ...-story building.
7階建て 7-story building.

了解(りょうかい)understanding; roger that.
見解(ケンカイ)opinion; point of view.
専門家見解(センモンカケンカイ)expert opinion.

名詞(メイシ)noun

操る(あやつる)(vt) to be fluent in (a language)

増える(ふえる)(v1,vi) to increase; to multiply

覧 perusal

報 report; news

選 elect; select

野 plains; rustic

転 revolve

連携(レンケイ)collaboration; cooperation

接吻(セップン)kiss

喧嘩(ケンカ)fight; brawl

博打(バクチ)gambling

間違う(まちがう)

国際(コクサイ)international

お絵描き(おエかき)oekaki; painting; drawing

ネタバレspoiler (of a movie, a story, etc.); something that spoils the end of a movie, a story, etc.

堕ちる(おちる)(v1,vi)to fall down; to drop (?)

関係(カンケイ)relation; connection

肉体関係(ニクタイカンケイ)sexual relations

垢と一切関係ないIt has absolutely nothing to do with dirt.

趣味(シュミ)hobby; taste, preference.

邪魔(ジャマ)intrusion
邪魔するto intrude
