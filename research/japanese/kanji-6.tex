\chapter{Kanji 6}

\section{6 在存有}

在(ザイ、あ)exist.

在る(ある)to exist.

自在(ジザイ)freely; at will.

有 depicts a hand holding a piece of 肉(meat).

有る(ある)to exist.

存(ソン)suppose.

存じる(ゾンじる)(v1,humble) to think, feel, consider, know.

存在(ソンザイ)existence; being.

共存(キョウゾン)coexistence.

存亡(ソンボウ)life-or-death; existence; destiny.

\subsection{9 陏 12 堕}

陏 is an uncommon kanji.

堕(ダ)degeneration; degradation.

堕胎(ダタイ)abortion; feticide; babykilling.

堕落(ダラク)depravity; corruption; degradation.

\section{6 共 8 供 9 洪}

共(とも)companion; follower; attendant; retinue.

供える(そなえる)(v1,vt) to offer; to sacrifice; to dedicate.

子供(こども)child.

洪水(コウズイ)(n) flood (of liquid)

大水(おおみず)(n) flood (of liquid)

\section{6 旨 9 指}

旨(むね)center; principle; meaning.

本旨(ホンシ)main object; principal object; true aim.

旨い(うまい)(adj-i) delicious.

指(ゆび)finger.

指す(さす)(vt) to point.

目指す(めざす)(vt) to aim at.

\section{6 曲 8 典}

曲(キョク)music.

作曲(サッキョク)musical composition.

典(テン)code.

\section{6 争当糸糹多竹}

争(ソウ)conflict.

争う(あらそう)
to dispute; to quarrel.
to compete; to contest; to contend.

当(トウ、あ)hit.
当てる(あてる)(v1,vt) to hit.
本当(ホントウ)truth; reality.
本当に(ホントウに)truly; really; seriously.

糸 combines 幺 and 小.
糸(いと)thread.

糹 is the left form of 糸.

多 depicts two pieces of meat.
多(ア、おお)many.
多(タ)(prefix) multi-.
多い(おおい)(adj-i) many; numerous.

竹(たけ)bamboo.
竹林(チクリン)bamboo thicket.

\section{6 死光}

死(シ)death.
死ぬ(しぬ)to die.
死亡(シボウ)death; mortality.
死去(シキョ)death.

光(コウ、ひかり)light (electromagnetic wave)

\section{6 艮 7 良 8 長 9 食 12 飲}

艮 depicts eye and spoon.

良(リョウ)good.

良い(いい)good.

良く(よく)well.

長(チョウ)
long (distance or time).
leader.
eldest.

長い(ながい)long (distance); long (time).

長女(チョウジョ)eldest daughter; first-born daughter.

市長(シチョウ)mayor (a government official).

身長(シンチョウ)height (of body).

最長(サイチョウ)longest, tallest.

社長(シャチョウ)company president.

食(ショク)eat.

食物(ショクもの)food.

食べ物(たべもの)food.

食べる(たべる)(v1) to eat.

食う(くう)(male, vulgar) to eat.

飲む(のむ)to drink (any liquid, not just liquor)

\section{6 交 10 校}

交(コウ、まじ)intersect.
交わる(まじわる)cross; intersect; join; meet.

国交(コッコウ)diplomatic relations

学校(ガッコウ)school

\section{6 式}

式(シキ)ceremony.

式(シキ)style.
公式(コウシキ)formal; official.
公式ブログ official blog.

式(シキ)numerical formula.

\section{6 次 7 吹}

次(つぎ)next (in sequence)

吹(スイ)blow.

吹く(ふく)to blow (wind, etc.).

吹雪(ふぶき)snow storm; blizzard.

\section{6 先 9 洗}

先(さき)before; ahead; previous; future.

先生(センセイ)teacher; master; doctor.

先日(センジツ)a few days ago; the other day.

洗う(あらう)(vt) to wash

\section{6 気 7 汽}

気(キ)spirit; mind; air; atmosphere

元気(ゲンキ)health(y); vigor; vitality; spirit.

天気(テンキ)weather.

気持ち(きもち)feeling.

汽(キ)steam.

汽車(キシャ)steam train.

\section{(6 糸) 7 系}

系(ケイ)lineage.

\section{(6 糸) 10 紙 11 細}

紙(シ、かみ)paper.

手紙(てがみ)letter (the document, not the alphabet).

細(サイ)thin.

細い(ほそい)(adj-i) thin; slender.

細る(ほそる)to become thin.

細かい(こまかい)(adj-i) small; trivial.

\section{(6 糸) 11 経}

経(ケイ、キョウ)manage.

経つ(たつ)(vi) to pass; to lapse.

経る(へる)to pass; to elapse; to experience.

\section{(6 糸) 11 率}

率(リツ)(suf) rate; ratio; proportion.

識字率(シキジリツ)literacy rate.

\section{6 西 10 配酒}

西(セイ、サイ、にし)west

配(ハイ)distribute.
配本(ハイホン)distribution of books.
配送センター(ハイソウセンター)distribution center.
配る(くばる)to distribute; to deliver.

酒(さけ)sake (a Japanese liquor)

\section{6 耳 14 聞}

耳(みみ)ear

聞(ブン、モン)hear.

聞く(きく)to hear.

\section{6 关 9 送 14 関}

送る(おくる)to send.
見送る(みおくる)to see off; to escort (for parting); to farewell.
先送る(さきおくる)to postpone.

関(カン、せき)barrier; gate

\section{6 羊 12 達}

羊(ヨウ、ひつじ)sheep.

山羊(やぎ)goat.

達(タツ)attain.

\section{6 各 9 客 10 格}

各(カク、おのおの)each.

客(キャク、カク)guest.

格(カク)personality; character.

性格(セイカク)
character; personality; disposition; nature.
