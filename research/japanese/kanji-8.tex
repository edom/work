\chapter{Kanji 8}

\section{8 录 9 泉}

录 means ``to carve wood'' or ``to record''.

泉(セン、いずみ)spring (source of water).

\subsection{(8 录) 16 録}

録(ロク)record.

登録(トウロク)registration; entry.

録音(ロクオン)recorded audio; audio recording.

\subsection{(9 泉) 15 線}

線(セン、すじ)line; stripe.
line (telephone line).
line (of a railroad).

打線(ダセン)baseball lineup.

前線(ゼンセン)front line; weather front.

\section{8 突 9 罙 11 深探}

突(トツ)stab.

罙 means deep.

深(シン)deep.

深い(ふかい)deep.

深める(ふかめる)(v1,vt) to deepen.

探 depicts a hand groping in a deep cave.

手探り(てさぐり)groping; fumbling.

探す(さがす)(vt)
to search for something lost.
to search for something desired.

探る(さぐる)to feel around for; to fumble for; to grope for.

\section{8 京 10 原 13 源}

京(キョウ、ケイ、みやこ)capital.

京(キョウ)imperial capital.

原(ゲン、はら)meadow; field; plain; prairie; tundra; moor; wilderness.

原(ゲン)original; source; raw; origin.
原文(ゲンブン)original text.
原油(ゲンユ)crude oil.
原料(ゲンリョウ)raw materials.

起源(キゲン)origin; beginning; rise.

\subsection{12 就}

就(シュウ)concerning.

\section{8 亟 12 極}

亟(キョク)fast; quick; sudden; urgently; immediately; extremely.

極(キョク、ゴク)poles.
\(10^{48}\) (since the 17th century).

積極的(セッキョクテキ)assertive; proactive.

\section{8 㑒 9 品 10 馬}

㑒 was simplified from the 13-stroke 僉
meaning ``all, together, unanimous''.

品(しな)article; item; thing; goods; stock.

品(ヒン)quality.

品物(しなもの)goods.

作品(サクヒン)work (book; film; composition; etc.). opus.

日用品(ニチヨウヒン)daily necessities.

一品(イッピン)one item; one article; one course of meal.

上品(ジョウヒン)elegant; refined; polished.

品性(ヒンセイ)character (elegant attitude; elegant behavior).

馬(バ、うま、ま)horse.

\subsection{12 検}

検(ケン)examine.

検索(ケンサク)search.

検索エンジンsearch engine.

検索結果(ケンサクケッカ)search results.

コンピューターを検索Search in computer.

Wikipedia内を検索Search in Wikipedia.

検出(ケンシュツ)(n,vs) detection.

検討(ケントウ)consideration.

\ruby{不}{フ}\ruby{正}{セイ}なソフトウェアを検出しました
The implied entity detected malicious software.

\subsection{14 駆駅}

駆(ク)drive.
This was simplified from 驅.

駆ける(かける)(v1,vi) to run (horse).

駅 is simplified from 23 驛.
駅(エキ)train station.

\subsection{16 操}

操る(あやつる)(vt) to be fluent in (a language)

体操(タイソウ)physical exercise; gymnastics; calisthenics.

\subsection{18 験}

験(ケン)verify.

実験(ジッケン)experiment.

治験(チケン)clinical trial.

受験(ジュケン)taking an examination (such as school and university entrance).

経験(ケイケン)experience.

\ruby{未}{み}経験\ruby{者}{しゃ}inexperienced person.

未経験者\ruby{歓}{かん}\ruby{迎}{げい}inexperienced people are welcome.

\section{8 隹}

隹(ふるとり)depicts a short-tailed bird or an old bird.

\subsection{13 催}

催(サイ)sponsor.

開催(カイサイ)holding a meeting; opening an exhibition.

主催(シュサイ)sponsorship.

\subsection{18 雚 13 勧 15 歓 18 観}

雚 depicts a stork or heron.

勧(カン)persuade.

歓(カン)delight.

歓迎(カンゲイ)welcome; reception.

観(カン)observe.

観光(カンコウ)sightseeing.

観光客(カンコウキャク)tourist.

\subsection{12 雇}

雇(コ)employ.

解雇(カイコ)dismissal; firing; layoff.

雇う(やとう)to employ. to hire; to charter.

\subsection{12 集 13 稚 14 雑}

集(シュウ)gather; meet; congregate; swarm; flock.

集める(あつめる)(v1,vt) to collect; to assemble; to gather.

稚(チ)immature; young.

幼稚(ヨウチ)infancy; childish; infantile.

\subsection{13 準}

準(シュン)level; standard.

水準(スイジュン)water level. level; standard.

雜(ザツ)miscellaneous.

\subsection{15 誰 18 曜}

誰(だれ)who

曜(ヨウ)(weekday name).

日曜日(ニチヨウビ)Sunday.

\subsection{18 難}

難(ナン)difficult.

海難(カイナン)shipwreck.

災難(サイナン)disaster.

多難(タナン)(adj-na,n) full of troubles.

難点(ナンテン)fault; weakness.

難しい(むずかしい)difficult.

有難い(ありがたい)(adj-i) grateful; thankful.

\section{8 直 9 県 10 真}

直(ジキ)soon.

県(ケン)prefecture (an administrative division).

直(チョク)direct; in person; frankness; honesty.

直す(なおす)to heal; to cure.
Chinese 直 has 7 strokes.
Japan adds the lower-left corner stroke.

直(ただち)

真(シン、ま)true.

真意(シンイ)real intention; true motive; true meaning.

真っ黒(まっくろ)pitch black.

真っ先(まっさき)the foremost; the beginning.

\section{8 金}

金has a lot to do with metals.

金(キン、かね)gold; money.

金属(キンゾク)metal.
重金属(ジュウキンゾク)heavy metal.
These are also the chemistry terms.

金色(キンいろ、コンジキ)golden (color)

\subsection{13 鉄 14 銅銀 16 鋼}

鉄(テツ、くろがね)iron (lit. 黒金 black metal).

鉄人(てつジン)iron man; strong man.

鉄道(テツドウ)railroad; railway

銅(ドウ、あかがね)copper (lit. 赤金 red metal)

銀(ギン、しろがね)silver (lit. 白金 white metal)

銀行(ギンコウ)bank.

鋼(コウ、はがね)steel

青銅(セイドウ)bronze

鋼鉄(コウテツ)steel

\subsection{14 銃 19 鏡}

銃(ジュウ)gun; small firearms

鏡(かがみ)mirror.

眼鏡(めがね)eyeglasses.

\section{8 実店例}

実(ジツ)truth; reality.
実(み)fruit.
実る(みのる)to bear fruit; to ripen.

店(テン)(n) store; shop.
ラメン店ramen shop (ramen is a kind of Japanese noodle).

店(みせ).

例えば(たとえば)for example.
例える(たとえる)(v1,vt)
to compare; to liken; to illustrate.
用例(ヨウレイ)example; illustration.

\section{8 和}

和(ワ)peace; Japan.
平和(ヘイワ)peace.
和む(なごむ)(vi) to be softened; to calm down.
和らげる(やわらげる)(v1,vt) to soften; to moderate; to relieve.

\section{8 画}

画 means picture.
画(カク)counter for kanji strokes.
字画(ジカク)number of strokes in a character.
画家(ガカ)painter.
企画(キカク)plan.

\section{8 非 10 俳 12 悲 13 罪}

非(ヒ)un-; non-; negative; mistake; wrong.

非常(ヒジョウ)extraordinary; unusual

非ず(あらず)(exp) no; never mind.

俳(ハイ)haiku.

俳優(ハイユウ)actor; actress.

悲恋(ヒレン)disappointed love

悲しい(かなしい)sad.

罪(ザイ)sin; guilt.

罪(つみ)sin; crime; fault.

犯罪(ハンザイ)crime.

七つの大罪(ななつのダイザイ)seven deadly sins.

\section{(8 雨) 11 雪 12 雲 13 雷電 16 曇}

雪(セツ、ゆき)snow

雲(ウン、くも)cloud

雷(かみなり)thunder.

電(デン)lightning.
電光(デンコウ)lightning.
電気(デンキ)electricity (lit. lightning spirit).
電話(デンワ)telephone (lit. lightning talk).
電車(デンシャ)electric train (lit. lightning carriage).
電気自動車(デンキジドウシャ)electric car.

曇(ドン)cloudy weather.

曇る(くもる)

\section{10 唇 15 震}

唇(くちびる)(n) lips.

震(シン)quake.

震う(ふるう)

震える(ふるえる)

\section{8 東 10 凍}

東(トン、ひがし、あずま)east

凍る(こおる)to freeze.
But the kanji for for ice is 氷(こおり).
