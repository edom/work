\chapter{Kanji 9}

\section{9 面}

面(おもて)mask.

面白い(おもしろい)interesting.

\section{9 音 13 暗意 17 闇}

音(オン、おと、ね)sound.

本音(ホンね)real intention; motive.

暗(アン)dark.

暗い(くらい)dark.

意(イ)feelings; thoughts.

意見(イケン)opinion.

意欲(イヨク)motivation; will.

意図(イト)(n,vs) intention; aim; design.

意味(イミ)meaning; significance.

意味合い(イミあい)implication; nuance

小生意気(こなまイキ)cheekiness; impudence.

闇(やみ)darkness.

(?) 闇で全ては黒く見える。Everything looks black in the dark.
(全て or 皆?)

\section{9 専臭}

専門家(センモンカ)expert; specialist

負(フ)negative; minus.
負う(おう)to bear; to carry on one's back.
負かす(まかす)(vt) to defeat.
負ける(まける)(v1,vi)
to lose; to be defeated.
to succumb; to give in; to surrender; to yield.
to be inferior to.

臭: In China 10 strokes, in Japan 9 strokes.
The lower character is 犬 in China and 大 in Japan.
臭い(くさい)(adj-i) stinking; malodorous; ill-smelling.

\section{7 余 9 叙 10 除}

余(ヨ)leave over.

叙(ジョ)confer; relate; narrate; describe.

自叙伝(ジジョデン)autobiography.

除(ジョ、ジ、のぞ)exclude.

除く(のぞく)to exclude.

削除(サクジョ)elimination; cancellation; deletion.

\section{9 胃政}

胃(イ)stomach.

政(セイ、まつりごと)rule; government.

\section{9 面 11 異}

異(イ)uncommon; different; unusual.
異国(イコク)foreign country.
異性(イセイ)different sex; opposite sex.
異なる(ことなる)to differ; to vary; to disagree.

\section{9 首 12 道}

首(シュ、くび)neck

道(ドウ、みち)street; road.
鉄道(テツドウ)railway.

\section{9 乗}

乗(ジョウ)ride.

乗る(のる)(v1) to get on a public transport vehicle; to aboard; to embark.
