\chapter{Kanji 1}

\section{1 一 2 二 3 三}

一(イチ、ひと)one

二(ニ、ふた)two

三(サン、み)three

\section{2 人 3 大 4 夫天失太犬 5 立}

人 depicts a person.
人(ジン、レン、ニン、ひと)person; people; human

大 depicts a person with outstretched arms.
大(タイ)big.
大きい(おおきい)big.

天(テン) sky; heaven

夫(フ、おっと)husband

失 depicts something falling from a hand.
失う(うしなう)(vt) to lose; to part with.
失明(シツメイ)loss of eyesight.
失血(シッケツ)loss of blood.

太(タイ) thick.
太い(ふとい)fat.

犬(いぬ)dog.
The kanji did not come
from the drawing of a person.

立 depicts a person standing on ground.
立つ(たつ)(vi) to stand

\section{2 入 4 内 6 肉}

入る(いる)(v5)
to enter; to go into; to get into.
Do not confuse this with 人(person).

内(うち)inside

肉 (ribs of an animal's torso)

肉(ニク)meat; flesh; body (as opposed to spirit)

\section{2 又刀力 3 土士}

又 depicts the right hand.
又(また)again; also.

刀(かたな)katana (Japanese sword).
刀(トウ)sword.

力(リョク、ちから)strength

土(ド、つち) earth; soil

士(さむらい)samurai.
士(シ)(suffix)
a qualified person;
a person with a qualified profession.
学士(ガクシ)university graduate (who has obtained a degree).
工学士(コウガクシ)Bachelor of Engineering.

Do not confuse 士(samurai) and 土(earth).
士(samurai) has longer upper horizontal stroke.
土(earth) has shorter upper horizontal stroke.

\section{2 了 3 子 6 字 8 学}

了depicts a wrapped baby with only head visible.

子is了 with arms visible.
3 子(シ、こ) child.

6 字(ジ)character, letter (of an alphabet, not the letter that is a document)

8 学(ガク)learning, scholarship, erudition, knowledge.
中学(チュウガク)middle school; junior high school.
大学(ダイガク)university.
学界(ガッカイ)academic world.

\section{3 小 4 少}

3 小(ショウ)small

(three sand granules)

小さい(ちいさい)small

4 少 few

少し(すこし)(adv, n) small quantity; few; a little

\section{3 口工山川上下女寸 4 中父止心手}

口(くち) mouth

女(おんな)woman

川(かわ)river

山(サン、やま)mount, mountain

上(うえ)up

下(した)down

% https://en.wiktionary.org/wiki/%E5%AF%B8#Japanese
寸depicts a position on the forearm
where the pulse can be palpated by compressing the radial artery.
寸(スン)an ancient unit of length, approximately 3 cm.

工(コウ、ク)craft.
工学(コウガク)engineering.
大工(ダイク)carpenter.
木工(モッコウ)carpenter.
工(たくみ)(name) Takumi.

心(シン、こころ)heart

父(ちち)the inner group's father.
お父さん(おとうさん)(polite) the outer group's father.

手(て)hand.
手がける(てがける)(v1) to make; to do; to produce; to work on.

中(チュウ、なか)middle.
中学(チュウガク)middle school; junior high school.
…の中(…のなか)middle of something.
田中(たなか)(name) Tanaka.
中山(なかやま)(name) Nakayama.
中川(なかがわ)(name) Nakagawa.

止depicts a footprint.

止める(とめる)(v1) to stop moving (walking, etc.); to park (a car)

\section{3 幺 6 糸}

3 幺 depicts a tiny/small/short thread.

6 糸(いと)thread.
糸 combines 幺 and 小.

\section{4 日月 5 白 8 明}

日(ひ)sun.
日々(ひび)daily; days; old days.

月(つき)moon.

白(ハク)white.
白い(しろい)white.

明(メイ)bright.
明るい(あかるい)bright.

\section{4 木 5 本未 7 来 8 林東 12 森}

4 木(き)tree

8 林(はやし)

12 森(もり)

5 本(ホン)book, counter for long cylindrical objects

一本(イッポン)one long cylindrical object, one point

日本(ニホン)Japan

5 未(ミ) not yet

未来(ミライ)future (lit. not yet come)

7 来(ライ)come; next

来depicts fruits hanging on a tree.
It means that the time to harvest has come.

来る(くる)to come. This is an irregular verb. The past form is 来た(きた).

来月(ライゲツ)next month.
来年(ライネン)next year.
「来年田中さんが日本に行きます。」Next year, Mr. Tanaka will go to Japan.
(``Next year'' means one year after the moment the speaker says it.)

8 東(トン、ひがし)east

\section{4 火 8 炎}

As the bottom part of another character,
the fire character is written as four dot strokes.

火(カ、ひ)fire

火(ひ)fire, flame

大火(タイカ)big fire

炎(ほのお)flame, blaze

炎天(エンテン)scorching sun

\section{4 王 5 玉生 8 国}

4 王(オウ)king

5 玉(ギョク、たま) ball

生depicts a sprout, something sprouting from the ground.
It has no relationship with 王.
Don't confuse them.
They only look similar.
生(セイ)nature; sex; gender.
学生(ガクセイ)student.
生まれる(うまれる)(v1,vi) to be born.

8 国(コク、くに) country

国王(コクオウ)king

国内(コクナイ)internal; domestic

\section{4 井开}

井 depicts a square well.
井(い)well (water reservoir).

开 is a simplification of 幵 depicting raising both hands.

\section{4 牛}

牛(ギュウ、うし)
cow; bull; ox; buffalo.
beef.

\section{4 水 5 氷}

4 水(スイ、みず)water

5 氷(こおり) ice

\section{4 氏}

氏 depicts a man bowing to the left.

氏(シ)(suffix,honorific) Mr.; Mrs.. family. clan.

氏(うじ)family name; birth; lineage.

\section{5 田申由出母石左右 6 回 8 門}

田(た)rice field

申 depicts a bolt of lightning.
申す(もうす)(humble,vt) to say; to speak.

由(よし)cause; reason.

母depicts a pair of breasts.
母(はは)mother

石(いし)stone

出 depicts something coming out of an open box.
出る(でる)(v1,vi) to go out; to exit; to leave.
出来る(できる)(v1,vi) to be able to do.
出来上がる(できあがる)(vi) to be finished; to be completed; to be ready

左(ひだり)left

右(みぎ)right.
Don't confuse this with 石(stone).

回 depicts a spiral.
回す(まわす)(vt) to turn; to rotate.
一回(イッカイ)once; one time.
一回目(イッカイめ)first.

門(かど) gate

\section{5 目 7 見 12 覚}

5 目(め)eye

7 見る(みる)(v1,vt) to see

見える(みえる)(v1,vi) to appear

12 覚める(さめる)(v1,vi) to wake up

\section{5 皿 6 血}

5 皿(さら)dish, plate

6 血(ち)blood

\section{6 両}

両(リョウ)both

\section{6 耳 14 聞}

6 耳(みみ)ear

14 聞 hear

聞く(きく)to hear

\section{7 車 8 雨店 9 面重}

車(くるま)car

雨(あめ)rain

店(テン)(n) store; shop.
ラメン店ramen shop (ramen is a kind of Japanese noodle).

面(おもて)mask

面白い(おもしろい)interesting

重 depicts a man carrying a bag.

重い(おもい)heavy (of weight)

\section{Others: 3 夂夊 4 攵 8 㑒}

In the Japanese language,
these characters become parts of other characters
instead of being used on their own.

夂 depicts two legs followed by something from behind.

夊 depicts a footprint.

攵 is a variant of 攴 depicting a branch and a hand.

㑒 is simplified from the 13-stroke 僉
meaning ``all, together, unanimous''.
