\chapter{Kanji 1}

Kanji is recursive: it may be made of other kanji.

The kanji in this chapter should be considered irreducible.
They cannot be broken apart into subkanji.

\section{Non-standalone parts}

\subsection{2 厶冂亻冫丩卜 3 彳㔾}

These parts rarely stand on their own.

厶 depicts I, myself, private.

冂 depicts an inverted box.

亻 is the side form of 人 (person).

冫 is the left form of 氷 (ice).

丩 depicts entanglement; intertwining.

卜 depicts a crack.

彳 depicts stepping or walking slowly.

What 㔾 depicts is not known.

\subsection{3 宀辶艹忄氵}

宀 depicts a roof.

辶 is the combining form of 辵 (walking).
Chinese dictionaries say this has 3 strokes.
Japanese dictionaries say this has 4 strokes.

艹 is the top form of 艸 (grass).

忄 is the left form of 心 (heart).

氵 is the left form of 水 (water).

\subsection{4 灬礻爪爫 5 衤}

灬 is the bottom form of 火 (fire).

礻 is the left form of 示 (spirit).

衤 is the left form of 衣 (cloth).

爪 depicts claw.

爫 is the top form of 爪.

\subsection{Disease: 5 疒}

疒depicts disease.

\subsection{6 艸}

艸 depicts grass.

\section{Numbers-related}

\subsection{1 一 2 二十八七九 3 三千万 4 五六 5 四 6 両百}

一(イチ、ひと)one

二(ニ、ふた)two

三(サン、み)three

四 four

五 five

六 six

七 seven

八 eight

九 nine

十 ten

両(リョウ)both

百(ヒャク)hundred

千(セン)thousand

万(マン、バン、よろず)ten thousand

\subsection{9 点}

点(テン、ぼち)point; spot; speck; mark.

\section{4 予 7 序}

予(ヨ、あらかじ)beforehand.

予め(あらかじめ)beforehand.

序(ジョ)preface

\section{7 図}

図(ズ)drawing; map; picture; plan; illustration; diagram; figure; chart.

\section{7 声}

図る(はかる)to plot; to plan; to design.

声(セイ、こえ)voice.
This character was simplified from 17 聲.

\section{8 門 9 面 10 書 12 傘}

門(モン、かど) gate

面(おもて)mask.
面白い(おもしろい)interesting.

書 depicts a hand holding a pen writing on paper.
書(ショ)writing.
辞書(ジショ)dictionary.
書く(かく)to write.

傘(かさ)umbrella.
