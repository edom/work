\chapter{Kanji 1}

Here are some simple characters, their stroke counts, and their meaning.
Parenthesized meaning means that it is the etymology
(the origin of the character, what it depicts),
not today's meaning.

Sometimes you understand kanji but you don't know how to pronounce it.

2 人 person\par
2 了 (wrapped baby with only head visible)\par
2 刀 sword\par
2 力 strength\par
3 女 woman\par
3 工 craft\par
3 小 small\par
3 土 earth\par
3 口 mouth\par
3 川 river\par
3 幺 (tiny; small; short thread)\par
3 山 mountain\par
4 手 hand\par
4 日 sun\par
4 月 moon\par
4 中 middle\par
4 王 king\par
4 父 father\par
4 火 fire\par
4 心 heart\par
4 水 water\par
4 木 tree\par
5 目 eye\par
5 皿 dish\par
5 母 mother\par
5 田 rice field\par
6 回 spiral\par
7 車 car\par
8 雨 rain\par

\section{2 人 3 大 4 太夫天 5 立}

2 人(ジン、レン、ニン、ひと)person

人(ひと)person; people; human

3 大(タイ) big

大きい(おおきい)big

4 太(タイ) thick

太い(ふとい)fat

太陽(タイヨウ)sun (lit. big yang (the yang in yin and yang))

4 夫(フ、おっと)husband

夫(おっと)husband

夫妻(フサイ)married couple; husband and wife

4 天(テン) sky; heaven

天気(テンキ)weather

天才(テンサイ)genius; prodigy

5 立 (person standing on ground)

立つ(たつ)to stand

\section{4 内 6 肉}

4 内(うち)inside

6 肉 (ribs of an animal's torso)

When 肉 acts as a left-side radical, it changes shape to ⺼ .

6 肉(ニク)meat; flesh; body (as opposed to spirit)

肉体(ニクタイ)body

筋肉(キンニク)muscle


\section{2 刀 4 切方}

2 刀(トウ)sword

刀(かたな)katana (Japanese sword)

4 切 (spoon and sword)

切る(きる)(v5r) cut

大切(タイセツ)(adj-na,n) important

4 方(かた)(tip of a sword)

\section{2 了 3 子 6 字 8 学}

了depicts a wrapped baby with only head visible.

子is了 with arms visible.

3 子(シ、こ) child

子供(こども)child

男子(ダンシ)young man, boy

女子(ジョシ)young woman, girl

太子(タイシ)crown prince

好き(すき)like, love, prefer

6 字(ジ)character, letter

漢字(カンジ)Han character

8 学(ガク)learning, scholarship, erudition, knowledge

中学(チュウガク)middle school; junior high school

大学(ダイガク)university

学生(ガクセイ)student

学界(ガッカイ)academic world

\section{2 力 9 重 11 動}

2 力(リョク、ちから)strength

9 重 (man carrying bag)

重い(おもい)heavy (of weight)

11 動(ドウ) motion

動く(うごく)(vi) to move

動画(ドウガ)animation, motion picture

自動車(ジドウシャ)automobile (lit. self-moving carriage)

動力(ドウリョク)power; motive power

\chapter{3 口 mouth}

\section{3 口(くち) mouth}

\section{5 古 old}

古consists of 十(ten) and 口(mouth, generation).

5 古い(ふるい)old

\section{7 言(こと) say}

言う(いう)to say

言葉(ことば)word; dialect

10 記 record

記す(しるす)to record, to write down

記録(キロク)record

13 話 talk

話す(はなす)to talk

14 読 read

読む(よむ)to read

14 語(ゴ) language

…語(…ゴ)... language

日本語(ニホンゴ)Japanese language

英語(エイゴ)English language

15 誰(だれ)who

\section{8 味(あじ)flavor; taste}


\section{3 女 6 好 8 妹 13 嫌}

3 女(おんな)woman

6 好き(すき)like

8 妹(いもうと)younger sister.
妹consists of 女(lady) and 未(not yet).

13 嫌い(きらい)hate

\section{3 小 4 少}

3 小(ショウ)small

(three sand granules)

小さい(ちいさい)small

4 少 few

少し(すこし)(adv, n) small quantity; few; a little

\section{3 幺 4 幻 6 糸 9 紅 11 細}

4 幻(まぼろし)phantom; vision; illusion; dream

6 糸(いと)thread

糸combines幺and小.

細い(ほそい)thin; slender
The left-side part of 細is the left-side form of 糸.

紅(くれない)deep red; crimson

\section{3 川}

3 川(かわ)river

\section{4 水 5 氷 6 汗 9 海 10 涙}

4 水(スイ、みず)water

冷たい(つめたい)(adj) cold (to the touch)

沈む(しずむ)to sink (descend into liquid)

6 汗(あせ)(n) sweat

汗をかく(exp,v5k) to sweat

汗を流す(exp,v5s) to work hard; to sweat (lit. to flow sweat)

浮かぶ(うかぶ)to float (be supported by liquid)

滴る(したたる)to drip (fall one drop at a time)

漏れる(もれる)to leak (liquid)

酒(さけ)(water west) sake (a Japanese liquor)

9 海(うみ)sea.
The original character has 10 strokes.
Shinjitai replaces the two dots in the middle
with one vertical stroke.

清い(きよい)clear (the character consists of water and blue 青)

流す(ながす)to flow (liquid)

涙(なみだ)tear (eyewater)

泳ぐ(およぐ)to swim

洪水(コウズイ)(n) flood (of liquid)

大水(おおみず)(n) flood (of liquid)

波(なみ)(n) wave (of liquid)

津波(つなみ)(n) tsunami; tidal wave

5 氷(こおり) ice

10 凍る(こおる)to freeze

\section{4 木 5 本未禾 6 休 7 体困村来 8 林東 9 秋茶 12 森}

4 木(き)tree

8 林(はやし)

12 森(もり)

5 本(ホン)book, counter for long cylindrical objects

一本(イッポン)one long cylindrical object, one point

日本(ニホン)Japan

本当(ホントウ)

5 未(ミ) not yet

未来(ミライ)future (lit. not yet come)

5 禾 plant stalk

9 秋(あき)autumn; fall season.
The character depicts the burning of plant stalks (after harvest).

7 来(ライ)come; next

来depicts fruits hanging on a tree.
It means that the time to harvest has come.

来る(くる)to come. This is an irregular verb. The past form is 来た(きた).

来月(ライゲツ)next month.
来年(ライネン)next year.
「来年田中さんが日本に行きます。」Next year, Mr. Tanaka will go to Japan.
(``Next year'' means one year after the moment the speaker says it.)

7 村(むら)village

8 東(トン、ひがし)east

6 休 rest

休む(やすむ)to rest

7 体(タイ、からだ)body

肉体(ニクタイ)body; flesh.

9 茶(チャ) tea

7 困 trouble

困る(こまる)(vi) to be troubled; to be embarrassed
(example?)

\section{4 日 5 白}

4 日(ひ) sun

日々(ひび)daily; days; old days

あの日(あのひ)that day

毎日(マイニチ)everyday

5 白(ハク) white

白い(しろい)white

告白(コクハク)confess (usually of love)

面白い(おもしろい)interesting

\section{4 月 8 明}

4 月(つき) moon

8 明(メイ) bright

明るい(あかるい)bright

\chapter{4 心(シン) heart}

Mostly about feelings.

心(こころ)heart

心配(シンパイ)(adj-na,n,vs) worry, concern, anxiety

心配(シンパイ)(n,vs) care, help

\section{5 必(ヒツ)}

Unrelated to 心. Only look similar.

必ず(かならず)(adv) always, invariably, certainly

必要(ヒツヨウ)(adj-na); (n) necessity, need

\section{7 忘 forget}

忘れる(わすれる)(v1) to forget

忘年会(ボウネンカイ)year-end party (lit. forget-year meeting, a meeting to forget the year)

\section{9 急(キュウ) urgent, sudden, abrupt}

急ぐ(いそぐ)to hurry

\section{9 思 think}

思is田(rice field) and 心(heart).

思う(おもう)to think

\section{10 恋(レン、こい) romance, love}

恋(こい)(n) love, tender passion

恋人(こいびと)lover; sweetheart

恋文(こいぶみ)love letter

\section{11 悪(アク) bad, evil, wicked}

悪(アク)evil, wickedness

悪人(アクニン)bad person, villain

悪い(わるい)bad, poor; evil; unprofitable; at fault

\section{12 悲(ヒ) sad}

悲しい(かなしい)sad

悲恋(ヒレン)disappointed love


\section{4 火 8 炎}

The fire radical becomes four dot strokes at the bottom side.

4 火(カ、ひ)fire

4 火(ひ)fire, flame

大火(タイカ)big fire

8 炎(ほのお)flame, blaze

炎天(エンテン)scorching sun

\section{4 内 6 肉}

4 内(うち)inside

6 肉 (ribs of an animal's torso)

When 肉 acts as a left-side radical, it changes shape to ⺼ .

6 肉(ニク)meat; flesh; body (as opposed to spirit)

肉体(ニクタイ)body

筋肉(キンニク)muscle

\section{4 中}

中(なか)(n) middle

\section{4 手 9 持}

4 手(て)hand

持つ(もつ)to hold; to carry; to possess

手洗い(てあらい)restroom; lavatory; toilet; a place for washing hands

手紙(てがみ)letter (the document, not the alphabet)

\section{4 王 5 玉主 8 国 9 美皇}

4 王(オウ)king

国王(コクオウ)king

5 玉(ギョク、たま) ball

5 主(おも)chief; main; principal; important

主人公(シュジンコウ)hero; main character

8 国(コク、くに) country

9 美 beauty

美味しい(おいしい)delicious (idiosyncratic reading)

美しい(うつくしい)beautiful

9 皇(コウ)imperial

皇居(コウキョ)imperial palace

\section{5 目 6 自 7 見 12 覚}

5 目(め)eye

7 見る(みる)(v1,vt) to see

見える(みえる)(v1,vi) to appear

12 覚める(さめる)(v1,vi) to wake up

6 自(ジ) self

自ら(みずから)(adv) personally

自在(ジザイ)freely (at will)

自分(ジブン)self (context? example usage?)

自爆(ジバク)suicide bombing, self-destruct (lit. self explode)

自動車(ジドウシャ)automobile (lit. self-moving carriage)

\section{5 皿 6 血}

5 皿(さら)dish, plate

6 血(ち)blood

\section{5 田 7 男 9 界}

7 男(おとこ)man

男is田(rice field) and 力(strength).

9 界(カイ)world

学界(ガッカイ)academic world

世界(セカイ)world

業界(ギョウカイ)industry world, business world

\section{6 回}

6 回(カイ) (spiral)

回す(まわす)to turn; to rotate

一回(イッカイ)once; one time

一回目(イッカイめ)first

\section{7 車}

車(くるま)car

\section{8 雨 11 雪 12 雲 13 雷電}

8 雨(あめ)rain

11 雪(ゆき)snow

12 雲(くも)cloud

13 雷(かみなり)thunder

13 電(デン) lightning

電光(デンコウ)lightning

電気(デンキ)electricity (lit. lightning spirit)

電話(デンワ)telephone (lit. lightning talk)

電車(デンシャ)electric train (lit. lightning carriage)

電撃(デンゲキ)electric shock

電気自動車(デンキジドウシャ)electric car
