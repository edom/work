\chapter{Kanji 1}

\section{1 一 2 二 3 三}

一(イチ、ひと)one

二(ニ、ふた)two

三(サン、み)three

\section{2 人 3 大 4 太夫天 5 立}

2 人 depicts a person.

人(ジン、レン、ニン、ひと)person; people; human

3 大 depicts a person with outstretched arms.

大(タイ)big

大きい(おおきい)big

4 太(タイ) thick

太い(ふとい)fat

4 夫(フ、おっと)husband

夫(おっと)husband

4 天(テン) sky; heaven

5 立 depicts a person standing on ground.

立つ(たつ)(vi) to stand

\section{2 了 3 子 6 字 8 学}

了depicts a wrapped baby with only head visible.

子is了 with arms visible.

3 子(シ、こ) child

6 字(ジ)character, letter (of an alphabet, not the letter that is a document)

8 学(ガク)learning, scholarship, erudition, knowledge

中学(チュウガク)middle school; junior high school

大学(ダイガク)university

学生(ガクセイ)student

学界(ガッカイ)academic world

\section{2 刀力}

刀(かたな)katana (Japanese sword).
刀(トウ)sword.

力(リョク、ちから)strength

\section{3 小 4 少}

3 小(ショウ)small

(three sand granules)

小さい(ちいさい)small

4 少 few

少し(すこし)(adv, n) small quantity; few; a little

\section{3 幺 6 糸}

3 幺 (tiny; small; short thread)

6 糸(いと)thread

糸combines幺and小.

\section{3 土士}

土(ド、つち) earth; soil

士(さむらい)samurai.
士(シ)(suffix)
a qualified person;
a person with a qualified profession.
学士(ガクシ)university graduate (who has obtained a degree).
工学士(コウガクシ)Bachelor of Engineering.

Do not confuse 士(samurai) and 土(earth).
士(samurai) has longer upper horizontal stroke.
土(earth) has shorter upper horizontal stroke.

\section{3 口女川山上下工 4 心父手中止}

口(くち) mouth

女(おんな)woman

川(かわ)river

山(サン、やま)mount, mountain

上(うえ)up

下(した)down

工(コウ、ク)craft.
工学(コウガク)engineering.
大工(ダイク)carpenter.
木工(モッコウ)carpenter.
工(たくみ)(name) Takumi.

心(シン、こころ)heart

父(ちち)the inner group's father.
お父さん(おとうさん)(polite) the outer group's father.

手(て)hand

中(チュウ、なか)middle.
中学(チュウガク)middle school; junior high school.
…の中(…のなか)middle of something.
田中(たなか)(name) Tanaka.
中山(なかやま)(name) Nakayama.
中川(なかがわ)(name) Nakagawa.

止depicts a footprint.

止める(とめる)(v1) to stop moving (walking, etc.); to park (a car)

\section{4 水 5 氷}

4 水(スイ、みず)water

5 氷(こおり) ice

\section{4 木 5 本未 7 来 8 林東 12 森}

4 木(き)tree

8 林(はやし)

12 森(もり)

5 本(ホン)book, counter for long cylindrical objects

一本(イッポン)one long cylindrical object, one point

日本(ニホン)Japan

5 未(ミ) not yet

未来(ミライ)future (lit. not yet come)

7 来(ライ)come; next

来depicts fruits hanging on a tree.
It means that the time to harvest has come.

来る(くる)to come. This is an irregular verb. The past form is 来た(きた).

来月(ライゲツ)next month.
来年(ライネン)next year.
「来年田中さんが日本に行きます。」Next year, Mr. Tanaka will go to Japan.
(``Next year'' means one year after the moment the speaker says it.)

8 東(トン、ひがし)east

\section{4 日月 5 白 8 明}

4 日(ひ) sun

日々(ひび)daily; days; old days

4 月(つき) moon

5 白(ハク) white

白い(しろい)white

8 明(メイ) bright

明るい(あかるい)bright

\section{4 火 8 炎}

As the bottom part of another character,
the fire character is written as four dot strokes.

火(カ、ひ)fire

火(ひ)fire, flame

大火(タイカ)big fire

炎(ほのお)flame, blaze

炎天(エンテン)scorching sun

\section{4 内 6 肉}

4 内(うち)inside

6 肉 (ribs of an animal's torso)

6 肉(ニク)meat; flesh; body (as opposed to spirit)

\section{4 王 5 玉 8 国}

4 王(オウ)king

5 玉(ギョク、たま) ball

8 国(コク、くに) country

国王(コクオウ)king

\section{5 石母田 6 回 8 門}

石(いし)stone

母depicts a pair of breasts.

母(はは)mother

田(た)rice field

回 depicts a spiral.

回す(まわす)to turn; to rotate

一回(イッカイ)once; one time

一回目(イッカイめ)first

門(かど) gate

\section{5 左右}

左(ひだり)left

右(みぎ)right

\section{5 目 7 見 12 覚}

5 目(め)eye

7 見る(みる)(v1,vt) to see

見える(みえる)(v1,vi) to appear

12 覚める(さめる)(v1,vi) to wake up

\section{5 皿 6 血}

5 皿(さら)dish, plate

6 血(ち)blood

\section{6 耳 14 聞}

6 耳(みみ)ear

14 聞 hear

聞く(きく)to hear

\section{7 車 8 雨店 9 面重}

車(くるま)car

雨(あめ)rain

店(テン)(n) store; shop.
ラメン店ramen shop (ramen is a kind of Japanese noodle).

面(おもて)mask

面白い(おもしろい)interesting

重 depicts a man carrying a bag.

重い(おもい)heavy (of weight)
