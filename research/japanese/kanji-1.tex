\chapter{Kanji 1}

\section{Common parts}

Kanji is recursive: it may be made of other kanji.

This describes what the kanji used to depict, \emph{not} what it means.

\paragraph{2}
冂 inverted box.

\paragraph{3}
艹 top form of 艸 (grass).
忄 left form of 心 (heart).

\paragraph{4}
斤 axe.
心 heart.
礻 left form of ⽰ (spirit).
开 is a simplification of 幵 (raising both hands).

\paragraph{5}
⽰ spirit.
衤 left form of 衣 (cloth).

\paragraph{6}
艸 grass.
耂 bent-over figure with long hair; old man.
衣 cloth.
幵 raising both hands.

\paragraph{8}
隹 short-tailed bird.

\paragraph{9}
重 man carrying bag.

\section{People}

\subsection{People: 2 人亻 2 了 3 士女 7 男呆}

亻 is the side form of 人.

人(ジン、レン、ニン、ひと)person; people; human

士(さむらい)gentleman; samurai.
士(シ)(suffix)
a qualified person;
a person with a qualified profession.
学士(ガクシ)university graduate (who has obtained a degree).
工学士(コウガクシ)Bachelor of Engineering.

女(おんな)woman

呆 depicts a child.
呆れる(あきれる)(v1,vi) to be amazed, astonished, astounded.

男(おとこ)man.

\subsection{Babies and children: 3 子 6 字 7 児}

了 depicts a wrapped baby with only head visible.

子is了 with arms visible.
子(シ、こ) child.
男子(ダンシ)young man, boy.
女子(ジョシ)young woman, girl.
太子(タイシ)crown prince.

字(ジ)character, letter (of an alphabet, not the letter that is a document)
赤字(あかジ)deficit; red letter; red Han-character.
漢字(カンジ)Han character.

児 depicts an infant with imperfect cranium (fontanelles).
児 is simplified from 兒.
乳児(ニュウジ)infant; suckling baby.
男児(ダンジ)boy; son.

\subsection{Locomotion: 4 止 5 立}

止 depicts a footprint.

止める(やめる)(v1) to stop (doing something)

止める(とめる)(v1) to stop moving (walking, etc.); to park (a car)

立 depicts a person standing on ground.
立つ(たつ)(vi) to stand

\section{Nature: 3 山川 4 火灬 5 石}

川(かわ)river

山(サン、やま)mount, mountain

灬 is the bottom form of 火.

火(カ、ひ)fire, flame.
火事(カジ)fire (disaster).
大火(タイカ)big fire.

石(いし)stone

\subsection{Earth: 3 土 4 井介 5 田世 9 界}

土(ド、つち)
soil; earth; ground; dirt; ground (as opposed to the heavens).

井 depicts a square well.
井(い)well (water reservoir).

介 can mean mediation.
紹介(ショウカイ)introduction; referral.
仲介(チュウカイ)agency; intermediation.
介入(カイニュウ)intervention.
介助(カイジョ)help; assistance; aid.

介 can mean shellfish.
魚介(ギョカイ)marine products; seafood.

田(た)rice field.

世(よ)world; society; age; generation

界(カイ)world.
世界(セカイ)world.
学界(ガッカイ)academic world.
業界(ギョウカイ)industry world, business world.

\subsection{Water: 2 冫 3 氵 4 水 5 氷}

冫 is the left form of 氷 (ice).

氵 is the left form of 水 (water).

水(スイ、みず)water.
水車(スイシャ)water wheel.

氷(こおり)ice

\subsection{Trees: 4 木 5 禾本 6 米休 7 体 8 林 12 森}

木(き)tree

禾 depicts a plant stalk.

本(ホン)book, counter for long cylindrical objects.
(In Ancient China, a book is a scroll.)
一本(イッポン)one long cylindrical object, one point.
日本(ニホン)Japan.

米(ベイ、こめ)rice.

休む(やすむ)to rest

体(タイ、からだ)body.
肉体(ニクタイ)body; flesh.

林(はやし)woods; forest; copse; thicket

森(もり)forest

森林(シンリン)forest woods

\subsection{Not-yet and next: 5 未 7 来}

未(ミ)not yet.
未来(ミライ)future (lit. not yet come).
未 depicts a tree that has not fruited.
来 depicts fruits hanging on a tree.
They mean that the time to harvest has not or has come.

来(ライ)come; next.
来月(ライゲツ)next month.
来年(ライネン)next year.
「来年田中さんが日本に行きます。」Next year, Mr. Tanaka will go to Japan.
(``Next year'' means one year after the moment the speaker says it.)

来る(くる)to come.
This is an irregular verb.
The past form is 来た(きた).

\section{Space}

\subsection{Inside-outside: 2 入 4 中内 5 出 6 肉}

入る(いる)(v5)
to enter; to go into; to get into.

中(チュウ、なか)middle.
中学(チュウガク)middle school; junior high school.
…の中(…のなか)middle of something.
田中(たなか)(name) Tanaka.
中山(なかやま)(name) Nakayama.
中川(なかがわ)(name) Nakagawa.

内(うち)inside

出 depicts something coming out of an open box.
出る(でる)(v1,vi) to go out; to exit; to leave.
出来る(できる)(v1,vi) to be able to do.
出来上がる(できあがる)(vi) to be finished; to be completed; to be ready

肉 depicts the ribs of an animal's torso.
肉(ニク)meat; flesh; body (as opposed to spirit).

\subsection{Directions: 3 上下 5 左右}

上(うえ)up

下(した)down

左(ひだり)left

右(みぎ)right.

\subsection{This: 4 之}

之(これ)this

\section{Size and amount}

\subsection{Big-small: 3 大小}

大 depicts a person with outstretched arms.
大(タイ)big.
大きい(おおきい)big.

小 depicts three sand granules.
小さい(ちいさい)small.
小(ショウ)small.

\subsection{Thick-thin: 4 太 11 細}

太い(タイ、タ、ふとい)(adj-i) fat; thick.
太る(ふとる)to grow fat; to become fat; to gain weight.

細い(サイ、ほそい)(adj-i) thin; slender.
細る(ほそる)to become thin.
細かい(こまかい)(adj-i) small; trivial.

\subsection{Few-many: 4 少 6 多}

少し(すこし)(adv, n) small quantity; few; a little

多 depicts two pieces of meat.
多 means ``many''.
多(タ)(prefix) multi-.

\section{Small objects}

\subsection{Threads: 3 幺 6 糸}

幺 depicts a tiny/small/short thread.

糸 combines 幺 and 小.
糸(いと)thread.

The left-side part of 細is the left-side form of 糸.

\subsection{Blades: 2 刀刂 3 弋 4 戈 5 戊 6 戌 9 咸}

刀(トン、かたな)sword.

刂 depicts a knife.

弋 depicts shooting with a bow and an arrow.

戈 depicts a spear-axe (a halberd).

戊 depicts a dagger-axe (an ancient Chinese weapon).

戌 depicts an axe.

咸 depicts an axe and a mouth.

\subsection{Dish: 5 皿}

皿(さら)dish, plate

\subsection{Cowry: 7 貝}

貝 depicts a cowry (a kind of seashell used as money in ancient China).
貝(かい)shell; shellfish.

\subsection{Mask: 9 面}

面(おもて)mask

面白い(おもしろい)interesting

\section{Structures}

\subsection{Roofs: 3 宀}

宀 depicts a roof.

\subsection{Roads: 3 辶}

辶 is the combining form of 辵 meaning ``walking''.
Chinese dictionaries say this has 3 strokes.
Japanese dictionaries say this has 4 strokes,
but also say that 近 has 7 strokes.
If the Japanese dictionaries are to be followed,
then 近 should have 8 strokes.

\subsection{Gate: 8 門}

門(かど) gate

\subsection{Tower: 5 台}

台(うてな)tower; stand; pedestal.
台 is simplified form of 14 臺.

\section{Animals: 4 犬牛 6 虫 10 馬 11 魚猫豚鳥}

魚介(ギョカイ)seafood; marine products

動物(ドウブツ)animals.

犬(いぬ)dog.

牛(ギュウ、うし)
cow; bull; ox; buffalo.
beef.

虫(むし)insect; bug; cricket; moth

動物園(ドウブツエン)zoo (lit. animal garden)

馬(うま)horse.

鳥(とり)bird; chicken; chicken meat.
焼き鳥(やきとり)grilled chicken meat.

魚(さかな)fish.

猫(ねこ)cat.

豚(ぶた)pig.

\subsection{9 臭}

臭: In China 10 strokes, in Japan 9 strokes.
The lower character is 犬 in China and 大 in Japan.
臭い(くさい)(adj-i) stinking; malodorous; ill-smelling.

\section{Family members: 4 父 5 母兄 7 弟姉 8 妹 10 娘 16 親}

父(ちち)(humble) father.
お父さん(おとうさん)(honorific) father.

母 depicts a pair of breasts.
母(はは)mother.
お母さん(おかあさん)(honorific)

兄(あに)(humble) older brother.
お兄さん(おにいさん)(honorific) older brother.

弟(おとうと)(humble) younger brother.
弟さん(おとうとさん)(honorific) younger brother.

兄弟(キョウダイ)siblings;
brothers and sisters
(although the characters mean older brother and younger brother).

姉(あね)(humble) older sister.
お姉さん(おねえさん)(honorific) older sister.

妹(いもうと)(humble) younger sister.
妹さん(いもうとさん)(honorific) younger sister.

息子(むすこ)son

娘(むすめ)daughter

親(おや)parent.
両親(リョウシン)both parents.

\subsection{Body parts: 3 口 5 目 6 血耳舌 9 首}

口(くち)mouth.

目(め)eye

血(ち)blood.
止血(シケツ)stop bleeding; hemostasis.

耳(みみ)ear

舌(した)tongue.

首(くび)neck

\subsection{Hand: 2 又 4 手殳}

又 depicts the right hand.

又(また)again; also.

手(て)hand.
手がける(てがける)(v1) to make; to do; to produce; to work on.
手洗い(てあらい)restroom; lavatory; toilet; a place for washing hands

殳 depicts a hand holding a tool or a weapon.

\section{Administration}

\subsection{Literature: 4 文 10 書}

文(ブン)sentence (literature).
文字(モジ)letter (of alphabet); character (a Han character)
文書(ブンショ)sentence.
文化(ブンカ)culture.
作文(サクブン)writing.
小学生の作文(ショウガクセイのサクブン)elementary-schoolchild writing.

書 depicts a hand holding a pen writing on paper.
書(ショ)writing.
辞書(ジショ)dictionary.
書く(かく)to write.

\subsection{History: 5 史}

史家(シカ)historian.

\subsection{Regions: 4 区 5 市 8 京}

区(ク)ward; district (an administrative area).

市(シ)city (an administrative area).
市(いち)market; fair (trade show).

京(キョウ)imperial capital

\subsection{Ministers: 9 相省}

首相(シュショウ)prime minister

4 少 + 5 目

省く(はぶく)(vt)
to omit; to leave out; to exclude.
to curtail; to save; to cut down; to economize.

省(ショウ)(suf) ministry; department

国交省(コッコウショウ)(abbr)
Ministry of Land, Infrastructure, Transport, and Tourism

\section{Others}

\subsection{Strength: 2 力}

力(リョク、リキ、ちから)strength.
百人力(ヒャクジンリキ)tremendous strength.

\subsection{Weight: 9 重}

重い(おもい)heavy (of weight).

\subsection{Loss: 3 亡 5 失 6 死}

亡 means death; destruction; perishment; the deceased.
亡くす(なくす)(vt) to lose something (because he/she/it dies).

失 depicts something falling from a hand.
失う(うしなう)(vt) to lose; to part with.
失明(シツメイ)loss of eyesight.
失血(シッケツ)loss of blood.

死(シ)death.
死ぬ(しぬ)to die.
死亡(シボウ)death; mortality.
死去(シキョ)death.

\subsection{Sprout: 5 生}

\subsection{Meeting: 6 合会}

合う(あう)to fit

会う(あう)to meet (face to face)

会社(カイシャ)company; corporation

会社員(カイシャイン)company employee

年会(ネンカイ)yearly meeting; annual convention

社会(シャカイ)society

会見(カイケン)interview

\subsection{Wheel: 7 車}

車(くるま)car

\subsection{Bitter: 7 辛}

辛 depicts a tool used to mark slaves and criminals;
this sometimes also depicts a tree.

辛い(からい)(adj-i) spicy, salty, harsh, hot, acrid

辛い(つらい)(adj-i) painful; heartbreaking; difficult.
Suffix づらい(adj-i) means ``difficult to do''.
読みづらい(adj-i) difficult to read.
書きづらい(adj-i) difficult to write.
読みづらい漢字difficult-to-read Han character.

\subsection{Light: 6 光}

光(コウ、ひかり)light (electromagnetic wave)

\subsection{Hit: 6 当}

当てる(あてる)(v1,vt) to hit.
本当(ホントウ)truth; reality.
本当に(ホントウに)truly; really; seriously.

\subsection{Red: 7 赤}

赤い(あかい)(adj-i) red.

\subsection{Nail: 2 丁}

丁 depicts a nail.

丁(ひのと)

丁(チョウ)(suffix)
counter for long and narrow (relatively flat) thing
such as paper, guns, scissors, spades, hoes, guitars.
一丁(イッチョウ)one sheet; one page.
one serving (in a restaurant).

丁重(テイチョウ)polite; courteous; hospitable

包丁(ホウチョウ)kitchen knife
