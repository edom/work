\chapter{Kanji 1}

Kanji is recursive: it may be made of other kanji.

The kanji in this chapter should be considered irreducible.
They cannot be broken apart into subkanji.

\section{Non-standalone parts}

\subsection{2 厶冂亻冫丩卜 3 彳㔾}

These parts rarely stand on their own.

厶 depicts I, myself, private.

冂 depicts an inverted box.

亻 is the side form of 人 (person).

冫 is the left form of 氷 (ice).

丩 depicts entanglement; intertwining.

卜 depicts a crack.

彳 depicts stepping or walking slowly.

What 㔾 depicts is not known.

\subsection{Mound and city: 3 阝 7 邑 8 阜}

阝 has 2 strokes in Chinese and 3 strokes in Japanese.
On the left side, 阝 is 阜 (mound).
On the right side, 阝 is 邑 (city; village).

\subsection{3 宀辶艹忄氵}

宀 depicts a roof.

辶 is the combining form of 辵 (walking).
Chinese dictionaries say this has 3 strokes.
Japanese dictionaries say this has 4 strokes,
but also say that 近 has 7 strokes.
If the Japanese dictionaries are to be followed,
then 近 should have 8 strokes.

艹 is the top form of 艸 (grass).

忄 is the left form of 心 (heart).

氵 is the left form of 水 (water).

\subsection{4 灬礻爪爫 5 衤}

灬 is the bottom form of 火 (fire).

礻 is the left form of 示 (spirit).

衤 is the left form of 衣 (cloth).

爪 depicts claw.

爫 is the top form of 爪.

\subsection{Disease: 5 疒}

疒depicts disease.

\subsection{6 艸}

艸 depicts grass.

\section{Numbers-related}

\subsection{1 一 2 二十八七九 3 三千万 4 五六 5 四 6 両百}

一(イチ、ひと)one

二(ニ、ふた)two

三(サン、み)three

四 four

五 five

六 six

七 seven

八 eight

九 nine

十 ten

両(リョウ)both

百(ヒャク)hundred

千(セン)thousand

万(マン、バン、よろず)ten thousand

\subsection{9 点}

点(テン、ぼち)point; spot; speck; mark.

\section{People-related}

\subsection{2 人 3 己大士女 4 犬太父 5 立母兄民 6 共}

人(ジン、レン、ニン、ひと)person; people; human

己(コ、おのれ)self

大 depicts a person with outstretched arms.
大(タイ)big.
大きい(おおきい)big.

士(さむらい)gentleman; samurai.
士(シ)(suffix)
a qualified person;
a person with a qualified profession.
学士(ガクシ)university graduate (who has obtained a degree).
工学士(コウガクシ)Bachelor of Engineering.

女(おんな)woman

犬(いぬ)dog.

太い(タイ、タ、ふとい)(adj-i) fat; thick.
太る(ふとる)to grow fat; to become fat; to gain weight.

父(フ、ちち、とう)father.
父(ちち)(humble) father.
お父さん(おとうさん)(honorific) father.

立 depicts a person standing on ground.
立つ(たつ)(vi) to stand

母 depicts a pair of breasts.
母(ボ、はは、かあ)mother.
母(はは)(humble) mother.
お母さん(おかあさん)(honorific) mother.

兄(ケイ、キョウ、あに)older brother.
兄(あに)(humble) older brother.
お兄さん(おにいさん)(honorific) older brother.

民(ミン、たみ)people

共(とも)companion; follower; attendant; retinue.

\subsection{4 天夫 5 矢失}

天(テン)sky; heaven

夫(フ、おっと)husband

矢(シ、や)arrow

失 depicts something falling from a hand.
失う(うしなう)(vt) to lose; to part with.
失明(シツメイ)loss of eyesight.
失血(シッケツ)loss of blood.

\subsection{2 了 3 子 6 字 7 呆弟男児}

了 depicts a wrapped baby with only head visible.

子 depicts a child with arms visible.
子(シ、こ) child.
男子(ダンシ)young man, boy.
女子(ジョシ)young woman, girl.
太子(タイシ)crown prince.

字(ジ)character, letter (of an alphabet, not the letter that is a document)
赤字(あかジ)deficit; red letter; red Han-character.
漢字(カンジ)Han character.

呆 depicts a child.
呆れる(あきれる)(v1,vi) to be amazed, astonished, astounded.

弟(テイ、ダイ、おとうと)younger brother.
弟(おとうと)(humble) younger brother.
弟さん(おとうとさん)(honorific) younger brother.
兄弟(キョウダイ)siblings;
brothers and sisters
(although the characters mean older brother and younger brother).

男(おとこ)man.

児 depicts an infant with imperfect cranium (fontanelles).
児 is simplified from 兒.
乳児(ニュウジ)infant; suckling baby.
男児(ダンジ)boy; son.

\subsection{Body parts: 2 又 3 口 4 心氏止手毛开 5 目 6 自血耳舌 9 首}

又 depicts the right hand.
又(また)again; also.

口(くち)mouth.

心(シン、こころ)heart.

氏 depicts a man bowing to the left.
氏(シ)(suffix,honorific) Mr.; Mrs.. family. clan.
氏(うじ)family name; birth; lineage.

止 depicts a footprint.
止(シ、と)stop.
止まる(とまる)(vi) to stop (moving); to come to a halt.
止める(やめる)(v1) to stop (doing something).
止める(とめる)(v1) to stop moving (walking, etc.); to park (a car).

手(て)hand.
手がける(てがける)(v1) to make; to do; to produce; to work on.
手洗い(てあらい)restroom; lavatory; toilet; a place for washing hands

毛(モウ、け)hair.

开 is a simplification of 幵 (raising both hands).

目(め)eye

自(ジ、シ、みずか)self; oneself.
自ら(みずから)(adv) personally.
自在(ジザイ)freely (at will).
自分(ジブン)self.
自身(ジシン)self.

血(ち)blood.
止血(シケツ)stop bleeding; hemostasis.

耳(みみ)ear

舌(ゼツ、した)tongue.

首(シュ、くび)neck

\subsection{Blades: 2 刀刂 3 弋 4 戈斤 5 戊 6 戌 9 咸}

刀(トン、かたな)sword.

刂 depicts a knife.

弋 depicts shooting with a bow and an arrow.

戈 depicts a spear-axe (a halberd).

斤 depicts an axe.

戊 depicts a dagger-axe (an ancient Chinese weapon).

戌 depicts an axe.

咸 depicts an axe and a mouth.

\subsection{Administrative divisions: 4 区 5 市 7 町 8 京 9 県}

% https://en.wikipedia.org/wiki/Administrative_divisions_of_Japan

区(ク)ward; district (an administrative division).

市(シ)city (an administrative division).
市(いち)market; fair (trade show).

町(チョウ、まち)town.

京(キョウ、ケイ、みやこ).
京(キョウ)imperial capital.

県(ケン)prefecture.

\subsection{4 予 7 序}

予(ヨ、あらかじ)beforehand.
予め(あらかじめ)beforehand.

序(ジョ)preface

\subsection{4 介}

介(カイ)can mean mediation or shellfish.
紹介(ショウカイ)introduction; referral.
仲介(チュウカイ)agency; intermediation.
介入(カイニュウ)intervention.
介助(カイジョ)help; assistance; aid.
魚介(ギョカイ)marine products; seafood.

\subsection{4 今 6 合会}

今(コン、いま)now.
今回(コンカイ)this time; this occasion; this occurrence.

合(ゴウ、あ)fit.
合う(あう)to fit.

会(カイ、エ、あ)meet.
会う(あう)to meet (face to face).
会社(カイシャ)company; corporation.
会社員(カイシャイン)company employee.
年会(ネンカイ)yearly meeting; annual convention.
社会(シャカイ)society.
会見(カイケン)interview.

\subsection{2 丁 5 庁}

丁(ひのと)depicts a nail.
丁(チョウ)(suffix)
counter for long and narrow (relatively flat) thing
such as paper, guns, scissors, spades, hoes, guitars.
一丁(イッチョウ)one sheet; one page.
one serving (in a restaurant).
丁重(テイチョウ)polite; courteous; hospitable.
包丁(ホウチョウ)kitchen knife.

丁(チョウ)town section.

庁(チョウ)government office.

\subsection{2 力}

力(リョク、リキ、ちから)strength.
百人力(ヒャクジンリキ)tremendous strength.

\subsection{7 貝 9 頁}

貝 depicts a cowry (a kind of seashell used as money in ancient China).
貝(かい)shell; shellfish.

頁 originally depicts a head,
but comes to mean ``leaf'' (an organ in plants).

\subsection{7 図車声}

図(ズ)drawing; map; picture; plan; illustration; diagram; figure; chart.
図る(はかる)to plot; to plan; to design.

車(くるま)car; vehicle. wheel.
水車(スイシャ)water wheel.

声(セイ、こえ)voice.
This character was simplified from 17 聲.

\subsection{8 門 9 重面 10 書 12 傘}

門(モン、かど) gate

重 depicts a man carrying a bag.
重い(おもい)heavy (of weight).

面(おもて)mask.
面白い(おもしろい)interesting.

書 depicts a hand holding a pen writing on paper.
書(ショ)writing.
辞書(ジショ)dictionary.
書く(かく)to write.

傘(かさ)umbrella.


\section{Other animals: 10 烏 11 猫}

魚介(ギョカイ)seafood; marine products

動物(ドウブツ)animals.

動物園(ドウブツエン)zoo (lit. animal garden)

烏(からす)crow; raven.
More often written with katakana as カラス instead of with kanji.
烏羽色(からすばいろ)glossy black; crow feather color.

猫(ねこ)cat.

\section{4 予 7 序}

予(ヨ、あらかじ)beforehand.

予め(あらかじめ)beforehand.

序(ジョ)preface

\section{7 図声}

図(ズ)drawing; map; picture; plan; illustration; diagram; figure; chart.
図る(はかる)to plot; to plan; to design.

声(セイ、こえ)voice.
This character was simplified from 17 聲.

\section{8 門 9 面 10 書 12 傘}

門(モン、かど) gate

面(おもて)mask.
面白い(おもしろい)interesting.

書 depicts a hand holding a pen writing on paper.
書(ショ)writing.
辞書(ジショ)dictionary.
書く(かく)to write.

傘(かさ)umbrella.
