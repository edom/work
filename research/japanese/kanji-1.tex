\chapter{Kanji 1}

\section{Confusing characters}

\subsection{3 夂夊 4 攵 8 㑒}

In the Japanese language,
these characters become parts of other characters
instead of being used on their own.

夂 depicts two legs followed by something from behind.

夊 depicts a footprint.

攵 is a variant of 攴 depicting a branch and a hand.

㑒 is simplified from the 13-stroke 僉
meaning ``all, together, unanimous''.

\subsection{士 (samurai) and 土 (earth)}

士(samurai) has longer upper horizontal stroke.
土(earth) has shorter upper horizontal stroke.

\subsection{Hat, sun, moon, meat, inner}

冃(hat)

日(sun)

月(moon)

\subsection{石 (stone) and 右 (right)}

\subsection{人 (person) and 入 (enter)}

\subsection{王 (king) and 生 (sprout)}

\subsection{業菐美}

\section{2 入 4 内 6 肉}

入る(いる)(v5)
to enter; to go into; to get into.

内(うち)inside

肉 (ribs of an animal's torso)

肉(ニク)meat; flesh; body (as opposed to spirit)

\section{4 之}

之(これ)this

\section{3 工山川上下 4 中止}

川(かわ)river

山(サン、やま)mount, mountain

上(うえ)up

下(した)down

工(コウ、ク)craft.
工学(コウガク)engineering.
大工(ダイク)carpenter.
木工(モッコウ)carpenter.
工(たくみ)(name) Takumi.

中(チュウ、なか)middle.
中学(チュウガク)middle school; junior high school.
…の中(…のなか)middle of something.
田中(たなか)(name) Tanaka.
中山(なかやま)(name) Nakayama.
中川(なかがわ)(name) Nakagawa.

止depicts a footprint.

止める(とめる)(v1) to stop moving (walking, etc.); to park (a car)

\section{5 平半}

平 can mean flat, level (not tilted), ordinary, plain, non-special.
平ら(たいら)flatness.
平たい(ひらたい)(adj-i) flat; even; level; simple.
平皿(ひらざら)flat dish.
平安(ヘイアン)peace; tranquility.
平気(ヘイキ)coolness; calmness; composure; unconcern.
平日(ヘイジツ)weekday; ordinary day (non-holiday).
平年(ヘイネン)normal (non-leap) year; normal year (related to harvest; weather).
公平(コウヘイ)fairness; impartiality; justice
水平(スイヘイ)level; horizontally
平文(ヘイブン)plain (non-encrypted) text.
平面(ヘイメン)level (flat and not-tilted) surface.

半(ハン)half

\section{4 日月 5 白 6 早 8 明}

日(ひ)sun.
日々(ひび)daily; days; old days.

月(つき)moon.

白(ハク)white.
白い(しろい)white.

早い(はやい)(adj-i) early.

明(メイ)bright.
明るい(あかるい)bright.

\subsection{9 星 12 朝}

日付(ひづけ)date.日付別(ひづけベツ)separate by date.(context? usage?)

星(ほし)star

朝(あさ)morning.
今朝(けさ)this morning.
早朝(ソウチョウ)early morning.

\section{4 王 5 玉生 8 国}

4 王(オウ)king

5 玉(ギョク、たま) ball

生depicts a sprout, something sprouting from the ground.
生(セイ)nature; sex; gender.
学生(ガクセイ)student.
生まれる(うまれる)(v1,vi) to be born.

8 国(コク、くに) country.
国王(コクオウ)king.
国内(コクナイ)internal; domestic.

\subsection{5 主 9 美皇}

5 主(おも)chief; main; principal; important

主人公(シュジンコウ)hero; main character

9 美 beauty

美味しい(おいしい)delicious (idiosyncratic reading)

美しい(うつくしい)beautiful

9 皇(コウ)imperial

皇居(コウキョ)imperial palace

\section{4 井开}

井 depicts a square well.
井(い)well (water reservoir).

开 is a simplification of 幵 depicting raising both hands.

\section{5 田由出石左右 6 回向}

田(た)rice field

由(よし)cause; reason.

石(いし)stone

出 depicts something coming out of an open box.
出る(でる)(v1,vi) to go out; to exit; to leave.
出来る(できる)(v1,vi) to be able to do.
出来上がる(できあがる)(vi) to be finished; to be completed; to be ready

左(ひだり)left

右(みぎ)right.

回 depicts a spiral.
回す(まわす)(vt) to turn; to rotate.
一回(イッカイ)once; one time.
一回目(イッカイめ)first.

向 depicts a house and a window.
向かい(むかい)(n) facing; opposite; across the street; other side.
向く(むく)to face; to turn toward.
向こう(むこう)opposite side; other side; opposite direction.
向上(コウジョウ)improvement; advancement; progress.

\section{5 皿}

皿(さら)dish, plate

\section{5 世 6 両}

世(よ)world; society; age; generation

両(リョウ)both

\section{7 車 8 店 9 面}

車(くるま)car

店(テン)(n) store; shop.
ラメン店ramen shop (ramen is a kind of Japanese noodle).

面(おもて)mask

面白い(おもしろい)interesting
