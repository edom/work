\chapter{Kanji 1}

Kanji is recursive: it may be made of other kanji.

The kanji in this chapter should be considered irreducible.
They cannot be broken apart into subkanji.

\section{Non-standalone parts}

\subsection{2 厶冂亻冫丩卜 3 彳㔾}

These parts rarely stand on their own.

厶 depicts I, myself, private.

冂 depicts an inverted box.

亻 is the side form of 人 (person).

冫 is the left form of 氷 (ice).

丩 depicts entanglement; intertwining.

卜 depicts a crack.

彳 depicts stepping or walking slowly.

What 㔾 depicts is not known.

\subsection{Mound and city: 3 阝 7 邑 8 阜}

阝 has 2 strokes in Chinese and 3 strokes in Japanese.
On the left side, 阝 is 阜 (mound).
On the right side, 阝 is 邑 (city; village).

\subsection{3 宀辶艹忄氵}

宀 depicts a roof.

辶 is the combining form of 辵 (walking).
Chinese dictionaries say this has 3 strokes.
Japanese dictionaries say this has 4 strokes,
but also say that 近 has 7 strokes.
If the Japanese dictionaries are to be followed,
then 近 should have 8 strokes.

艹 is the top form of 艸 (grass).

忄 is the left form of 心 (heart).

氵 is the left form of 水 (water).

\subsection{4 灬礻爪爫 5 衤}

灬 is the bottom form of 火 (fire).

礻 is the left form of 示 (spirit).

衤 is the left form of 衣 (cloth).

爪 depicts claw.

爫 is the top form of 爪.

\subsection{6 艸 8 㑒}

艸 depicts grass.

㑒 is simplified from the 13-stroke 僉
meaning ``all, together, unanimous''.

\section{3 士女 4 氏 5 仕民 11 婚 12 結}

士(さむらい)gentleman; samurai.

士(シ)(suffix)
a qualified person;
a person with a qualified profession.

学士(ガクシ)university graduate (who has obtained a degree).

工学士(コウガクシ)Bachelor of Engineering.

女(おんな)woman

氏 depicts a man bowing to the left.

氏(シ)(suffix,honorific) Mr.; Mrs.. family. clan.

氏(うじ)family name; birth; lineage.

仕(シ)serve.

仕事(シごと)(n) work; job; business; occupation; employment.

仕方(シかた)way; method; manner.

仕方ない。It can't be helped. There's no other way.

仕える(つかえる)(v1,vi) to serve; to work; to attend.

民(ミン、たみ)people

婚(コン)marriage.

結 婚(ケッコン)marriage.

結婚式(ケッコンシキ)marriage ceremony.

新婚(シンコン)newlywed.

結(ケツ、むす、ゆ)bind; tie (not the wearable).

結う(ゆう)to do up hair.

結ぶ(むすぶ)to bind; to tie; to link.

結合(ケツゴウ)combination; union; binding; catenation; coupling; joining; bond.

\section{5 业 8 並 12 普}

业 is an alternative form of 北 (north).

並 depicts two people standing side-by-side.

並(ヘイ、なみ、なら)row.
並(なみ)row.
並ぶ(ならぶ)to stand in line; to line up.

普(フ)universal; wide; general.
普通(フツウ)(adj-no) general; ordinary; usual.
普通の人間(フツウのニンゲン)ordinary human.

\section{Numbers-related}

\subsection{1 一 2 二十八七九 3 三千万 4 五六 5 四 6 両百}

一(イチ、ひと)one

二(ニ、ふた)two

三(サン、み)three

四 four

五 five

六 six

七 seven

八 eight

九 nine

十 ten

両(リョウ)both

百(ヒャク)hundred

千(セン)thousand

万(マン、バン、よろず)ten thousand

\subsection{Geometry: 5 平半 7 形角 9 点}

平 can mean flat, level (not tilted), ordinary, plain, non-special.
平たい(ひらたい)(adj-i) flat; even; level; simple.

平ら(たいら)flatness.
平皿(ひらざら)flat dish.

半(ハン)half.
半ば(なかば)half; middle; semi.

水平(スイヘイ)level; horizontally.
平面(ヘイメン)level (flat and not-tilted) surface.

平安(ヘイアン)peace; tranquility.
平気(ヘイキ)coolness; calmness; composure; unconcern.
公平(コウヘイ)fairness; impartiality; justice.

平日(ヘイジツ)weekday; ordinary day (non-holiday).
平年(ヘイネン)normal (non-leap) year; normal year (related to harvest; weather).

平文(ヘイブン)plain (non-encrypted) text.

形(ケイ、ギョウ、かたち)shape; form.

角(カク、かど)corner; angle.

角(つの)horn (head protrusion)

点(テン、ぼち)point; spot; speck; mark.

\section{People-related}

\subsection{2 人 3 己大士女 4 犬太父 5 立母兄民 6 共}

人(ジン、レン、ニン、ひと)person; people; human

己(コ、おのれ)self

大 depicts a person with outstretched arms.
大(タイ)big.
大きい(おおきい)big.

士(さむらい)gentleman; samurai.
士(シ)(suffix)
a qualified person;
a person with a qualified profession.
学士(ガクシ)university graduate (who has obtained a degree).
工学士(コウガクシ)Bachelor of Engineering.

女(おんな)woman

犬(いぬ)dog.

太い(タイ、タ、ふとい)(adj-i) fat; thick.
太る(ふとる)to grow fat; to become fat; to gain weight.

父(フ、ちち、とう)father.
父(ちち)(humble) father.
お父さん(おとうさん)(honorific) father.

立 depicts a person standing on ground.
立つ(たつ)(vi) to stand

母 depicts a pair of breasts.
母(ボ、はは、かあ)mother.
母(はは)(humble) mother.
お母さん(おかあさん)(honorific) mother.

兄(ケイ、キョウ、あに)older brother.
兄(あに)(humble) older brother.
お兄さん(おにいさん)(honorific) older brother.

民(ミン、たみ)people

共(とも)companion; follower; attendant; retinue.

\subsection{4 天夫 5 矢失}

天(テン)sky; heaven

夫(フ、おっと)husband

矢(シ、や)arrow

失 depicts something falling from a hand.
失う(うしなう)(vt) to lose; to part with.
失明(シツメイ)loss of eyesight.
失血(シッケツ)loss of blood.

\subsection{2 了 3 子 6 字 7 呆弟男児}

了 depicts a wrapped baby with only head visible.

子 depicts a child with arms visible.
子(シ、こ) child.
男子(ダンシ)young man, boy.
女子(ジョシ)young woman, girl.
太子(タイシ)crown prince.

字(ジ)character, letter (of an alphabet, not the letter that is a document)
赤字(あかジ)deficit; red letter; red Han-character.
漢字(カンジ)Han character.

呆 depicts a child.
呆れる(あきれる)(v1,vi) to be amazed, astonished, astounded.

弟(テイ、ダイ、おとうと)younger brother.
弟(おとうと)(humble) younger brother.
弟さん(おとうとさん)(honorific) younger brother.
兄弟(キョウダイ)siblings;
brothers and sisters
(although the characters mean older brother and younger brother).

男(おとこ)man.

児 depicts an infant with imperfect cranium (fontanelles).
児 is simplified from 兒.
乳児(ニュウジ)infant; suckling baby.
男児(ダンジ)boy; son.

\subsection{Body parts: 2 又 3 口 4 心氏止手毛开 5 目 6 自血耳舌 9 首}

又 depicts the right hand.
又(また)again; also.

口(くち)mouth.

心(シン、こころ)heart.

氏 depicts a man bowing to the left.
氏(シ)(suffix,honorific) Mr.; Mrs.. family. clan.
氏(うじ)family name; birth; lineage.

止 depicts a footprint.
止(シ、と)stop.
止まる(とまる)(vi) to stop (moving); to come to a halt.
止める(やめる)(v1) to stop (doing something).
止める(とめる)(v1) to stop moving (walking, etc.); to park (a car).

手(て)hand.
手がける(てがける)(v1) to make; to do; to produce; to work on.
手洗い(てあらい)restroom; lavatory; toilet; a place for washing hands

毛(モウ、け)hair.

开 is a simplification of 幵 (raising both hands).

目(め)eye

自(ジ、シ、みずか)self; oneself.
自ら(みずから)(adv) personally.
自在(ジザイ)freely (at will).
自分(ジブン)self.
自身(ジシン)self.

血(ち)blood.
止血(シケツ)stop bleeding; hemostasis.

耳(みみ)ear

舌(ゼツ、した)tongue.

首(シュ、くび)neck

\subsection{Blades: 2 刀刂 3 弋 4 戈斤 5 戊 6 戌 9 咸}

刀(トン、かたな)sword.

刂 depicts a knife.

弋 depicts shooting with a bow and an arrow.

戈 depicts a spear-axe (a halberd).

斤 depicts an axe.

戊 depicts a dagger-axe (an ancient Chinese weapon).

戌 depicts an axe.

咸 depicts an axe and a mouth.

\subsection{Administrative divisions: 4 区 5 市 7 町 8 京 9 県}

% https://en.wikipedia.org/wiki/Administrative_divisions_of_Japan

区(ク)ward; district (an administrative division).

市(シ)city (an administrative division).
市(いち)market; fair (trade show).

町(チョウ、まち)town.

京(キョウ、ケイ、みやこ).
京(キョウ)imperial capital.

県(ケン)prefecture.

\subsection{4 予 7 序}

予(ヨ、あらかじ)beforehand.
予め(あらかじめ)beforehand.

序(ジョ)preface

\subsection{4 介}

介(カイ)can mean mediation or shellfish.
紹介(ショウカイ)introduction; referral.
仲介(チュウカイ)agency; intermediation.
介入(カイニュウ)intervention.
介助(カイジョ)help; assistance; aid.
魚介(ギョカイ)marine products; seafood.

\subsection{4 今 6 合会}

今(コン、いま)now.
今回(コンカイ)this time; this occasion; this occurrence.

合(ゴウ、あ)fit.
合う(あう)to fit.

会(カイ、エ、あ)meet.
会う(あう)to meet (face to face).
会社(カイシャ)company; corporation.
会社員(カイシャイン)company employee.
年会(ネンカイ)yearly meeting; annual convention.
社会(シャカイ)society.
会見(カイケン)interview.

\subsection{2 丁 5 庁}

丁(ひのと)depicts a nail.
丁(チョウ)(suffix)
counter for long and narrow (relatively flat) thing
such as paper, guns, scissors, spades, hoes, guitars.
一丁(イッチョウ)one sheet; one page.
one serving (in a restaurant).
丁重(テイチョウ)polite; courteous; hospitable.
包丁(ホウチョウ)kitchen knife.

丁(チョウ)town section.

庁(チョウ)government office.

\subsection{2 力}

力(リョク、リキ、ちから)strength.
百人力(ヒャクジンリキ)tremendous strength.

\subsection{7 貝 9 頁}

貝 depicts a cowry (a kind of seashell used as money in ancient China).
貝(かい)shell; shellfish.

頁 originally depicts a head,
but comes to mean ``leaf'' (an organ in plants).

\subsection{7 図車声}

図(ズ)drawing; map; picture; plan; illustration; diagram; figure; chart.
図る(はかる)to plot; to plan; to design.

車(くるま)car; vehicle. wheel.
水車(スイシャ)water wheel.

声(セイ、こえ)voice.
This character was simplified from 17 聲.

\subsection{8 門 9 重面 10 書 12 傘}

門(モン、かど) gate

重 depicts a man carrying a bag.
重い(おもい)heavy (of weight).

面(おもて)mask.
面白い(おもしろい)interesting.

書 depicts a hand holding a pen writing on paper.
書(ショ)writing.
辞書(ジショ)dictionary.
書く(かく)to write.

傘(かさ)umbrella.


\section{4 中 5 央 6 仲}

中(チュウ、なか)middle.

中学(チュウガク)middle school; junior high school.

…の中(…のなか)middle of something.

田中(たなか)(name) Tanaka.

中山(なかやま)(name) Nakayama.

中川(なかがわ)(name) Nakagawa.

央(オウ)center.
中央(チュウオウ)center; central.

仲(なか)relation; relationship.

仲間(なかま)company; fellow; colleague; associate; comrade; partner.

\section{2 入 5 込}

入(ニュウ)enter.
入る(いる)(v5i,vt)
to enter; to go into; to get into.

込む(こむ)to be crowded.

\section{4 耂 6 考老 7 孝 8 者 11 著}

耂 depicts an old man, a bent-over figure with long hair.

考(コウ)consider.

孝(コウ)filial piety.

考える(かんがえる)(v1,vt) to consider; to think about.

老い(おい)old age; old (of person); at a late time in life.
老人(ロウジン)old person.
老若(ロウニャク)old and young; all ages.

者(シャ)(n,suf) someone of that nature; someone doing that work.

者(もの)(n) person (rarely used without a qualifier).

学者(ガクシャ)scholar.

作者(サクシャ)author.

業者(ギョウシャ)trader; merchant.

研究者(ケンキュウシャ)researcher.

著(チョ)renowned.

著書(チョショ)literary work; book; textbook.

著名人(チョメイジン)celebrity.

\section{5 左右 8 若}

左(ひだり)left.

右(みぎ)right.

若い(わかい)young; at an early time in life.
若年(ジャクネン)the time when one was young.

\section{3 上下 4 内 出 6 肉}

上(うえ)up

下(した)down

内(ナイ、うち)inside.

出 depicts something coming out of an open box.
出る(でる)(v1,vi) to go out; to exit; to leave.
出来る(できる)(v1,vi) to be able to do.
出来上がる(できあがる)(vi) to be finished; to be completed; to be ready

肉 depicts the ribs of an animal's torso.
肉(ニク)meat; flesh; body (as opposed to spirit).

\section{2 丁 5 庁打}

丁(ひのと)depicts a nail.
丁(チョウ)(suffix)
counter for long and narrow (relatively flat) thing
such as paper, guns, scissors, spades, hoes, guitars.
一丁(イッチョウ)one sheet; one page.
one serving (in a restaurant).
丁重(テイチョウ)polite; courteous; hospitable.
包丁(ホウチョウ)kitchen knife.

丁(チョウ)town section.

庁(チョウ)government office.

打(ダ)strike; hit; knock; pound.

安打(アンダ)safe hit (baseball).

打つ(うつ)to hit; to strike; to knock; to beat; to punch; to slap.

\section{3 干 4 午牛 6 汗}

干(カン、ほ)dry.
干す(ほす)(vt) to dry (to make something dry).

午(ゴ、うま)noon.

牛(ギュウ、うし)
cow; bull; ox; buffalo.
beef.

汗(あせ)(n) sweat.

汗をかく(exp,v5k) to sweat.

汗を流す(exp,v5s) to work hard; to sweat.

\section{3 土 5 圧}

土(ド、つち)
soil; earth; ground; dirt; ground (as opposed to the heavens).

圧(アツ)pressure.

気圧(キアツ)air pressure.

指圧(シアツ)finger pressure massage.

\section{Nature}

\subsection{3 川 10 流}

川(かわ)river

流す(ながす)to flow (liquid)

\subsection{3 夕 4 日月 5 白田生 7 麦谷里}

夕(ゆう)evening

日(ひ)sun.
日々(ひび)daily; days; old days.

月(つき)moon.

白(ハク)white.
白い(しろい)white.

田(た)rice field.

生 depicts a sprout, something sprouting from the ground.
生(セイ)nature; sex; gender.
学生(ガクセイ)student.
生まれる(うまれる)(v1,vi) to be born.

麦(バク、むぎ)wheat.

谷(コク、たに)valley.

里(リ、さと)hometown.

\subsection{4 火水开 5 氷永}

火(カ、ひ)fire, flame.
火事(カジ)fire (disaster).
大火(タイカ)big fire.

水(スイ、みず)water.

井 depicts a square well.
井(い)well (water reservoir).

氷(こおり)ice

永(エイ、なが)eternity.
永い(ながい)(adj-i) very long (time).

\subsection{4 木 5 本 6 米休 7 体 8 林 12 森}

木(き)tree

本(ホン)book, counter for long cylindrical objects.
(In Ancient China, a book is a scroll.)
一本(イッポン)one long cylindrical object, one point.
日本(ニホン)Japan.

米(ベイ、マイ、こめ)rice.

休(キュウ)rest.
休日(キュウジツ)day off; holiday.

休む(やすむ)to rest.

体(タイ、からだ)body.

肉体(ニクタイ)body; flesh.

重体(ジュウタイ)seriously ill; critically ill.

林(はやし)woods; forest; copse; thicket

森(もり)forest.
森林(シンリン)forest woods.

\subsection{5 禾 9 秋}

禾 depicts a plant stalk.

秋 depicts the burning of plant stalks (after harvest).

秋(シュウ、あき)autumn; fall season.

\subsection{4 木 5 未末 7 来}

未(ミ)not yet.

未来(ミライ)future (lit. not yet come).

未 depicts a tree that has not fruited.
来 depicts fruits hanging on a tree.
They mean that the time to harvest has not or has come.

末(マツ、すえ)end.

来(ライ)come; next.

来月(ライゲツ)next month.

来年(ライネン)next year.

「来年田中さんが日本に行きます。」Next year, Mr. Tanaka will go to Japan.
(``Next year'' means one year after the moment the speaker says it.)

来る(くる)to come.
This is an irregular verb.
The past form is 来た(きた).

\section{3 山 5 石 6 虫 9 風 11 鳥 : 8 岩 10 島 11 嵐}

山(サン、やま)mount; mountain.

石(セキ、コク、いし)stone.

虫(むし)insect; bug; cricket; moth

風(フウ、かぜ)wind; -like.

鳥(チョウ、とり)bird; chicken; chicken meat.

岩(ガン、いわ)rock.

島(トウ、しま)island.

嵐(あらし)storm.

\section{3 工小亡才幺纟 4 少}

工(コウ、ク)craft.

工学(コウガク)engineering.

大工(ダイク)carpenter.

木工(モッコウ)carpenter.

工(たくみ)(name) Takumi.

小 depicts three sand granules.
小さい(ちいさい)small.
小(ショウ)small.

亡(ボウ)death; destruction; perishment; the deceased.

亡くす(なくす)(vt) to lose something (because he/she/it dies).

マンション\ruby{火}{か}\ruby{災}{さい}、4\ruby{歳}{さい}\ruby{女}{じょ}\ruby{児}{じ}\ruby{死}{し}\ruby{亡}{ぼう} きょうだい2人\ruby{重}{じゅう}\ruby{体}{たい}
Fire in a large apartment: 4-year girl dead, 2 siblings in critical state.

才(サイ)ability; talent.
In China 3 strokes, in Japan 4 strokes.

天才(テンサイ)genius; prodigy.

幺 depicts a tiny/small/short thread.

纟 is the simplified form of 糹
(left form of 6 糸 (thread)).

少(ショウ、すく、すこ)a few; a little.
少ない(すくない)(adj-i) few; little; scarce; limited; insufficient.
少し(すこし)(adv,n) small quantity; few; a little.

\section{3 弓 4 引}

弓(キュウ、ゆみ)bow (archery, violin).

引(イン、ひ)pull.
引用(インヨウ)quotation; citation; reference.
引く(ひく)(vi,vt) to pull.

\section{3 寸 5 付 7 対}

% https://en.wiktionary.org/wiki/%E5%AF%B8#Japanese
寸depicts a position on the forearm
where the pulse can be palpated by compressing the radial artery.
寸(スン)an ancient unit of length, approximately 3 cm.

一寸(ちょっと)just a minute; short time; just a little.

一寸待って下さい。Please wait a moment.

一寸!Hey!

付ける(つける)(v1,vt) to attach, join, stick, glue, fasten.

口付け(くちづけ)(n) kiss.

口付ける(くちづける)(v1) to kiss.

日付(ひづけ)date.

対(タイ)versus...
対する(タイする)to face each other.

\section{3 丸 4 円 6 色}

丸(ガン、まる)circle

円(エン)yen (Japanese currency).
円(まる)circle.

色(ショク、いろ)color.

\section{4 王 8 国 9 美皇}

王(オウ)king

国(コク、くに) country.

国王(コクオウ)king.

国内(コクナイ)internal; domestic.

美(ビ)beauty.

美人(ビジン)beautiful woman.

美しい(うつくしい)beautiful.

美味しい(おいしい)delicious (idiosyncratic reading).

皇(コウ)imperial.

皇居(コウキョ)imperial palace.

\section{5 玉 6 全}

玉(ギョク、たま) ball

全 depicts a whole piece of jade.
全(ゼン)whole.
全部(ゼンブ)altogether; everything.
全国(ゼンコク)countrywide; national.

全く(まったく)(adv) completely, entirely, wholly, totally

\section{5 主 7 住 8 往}

主(おも)chief; main; principal; important.

主人公(シュジンコウ)hero; main character.

住(ジュウ)dwelling; living.

永住(エイジュウ)permanent residence.

居住(キョジュウ)residence.

住人(ジュウニン)inhabitant; resident; dweller.

住む(すむ)to live (of humans); to reside; to inhabit.

日本に住んでいるto be living/residing in Japan.

往(オウ)outward; journey.

\section{4 元之文}

元(ゲン、ガン,もと)origin; beginning; source.
元(もと)origin; source.
元々(もともと)
originally; by nature; from the start; since the beginning.

之(これ)this

文(ブン)sentence (literature).
文字(モジ)letter (of alphabet); character (a Han character)
文書(ブンショ)sentence.
文化(ブンカ)culture.
作文(サクブン)writing.
小学生の作文(ショウガクセイのサクブン)elementary-schoolchild writing.

\section{5 以}

以(イ)since.

\section{5 去 8 法}

去(キョ、コ)leave

法(ホウ)method.

\section{5 令 7 冷}

令(レイ)orders

冷たい(つめたい)(adj-i) cold (of a tangible object; to the touch)

\section{5 号}

号(ゴウ、よびな、さけ)number

\section{5 由 6 因 7 困}

由(よし)cause; reason.

因(イン)cause; factor

因る(よる)(vi) to be caused by.

困(コン)quandary.

困る(こまる)(vi) to be troubled; to be embarrassed.

\section{5 皿用求且}

皿(さら)dish, plate

用 utilize; business; service; employ.
用(ヨウ、もち)use.
用いる(もちいる)(v1,vt) to use; to make use of; to utilize.
常用(ジョウヨウ)(n) common-use; in common use; commonly used.

求める(もとめる)to want

且つ(かつ)and.
且又(かつまた)besides; furthermore; moreover

\section{4 云 6 会伝 7 芸}

云 depicts clouds.

云う(いう)to say; to call; to name.

会(カイ、エ、あ)meet.

会う(あう)to meet (face to face).

会社(カイシャ)company; corporation.

会社員(カイシャイン)company employee.

年会(ネンカイ)yearly meeting; annual convention.

社会(シャカイ)society.

会見(カイケン)interview.

伝(デン)transmit.

自伝(ジデン)autobiography.

手伝う(てつだう)to help; to assist; to take part in.

伝言(デンゴン)verbal message.

伝記(デンキ)life story.

伝道(デンドウ)proselytizing; evangelism; missionary work.

伝える(つたえる)to convey; to report; to transmit; to communicate.

芸(ゲイ)art.

芸能界(ゲイノウカイ)world of show business.

芸能人(ゲイノウジン)performer; a talent in show business.

\section{4 今 6 合}

今(コン、いま)now.
今回(コンカイ)this time; this occasion; this occurrence.

合(ゴウ、あ)fit.
合う(あう)to fit.

\section{5 示申}

示 depicts a spirit.
There are two Unicode codepoints:
⽰ (U+2F70 in the CJK Radicals Supplement block)
and 示 (U+793A in the CJK Unified Ideographs block).
To machines they differ,
but to humans they look the same.

示(シ)indicate.

示す(しめす)to indicate.

申 depicts a bolt of lightning.
申す(もうす)(humble,vt) to say; to speak.

\section{5 世台}

世(よ)world; society; age; generation

台(ダイ、タイ、うてな)tower; stand; pedestal.
台 is simplified form of 14 臺.
仙台(センダイ)(city name) Sendai.

\section{6 共 8 供 9 洪}

共(とも)companion; follower; attendant; retinue.

供える(そなえる)(v1,vt) to offer; to sacrifice; to dedicate.

子供(こども)child.

洪水(コウズイ)(n) flood (of liquid)

大水(おおみず)(n) flood (of liquid)

\section{6 旨 9 指}

旨(むね)center; principle; meaning.

本旨(ホンシ)main object; principal object; true aim.

旨い(うまい)(adj-i) delicious.

指(ゆび)finger.

指す(さす)(vt) to point.

目指す(めざす)(vt) to aim at.

\section{6 争当曲糸糹多竹}

争(ソウ)conflict.

争う(あらそう)
to dispute; to quarrel.
to compete; to contest; to contend.

当(トウ、あ)hit.
当てる(あてる)(v1,vt) to hit.
本当(ホントウ)truth; reality.
本当に(ホントウに)truly; really; seriously.

曲(キョク)music.
作曲(サッキョク)musical composition.

糸 combines 幺 and 小.
糸(いと)thread.

糹 is the left form of 糸.

多 depicts two pieces of meat.
多(ア、おお)many.
多(タ)(prefix) multi-.
多い(おおい)(adj-i) many; numerous.

竹(たけ)bamboo.
竹林(チクリン)bamboo thicket.

\section{6 死光}

死(シ)death.
死ぬ(しぬ)to die.
死亡(シボウ)death; mortality.
死去(シキョ)death.

光(コウ、ひかり)light (electromagnetic wave)

衣 depicts a cloth.
衣(ころも)
clothes;
garment. gown;
robe. coating (glaze; batter; icing).

\section{6 衣 8 表}

表(ヒョウ)express.

表示(ヒョウジ)manifestation; demonstration. display. representation.

\section{5 可司 6 同回 7 何 8 河 10 哥高}

可(カ)passable; acceptable; tolerable.

許可(キョカ)permission; authorization; approval.

司(シ)director.

同(ドウ、おな)same.

同じ(おなじ)same.

同性愛(ドウセイアイ)same-sex love.

回 depicts a spiral.

回す(まわす)(vt) to turn; to rotate.

今回(コンカイ)this time.

一回(イッカイ)once; one time.

一回目(イッカイめ)first.

何(カ、なに、なん)what.

河(かわ)river; stream.

河川(カセン)rivers.

大河(タイガ)large river.

哥 means older brother.

高(コウ)high.

高い(たかい)high; expensive.

\section{6 艮 7 良 8 長 9 食 12 飲}

艮 depicts eye and spoon.

良(リョウ)good.

良い(いい)good.

良く(よく)well.

長(チョウ)
long (distance or time).
leader.
eldest.

長い(ながい)long (distance); long (time).

長女(チョウジョ)eldest daughter; first-born daughter.

市長(シチョウ)mayor (a government official).

身長(シンチョウ)height (of body).

最長(サイチョウ)longest, tallest.

社長(シャチョウ)company president.

食(ショク)eat.

食物(ショクもの)food.

食べ物(たべもの)food.

食べる(たべる)(v1) to eat.

食う(くう)(male, vulgar) to eat.

飲む(のむ)to drink (any liquid, not just liquor)

\section{5 母 7 毎 9 海}

母 depicts a pair of breasts.
母(ボ、はは、かあ)mother.
母(はは)(humble) mother.
お母さん(おかあさん)(honorific) mother.

毎(マイ、ごと)every.
毎日(マイニチ)everyday.
毎月(マイゲツ、マイつき)every month.
毎時(マイジ)every hour.
毎回(マイカイ)every time (every time it happens); every occurrence.
毎年(マイネン、マイとし)every year.

海(カイ、うみ)sea; beach.
The original character has 10 strokes.
Shinjitai replaces the two dots in the middle
with one vertical stroke.

\section{7 辛 13 新}

辛 depicts a tool used to mark slaves and criminals;
this sometimes also depicts a tree.

辛い(からい)(adj-i) spicy, salty, harsh, hot, acrid.

辛い(つらい)(adj-i) bitter; painful; heartbreaking; difficult.
Suffix づらい(adj-i) means ``difficult to do''.

読みづらい(adj-i) difficult to read.

書きづらい(adj-i) difficult to write.

読みづらい漢字difficult-to-read Han character.

新 depicts cutting tree down with axe.
新(シン)new.
新しい(あたらしい)new.
新聞(シンブン)news.
新車(シンシャ)new car.

\section{7 赤}

赤い(あかい)(adj-i) red.

\section{10 馬 14 駆駅}

馬(バ、うま、ま)horse.

駆(ク)drive.
This was simplified from 驅.

駆ける(かける)(v1,vi) to run (horse).

駅 is simplified from 23 驛.
駅(エキ)train station.

\section{6 羽 7 卵 9 飛 10 弱 11 習}

羽(ウ、は、はね)feather

卵(ラン、たまご)egg.

飛(ヒ)fly; jump.

飛ぶ(とぶ)to jump.

弱(ジャク)weak.

弱い(よわい)weak.

練習(レンシュウ)training; practice.

習う(ならう)(vt) to learn.

Note that the bottom component of 習 is 自, not 日.

\section{Animals}

\subsection{10 烏 11 魚猫豚}

魚介(ギョカイ)seafood; marine products

動物(ドウブツ)animals.

動物園(ドウブツエン)zoo (lit. animal garden)

烏(からす)crow; raven.
More often written with katakana as カラス instead of with kanji.
烏羽色(からすばいろ)glossy black; crow feather color.

魚(ギョ、うお、さかな)fish.
金魚(キンギョ)goldfish.

猫(ねこ)cat.

豚(ぶた)pig.

\section{Negation: 4 不 12 無}

不(フ)(prefix) not; bad; poor.
不安(フアン)anxiety; insecurity.
不明(フメイ)unknown; obscure; anonymous; unidentified.

無(ム) no, -less, without.
無駄(ムダ)uselessness.
無用(ムヨウ)uselessness.
無敵(ムテキ)invincible, unrivaled (lit. no-enemy).
無茶(ムチャ)absurd, unreasonable (lit. no-tea).
無人(ムジン)unmanned (lit. no-human).
無言(ムゴン)silence (lit. no-say).
