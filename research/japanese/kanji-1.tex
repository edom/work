\chapter{Kanji 1}

Kanji is recursive: it may be made of other kanji.

The kanji in this chapter should be considered irreducible.
They cannot be broken apart into subkanji.

\section{Non-standalone parts}

\subsection{2 厶冂亻冫丩卜 3 彳㔾}

These parts rarely stand on their own.

厶 depicts I, myself, private.

冂 depicts an inverted box.

亻 is the side form of 人 (person).

冫 is the left form of 氷 (ice).

丩 depicts entanglement; intertwining.

卜 depicts a crack.

彳 depicts stepping or walking slowly.

What 㔾 depicts is not known.

\subsection{Mound and city: 3 阝 7 邑 8 阜}

阝 has 2 strokes in Chinese and 3 strokes in Japanese.
On the left side, 阝 is 阜 (mound).
On the right side, 阝 is 邑 (city; village).

\subsection{3 宀辶艹忄氵}

宀 depicts a roof.

辶 is the combining form of 辵 (walking).
Chinese dictionaries say this has 3 strokes.
Japanese dictionaries say this has 4 strokes,
but also say that 近 has 7 strokes.
If the Japanese dictionaries are to be followed,
then 近 should have 8 strokes.

艹 is the top form of 艸 (grass).

忄 is the left form of 心 (heart).

氵 is the left form of 水 (water).

\subsection{4 灬礻爪爫 5 衤}

灬 is the bottom form of 火 (fire).

礻 is the left form of 示 (spirit).

衤 is the left form of 衣 (cloth).

爪 depicts claw.

爫 is the top form of 爪.

\subsection{Disease: 5 疒}

疒depicts disease.

\subsection{6 艸}

艸 depicts grass.

\section{Numbers-related}

\subsection{1 一 2 二十八七九 3 三千万 4 五六 5 四 6 両百}

一(イチ、ひと)one

二(ニ、ふた)two

三(サン、み)three

四 four

五 five

六 six

七 seven

八 eight

九 nine

十 ten

両(リョウ)both

百(ヒャク)hundred

千(セン)thousand

万(マン、バン、よろず)ten thousand

\subsection{Geometry: 5 平半 7 角 9 点}

平 can mean flat, level (not tilted), ordinary, plain, non-special.
平たい(ひらたい)(adj-i) flat; even; level; simple.

平ら(たいら)flatness.
平皿(ひらざら)flat dish.

半(ハン)half.
半ば(なかば)half; middle; semi.

水平(スイヘイ)level; horizontally.
平面(ヘイメン)level (flat and not-tilted) surface.

平安(ヘイアン)peace; tranquility.
平気(ヘイキ)coolness; calmness; composure; unconcern.
公平(コウヘイ)fairness; impartiality; justice.

平日(ヘイジツ)weekday; ordinary day (non-holiday).
平年(ヘイネン)normal (non-leap) year; normal year (related to harvest; weather).

平文(ヘイブン)plain (non-encrypted) text.

角(カク、かど)corner; angle.

角(つの)horn (head protrusion)

点(テン、ぼち)point; spot; speck; mark.

\section{People-related}

\subsection{7 弟児}

弟(テイ、ダイ、おとうと)younger brother.
弟(おとうと)(humble) younger brother.
弟さん(おとうとさん)(honorific) younger brother.
兄弟(キョウダイ)siblings;
brothers and sisters
(although the characters mean older brother and younger brother).

児 depicts an infant with imperfect cranium (fontanelles).
児 is simplified from 兒.
乳児(ニュウジ)infant; suckling baby.
男児(ダンジ)boy; son.

\subsection{Body parts: 4 心 6 自血}

心(シン、こころ)heart.

自(ジ、シ、みずか)self; oneself.
自ら(みずから)(adv) personally.
自在(ジザイ)freely (at will).
自分(ジブン)self.
自身(ジシン)self.

血(ち)blood.
止血(シケツ)stop bleeding; hemostasis.

\subsection{Administrative divisions: 5 市 9 県}

% https://en.wikipedia.org/wiki/Administrative_divisions_of_Japan

市(シ)city (an administrative division).

市(いち)market; fair (trade show).

県(ケン)prefecture.


\section{Other animals: 10 烏 11 魚猫}

魚介(ギョカイ)seafood; marine products

動物(ドウブツ)animals.

動物園(ドウブツエン)zoo (lit. animal garden)

烏(からす)crow; raven.
More often written with katakana as カラス instead of with kanji.
烏羽色(からすばいろ)glossy black; crow feather color.

魚(ギョ、うお、さかな)fish.
金魚(キンギョ)goldfish.

猫(ねこ)cat.

\section{4 予 7 序}

予(ヨ、あらかじ)beforehand.

予め(あらかじめ)beforehand.

序(ジョ)preface

\section{7 図声}

図(ズ)drawing; map; picture; plan; illustration; diagram; figure; chart.
図る(はかる)to plot; to plan; to design.

声(セイ、こえ)voice.
This character was simplified from 17 聲.

\section{8 門 9 面 10 書 12 傘}

門(モン、かど) gate

面(おもて)mask.
面白い(おもしろい)interesting.

書 depicts a hand holding a pen writing on paper.
書(ショ)writing.
辞書(ジショ)dictionary.
書く(かく)to write.

傘(かさ)umbrella.
