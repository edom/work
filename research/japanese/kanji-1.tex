\chapter{Kanji 1}

\section{1 一 2 二 3 三 ... 四五六七八九十}

一(イチ、ひと)one

二(ニ、ふた)two

三(サン、み)three

四 four

五 five

六 six

七 seven

八 eight

九 nine

十 ten

\section{2 人 3 大 4 夫天失太犬 5 立}

人 depicts a person.
人(ジン、レン、ニン、ひと)person; people; human

大 depicts a person with outstretched arms.
大(タイ)big.
大きい(おおきい)big.

天(テン) sky; heaven

夫(フ、おっと)husband

失 depicts something falling from a hand.
失う(うしなう)(vt) to lose; to part with.
失明(シツメイ)loss of eyesight.
失血(シッケツ)loss of blood.

太(タイ) thick.
太い(ふとい)fat.

犬(いぬ)dog.
The kanji did not come
from the drawing of a person.

立 depicts a person standing on ground.
立つ(たつ)(vi) to stand

\section{2 入 4 内 6 肉}

入る(いる)(v5)
to enter; to go into; to get into.
Do not confuse this with 人(person).

内(うち)inside

肉 (ribs of an animal's torso)

肉(ニク)meat; flesh; body (as opposed to spirit)

\section{2 刀力 3 亡万土士 4 区之 6 虫多}

刀(トン、かたな)sword.

力(リョク、ちから)strength

亡 means death; destruction; perishment; the deceased.
亡くす(なくす)(vt) to lose something (because he/she/it dies).

万(マン)ten thousand

土(ド、つち) earth; soil

士(さむらい)samurai.
士(シ)(suffix)
a qualified person;
a person with a qualified profession.
学士(ガクシ)university graduate (who has obtained a degree).
工学士(コウガクシ)Bachelor of Engineering.

Do not confuse 士(samurai) and 土(earth).
士(samurai) has longer upper horizontal stroke.
土(earth) has shorter upper horizontal stroke.

区(ク)ward; district (an administrative region).

之(これ)this

虫(むし)insect; bug; cricket; moth

多 depicts two pieces of meat.
多 means ``many''.
多(タ)(prefix) multi-.

\section{2 了 3 子 6 字 8 学}

了depicts a wrapped baby with only head visible.

子is了 with arms visible.
3 子(シ、こ) child.

6 字(ジ)character, letter (of an alphabet, not the letter that is a document)

8 学(ガク)learning, scholarship, erudition, knowledge.
中学(チュウガク)middle school; junior high school.
大学(ダイガク)university.
学界(ガッカイ)academic world.

\section{3 口工山川上下女 4 中不父止心手}

口(くち) mouth

女(おんな)woman

川(かわ)river

山(サン、やま)mount, mountain

上(うえ)up

下(した)down

工(コウ、ク)craft.
工学(コウガク)engineering.
大工(ダイク)carpenter.
木工(モッコウ)carpenter.
工(たくみ)(name) Takumi.

心(シン、こころ)heart

父(ちち)the inner group's father.
お父さん(おとうさん)(polite) the outer group's father.

手(て)hand.
手がける(てがける)(v1) to make; to do; to produce; to work on.

中(チュウ、なか)middle.
中学(チュウガク)middle school; junior high school.
…の中(…のなか)middle of something.
田中(たなか)(name) Tanaka.
中山(なかやま)(name) Nakayama.
中川(なかがわ)(name) Nakagawa.

不(フ)(prefix) not; bad; poor.
不安(フアン)anxiety; insecurity.
不明(フメイ)unknown; obscure; anonymous; unidentified.

止depicts a footprint.

止める(とめる)(v1) to stop moving (walking, etc.); to park (a car)

\section{3 小幺 4 少 6 糸}

小 depicts three sand granules.
小さい(ちいさい)small.
小(ショウ)small.

少し(すこし)(adv, n) small quantity; few; a little

幺 depicts a tiny/small/short thread.

糸 combines 幺 and 小.
糸(いと)thread.

\section{3 千 4 牛 5 平半}

千(セン)thousand

牛(ギュウ、うし)
cow; bull; ox; buffalo.
beef.

平 can mean flat, level (not tilted), ordinary, plain, non-special.
平ら(たいら)flatness.
平たい(ひらたい)(adj-i) flat; even; level; simple.
平皿(ひらざら)flat dish.
平安(ヘイアン)peace; tranquility.
平気(ヘイキ)coolness; calmness; composure; unconcern.
平日(ヘイジツ)weekday; ordinary day (non-holiday).
平年(ヘイネン)normal (non-leap) year; normal year (related to harvest; weather).
公平(コウヘイ)fairness; impartiality; justice
水平(スイヘイ)level; horizontally
平文(ヘイブン)plain (non-encrypted) text.
平面(ヘイメン)level (flat and not-tilted) surface.

半(ハン)half

\section{4 日月 5 白 6 百 8 明}

日(ひ)sun.
日々(ひび)daily; days; old days.

月(つき)moon.

白(ハク)white.
白い(しろい)white.

百(ヒャク)hundred

明(メイ)bright.
明るい(あかるい)bright.

\section{4 王 5 玉生 8 国}

4 王(オウ)king

5 玉(ギョク、たま) ball

生depicts a sprout, something sprouting from the ground.
It has no relationship with 王.
Don't confuse them.
They only look similar.
生(セイ)nature; sex; gender.
学生(ガクセイ)student.
生まれる(うまれる)(v1,vi) to be born.

8 国(コク、くに) country.
国王(コクオウ)king.
国内(コクナイ)internal; domestic.

\section{4 井开氏}

井 depicts a square well.
井(い)well (water reservoir).

开 is a simplification of 幵 depicting raising both hands.

氏 depicts a man bowing to the left.
氏(シ)(suffix,honorific) Mr.; Mrs.. family. clan.
氏(うじ)family name; birth; lineage.

\section{5 田申由出母石左右 6 回向}

田(た)rice field

申 depicts a bolt of lightning.
申す(もうす)(humble,vt) to say; to speak.

由(よし)cause; reason.

母depicts a pair of breasts.
母(はは)mother

石(いし)stone

出 depicts something coming out of an open box.
出る(でる)(v1,vi) to go out; to exit; to leave.
出来る(できる)(v1,vi) to be able to do.
出来上がる(できあがる)(vi) to be finished; to be completed; to be ready

左(ひだり)left

右(みぎ)right.
Don't confuse this with 石(stone).

回 depicts a spiral.
回す(まわす)(vt) to turn; to rotate.
一回(イッカイ)once; one time.
一回目(イッカイめ)first.

向 depicts a house and a window.
向かい(むかい)(n) facing; opposite; across the street; other side.
向く(むく)to face; to turn toward.
向こう(むこう)opposite side; other side; opposite direction.
向上(コウジョウ)improvement; advancement; progress.

\section{5 目皿 6 血耳 7 貝見 12 覚}

目(め)eye

皿(さら)dish, plate

血(ち)blood

耳(みみ)ear

貝 depicts a cowry (a kind of seashell used as money in ancient China).
貝 is not related to 目.
貝(かい)shell; shellfish.

見る(みる)(v1,vt) to see.
見える(みえる)(v1,vi) to appear.

覚める(さめる)(v1,vi) to wake up

\section{6 両}

両(リョウ)both

\section{7 車 8 雨店 9 面重}

車(くるま)car

雨(あめ)rain

店(テン)(n) store; shop.
ラメン店ramen shop (ramen is a kind of Japanese noodle).

面(おもて)mask

面白い(おもしろい)interesting

重 depicts a man carrying a bag.

重い(おもい)heavy (of weight)

\section{Others: 3 夂夊 4 攵 8 㑒}

In the Japanese language,
these characters become parts of other characters
instead of being used on their own.

夂 depicts two legs followed by something from behind.

夊 depicts a footprint.

攵 is a variant of 攴 depicting a branch and a hand.

㑒 is simplified from the 13-stroke 僉
meaning ``all, together, unanimous''.
