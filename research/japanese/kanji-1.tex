\chapter{Kanji 1}

Kanji is recursive: it may be made of other kanji.

\section{Non-standalone parts}

\subsection{2 冂亻冫 3 宀辶艹忄氵 4 灬殳礻 5 衤}

These parts rarely stand on their own.

冂 depicts an inverted box.

亻 is the side form of 人.

冫 is the left form of 氷 (ice).

宀 depicts a roof.

辶 is the combining form of 辵 meaning ``walking''.
Chinese dictionaries say this has 3 strokes.
Japanese dictionaries say this has 4 strokes,
but also say that 近 has 7 strokes.
If the Japanese dictionaries are to be followed,
then 近 should have 8 strokes.

艹 is the top form of 艸 (grass).

忄 is the left form of 心 (heart).

氵 is the left form of 水 (water).

灬 is the bottom form of 火.

殳 depicts a hand holding a tool or a weapon.

礻 is the left form of ⽰ (spirit).

衤 is the left form of 衣 (cloth).

\subsection{3 㔾}

What 㔾 depicts is not known.

\subsection{6 艸耂艮 8 隹 10 哥}

艸 depicts grass.

耂 depicts a bent-over figure with long hair; an old man.

艮 depicts stopping; eye and spoon.

隹 depicts a short-tailed bird.

哥 means older brother.

\section{2 人又了丁力厶入}

人(ジン、レン、ニン、ひと)person; people; human

又 depicts the right hand.
又(また)again; also.

了 depicts a wrapped baby with only head visible.

丁 depicts a nail.

丁(ひのと)

丁(チョウ)(suffix)
counter for long and narrow (relatively flat) thing
such as paper, guns, scissors, spades, hoes, guitars.
一丁(イッチョウ)one sheet; one page.
one serving (in a restaurant).
丁重(テイチョウ)polite; courteous; hospitable.
包丁(ホウチョウ)kitchen knife.

力(リョク、リキ、ちから)strength.
百人力(ヒャクジンリキ)tremendous strength.

厶 means I, myself, private.

入る(いる)(v5)
to enter; to go into; to get into.

\section{3 口子士女川山土上下大小亡工夕才寸幺纟弓}

口(くち)mouth.

子 depicts a child with arms visible.
子(シ、こ) child.
男子(ダンシ)young man, boy.
女子(ジョシ)young woman, girl.
太子(タイシ)crown prince.

士(さむらい)gentleman; samurai.
士(シ)(suffix)
a qualified person;
a person with a qualified profession.
学士(ガクシ)university graduate (who has obtained a degree).
工学士(コウガクシ)Bachelor of Engineering.

女(おんな)woman

川(かわ)river

山(サン、やま)mount, mountain

土(ド、つち)
soil; earth; ground; dirt; ground (as opposed to the heavens).

上(うえ)up

下(した)down

大 depicts a person with outstretched arms.
大(タイ)big.
大きい(おおきい)big.

小 depicts three sand granules.
小さい(ちいさい)small.
小(ショウ)small.

亡 means death; destruction; perishment; the deceased.
亡くす(なくす)(vt) to lose something (because he/she/it dies).

工(コウ、ク)craft.
工学(コウガク)engineering.
大工(ダイク)carpenter.
木工(モッコウ)carpenter.
工(たくみ)(name) Takumi.

夕(ゆう)evening

才 talent.
In China 3 strokes, in Japan 4 strokes.
天才(テンサイ)genius; prodigy.

% https://en.wiktionary.org/wiki/%E5%AF%B8#Japanese
寸depicts a position on the forearm
where the pulse can be palpated by compressing the radial artery.
寸(スン)an ancient unit of length, approximately 3 cm.

幺 depicts a tiny/small/short thread.

纟 is the simplified form of 糹
(left form of 6 糸 (thread)).

弓(キュウ、ゆみ)bow (archery, violin).

\section{4 円父开井火水斤心中内}

円(エン)yen (Japanese currency).
円(まる)circle.

父(ちち)(humble) father.
お父さん(おとうさん)(honorific) father.

开 is a simplification of 幵 (raising both hands).
幵 raising both hands.

井 depicts a square well.
井(い)well (water reservoir).

火(カ、ひ)fire, flame.
火事(カジ)fire (disaster).
大火(タイカ)big fire.

水(スイ、みず)water.
水車(スイシャ)water wheel.

斤 depicts an axe.

心 depicts a heart.

中(チュウ、なか)middle.
中学(チュウガク)middle school; junior high school.
…の中(…のなか)middle of something.
田中(たなか)(name) Tanaka.
中山(なかやま)(name) Nakayama.
中川(なかがわ)(name) Nakagawa.

内(うち)inside

\section{4 王天介太少止手}

王(オウ)king

天(テン)sky; heaven

介 can mean mediation.
紹介(ショウカイ)introduction; referral.
仲介(チュウカイ)agency; intermediation.
介入(カイニュウ)intervention.
介助(カイジョ)help; assistance; aid.

介 can mean shellfish.
魚介(ギョカイ)marine products; seafood.

太い(タイ、タ、ふとい)(adj-i) fat; thick.
太る(ふとる)to grow fat; to become fat; to gain weight.

少し(すこし)(adv, n) small quantity; few; a little

止 depicts a footprint.
止める(やめる)(v1) to stop (doing something).
止める(とめる)(v1) to stop moving (walking, etc.); to park (a car).

手(て)hand.
手がける(てがける)(v1) to make; to do; to produce; to work on.
手洗い(てあらい)restroom; lavatory; toilet; a place for washing hands

\section{4 今元戸之氏文夫}

今(いま)now

元 beginning; origin.
元(もと)origin; source.
元々(もともと)
originally; by nature; from the start; since the beginning.

戸(と)door.

之(これ)this

氏 depicts a man bowing to the left.
氏(シ)(suffix,honorific) Mr.; Mrs.. family. clan.
氏(うじ)family name; birth; lineage.

文(ブン)sentence (literature).
文字(モジ)letter (of alphabet); character (a Han character)
文書(ブンショ)sentence.
文化(ブンカ)culture.
作文(サクブン)writing.
小学生の作文(ショウガクセイのサクブン)elementary-schoolchild writing.

夫(フ、おっと)husband

\section{4 木}

木(き)tree

\section{5 皿用求业矢史且}

皿(さら)dish, plate

用 utilize; business; service; employ.
用いる(もちいる)(v1,vt) to use; to make use of; to utilize.
常用(ジョウヨウ)(n) common-use; in common use; commonly used.

求める(もとめる)to want

业 alternative form of 北 (north).

矢(や)arrow

史家(シカ)historian.

且つ(かつ)and.
且又(かつまた)besides; furthermore; moreover

\section{5 非⽰失申主玉生禾本}

非(ヒ)un-; non-; negative; mistake; wrong.
非ず(あらず)(exp) no; never mind.

⽰ depicts a spirit.

失 depicts something falling from a hand.
失う(うしなう)(vt) to lose; to part with.
失明(シツメイ)loss of eyesight.
失血(シッケツ)loss of blood.

申 depicts a bolt of lightning.
申す(もうす)(humble,vt) to say; to speak.

主(おも)chief; main; principal; important.
主人公(シュジンコウ)hero; main character.

玉(ギョク、たま) ball

生depicts a sprout, something sprouting from the ground.
生(セイ)nature; sex; gender.
学生(ガクセイ)student.
生まれる(うまれる)(v1,vi) to be born.

禾 depicts a plant stalk.

本(ホン)book, counter for long cylindrical objects.
(In Ancient China, a book is a scroll.)
一本(イッポン)one long cylindrical object, one point.
日本(ニホン)Japan.

\section{5 左右央出召皮}

左(ひだり)left

右(みぎ)right.

央(オウ)center.
中央(チュウオウ)center; central.

出 depicts something coming out of an open box.
出る(でる)(v1,vi) to go out; to exit; to leave.
出来る(できる)(v1,vi) to be able to do.
出来上がる(できあがる)(vi) to be finished; to be completed; to be ready

召す(めす)(honorific) to invite; to eat

皮(かわ)skin; hide.

\section{5 母兄目立田世石氷台}

母 depicts a pair of breasts.
母(はは)mother.
お母さん(おかあさん)(honorific)

兄(あに)(humble) older brother.
お兄さん(おにいさん)(honorific) older brother.

目(め)eye

立 depicts a person standing on ground.
立つ(たつ)(vi) to stand

田(た)rice field.

世(よ)world; society; age; generation

石(いし)stone

氷(こおり)ice

台(うてな)tower; stand; pedestal.
台 is simplified form of 14 臺.
仙台(センダイ)(city name) Sendai.

\section{6 合会当曲羽字糸糹多}

合う(あう)to fit

会う(あう)to meet (face to face).
会社(カイシャ)company; corporation.
会社員(カイシャイン)company employee.
年会(ネンカイ)yearly meeting; annual convention.
社会(シャカイ)society.
会見(カイケン)interview.

当てる(あてる)(v1,vt) to hit.
本当(ホントウ)truth; reality.
本当に(ホントウに)truly; really; seriously.

曲(キョク)music.
作曲(サッキョク)musical composition.

羽(はね)feather

字(ジ)character, letter (of an alphabet, not the letter that is a document)
赤字(あかジ)deficit; red letter; red Han-character.
漢字(カンジ)Han character.

糸 combines 幺 and 小.
糸(いと)thread.

糹 is the left form of 糸.

多 depicts two pieces of meat.
多 means ``many''.
多(タ)(prefix) multi-.
多い(おおい)(adj-i) many; numerous.

\section{6 血耳舌肉死衣光}

血(ち)blood.
止血(シケツ)stop bleeding; hemostasis.

耳(みみ)ear

舌(した)tongue.

肉 depicts the ribs of an animal's torso.
肉(ニク)meat; flesh; body (as opposed to spirit).

死(シ)death.
死ぬ(しぬ)to die.
死亡(シボウ)death; mortality.
死去(シキョ)death.

衣 depicts a cloth.
衣(ころも)
clothes;
garment. gown;
robe. coating (glaze; batter; icing).

光(コウ、ひかり)light (electromagnetic wave)

\section{6 共米休}

共(とも)companion; follower; attendant; retinue.

米(ベイ、こめ)rice.

休む(やすむ)to rest

\section{7 呆貝車弟男児}

呆 depicts a child.
呆れる(あきれる)(v1,vi) to be amazed, astonished, astounded.

貝 depicts a cowry (a kind of seashell used as money in ancient China).
貝(かい)shell; shellfish.

車(くるま)car; vehicle. wheel.

弟(おとうと)(humble) younger brother.
弟さん(おとうとさん)(honorific) younger brother.
兄弟(キョウダイ)siblings;
brothers and sisters
(although the characters mean older brother and younger brother).

男(おとこ)man.

児 depicts an infant with imperfect cranium (fontanelles).
児 is simplified from 兒.
乳児(ニュウジ)infant; suckling baby.
男児(ダンジ)boy; son.

\section{7 声辛赤体}

声(こえ)voice.
This character was simplified from 17 聲.

辛 depicts a tool used to mark slaves and criminals;
this sometimes also depicts a tree.
辛い(からい)(adj-i) spicy, salty, harsh, hot, acrid.
辛い(つらい)(adj-i) bitter; painful; heartbreaking; difficult.
Suffix づらい(adj-i) means ``difficult to do''.
読みづらい(adj-i) difficult to read.
書きづらい(adj-i) difficult to write.
読みづらい漢字difficult-to-read Han character.

赤い(あかい)(adj-i) red.

体(タイ、からだ)body.
肉体(ニクタイ)body; flesh.

\section{8 門雨林}

門(かど) gate

雨(あめ)rain

林(はやし)woods; forest; copse; thicket

\section{9 首重面食}

首(くび)neck

重 depicts a man carrying a bag.
重い(おもい)heavy (of weight).

面(おもて)mask.
面白い(おもしろい)interesting.

食べ物(たべもの)food.
食物(ショクもの)food.
食べる(たべる)(v1) to eat.

\section{10 書 12 傘森}

書 depicts a hand holding a pen writing on paper.
書(ショ)writing.
辞書(ジショ)dictionary.
書く(かく)to write.

傘(かさ)umbrella.

森(もり)forest.
森林(シンリン)forest woods.

\section{Blades: 2 刀刂 3 弋 4 戈 5 戊 6 戌 9 咸}

刀(トン、かたな)sword.

刂 depicts a knife.

弋 depicts shooting with a bow and an arrow.

戈 depicts a spear-axe (a halberd).

戊 depicts a dagger-axe (an ancient Chinese weapon).

戌 depicts an axe.

咸 depicts an axe and a mouth.

\section{Administrative regions: 4 区 5 市 8 京}

区(ク)ward; district (an administrative area).

市(シ)city (an administrative area).
市(いち)market; fair (trade show).

京(キョウ)imperial capital

\section{Mound and city: 3 阝 7 邑 8 阜}

阝 has 2 strokes in Chinese and 3 strokes in Japanese.
On the left side, 阝 is 阜 (mound).
On the right side, 阝 is 邑 (city; village).

\section{Animals: 4 犬牛 6 虫 10 馬 11 魚猫豚鳥}

魚介(ギョカイ)seafood; marine products

動物(ドウブツ)animals.

犬(いぬ)dog.

牛(ギュウ、うし)
cow; bull; ox; buffalo.
beef.

虫(むし)insect; bug; cricket; moth

動物園(ドウブツエン)zoo (lit. animal garden)

馬(うま)horse.

鳥(とり)bird; chicken; chicken meat.
焼き鳥(やきとり)grilled chicken meat.

魚(さかな)fish.

猫(ねこ)cat.

豚(ぶた)pig.

\section{Not-yet and next: 5 未 7 来}

未(ミ)not yet.
未来(ミライ)future (lit. not yet come).
未 depicts a tree that has not fruited.
来 depicts fruits hanging on a tree.
They mean that the time to harvest has not or has come.

来(ライ)come; next.
来月(ライゲツ)next month.
来年(ライネン)next year.
「来年田中さんが日本に行きます。」Next year, Mr. Tanaka will go to Japan.
(``Next year'' means one year after the moment the speaker says it.)

来る(くる)to come.
This is an irregular verb.
The past form is 来た(きた).
