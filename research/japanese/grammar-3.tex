\chapter{Grammar 3}

All truth values are to be interpreted probabilistically.
The statement ``everybody has a chicken'' is neither true nor false;
it has truth value somewhere between 0 and 1.

Probabilistic temporal modal logic?

An adjectival is a thing that modifies a nominal.

If 「笑った。」 is true, then 「笑う。」 is false,
but there exists a past time interval where 「笑う。」 is true.

The negative form is an i-form, so the i-to-katta rule perfects it:
笑わない becomes 笑わなかった.

笑わなかった。The implied entity did not laugh.

Iff 「笑わない。」, then not 「笑う。」.

Iff 「笑わなかった。」, then 「笑わない。」.

Iff 「笑わなかった。」, then not 「笑った。」.

\subsection{In doubt: Nominalization: の}

犬(いぬ)、来る(くる)、来た(きた):

田中さんは来た。Tanaka-san came.
\[
    \vdash 来た(田中さん)
\]
About Tanaka-san, he came.
What Tanaka-san did was coming.

来たのは田中さんです。
About having come,
it is Tanaka-san who did that (and not somebody else).
\[
    来た(x) \vdash x = 田中さん
\]
If anyone came, then it was Tanaka-san.
Who came is Tanaka-san.
Tanaka-san is the person who came (and not somebody else).

勉強(ベンキョウ)、貴方(あなた):
勉強するのは貴方にいい。
Studyingはyouにgood。
Studying is good for you.
Parse tree:
貴方にいいmodifiesの,
貴方にいいのis the topic,
貴方にいいのはmodifiesいい.
Logic: good-for(studying,you).

貴方にいいのは勉強するのです。
What is good for youはstudyingです。
What is good for you is studying.
Parse tree:((貴方)に・いい・の)は・勉強・する・の・です。
Another possible parse tree:
(貴方)に・(いい・の)は・勉強・する・の・です。
For you, what is good is studying.

勉強するのは何のため?
What is studying for?
What is the purpose of studying?

\section{Example constructions: たい}

魚(さかな)、食べる(たべる):魚を食べたくありません。The implied entity does not want to eat fish.

\section{Questionable}

良い(いい)、良くなかった(よくなかった):
良くなかった良い事things that was good that was not good.

Does AのBとC parse as Aの(BとC) or (AのB)とC?

\section{In doubt}

If X is a clause, then Xの is a nominal. (?)
Or is this a のは particle?
Example:
〈魚を食べる・の〉は・田中さんです。
or
〈魚を食べる〉のは・田中さんです。
?

\section{Adjectival}

If X is an i-adjective, then X is an adjectival.

If X is an no-adjectival, then X is an adjectival.

\subsection{I-adjectival}

\subsection{Verbal adjectival}

A verb by itself readily forms an adjectival.

笑う

笑った

\subsection{No-adjectival}

If X is a nominal, then Xの is a no-adjectival.

\section{Predicate}

A predicate is a nominal, an i-adjectival, or a verbal.

\section{Clause}

A clause is an adjectival.

食べる(たべる)by itself can be the main clause
``The implied entity eats.'' or the relative clause ``who eats''.
「食べる。」The implied entity will eat.
「食べる人」The person who eats.

\section{What?}

Phrases.

形容詞(ケイヨウシ)

i-adjective

連用形(レンヨウケイ)

continuative form?

How verbs change forms.

Conjugation is inflection of verb.
Inflection is a change of form that does not change syntactic category.
Derivation is a change of form that changes syntactic category.
