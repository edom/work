\chapter{Kanji 5}

\section{5 市}

市(シ)city (an administrative division).

市(いち)market; fair (trade show).

\section{5 矛务}

矛 Kangxi radical 110 (``spear'').

\subsection{11 務}

務(ム)duty.

事務所(ジムショ)office.

\section{5 乎平半 8 呼}

乎 question mark.

平(ヘイ)flat.

平たい(ひらたい)(adj-i) flat; even; level; simple.

平ら(たいら)flatness.

平皿(ひらざら)flat dish.

水平(スイヘイ)level; horizontally.

平面(ヘイメン)level (flat and not-tilted) surface.

平安(ヘイアン)peace; tranquility.

平気(ヘイキ)coolness; calmness; composure; unconcern.

公平(コウヘイ)fairness; impartiality; justice.

平日(ヘイジツ)weekday; ordinary day (non-holiday).

平年(ヘイネン)normal (non-leap) year; normal year (related to harvest; weather).

平文(ヘイブン)plain (non-encrypted) text.

半(ハン)half.

半ば(なかば)half; middle; semi.

呼(コ)call.

呼ぶ(よぶ)to call; to invoke; to summon.

\section{5 乍 7 作 9 昨}

作(サク)make; work; harvest.

作る(つくる)to make.

昨(サク)previous.

昨日(きのう)yesterday.

\section{5 㠯 8 官 9 追}

官 depicts many rooms in a building.

官(カン)government official.

官界(カンカイ)bureaucracy.

長官(チョウカン)secretary; director; chief; director general.

追(ツイ、お)chase; follow; pursue.

追う(おう)to chase; to run after; to follow.

追い風(おいかぜ)tailwind; favorable wind.

\subsection{7 呂 12 営}

呂(ロ)spine.

営(エイ)manage.

\section{5 可司 6 向同回 10 高}

可(カ)passable; acceptable; tolerable.

許可(キョカ)permission; authorization; approval.

司(シ)director.

司る(つかさどる)to rule; to govern.

向 depicts a house and a window.

向かい(むかい)(n) facing; opposite; across the street; other side.

向く(むく)to face; to turn toward.

向ける(むける)(v1,vt) to turn towards.

向こう(むこう)opposite side; other side; opposite direction.

向上(コウジョウ)improvement; advancement; progress.

同(ドウ、おな)same.

同じ(おなじ)same.

同性愛(ドウセイアイ)same-sex love.

回 depicts a spiral.

回す(まわす)(vt) to turn; to rotate.

今回(コンカイ)this time.

一回(イッカイ)once; one time.

一回目(イッカイめ)first.

高(コウ)high.

高い(たかい)high; expensive.

\subsection{7 何 8 河}

何(カ、なに、なん)what.

河(かわ)river; stream.

河川(カセン)rivers.

大河(タイガ)large river.

\subsection{11 寄}

寄(キ)approach.

寄り(より)(n) being close to.

寄せる(よせる)(v1,vt) to come near.

寄り掛かる(よりかかる)to lean against; to lean on.

\ruby{柵}{さく}に寄りかかるto lean on the fence.

\subsection{10 哥 14 歌}

哥 means older brother.

歌(カ、うた)song.

歌声(うたごえ)singing voice.

歌う(うたう)to sing.

\subsection{13 稿}

稿(コウ)draft; copy; manuscript

\section{5 业}

业 is an alternative form of 北 (north).

\subsection{8 並 12 普}

並 depicts two people standing side-by-side.

並(ヘイ、なみ、なら)row.

並(なみ)row.

並ぶ(ならぶ)to stand in line; to line up.

普(フ)universal; wide; general.

普通(フツウ)(adj-no) general; ordinary; usual.

普通の人間(フツウのニンゲン)ordinary human.

\subsection{13 業 14 菐僕}

業(ギョウ)industry.

工業(コウギョウ)manufacturing industry.

企業(キギョウ)enterprise; business.

菐 is the 14-stroke ``thicket'' radical.

僕(ボク)I; me (male).

\section{5 左右 8 若}

左(ひだり)left.

右(みぎ)right.

若い(わかい)young; at an early time in life.
若年(ジャクネン)the time when one was young.

\section{5 玉 8 咅}

玉(ギョク、たま) ball

咅 means ``to spit out''.

\subsection{6 全}

全 depicts a whole piece of jade.

全(ゼン)whole.

全国(ゼンコク)countrywide; national.

全て(すべて)all; everything.

全く(まったく)(adv) completely, entirely, wholly, totally

\subsection{10 剖 11 部}

剖(ボウ)dissection.

部(ブ)section; department; part.

市部(シブ)urban areas.

全部(ゼンブ)all; entire; whole.

学部(ガクブ)a department in a faculty in a university.

東部(トウブ)eastern part.

部門(ブモン)division (of a larger group).

一部(イチブ)one copy (of a document).

\section{5 主 7 住 8 往注}

主(おも)chief; main; principal; important.

主人公(シュジンコウ)hero; main character.

住(ジュウ)dwelling; living.

永住(エイジュウ)permanent residence.

居住(キョジュウ)residence.

住人(ジュウニン)inhabitant; resident; dweller.

住む(すむ)to live (of humans); to reside; to inhabit.

日本に住んでいるto be living/residing in Japan.

往(オウ)outward; journey.

注(チュウ)pour.

注文(チュウモン)order (in a restaurant); request.

\section{5 以}

以(イ)by means of.

以前(イゼン)previously; formerly; the former.

以上(イジョウ)above; more.

10以上above 10; over 10; more than 10.

以下(イカ)below; less.

10以下below 10; under 10; less than 10.

\section{5 去 8 法}

去(キョ、コ)leave

法(ホウ)method.

\section{5 令 (6 米) 7 冷}

令(レイ)orders

冷たい(つめたい)(adj-i) cold (of a tangible object; to the touch)

\subsection{8 命}

命(メイ、ミョウ、いのち)fate.

運命(ウンメイ)fate; destiny.

使命(シメイ)mission.

\subsection{12 奥}

奥(おく)interior.

奥山(おくやま)remote mountain.

\subsection{12 歯 17 齢}

歯(シ、は)tooth.

歯車(はぐるま)gear; cog-wheel.

齢(レイ)age.

齢は、23\ruby{歳}{サイ}。The age is 23 years.

\section{5 号}

号(ゴウ、よびな、さけ)number

\section{5 皿且 6 血}

皿(さら)dish, plate

且つ(かつ)and.

且又(かつまた)besides; furthermore; moreover

血(ち)blood.

止血(シケツ)stop bleeding; hemostasis.

\section{5 示申}

示 depicts a spirit.
There are two Unicode codepoints:
⽰ (U+2F70 in the CJK Radicals Supplement block)
and 示 (U+793A in the CJK Unified Ideographs block).
To machines they differ,
but to humans they look the same.

示(シ)indicate.

示す(しめす)to indicate.

申 depicts a bolt of lightning.
申す(もうす)(humble,vt) to say; to speak.

\section{5 世台}

世(よ)world; society; age; generation

台(ダイ、タイ、うてな)tower; stand; pedestal.
台 is simplified form of 14 臺.
仙台(センダイ)(city name) Sendai.

\section{5 冬 10 夏 11 終}

冬(トウ、ふゆ)winter

夏(カ、なつ)summer

終(シュウ)end.

最終(サイシュウ)last; final; closing.

終了(シュウリョウ)(n,vs)
end; close; termination.
to quit or exit (a computer program).

終わる(おわる)to finish; to end; to close.

\section{5 犯 8 狙 9 狭}

犯(ハン、おか)crime.
犯す(おかす)to commit (a crime); to perpetrate (a crime).

狙う(ねらう)(vt) to aim at

狭める(せばめる)(v1,vt) to narrow.
狭い(せまい)(adj-i) narrow; confined; small.

\section{5 癶 7 豆}

癶 depicts footsteps, dotted tent, or legs.

豆(トウ、まめ)beans; pea.

\subsection{9 発 12 廃}

発(ハツ、ホツ)departure.

発車(ハッシャ)(n,vs) departure of a vehicle.

発達(ハッタツ)development; growth.

発言(ハツゲン)utterance; speech; proposal.

発信 (ハッシン) (n,vs) dispatch; transmission; submission.

廃(ハイ)abolish.

廃人(ハイジン)crippled/disabled/invalid person.

\subsection{12 登}

登(トウ)climb.

\section{5 史 6 吏 7 更 8 使 9 便}

史(シ)history.

史家(シカ)historian.

吏(リ)officer.

更(コウ)grow late.

更ける(ふける)(vi) to get late; to advance; to wear on.

使(シ)use.

使用(シヨウ)(n) use.

使う(つかう)to use.

便(ベン、ビン)convenience.

便利(ベンリ)convenient; handy; useful.

不便(フベン)inconvenience.

便所(ベンジョ)lavatory.

大便(ダイベン)feces; excrement; shit.

便り(たより)news; tidings; information; letter.

\section{5 召 8 沼招 9 昭 11 紹}

召す(めす)(honorific) to invite; to eat

沼(ぬま)marsh; swamp; bog

招(ショウ)beckon.

招待(ショウタイ)invitation.

昭(ショウ)shining.

紹(ショウ)introduce.

紹介(ショウカイ)introduction; referral.

\section{5 目 6 自}

目(め)eye

自(ジ、シ、みずか)self; oneself.

自ら(みずから)(adv) personally.

自在(ジザイ)freely (at will).

自分(ジブン)self.

自身(ジシン)self.

\subsection{9 相省冒 11 眼}

相(ソウ、あい)mutual.
相手(あいて)companion; partner; company.

相(ショウ)minister.
首相(シュショウ)prime minister; chancellor; premier.

省(ショウ)(suf) ministry; department.
国交省(コッコウショウ)(abbr)
Ministry of Land, Infrastructure, Transport, and Tourism.
省く(はぶく)(vt)
to omit; to leave out; to exclude.
to curtail; to save; to cut down; to economize.

冒 depicts a hat obstructing the sight, implying rashness
(acting without enough thought).
冒す(おかす)(vt) to risk.
冒険(ボウケン)adventure.

眼(まなこ)eyeball.
両眼(リョウガン)both eyes.

\subsection{17 瞳}

瞳(ひとみ)pupil (of the eye).

\subsection{7 見 11 現 12 覚}

見る(みる)(v1,vt) to see.

見える(みえる)(v1,vi) to appear.

現(ゲン)appear.

表現(ヒョウゲン)expression; presentation.

現す(あらわす)(vt) to reveal; to show; to display.

現れる(あらわれる)(v1,vi) to appear; to become visible; to materialize.

覚める(さめる)(v1,vi) to wake up
