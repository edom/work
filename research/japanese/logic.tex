\chapter{Logic}

\section{Counting}

\subsection{Numbers: 1 一 2 二十八七九 3 三千万 4 五六 5 半四 6 両百}

一(イチ、ひと)one

二(ニ、ふた)two

三(サン、み)three

四 four

五 five

六 six

七 seven

八 eight

九 nine

十 ten

半(ハン)half

両(リョウ)both

百(ヒャク)hundred

千(セン)thousand

万(マン)ten thousand

\subsection{Rotation: 6 回}

回 depicts a spiral.
回す(まわす)(vt) to turn; to rotate.
一回(イッカイ)once; one time.
一回目(イッカイめ)first.

\section{Contrast and opposition}

\subsection{Negation: 4 不反 12 無}

不(フ)(prefix) not; bad; poor.
不安(フアン)anxiety; insecurity.
不明(フメイ)unknown; obscure; anonymous; unidentified.

反(ハン)anti-.
反する(ハンする)to oppose; to rebel; to revolt.
反体制(ハンタイセイ)anti-establishment.

無(ム) no, -less, without.
無駄(ムダ)uselessness.
無用(ムヨウ)uselessness.
無敵(ムテキ)invincible, unrivaled (lit. no-enemy).
無茶(ムチャ)absurd, unreasonable (lit. no-tea).
無人(ムジン)unmanned (lit. no-human).
無言(ムゴン)silence (lit. no-say).

\section{Opposite: 6 向}

向 depicts a house and a window.
向かい(むかい)(n) facing; opposite; across the street; other side.
向く(むく)to face; to turn toward.
向ける(むける)(v1,vt) to turn towards.
向こう(むこう)opposite side; other side; opposite direction.
向上(コウジョウ)improvement; advancement; progress.

\subsection{Versus: 7 対}

対(タイ)versus...
対する(タイする)to face each other.

\section{Productive abstract concepts}

\subsection{Turning-into: 4 化}

化(カ)(suffix) -ization, -ification.
グローバル化(グローバルカ)globalization.
化ける(ばける)(v1,vi) to take the form of.
化学(カガク)chemistry.
化石(カセキ)fossilization.
分化(ブンカ)specialization.

\subsection{Self: 6 自}

自(ジ)self.
自ら(みずから)(adv) personally.
自在(ジザイ)freely (at will).
自分(ジブン)self (context? example usage?).

\subsection{Repetition: 6 再}

再(サイ)again, re-

再生(サイセイ)playback; rebirth

再開(サイカイ)reopening

再来(サイライ)return, comeback

\subsection{Same: 6 同}

同じ(おなじ)same.
同性愛(ドウセイアイ)same-sex love.

\subsection{Whole: 6 全}

全 depicts a whole piece of jade.

全(ゼン) whole

全部(ゼンブ)altogether; everything

全く(まったく)(adv) completely, entirely, wholly, totally

\subsection{Every: 7 毎}

毎(マイ)every.
毎日(マイニチ)everyday.
毎月(マイゲツ、マイつき)every month.
毎時(マイジ)every hour.
毎回(マイカイ)every time (every time it happens); every occurrence.
毎年(マイネン、マイとし)every year.

\subsection{Most: 12 最}

最(サイ)most.
最も(もっとも)most.
日本の最も高い山(ニホンのもっともたかいやま)Japan's highest mountain.
世界で最も太い人(セカイでもっともふといひと)The fattest person in the world.
最小(サイショウ)smallest.
最大(サイダイ)biggest.
最初(サイショ)first.
最後(サイゴ)last.
最新(サイシン)newest.
最高(サイコウ)best, highest, tallest.

\section{Foreign: 6 外}

6 外(ガイ)foreign (not from somewhere nearby).
外人(ガイジン)foreigner, foreign person.
外国(ガイコク)foreign country.
外界(ガイカイ)outside world.
海外(カイガイ)foreign; abroad; overseas.

\section{Existence and truth}

\subsection{Existence: 6 在存有}

有 depicts a hand holding a piece of 肉(meat).
有る(ある)to exist.

存じる(ゾンじる)(v1,humble) to think, feel, consider, know.
存在(ソンザイ)existence; being.
共存(キョウゾン)coexistence.
存亡(ソンボウ)life-or-death; existence; destiny.

\subsection{Truth: 10 真}

真(シン)truth; reality

\section{Cause and reason: 5 由 6 因}

由(よし)cause; reason.

因(イン)cause; factor

\section{Geometry}

\subsection{Flat: 5 平}

平 can mean flat, level (not tilted), ordinary, plain, non-special.
平ら(たいら)flatness.
平たい(ひらたい)(adj-i) flat; even; level; simple.
平皿(ひらざら)flat dish.
平安(ヘイアン)peace; tranquility.
平気(ヘイキ)coolness; calmness; composure; unconcern.
平日(ヘイジツ)weekday; ordinary day (non-holiday).
平年(ヘイネン)normal (non-leap) year; normal year (related to harvest; weather).
公平(コウヘイ)fairness; impartiality; justice
水平(スイヘイ)level; horizontally
平文(ヘイブン)plain (non-encrypted) text.
平面(ヘイメン)level (flat and not-tilted) surface.

\subsection{Point: 9 点}

点(テン)point; spot; speck; mark.

\subsection{Shape: 7 形}

形(ケイ、かたち)shape; form.

\subsection{Circle: 3 丸}

丸(まる)circle

\subsection{Intersect: 6 交 10 校}

交わる(まじわる)cross; intersect; join; meet

国交(コッコウ)diplomatic relations

学校(ガッコウ)school

\subsection{Corner: 7 角}

角(カク、かど)corner

角(つの)horn (head protrusion)

\section{Spacetime points and intervals: 5 古 7 近 8 若長 9 前 13 新遠}

古 consists of 十(ten) and 口(mouth, generation).
古い(ふるい)old (not of person); ancient; obsolete.

近い(ちかい)(adj-i) near (spatial distance).
近々(ちかぢか)soon.
近作(キンサク)recent work.
最近(サイキン)most recent; recently; these days; nowadays.

若い(わかい)young; at an early time in life.
若年(ジャクネン)the time when one was young.

長(チョウ)
long (distance or time).
leader.
eldest.
長い(ながい)long (distance); long (time).
長女(チョウジョ)eldest daughter; first-born daughter.
市長(シチョウ)mayor (a government official).
身長(シンチョウ)height (of body).
最長(サイチョウ)longest, tallest.
社長(シャチョウ)company president.

老い(おい)old age; old (of person); at a late time in life.
老人(ロウジン)old person.
老若(ロウニャク)old and young; all ages.

前(まえ)before (time), in front of.
午前(ゴゼン)morning; before noon; a.m. (ante meridien).

新 depicts cutting tree down with axe.
新(シン)new.
新しい(あたらしい)new.
新聞(シンブン)news.
新車(シンシャ)new car.

遠い(とおい)(adj-i) far (spatial distance).
