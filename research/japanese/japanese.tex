\documentclass[12pt,openany]{book}
\usepackage[CJKmath=true]{xeCJK}
\usepackage{amsmath}
\usepackage{hyperref}
\usepackage[
    paperwidth=6in,paperheight=9in
    ,top=1.00in,bottom=1.00in
    ,inner=1.00in,outer=0.75in
]{geometry}
\setCJKmainfont{HanaMinA}
\begin{document}
\frontmatter
\phantomsection
\addcontentsline{toc}{chapter}{Cover}
\begin{titlepage}
{%
    \setlength\parindent{0em}%
    {\Large Cold-Turkey Head-First Dive into Kanji}\par\vspace{2em}
    Erik Dominikus\par
    2017-04-02\par
    (Draft)\par\vfill
    {\small
        This book aims to optimize your kanji learning loop.

        This book does not discuss about grammar, register, and culture.
    }
}
\end{titlepage}
\phantomsection
\addcontentsline{toc}{chapter}{\contentsname}
\tableofcontents
\mainmatter
\chapter{Strategy}

\section{Mindful consistent spaced repetition}

The key to remembering kanji is
\emph{mindful consistent spaced repetition}.
Don't just gloss over the kanji.
\emph{Internalize}.

Don't read one stroke at a time,
but read one component at a time.

\emph{Expect to forget a lot, many times}.
You will forget.
Reread this book.
This book is meant to be read and read again many times.
The aim of this book is to help you optimize your loop.

Mindful consistent spaced repetition is \emph{the only way},
at least until mind-upload is invented.
There is no shortcut.
It's just how the brain works.

The loop:
For each kanji:
Read the kanji out (voice it out).

\section{Target audience}

You read, write, speak, and listen to English but don't read, write, speak, and listen to Japanese.

\section{Memorize the syllabaries}

Memorize all hiragana and katakana.
It's a must.

\section{Understand the principles}

Understand that most kanjis are composed using other kanjis.

Use Wiktionary to found the oracle bone script version of Han characters.

Don't focus on both grammar and vocabulary at the same time.
One who chases two rabbits catches neither.

\section{Read and hear native speakers}

Go to YouTube.
Hear how they speak, even if you don't understand.

Use Google Translate, but do not trust it.
Sometimes it gives the wrong reading.
Use Wiktionary, Wikipedia.

Find Japanese people on Twitter and understand their tweets.

Find Japanese blog posts.

Set up an input method, or use Google Translate.

\chapter{Kanji}

Information is compiled from Wiktionary,
edict, compdic, kanjidic, gjiten,
and Japanese-to-English Google-Translate.

Japanese schoolgoers take 12 years to study 2,000 characters.

\section{History}

Japan borrowed Han characters.

To understand a character, see how it combines with other characters.

To keep you motivated, the characters have been sorted by number of strokes.

A kanji can have several readings:
go'on, kan'on, kun-yomi, name reading, and stylistic/idiosyncratic/ad-hoc reading.

新字体(シンジタイ)(lit. new Han-character body)

旧字体(キュウジタイ)(lit. old Han-character body)

A Latin alphabet letter encodes a \emph{sound}.
You can pronounce the word without understanding it.
A Han character encodes a \emph{concept}.
You can understand the meaning without pronouncing it.

Katakana shows Chinese reading and hiragana shows Japanese reading.

\section{Grouping}

The goal of grouping is to allow the brain to \emph{chunk}.
When the brain recalls any of the member of the group,
it automatically recalls other members of the group.

Use recursive grouping that is 4 levels deep because \(7^4 = 2401\).

\section{Grouping by reducibility}

Chapter \(n\) should not define a kanji
that has more than \(n\) components.
Each section in a chapter groups kanji by semantic closeness
(not by visual closeness).

Example of irreducibility:
台 (tower) is treated as one component;
for pattern recognition purposes,
it is not treated as 厶 stacked on 口.

\section{A possible grouping: scratchiness}

The characters are sorted ascending by ``scratchiness''.
Boxes are less scratchy than lines.
Lines are less scratchy than dots.
Symmetry is less scratchy than asymmetry.
Fewer strokes are less scratchy than more strokes.
However, this is subjective, approximate, and inconsistent.
I don't always follow my own rules.
The goal is to optimize the learning rate of the reader.
The reader is expected to read the character list
sequentially and repeatedly.

Level 1 looks like
an aspiring artist's minimalist abstract paintings.
Level 2 is a combinatorial explosion
of two juxtaposed abstract paintings.
Level 3 ramps up the difficulty even more with more strokes and less symmetry.
Finally, Level 4 looks like total chicken scratch to the uninitiated,
is barely legible on screen even in a 12-point sans-serif font,
and may raise questions such as
``You have a character for that?''
and
``What drug were the sages smoking?''

\section{Another grouping: recency}

Group characters invented in the same period together.
Ancient Chinese concepts,
Old Chinese concepts,
Middle Chinese concepts,
Contemporary Chinese concepts.

\section{Another grouping}

Use visual similarity to group characters by concept.

Visual similarity does not always imply conceptual similarity.

\section{Historical context}

All dates are approximate.

In 12,000 BCE, dogs had been domesticated.

In 4,000 BCE, Egypt had had papyrus.

In 2,500 BCE, Egypt had hieroglyphs.

In 1,000 BCE, China had had oracle bone writing.

In 479 BCE, Confucius died.

Around 400 BCE, Gautama Buddha died.

In 399 BCE, Socrates died.

In 206 BCE, China began the Han dynasty.

Around 100 BCE, China had paper.

Around 30, Jesus died.

In 220, China ended the Han dynasty.

In 632, Muhammad died.

In 1000, China had had gunpowder.

Around 1056, Benedict IX died.

In 1185, Japan ended the Heian period.
Japan also had imported Han characters from China.

In 1546, Martin Luther died.

In October 1582, Gregory XIII introduced
what would later be known as the Gregorian calendar.

In 1750, Johann Sebastian Bach died.

In 1819, James Watt died.

In 1823, David Ricardo died.

In 1865, Abraham Lincoln died.

In 1897, Henry George died.

In 1917, the Balfour Declaration supporting Zionism was made.

In 1945, Adolf Hitler died.

1942--1945: World War II

In 1946, John Maynard Keynes died.

In 1947, the Truman Doctrine was announced,
marking the beginning of the Cold War.

1950--1953: Korean War

In 1954, Alan Turing died.

In 1955, the Vietnam War began.

In 1961, Yuri Gagarin in Vostok 1
became the first man to travel into space.

In 1963, John Fitzgerald Kennedy died.

In 1968, Martin Luther King Jr. died.

In 1970, Sukarno died.

In 1975, the Vietnam War ended.

In 1980, Mohammad Hatta died.

In 1979--1989, the United States Central Intelligence Agency
carried out Operation Cyclone,
arming and financing Afghanistan mujahideen.

In 1986, Lafayette Ronald Hubbard died.

In 1991, the Soviet union ended, marking the end of the Cold War.

In 2001, the World Trade Center was destroyed.

In 2006, Milton Friedman died.

\chapter{Kanji 1}

Kanji is recursive: it may be made of other kanji.

\section{Non-standalone parts}

\subsection{2 冂亻冫 3 宀辶艹忄氵 4 灬殳礻 5 衤}

These parts rarely stand on their own.

冂 depicts an inverted box.

亻 is the side form of 人.

冫 is the left form of 氷 (ice).

宀 depicts a roof.

辶 is the combining form of 辵 meaning ``walking''.
Chinese dictionaries say this has 3 strokes.
Japanese dictionaries say this has 4 strokes,
but also say that 近 has 7 strokes.
If the Japanese dictionaries are to be followed,
then 近 should have 8 strokes.

艹 is the top form of 艸 (grass).

忄 is the left form of 心 (heart).

氵 is the left form of 水 (water).

灬 is the bottom form of 火.

殳 depicts a hand holding a tool or a weapon.

礻 is the left form of ⽰ (spirit).

衤 is the left form of 衣 (cloth).

\subsection{6 艸耂艮 8 隹 10 哥}

艸 depicts grass.

耂 depicts a bent-over figure with long hair; an old man.

艮 depicts stopping; eye and spoon.

隹 depicts a short-tailed bird.

哥 means older brother.

\section{2 人又了}

人(ジン、レン、ニン、ひと)person; people; human

又 depicts the right hand.
又(また)again; also.

了 depicts a wrapped baby with only head visible.

\section{3 口子士女}

口(くち)mouth.

子 depicts a child with arms visible.
子(シ、こ) child.
男子(ダンシ)young man, boy.
女子(ジョシ)young woman, girl.
太子(タイシ)crown prince.

士(さむらい)gentleman; samurai.
士(シ)(suffix)
a qualified person;
a person with a qualified profession.
学士(ガクシ)university graduate (who has obtained a degree).
工学士(コウガクシ)Bachelor of Engineering.

女(おんな)woman

\section{4 父止开手井火水斤心}

父(ちち)(humble) father.
お父さん(おとうさん)(honorific) father.

止 depicts a footprint.
止める(やめる)(v1) to stop (doing something).
止める(とめる)(v1) to stop moving (walking, etc.); to park (a car).

开 is a simplification of 幵 (raising both hands).
幵 raising both hands.

手(て)hand.
手がける(てがける)(v1) to make; to do; to produce; to work on.
手洗い(てあらい)restroom; lavatory; toilet; a place for washing hands

井 depicts a square well.
井(い)well (water reservoir).

火(カ、ひ)fire, flame.
火事(カジ)fire (disaster).
大火(タイカ)big fire.

水(スイ、みず)water.
水車(スイシャ)water wheel.

斤 depicts an axe.

心 depicts a heart.

\section{7 弟男呆児}

弟(おとうと)(humble) younger brother.
弟さん(おとうとさん)(honorific) younger brother.
兄弟(キョウダイ)siblings;
brothers and sisters
(although the characters mean older brother and younger brother).

男(おとこ)man.

呆 depicts a child.
呆れる(あきれる)(v1,vi) to be amazed, astonished, astounded.

児 depicts an infant with imperfect cranium (fontanelles).
児 is simplified from 兒.
乳児(ニュウジ)infant; suckling baby.
男児(ダンジ)boy; son.

\section{6 血耳舌}

血(ち)blood.
止血(シケツ)stop bleeding; hemostasis.

耳(みみ)ear

舌(した)tongue.

\section{4 木 5 禾本 6 米休 7 体 8 林 12 森}

木(き)tree

禾 depicts a plant stalk.

本(ホン)book, counter for long cylindrical objects.
(In Ancient China, a book is a scroll.)
一本(イッポン)one long cylindrical object, one point.
日本(ニホン)Japan.

米(ベイ、こめ)rice.

休む(やすむ)to rest

体(タイ、からだ)body.
肉体(ニクタイ)body; flesh.

林(はやし)woods; forest; copse; thicket

森(もり)forest

森林(シンリン)forest woods

\section{Not-yet and next: 5 未 7 来}

未(ミ)not yet.
未来(ミライ)future (lit. not yet come).
未 depicts a tree that has not fruited.
来 depicts fruits hanging on a tree.
They mean that the time to harvest has not or has come.

来(ライ)come; next.
来月(ライゲツ)next month.
来年(ライネン)next year.
「来年田中さんが日本に行きます。」Next year, Mr. Tanaka will go to Japan.
(``Next year'' means one year after the moment the speaker says it.)

来る(くる)to come.
This is an irregular verb.
The past form is 来た(きた).

\section{4 介 6 光 7 辛赤 9 界}

介 can mean mediation.
紹介(ショウカイ)introduction; referral.
仲介(チュウカイ)agency; intermediation.
介入(カイニュウ)intervention.
介助(カイジョ)help; assistance; aid.

介 can mean shellfish.
魚介(ギョカイ)marine products; seafood.

光(コウ、ひかり)light (electromagnetic wave)

辛 depicts a tool used to mark slaves and criminals;
this sometimes also depicts a tree.
辛い(からい)(adj-i) spicy, salty, harsh, hot, acrid.
辛い(つらい)(adj-i) bitter; painful; heartbreaking; difficult.
Suffix づらい(adj-i) means ``difficult to do''.
読みづらい(adj-i) difficult to read.
書きづらい(adj-i) difficult to write.
読みづらい漢字difficult-to-read Han character.

赤い(あかい)(adj-i) red.

界(カイ)world.
世界(セカイ)world.
学界(ガッカイ)academic world.
業界(ギョウカイ)industry world, business world.

\section{2 入 4 中内 5 央出 6 肉}

入る(いる)(v5)
to enter; to go into; to get into.

中(チュウ、なか)middle.
中学(チュウガク)middle school; junior high school.
…の中(…のなか)middle of something.
田中(たなか)(name) Tanaka.
中山(なかやま)(name) Nakayama.
中川(なかがわ)(name) Nakagawa.

央(オウ)center.
中央(チュウオウ)center; central.

内(うち)inside

出 depicts something coming out of an open box.
出る(でる)(v1,vi) to go out; to exit; to leave.
出来る(できる)(v1,vi) to be able to do.
出来上がる(できあがる)(vi) to be finished; to be completed; to be ready

肉 depicts the ribs of an animal's torso.
肉(ニク)meat; flesh; body (as opposed to spirit).

\section{3 川山土上下大小亡工夕才寸幺纟}

川(かわ)river

山(サン、やま)mount, mountain

土(ド、つち)
soil; earth; ground; dirt; ground (as opposed to the heavens).

上(うえ)up

下(した)down

大 depicts a person with outstretched arms.
大(タイ)big.
大きい(おおきい)big.

小 depicts three sand granules.
小さい(ちいさい)small.
小(ショウ)small.

亡 means death; destruction; perishment; the deceased.
亡くす(なくす)(vt) to lose something (because he/she/it dies).

工(コウ、ク)craft.
工学(コウガク)engineering.
大工(ダイク)carpenter.
木工(モッコウ)carpenter.
工(たくみ)(name) Takumi.

夕(ゆう)evening

才 talent.
In China 3 strokes, in Japan 4 strokes.
天才(テンサイ)genius; prodigy.

% https://en.wiktionary.org/wiki/%E5%AF%B8#Japanese
寸depicts a position on the forearm
where the pulse can be palpated by compressing the radial artery.
寸(スン)an ancient unit of length, approximately 3 cm.

幺 depicts a tiny/small/short thread.

纟 is the simplified form of 糹
(left form of 6 糸 (thread)).

\section{5 左右}

左(ひだり)left

右(みぎ)right.

\section{4 太少 6 多 11 細}

太い(タイ、タ、ふとい)(adj-i) fat; thick.
太る(ふとる)to grow fat; to become fat; to gain weight.

少し(すこし)(adv, n) small quantity; few; a little

多 depicts two pieces of meat.
多 means ``many''.
多(タ)(prefix) multi-.

細い(サイ、ほそい)(adj-i) thin; slender.
細る(ほそる)to become thin.
細かい(こまかい)(adj-i) small; trivial.

\section{Blades: 2 刀刂 3 弋 4 戈 5 戊 6 戌 9 咸}

刀(トン、かたな)sword.

刂 depicts a knife.

弋 depicts shooting with a bow and an arrow.

戈 depicts a spear-axe (a halberd).

戊 depicts a dagger-axe (an ancient Chinese weapon).

戌 depicts an axe.

咸 depicts an axe and a mouth.

\section{5 母兄目立田世石氷台}

母 depicts a pair of breasts.
母(はは)mother.
お母さん(おかあさん)(honorific)

兄(あに)(humble) older brother.
お兄さん(おにいさん)(honorific) older brother.

目(め)eye

立 depicts a person standing on ground.
立つ(たつ)(vi) to stand

田(た)rice field.

世(よ)world; society; age; generation

石(いし)stone

氷(こおり)ice

台(うてな)tower; stand; pedestal.
台 is simplified form of 14 臺.
仙台(センダイ)(city name) Sendai.

\section{Administrative regions: 4 区 5 市 8 京}

区(ク)ward; district (an administrative area).

市(シ)city (an administrative area).
市(いち)market; fair (trade show).

京(キョウ)imperial capital

\section{Mound and city: 3 阝 7 邑 8 阜}

阝 has 2 strokes in Chinese and 3 strokes in Japanese.
On the left side, 阝 is 阜 (mound).
On the right side, 阝 is 邑 (city; village).

\section{6 死}

死(シ)death.
死ぬ(しぬ)to die.
死亡(シボウ)death; mortality.
死去(シキョ)death.

\section{6 衣}

衣 depicts a cloth.

\section{Nourishment, skin, and cloth: 5 召皮 6 衣 9 食 12 飲}

召す(めす)(honorific) to invite; to eat

皮(かわ)skin; hide.

衣(ころも)
clothes;
garment. gown;
robe. coating (glaze; batter; icing).

食べ物(たべもの)food

食物(ショクもの)food

食べる(たべる)(v1) to eat

飲む(のむ)to drink (any liquid, not just liquor)

\section{Animals: 4 犬牛 6 虫 10 馬 11 魚猫豚鳥}

魚介(ギョカイ)seafood; marine products

動物(ドウブツ)animals.

犬(いぬ)dog.

牛(ギュウ、うし)
cow; bull; ox; buffalo.
beef.

虫(むし)insect; bug; cricket; moth

動物園(ドウブツエン)zoo (lit. animal garden)

馬(うま)horse.

鳥(とり)bird; chicken; chicken meat.
焼き鳥(やきとり)grilled chicken meat.

魚(さかな)fish.

猫(ねこ)cat.

豚(ぶた)pig.

\section{2 丁力厶}

丁 depicts a nail.

丁(ひのと)

丁(チョウ)(suffix)
counter for long and narrow (relatively flat) thing
such as paper, guns, scissors, spades, hoes, guitars.
一丁(イッチョウ)one sheet; one page.
one serving (in a restaurant).
丁重(テイチョウ)polite; courteous; hospitable.
包丁(ホウチョウ)kitchen knife.

力(リョク、リキ、ちから)strength.
百人力(ヒャクジンリキ)tremendous strength.

厶 means I, myself, private.

\section{4 王天}

王(オウ)king

天(テン)sky; heaven

\section{4 今元戸之氏文夫}

今(いま)now

元 beginning; origin.
元(もと)origin; source.
元々(もともと)
originally; by nature; from the start; since the beginning.

戸(と)door.

之(これ)this

氏 depicts a man bowing to the left.
氏(シ)(suffix,honorific) Mr.; Mrs.. family. clan.
氏(うじ)family name; birth; lineage.

文(ブン)sentence (literature).
文字(モジ)letter (of alphabet); character (a Han character)
文書(ブンショ)sentence.
文化(ブンカ)culture.
作文(サクブン)writing.
小学生の作文(ショウガクセイのサクブン)elementary-schoolchild writing.

夫(フ、おっと)husband

\section{5 皿用求业矢史且}

皿(さら)dish, plate

用 utilize; business; service; employ.
用いる(もちいる)(v1,vt) to use; to make use of; to utilize.
常用(ジョウヨウ)(n) common-use; in common use; commonly used.

求める(もとめる)to want

业 alternative form of 北 (north).

矢(や)arrow

史家(シカ)historian.

且つ(かつ)and.
且又(かつまた)besides; furthermore; moreover

\section{5 ⽰失申主玉生}

⽰ depicts a spirit.

失 depicts something falling from a hand.
失う(うしなう)(vt) to lose; to part with.
失明(シツメイ)loss of eyesight.
失血(シッケツ)loss of blood.

申 depicts a bolt of lightning.
申す(もうす)(humble,vt) to say; to speak.

主(おも)chief; main; principal; important.
主人公(シュジンコウ)hero; main character.

玉(ギョク、たま) ball

生depicts a sprout, something sprouting from the ground.
生(セイ)nature; sex; gender.
学生(ガクセイ)student.
生まれる(うまれる)(v1,vi) to be born.

\section{6 合会当曲羽字糸糹}

合う(あう)to fit

会う(あう)to meet (face to face).
会社(カイシャ)company; corporation.
会社員(カイシャイン)company employee.
年会(ネンカイ)yearly meeting; annual convention.
社会(シャカイ)society.
会見(カイケン)interview.

当てる(あてる)(v1,vt) to hit.
本当(ホントウ)truth; reality.
本当に(ホントウに)truly; really; seriously.

曲(キョク)music.
作曲(サッキョク)musical composition.

羽(はね)feather

字(ジ)character, letter (of an alphabet, not the letter that is a document)
赤字(あかジ)deficit; red letter; red Han-character.
漢字(カンジ)Han character.

糸 combines 幺 and 小.
糸(いと)thread.

糹 is the left form of 糸.

\section{7 貝声}

貝 depicts a cowry (a kind of seashell used as money in ancient China).
貝(かい)shell; shellfish.

声(こえ)voice.
This character was simplified from 17 聲.

\section{7 車}

車(くるま)car; vehicle. wheel.

\section{8 門}

門(かど) gate

\section{8 雨}

雨(あめ)rain

\section{9 首重面}

首(くび)neck

重 depicts a man carrying a bag.
重い(おもい)heavy (of weight).

面(おもて)mask.
面白い(おもしろい)interesting.

\section{10 書 12 傘}

書 depicts a hand holding a pen writing on paper.
書(ショ)writing.
辞書(ジショ)dictionary.
書く(かく)to write.

傘(かさ)umbrella.

\chapter{Grammar 1}

Phrases.

(Actually, not only grammar, but also semantics, in logic.)

\section{Existence: ある、いる}

有る(ある)possess

在る(ある)exist (inanimate)

居る(いる)exist (animate)

リンゴが有る。
The implied entity has an apple.

リンゴが在る。
There is an apple.
An apple exists.

犬は居る。
There is a dog.

妹が居る。
The implied entity has a younger sister.

\section{Negation: ない}

無い(ない)not; not exist

リンゴが無い。
The implied entity does not have any apples,
or there are no apples.
(It depends on context.)

妹が居ない。
The implied entity does not have a younger sister.

\section{Perfection: た}

(``Perfect'' means ``complete'', as in ``past perfect'', not flawless.)

笑う。
The implied entity laughs.
The implied entity will laugh.

笑った。
The implied entity laughed.
The implied entity has laughed.
The implied entity had laughed.

\section{Politification: u-to-imasu}

Irregular:

する becomes します

来る(くる) becomes 来ます(きます)

Regular:

在る becomes 在ります

居る becomes 居ます

笑う becomes 笑います

\section{Polite negation: masu-to-masen}

In polite speech, to negate a verb,
politely-negate the polite form.

笑わない becomes 笑いません.
Politify 笑う to 笑います.
Politely-negate 笑います to 笑いません.

Do not politify the negative form.
Doing so would turn 笑わない to 笑わないです,
which is not the polite counterpart of 笑わない.

Do not casually-negate the polite form.
Doing so would turn 笑います to 笑いましない,
which is not Japanese.

\section{Polite perfection: masu-to-mashita}

Politely-perfect the polite form.

笑います becomes 笑いました

Politify, politely-negate, and then politely-perfect: imasen-to-imasendeshita.

笑わなかった corresponds to 笑いませんでした

\section{I-modified noun phrase}

車car

黒いblack

黒い車black car

高い人tall person

黒くない車non-black car

黒かった車formerly-black car (a car that was black)

黒くなかった車formerly-nonblack car (a car that was not black)

Start with 黒い. Negate it to 黒くない
and then perfect it to 黒くなかった.

高い黒い車expensive black car

高い黒くない車expensive non-black car

高くない黒い車non-expensive black car

高くない黒くない車non-expensive non-black car

高かった黒かった車formerly-expensive formerly-black car

\section{U-modified noun phrase}

人person

笑う人laughing person

笑わない人non-laughing person; person who does not laugh; person who is not laughing

笑わなかった人formerly-non-laughing person; person who did not laugh

笑った人person who laughed

愛される人loved person

愛されない人unloved person

愛された人formerly-loved person

愛されなかった人formerly-nonloved person; a person who was not loved

Begin with 愛 (love).
Append する, producing 愛する (to love; who loves).
Passivate する to される by u-to-areru, producing 愛される (to be loved; who is loved).
Negate される to されない by v1-ru-to-nai, producing 愛されない (who is not loved).
Perfect されない to されなかった by i-to-katta, producing 愛されなかった (who was not loved).

話されなかった言葉
words that were not spoken

\subsection{In doubt}

愛される田中さんの女性Beloved Tanaka-san's lady (or ladies)

田中さんの愛される女性Tanaka-san's beloved lady (or ladies)

\chapter{Inflectables}

\section{Perfection of dictionary-form verb: た、だ}

``Perfect'' means ``complete'', as in ``past perfect'', not flawless.
Every dictionary-form verb can be perfected into an perfect-form verb.
The result is still a verb.

笑う。
The implied entity laughs.
The implied entity will laugh.

笑った。
The implied entity laughed (and is no longer laughing now).
The implied entity has laughed.
The implied entity had laughed.

\section{Perfection of i-adjective: i-to-katta}

速い becomes 速かった

\section{Negation of i-adjective: i-to-kunai}

速い becomes 速くない

\section{i-to-kunakatta = i-to-kunai + i-to-katta}

\section{Volitionalization: u-to-itai}

笑う becomes 笑いたい

話す becomes 話したい

\section{Conjunction: i-form}

食べる becomes 食べ

話す becomes 話し

立つ becomes 立ち

\section{Progressive form}

Examples:
the 歩き in 歩きタバコ(smoking while walking).
歩きタバコするto smoke while walking.

\section{Concurrency: X-while-Y: i-form of Y + nagara + X}

Japanese is consistently head-final,
and while-Y explains X,
so while-Y comes before X.
The modifier always comes before the head.

食べながら while eating

立ちながら while standing

食べながら話すto speak while eating

立ちながら食べるto eat while standing

\section{In doubt}

A verbal can be modified by adverbials.
A nominal can be modified by adjectivals.

\chapter{Nominal}

A \emph{nominal} is a construction that functions like a noun.

Every noun is a nominal.

\section{Conjunction: と}

Unlike English ``and'', Japanese と conjoins \emph{nominals} only.
English ``and'' can conjoin noun phrases or sentences.
田中さんと中川さんは東京に行く。Tanaka-san and Nakagawa-san goes to Toukyou.

[〈鳥を食べる人〉と〈魚を食べる人〉]が良いです。
People who eat chicken and people who eat fish are okay.

彼と私は同じ夢を見ました。
He and I had the same dream.

\section{Disjunction: か}

鳥か魚は良いです。
Chicken or fish are okay.
(Whichever of chicken or fish you give me, I will eat it.)

\section{Possession: の}

カラスの羽feather of crow

父の車father's car

彼の友達の娘 his friend's daughter

彼の妹の娘の友達his younger sister's daughter's friend

父の黒い車father's black car

細い俳優の車skinny actor's car

私の家の右the house at the right side of my house

私の家の右の右the house at the right side of the house at the right side of my house

\section{Modification by i-adjectives}

Every nominal can be modified by prepending an adjectival.

Every i-adjective is an adjectival.

黒い車black car

悲しい人sad person

細い悲しい人thin sad person

Every dictionary-form verb is an adjectival.

Every perfect-form verb is an adjectival.

黒い is an i-adjective.
車 is a noun.

車car.

黒いblack

黒い車 is an i-adjectival-modified nominal. It means ``black car''.

高い人tall person

黒くない車non-black car

黒かった車formerly-black car (a car that was black)

黒くなかった車formerly-nonblack car (a car that was not black)

Start with 黒い. Negate it to 黒くない
and then perfect it to 黒くなかった.

高い黒い車expensive black car

高い黒くない車expensive non-black car

高くない黒い車non-expensive black car

高くない黒くない車non-expensive non-black car

高かった黒かった車formerly-expensive formerly-black car

\section{More modifications by adjectivals}

Japanese subordinate clauses do not use relative pronouns.
魚を食べる is both a complete statement and a subordinate adjective clause.
As a complete statement, it means ``The implied entity eats fish.''
As a subordinate clause, it means ``who eats fish'' or ``which eats fish.''
Thus, 魚を食べる人 is a nominal that means ``the person who eats fish,''
and parses as 魚を食べる・人.

魚を食べない人person who does not eat fish

魚を食べなかった人person who did not eat fish

結婚出来ない男a man who cannot marry

妹が居ない人
people who do not have younger sisters

お金がない人
people who do not have money

夜で魚を食べる人
people who eat fish at night

石を投げる子供stone-throwing child

愛されなかった人a person who was not loved

笑う人laughing person

笑わない人non-laughing person; person who does not laugh; person who is not laughing

笑わなかった人formerly-non-laughing person; person who did not laugh

笑った人person who laughed

愛される人loved person

愛されない人unloved person

愛された人formerly-loved person

愛されなかった人formerly-nonloved person; a person who was not loved

Begin with 愛 (love).
Append する, producing 愛する (to love; who loves).
Passivate する to される by u-to-areru, producing 愛される (to be loved; who is loved).
Negate される to されない by v1-ru-to-nai, producing 愛されない (who is not loved).
Perfect されない to されなかった by i-to-katta, producing 愛されなかった (who was not loved).

話されなかった言葉
words that were not spoken

\section{In a relative clause, が can change to の}

In a relative clause, が can change to の.
For example, both 海が見え街 and 海の見え街 is to be taken to mean
``seaside town'' (a town where the sea is visible)
where the adjectival 海が見える(where the sea is visible) explains the noun 街.
Both 日本人が知らない日本語
and 日本人の知らない日本語
means ``the Japanese language that the Japanese people do not know.''
Because の was が, the nominal 日本人の知らない日本語 parses as
〈日本人・の・知らない〉日本語 (the Japanese 〈that the Japanese people do not know〉)
and not as
[日本人・の]〈知らない・日本語〉
([The Japanese people's]〈unknown・Japanese language〉).

\section{In doubt}

愛される田中さんの女性Beloved Tanaka-san's lady (or ladies)

田中さんの愛される女性Tanaka-san's beloved lady (or ladies)

\chapter{Statement}

A statement can be assigned truth value.

\section{Predication}

車は高い。
Cars are expensive.

黒い車は良い。
Black cars are good.

中村さんは魚を食べる。
Nakamura-san eats fish.

\section{Subsumption (is-a)}

犬は動物。
Dog is a mammal.

中村さんは俳優。Nakamura-san is an actor.

\section{Equation}

あの人は田中。
That person is Nakamura.

\section{Existential quantification}

Relative clause is adjectival.

人が居る。
There is a human.
There are humans.

お金がない人が居る。
There are people who do not have money.

お金がない人が悲しい。
People who do not have money are sad (feel sad).
(This is just an example sentence.
It has nothing to do with the real world.)

\section{Example constructions: が、は}

お金がありません。The implied entity does not have money.

この山は高いです。This mountain is tall.

この川は広いです。This river is wide.

名前(なまえ):お名前は?
And your name is...?

\section{Example constructions: て下さい}

戸(と)、開く(ひらく)、下さい(ください):
戸を開いてく下さい。Please open the door.

\chapter{Grammar 2}

All truth values are to be interpreted probabilistically.
The statement ``everybody has a chicken'' is neither true nor false;
it has truth value somewhere between 0 and 1.

Probabilistic temporal modal logic?

I-phrase

I-clause

U-phrase

U-clause

A nominal is a thing that acts like a noun.

An adjectival is a thing that modifies a nominal.

\section{Predication and is-a}

車は高い。
Cars are expensive.
\[
    車(x) \vdash 高い(x)
\]

車は高いか?
Are cars expensive?
\[
    ? : 車(x) \vdash 高い(x)
\]

黒い車は良いです。
Black cars are good.
\[
    車(x), 黒い(x) \vdash 良い(x)
\]

犬は動物。
Dog is a mammal.
\[
    犬(x) \vdash 動物(x)
\]

中村さんは俳優です。Nakamura-san is an actor.
\[
    \vdash 俳優(中村さん)
\]
or
\[
    中村さん(x) \vdash 俳優(x)
\]

\section{Equation}

あの人は田中さんです。

\[
    あの人 = 田中さん
\]

\section{Relative clauses}

妹が居ない人
people who do not have younger sisters
\[
    人(x), \neg \exists y (y \text{ is an 妹 of } x)
\]

お金がない人
people who do not have money
\[
    人(x), \neg \text{possess}(x,お金)
\]

お金がない人がいる。
There are people who do not have money.
\[
    \exists x (人(x) \wedge \neg \text{possess}(x,お金))
\]

お金がない人が悲しい。
People who do not have money are sad (feel sad).
(This is just an example sentence.
It has nothing to do with the real world.)
\[
    人(x), \neg \text{possess}(x,お金) \vdash 悲しい(x)
\]

\section{Example constructions: が、は}

お金(おかね):お金がありません。The implied entity does not have money.

山(やま)、高い(たかい):
この山は高いです。This mountain is tall.

川(かわ)、広い(ひろい):
この川は広いです。This river is wide.

石(いし)、子供(こども)、投げる(なげる):
石を投げる子供stone-throwing child

名前(なまえ):お名前は?
And your name is...?

\section{Example constructions: て下さい}

戸(と)、開く(ひらく)、下さい(ください):
戸を開いてく下さい。Please open the door.

\section{Nouns made by conjoining nouns: と}

Unlike English ``and'', Japanese と conjoins \emph{nouns} only.
English ``and'' can conjoin noun phrases or sentences.
田中さんと中川さんは東京に行く。Tanaka-san and Nakagawa-san goes to Toukyou.
\[
    \vdash \text{destination}(東京),行く(田中さん),行く(中川さん)
\]

彼(かれ)、私(わたし)、同じ(おなじ)、見る(みる):
彼と私は同じ夢を見ました。
He and I had the same dream.
\[
    夢(彼,x), 夢(私,y) \vdash 同じ(x,y)
\]

\section{Example constructions: Genitive: の}

羽(はね)、黒(くろ):カラスの羽は黒い。The color of a crow's feathers is black.

Crowのfeatherのcolorはblack。

\[
    羽(x), カラス(y), \text{belongs-to}(x,y) \vdash 黒い(x)
\]

\section{Example constructions: Nominalization: の}

\subsection{In doubt}

犬(いぬ)、来る(くる)、来た(きた):

田中さんは来た。Tanaka-san came.
\[
    \vdash 来た(田中さん)
\]
About Tanaka-san, he came.
What Tanaka-san did was coming.

来たのは田中さんです。
About having come,
it is Tanaka-san who did that (and not somebody else).
\[
    来た(x) \vdash x = 田中さん
\]
If anyone came, then it was Tanaka-san.
Who came is Tanaka-san.
Tanaka-san is the person who came (and not somebody else).

勉強(ベンキョウ)、貴方(あなた):
勉強するのは貴方にいい。
Studyingはyouにgood。
Studying is good for you.
Parse tree:
貴方にいいmodifiesの,
貴方にいいのis the topic,
貴方にいいのはmodifiesいい.
Logic: good-for(studying,you).

貴方にいいのは勉強するのです。
What is good for youはstudyingです。
What is good for you is studying.
Parse tree:((貴方)に・いい・の)は・勉強・する・の・です。
Another possible parse tree:
(貴方)に・(いい・の)は・勉強・する・の・です。
For you, what is good is studying.

勉強するのは何のため?
What is studying for?
What is the purpose of studying?

\section{Example constructions: たい}

魚(さかな)、食べる(たべる):魚を食べたくありません。The implied entity does not want to eat fish.

\section{Questionable}

良い(いい)、良くなかった(よくなかった):
良くなかった良い事things that was good that was not good.

Does AのBとC parse as Aの(BとC) or (AのB)とC?

\chapter{Grammar 3}

All truth values are to be interpreted probabilistically.
The statement ``everybody has a chicken'' is neither true nor false;
it has truth value somewhere between 0 and 1.

Probabilistic temporal modal logic?

An adjectival is a thing that modifies a nominal.

If 「笑った。」 is true, then 「笑う。」 is false,
but there exists a past time interval where 「笑う。」 is true.

The negative form is an i-form, so the i-to-katta rule perfects it:
笑わない becomes 笑わなかった.

笑わなかった。The implied entity did not laugh.

Iff 「笑わない。」, then not 「笑う。」.

Iff 「笑わなかった。」, then 「笑わない。」.

Iff 「笑わなかった。」, then not 「笑った。」.

\subsection{In doubt: Nominalization: の}

犬(いぬ)、来る(くる)、来た(きた):

田中さんは来た。Tanaka-san came.
\[
    \vdash 来た(田中さん)
\]
About Tanaka-san, he came.
What Tanaka-san did was coming.

来たのは田中さんです。
About having come,
it is Tanaka-san who did that (and not somebody else).
\[
    来た(x) \vdash x = 田中さん
\]
If anyone came, then it was Tanaka-san.
Who came is Tanaka-san.
Tanaka-san is the person who came (and not somebody else).

勉強(ベンキョウ)、貴方(あなた):
勉強するのは貴方にいい。
Studyingはyouにgood。
Studying is good for you.
Parse tree:
貴方にいいmodifiesの,
貴方にいいのis the topic,
貴方にいいのはmodifiesいい.
Logic: good-for(studying,you).

貴方にいいのは勉強するのです。
What is good for youはstudyingです。
What is good for you is studying.
Parse tree:((貴方)に・いい・の)は・勉強・する・の・です。
Another possible parse tree:
(貴方)に・(いい・の)は・勉強・する・の・です。
For you, what is good is studying.

勉強するのは何のため?
What is studying for?
What is the purpose of studying?

\section{Example constructions: たい}

魚(さかな)、食べる(たべる):魚を食べたくありません。The implied entity does not want to eat fish.

\section{Questionable}

良い(いい)、良くなかった(よくなかった):
良くなかった良い事things that was good that was not good.

Does AのBとC parse as Aの(BとC) or (AのB)とC?

\section{In doubt}

If X is a clause, then Xの is a nominal. (?)
Or is this a のは particle?
Example:
〈魚を食べる・の〉は・田中さんです。
or
〈魚を食べる〉のは・田中さんです。
?

\section{Adjectival}

If X is an i-adjective, then X is an adjectival.

If X is an no-adjectival, then X is an adjectival.

\subsection{I-adjectival}

\subsection{Verbal adjectival}

A verb by itself readily forms an adjectival.

笑う

笑った

\subsection{No-adjectival}

If X is a nominal, then Xの is a no-adjectival.

\section{Predicate}

A predicate is a nominal, an i-adjectival, or a verbal.

\section{Clause}

A clause is an adjectival.

食べる(たべる)by itself can be the main clause
``The implied entity eats.'' or the relative clause ``who eats''.
「食べる。」The implied entity will eat.
「食べる人」The person who eats.

\section{What?}

Phrases.

形容詞(ケイヨウシ)

i-adjective

連用形(レンヨウケイ)

continuative form?

How verbs change forms.

Conjugation is inflection of verb.
Inflection is a change of form that does not change syntactic category.
Derivation is a change of form that changes syntactic category.

\chapter{Polite}

\section{Politification: u-to-imasu}

Irregular:

する becomes します

来る(くる) becomes 来ます(きます)

Regular:

在る becomes 在ります

居る becomes 居ます

笑う becomes 笑います

\section{Polite negation: masu-to-masen}

In polite speech, to negate a verb,
politely-negate the polite form.

笑わない becomes 笑いません.
Politify 笑う to 笑います.
Politely-negate 笑います to 笑いません.

Do not politify the negative form.
Doing so would turn 笑わない to 笑わないです,
which is not the polite counterpart of 笑わない.

Do not casually-negate the polite form.
Doing so would turn 笑います to 笑いましない,
which is not Japanese.

\section{Polite perfection: masu-to-mashita}

Politely-perfect the polite form.

笑います becomes 笑いました

Politify, politely-negate, and then politely-perfect: imasen-to-imasendeshita.

笑わなかった corresponds to 笑いませんでした

\chapter{Kanji 2}

\section{2 入ヒ七匕 4 区凶 5 兄}

入(ニュウ)enter.

入る(いる)(v5i,vt)
to enter; to go into; to get into.

入力(ニュウリョク)input. data entry.

ヒ is the katakana ``hi''.

七(シチ、なな)seven.

匕(ヒ、さじ)spoon.

区(ク)ward; district (an administrative division).

凶(キョウ)evil; villain; bad luck; disaster.

凶悪(キョウアク)(adj-na) atrocious; fiendish; brutal; villainous.

兄(ケイ、キョウ、あに)older brother.

兄(あに)(humble) older brother.

お兄さん(おにいさん)(honorific) older brother.

\subsection{(2 入) 4 内 5 込}

内(ナイ、うち)inside.

込む(こむ)to be crowded.

\subsection{(2 入) 6 肉}

肉 depicts the ribs of an animal's torso.

肉(ニク)meat; flesh; body (as opposed to spirit).

肉 is sometimes corrupted to 月.

\subsection{(2 匕) 5 北}

北(ホク、きた)north

\subsection{(2 匕) 4 化 7 花}

化(カ、ばけ)change.

化(カ)(suffix) -ization, -ification.

化学(カガク)chemistry.

化石(カセキ)fossilization.

分化(ブンカ)specialization.

化ける(ばける)(v1,vi) to take the form of.

花(はな)flower

\subsection{(2 匕) 4 比 9 皆}

比 depicts two men.

比(ヒ)compare.

皆(カイ、みな)all.

\subsection{(5 兄) 10 党}

党(トウ)political party; faction.

党首(トウシュ)political party leader.

\subsection{(6 肉 4 心 2 匕) 10 能 14 態}

能(のう)talent; gift; function.

態(ざま)mess; sorry state; plight; sad sight.

変態(ヘンタイ)sexual perversion.

\subsection{(7 兑 6 肉) 11 脱 14 説 15 閲}

兑 is not used on its own in Japanese.

脱出(ダッシュツ)escape.

説(セツ)theory.

説明(セツメイ)explanation.

閲(エツ)review.

\subsection{(6 肉 4 凶) 10 悩 11 脳}

悩(ノウ)trouble.

脳(ノウ)brain.

The left radical of 脳 is 肉.

脳内(ノウナイ)intracranial; inside the brain

\section{2 了}

了 depicts a wrapped baby with only head visible.

了(リョウ)finish.

\section{2 人 4 仁 5 代}

仁(ジン)kindness.

グローバル化(グローバルカ)globalization.

代(ダイ、タイ)substitute.

代わる(かわる)(vi) to substitute for.

\section{2 ト卜 6 外}

ト is the katakana ``to''.

卜 is the ``divination'' radical.

For the computer, those are different characters.

外(ガイ、ゲ、そと、ほか、はず)outside; foreign.

外(そと)outside; exterior. open air.

外(ほか)other (places and things); the rest.

外す(はずす)to unfasten. to remove. to leave. to miss (a target).

外人(ガイジン)foreigner, foreign person.

外国(ガイコク)foreign country.

外界(ガイカイ)outside world.

海外(カイガイ)foreign; abroad; overseas.

外見(ガイケン)outward apperarance.

\section{2 刀 5 刊 7 利判}

刊行(カンコウ)publication; issue.
月刊(ゲッカン)monthly publication; monthly issue.
夕刊(ユウカン)evening newspaper.

利(リ)
advantage; benefit; profit;
interest (the amount added on top of the principal that is paid back).

判(ハン、わか)judge.
判る(わかる)

\section{2 力 4 木介 5 田 9 重}

力(リョク、リキ、ちから)strength.

百人力(ヒャクジンリキ)tremendous strength.

木(モク、き)tree.

介(カイ)can mean mediation or shellfish.

仲介(チュウカイ)agency; intermediation.

介入(カイニュウ)intervention.

介助(カイジョ)help; assistance; aid.

魚介(ギョカイ)marine products; seafood.

田(た)rice field.

重 depicts a man carrying a bag.

重い(おもい)heavy (of weight).

\subsection{(2 力) 5 功 8 協}

功(コウ)achievement

協(キョウ)cooperation.

\subsection{(4 木) 8 林 12 森}

林(リン、はやし)woods; forest; copse; thicket.

森(シン、もり)forest.

森林(シンリン)forest woods.

\subsection{(4 木) 8 果 11 菓}

果(カ)fruit.

菓(カ)confectionery.

菓子(カシ)pastry; confectionery.

\subsection{(4 木) 6 米}

米(ベイ、マイ、こめ)rice.

\subsection{(4 木) 5 本 6 休 7 体}

本(ホン)book, counter for long cylindrical objects.
(In Ancient China, a book is a scroll.)

一本(イッポン)one long cylindrical object, one point.

日本(ニホン)Japan.

休(キュウ)rest.

休日(キュウジツ)day off; holiday.

休む(やすむ)to rest.

体(タイ、からだ)body.

肉体(ニクタイ)body; flesh.

重体(ジュウタイ)seriously ill; critically ill.

\subsection{(4 木) 5 禾 9 秋 11 菌}

禾 depicts a plant stalk.

秋 depicts the burning of plant stalks (after harvest).

秋(シュウ、あき)autumn; fall season.

菌(キン)fungus; germ; bacterium.

\subsection{(4 木) 9 茶}

茶(チャ、サ)tea.

\subsection{(4 木) 5 未末 7 来}

未 depicts a tree that has not fruited.

未(ミ)not yet.

来 depicts fruits hanging on a tree.

未来(ミライ)future (lit. not yet come).

末(マツ、すえ)end.

来(ライ)come; next.

来月(ライゲツ)next month.

来年(ライネン)next year.

「来年田中さんが日本に行きます。」Next year, Mr. Tanaka will go to Japan.
(``Next year'' means one year after the moment the speaker says it.)

来る(くる)to come.
This is an irregular verb.
The past form is 来た(きた).

\subsection{(4 木) 13 楽}

楽 depicts a wooden stringed musical instrument.

楽(ガク、ラク)pleasure.

楽天(ラクテン)optimism.

音楽(オンガク)music.

楽しい(たのしい)happy.

\subsection{(4 田) 7 町}

町(チョウ、まち)town.

小町(こまち)belle; town beauty.

\subsection{(4 田) 7 男 9 勇}

男(おとこ)man.

勇(ユウ)courage.

勇む(いさむ)to be in high spirits.

通(ツウ)pass through.

通る(とおる)to go by.

\subsection{(4 田) 9 界 12 堺}

界(カイ)world.

世界(セカイ)world.

学界(ガッカイ)academic world.

業界(ギョウカイ)industry world, business world.

堺(カイ、さかい)world.

\subsection{(4 田) 7 里 11 理野 12 量}

里(リ、さと)hometown.

理(リ、ことわり)reason.

心理(シンリ)state of mind; mentality; psychology.

料理(リョウリ)cooking; cookery; cuisine.

野(ヤ、の)field; plains; rustic.

量(リョウ)quantity.

大量(タイリョウ)large quantity.

\subsection{5 甲 8 押}

甲(コウ)armor.

押(オウ)push.

押し(おし)(n) push.

押す(おす)to push; to press; to cram into; to force.
to stamp.
to overwhelm.

\subsection{5 由 6 因 7 困}

由(ユ、ユウ、よし)cause; reason.

自由(ジユウ)(n,adj-na) freedom; liberty.
(exp) as it pleases you.

自由なソフトウェアlibre software.
\ruby{無}{ム}\ruby{料}{リョウ}ソフトウェアgratis (no-fee) software.

因(イン)cause; factor

因る(よる)(vi) to be caused by.

因みに(ちなみに)(conj) by the way.
Usually written ちなみに.

困(コン)quandary.

困る(こまる)(vi) to be troubled; to be embarrassed.

\subsection{(5 由) 8 抽届}

抽(チュウ)pluck.

抽出(チュウシュツ)selection (from a group); sampling.

届(カイ)deliver.

届ける(とどける)(v1,vt) to make sure of.

\subsection{11 動}

動(ドウ)motion.

不動(フドウ)immobile.

動画(ドウガ)animation, motion picture.

自動車(ジドウシャ)automobile.

動力(ドウリョク)power; motive power.

動く(うごく)(vi) to move.

\section{2 又 5 求}

又 depicts the right hand.

又(また)again; also.

求(キュウ)request.

求める(もとめる)to want

\subsection{4 収反 8 取}

収(シュウ)obtain; income (monetary).

年収(ネンシュウ)annual income.

収める(おさめる)to obtain.

反(ハン)anti-.

反する(ハンする)to oppose; to rebel; to revolt.

反体制(ハンタイセイ)anti-establishment.

取る(とる)(vt) take; fetch; take up.
買い取り(かいとり)purchase; sale. purchase on a non-return policy.

\subsection{4 攴支 9 度}

攴(ぼくづくり)represents folding chair.

支(シ)branch.

度(ド、たび)degrees.

\ruby{毎}{マイ}\ruby{回}{カイ}\ruby{来}{く}る\ruby{度}{たび}
every time the implied entity comes.

今度(コンド)
now; this time; this occurrence.
next time; another time.

もう一度(もうイチド)once more.

\subsection{(4 攴 5 求) 11 救}

救(キュウ)salvation.

\subsection{4 友 7 抜 13 暖}

友(ユウ、とも)friend.

抜(バツ、ぬ)slip out.
抜ける(ぬける)(v1,vi) to escape.

暖かい(あたたかい)(adj-i) warm; genial

\subsection{8 受 11 授}

受ける(うける)(v1,vt) to receive

授ける(さずける)(v1,vt) to grant; to award.
授受(ジュジュ)give-and-receive.

\section{2 勹 3 勺 5 包 8 抱}

勹 is Kangxi radical 20 meaning ``wrap''.

勺 depicts something in the spoon.

包(ホウ)wrap.

包む(つつむ)(vt) to wrap up; to pack.

抱(ホウ)embrace; hug.

抱く(だく)to embrace; to hug.

\subsection{(3 勺) 8 的 9 約}

的(テキ)(suffix) -like; typical.

男性的(ダンセイテキ)manly.

的(テキ、まと)mark; target.

目的(モクテキ)purpose; goal; aim; objective; intention.

約(ヤク)promise.

約束(ヤクソク)arrangement; promise; pact; engagement.

\subsection{6 危}

危(キ、あや)dangerous.

危急(キキュウ)emergency.

危険(キケン)danger.

危うい(あやうい)dangerous.

危ない(あぶない)dangerous.

\section{Cardinal directions: 9 南}

南(ナン、みなみ)south

They combine as in English.

北西(ホクセイ)northwest

北東(ホクトウ)northeast

東北(トウホク)Touhoku (a prefecture)

南西(ナンセイ)southwest

南東(ナントウ)southeast

\section{Thing: 6 件 8 物事}

件(ケン)matter; case; item

物(ブツ、モツ、もの)thing; object; matter.

書物(ショモツ)books.

食べ物(たべもの)food.

物語(ものがたり)tale; story; legend.

物語る(ものがたる)(vt) to tell; to indicate.

火事(カジ)fire (as a disaster).

ラメン店で火事fire at a ramen shop.

有事(ユウジ)emergency.

無事(ブジ)safety; peace; quietness.

\section{Blades}

\subsection{Tangible cutting: 4 切}

切 depicts spoon and sword.
切(セツ、サイ、き)cut.
切る(きる)(v5r) cut.
大切(タイセツ)(adj-na,n) important.
一切(イッサイ)absolutely; (when used with negative) at all.

\subsection{Intangible cutting: 4 分}

分 depicts something separated by a blade.
分(フン、ブン)minute.
1分(イップン)one minute.
12時34分(ジュウニジサンジュウヨンプン)12:34 (time).

分(わ)understand.
分かる(わかる)to be understood.

分ける(わける)(v1,vt) to divide; to split; to share; to distribute.

\subsection{Swordtip: 4 方}

方 depicts the tip of a sword.
方(ホウ、かた)direction.
方(かた)(honorific) person.
あの方(あのかた)that person.

\subsection{Separation: 7 別}

別 depicts sword cutting bone.
別な(ベツな)(adj-na) different; separate; another
日付別(ひづけベツ)separate by date.

\section{2 丁 5 庁打}

丁(ひのと)depicts a nail.
丁(チョウ)(suffix)
counter for long and narrow (relatively flat) thing
such as paper, guns, scissors, spades, hoes, guitars.
一丁(イッチョウ)one sheet; one page.
one serving (in a restaurant).
丁重(テイチョウ)polite; courteous; hospitable.
包丁(ホウチョウ)kitchen knife.

丁(チョウ)town section.

庁(チョウ)government office.

打(ダ)strike; hit; knock; pound.

安打(アンダ)safe hit (baseball).

打つ(うつ)to hit; to strike; to knock; to beat; to punch; to slap.

\section{2 刀 9 削前契 13 解}

削(サク)shave; sharpen; delete.

削る(けずる)to shave; to sharpen; to erase; to delete.

前(ゼン、まえ)before (time), in front of.

午前(ゴゼン)morning; before noon; a.m. (ante meridien).

契(ケイ)pledge.

契約(ケイヤク)contract.

契る(ちぎる)(v5r,vt) to pledge; to promise; to swear.

解(ゲ、カイ)untie.

了解(リョウカイ)understanding; roger that.

見解(ケンカイ)opinion; point of view.

専門家見解(センモンカケンカイ)expert opinion.

解熱(ゲネツ)alleviation of fever.

解約(カイヤク)cancellation of contract.

解禁(カイキン)lifting of a ban.

解く(とく)to solve; to answer. to untangle (hair).

\section{2 力 7 努 12 筋}

努める(つとめる)(v1,vt) to endeavor; to try; to strive.
努力(ドリョク)great effort; exertion; endeavor

筋肉(キンニク)muscle.

\section{2 人 7 伴}

伴(ハン、バン)consort.

伴う(ともなう)to accompany.

\section{Blades: 2 刀刂 3 弋 4 戈 5 戊 6 戌 9 咸}

刀(トン、かたな)sword.

刂 depicts a knife.

弋 depicts shooting with a bow and an arrow.

戈 depicts a spear-axe (a halberd).

戊 depicts a dagger-axe (an ancient Chinese weapon).

戌 depicts an axe.

咸 depicts an axe and a mouth.

\section{(4 戈) 12 減 13 戦賊}

減(ゲン)reduction; 10\%減 ten percent reduction.

戦(いくさ)war.
内戦(ナイセン)civil war.
世界大戦(セカイタイセン)World War.

賊(ゾク)burglar; robber.
海賊(カイゾク)pirate; sea robber.

\chapter{Kanji 3}

\section{Components: 14 菐}

菐 is the 14-stroke ``thicket'' radical.

\section{6 危}

危ない(あぶない)dangerous

\section{7 改花努}

改める(あらためる)(v1,vt) to revise.

花(はな)flower

努める(つとめる)(v1,vt) to endeavor; to try; to strive.
努力(ドリョク)great effort; exertion; endeavor

\section{8 店河例}

店(テン)(n) store; shop.
ラメン店ramen shop (ramen is a kind of Japanese noodle).

河(かわ)river; stream.
河川(カセン)rivers.
大河(タイガ)large river.

例えば(たとえば)for example.
例える(たとえる)(v1,vt)
to compare; to liken; to illustrate.
用例(ヨウレイ)example; illustration.

\section{9 品草茶垢信計訂}

品(しな)article; item; thing; goods; stock.
品(ヒン)quality.
品物(しなもの)goods.
作品(サクヒン)work (book; film; composition; etc.). opus.
日用品(ニチヨウヒン)daily necessities.
一品(イッピン)one item; one article; one course of meal.
上品(ジョウヒン)elegant; refined; polished.
品性(ヒンセイ)character (elegant attitude; elegant behavior).

草(くさ)grass

茶(チャ) tea

垢(あか)dirt; filth; grime

信(シン)faith; trust.
信じる(シンじる)(v1,vt) to believe; to have faith in.

計(ケイ)plan.
計画(ケイカク)plan; project; schedule; scheme; program; programme.

訂正(テイセイ)correction; revision; amendment

\section{10 帰討通時唇特}

帰る(かえる)(v5r,vi)
to return; to come home; to go home; to go back.

討(トウ)chastise.
討伐(トウバツ)subjugation; suppression.
討つ(うつ)to shoot at; to attack, defeat, destroy, avenge.

通(ツウ)pass through.
通る(とおる)to go by.

時(とき)time.
時代(ジダイ)era.
三国時代(サンゴクジダイ)The Three Kingdoms period.
戦国時代(センゴクジダイ)The Warring States period.

唇(くちびる)(n) lips.

特(トク)special.
特に(トクに)particularly; especially.
特技(トクギ)special skill.

\section{11 異野転接菌}

異(イ)uncommon; different; unusual.
異国(イコク)foreign country.
異性(イセイ)different sex; opposite sex.
異なる(ことなる)to differ; to vary; to disagree.

野(ヤ、の)field; plains; rustic.

転(テン)revolve; turn around; change.
反転(ハンテン)rolling over; turning around.
転ぶ(ころぶ)(vi) to fall down; to fall over.

接(セツ)touch; contact.
接ぐ(つぐ)to join (two things into one); to piece together; to graft.
接吻(セップン)kiss.
接する(セッする)to come in contact with; to touch.

菌(キン)fungus; germ; bacterium

\section{11 経済脳情啓祭教清}

経(キョウ)sutra; Buddhist scripture.
経つ(たつ)(vi) to pass; to lapse.
経る(へる)to pass; to elapse; to experience.

済ませる(すませる)(v1,vt) to finish; to end.
経済(ケイザイ)economy.

脳内(ノウナイ)intracranial; inside the brain

情(ジョウ)feelings; emotion; passion; sympathy.

啓(ケイ).
拝啓(ハイケイ):拝啓… Dear ...

祭り(まつり)feast; festival

教え(おしえ)teaching; doctrine.
教育(キョウイク)training; education.
教会(キョウカイ)church.

清い(きよい)clear; pure; noble

\section{12 堺堕貿普奥詞筋報}

堺(カイ、さかい)world.

堕(ダ)degeneration; degradation.
堕胎(ダタイ)abortion; feticide; babykilling.
堕落(ダラク)depravity; corruption; degradation.

貿(ボウ)trade.
貿易(ボウエキ)trade (foreign)

普(フ)universal; wide; general.
普通(フツウ)general; ordinary; usual.
普通の人間(フツウのニンゲン)ordinary human.

奥(おく)interior.
奥山(おくやま)remote mountain.

詞(シ).
名詞(メイシ)noun.

筋肉(キンニク)muscle.

報(ホウ)report; information; news.
報じる(ホウじる)(v1,vt) to report; to inform.
報いる(むくいる)(v1,vt) to reward; to recompense; to repay.

\section{13 違僧話詳園数禁準楽暖業}

相違(ソウイ)difference; discrepancy; variation.
違う(ちがう)(vi) to differ; to not match the correct answer.
間違う(まちがう)to make a mistake; to be incorrect; to be mistaken.

僧(ソウ)monk; priest.

話す(はなす)to talk.
会話(カイワ)conversation.

詳

園(エン)park; garden; yard; farm.
動物園(ドウブツエン)zoo; animal park; zoological garden.
幼稚園(ヨウチエン)kindergarten.

数(かず)number; amount.
数える(かぞえる)(v1,vt) to count; to enumerate.
算数(サンスウ)arithmetics.
数万(スウマン)tens of thousands.

禁じる(キンじる)(v1,vt) to prohibit.

準(シュン).
水準(スイジュン)water level. level; standard.

楽しい(たのしい)happy.
音楽(オンガク)music.

暖かい(あたたかい)(adj-i) warm; genial

業(ギョウ)industry.
工業(コウギョウ)manufacturing industry.

\section{14 際僕概}

算(サン)calculation.
算出(サンシュツ)calculation; computation.
加算(カサン)addition.
引き算(ひきザン)subtraction.
公算(コウサン)probability; likelihood.

際(サイ)occasion; circumstances.
際限(サイゲン)limits; bounds.
学祭(ガクサイ)interdisciplinary.
国際(コクサイ)international.

僕(ボク)I; me (male).

概要(ガイヨウ)outline; summary

\section{15 膝質談線監編稿誰}

膝(ひざ)knee; lap

質(シツ)(suffix) substance; quality; matter.
質問(シツモン)question.

談(ダン)discuss.
相談(ソウダン)consultation.
示談(ジダン)out-of-court settlement.
座談会(ザダンカイ)symposium; round-table discussion.

線(セン)line; stripe.
line (telephone line).
line (of a railroad).
打線(ダセン)baseball lineup.

監(カン)government official; rule; administer.
監禁(カンキン)confinement; bondage.

編む(あむ)(vt)
to knit; to plait; to braid.
to compile (an anthology); to edit.

稿(コウ)draft; copy; manuscript

誰(だれ)who

\section{16 親諦頭}

親(おや)parent.
両親(リョウシン)both parents.

諦める(あきらめる)(v1,vt)
to give up; to abandon.

頭(あたま)head

\section{18 顔曜類}

顔(かお)face.
顔面(ガンメン)face (of a person).

曜(ヨウ)(weekday name).
日曜日(ニチヨウビ)Sunday.

類(ルイ)kind; sort; type.
人類(ジンルイ)mankind.

\section{Written: 10 記 14 読語}

記(キ)record.
記す(しるす)to record, to write down.
記録(キロク)record.
記事(キジ)article (writing).
選り抜き記事(よりぬきキジ)selected articles.
新しい記事(あたらしいキジ)new articles.

読む(よむ)to read.

語(ゴ)language.
日本語(ニホンゴ)Japanese language.
英語(エイゴ)English language.

\section{Craft: 3 工}

工作(コウサク)work; construction; handicraft.

\section{Countries}

日本(ニホン)Japan

中国(チュウゴク)People's Republic of China

英国(エイコク)United Kingdom

米国(ベイコク)United States of America

\section{Personal data}

出身(シュッシン)person's origin (town, city, country, etc.)

出身地(シュッシンチ)birthplace

誕生日(タンジョウビ)birthday; birth date; day of birth.

身長(シンチョウ)height (of body)

体重(タイジュウ)body weight

血液型(ケツエキガタ)blood type.「A型」type a.

好きなもの(すきなもの)likes

嫌いなもの(きらいなもの)dislikes

\section{Ungrouped}

警察(ケイサツ)police

素晴らしい(すばらしい)

卒業(ソツギョウ)

擦り傷(すりきず)(n) scratch; graze; abrasion

今度(コンド)
now; this time; this occurrence.
next time; another time.

率(リツ)(suf) rate; ratio; proportion.
識字率(シキジリツ)literacy rate.

…階建て(…カイだて)(suf) ...-story building.
7階建て 7-story building.

13
了解(りょうかい)understanding; roger that.
見解(ケンカイ)opinion; point of view.
専門家見解(センモンカケンカイ)expert opinion.

16 操る(あやつる)(vt) to be fluent in (a language)

14 増える(ふえる)(v1,vi) to increase; to multiply

17 覧 perusal

15 選 elect; select

13 連携(レンケイ)collaboration; cooperation

喧嘩(ケンカ)fight; brawl

博打(バクチ)gambling

お絵描き(おエかき)oekaki; painting; drawing

ネタバレspoiler (of a movie, a story, etc.); something that spoils the end of a movie, a story, etc.

関係(カンケイ)relation; connection

肉体関係(ニクタイカンケイ)sexual relations

垢と一切関係ないIt has absolutely nothing to do with dirt.

\chapter{Kanji 4}

\section{10 姫}

姫(ひめ)princess.

\section{11 終}

最終(サイシュウ)last; final; closing.
終わる(おわる)to finish; to end; to close.

\section{11 率}

率(リツ)(suf) rate; ratio; proportion.
識字率(シキジリツ)literacy rate.

\section{12 備}

備(ビ)provision.
備考(ビコウ)note; remarks; nota bene; NB.
備える(そなえる)to provide; to equip; to install.

\section{12 貿}

貿(ボウ)trade.
貿易(ボウエキ)trade (foreign)

\section{13 違 15 遺}

相違(ソウイ)difference; discrepancy; variation.
違う(ちがう)(vi) to differ; to not match the correct answer.
間違う(まちがう)to make a mistake; to be incorrect; to be mistaken.
違反(イハン)violation of law.

遺(イ、ユイ、のこ)bequeath; leave behind.
遺児(イジ)orphan.
遺物(イブツ)relic; memento.
遺品(イヒン)articles left by the deceased.
遺書(イショ)will; testament.
遺体(イタイ)corpse; remains.
遺言(ユイゴン)will; testament.
遺す(のこす)to bequeath; to leave behind; to save; to reserve.

\section{13 愛続}

愛(アイ)(n) love.
愛す(あいす)(vt) to love.

続(ゾク、つづ)continue.
相続(ソウゾク)succession; inheritance.
存続(ソンゾク)duration; continuance.
続く(つづく)to continue.

\section{14 暮魂}

暮(ボ、くら)livelihood.
暮らす(くらす)to live; to get along.

魂(たましい)soul.

\section{15 暴 19 爆}

暴 depicts the antler of a buck, representing a savage attack, a violence.
暴動(ボウドウ)insurrection; rebellion; revolt; riot; uprising.
暴風(ボウフウ)storm; windstorm; gale.
暴れる(あばれる)(v1,vi) to rage; to act violently.

爆(バク)burst; explode; bomb.
自爆(ジバク)suicide bombing; self-destruct.
水爆(スイバク)hydrogen bomb.
原爆(ゲンバク)atomic bomb; nuclear bomb.
空爆(クウバク)aerial bombing; air raid.
爆殺(バクサツ)killing by bombing.
爆死(バクシ)death by explosion.
爆音(バクオン)sound of explosion or detonation.

\section{15 熱趣撃}

熱い(あつい)(adj)hot (temperature)

趣味(シュミ)hobby; taste, preference.

電撃(デンゲキ)electric shock.

\section{16 燃}

燃える(もえる)(v1,vi) to burn; to get fired up

\section{16 操}

操る(あやつる)(vt) to be fluent in (a language)

\section{17 優}

優しい(やさしい)tender; kind; gentle; affectionate; suave

\section{17 覧}

覧(ラン)perusal.
回覧(カイラン)circulation.
回覧板(カイランバン)circular notice
(especially those distributed to households within a neighborhood association).

\section{20 議}

議(ギ)deliberation; consultation; debate; consideration.
議論(ギロン)argument; discussion; dispute; controversy.

\section{21 魔}

邪魔(ジャマ)intrusion.
邪魔するto intrude.

\section{Confusing characters}

\subsection{Spirit and cloth: 4 礻 5 衤}

\subsection{3 夂夊 4 攵}

In the Japanese language,
these characters become parts of other characters
instead of being used on their own.

夂 depicts two legs followed by something from behind.

夊 depicts a footprint.

攵 is a variant of 攴 depicting a branch and a hand.

\subsection{Samurai and earth: 3 士土}

士(samurai) has longer upper horizontal stroke.
土(earth) has shorter upper horizontal stroke.

\subsection{Hat, sun, moon, meat, inner}

冃(hat)

日(sun)

月(moon)

\subsection{石 (stone) and 右 (right)}

\subsection{人 (person) and 入 (enter)}

\subsection{王 (king) and 生 (sprout)}

\subsection{業菐美}

13 業 contains 业未.
14 菐 contains 业土人.

\section{Calendar}

These kanji readings for today, yesterday, and tomorrow are irregular.

今日(きょう)today

昨日(きのう)yesterday

明日(あした)tomorrow

Names of weekdays.

日曜日(ニチヨウび)Sunday

月曜日(ゲツヨウび)Monday

火曜日(カヨウび)Tuesday

水曜日(スイヨウび)Wednesday

木曜日(モクヨウび)Thursday

金曜日(キンヨウび)Friday

土曜日(ドヨウび)Saturday

毎日(マイニチ)everyday

Expressions.

また明日(あした)see you again tomorrow; means 'again' and 'tomorrow'

\section{Homophones}

\subsection{おさめる: 4 収 8 治 10 納}

収(シュウ)income (monetary).
年収(ネンシュウ)annual income.
収める(おさめる)see 納める below.

納める(おさめる)(v1,vt)
to dedicate; to make an offering to; to pay (fees) to.
収納(シュウノウ)crop; harvest.
未納(ミノウ)default; failure to pay; overdue payment.

治(おさむ)(name) Osamu.
治る(なおる)(vi) to heal.
治める(おさめる)(v1,vt)
to dominate; to rule; to govern; to manage.
to tranquilize; to pacify; to subdue.
to suppress.
政治(セイジ)politics.

\subsection{うつ:}

\chapter{Logic}

Every: 6 毎(マイ)

Negation: 4 不反 12 無

Versus: 7 対

Whole: 6 全

Opposite: 6 向

\section{Productive abstract concepts}

Turning-into: 4 化

Repetition: 6 再

Most: 12 最

\section{Existence and truth}

Truth: 10 真

Existence: 6 在存有

\chapter{Scratch space}

\section{12 塔}

塔(トウ)tower

管制塔(カンセイトウ)control tower

\section{Not yet categorized}

頭(あたま)head

歌う(うたう)to sing

笑う(わらう)to laugh; to smile

若い(わかい)

TODO

the right-side radical of 使う(つかう)to use

喧嘩(ケンカ)fight; brawl

博打(バクチ)gambling

北東(ホクトウ)northeast

空(そら)sky

空港(クウコウ)airport

間違う(まちがう)

百人力(ヒャクジンリキ)tremendous strength (lit. one-hundred-people strength)

辞書(ジショ)dictionary

危ない(あぶない)dangerous

政治(セイジ)politics

経済(ケイザイ)economy

国際(コクサイ)international

写真(シャシン)multimedia; photograph; movie (?)

毒ガス(ドクガス)poison gas

連携(レンケイ)collaboration; cooperation

意欲(イヨク)motivation; will

貿易(ボウエキ)trade (foreign)

対(タイ)versus... (usage?)

お絵描き(おエかき)oekaki; painting; drawing

ネタバレspoiler (of a movie, a story, etc.); something that spoils the end of a movie, a story, etc.

堕ちる(おちる)(v1,vi)to fall down; to drop (?)

脳内(ノウナイ)intracranial; inside the brain

授乳(ジュニュウ)breast-feeding

質問(シツモン)question; inquiry; enquiry

問題(モンダイ)problem

吸う(すう)to suck with mouth

仕事(シごと)work

仕方(シかた)way.仕方ないit can't be helped; there's no other way.

平和(ヘイワ)peace.平和だpeace (used in a post that may easily anger the reader) (?)

垢(あか)dirt; filth

関係(カンケイ)relation; connection

肉体関係(ニクタイカンケイ)sexual relations

垢と一切関係ないIt has absolutely nothing to do with dirt.

趣味(シュミ)hobby; taste, preference.

邪魔(ジャマ)intrusion
邪魔するto intrude

接吻(セップン)kiss

待つ(まつ)(vt,vi) to wait; to wait for; to await.

春(はる)spring (season)

夏(なつ)summer

冬(ふゆ)winter

一寸(ちょっと)just a minute; short time; just a little.
「一寸待って下さい。」``Please wait a moment.''
「一寸!」``Hey!''

諦める(あきらめる)(v1,vt)to give up; to abandon

一部
方
需要
編
一般
閲覧
お気

今日も
元気
な膝小僧ときれいな
お目眼
多少画質

お巡り(おまわり)
policeman

ぐるぐる

警察(ケイサツ)police

専門家(センモンカ)expert; specialist

見解(ケンカイ)opinion; point of view

専門家見解(センモンカケンカイ)expert opinion

納得(ナットク)understanding, agreement.
納得するto consent, agree; to understand the reason for something.

番組(バンぐみ)television program

人間(ニンゲン)humanity?

日本人(にホンジン)Japanese person

禁じる(キンじる)(v1,vt) to prohibit

監禁(カンキン)confinement; bondage

% https://en.wiktionary.org/wiki/%E8%BE%9B%E3%81%84#Japanese
% https://en.wiktionary.org/wiki/%E8%BE%9B#Japanese

飛ぶ(とぶ)

暖かい(あたたかい)(adj-i) warm; genial

素晴らしい(すばらしい)

天才(テンサイ)genius; prodigy.
Sometimes 3 strokes, sometimes 4 strokes.

卒業(ソツギョウ)

議論(ギロン)

傘(かさ)

すてきlovely; beautiful; dreamy; great; superb; cool

求める(もとめる)to want

擦り傷(すりきず)(n) scratch; graze; abrasion

今度(コンド)
now; this time; this occurrence.
next time; another time.

子供を作るto make a child

受賞(ジュショウ)winning (a prize)

次(つぎ)next (in sequence) (?)

拝啓(ハイケイ):拝啓… Dear ...

水準(スイジュン)water level. level; standard.

率(リツ)(suf) rate; ratio; proportion.
識字率(シキジリツ)literacy rate.

木造(モクゾウ)wooden; made of wood

…階建て(…カイだて)(suf) ...-story building.
7階建て 7-story building.

関連ニュース(カンレンニュース)related news

団地(ダンチ)multi-unit apartments

了解(りょうかい)understanding; roger that

息子(むすこ)son

娘(むすめ)daughter

名詞(メイシ)noun

\chapter{Grammar}

% https://en.wikipedia.org/wiki/Japanese_verb_conjugation

\section{Noun clause}

「\ruby{文}{も}\ruby{字}{じ}を\ruby{読}{よ}み\ruby{書}{か}き\ruby{出}{で}%
\ruby{来}{き}ない\ruby{子}{こ}\ruby{供}{ども}たちの\ruby{未}{み}\ruby{来}{らい}」means
``the future of children who cannot read and write Chinese characters''.
The primary noun is 「未来」 (``future''),
which is modified by the adjective
「文字を読み書き出来ない子供たちの」
(``of children who cannot read and write Chinese characters'').
「文字を読み書き出来ない」(``unable to read and write Chinese characters'')
is an adjective that modifies 「子供たち」 (children).

\section{Vocabulary}

本(ほん)book

魚(さかな)fish

食べる(たべる)eat

読む(よむ)read

(S) is the implied subject.

\section{Present imperfect}

魚を食べる。(S) eats fish.

魚を食べない。(S) doesn't eat fish.

\section{Past perfect}

魚を食べなかった。(S) didn't eat fish.

本を読んだ。(S) read a book.

\section{Command}

魚を食べなさい。Eat fish.

魚を食べな。(childspeak?) Eat fish.

魚を食べて。Eat fish.

魚を食べないで。Don't eat fish.

\section{Passive}

食べられた。(S) was eaten.

読まれた。(S) was read.

\section{I want to ...}

食べたい。I want to eat.

読みたい。I want to read.

\section{Polite}

You have to learn to say the same thing all over again in different registers.

食べます(polite) to eat

食べません(polite) not eat

食べました(polite) ate

魚を食べました。(polite) (S) ate fish.

魚が食べられました。(polite)Fish was eaten. (???)

魚を食べません。(polite) (S) doesn't eat fish.

魚を食べないでください。(polite command) Please do not eat fish.

\section{But, although, despite, in spite of}

でも

けど

けれど

\section{Then}

そして

\section{So, thus, therefore}

だから

\section{To, for}

\section{From, to (place)}

から

より

\section{From, until (time)}

\section{In order to}

\section{Nevertheless}

\section{Whereas}

\section{On the other hand}

\section{Too bad ...}

\section{Because, because of, due to}

から

たら

ば

\section{About, regarding}

ついて

\section{About, approximately}

くらい

\section{Hmm...}

あの

えと

\section{Huh?}

え?

おれ?

あれ?

\section{Let's ...}

\section{Don't ...}

\section{Please ...}

\section{Are you sure?} 

\section{Why not ...}

\section{I think ...}

\section{Perhaps ...}

たぶん

もしかして

\section{I mean..., What I'm trying to say is..., How do I say this...}

\section{Absolutely}

\section{Precisely}

\section{Actually}

\section{Frankly}

\section{Indeed}

\section{Anyway}

\section{I'm sorry}

すま

すみません

ごめん

ごめんなさい

\section{Excuse me}

しつれいします

おじゃまします

\section{Long time no see}

おひさしぶり

\section{How dare you...}

\section{Like it or not}

\section{Or else}

\section{Questions: 7 何 15 誰}

何(なに)what

誰(だれ)who

\section{Uncategorized}

として

さすが

まじで


\end{document}
