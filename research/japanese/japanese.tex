\documentclass[12pt,openany]{book}
\usepackage{xeCJK}
\usepackage{hyperref}
\usepackage[
    paperwidth=6in,paperheight=9in
    ,top=1.00in,bottom=1.00in
    ,inner=1.00in,outer=0.75in
]{geometry}
\setCJKmainfont{Droid Sans Fallback}
\begin{document}
\frontmatter
\phantomsection
\addcontentsline{toc}{chapter}{Cover}
\begin{titlepage}
{%
    \setlength\parindent{0em}%
    {\Large Bootstrap your Japanese with English}\par\vspace{2em}
    Erik Dominikus\par
    2017-04-02\par
    (Draft)\par
}
\end{titlepage}
\phantomsection
\addcontentsline{toc}{chapter}{\contentsname}
\tableofcontents
\mainmatter
\chapter{Strategy}

\section{Mindful consistent spaced repetition}

The key to remembering kanji is
\emph{mindful consistent spaced repetition}.
Don't just gloss over the kanji.
\emph{Internalize}.

Don't read one stroke at a time,
but read one component at a time.

\emph{Expect to forget a lot, many times}.
You will forget.
Reread this book.
This book is meant to be read and read again many times.
The aim of this book is to help you optimize your loop.

Mindful consistent spaced repetition is \emph{the only way},
at least until mind-upload is invented.
There is no shortcut.
It's just how the brain works.

\section{Target audience}

You read, write, speak, and listen to English but don't read, write, speak, and listen to Japanese.

\section{Memorize the syllabaries}

Memorize all hiragana and katakana.
It's a must.

\section{Understand the principles}

Understand that most kanjis are composed using other kanjis.

Use Wiktionary to found the oracle bone script version of Han characters.

Don't focus on both grammar and vocabulary at the same time.
One who chases two rabbits catches neither.

\section{Read and hear native speakers}

Go to YouTube.
Hear how they speak, even if you don't understand.

Use Google Translate, but do not trust it.
Sometimes it gives the wrong reading.
Use Wiktionary, Wikipedia.

Find Japanese people on Twitter and understand their tweets.

Find Japanese blog posts.

Set up an input method, or use Google Translate.

\chapter{Kanji}

Information is compiled from Wiktionary,
edict, compdic, kanjidic, gjiten,
and Japanese-to-English Google-Translate.

Japanese schoolgoers take 12 years to study 2,000 characters.

\section{History}

Japan borrowed Han characters.

To understand a character, see how it combines with other characters.

To keep you motivated, the characters have been sorted by number of strokes.

A kanji can have several readings:
go'on, kan'on, kun-yomi, name reading, and stylistic/idiosyncratic/ad-hoc reading.

新字体(シンジタイ)(lit. new Han-character body)

旧字体(キュウジタイ)(lit. old Han-character body)

A Latin alphabet letter encodes a \emph{sound}.
You can pronounce the word without understanding it.
A Han character encodes a \emph{concept}.
You can understand the meaning without pronouncing it.

Katakana shows Chinese reading and hiragana shows Japanese reading.

\section{Grouping}

The goal of grouping is to allow the brain to \emph{chunk}.
When the brain recalls any of the member of the group,
it automatically recalls other members of the group.

Use recursive grouping that is 4 levels deep because \(7^4 = 2401\).

\section{Grouping by reducibility}

Chapter \(n\) should not define a kanji
that has more than \(n\) components.
Each section in a chapter groups kanji by semantic closeness
(not by visual closeness).

Example of irreducibility:
台 (tower) is treated as one component;
for pattern recognition purposes,
it is not treated as 厶 stacked on 口.

\section{A possible grouping: scratchiness}

The characters are sorted ascending by ``scratchiness''.
Boxes are less scratchy than lines.
Lines are less scratchy than dots.
Symmetry is less scratchy than asymmetry.
Fewer strokes are less scratchy than more strokes.
However, this is subjective, approximate, and inconsistent.
I don't always follow my own rules.
The goal is to optimize the learning rate of the reader.
The reader is expected to read the character list
sequentially and repeatedly.

Level 1 looks like
an aspiring artist's minimalist abstract paintings.
Level 2 is a combinatorial explosion
of two juxtaposed abstract paintings.
Level 3 ramps up the difficulty even more with more strokes and less symmetry.
Finally, Level 4 looks like total chicken scratch to the uninitiated,
is barely legible on screen even in a 12-point sans-serif font,
and may raise questions such as
``You have a character for that?''
and
``What drug were the sages smoking?''

\section{Another grouping: recency}

Group characters invented in the same period together.
Ancient Chinese concepts,
Old Chinese concepts,
Middle Chinese concepts,
Contemporary Chinese concepts.

\section{Another grouping}

Use visual similarity to group characters by concept.

Visual similarity does not always imply conceptual similarity.

\section{Historical context}

All dates are approximate.

In 12,000 BCE, dogs had been domesticated.

In 4,000 BCE, Egypt had had papyrus.

In 2,500 BCE, Egypt had hieroglyphs.

In 1,000 BCE, China had had oracle bone writing.

In 479 BCE, Confucius died.

Around 400 BCE, Gautama Buddha died.

In 399 BCE, Socrates died.

In 206 BCE, China began the Han dynasty.

Around 100 BCE, China had paper.

Around 30, Jesus died.

In 220, China ended the Han dynasty.

In 632, Muhammad died.

In 1000, China had had gunpowder.

Around 1056, Benedict IX died.

In 1185, Japan ended the Heian period.
Japan also had imported Han characters from China.

In 1546, Martin Luther died.

In October 1582, Gregory XIII introduced
what would later be known as the Gregorian calendar.

In 1750, Johann Sebastian Bach died.

In 1819, James Watt died.

In 1823, David Ricardo died.

In 1865, Abraham Lincoln died.

In 1897, Henry George died.

In 1917, the Balfour Declaration supporting Zionism was made.

In 1945, Adolf Hitler died.

1942--1945: World War II

In 1946, John Maynard Keynes died.

In 1947, the Truman Doctrine was announced,
marking the beginning of the Cold War.

1950--1953: Korean War

In 1954, Alan Turing died.

In 1955, the Vietnam War began.

In 1961, Yuri Gagarin in Vostok 1
became the first man to travel into space.

In 1963, John Fitzgerald Kennedy died.

In 1968, Martin Luther King Jr. died.

In 1970, Sukarno died.

In 1975, the Vietnam War ended.

In 1980, Mohammad Hatta died.

In 1979--1989, the United States Central Intelligence Agency
carried out Operation Cyclone,
arming and financing Afghanistan mujahideen.

In 1986, Lafayette Ronald Hubbard died.

In 1991, the Soviet union ended, marking the end of the Cold War.

In 2001, the World Trade Center was destroyed.

In 2006, Milton Friedman died.

\chapter{Kanji 1}

\section{Confusing characters}

\subsection{3 夂夊 4 攵 8 㑒}

In the Japanese language,
these characters become parts of other characters
instead of being used on their own.

夂 depicts two legs followed by something from behind.

夊 depicts a footprint.

攵 is a variant of 攴 depicting a branch and a hand.

㑒 is simplified from the 13-stroke 僉
meaning ``all, together, unanimous''.

\subsection{士 (samurai) and 土 (earth)}

士(samurai) has longer upper horizontal stroke.
土(earth) has shorter upper horizontal stroke.

\subsection{Hat, sun, moon, meat, inner}

冃(hat)

日(sun)

月(moon)

\subsection{石 (stone) and 右 (right)}

\subsection{人 (person) and 入 (enter)}

\subsection{王 (king) and 生 (sprout)}

\subsection{業菐美}

\section{2 入 4 内 6 肉}

入る(いる)(v5)
to enter; to go into; to get into.

内(うち)inside

肉 (ribs of an animal's torso)

肉(ニク)meat; flesh; body (as opposed to spirit)

\section{4 之}

之(これ)this

\section{3 工山川上下 4 中止}

川(かわ)river

山(サン、やま)mount, mountain

上(うえ)up

下(した)down

工(コウ、ク)craft.
工学(コウガク)engineering.
大工(ダイク)carpenter.
木工(モッコウ)carpenter.
工(たくみ)(name) Takumi.

中(チュウ、なか)middle.
中学(チュウガク)middle school; junior high school.
…の中(…のなか)middle of something.
田中(たなか)(name) Tanaka.
中山(なかやま)(name) Nakayama.
中川(なかがわ)(name) Nakagawa.

止depicts a footprint.

止める(とめる)(v1) to stop moving (walking, etc.); to park (a car)

\section{5 平半}

平 can mean flat, level (not tilted), ordinary, plain, non-special.
平ら(たいら)flatness.
平たい(ひらたい)(adj-i) flat; even; level; simple.
平皿(ひらざら)flat dish.
平安(ヘイアン)peace; tranquility.
平気(ヘイキ)coolness; calmness; composure; unconcern.
平日(ヘイジツ)weekday; ordinary day (non-holiday).
平年(ヘイネン)normal (non-leap) year; normal year (related to harvest; weather).
公平(コウヘイ)fairness; impartiality; justice
水平(スイヘイ)level; horizontally
平文(ヘイブン)plain (non-encrypted) text.
平面(ヘイメン)level (flat and not-tilted) surface.

半(ハン)half

\section{4 日月 5 白 6 早 8 明}

日(ひ)sun.
日々(ひび)daily; days; old days.

月(つき)moon.

白(ハク)white.
白い(しろい)white.

早い(はやい)(adj-i) early.

明(メイ)bright.
明るい(あかるい)bright.

\subsection{9 星 12 朝}

日付(ひづけ)date.日付別(ひづけベツ)separate by date.(context? usage?)

星(ほし)star

朝(あさ)morning.
今朝(けさ)this morning.
早朝(ソウチョウ)early morning.

\section{4 王 5 玉生 8 国}

4 王(オウ)king

5 玉(ギョク、たま) ball

生depicts a sprout, something sprouting from the ground.
生(セイ)nature; sex; gender.
学生(ガクセイ)student.
生まれる(うまれる)(v1,vi) to be born.

8 国(コク、くに) country.
国王(コクオウ)king.
国内(コクナイ)internal; domestic.

\subsection{5 主 9 美皇}

5 主(おも)chief; main; principal; important

主人公(シュジンコウ)hero; main character

9 美 beauty

美味しい(おいしい)delicious (idiosyncratic reading)

美しい(うつくしい)beautiful

9 皇(コウ)imperial

皇居(コウキョ)imperial palace

\section{4 井开}

井 depicts a square well.
井(い)well (water reservoir).

开 is a simplification of 幵 depicting raising both hands.

\section{5 田由出石左右 6 回向}

田(た)rice field

由(よし)cause; reason.

石(いし)stone

出 depicts something coming out of an open box.
出る(でる)(v1,vi) to go out; to exit; to leave.
出来る(できる)(v1,vi) to be able to do.
出来上がる(できあがる)(vi) to be finished; to be completed; to be ready

左(ひだり)left

右(みぎ)right.

回 depicts a spiral.
回す(まわす)(vt) to turn; to rotate.
一回(イッカイ)once; one time.
一回目(イッカイめ)first.

向 depicts a house and a window.
向かい(むかい)(n) facing; opposite; across the street; other side.
向く(むく)to face; to turn toward.
向こう(むこう)opposite side; other side; opposite direction.
向上(コウジョウ)improvement; advancement; progress.

\section{5 皿}

皿(さら)dish, plate

\section{5 世 6 両}

世(よ)world; society; age; generation

両(リョウ)both

\section{7 車 8 店 9 面}

車(くるま)car

店(テン)(n) store; shop.
ラメン店ramen shop (ramen is a kind of Japanese noodle).

面(おもて)mask

面白い(おもしろい)interesting

\chapter{Kanji 2}

\section{Pictures: 5 写 7 図 8 画}

写生(シャセイ)sketch.
写真(シャシン)multimedia; photograph; movie
写す(うつす)(vt) to photograph.

図(ズ)map; drawing; picture; plan; illustration; diagram; figure; chart

画 means picture.
画(カク)counter for kanji strokes.
字画(ジカク)number of strokes in a character.
画家(ガカ)painter.

\section{Cardinal directions: 5 北 6 西 8 東 9 南}

北(ホク、きた)north

西(セイ、にし)west

東(トン、ひがし)east

南(ナン、みなみ)south

They combine as in English.

北西(ホクセイ)northwest

北東(ホクトウ)northeast

東北(トウホク)Touhoku (a prefecture)

南西(ナンセイ)southwest

南東(ナントウ)southeast

\section{Calendar}

These kanji readings for today, yesterday, and tomorrow are irregular.

今日(きょう)today

昨日(きのう)yesterday

明日(あした)tomorrow

Names of weekdays.

日曜日(ニチヨウび)Sunday

月曜日(ゲツヨウび)Monday

火曜日(カヨウび)Tuesday

水曜日(スイヨウび)Wednesday

木曜日(モクヨウび)Thursday

金曜日(キンヨウび)Friday

土曜日(ドヨウび)Saturday

毎日(マイニチ)everyday

Expressions.

また明日(あした)see you again tomorrow; means 'again' and 'tomorrow'

\section{Shops: 8 店}

店(テン)(n) store; shop.
ラメン店ramen shop (ramen is a kind of Japanese noodle).

\section{Myself: 2 厶 4 公 7 私}

厶 means I, myself, private.

公(おおやけ)public; communal; official; governmental.
公安(コウアン)public safety; public welfare.

私(わたし)I; me

\section{Inches: 3 寸 5 付 7 村 10 時}

% https://en.wiktionary.org/wiki/%E5%AF%B8#Japanese
寸depicts a position on the forearm
where the pulse can be palpated by compressing the radial artery.
寸(スン)an ancient unit of length, approximately 3 cm.

付ける(つける)(v1,vt) to attach, join, stick, glue, fasten.

村(むら)village

時(とき)time.
時代(ジダイ)era.
三国時代(サンゴクジダイ)The Three Kingdoms period.
戦国時代(センゴクジダイ)The Warring States period.

\section{Threads}

\subsection{4 幻 9 紅 10 紙}

幻(まぼろし)phantom; vision; illusion; dream

紅(くれない)deep red; crimson

紙(かみ)paper

手紙(てがみ)letter (the document, not the alphabet)

\section{Passable: 5 可 10 哥 14 歌}

可 passable
許可(キョカ)permission; authorization; approval.

哥 means older brother.
This character is not used on its own in Japan.

歌う(うたう)to sing

\section{Escape: 11 脱}

The left side is 肉.
脱出(ダッシュツ)escape.

\section{Foresight: 6 先 8 洗}

先生(せんせい)teacher; master; doctor

洗う(あらう)(vt) to wash

\section{Who: 15 誰}

誰(だれ)who

\subsection{Wide-narrow: 5 広 9 狭}

広い(ひろい)(adj-i) spacious; vast; wide.
広告(コウコク)advertisement.

狭める(せばめる)(v1,vt) to narrow.
狭い(せまい)(adj-i) narrow; confined; small.

\section{Metals: 8 金 13 鉄 14 銅銀 16 鋼}

金has a lot to do with metals.
金(キン、かね)gold; money.

金属(キンゾク)metal.
重金属(ジュウキンゾク)heavy metal.
These are also the chemistry terms.

金色(キンいろ、コンジキ)golden (color)

鉄(テツ、くろがね)iron (lit. 黒金 black metal).
鉄人(てつジン)iron man; strong man.

鉄道(テツドウ)railroad; railway

銅(ドウ、あかがね)copper (lit. 赤金 red metal)

銀(ギン、しろがね)silver (lit. 白金 white metal)

鋼(コウ、はがね)steel

青銅(セイドウ)bronze

鋼鉄(コウテツ)steel

\subsection{Firearms: 14 銃}

銃(ジュウ)gun; small firearms

\subsection{Mirror: 19 鏡}

8 金 + 5 立 + 7 見 - 1

鏡(かがみ)mirror

眼鏡(めがね)eyeglasses

\section{Trouble: 7 困}

困る(こまる)(vi) to be troubled; to be embarrassed

\section{Grass: 6 艸}

\subsection{7 花 9 草}

花(はな)flower

草(くさ)grass

\subsection{Tea: 9 茶}

茶(チャ) tea

\subsection{11 菌}

菌(キン)fungus; germ; bacterium

\section{Fire: 7 災 8 炎}

災い(わざわい)(n) calamity; catastrophe.
火災(カサイ)fire (disaster).

炎(ほのお)flame, blaze.
炎天(エンテン)scorching sun.

\subsection{15 熱 16 燃}

熱い(あつい)(adj)hot (temperature)

燃える(もえる)(v1,vi) to burn; to get fired up

火事(カジ)fire (disaster).
ラメン店で火事fire at a ramen shop.

\section{Gate: 12 間 12 開 14 関}

間 interval

開く(ひらく)to open (door, business, eye, mouth, ...)

関(カン、せき)barrier; gate

\chapter{Kanji 3}

\section{4 凶}

凶(キョウ)evil; villain; bad luck; disaster.
凶悪(キョウアク)(adj-na) atrocious; fiendish; brutal; villainous.

\section{6 危}

危ない(あぶない)dangerous

\section{7 努}

努める(つとめる)(v1,vt) to endeavor; to try; to strive.
努力(ドリョク)great effort; exertion; endeavor

\section{8 店}

店(テン)(n) store; shop.
ラメン店ramen shop (ramen is a kind of Japanese noodle).

\section{9 垢信計}

垢(あか)dirt; filth; grime

信(シン)faith; trust.
信じる(シンじる)(v1,vt) to believe; to have faith in.

計(ケイ)plan.
計画(ケイカク)plan; project; schedule; scheme; program; programme.

\section{10 通時唇特}

通(ツウ)pass through.
通る(とおる)to go by.

時(とき)time.
時代(ジダイ)era.
三国時代(サンゴクジダイ)The Three Kingdoms period.
戦国時代(センゴクジダイ)The Warring States period.

唇(くちびる)(n) lips.

特(トク)special.
特に(トクに)particularly; especially.
特技(トクギ)special skill.

\section{11 野転接}

野(ヤ、の)field; plains; rustic.

転(テン)revolve; turn around; change.
反転(ハンテン)rolling over; turning around.
転ぶ(ころぶ)(vi) to fall down; to fall over.

接(セツ)touch; contact.
接ぐ(つぐ)to join (two things into one); to piece together; to graft.
接吻(セップン)kiss.
接する(セッする)to come in contact with; to touch.

\section{11 経済脳情啓祭教清}

経(キョウ)sutra; Buddhist scripture.
経つ(たつ)(vi) to pass; to lapse.
経る(へる)to pass; to elapse; to experience.

済ませる(すませる)(v1,vt) to finish; to end.
経済(ケイザイ)economy.

脳内(ノウナイ)intracranial; inside the brain

情(ジョウ)feelings; emotion; passion; sympathy.

啓(ケイ).
拝啓(ハイケイ):拝啓… Dear ...

祭り(まつり)feast; festival

教え(おしえ)teaching; doctrine.
教育(キョウイク)training; education.
教会(キョウカイ)church.

清い(きよい)clear; pure; noble

\section{12 堺堕貿普奥詞筋}

堺(カイ、さかい)world.

堕(ダ)degeneration; degradation.
堕胎(ダタイ)abortion; feticide; babykilling.
堕落(ダラク)depravity; corruption; degradation.

貿(ボウ)trade.
貿易(ボウエキ)trade (foreign)

普(フ)universal; wide; general.
普通(フツウ)general; ordinary; usual.

奥(おく)interior.
奥山(おくやま)remote mountain.

詞(シ).
名詞(メイシ)noun.

筋肉(キンニク)muscle.

\section{13 違僧話園数禁準楽暖業}

相違(ソウイ)difference; discrepancy; variation.
違う(ちがう)(vi) to differ; to not match the correct answer.
間違う(まちがう)to make a mistake; to be incorrect; to be mistaken.

僧(ソウ)monk; priest.

話す(はなす)to talk.

園(エン)park; garden; yard; farm.
動物園(ドウブツエン)zoo; animal park; zoological garden.
幼稚園(ヨウチエン)kindergarten.

数(かず)number; amount.
数える(かぞえる)(v1,vt) to count; to enumerate.
算数(サンスウ)arithmetics.

禁じる(キンじる)(v1,vt) to prohibit.

準(シュン).
水準(スイジュン)water level. level; standard.

楽しい(たのしい)happy.
音楽(オンガク)music.

暖かい(あたたかい)(adj-i) warm; genial

業(ギョウ)industry.
工業(コウギョウ)manufacturing industry.

\section{14 際菐僕概}

算(サン)calculation.
算出(サンシュツ)calculation; computation.
加算(カサン)addition.
引き算(ひきザン)subtraction.
公算(コウサン)probability; likelihood.

際(サイ)occasion; circumstances.
際限(サイゲン)limits; bounds.
学祭(ガクサイ)interdisciplinary.
国際(コクサイ)international.

菐 is the 14-stroke ``thicket'' radical.

僕(ボク)I; me (male).

概要(ガイヨウ)outline; summary

\section{15 膝質談線監編稿誰撃}

膝(ひざ)knee; lap

質(シツ)(suffix) substance; quality; matter.
質問(シツモン)question.

談(ダン)discuss.
相談(ソウダン)consultation.
示談(ジダン)out-of-court settlement.
座談会(ザダンカイ)symposium; round-table discussion.

線(セン)line; stripe.
line (telephone line).
line (of a railroad).
打線(ダセン)baseball lineup.

監(カン)government official; rule; administer.
監禁(カンキン)confinement; bondage.

編む(あむ)(vt)
to knit; to plait; to braid.
to compile (an anthology); to edit.

稿(コウ)draft; copy; manuscript

誰(だれ)who

電撃(デンゲキ)electric shock.

\section{16 諦頭}

諦める(あきらめる)(v1,vt)
to give up; to abandon.

頭(あたま)head

\section{18 顔曜類}

顔(かお)face.
顔面(ガンメン)face (of a person).

曜(ヨウ)(weekday name).
日曜日(ニチヨウビ)Sunday.

類(ルイ)kind; sort; type.
人類(ジンルイ)mankind.

\section{Revision: 7 改 9 訂}

7 改める(あらためる)(v1,vt) to revise.

訂正(テイセイ)correction; revision; amendment

\section{Written: 10 記 14 読語}

記(キ)record.
記す(しるす)to record, to write down.
記録(キロク)record.
記事(キジ)article (writing).
選り抜き記事(よりぬきキジ)selected articles.
新しい記事(あたらしいキジ)new articles.

読む(よむ)to read.

語(ゴ)language.
日本語(ニホンゴ)Japanese language.
英語(エイゴ)English language.

\section{Calendar}

These kanji readings for today, yesterday, and tomorrow are irregular.

今日(きょう)today

昨日(きのう)yesterday

明日(あした)tomorrow

Names of weekdays.

日曜日(ニチヨウび)Sunday

月曜日(ゲツヨウび)Monday

火曜日(カヨウび)Tuesday

水曜日(スイヨウび)Wednesday

木曜日(モクヨウび)Thursday

金曜日(キンヨウび)Friday

土曜日(ドヨウび)Saturday

毎日(マイニチ)everyday

Expressions.

また明日(あした)see you again tomorrow; means 'again' and 'tomorrow'

\section{Grass: 7 花 9 草茶 11 菌}

花(はな)flower

草(くさ)grass

茶(チャ) tea

菌(キン)fungus; germ; bacterium

\section{Craft: 3 工}

工作(コウサク)work; construction; handicraft.

\section{Countries}

日本(ニホン)Japan

中国(チュウゴク)People's Republic of China

英国(エイコク)United Kingdom

米国(ベイコク)United States of America

\section{Personal data}

出身(シュッシン)person's origin (town, city, country, etc.)

出身地(シュッシンチ)birthplace

誕生日(タンジョウビ)birthday; birth date; day of birth.

身長(シンチョウ)height (of body)

体重(タイジュウ)body weight

血液型(ケツエキガタ)blood type.「A型」type a.

好きなもの(すきなもの)likes

嫌いなもの(きらいなもの)dislikes

\section{Ungrouped}

警察(ケイサツ)police

素晴らしい(すばらしい)

卒業(ソツギョウ)

擦り傷(すりきず)(n) scratch; graze; abrasion

今度(コンド)
now; this time; this occurrence.
next time; another time.

率(リツ)(suf) rate; ratio; proportion.
識字率(シキジリツ)literacy rate.

…階建て(…カイだて)(suf) ...-story building.
7階建て 7-story building.

13
了解(りょうかい)understanding; roger that.
見解(ケンカイ)opinion; point of view.
専門家見解(センモンカケンカイ)expert opinion.

16 操る(あやつる)(vt) to be fluent in (a language)

14 増える(ふえる)(v1,vi) to increase; to multiply

17 覧 perusal

12 報 report; news

15 選 elect; select

13 連携(レンケイ)collaboration; cooperation

喧嘩(ケンカ)fight; brawl

博打(バクチ)gambling

お絵描き(おエかき)oekaki; painting; drawing

ネタバレspoiler (of a movie, a story, etc.); something that spoils the end of a movie, a story, etc.

関係(カンケイ)relation; connection

肉体関係(ニクタイカンケイ)sexual relations

垢と一切関係ないIt has absolutely nothing to do with dirt.

\chapter{People}

\section{People}

\subsection{Followers and friends: 6 仲共 8 供}

仲間(なかま)friend (how does this differ from 友達(ともだち)?)

共(とも)companion; follower; attendant; retinue

子供(こども)child

\subsection{Returning: 7 戻}

戻る(もどる)(vi) to turn back; to return; to go back;
to come back to a previously visited place

\subsection{Children: 9 保}

保する(ほする)to guarantee

\subsection{Child-rearing: 8 学乳}

学(ガク)learning, scholarship, erudition, knowledge.
中学(チュウガク)middle school; junior high school.
大学(ダイガク)university.
学界(ガッカイ)academic world.
科学(カガク)science.

乳(ちち)breasts.
授乳(ジュニュウ)breast-feeding.

\subsection{Spouses: 4 夫 8 妻}

夫(フ、おっと)husband

夫妻(フサイ)married couple; husband and wife

\subsection{Feelings: 6 好 8 委 13 嫌}

好き(すき)like; love; prefer

委員(イイン)committee member.
委ねる(ゆだねる)(v1,vt) to entrust to.

嫌い(きらい)hate

\subsection{4 氏}

氏 depicts a man bowing to the left.
氏(シ)(suffix,honorific) Mr.; Mrs.. family. clan.
氏(うじ)family name; birth; lineage.

\section{Body parts}

\subsection{Externals: 10 唇 16 頭}

口付け(くちづけ)(n) kiss.
口付ける(くちづける)(v1) to kiss.

唇(くちびる)(n) lips.

頭(あたま)head

\subsection{Eye-related: 7 見 9 冒 11 現 12 覚}

見る(みる)(v1,vt) to see.
見える(みえる)(v1,vi) to appear.

冒 depicts a hat obstructing the sight, implying rashness
(acting without enough thought).
冒す(おかす)(vt) to risk.
冒険(ボウケン)adventure.

現す(あらわす)(vt) to reveal; to show; to display.
現れる(あらわれる)(v1,vi) to appear; to become visible; to materialize.

覚める(さめる)(v1,vi) to wake up

\subsection{Sound-related: 6 曲 7 声 14 聞}

曲(キョク)music.
作曲(サッキョク)musical composition.

声(こえ)voice.
This character was simplified from 17 聲.

聞く(きく)to hear

\subsection{Mouth-related: 8 味 11 問}

味(あじ)flavor; taste.
美味しい(おいしい)delicious.

問(モン)(suffix, counter) counter for questions.
問う(とう)to ask (a question).
質問(シツモン)question; inquiry; enquiry.
問題(モンダイ)problem.

\subsection{Mouth-related: 6 名吸 7 告 10 舐 11 唾}

名前(なまえ)
name; full name.
given name; first name.

吸う(すう)to suck with mouth

告げる(つげる)to inform; to tell.
告白(コクハク)confess (usually of love).

舐める(なめる)(v1,vt) to lick

唾(つば)spit

\subsection{Locomotion: 7 走 8 歩}

走る(はしる)(v5r,vi) to run

歩く(あるく)to walk

\subsection{Internals: 12 筋}

筋肉(キンニク)muscle

\section{Work, craftsmanship, and family}

\subsection{Craftmanship: 3 工 7 作}

工(コウ、ク)craft.
工学(コウガク)engineering.
大工(ダイク)carpenter.
木工(モッコウ)carpenter.
工(たくみ)(name) Takumi.

作(サク)work; harvest.
作る(つくる)to make.

\subsection{Occupation: 5 仕 8 者 10 家}

仕事(シごと)work.
仕方(シかた)way; method; manner.
仕方ないit can't be helped; there's no other way.

者(シャ)(n,suf) someone of that nature; someone doing that work.
者(もの)(n) person (rarely used without a qualifier).
学者(ガクシャ)scholar.
業者(ギョウシャ)trader; merchant.
研究者(ケンキュウシャ)researcher.
作者(サクシャ)author.

家(カ)-er; -ist; someone who does something.
書家(ショカ)calligrapher.
画家(ガカ)painter.
漫画家(マンガカ)Japanese-comic-book-drawing artist.
活動家(カツドウカ)activist.
研究科(ケンキュウカ)researcher.
作家(サッカ)author; creator; writer; artist.
小説家(ショウセツカ)novelist; fiction writer.
政治家(セイジカ)politician; statesman.
作曲家(サッキョクカ)music composer.
史家(シカ)historian.

Difference between 者 and 家(カ): ???

\subsection{Domicile: 10 家}

家(うち)house.
「今夜私の家(うち)に来てください。」Please come to my house tonight.

\subsection{Family: 10 家}

家(ケ)family.
中川家(なかがわケ)the Nakagawa family.
田中家(たなかケ)the Tanaka family.
マッカーサー家(マッカーサーケ)the MacArthur family; the MacArthurs.

\section{Celestials, skies, gods, and flying: 4 日月 5 白 6 早 8 明}

日(ひ)sun.
日々(ひび)daily; days; old days.

月(つき)moon.

白(ハク)white.
白い(しろい)white.

早い(はやい)(adj-i) early.

明(メイ)bright.
明るい(あかるい)bright.

\subsection{9 星 12 朝}

日付(ひづけ)date.日付別(ひづけベツ)separate by date.(context? usage?)

星(ほし)star

朝(あさ)morning.
今朝(けさ)this morning.
早朝(ソウチョウ)early morning.

\subsection{Skies: 4 天 8 空}

天(テン) sky; heaven

空(そら)sky.
空港(クウコウ)airport.
空く(すく)(vi) to become less crowded; to get empty.
空く(あく)(vi) to be open; to be empty.

\subsection{Gods: 5 申 10 神}

申 depicts a bolt of lightning.
申す(もうす)(humble,vt) to say; to speak.

神(かみ)god; spirit; thunder

\subsection{Atmosphere: 6 気}

気(キ)spirit; mind; air; atmosphere

元気(ゲンキ)

天気(テンキ)weather

気持ち(きもち)feeling

\subsection{Seasons: 5 冬 9 春秋 10 夏}

春(はる)spring (season)

夏(なつ)summer

秋 depicts the burning of plant stalks (after harvest).
秋(あき)autumn; fall season.

冬(ふゆ)winter

\subsection{Atmospheric conditions: 8 雨 11 雪 12 雲 13 雷電}

雨(あめ)rain

雪(ゆき)snow

雲(くも)cloud

雷(かみなり)thunder

電(デン) lightning

電光(デンコウ)lightning

電気(デンキ)electricity (lit. lightning spirit)

電話(デンワ)telephone (lit. lightning talk)

電車(デンシャ)electric train (lit. lightning carriage)

電撃(デンゲキ)electric shock

電気自動車(デンキジドウシャ)electric car

\subsection{Feathers: 6 羽 10 弱 11 習}

6 羽(はね)feather

10 弱

弱い(よわい)weak

11 習 (the bottom character is自not日).

習う(ならう)(vt) to learn

練習(レンシュウ)training; practice

\section{Government}

\subsection{King: 4 王 5 玉生 8 国}

4 王(オウ)king

5 玉(ギョク、たま) ball

生depicts a sprout, something sprouting from the ground.
生(セイ)nature; sex; gender.
学生(ガクセイ)student.
生まれる(うまれる)(v1,vi) to be born.

8 国(コク、くに) country.
国王(コクオウ)king.
国内(コクナイ)internal; domestic.

\subsection{5 主 9 美皇}

5 主(おも)chief; main; principal; important

主人公(シュジンコウ)hero; main character

9 美 beauty

美味しい(おいしい)delicious (idiosyncratic reading)

美しい(うつくしい)beautiful

9 皇(コウ)imperial

皇居(コウキョ)imperial palace

\subsection{8 治}

政治(セイジ)politics.
治める(おさめる)(v1,vt)
to dominate; to rule; to govern; to manage.
to tranquilize; to pacify; to subdue.
to suppress.

治る(なおる)(vi) to heal

治(おさむ)(name) Osamu

仙台(センダイ)(city name) Sendai

\subsection{12 塔}

塔(トウ)tower

管制塔(カンセイトウ)control tower

\section{Strength: 7 助 11 動}

助ける(たすける)(v1,vt) to help

動(ドウ)motion.
動く(うごく)(vi) to move.
動画(ドウガ)animation, motion picture.
自動車(ジドウシャ)automobile.
動力(ドウリョク)power; motive power.

\section{Earth}

\subsection{Utilized land: 6 地}

地(チ)land (that is being used for an activity).
空き地(あきチ)vacant land.
耕地(コウチ)arable land.
団地(ダンチ)multi-unit apartments.

\subsection{Dirt: 9 垢}

垢(あか)dirt; filth; grime

\section{Blades}

\subsection{Tangible cutting: 4 切}

切 (spoon and sword).
切る(きる)(v5r) cut.
切(サイ、セツ).
一切(イッサイ)absolutely; (when used with negative) at all.
大切(タイセツ)(adj-na,n) important.

\subsection{Intangible cutting: 4 分}

分 depicts something separated by a blade.
分ける(わける)(v1,vt) to divide; to split; to share; to distribute.
1分(イップン)one minute.
12時34分(ジュウニジサンジュウヨンプン)12:34 (time).

\subsection{Swordtip: 4 方}

方 depicts the tip of a sword.

方(かた)(honorific) person.
あの方(あのかた)that person.

\subsection{Dissection: 10 剖}

剖(ボウ)dissection.

\subsection{Law: 6 刑 8 法 9 則 14 罰}

刑(ケイ)(n,n-suf) penalty; sentence; punishment

法(ホウ)law; rule; method; principle.

則(のり)law; rule; regulation.
法則(ホウソク)law; rule.

罰(バツ)punishment; penalty.
罰する(ばっする)to punish; to penalize.
罰金(バッキン)fine; monetary penalty.

\subsection{Separation: 7 別}

別 depicts sword cutting bone.
別な(ベツな)(adj-na) different; separate; another

\subsection{Reduction: 12 減}

減(ゲン)reduction; 10\%減 ten percent reduction.

\subsection{Burglar: 13 賊}

賊(ゾク)burglar; robber.
海賊(カイゾク)pirate; sea robber.

\subsection{War, savagery, and violence: 13 戦 15 暴 19 爆}

戦(いくさ)war.
内戦(ナイセン)civil war.
世界大戦(セカイタイセン)World War.

暴 depicts the antler of a buck, representing a savage attack, a violence.
暴動(ボウドウ)insurrection; rebellion; revolt; riot; uprising.
暴風(ボウフウ)storm; windstorm; gale.
暴れる(あばれる)(v1,vi) to rage; to act violently.

爆(バク)burst; explode; bomb.
自爆(ジバク)suicide bombing; self-destruct.
水爆(スイバク)hydrogen bomb.
原爆(ゲンバク)atomic bomb; nuclear bomb.
空爆(クウバク)aerial bombing; air raid.
爆殺(バクサツ)killing by bombing.
爆死(バクシ)death by explosion.

\section{Hand}

\subsection{Arm or hand movements: 7 投 9 指}

投げる(なげる)to throw

指(ゆび)finger.
指す(さす)(vt) to point.

\subsection{Ambiguously figurative: 11 探}

探 depicts a hand groping in a deep cave.
手探り(てさぐり)groping; fumbling.
探す(さがす)(vt)
to search for something lost.
to search for something desired.
探る(さぐる)to feel around for; to fumble for; to grope for.

\subsection{Figurative: 8 押 9 持 10 殺 11 設}

押す(おす)to push; to press; to cram into; to force.
to stamp.
to overwhelm.
押し(おし)(n) push.

持つ(もつ)to hold; to carry; to possess

殺す(ころす)to kill.
殺害(サツガイ)murder.
殺人(サツジン)murder.

設ける(もうける)to establish.

\subsection{Giving and taking: 8 取受 11 授}

取る(とる)(vt) take; fetch; take up.
買い取り(かいとり)purchase; sale. purchase on a non-return policy.

受ける(うける)(v1,vt) to receive

授ける(さずける)(v1,vt) to grant; to award.
授受(ジュジュ)give-and-receive.

\subsection{15 撃}

電撃(デンゲキ)electric shock

\section{Water}

\subsection{Feelings: 7 冷 12 温}

These adjectives describe the feeling when
touching something, not of weather or wind.

冷たい(つめたい)(adj-i) cold (of a tangible object)

温かい(あたたかい)(adj-i) warm (of a tangible object)

\subsection{State: 9 洪}

洪水(コウズイ)(n) flood (of liquid)

大水(おおみず)(n) flood (of liquid)

\subsection{Places: 6 江 8 沼 9 海津}

江(え)inlet; bay

沼(ぬま)swamp; bog

海(うみ)sea; beach.
The original character has 10 strokes.
Shinjitai replaces the two dots in the middle
with one vertical stroke.

津(つ)seaport; harbor.
津波(つなみ)(n) tsunami; tidal wave.

\subsection{Bodily fluids: 6 汗 10 涙}

汗(あせ)(n) sweat.
汗をかく(exp,v5k) to sweat.
汗を流す(exp,v5s) to work hard; to sweat.

涙(なみだ)tear (eyewater)

\subsection{Action done by the liquid: 8 波 10 凍流}

波(なみ)(n) wave (of liquid)

凍る(こおる)to freeze.
But the kanji for for ice is 氷(こおり).

流す(ながす)to flow (liquid)

\subsection{Action done to or with the liquid: 7 没沈 8 泳 9 洗 10 浮}

没(ボツ)drowning

沈む(しずむ)(vi) to sink (descend into liquid)

泳ぐ(およぐ)(vi) to swim

洗う(あらう)(vt) to wash

浮かぶ(うかぶ)to float (be supported by liquid)

\subsection{Drinks: 10 酒}

酒(さけ)sake (a Japanese liquor)

\subsection{14 滴漏}

滴(しずく)a drop of water; a drip.
滴る(したたる)to drip (fall one drop at a time).

漏れる(もれる)to leak (liquid)

\section{Heart: 4 心 5 必}

心 is involved in a lot of feeling-related characters.
心(シン、こころ)heart.
心配(シンパイ)(adj-na,n,vs) worry, concern, anxiety.
心配(シンパイ)(n,vs) care, help.

必 is unrelated to 心. They only look similar.
必ず(かならず)(adv) always, invariably, certainly.
必要(ヒツヨウ)(adj-na,n) necessity, need.

\subsection{Response: 7 応}

応え(こたえ)response; reply; answer; solution.
応える(こたえる)(v1) to respond; to reply; to answer.

\subsection{Thoughts: 7 忘 9 思}

忘れる(わすれる)(v1) to forget.
忘年会(ボウネンカイ)year-end party
(lit. forget-year meeting, a meeting to forget the year).

思う(おもう)to think

\subsection{Feelings: 12 悲 13 意感}

悲しい(かなしい)sad.
悲恋(ヒレン)disappointed love

意(イ)feelings; thoughts.
意欲(イヨク)motivation; will.
意味合い(イミあい)implication; nuance
小生意気(こなまイキ)cheekiness; impudence.

感じる(カンじる)(v1) to feel.

\subsection{Love: 10 恋 13 愛 17 優}

恋(レン、こい)romance; love; tender passion.
恋人(こいびと)lover; sweetheart.
恋文(こいぶみ)love letter.

愛(アイ)(n) love

優しい(やさしい)tender; kind; gentle; affectionate; suave

\subsection{Other: 9 急 11 悪 14 態}

急(キュウ)urgent, sudden, abrupt.
急ぐ(いそぐ)to hurry.

悪(アク)evil, wickedness.
悪人(アクニン)bad person, villain.
悪い(わるい)bad, poor; evil; unprofitable; at fault.

態(ざま)mess; sorry state; plight; sad sight.
変態(ヘンタイ)sexual perversion.

\section{Money: 10 員 11 側 12 買 15 賞賣(売)}

員(イン)(suffix) member.
工員(コウイン)factory worker.
会社員(カイシャイン)company employee.

側(がわ、かわ)side

側(そば)vicinity; near; beside

買う(かう)to buy; to purchase

賞(ショウ)prize; award

賣る(うる)to sell.
This kanji has been simplified to 7 売る.

\section{Communication: 7 言}

言contains口(mouth).
言(こと)saying.
言う(いう)to say.
言葉(ことば)word; dialect.

\subsection{Faith: 9 信}

信(シン)faith; trust.
信じる(シンじる)(v1,vt) to believe; to have faith in.

\subsection{Schemes: 9 計訂}

計(ケイ)plan.
計画(ケイカク)plan; project; schedule; scheme; program; programme.

訂正(テイセイ)correction; revision; amendment

\subsection{Mouth: 13 話 15 談}

話す(はなす)to talk.

談(ダン)discuss.
相談(ソウダン)consultation.
示談(ジダン)out-of-court settlement.
座談会(ザダンカイ)symposium; round-table discussion.

\subsection{Written: 10 記 14 読語}

記(キ)record.
記す(しるす)to record, to write down.
記録(キロク)record.
記事(キジ)article (writing).
選り抜き記事(よりぬきキジ)selected articles.
新しい記事(あたらしいキジ)new articles.

読む(よむ)to read.

語(ゴ)language.
日本語(ニホンゴ)Japanese language.
英語(エイゴ)English language.

\section{Roofs: 6 安宅 9 室}

安い(やすい)cheap; inexpensive

安全(アンゼン)safety; security

安心(アンシン)relief; peace of mind

住宅(ジュウタク)residence; housing; residential building

自宅(ジタク)one's home

自宅火災(ジタクカサイ)house fire; home fire (disaster)

室(むろ)room.

\section{Roads: 12 道 13 違}

道(みち)street; road.
鉄道(テツドウ)railway.

違う(ちがう)(vi) to differ; to not match the correct answer.

\section{Nourishment: 9 食 12 飲}

食べ物(たべもの)food

食物(ショクもの)food

食べる(たべる)(v1) to eat

飲む(のむ)to drink (any liquid, not just liquor)

\section{Illness: 10 症}

症(ショウ)(n,suf) illness

\section{Arrow, medicine, and knowledge: 5 矢 7 医 8 知}

矢(や)arrow

医(イ)medicine; healing; curing; doctor (medical)

日本人の知らない日本語the Japanese language that the Japanese people don't know

\section{Usage: 8 使}

使用(シヨウ)(n) use.
使う(つかう)to use.

\section{Sprout: 8 青毒性 11 清}

性(セイ)nature; sex; gender

男性(だんせい)male

女性(じょせい)female

青(あお)(n) blue; green.
青い(あおい)(adj-i) blue; green.
青ざめる(あおざめる)(v1,vi) to become pale.

毒(ドク)poison.
毒ガス(ドクガス)poison gas.

清い(きよい)clear; pure; noble

\section{Conjunction: 5 且 7 助 8 狙}

且つ(かつ)and.
且又(かつまた)besides; furthermore; moreover

助(すけ)assistance

助(ジョ)(pref) help; rescue; assistant

狙う(ねらう)(vt) to aim at

\section{Thing: 6 件 8 物事}

件(ケン)matter; case; item

物(ブツ、モツ、もの)thing; object; matter.
物語る(ものがたる)(vt) to tell; to indicate.
書物(ショモツ)books.
食べ物(たべもの)food.

仕事(シごと)(n) work; job; business; occupation; employment.
火事(カジ)fire (as a disaster).
有事(ユウジ)emergency.
無事(ブジ)safety; peace; quietness.

\section{Yin-yang: 12 陽}

陽(ヨウ)the yang in yin and yang

太陽(タイヨウ)sun

\chapter{Logic}

\section{Counting}

\subsection{Numbers: 1 一 2 二十八七九 3 三千万 4 五六 5 半四 6 両百}

一(イチ、ひと)one

二(ニ、ふた)two

三(サン、み)three

四 four

五 five

六 six

七 seven

八 eight

九 nine

十 ten

半(ハン)half

両(リョウ)both

百(ヒャク)hundred

千(セン)thousand

万(マン)ten thousand

\subsection{Rotation: 6 回}

回 depicts a spiral.
回す(まわす)(vt) to turn; to rotate.
一回(イッカイ)once; one time.
一回目(イッカイめ)first.

\section{Contrast and opposition}

\subsection{Negation: 4 不反 12 無}

不(フ)(prefix) not; bad; poor.
不安(フアン)anxiety; insecurity.
不明(フメイ)unknown; obscure; anonymous; unidentified.

反(ハン)anti-.
反する(ハンする)to oppose; to rebel; to revolt.
反体制(ハンタイセイ)anti-establishment.

無(ム) no, -less, without.
無駄(ムダ)uselessness.
無用(ムヨウ)uselessness.
無敵(ムテキ)invincible, unrivaled (lit. no-enemy).
無茶(ムチャ)absurd, unreasonable (lit. no-tea).
無人(ムジン)unmanned (lit. no-human).
無言(ムゴン)silence (lit. no-say).

\section{Opposite: 6 向}

向 depicts a house and a window.
向かい(むかい)(n) facing; opposite; across the street; other side.
向く(むく)to face; to turn toward.
向ける(むける)(v1,vt) to turn towards.
向こう(むこう)opposite side; other side; opposite direction.
向上(コウジョウ)improvement; advancement; progress.

\subsection{Versus: 7 対}

対(タイ)versus...
対する(タイする)to face each other.

\section{Productive abstract concepts}

\subsection{Turning-into: 4 化}

化(カ)(suffix) -ization, -ification.
グローバル化(グローバルカ)globalization.
化ける(ばける)(v1,vi) to take the form of.
化学(カガク)chemistry.
化石(カセキ)fossilization.
分化(ブンカ)specialization.

\subsection{Self: 6 自}

自(ジ)self.
自ら(みずから)(adv) personally.
自在(ジザイ)freely (at will).
自分(ジブン)self (context? example usage?).

\subsection{Repetition: 6 再}

再(サイ)again, re-

再生(サイセイ)playback; rebirth

再開(サイカイ)reopening

再来(サイライ)return, comeback

\subsection{Same: 6 同}

同じ(おなじ)same.
同性愛(ドウセイアイ)same-sex love.

\subsection{Whole: 6 全}

全 depicts a whole piece of jade.

全(ゼン) whole

全部(ゼンブ)altogether; everything

全く(まったく)(adv) completely, entirely, wholly, totally

\subsection{Every: 7 毎}

毎(マイ)every.
毎日(マイニチ)everyday.
毎月(マイゲツ、マイつき)every month.
毎時(マイジ)every hour.
毎回(マイカイ)every time (every time it happens); every occurrence.
毎年(マイネン、マイとし)every year.

\subsection{Most: 12 最}

最(サイ)most.
最も(もっとも)most.
日本の最も高い山(ニホンのもっともたかいやま)Japan's highest mountain.
世界で最も太い人(セカイでもっともふといひと)The fattest person in the world.
最小(サイショウ)smallest.
最大(サイダイ)biggest.
最初(サイショ)first.
最後(サイゴ)last.
最新(サイシン)newest.
最高(サイコウ)best, highest, tallest.

\section{Foreign: 6 外}

6 外(ガイ)foreign (not from somewhere nearby).
外人(ガイジン)foreigner, foreign person.
外国(ガイコク)foreign country.
外界(ガイカイ)outside world.
海外(カイガイ)foreign; abroad; overseas.

\section{Existence and truth}

\subsection{Existence: 6 在存有}

有 depicts a hand holding a piece of 肉(meat).
有る(ある)to exist.

存じる(ゾンじる)(v1,humble) to think, feel, consider, know.
存在(ソンザイ)existence; being.
共存(キョウゾン)coexistence.
存亡(ソンボウ)life-or-death; existence; destiny.

\subsection{Truth: 10 真}

真(シン)truth; reality

\section{Cause and reason: 5 由 6 因}

由(よし)cause; reason.

因(イン)cause; factor

\section{Geometry}

\subsection{Flat: 5 平}

平 can mean flat, level (not tilted), ordinary, plain, non-special.
平ら(たいら)flatness.
平たい(ひらたい)(adj-i) flat; even; level; simple.
平皿(ひらざら)flat dish.
平安(ヘイアン)peace; tranquility.
平気(ヘイキ)coolness; calmness; composure; unconcern.
平日(ヘイジツ)weekday; ordinary day (non-holiday).
平年(ヘイネン)normal (non-leap) year; normal year (related to harvest; weather).
公平(コウヘイ)fairness; impartiality; justice
水平(スイヘイ)level; horizontally
平文(ヘイブン)plain (non-encrypted) text.
平面(ヘイメン)level (flat and not-tilted) surface.

\subsection{Point: 9 点}

点(テン)point; spot; speck; mark.

\subsection{Shape: 7 形}

形(ケイ、かたち)shape; form.

\subsection{Circle: 3 丸}

丸(まる)circle

\subsection{Intersect: 6 交 10 校}

交わる(まじわる)cross; intersect; join; meet

国交(コッコウ)diplomatic relations

学校(ガッコウ)school

\subsection{Corner: 7 角}

角(カク、かど)corner

角(つの)horn (head protrusion)

\section{Spacetime points and intervals: 5 古 7 近 8 若長 9 前 13 新遠}

古 consists of 十(ten) and 口(mouth, generation).
古い(ふるい)old (not of person); ancient; obsolete.

近い(ちかい)(adj-i) near (spatial distance).
近々(ちかぢか)soon.
近作(キンサク)recent work.
最近(サイキン)most recent; recently; these days; nowadays.

若い(わかい)young; at an early time in life.
若年(ジャクネン)the time when one was young.

長(チョウ)
long (distance or time).
leader.
eldest.
長い(ながい)long (distance); long (time).
長女(チョウジョ)eldest daughter; first-born daughter.
市長(シチョウ)mayor (a government official).
身長(シンチョウ)height (of body).
最長(サイチョウ)longest, tallest.
社長(シャチョウ)company president.

老い(おい)old age; old (of person); at a late time in life.
老人(ロウジン)old person.
老若(ロウニャク)old and young; all ages.

前(まえ)before (time), in front of.
午前(ゴゼン)morning; before noon; a.m. (ante meridien).

新 depicts cutting tree down with axe.
新(シン)new.
新しい(あたらしい)new.
新聞(シンブン)news.
新車(シンシャ)new car.

遠い(とおい)(adj-i) far (spatial distance).

\chapter{Scratch space}

同人(ドウジン)literary group (coterie); same people; clique; fraternity; comrade; colleague.

1252	騒	騷	馬	18	S		boisterous	ソウ、さわ-ぐ

353	詰		言	13	S		packed	キツ、つ-める、つ-まる、つ-む

追い詰める(おいつめる)(v1,vt) to corner; to drive to the wall.

1783	憤		心	15	S		aroused	フン、いきどお-る

憤り(いきどおり)resentment; indignation.

1261	即	卽	卩	7	S		instant	ソク

1013	状	狀	犬	7	5		form	ジョウ

402	況		水	8	S		condition	キョウ

状況(ジョウキョウ)state of affairs (around you).

1709	膝		肉	15	S	2010	knee	ひざ

883	収	收	攴	4	6		take in	シュウ、おさ-める、おさ-まる

収まる(おさまる)to be in one's (supposed/intended) place.

1054	振		手	10	S		shake	シン、ふ-る、ふ-るう、ふ-れる

1809	返		辵	7	3		return	ヘン、かえ-す、かえ-る

振り返る(ふりかえる)(v5r,vi)to turn head; to look over one's shoulder. to turn around. to look back.

1359	恥		心	10	S		shame	チ、は-じる、はじ、は-じらう、は-ずかしい

43	逸	逸 [4]	辵	11	S		deviate	イツ

逸らす(そらす)(vt) to turn away; to avert.

問い(とい)(n) question.

全く(まったく)(adv) really; truly; entirely; completely; wholly; indeed.

2004	腰		肉	13	S		loins	ヨウ、こし

腰(こし)back; lower back; loins; hip; waist; lumbar region.

28	移		禾	11	5		shift	イ、うつ-る、うつ-す

乱暴(ランボウ)rude; lawless.

抓る(つねる)(v5r,vt) to pinch.

1807	片		片	4	6		one-sided	ヘン、かた

1043	伸		人	7	S		lengthen	シン、の-びる、の-ばす、の-べる

掴まる(つかまる)(v5r,vi)to be caught; to be arrested. to hold on to; to grasp.

巡る(めぐる)(v5r,vi) to go around.

1213	訴		言	12	S		sue	ソ、うった-える

訴える(うったえる)(v1,vt) to raise; to bring to (someone's attention).

1917	妙		女	7	S		exquisite	ミョウ

どんどんdrumming (noise). rapidly; steadily.

1035	触	觸	角	13	S		contact	ショク、ふ-れる、さわ-る

触れる(ふれる)(v1,vi) to touch; to feel.

1400	張		弓	11	5		stretch	チョウ、は-る

637	絞		糸	12	S		strangle	コウ、しぼ-る、し-める、し-まる

絞る(しぼる)to wring; to squeeze; to press; to extract.

730	搾		手	13	S		squeeze	サク、しぼ-る

しぼる:
絞る vs 搾る

1727	描		手	11	S		sketch	ビョウ、えが-く、か-く

565	弧		弓	9	S		arc	コ

1412	嘲 [7]		口	15	S	2010	ridicule	チョウ、あざけ-る

741	擦		手	17	S		grate	サツ、す-る、す-れる

1491	塗		土	13	S		paint	ト、ぬ-る

布団(ふとん)(ateji) futon (Japanese mattress)

404	挟	挾	手	9	S	1981	pinch	キョウ、はさ-む、はさ-まる

立てる(たてる)(v1,vt) to stand up.

2026	落		艸	12	3		fall	ラク、お-ちる、お-とす

落ちる(おちる)

落とす(おとす)

2117	弄		廾	7	S	2010	tamper with	ロウ、もてあそ-ぶ

弄る(いじる)(v5r,vt) to tamper with.

1603	濃		水	16	S		concentrated	ノウ、こ-い

756	残	殘	歹	10	4		remainder	ザン、のこ-る、のこ-す

煽る(あおる)(v5r) to fan; to agitate; to stir up.

1136	醒		酉	16	S	2010	be disillusioned	セイ

醒める(さめる)(v1,vi) to be disillusioned.

317	起		走	10	3		wake up	キ、お-きる、お-こる、お-こす

121	憶		心	16	S		recollection	オク

記憶(キオク)memory; recollection; remembrance. storage.

愛撫(アイブ)caress

1330	奪		大	14	S		rob	ダツ、うば-う

奪う(うばう)(v5u,vt) to snatch away.

すらすらsmoothly

426	極		木	12	4		poles	キョク、ゴク、きわ-める、きわ-まる、きわ-み

極まる(きわまる)(v5r,vi) to terminate; to reach an extreme.

1457	締		糸	15	S		tighten	テイ、し-まる、し-める

反応(ハンノウ)reaction; response.

顔(かお)face (person).

固い(かたい)(adj-i) hard; solid; tough.

1638	薄		艸	16	S		dilute	ハク、うす-い、うす-める、うす-まる、うす-らぐ、うす-れる

薄い(うすい)(adj-i) thin. pale; light. watery; dilute; sparse.

ごりごりscraping; scratching. hard (to the bite, to the touch).

抉る(えぐる)(v5r,vt) to gouge; to hollow out.

29	萎		艸	11	S	2010	wither	イ、な-える

萎える(なえる)(v1,vi) to wither. to droop. to be lame.

1421	沈		水	7	S		sink	チン、しず-む、しず-める

沈む(しずむ)(v5m,vi) to sink; to feel depressed.

408	胸		肉	10	6		bosom	キョウ、むね、(むな)

1916	脈		肉	10	4		vein	ミャク

2025	絡		糸	12	S		entwine	ラク、から-む、から-まる、から-める

深々(シンシン)(adj-t,adv-to)silent (especially of the passing of the night). piercing.

深々(フカブカ)(adv-to) very deeply.

1229	挿	插	手	10	S	1981	insert	ソウ、さ-す

挿入(ソウニュウ)insertion.

2046	律		彳	9	6		law	リツ、(リチ)

脈絡(ミャクラク)chain of reasoning; logical connection; coherence.

310	奇		大	8	S		strange	キ

1130	精		米	14	5		refined	セイ、(ショウ)

落書き(ラクがき)scrawl.

搾乳(サクニュウ)milking.

記念(キネン)commemoration.

2 乃

1513	透		辵	10	S		transparent	トウ、す-く、す-かす、す-ける

556	厳	嚴	口	17	6		strict	ゲン、(ゴン)、おごそ-か、きび-しい

1256	増	增	土	14	5		increase	ゾウ、ま-す、ふ-える、ふ-やす

増す(ます)to increase.

厳しさ(きびしさ)strictness.

含む(ふくむ)(v5m,vt) to contain.

1211	組		糸	11	2		association	ソ、く-む、くみ

取り組む(とりくむ)(v5m,vi) to tackle; to deal with.

1427	追		辵	9	3		follow	ツイ、お-う

追い付く(おいつく)(v5,vi) to catch up with.

1256	増	增	土	14	5		increase	ゾウ、ま-す、ふ-える、ふ-やす

増やす(ふやす)(v5,vt) to increase.

対応(タイオウ)interaction; correspondence.

実態(ジッタイ)true state; actual condition; reality.

653	拷		手	9	S		torture	ゴウ

拷問(ゴウモン)torture.

公開(コウカイ)open to the public; exhibit.

聞き取る(ききとる)to catch (a person's words); to follow; to understand.

拘束(コウソク)restriction.

806	示		示	5	5		indicate	ジ、シ、しめ-す

示す(しめす)(v5,vt) to denote; to show; to indicate.

明らか(あきらか)obvious.

告発(コクハツ)inditement; prosecution; complaint.

判明(ハンメイ)establishing; proving; identifying; confirming.

新た(あらた)new; fresh.

1529	踏		足	15	S		step	トウ、ふ-む、ふ-まえる

踏む(ふむ)(v5,vt) to step on.

1060	進		辵	11	3		advance	シン、すす-む、すす-める

前進(ゼンシン)advance; drive; progress.

待合室(まちあいシツ)waiting room.

低速(テイソク)low gear; slow speed.

走行(ソウコウ)running a wheeled vehicle.

1726	病		疒	10	3		sick	ビョウ、(ヘイ)、や-む、やまい

52	院		阜	10	3		institution	イン

病院(ビョウイン)hospital.

突入(トツニュウ)rushing; breaking into.

566	故		攴	9	5		circumstances	コ、ゆえ

事故(ジコ)accident; incident; trouble.

1437	低		人	7	4		low	テイ、ひく-い、ひく-める、ひく-まる

大分(ダイブン)considerably; greatly; a lot.

261	患		心	11	S		afflicted	カン、わずら-う

1002	障		阜	14	6		hurt	ショウ、さわ-る

507	穴		穴	5	6		hole	ケツ、あな

2104	裂		衣	12	S		split	レツ、さ-く、さ-ける

1607	破		石	10	5		rend	ハ、やぶ-る、やぶ-れる

破る(やぶる)to tear (such as paper).

1282	損		手	13	5		loss	ソン、そこ-なう、そこ-ねる

破損(ハソン)damage.

1415	調		言	15	3		investigate	チョウ、しら-べる、ととの-う、ととの-える

110	押		手	8	S		push	オウ、お-す、お-さえる

押す(おす)(v5,vt) to push.

押し退ける(おしのける)(v1,vt) to push aside.

調べる(しらべる)(v1,vt) to investigate.

1451	停		人	11	4		halt	テイ

喰う(くう)(male,vulgar) to eat

1430	通		辵	10	2		pass through	ツウ、(ツ)、とお-る、とお-す、かよ-う

妊娠(ニンシン)conception; pregnancy.

伝える(つたえる)(vt) to convey.

ふくよかplump; well-rounded.

聞く(きく)to ask; to enquire; to query.

無神経(ムシンケイ)thick-skinned; insensitive to criticism or insults.

質問(シツモン)question.

妊娠していないふくよかな女性
a plump woman who is not pregnant

無神経な質問
insensible question

妊娠していないふくよかな女性に「予定日はいつ?」と聞く無神経な質問。

関連企業(カンレンキギョウ)associated company; affiliated business.

387	挙	擧	手	10	4		raise	キョ、あ-げる、あ-がる

\ruby{4}{ヨ}\ruby{人}{ニン}に\ruby{1}{ひと}\ruby{人}{り}がX 1 in 4 people X.

新社会人(シンシャカイジン)new members of society (especially after turning 20 or joining a company); new working adults

997	傷		人	13	6		wound	ショウ、きず、いた-む、いた-める

傷付ける(きずつける)to hurt someone's feelings or pride

多過ぎる(おおすぎる)to be too many; to be too much

1497	怒		心	9	S		angry	ド、いか-る、おこ-る

怒り(いかり)anger.

挙げる(あげる)(vt)

上げる(あげる)

予定日(ヨテイび)scheduled date; expected date.

660	刻		刀	8	6		engrave	コク、きざ-む

深刻(シンコク)serious.

面倒臭い(メンドウくさい)bothersome (to do); troublesome.

愛情(アイジョウ)love; affection

安心感(アンシンカン)sense of security

1550	得		彳	11	4		acquire	トク、え-る、う-る

得る(える)(v1,vt) to get; to obtain.

560	呼		口	8	6		call	コ、よ-ぶ

悩む(なやむ)to be worried.

多い(おおい)many; numerous.

物言う(ものいう)to talk; to carry meaning.

気分(キブン)feeling; mood.

気持ち(きもち)feeling; sensation; mood.

対等(タイトウ)equality (especially of status or terms).

清楚(セイソ)neat and clean; tidy; trim.

生返事(なまヘンジ)half-hearted reply; vague answer; reluctant answer.

645	興		臼	16	5		entertain	コウ、キョウ、おこ-る、おこ-す

興味(キョウミ)interest (in something).

私に興味ないの?Aren't you interested in me?

533	献	獻	犬	13	S		offering	ケン、(コン)

1766	服		月	8	3		clothes	フク

1241	装	裝	衣	12	6		attire	ソウ、ショウ、よそお-う

1850	褒	襃	衣	15	S	1981	praise	ホウ、ほ-める

理想的(リソウテキ)ideal.

219	較		車	13	S		contrast	カク

比較(ヒカク)comparison.

共通(キョウツウ)

語らう(かたらう)(vt) to talk; to tell

155	暇		日	13	S		spare time	カ、ひま

あまりremainder; rest; residue; remnant

792	視	視 [4]	見	11	6		look at	シ

視点(シテン)opinion; point of view.

よりvsから?

293	含		口	7	S		include	ガン、ふく-む、ふく-める

含む(ふくむ)(vt) to contain

含める(ふくめる)(vt) to include

1836	放		攴	8	3		release	ホウ、はな-す、はな-つ、はな-れる、ほう-る

457	屈		尸	8	S		yield	クツ

1039	辱		辰	10	S		embarrass	ジョク、はずかし-める

屈辱的(クツジョクテキ)humiliating.

???
前で後で
前に後に
???

行動(コウドウ)action

感性(カンセイ)sensitivity

存分(ゾンブン)to one's heart's content; as much as one wants.

無意味(ムイミ)meaningless; nonsense.

料金(リョウキン)fee; price.

偏見(ヘンケン)prejudice.

外人は外国の方。

迷惑(メイワク)trouble; bother; annoyance.

自身(ジシン)self-confidence.

支配(シハイ)control.

太め(ふとめ)chubby; plump.

1191	遷		辵	15	S		transition	セン

1192	選		辵	15	4		choose	セン、えら-ぶ

1619	敗		攴	11	4		failure	ハイ、やぶ-れる

447	苦		艸	8	3		suffer	ク、くる-しい、くる-しむ、くる-しめる、にが-い、にが-る

閉じる(とじる) vs 閉める(しめる)?

窓を閉めるclose window (of a building)

タブを閉じるclose tab

1802	壁		土	16	S		wall	ヘキ、かべ

27	異		田	11	6		uncommon	イ、こと

481	掲	揭	手	11	S		put up (a notice)	ケイ、かか-げる

「小説を読もう!」は約471,337作品の小説が無料で読める小説サイトです。
``Shousetsu-wo yomou!''is a site of about 471,337 freely-readable literary-work novels/short-stories.

読める(よめる)(v1,vi)to be legible; to be readable.
to be pronounceable.
to be predictable.

9 耶(ヤ)question mark

有耶無耶(ウヤムヤ)indefinite; hazy; vague; unsettled; undecided.

1468	哲		口	10	S		philosophy	テツ

哲学者(テツガクシャ)philosopher

1192	選		辵	15	4		choose	セン、えら-ぶ

1273	属	屬	尸	12	5		belong	ゾク

無所属(ムショゾク)independent (in politics); non-partisan.

1770	福	福 [4]	示	13	3		luck	フク

幸福(コウフク)happiness.

先発(センパツ)forerunner

987	勝		力	12	3		win	ショウ、か-つ、まさ-る

決勝(ケッショウ)decision of a contest; finals (in sports).

大会(タイカイ)convention; tournament; mass meeting; rally

375	宮		宀	10	3		Shinto shrine	キュウ、グウ、(ク)、みや

23	為	爲	爪	9	S		do	イ

為(ため)sake; purpose; benefit.

お前の為に!
It's for your own good!

僕はお前の為にこんな事をしている。
I'm doing things like this for your sake.

What is the difference between 者 and 家(カ)?

掛ける、さえ

1704	眉		目	9	S	2010	eyebrow	ビ、(ミ)、まゆ

眉間(ミケン)glabella; middle forehead; area between the eyebrows.

13 睨

睨み(にらみ)glare; sharp look.

睨み付ける(にらみつける)(v1,vt) to glare at; to scowl at.

2133	惑		心	12	S		beguile	ワク、まど-う

231	掛		手	11	S		hang	か-ける、か-かる、かかり

874	種		禾	14	4		kind	シュ、たね

流石(さすが)(ateji)as one would expect.

布団(フトン)(ateji)futon (Japanese mattress).

「X」と「Y」は、どう違う?
How does X and Y differ?

「X」と「Y」の違いは?
What is the difference between X and Y?
Technically, we can parse it as
``What can you say about X and 「Y」の違い (the difference of Y)?''

素  糸 10 5  elementary ソ、ス
遊  辵 12 3  play ユウ、(ユ)、あそ-ぶ
織  糸 18 5  weave ショク、シキ、お-る
状 狀 犬 7 5  form ジョウ

現状

このページを翻訳しますか?Translate this page?

すべてall.

すべてのタブをブックマークに\ruby{追}{つい}\ruby{加}{か}Add all tabs to bookmarks.

次回(ジカイ)next time.

上書き(うわがき)(n,vs) overwrite.

ぴりりtingling; stinging; pungently.

報告(ホウコク)report; information.

1846 報  土 12 5  report ホウ、むく-いる

1343 端  立 14 S  edge タン、はし、は、はた
端末(タンマツ)(comp) (abbr) terminal; computer terminal.

為 爲 爪 9 S  do イ

行為(コウイ)act; deed; conduct.

指示(シジ)(n,vs) instruction

Xように(exp) in order to X

在日(ザイニチ)in Japan; people in Japan.

現状(ゲンジョウ)present condition; existing state; status quo.

そもそもin the first place

解決(カイケツ)(n,vs) solution.

取得(シュトク)acquisition.
資格(シカク)qualifications.
取得資格vocational certificate.

979	称	稱	禾	10	S		appellation	ショウ
愛称(アイショウ)pet name.

舞台(ブタイ)
stage (theater).

写真集(シャシンシュウ)photo album.

1153	籍		竹	20	S		enroll	セキ
書籍(ショセキ)
book; publication.

Xきっかけに(exp) with X as a start

291	鑑		金	23	S		specimen	カン、かんが-みる
鑑賞(カンショウ)appreciation (of art).

人口(ジンコウ)population.
人口10,000人(ジインコウイチマンニン)
Population: 10,000 people.

1720	票		示	11	4		ballot	ヒョウ
投票(トウヒョウ)voting; poll.

944	初		刀	7	4		first	ショ、はじ-め、はじ-めて、はつ、うい、そ-める
初出(ショシュツ)first appearance.
初め(はじめ)first doing of something.
Compare: 始め(はじめ)beginning of something.

努める(つとめる)(v1,vt)to endeavor

2048	略		田	11	5		abbreviation	リャク
2100	歴	歷	止	14	4		curriculum	レキ

略歴(リャクレキ)
brief personal record;
short curriculum vitae.

1038	職		耳	18	5		employment	ショク
就職(シュウショク)finding employment.

801	誌		言	14	6		document	シ
雑誌(ザッシ)magazine (a periodical publication).

387	挙	擧	手	10	4		raise	キョ、あ-げる、あ-がる

153	過		辵	12	5		go beyond	カ、す-ぎる、す-ごす、あやま-つ、あやま-ち

過去 (カコ) (n-adv,n)
past; bygone days.
a past (a personal history one would prefer remained secret).

過程(カテイ)progress; mechanism.

過ごす(すごす)(v5,vt) to pass time. to spend. to overdo.

1507	倒		人	10	S		overthrow	トウ、たお-れる、たお-す

1945	猛		犬	11	S		fierce	モウ

627	降		阜	10	6		descend	コウ、お-りる、お-ろす、ふ-る
降る(ふる)to come down.
雨降り(あめふり)rainfall; rainy weather.
雨が降る。It's raining.
雪が降る。It's snowing.
雨だ。It's raining.

事実上(ジジツジョウ)(n,adj-no)
as a matter of fact; actually; in reality.

長編小説(チョウヘンショウセツ)novel.

需 demand

要 need

一般(イッパン)general; universal; ordinary; average.

ぐるぐる

すてきlovely; beautiful; dreamy; great; superb; cool

% https://en.wiktionary.org/wiki/%E8%BE%9B%E3%81%84#Japanese
% https://en.wiktionary.org/wiki/%E8%BE%9B#Japanese

子供を作るto make a child

建築家(ケンチクカ)architect.

制作(セイサク)
work (film, book).
production; creation.

注目(チュウモク)attention.

活動(カツドウ)activity.

1594	念		心	8	4		thought	ネン

1700	碑	碑 [4]	石	14	S		tombstone	ヒ

懺悔(ザンゲ)repentance; confession; penitence

値打ち(ねうち)value; worth; price; dignity

寒い(さむい)cold (weather; wind); chilly

警察(ケイサツ)police

素晴らしい(すばらしい)

卒業(ソツギョウ)

擦り傷(すりきず)(n) scratch; graze; abrasion

…階建て(…カイだて)(suf) ...-story building.
7階建て 7-story building.

14 増える(ふえる)(v1,vi) to increase; to multiply

15 選 elect; select

13 連携(レンケイ)collaboration; cooperation

喧嘩(ケンカ)fight; brawl

博打(バクチ)gambling

お絵描き(おエかき)oekaki; painting; drawing

ネタバレspoiler (of a movie, a story, etc.); something that spoils the end of a movie, a story, etc.

% https://en.wikipedia.org/wiki/Administrative_divisions_of_Japan

\chapter{Grammar}

% https://en.wikipedia.org/wiki/Japanese_verb_conjugation

\section{Noun clause}

「文字を読み書き出来ない子供たちの未来」means
``the future of children who cannot read and write Chinese characters''.
The primary noun is 「未来」 (``future''),
which is modified by the adjective
「文字を読み書き出来ない子供たちの」
(``of children who cannot read and write Chinese characters'').
「文字を読み書き出来ない」(``unable to read and write Chinese characters'')
is an adjective that modifies 「子供たち」 (children).

\section{Vocabulary}

本(ほん)book

魚(さかな)fish

食べる(たべる)eat

読む(よむ)read

(S) is the implied subject.

\section{Present imperfect}

魚を食べる。(S) eats fish.

魚を食べない。(S) doesn't eat fish.

\section{Past perfect}

魚を食べなかった。(S) didn't eat fish.

本を読んだ。(S) read a book.

\section{Command}

魚を食べなさい。Eat fish.

魚を食べな。(childspeak?) Eat fish.

魚を食べて。Eat fish.

魚を食べないで。Don't eat fish.

\section{Passive}

食べられた。(S) was eaten.

読まれた。(S) was read.

\section{I want to ...}

食べたい。I want to eat.

読みたい。I want to read.

\section{Polite}

You have to learn to say the same thing all over again in different registers.

食べます(polite) to eat

食べません(polite) not eat

食べました(polite) ate

魚を食べました。(polite) (S) ate fish.

魚が食べられました。(polite)Fish was eaten. (???)

魚を食べません。(polite) (S) doesn't eat fish.

魚を食べないでください。(polite command) Please do not eat fish.

\section{But, although, despite, in spite of}

でも

けど

けれど

\section{Then}

そして

\section{So, thus, therefore}

だから

\section{To, for}

\section{From, to (place)}

から

より

\section{From, until (time)}

\section{In order to}

\section{Nevertheless}

\section{Whereas}

\section{On the other hand}

\section{Too bad ...}

\section{Because, because of, due to}

から

たら

ば

\section{About, regarding}

ついて

\section{About, approximately}

くらい

\section{Hmm...}

あの

えと

\section{Huh?}

え?

おれ?

あれ?

\section{Let's ...}

\section{Don't ...}

\section{Please ...}

\section{Are you sure?} 

\section{Why not ...}

\section{I think ...}

\section{Perhaps ...}

たぶん

もしかして

\section{I mean..., What I'm trying to say is..., How do I say this...}

\section{Absolutely}

\section{Precisely}

\section{Actually}

\section{Frankly}

\section{Indeed}

\section{Anyway}

\section{I'm sorry}

すま

すみません

ごめん

ごめんなさい

\section{Excuse me}

しつれいします

おじゃまします

\section{Long time no see}

おひさしぶり

\section{How dare you...}

\section{Like it or not}

\section{Or else}

\section{Questions: 7 何 15 誰}

何(なに)what

誰(だれ)who

\section{Uncategorized}

として

さすが

まじで


\end{document}
