\documentclass[12pt,openany]{book}
\usepackage{tabularx}
\usepackage[CJKmath=true]{xeCJK}
\xeCJKDeclareSubCJKBlock{RadicalB}{"20000 -> "2A6DF}
\setCJKmainfont{HanaMinA}
\setCJKmainfont[RadicalB]{HanaMinB}
\usepackage{ruby}
\renewcommand\rubysize{0.5}
\renewcommand\rubysep{0.0em}
\usepackage{amsmath}
\usepackage[
    paperwidth=6in,paperheight=9in
    ,top=0.75in,bottom=0.75in
    ,inner=0.67in,outer=0.33in
]{geometry}
\usepackage{hyperref}
\begin{document}
\frontmatter
\phantomsection
\addcontentsline{toc}{chapter}{Cover}
\begin{titlepage}
{%
    \setlength\parindent{0em}%
    {\Large Cold-Turkey Head-First Dive into Kanji}\par\vspace{2em}
    Erik Dominikus\par
    2017-04-02\par
    (Draft)\par\vfill
    {\small
        This book aims to optimize your kanji learning loop.

        This book does not discuss about grammar, register, and culture.
    }
}
\end{titlepage}
\phantomsection
\addcontentsline{toc}{chapter}{\contentsname}
\tableofcontents
\mainmatter
\chapter{Strategy}

\section{Mindful consistent spaced repetition}

The key to remembering kanji is
\emph{mindful consistent spaced repetition}.
Don't just gloss over the kanji.
\emph{Internalize}.

Don't read one stroke at a time,
but read one component at a time.

\emph{Expect to forget a lot, many times}.
You will forget.
Reread this book.
This book is meant to be read and read again many times.
The aim of this book is to help you optimize your loop.

Mindful consistent spaced repetition is \emph{the only way},
at least until mind-upload is invented.
There is no shortcut.
It's just how the brain works.

The loop:
For each kanji:
Read the kanji out (voice it out).

\section{Target audience}

You read, write, speak, and listen to English but don't read, write, speak, and listen to Japanese.

\section{Memorize the syllabaries}

Memorize all hiragana and katakana.
It's a must.

\section{Understand the principles}

Understand that most kanjis are composed using other kanjis.

Use Wiktionary to found the oracle bone script version of Han characters.

Don't focus on both grammar and vocabulary at the same time.
One who chases two rabbits catches neither.

\section{Read and hear native speakers}

Go to YouTube.
Hear how they speak, even if you don't understand.

Use Google Translate, but do not trust it.
Sometimes it gives the wrong reading.
Use Wiktionary, Wikipedia.

Find Japanese people on Twitter and understand their tweets.

Find Japanese blog posts.

Set up an input method, or use Google Translate.

\chapter{Kanji}

Information is compiled from Wiktionary,
edict, compdic, kanjidic, gjiten,
and Japanese-to-English Google-Translate.

Japanese schoolgoers take 12 years to study 2,000 characters.

\section{History}

Japan borrowed Han characters.

To understand a character, see how it combines with other characters.

To keep you motivated, the characters have been sorted by number of strokes.

A kanji can have several readings:
go'on, kan'on, kun-yomi, name reading, and stylistic/idiosyncratic/ad-hoc reading.

新字体(シンジタイ)(lit. new Han-character body)

旧字体(キュウジタイ)(lit. old Han-character body)

A Latin alphabet letter encodes a \emph{sound}.
You can pronounce the word without understanding it.
A Han character encodes a \emph{concept}.
You can understand the meaning without pronouncing it.

Katakana shows Chinese reading and hiragana shows Japanese reading.

\section{Grouping}

The goal of grouping is to allow the brain to \emph{chunk}.
When the brain recalls any of the member of the group,
it automatically recalls other members of the group.

Use recursive grouping that is 4 levels deep because \(7^4 = 2401\).

\section{Grouping by reducibility}

Chapter \(n\) should not define a kanji
that has more than \(n\) components.
Each section in a chapter groups kanji by semantic closeness
(not by visual closeness).

Example of irreducibility:
台 (tower) is treated as one component;
for pattern recognition purposes,
it is not treated as 厶 stacked on 口.

\section{A possible grouping: scratchiness}

The characters are sorted ascending by ``scratchiness''.
Boxes are less scratchy than lines.
Lines are less scratchy than dots.
Symmetry is less scratchy than asymmetry.
Fewer strokes are less scratchy than more strokes.
However, this is subjective, approximate, and inconsistent.
I don't always follow my own rules.
The goal is to optimize the learning rate of the reader.
The reader is expected to read the character list
sequentially and repeatedly.

Level 1 looks like
an aspiring artist's minimalist abstract paintings.
Level 2 is a combinatorial explosion
of two juxtaposed abstract paintings.
Level 3 ramps up the difficulty even more with more strokes and less symmetry.
Finally, Level 4 looks like total chicken scratch to the uninitiated,
is barely legible on screen even in a 12-point sans-serif font,
and may raise questions such as
``You have a character for that?''
and
``What drug were the sages smoking?''

\section{Another grouping: recency}

Group characters invented in the same period together.
Ancient Chinese concepts,
Old Chinese concepts,
Middle Chinese concepts,
Contemporary Chinese concepts.

\section{Another grouping}

Use visual similarity to group characters by concept.

Visual similarity does not always imply conceptual similarity.

\section{Historical context}

All dates are approximate.

In 12,000 BCE, dogs had been domesticated.

In 4,000 BCE, Egypt had had papyrus.

In 2,500 BCE, Egypt had hieroglyphs.

In 1,000 BCE, China had had oracle bone writing.

In 479 BCE, Confucius died.

Around 400 BCE, Gautama Buddha died.

In 399 BCE, Socrates died.

In 206 BCE, China began the Han dynasty.

Around 100 BCE, China had paper.

Around 30, Jesus died.

In 220, China ended the Han dynasty.

In 632, Muhammad died.

In 1000, China had had gunpowder.

Around 1056, Benedict IX died.

In 1185, Japan ended the Heian period.
Japan also had imported Han characters from China.

In 1546, Martin Luther died.

In October 1582, Gregory XIII introduced
what would later be known as the Gregorian calendar.

In 1750, Johann Sebastian Bach died.

In 1819, James Watt died.

In 1823, David Ricardo died.

In 1865, Abraham Lincoln died.

In 1897, Henry George died.

In 1917, the Balfour Declaration supporting Zionism was made.

In 1945, Adolf Hitler died.

1942--1945: World War II

In 1946, John Maynard Keynes died.

In 1947, the Truman Doctrine was announced,
marking the beginning of the Cold War.

1950--1953: Korean War

In 1954, Alan Turing died.

In 1955, the Vietnam War began.

In 1961, Yuri Gagarin in Vostok 1
became the first man to travel into space.

In 1963, John Fitzgerald Kennedy died.

In 1968, Martin Luther King Jr. died.

In 1970, Sukarno died.

In 1975, the Vietnam War ended.

In 1980, Mohammad Hatta died.

In 1979--1989, the United States Central Intelligence Agency
carried out Operation Cyclone,
arming and financing Afghanistan mujahideen.

In 1986, Lafayette Ronald Hubbard died.

In 1991, the Soviet union ended, marking the end of the Cold War.

In 2001, the World Trade Center was destroyed.

In 2006, Milton Friedman died.

\chapter{Kanji 1}

\section{Confusing characters}

\subsection{3 夂夊 4 攵 8 㑒}

In the Japanese language,
these characters become parts of other characters
instead of being used on their own.

夂 depicts two legs followed by something from behind.

夊 depicts a footprint.

攵 is a variant of 攴 depicting a branch and a hand.

㑒 is simplified from the 13-stroke 僉
meaning ``all, together, unanimous''.

\subsection{士 (samurai) and 土 (earth)}

士(samurai) has longer upper horizontal stroke.
土(earth) has shorter upper horizontal stroke.

\subsection{Hat, sun, moon, meat, inner}

冃(hat)

日(sun)

月(moon)

\subsection{石 (stone) and 右 (right)}

\subsection{人 (person) and 入 (enter)}

\subsection{王 (king) and 生 (sprout)}

\subsection{業菐美}

\section{2 入 4 内 6 肉}

入る(いる)(v5)
to enter; to go into; to get into.

内(うち)inside

肉 (ribs of an animal's torso)

肉(ニク)meat; flesh; body (as opposed to spirit)

\section{4 之}

之(これ)this

\section{3 工山川上下 4 中止}

川(かわ)river

山(サン、やま)mount, mountain

上(うえ)up

下(した)down

工(コウ、ク)craft.
工学(コウガク)engineering.
大工(ダイク)carpenter.
木工(モッコウ)carpenter.
工(たくみ)(name) Takumi.

中(チュウ、なか)middle.
中学(チュウガク)middle school; junior high school.
…の中(…のなか)middle of something.
田中(たなか)(name) Tanaka.
中山(なかやま)(name) Nakayama.
中川(なかがわ)(name) Nakagawa.

止depicts a footprint.

止める(とめる)(v1) to stop moving (walking, etc.); to park (a car)

\section{5 平半}

平 can mean flat, level (not tilted), ordinary, plain, non-special.
平ら(たいら)flatness.
平たい(ひらたい)(adj-i) flat; even; level; simple.
平皿(ひらざら)flat dish.
平安(ヘイアン)peace; tranquility.
平気(ヘイキ)coolness; calmness; composure; unconcern.
平日(ヘイジツ)weekday; ordinary day (non-holiday).
平年(ヘイネン)normal (non-leap) year; normal year (related to harvest; weather).
公平(コウヘイ)fairness; impartiality; justice
水平(スイヘイ)level; horizontally
平文(ヘイブン)plain (non-encrypted) text.
平面(ヘイメン)level (flat and not-tilted) surface.

半(ハン)half

\section{4 日月 5 白 6 早 8 明}

日(ひ)sun.
日々(ひび)daily; days; old days.

月(つき)moon.

白(ハク)white.
白い(しろい)white.

早い(はやい)(adj-i) early.

明(メイ)bright.
明るい(あかるい)bright.

\subsection{9 星 12 朝}

日付(ひづけ)date.日付別(ひづけベツ)separate by date.(context? usage?)

星(ほし)star

朝(あさ)morning.
今朝(けさ)this morning.
早朝(ソウチョウ)early morning.

\section{4 王 5 玉生 8 国}

4 王(オウ)king

5 玉(ギョク、たま) ball

生depicts a sprout, something sprouting from the ground.
生(セイ)nature; sex; gender.
学生(ガクセイ)student.
生まれる(うまれる)(v1,vi) to be born.

8 国(コク、くに) country.
国王(コクオウ)king.
国内(コクナイ)internal; domestic.

\subsection{5 主 9 美皇}

5 主(おも)chief; main; principal; important

主人公(シュジンコウ)hero; main character

9 美 beauty

美味しい(おいしい)delicious (idiosyncratic reading)

美しい(うつくしい)beautiful

9 皇(コウ)imperial

皇居(コウキョ)imperial palace

\section{4 井开}

井 depicts a square well.
井(い)well (water reservoir).

开 is a simplification of 幵 depicting raising both hands.

\section{5 田由出石左右 6 回向}

田(た)rice field

由(よし)cause; reason.

石(いし)stone

出 depicts something coming out of an open box.
出る(でる)(v1,vi) to go out; to exit; to leave.
出来る(できる)(v1,vi) to be able to do.
出来上がる(できあがる)(vi) to be finished; to be completed; to be ready

左(ひだり)left

右(みぎ)right.

回 depicts a spiral.
回す(まわす)(vt) to turn; to rotate.
一回(イッカイ)once; one time.
一回目(イッカイめ)first.

向 depicts a house and a window.
向かい(むかい)(n) facing; opposite; across the street; other side.
向く(むく)to face; to turn toward.
向こう(むこう)opposite side; other side; opposite direction.
向上(コウジョウ)improvement; advancement; progress.

\section{5 皿}

皿(さら)dish, plate

\section{5 世 6 両}

世(よ)world; society; age; generation

両(リョウ)both

\section{7 車 8 店 9 面}

車(くるま)car

店(テン)(n) store; shop.
ラメン店ramen shop (ramen is a kind of Japanese noodle).

面(おもて)mask

面白い(おもしろい)interesting

\chapter{Kanji 2}

\section{Pictures: 5 写 7 図 8 画}

写生(シャセイ)sketch.
写真(シャシン)multimedia; photograph; movie
写す(うつす)(vt) to photograph.

図(ズ)map; drawing; picture; plan; illustration; diagram; figure; chart

画 means picture.
画(カク)counter for kanji strokes.
字画(ジカク)number of strokes in a character.
画家(ガカ)painter.

\section{Cardinal directions: 5 北 6 西 8 東 9 南}

北(ホク、きた)north

西(セイ、にし)west

東(トン、ひがし)east

南(ナン、みなみ)south

They combine as in English.

北西(ホクセイ)northwest

北東(ホクトウ)northeast

東北(トウホク)Touhoku (a prefecture)

南西(ナンセイ)southwest

南東(ナントウ)southeast

\section{Calendar}

These kanji readings for today, yesterday, and tomorrow are irregular.

今日(きょう)today

昨日(きのう)yesterday

明日(あした)tomorrow

Names of weekdays.

日曜日(ニチヨウび)Sunday

月曜日(ゲツヨウび)Monday

火曜日(カヨウび)Tuesday

水曜日(スイヨウび)Wednesday

木曜日(モクヨウび)Thursday

金曜日(キンヨウび)Friday

土曜日(ドヨウび)Saturday

毎日(マイニチ)everyday

Expressions.

また明日(あした)see you again tomorrow; means 'again' and 'tomorrow'

\section{Shops: 8 店}

店(テン)(n) store; shop.
ラメン店ramen shop (ramen is a kind of Japanese noodle).

\section{Myself: 2 厶 4 公 7 私}

厶 means I, myself, private.

公(おおやけ)public; communal; official; governmental.
公安(コウアン)public safety; public welfare.

私(わたし)I; me

\section{Inches: 3 寸 5 付 7 村 10 時}

% https://en.wiktionary.org/wiki/%E5%AF%B8#Japanese
寸depicts a position on the forearm
where the pulse can be palpated by compressing the radial artery.
寸(スン)an ancient unit of length, approximately 3 cm.

付ける(つける)(v1,vt) to attach, join, stick, glue, fasten.

村(むら)village

時(とき)time.
時代(ジダイ)era.
三国時代(サンゴクジダイ)The Three Kingdoms period.
戦国時代(センゴクジダイ)The Warring States period.

\section{Threads}

\subsection{4 幻 9 紅 10 紙}

幻(まぼろし)phantom; vision; illusion; dream

紅(くれない)deep red; crimson

紙(かみ)paper

手紙(てがみ)letter (the document, not the alphabet)

\section{Passable: 5 可 10 哥 14 歌}

可 passable
許可(キョカ)permission; authorization; approval.

哥 means older brother.
This character is not used on its own in Japan.

歌う(うたう)to sing

\section{Escape: 11 脱}

The left side is 肉.
脱出(ダッシュツ)escape.

\section{Foresight: 6 先 8 洗}

先生(せんせい)teacher; master; doctor

洗う(あらう)(vt) to wash

\section{Who: 15 誰}

誰(だれ)who

\subsection{Wide-narrow: 5 広 9 狭}

広い(ひろい)(adj-i) spacious; vast; wide.
広告(コウコク)advertisement.

狭める(せばめる)(v1,vt) to narrow.
狭い(せまい)(adj-i) narrow; confined; small.

\section{Metals: 8 金 13 鉄 14 銅銀 16 鋼}

金has a lot to do with metals.
金(キン、かね)gold; money.

金属(キンゾク)metal.
重金属(ジュウキンゾク)heavy metal.
These are also the chemistry terms.

金色(キンいろ、コンジキ)golden (color)

鉄(テツ、くろがね)iron (lit. 黒金 black metal).
鉄人(てつジン)iron man; strong man.

鉄道(テツドウ)railroad; railway

銅(ドウ、あかがね)copper (lit. 赤金 red metal)

銀(ギン、しろがね)silver (lit. 白金 white metal)

鋼(コウ、はがね)steel

青銅(セイドウ)bronze

鋼鉄(コウテツ)steel

\subsection{Firearms: 14 銃}

銃(ジュウ)gun; small firearms

\subsection{Mirror: 19 鏡}

8 金 + 5 立 + 7 見 - 1

鏡(かがみ)mirror

眼鏡(めがね)eyeglasses

\section{Trouble: 7 困}

困る(こまる)(vi) to be troubled; to be embarrassed

\section{Grass: 6 艸}

\subsection{7 花 9 草}

花(はな)flower

草(くさ)grass

\subsection{Tea: 9 茶}

茶(チャ) tea

\subsection{11 菌}

菌(キン)fungus; germ; bacterium

\section{Fire: 7 災 8 炎}

災い(わざわい)(n) calamity; catastrophe.
火災(カサイ)fire (disaster).

炎(ほのお)flame, blaze.
炎天(エンテン)scorching sun.

\subsection{15 熱 16 燃}

熱い(あつい)(adj)hot (temperature)

燃える(もえる)(v1,vi) to burn; to get fired up

火事(カジ)fire (disaster).
ラメン店で火事fire at a ramen shop.

\section{Gate: 12 間 12 開 14 関}

間 interval

開く(ひらく)to open (door, business, eye, mouth, ...)

関(カン、せき)barrier; gate

\chapter{Kanji 3}

\section{4 凶}

凶(キョウ)evil; villain; bad luck; disaster.
凶悪(キョウアク)(adj-na) atrocious; fiendish; brutal; villainous.

\section{6 危}

危ない(あぶない)dangerous

\section{7 努}

努める(つとめる)(v1,vt) to endeavor; to try; to strive.
努力(ドリョク)great effort; exertion; endeavor

\section{8 店}

店(テン)(n) store; shop.
ラメン店ramen shop (ramen is a kind of Japanese noodle).

\section{9 垢信計}

垢(あか)dirt; filth; grime

信(シン)faith; trust.
信じる(シンじる)(v1,vt) to believe; to have faith in.

計(ケイ)plan.
計画(ケイカク)plan; project; schedule; scheme; program; programme.

\section{10 通時唇特}

通(ツウ)pass through.
通る(とおる)to go by.

時(とき)time.
時代(ジダイ)era.
三国時代(サンゴクジダイ)The Three Kingdoms period.
戦国時代(センゴクジダイ)The Warring States period.

唇(くちびる)(n) lips.

特(トク)special.
特に(トクに)particularly; especially.
特技(トクギ)special skill.

\section{11 野転接}

野(ヤ、の)field; plains; rustic.

転(テン)revolve; turn around; change.
反転(ハンテン)rolling over; turning around.
転ぶ(ころぶ)(vi) to fall down; to fall over.

接(セツ)touch; contact.
接ぐ(つぐ)to join (two things into one); to piece together; to graft.
接吻(セップン)kiss.
接する(セッする)to come in contact with; to touch.

\section{11 経済脳情啓祭教清}

経(キョウ)sutra; Buddhist scripture.
経つ(たつ)(vi) to pass; to lapse.
経る(へる)to pass; to elapse; to experience.

済ませる(すませる)(v1,vt) to finish; to end.
経済(ケイザイ)economy.

脳内(ノウナイ)intracranial; inside the brain

情(ジョウ)feelings; emotion; passion; sympathy.

啓(ケイ).
拝啓(ハイケイ):拝啓… Dear ...

祭り(まつり)feast; festival

教え(おしえ)teaching; doctrine.
教育(キョウイク)training; education.
教会(キョウカイ)church.

清い(きよい)clear; pure; noble

\section{12 堺堕貿普奥詞筋}

堺(カイ、さかい)world.

堕(ダ)degeneration; degradation.
堕胎(ダタイ)abortion; feticide; babykilling.
堕落(ダラク)depravity; corruption; degradation.

貿(ボウ)trade.
貿易(ボウエキ)trade (foreign)

普(フ)universal; wide; general.
普通(フツウ)general; ordinary; usual.

奥(おく)interior.
奥山(おくやま)remote mountain.

詞(シ).
名詞(メイシ)noun.

筋肉(キンニク)muscle.

\section{13 違僧話園数禁準楽暖業}

相違(ソウイ)difference; discrepancy; variation.
違う(ちがう)(vi) to differ; to not match the correct answer.
間違う(まちがう)to make a mistake; to be incorrect; to be mistaken.

僧(ソウ)monk; priest.

話す(はなす)to talk.

園(エン)park; garden; yard; farm.
動物園(ドウブツエン)zoo; animal park; zoological garden.
幼稚園(ヨウチエン)kindergarten.

数(かず)number; amount.
数える(かぞえる)(v1,vt) to count; to enumerate.
算数(サンスウ)arithmetics.

禁じる(キンじる)(v1,vt) to prohibit.

準(シュン).
水準(スイジュン)water level. level; standard.

楽しい(たのしい)happy.
音楽(オンガク)music.

暖かい(あたたかい)(adj-i) warm; genial

業(ギョウ)industry.
工業(コウギョウ)manufacturing industry.

\section{14 際菐僕概}

算(サン)calculation.
算出(サンシュツ)calculation; computation.
加算(カサン)addition.
引き算(ひきザン)subtraction.
公算(コウサン)probability; likelihood.

際(サイ)occasion; circumstances.
際限(サイゲン)limits; bounds.
学祭(ガクサイ)interdisciplinary.
国際(コクサイ)international.

菐 is the 14-stroke ``thicket'' radical.

僕(ボク)I; me (male).

概要(ガイヨウ)outline; summary

\section{15 膝質談線監編稿誰撃}

膝(ひざ)knee; lap

質(シツ)(suffix) substance; quality; matter.
質問(シツモン)question.

談(ダン)discuss.
相談(ソウダン)consultation.
示談(ジダン)out-of-court settlement.
座談会(ザダンカイ)symposium; round-table discussion.

線(セン)line; stripe.
line (telephone line).
line (of a railroad).
打線(ダセン)baseball lineup.

監(カン)government official; rule; administer.
監禁(カンキン)confinement; bondage.

編む(あむ)(vt)
to knit; to plait; to braid.
to compile (an anthology); to edit.

稿(コウ)draft; copy; manuscript

誰(だれ)who

電撃(デンゲキ)electric shock.

\section{16 諦頭}

諦める(あきらめる)(v1,vt)
to give up; to abandon.

頭(あたま)head

\section{18 顔曜類}

顔(かお)face.
顔面(ガンメン)face (of a person).

曜(ヨウ)(weekday name).
日曜日(ニチヨウビ)Sunday.

類(ルイ)kind; sort; type.
人類(ジンルイ)mankind.

\section{Revision: 7 改 9 訂}

7 改める(あらためる)(v1,vt) to revise.

訂正(テイセイ)correction; revision; amendment

\section{Written: 10 記 14 読語}

記(キ)record.
記す(しるす)to record, to write down.
記録(キロク)record.
記事(キジ)article (writing).
選り抜き記事(よりぬきキジ)selected articles.
新しい記事(あたらしいキジ)new articles.

読む(よむ)to read.

語(ゴ)language.
日本語(ニホンゴ)Japanese language.
英語(エイゴ)English language.

\section{Calendar}

These kanji readings for today, yesterday, and tomorrow are irregular.

今日(きょう)today

昨日(きのう)yesterday

明日(あした)tomorrow

Names of weekdays.

日曜日(ニチヨウび)Sunday

月曜日(ゲツヨウび)Monday

火曜日(カヨウび)Tuesday

水曜日(スイヨウび)Wednesday

木曜日(モクヨウび)Thursday

金曜日(キンヨウび)Friday

土曜日(ドヨウび)Saturday

毎日(マイニチ)everyday

Expressions.

また明日(あした)see you again tomorrow; means 'again' and 'tomorrow'

\section{Grass: 7 花 9 草茶 11 菌}

花(はな)flower

草(くさ)grass

茶(チャ) tea

菌(キン)fungus; germ; bacterium

\section{Craft: 3 工}

工作(コウサク)work; construction; handicraft.

\section{Countries}

日本(ニホン)Japan

中国(チュウゴク)People's Republic of China

英国(エイコク)United Kingdom

米国(ベイコク)United States of America

\section{Personal data}

出身(シュッシン)person's origin (town, city, country, etc.)

出身地(シュッシンチ)birthplace

誕生日(タンジョウビ)birthday; birth date; day of birth.

身長(シンチョウ)height (of body)

体重(タイジュウ)body weight

血液型(ケツエキガタ)blood type.「A型」type a.

好きなもの(すきなもの)likes

嫌いなもの(きらいなもの)dislikes

\section{Ungrouped}

警察(ケイサツ)police

素晴らしい(すばらしい)

卒業(ソツギョウ)

擦り傷(すりきず)(n) scratch; graze; abrasion

今度(コンド)
now; this time; this occurrence.
next time; another time.

率(リツ)(suf) rate; ratio; proportion.
識字率(シキジリツ)literacy rate.

…階建て(…カイだて)(suf) ...-story building.
7階建て 7-story building.

13
了解(りょうかい)understanding; roger that.
見解(ケンカイ)opinion; point of view.
専門家見解(センモンカケンカイ)expert opinion.

16 操る(あやつる)(vt) to be fluent in (a language)

14 増える(ふえる)(v1,vi) to increase; to multiply

17 覧 perusal

12 報 report; news

15 選 elect; select

13 連携(レンケイ)collaboration; cooperation

喧嘩(ケンカ)fight; brawl

博打(バクチ)gambling

お絵描き(おエかき)oekaki; painting; drawing

ネタバレspoiler (of a movie, a story, etc.); something that spoils the end of a movie, a story, etc.

関係(カンケイ)relation; connection

肉体関係(ニクタイカンケイ)sexual relations

垢と一切関係ないIt has absolutely nothing to do with dirt.

\chapter{Kanji 4}

\section{4 火水 5 氷永 8 泳}

火(カ、ひ)fire, flame.

火事(カジ)fire (disaster).

大火(タイカ)big fire.

水(スイ、みず)water.

井 depicts a square well.

井(い)well (water reservoir).

氷(こおり)ice

永(エイ、なが)eternity.

永い(ながい)(adj-i) very long (time).

泳ぐ(およぐ)(vi) to swim

\subsection{(4 火) 6 灰 7 災 8 炎}

灰(カイ、はい)ashes.

火災(カサイ)fire (disaster).

災い(わざわい)(n) calamity; catastrophe.

炎(ほのお)flame, blaze.

炎天(エンテン)scorching sun.

\subsection{(4 火) 12 焼 15 熱 16 燃}

焼(ショウ)bake; burn.

焼き鳥(やきとり)grilled chicken meat.

熱い(あつい)(adj)hot (temperature)

燃える(もえる)(v1,vi) to burn; to get fired up.

再燃(サイネン)recurrence; revival; resuscitation;
getting spirited/interested once again.

\section{4 心 5 必}

心 is involved in a lot of feeling-related characters.

心配(シンパイ)(adj-na,n,vs) worry, concern, anxiety.

心配(シンパイ)(n,vs) care, help.

必 is unrelated to 心. They only look similar.

必(ヒツ、かなら)inevitable.

必ず(かならず)(adv) always, invariably, certainly.

必要(ヒツヨウ)(adj-na,n) necessity, need.

\subsection{Thoughts: 7 忘 9 思}

忘(ボウ、わす)forget.

忘れる(わすれる)(v1) to forget.

忘年会(ボウネンカイ)year-end party
(lit. forget-year meeting, a meeting to forget the year).

思(シ)think.
思う(おもう)to think

\subsection{10 息}

息(ソク、いき)breath.

息子(むすこ)son

\subsection{13 感}

感じる(カンじる)(v1) to feel.

感心(カンシン)(adj-na,n,vs) admiration. Well done!

\subsection{Other: 9 急 11 悪}

急(キュウ)urgent, sudden, abrupt.

急ぐ(いそぐ)to hurry.

悪(アク)evil, wickedness.

悪人(アクニン)bad person, villain.

悪い(わるい)bad, poor; evil; unprofitable; at fault.

\subsection{(4 心) 7 応 13 愛 17 優}

応え(こたえ)response; reply; answer; solution.

応える(こたえる)(v1) to respond; to reply; to answer.

愛(アイ)(n) love.

愛す(あいす)(vt) to love.

同性愛(ドウセイアイ)homosexuality; same-sex love.

優しい(やさしい)tender; kind; gentle; affectionate; suave

\ruby{愛}{まな}\ruby{弟}{デ}\ruby{子}{シ}favorite pupil/student.

\section{4 水 16 激}

激(ゲキ)violent.

激しい(はげしい)violent.

激しい雨(はげしいあめ)violent rain.

\section{4 中夬 5 央 6 仲 7 決}

中(チュウ、なか)middle.

中学(チュウガク)middle school; junior high school.

…の中(…のなか)middle of something.

田中(たなか)(name) Tanaka.

中山(なかやま)(name) Nakayama.

中川(なかがわ)(name) Nakagawa.

夬Kangxi radical 37.

央(オウ)center.
中央(チュウオウ)center; central.

仲(なか)relation; relationship.

仲間(なかま)company; fellow; colleague; associate; comrade; partner.

決(ケツ)decide.

決まる(きまる)to be decided; to be settled. to look good in clothes.

\subsection{9 映}

映(エイ)reflect.

\section{4 耂 6 考老}

耂 depicts an old man, a bent-over figure with long hair.

考(コウ)consider.

考える(かんがえる)(v1,vt) to consider; to think about.

老い(おい)old age; old (of person); at a late time in life.

老人(ロウジン)old person.

老若(ロウニャク)old and young; all ages.

\subsection{7 孝 11 教}

孝(コウ)filial piety.

教(キョウ)teach.

教育(キョウイク)training; education.

教会(キョウカイ)church.

教え(おしえ)teaching; doctrine.

教える(おしえる)(v1,vt) to teach.

\subsection{8 者 11 著都 14 緒}

者(シャ)(n,suf) someone of that nature; someone doing that work.

者(もの)(n) person (rarely used without a qualifier).

学者(ガクシャ)scholar.

作者(サクシャ)author.

業者(ギョウシャ)trader; merchant.

研究者(ケンキュウシャ)researcher.

著(チョ)renowned.

著書(チョショ)literary work; book; textbook.

著名人(チョメイジン)celebrity.

都(ト、ツ、みやこ)metropolis.

緒(ショ)thong.

一緒(イッショ)together.

\section{4 王 8 国 9 美皇}

王(オウ)king

国(コク、くに) country.

国王(コクオウ)king.

国内(コクナイ)internal; domestic.

美(ビ)beauty.

美人(ビジン)beautiful woman.

美しい(うつくしい)beautiful.

美味しい(おいしい)delicious (idiosyncratic reading).

皇(コウ)imperial.

皇居(コウキョ)imperial palace.

\section{4 元之文}

元(ゲン、ガン,もと)origin; beginning; source.
元(もと)origin; source.
元々(もともと)
originally; by nature; from the start; since the beginning.

之(これ)this

文(ブン)sentence (literature).
文字(モジ)letter (of alphabet); character (a Han character)
文書(ブンショ)sentence.
文化(ブンカ)culture.
作文(サクブン)writing.
小学生の作文(ショウガクセイのサクブン)elementary-schoolchild writing.

\section{4 今 6 合}

今(コン、いま)now.
今回(コンカイ)this time; this occasion; this occurrence.

合(ゴウ、あ)fit.
合う(あう)to fit.

\section{4 殳 7 投没 10 殺 11 設 15 撃}

殳 depicts a tool or a weapon.

投げる(なげる)to throw

没(ボツ)drowning

殺す(ころす)to kill.

殺害(サツガイ)murder.

殺人(サツジン)murder.

設ける(もうける)to establish.

撃(ゲキ)beat.

電撃(デンゲキ)electric shock.

衝撃(ショウゲキ)shock; crash; impact.

\section{3 尸 4 戸 5 冊 8 雨}

尸 Kangxi radical 44 (``corpse'').

戸(コ、と、べ)door.

神戸(こうべ)Koube (an area in Japan).

冊(サツ)counter for books.

冊子(サッシ)booklet.

小冊子(ショウサッシ)booklet; pamphlet.

一冊(イッサツ)one book.

雨(あ\め)rain

\subsection{6 尺 11 訳}

尺(シャク)shaku (a unit of length).

訳(ヤク)translate.

訳す(ヤクす)(vt) to translate.

\subsection{7 戻 所 10 涙 14 漏 15 編}

戻る(もどる)(vi) to turn back; to return; to go back;
to come back to a previously visited place

所(ところ)place.

近所(キンジョ)neighborhood.

名所(メイショ)famous place.

涙(なみだ)tear (eyewater)

漏れる(もれる)to leak (liquid)

編(ヘン)editing; compilation.

編集(ヘンシュウ)Edit (menu item in computer user interface).

編む(あむ)(vt)
to knit; to plait; to braid.
to compile (an anthology); to edit.

\section{4 勿 8 易 12 陽場}

勿(ブツ)

易(エキ)easy; simple.

易しい(やさしい)(adj-i) easy; plain; simple.

交易(コウエキ)trade; commerce.

易い(やすい)(adj-i) easy (not difficult).

陽(ヨウ)the yang in yin and yang.

太陽(タイヨウ)sun.

場(ジョウ、ば)place.

場所(ばショ)place.

場合(ばあい)case; situation.

\section{4 日月 9 韋}

日(ひ)sun.

日々(ひび)daily; days; old days.

月(つき)moon.

韋 is Kangxi radical 178 (``tanned leather'').

\subsection{5 白 7 伯 8 拍泊迫}

白(ハク)white.

白い(しろい)white.

伯(ハク)chief.

拍(ハク)clap.

泊(ハク)overnight.

迫(ハク)urge.

\subsection{8 昔 10 借}

昔(セキ、むかし)long ago.

借(シャク)borrow.

\subsection{12 散}

散(サン)scatter.

\subsection{12 間温}

間(カン、ケン)interval; space.

間(あいだ)gap; interval; distance; span; stretch (space or time).

間(ま)space; room; time; pause.

人間(ニンゲン)human being.

世間(セケン)world; society.

温かい(あたたかい)(adj-i) warm (of a tangible object; to the touch)

\subsection{9 昼}

昼(チュウ、ひる)daytime.

\subsection{6 早 9 草 12 朝}

早い(はやい)(adj-i) early.

草(くさ)grass

朝(チョウ、あさ)morning.

今朝(けさ)this morning.

早朝(ソウチョウ)early morning.

\subsection{8 明 9 春星}

明(メイ)bright.
明るい(あかるい)bright.

春(シュン、はる)spring (season)

星(セイ、ほし)star

\subsection{(6 早 9 韋) 8 𠦝卓 13 違 18 韓}

𠦝 is an alternative form of 卓.

卓(タク)eminent.

違(イ)differ.

相違(ソウイ)difference; discrepancy; variation.

違う(ちがう)(vi) to differ; to not match the correct answer.

間違う(まちがう)to make a mistake; to be incorrect; to be mistaken.

違反(イハン)violation of law.

韓(カン)Korea.

\section{4 公 5 広払 7 私 8 拡}

公(コウ、おおやけ)public; communal; official; governmental.

公安(コウアン)public safety; public welfare.

広(コウ、ひろ)wide.

広い(ひろい)(adj-i) spacious; vast; wide.

広告(コウコク)advertisement.

払(フツ)pay.

私(シ、わたくし、わたし)I; me.

拡(カク)broaden.

\section{4 幻 5 幼}

幻(まぼろし)phantom; vision; illusion; dream

幼(ヨウ、おさな)infancy.
幼い(おさない)very young; immature. childish.

\section{4 心 7 志快}

志(シ、こころざし)intention

快(カイ)cheerful.
快い(こころよい)cheerful.

\section{4 不 7 否}

不(フ)(prefix) not; bad; poor.

不安(フアン)anxiety; insecurity.

不明(フメイ)unknown; obscure; anonymous; unidentified.

否(イナ、いや)negate.

\section{(4 木) 7 村枚}

村(むら)village

枚(マイ)(counter) sheet; thin flat object.
一枚(イチマイ)one sheet.

\section{(4 戈) 6 伐成 7 戒}

伐(バツ)fell; strike; attack; punish.

成(セイ)become.

作成(サクセイ)(n,vs) writing; creation.

コンピュータプログラムを作成するto write a computer program.

戒(カイ)commandment.

十戒(ジッカイ)
(Buddhist) 10 precepts.
(Christian) 10 commandments.

戒める(いましめる)(vt) to admonish.

\section{(4 god radical) 7 社 10 神}

社(シャ)company; firm; association; shrine

社(やしろ)shrine (usually Shinto).

神(かみ)god; spirit; thunder

\section{4 水 7 沈 10 浮}

沈下(チンカ)sinking; subsidence.

沈む(しずむ)(vi) to sink (descend into liquid).
This was simplified from 18 瀋.

浮かぶ(うかぶ)to float (be supported by liquid)

\section{4 水 9 津}

津(つ)seaport; harbor.

津波(つなみ)(n) tsunami; tidal wave.

\section{4 壬 6 任 13 賃}

壬(ニン、ジン、みずのえ)depicts a carrying pole.

任(ニン)responsibility.

賃(チン)fare.

\section{4 天夫 5 矢失}

天(テン)sky; heaven

夫(フ、おっと)husband

矢(シ、や)arrow

失 depicts something falling from a hand.
失う(うしなう)(vt) to lose; to part with.
失明(シツメイ)loss of eyesight.
失血(シッケツ)loss of blood.

\section{4 止 8 歩}

止 depicts a footprint.

止(シ)stop.

止まる(とまる)(vi) to stop (moving); to come to a halt.

止める(やめる)(v1) to stop (doing something).

止める(とめる)(v1) to stop moving (walking, etc.); to park (a car).

歩(ホ、フ、ブ)walk.

歩く(あるく)to walk.

\subsection{6 企}

企(キ)plan.

\subsection{7 足⻊ 14 踊}

足(ソク、あし)foot.
Kangxi radical 157.

⻊ is the component form of 足 (foot).

踊(ヨウ)jump.

\subsection{9 是 12 提 18 題}

是(ゼ)just so.

提(テイ)present.

提供(テイキョウ)offer.

題(ダイ)topic.

問題(モンダイ)problem.

\subsection{5 正 10 症}

正(セイ、ショウ)correct.

不正(フセイ)(adj-na,n) injustice; illegality; fraud.

不正なソフトウェアmalicious software.

正しい(ただしい)right; correct.

症(ショウ)(n,suf) illness.

\subsection{7 走 8 定}

走(ソウ)run.

走る(はしる)(v5r,vi) to run

定(テイ、ジョウ)fix; determine; establish; settle; decide.

安定(アンテイ)stability; equilibrium.

予定(ヨテイ)(n,vs) plan; arrangement; schedule; program.

定住(テイジュウ)settlement; permanent residence.

定める(さだめる)(v1,vt) to decide; to establish; to determine.

未定(ミテイ)not yet fixed; undecided; pending.

\section{4 斤 7 辛}

\subsection{7 近 8 析}

斤 depicts an axe.

辛 depicts a tool used to mark slaves and criminals;
this sometimes also depicts a tree.

辛い(からい)(adj-i) spicy, salty, harsh, hot, acrid.

辛い(つらい)(adj-i) bitter; painful; heartbreaking; difficult.
Suffix づらい(adj-i) means ``difficult to do''.

読みづらい(adj-i) difficult to read.

書きづらい(adj-i) difficult to write.

読みづらい漢字difficult-to-read Han character.

近い(ちかい)(adj-i) near (spatial distance).

近々(ちかぢか)soon.

近作(キンサク)recent work.

最近(サイキン)most recent; recently; these days; nowadays.

析(セキ)chop.

分析(ブンセキ)analysis.

\subsection{13 新 16 親}

新 depicts cutting tree down with axe.

新(シン)new.

新聞(シンブン)news.

新車(シンシャ)new car.

最新(サイシン)newest.

新しい(あたらしい)(adj-i) new.

親(シン、おや)parent.

両親(リョウシン)both parents.

親友(シニュウ)close friend.

母親(ははおや)mother.

\chapter{Kanji 5}

\section{5 乍 7 作 9 昨}

作(サク)make; work; harvest.

作る(つくる)to make.

昨(サク)previous.

昨日(きのう)yesterday.

\section{5 㠯 8 官 9 追}

官 depicts many rooms in a building.

官(カン)government official.

官界(カンカイ)bureaucracy.

長官(チョウカン)secretary; director; chief; director general.

追(ツイ、お)chase; follow; pursue.

追う(おう)to chase; to run after; to follow.

追い風(おいかぜ)tailwind; favorable wind.

\section{5 可司 6 向同回 10 高}

可(カ)passable; acceptable; tolerable.

許可(キョカ)permission; authorization; approval.

司(シ)director.

司る(つかさどる)to rule; to govern.

向 depicts a house and a window.

向かい(むかい)(n) facing; opposite; across the street; other side.

向く(むく)to face; to turn toward.

向ける(むける)(v1,vt) to turn towards.

向こう(むこう)opposite side; other side; opposite direction.

向上(コウジョウ)improvement; advancement; progress.

同(ドウ、おな)same.

同じ(おなじ)same.

同性愛(ドウセイアイ)same-sex love.

回 depicts a spiral.

回す(まわす)(vt) to turn; to rotate.

今回(コンカイ)this time.

一回(イッカイ)once; one time.

一回目(イッカイめ)first.

高(コウ)high.

高い(たかい)high; expensive.

\subsection{7 何 8 河}

何(カ、なに、なん)what.

河(かわ)river; stream.

河川(カセン)rivers.

大河(タイガ)large river.

\subsection{10 哥 14 歌}

哥 means older brother.

歌(カ、うた)song.

歌声(うたごえ)singing voice.

歌う(うたう)to sing.

\subsection{13 稿}

稿(コウ)draft; copy; manuscript

\section{5 业}

业 is an alternative form of 北 (north).

\subsection{8 並 12 普}

並 depicts two people standing side-by-side.

並(ヘイ、なみ、なら)row.

並(なみ)row.

並ぶ(ならぶ)to stand in line; to line up.

普(フ)universal; wide; general.

普通(フツウ)(adj-no) general; ordinary; usual.

普通の人間(フツウのニンゲン)ordinary human.

\subsection{13 業 14 菐僕}

業(ギョウ)industry.

工業(コウギョウ)manufacturing industry.

企業(キギョウ)enterprise; business.

菐 is the 14-stroke ``thicket'' radical.

僕(ボク)I; me (male).

\section{5 左右 8 若}

左(ひだり)left.

右(みぎ)right.

若い(わかい)young; at an early time in life.
若年(ジャクネン)the time when one was young.

\section{5 玉 8 咅}

玉(ギョク、たま) ball

咅 means ``to spit out''.

\subsection{6 全}

全 depicts a whole piece of jade.

全(ゼン)whole.

全国(ゼンコク)countrywide; national.

全く(まったく)(adv) completely, entirely, wholly, totally

\subsection{10 剖 11 部}

剖(ボウ)dissection.

部(ブ)section; department; part.

市部(シブ)urban areas.

全部(ゼンブ)all; entire; whole.

学部(ガクブ)a department in a faculty in a university.

東部(トウブ)eastern part.

部門(ブモン)division (of a larger group).

一部(イチブ)one copy (of a document).

\section{5 主 7 住 8 往注}

主(おも)chief; main; principal; important.

主人公(シュジンコウ)hero; main character.

住(ジュウ)dwelling; living.

永住(エイジュウ)permanent residence.

居住(キョジュウ)residence.

住人(ジュウニン)inhabitant; resident; dweller.

住む(すむ)to live (of humans); to reside; to inhabit.

日本に住んでいるto be living/residing in Japan.

往(オウ)outward; journey.

注(チュウ)pour.

注文(チュウモン)order (in a restaurant); request.

\section{5 以}

以(イ)by means of.

以前(イゼン)previously; formerly; the former.

以上(イジョウ)above; more.

10以上above 10; over 10; more than 10.

以下(イカ)below; less.

10以下below 10; under 10; less than 10.

\section{5 去 8 法}

去(キョ、コ)leave

法(ホウ)method.

\section{5 令 (6 米) 7 冷}

令(レイ)orders

冷たい(つめたい)(adj-i) cold (of a tangible object; to the touch)

\subsection{12 奥}

奥(おく)interior.

奥山(おくやま)remote mountain.

\subsection{12 歯 17 齢}

歯(シ、は)tooth.

齢(レイ)age.

齢は、23\ruby{歳}{サイ}。The age is 23 years.

\section{5 号}

号(ゴウ、よびな、さけ)number

\section{5 皿且}

皿(さら)dish, plate

且つ(かつ)and.
且又(かつまた)besides; furthermore; moreover

\section{5 示申}

示 depicts a spirit.
There are two Unicode codepoints:
⽰ (U+2F70 in the CJK Radicals Supplement block)
and 示 (U+793A in the CJK Unified Ideographs block).
To machines they differ,
but to humans they look the same.

示(シ)indicate.

示す(しめす)to indicate.

申 depicts a bolt of lightning.
申す(もうす)(humble,vt) to say; to speak.

\section{5 世台}

世(よ)world; society; age; generation

台(ダイ、タイ、うてな)tower; stand; pedestal.
台 is simplified form of 14 臺.
仙台(センダイ)(city name) Sendai.

\section{5 冬 10 夏 11 終}

冬(トウ、ふゆ)winter

夏(カ、なつ)summer

終(シュウ)end.

最終(サイシュウ)last; final; closing.

終了(シュウリョウ)(n,vs)
end; close; termination.
to quit or exit (a computer program).

終わる(おわる)to finish; to end; to close.

\section{5 犯 8 狙 9 狭}

犯(ハン、おか)crime.
犯す(おかす)to commit (a crime); to perpetrate (a crime).

狙う(ねらう)(vt) to aim at

狭める(せばめる)(v1,vt) to narrow.
狭い(せまい)(adj-i) narrow; confined; small.

\section{5 写}

写生(シャセイ)sketch.
写真(シャシン)multimedia; photograph; movie
写す(うつす)(vt) to photograph.

\section{5 癶 7 豆}

癶 depicts footsteps, dotted tent, or legs.

豆(トウ、まめ)beans; pea.

\subsection{9 発 12 廃}

発(ハツ、ホツ)departure.

発車(ハッシャ)(n,vs) departure of a vehicle.

発達(ハッタツ)development; growth.

発言(ハツゲン)utterance; speech; proposal.

発信 (ハッシン) (n,vs) dispatch; transmission; submission.

廃(ハイ)abolish.

廃人(ハイジン)crippled/disabled/invalid person.

\subsection{12 登}

登(トウ)climb.

\section{5 史 6 吏 7 更 8 使 9 便}

史(シ)history.

史家(シカ)historian.

吏(リ)officer.

更(コウ)grow late.

更ける(ふける)(vi) to get late; to advance; to wear on.

使(シ)use.

使用(シヨウ)(n) use.

使う(つかう)to use.

便(ベン、ビン)convenience.

便利(ベンリ)convenient; handy; useful.

不便(フベン)inconvenience.

便所(ベンジョ)lavatory.

大便(ダイベン)feces; excrement; shit.

便り(たより)news; tidings; information; letter.

\section{5 召 8 沼 11 紹}

召す(めす)(honorific) to invite; to eat

沼(ぬま)swamp; bog

紹(ショウ)introduce.

紹介(ショウカイ)introduction; referral.

\section{5 皮 8 波 10 疲}

皮(かわ)skin; hide.

波(なみ)(n) wave (of liquid)

疲(ヒ)exhausted.

疲れる(つかれる)(vi) to get tired.

\section{5 目}

目(め)eye

\subsection{9 相省冒 11 眼}

相(ソウ、あい)mutual.
相手(あいて)companion; partner; company.

相(ショウ)minister.
首相(シュショウ)prime minister; chancellor; premier.

省(ショウ)(suf) ministry; department.
国交省(コッコウショウ)(abbr)
Ministry of Land, Infrastructure, Transport, and Tourism.
省く(はぶく)(vt)
to omit; to leave out; to exclude.
to curtail; to save; to cut down; to economize.

冒 depicts a hat obstructing the sight, implying rashness
(acting without enough thought).
冒す(おかす)(vt) to risk.
冒険(ボウケン)adventure.

眼(まなこ)eyeball.
両眼(リョウガン)both eyes.

\subsection{17 瞳}

瞳(ひとみ)pupil (of the eye).

\subsection{7 見 11 現 12 覚}

見る(みる)(v1,vt) to see.

見える(みえる)(v1,vi) to appear.

現(ゲン)appear.

表現(ヒョウゲン)expression; presentation.

現す(あらわす)(vt) to reveal; to show; to display.

現れる(あらわれる)(v1,vi) to appear; to become visible; to materialize.

覚める(さめる)(v1,vi) to wake up

\section{5 用 10 通 12 痛備}

用(ヨウ、もち)use.

用いる(もちいる)(v1,vt) to use; to make use of; to utilize.

常用(ジョウヨウ)(n) common-use; in common use; commonly used.

痛(ツウ、いた)pain; hurt; damage; bruise.

苦痛(クツウ)pain; agony; bitterness.

痛む(いたむ)to hurt; to feel pain. to be injured.

痛い(いたい)painful; sore.

胃が痛い(イがいたい)The stomach is aching.

備(ビ)provision.

備考(ビコウ)note; remarks; nota bene; NB.

備える(そなえる)to provide; to equip; to install.

\section{5 冉 6 再 10 冓 14 構}

冉 may mean ``red'', ``tan'', ``weak'', or ``proceed gradually''.

再(サイ、ふたた)again, re-

再び(ふたたび)again; once more; a second time.

再生(サイセイ)playback; rebirth

再開(サイカイ)reopening

再来(サイライ)return, comeback

冓(コウ) depicts blocks of wood.

構(コウ)construct.

結構(ケッコウ)nice.

\chapter{Kanji 6}

\section{6 在存有}

在(ザイ、あ)exist.

在る(ある)to exist.

自在(ジザイ)freely; at will.

有 depicts a hand holding a piece of 肉(meat).

有る(ある)to exist.

存(ソン)suppose.

存じる(ゾンじる)(v1,humble) to think, feel, consider, know.

存在(ソンザイ)existence; being.

共存(キョウゾン)coexistence.

存亡(ソンボウ)life-or-death; existence; destiny.

\subsection{9 陏 12 堕}

陏 is an uncommon kanji.

堕(ダ)degeneration; degradation.

堕胎(ダタイ)abortion; feticide; babykilling.

堕落(ダラク)depravity; corruption; degradation.

\section{6 共 8 供 9 洪}

共(とも)companion; follower; attendant; retinue.

供える(そなえる)(v1,vt) to offer; to sacrifice; to dedicate.

子供(こども)child.

洪水(コウズイ)(n) flood (of liquid)

大水(おおみず)(n) flood (of liquid)

\section{6 旨 9 指}

旨(むね)center; principle; meaning.

本旨(ホンシ)main object; principal object; true aim.

旨い(うまい)(adj-i) delicious.

指(ゆび)finger.

指す(さす)(vt) to point.

目指す(めざす)(vt) to aim at.

\section{6 曲 8 典}

曲(キョク)music.

作曲(サッキョク)musical composition.

典(テン)code.

\section{6 争当糸糹多竹}

争(ソウ)conflict.

争う(あらそう)
to dispute; to quarrel.
to compete; to contest; to contend.

当(トウ、あ)hit.
当てる(あてる)(v1,vt) to hit.
本当(ホントウ)truth; reality.
本当に(ホントウに)truly; really; seriously.

糸 combines 幺 and 小.
糸(いと)thread.

糹 is the left form of 糸.

多 depicts two pieces of meat.
多(ア、おお)many.
多(タ)(prefix) multi-.
多い(おおい)(adj-i) many; numerous.

竹(たけ)bamboo.
竹林(チクリン)bamboo thicket.

\section{6 死光}

死(シ)death.
死ぬ(しぬ)to die.
死亡(シボウ)death; mortality.
死去(シキョ)death.

光(コウ、ひかり)light (electromagnetic wave)

\section{6 艮 7 良 8 長 9 食 12 飲}

艮 depicts eye and spoon.

良(リョウ)good.

良い(いい)good.

良く(よく)well.

長(チョウ)
long (distance or time).
leader.
eldest.

長い(ながい)long (distance); long (time).

長女(チョウジョ)eldest daughter; first-born daughter.

市長(シチョウ)mayor (a government official).

身長(シンチョウ)height (of body).

最長(サイチョウ)longest, tallest.

社長(シャチョウ)company president.

食(ショク)eat.

食物(ショクもの)food.

食べ物(たべもの)food.

食べる(たべる)(v1) to eat.

食う(くう)(male, vulgar) to eat.

飲む(のむ)to drink (any liquid, not just liquor)

\section{6 交 10 校}

交(コウ、まじ)intersect.
交わる(まじわる)cross; intersect; join; meet.

国交(コッコウ)diplomatic relations

学校(ガッコウ)school

\section{6 式}

式(シキ)ceremony.

式(シキ)style.
公式(コウシキ)formal; official.
公式ブログ official blog.

式(シキ)numerical formula.

\section{6 次 7 吹 9 姿}

次(つぎ)next (in sequence)

吹(スイ)blow.

吹く(ふく)to blow (wind, etc.).

吹雪(ふぶき)snow storm; blizzard.

姿(シ、すがた)figure.

\section{6 先 9 洗}

先(さき)before; ahead; previous; future.

先生(センセイ)teacher; master; doctor.

先日(センジツ)a few days ago; the other day.

洗う(あらう)(vt) to wash

\section{6 気 7 汽}

気(キ)spirit; mind; air; atmosphere

元気(ゲンキ)health(y); vigor; vitality; spirit.

天気(テンキ)weather.

気持ち(きもち)feeling.

汽(キ)steam.

汽車(キシャ)steam train.

\section{(6 糸) 7 系}

系(ケイ)lineage.

\section{(6 糸) 10 紙 11 細}

紙(シ、かみ)paper.

手紙(てがみ)letter (the document, not the alphabet).

細(サイ)thin.

細い(ほそい)(adj-i) thin; slender.

細る(ほそる)to become thin.

細かい(こまかい)(adj-i) small; trivial.

\section{(6 糸) 11 経}

経(ケイ、キョウ)manage.

経つ(たつ)(vi) to pass; to lapse.

経る(へる)to pass; to elapse; to experience.

\section{(6 糸) 11 率}

率(リツ)(suf) rate; ratio; proportion.

識字率(シキジリツ)literacy rate.

\section{6 西 10 配酒}

西(セイ、サイ、にし)west

配(ハイ)distribute.
配本(ハイホン)distribution of books.
配送センター(ハイソウセンター)distribution center.
配る(くばる)to distribute; to deliver.

酒(さけ)sake (a Japanese liquor)

\section{6 耳 14 聞}

耳(みみ)ear

聞(ブン、モン)hear.

聞く(きく)to hear.

\section{6 关 (10 笑) 9 咲送 14 関}

关 is a simplification of 10 笑 (to smile).

咲(ショウ)blossom.

咲く(さく)(vi) to bloom.

送る(おくる)to send.

見送る(みおくる)to see off; to escort (for parting); to farewell.

先送る(さきおくる)to postpone.

関(カン、せき)barrier; gate

\section{6 羊 12 達}

羊(ヨウ、ひつじ)sheep.

山羊(やぎ)goat.

達(タツ)attain.

\section{6 各 9 客 10 格}

各(カク、おのおの)each.

客(キャク、カク)guest.

格(カク)personality; character.

性格(セイカク)
character; personality; disposition; nature.

\chapter{Kanji 7}

\section{7 弟}

弟(テイ、ダイ、おとうと)younger brother.

弟(おとうと)(humble) younger brother.

弟さん(おとうとさん)(honorific) younger brother.

兄弟(キョウダイ)siblings;
brothers and sisters
(although the characters mean older brother and younger brother).

\section{7 児}

児 depicts an infant with imperfect cranium (fontanelles).

児 is simplified from 兒.

乳児(ニュウジ)infant; suckling baby.

男児(ダンジ)boy; son.

\section{7 臣 10 姫 17 覧}

臣(シン、ジン)retainer.
Kangxi radical 131.

姫(ひめ)princess.

覧(ラン)perusal.

回覧(カイラン)circulation.

回覧板(カイランバン)circular notice
(especially those distributed to households within a neighborhood association).

\section{7 旲 10 莫 12 募 13 墓 14 慕暮}

旲 is an uncommon kanji.

莫大(バクダイ)enormous; vast.

募(ボ)recruit.

募集(ボシュウ)recruiting; taking applications.

応募(オウボ)application (registration).

墓(ボ、はか)grave; tomb.

慕(ボ)pining.

慕う(したう)(vt) to yearn for; to love dearly.

Xを兄のように慕っている。
The implied entity loves X as if
X were the implied entity's own older brother.

暮(ボ、くら)livelihood.

暮らす(くらす)to live; to get along.

\section{7 赤 9 変 10 恋}

赤い(あかい)(adj-i) red.

変(ヘン)strange.

変わる(かわる)to change; to transform.

変身(ヘンシン)metamorphosis; transformation.

大変(タイヘン)(adv,adj-na,n)very.

恋(レン、こい)romance; love; tender passion.

恋人(こいびと)lover; sweetheart.

恋文(こいぶみ)love letter.

\section{7 車 10 庫 11 転 15 輪}

車(くるま)car; vehicle. wheel.

水車(スイシャ)water wheel.

庫(コ)warehouse.

車庫(シャコ)garage.

文庫(ブンコ)library; book collection. paperback book.

転(テン)revolve; turn around; change.

反転(ハンテン)rolling over; turning around.

転ぶ(ころぶ)(vi) to fall down; to fall over.

輪(リン、わ)ring; circle; hoop; wheel.

指輪(ゆびわ)ring (finger accessory).

車輪(シャリン)car wheel.

五輪(ゴリン)the Olympics.

\subsection{9 軍 10 連 12 運}

軍(グン)army.

連(レン)connection; sequence; chaining.

連休(レンキュウ)consecutive holidays.

常連(ジョウレン)regular customer; regular patron.

連れる(つれる)(v1) to lead or take a person.

関連ニュース(カンレンニュース)related news.

運(ウン、はこ)carry; transport.

運ぶ(はこぶ)to carry; to transport; to move.

自動運転(ジドウウンテン)
automatic operation (machine); automatic driving (vehicle).

\section{7 改}

改める(あらためる)(v1,vt) to revise.

\section{7 我 13 義 20 議}

我(ガ、われ)ego; I; me; oneself.

自我(ジガ)self.

無我(ムガ)selflessness.

義(ギ)in-law.

義兄(ギケイ)elder-brother-in-law; elder stepbrother.

義姉(ギシ)elder-sister-in-law; elder stepsister.

義父(ギフ)father-in-law; foster father; stepfather.

定義(テイギ)definition (of terms).

原義(ゲンギ)original meaning.

同義語(ドウギゴ)synonym.

議(ギ)deliberation; consultation; debate; consideration.

議論(ギロン)argument; discussion; dispute; controversy.

\section{7 言}

言(ゲン、ゴン)say.

失言(シツゲン)slip of tongue; using improper words.

文言(ブンゲン)wording.

言葉(ことば)word; dialect.

言う(いう)to say.

\subsection{9 信}

信(シン)faith; trust.

信じる(シンじる)(v1,vt) to believe; to have faith in.

\subsection{9 計訂 10 記討 11 訪 12 詞 13 詳 14 語 15 談}

計(ケイ)measure; plan.

計画(ケイカク)plan; project; schedule; scheme; program; programme.

計算(ケイサン)forecast.

(?) 金利計算interest rate forecast.

計る(はかる)(vt) to measure.

訂正(テイセイ)correction; revision; amendment

記(キ)record.

記録(キロク)record.

記録するto set a record (such as in sports); to break a record.

記録をもつto hold such record.

記事(キジ)article (writing).

選り抜き記事(よりぬきキジ)selected articles.

新しい記事(あたらしいキジ)new articles.

記す(しるす)to record, to write down.

討(トウ)chastise.

討伐(トウバツ)subjugation; suppression.

討つ(うつ)to shoot at; to attack, defeat, destroy, avenge.

訪問(ホウモン)call; visit.

訪中(ホウチュウ)a visit to China.

訪日(ホウニチ)a visit to Japan.

訪米(ホウベイ)a visit to America.

訪れる(おとずれる)(v1,vt) to visit.

訪ねる(たずねる)to visit.

詞(シ、ことば)part of speech; words; poetry.

歌詞(カシ)song lyrics.

作詞(サクシ)song lyrics.

名詞(メイシ)noun.

詳(ショウ、くわ)detailed.

詳細(ショウサイ)details; particulars.

不詳(フショウ)unknown; unidentified; unspecified.

詳しい(くわしい)detailed; full; accurate.

語(ゴ)language.

日本語(ニホンゴ)Japanese language.

英語(エイゴ)English language.

用語(ヨウゴ)term.

言語(ゲンゴ)(linguistic) language.

プログラミング言語programming language.

語る(かたる)(vt) to talk; to tell; to recite.

談(ダン)discuss.

相談(ソウダン)consultation.

示談(ジダン)out-of-court settlement.

座談会(ザダンカイ)symposium; round-table discussion.

\section{7 攻}

攻(コウ、せめ)aggression.
攻める(せめる)(v1,vt) to attack; to assault; to assail.
攻防(コウボウ)attack and defense.

\section{7 君}

君(きみ)you.
…君(…クン)(suffix) Mr. (junior).
君主(クンシュ)ruler; monarch; sovereign.

\section{7 防}

防(ボウ)defense; resistance.

防ぐ(ふせぐ)to resist; to defend against.

防止(ボウシ)prevention; check.

\section{7 助 9 査}

助(すけ)assistance.
助(ジョ)(pref) help; rescue; assistant.
助ける(たすける)(v1,vt) to help.

査(サ)investigate.
巡査(ジュンサ)policeperson.
主査(シュサ)chief examiner; chief investigator.
査問(サモン)enquiry; hearing.

\section{7 医 8 知}

医(イ)medicine; healing; curing; doctor (medical)

知(チ)know.
知る(しる)to know.
日本人の知らない日本語the Japanese language that the Japanese people don't know

\section{7 身}

身(シン)somebody; person.

自身(ジシン)self.

私自身(わたしジシン)I myself; me myself.

出身(シュッシン)person's origin (town, city, country, etc.).

出身地(シュッシンチ)birthplace.

身長(シンチョウ)height (of body).

\section{7 売 13 続 14 読}

売 is simplified from 15 賣.

賣(バイ)sell.

賣る(うる)to sell.

続(ゾク、つづ)continue.

相続(ソウゾク)succession; inheritance.

存続(ソンゾク)duration; continuance.

続く(つづく)to continue.

読(ドク)read.

読者(ドクシャ)reader.

読む(よむ)to read.

\section{7 呆 9 保}

呆 depicts a child.

呆れる(あきれる)(v1,vi) to be amazed, astonished, astounded.

保する(ホする)to guarantee.

保つ(たもつ)to preserve.

保安(ホアン)peace preservation; security.

\section{7 豕 10 家 11 豚 12 象}

豕(いのこ)(pig radical).

家 has at least 3 meanings, depending on how it is read.

家(カ)-er; -ist; someone who does something.

書家(ショカ)calligrapher.

画家(ガカ)painter.

漫画家(マンガカ)Japanese-comic-book-drawing artist.

活動家(カツドウカ)activist.

研究科(ケンキュウカ)researcher.

作家(サッカ)author; creator; writer; artist.

小説家(ショウセツカ)novelist; fiction writer.

政治家(セイジカ)politician; statesman.

作曲家(サッキョクカ)music composer.

史家(シカ)historian.

家(ケ)family.

中川家(なかがわケ)the Nakagawa family.

田中家(たなかケ)the Tanaka family.

マッカーサー家(マッカーサーケ)the MacArthur family; the MacArthurs.

家(うち)house.

「今夜私の家(うち)に来てください。」Please come to my house tonight.

豚(ぶた)pig.

象(ショウ、ゾウ)elephant.

対象(タイショウ)target; object (of worship, study, etc.).

\section{7 谷 10 浴 11 欲}

谷(コク、たに)valley.

浴(ヨク)bathe.

欲(ヨク)longing.

欲しい(ほしい)(adj-i) want.

...欲しい(ほしい)(aux-adj) I want you to ...

食べて欲しいI want you to eat.

黙って欲しいI want you to shut up.

\chapter{Kanji 8}

\section{8 京 10 原 13 源}

京(キョウ、ケイ、みやこ)capital.

京(キョウ)imperial capital.

原(ゲン、はら)meadow; field; plain; prairie; tundra; moor; wilderness.

原(ゲン)original; source; raw; origin.
原文(ゲンブン)original text.
原油(ゲンユ)crude oil.
原料(ゲンリョウ)raw materials.

起源(キゲン)origin; beginning; rise.

\subsection{12 就}

就(シュウ)concerning.

\section{8 亟 12 極}

亟(キョク)fast; quick; sudden; urgently; immediately; extremely.

極(キョク、ゴク)poles.
\(10^{48}\) (since the 17th century).

積極的(セッキョクテキ)assertive; proactive.

\section{8 㑒 9 品 10 馬}

㑒 was simplified from the 13-stroke 僉
meaning ``all, together, unanimous''.

品(しな)article; item; thing; goods; stock.

品(ヒン)quality.

品物(しなもの)goods.

作品(サクヒン)work (book; film; composition; etc.). opus.

日用品(ニチヨウヒン)daily necessities.

一品(イッピン)one item; one article; one course of meal.

上品(ジョウヒン)elegant; refined; polished.

品性(ヒンセイ)character (elegant attitude; elegant behavior).

馬(バ、うま、ま)horse.

\subsection{12 検}

検(ケン)examine.

検索(ケンサク)search.

検索エンジンsearch engine.

検索結果(ケンサクケッカ)search results.

コンピューターを検索Search in computer.

Wikipedia内を検索Search in Wikipedia.

検出(ケンシュツ)(n,vs) detection.

検討(ケントウ)consideration.

\ruby{不}{フ}\ruby{正}{セイ}なソフトウェアを検出しました
The implied entity detected malicious software.

\subsection{14 駆駅}

駆(ク)drive.
This was simplified from 驅.

駆ける(かける)(v1,vi) to run (horse).

駅 is simplified from 23 驛.
駅(エキ)train station.

\subsection{16 操}

操る(あやつる)(vt) to be fluent in (a language)

体操(タイソウ)physical exercise; gymnastics; calisthenics.

\subsection{18 験}

験(ケン)verify.

実験(ジッケン)experiment.

治験(チケン)clinical trial.

受験(ジュケン)taking an examination (such as school and university entrance).

経験(ケイケン)experience.

\ruby{未}{み}経験\ruby{者}{しゃ}inexperienced person.

未経験者\ruby{歓}{かん}\ruby{迎}{げい}inexperienced people are welcome.

\section{8 隹}

隹(ふるとり)depicts a short-tailed bird or an old bird.

\subsection{13 催}

催(サイ)sponsor.

開催(カイサイ)holding a meeting; opening an exhibition.

主催(シュサイ)sponsorship.

\subsection{18 雚 13 勧 15 歓 18 観}

雚 depicts a stork or heron.

勧(カン)persuade.

歓(カン)delight.

歓迎(カンゲイ)welcome; reception.

観(カン)observe.

観光(カンコウ)sightseeing.

観光客(カンコウキャク)tourist.

\subsection{12 雇}

雇(コ)employ.

解雇(カイコ)dismissal; firing; layoff.

雇う(やとう)to employ. to hire; to charter.

\subsection{12 集 13 稚 14 雑}

集(シュウ)gather; meet; congregate; swarm; flock.

集める(あつめる)(v1,vt) to collect; to assemble; to gather.

稚(チ)immature; young.

幼稚(ヨウチ)infancy; childish; infantile.

\subsection{13 準}

準(シュン)level; standard.

水準(スイジュン)water level. level; standard.

雜(ザツ)miscellaneous.

\subsection{15 誰 18 曜}

誰(だれ)who

曜(ヨウ)(weekday name).

日曜日(ニチヨウビ)Sunday.

\subsection{18 難}

難(ナン)difficult.

海難(カイナン)shipwreck.

災難(サイナン)disaster.

多難(タナン)(adj-na,n) full of troubles.

難点(ナンテン)fault; weakness.

難しい(むずかしい)difficult.

有難い(ありがたい)(adj-i) grateful; thankful.

\section{8 直 9 県 10 真}

直(ジキ)soon.

県(ケン)prefecture (an administrative division).

直(チョク)direct; in person; frankness; honesty.

直す(なおす)to heal; to cure.
Chinese 直 has 7 strokes.
Japan adds the lower-left corner stroke.

直(ただち)

真(シン、ま)true.

真意(シンイ)real intention; true motive; true meaning.

真っ黒(まっくろ)pitch black.

真っ先(まっさき)the foremost; the beginning.

\section{8 金}

金has a lot to do with metals.

金(キン、かね)gold; money.

金属(キンゾク)metal.
重金属(ジュウキンゾク)heavy metal.
These are also the chemistry terms.

金色(キンいろ、コンジキ)golden (color)

\subsection{13 鉄 14 銅銀 16 鋼}

鉄(テツ、くろがね)iron (lit. 黒金 black metal).

鉄人(てつジン)iron man; strong man.

鉄道(テツドウ)railroad; railway

銅(ドウ、あかがね)copper (lit. 赤金 red metal)

銀(ギン、しろがね)silver (lit. 白金 white metal)

銀行(ギンコウ)bank.

鋼(コウ、はがね)steel

青銅(セイドウ)bronze

鋼鉄(コウテツ)steel

\subsection{14 銃 19 鏡}

銃(ジュウ)gun; small firearms

鏡(かがみ)mirror.

眼鏡(めがね)eyeglasses.

\section{8 実店例}

実(ジツ)truth; reality.
実(み)fruit.
実る(みのる)to bear fruit; to ripen.

店(テン)(n) store; shop.
ラメン店ramen shop (ramen is a kind of Japanese noodle).

店(みせ).

例えば(たとえば)for example.
例える(たとえる)(v1,vt)
to compare; to liken; to illustrate.
用例(ヨウレイ)example; illustration.

\section{8 和}

和(ワ)peace; Japan.
平和(ヘイワ)peace.
和む(なごむ)(vi) to be softened; to calm down.
和らげる(やわらげる)(v1,vt) to soften; to moderate; to relieve.

\section{8 画}

画 means picture.
画(カク)counter for kanji strokes.
字画(ジカク)number of strokes in a character.
画家(ガカ)painter.
企画(キカク)plan.

\section{8 非 10 俳 12 悲 13 罪}

非(ヒ)un-; non-; negative; mistake; wrong.

非常(ヒジョウ)extraordinary; unusual

非ず(あらず)(exp) no; never mind.

俳(ハイ)haiku.

俳優(ハイユウ)actor; actress.

悲恋(ヒレン)disappointed love

悲しい(かなしい)sad.

罪(ザイ)sin; guilt.

罪(つみ)sin; crime; fault.

犯罪(ハンザイ)crime.

七つの大罪(ななつのダイザイ)seven deadly sins.

\section{(8 雨) 11 雪 12 雲 13 雷電}

雪(セツ、ゆき)snow

雲(ウン、くも)cloud

雷(かみなり)thunder.

電(デン)lightning.
電光(デンコウ)lightning.
電気(デンキ)electricity (lit. lightning spirit).
電話(デンワ)telephone (lit. lightning talk).
電車(デンシャ)electric train (lit. lightning carriage).
電気自動車(デンキジドウシャ)electric car.

\section{8 東 10 凍}

東(トン、ひがし、あずま)east

凍る(こおる)to freeze.
But the kanji for for ice is 氷(こおり).

\chapter{Kanji 9}

\section{9 面}

面(おもて)mask.

面白い(おもしろい)interesting.

\section{9 音 13 暗意 17 闇}

音(オン、おと、ね)sound.

本音(ホンね)real intention; motive.

暗(アン)dark.

暗い(くらい)dark.

意(イ)feelings; thoughts.

意見(イケン)opinion.

意欲(イヨク)motivation; will.

意図(イト)(n,vs) intention; aim; design.

意味(イミ)meaning; significance.

意味合い(イミあい)implication; nuance

小生意気(こなまイキ)cheekiness; impudence.

闇(やみ)darkness.

(?) 闇で全ては黒く見える。Everything looks black in the dark.
(全て or 皆?)

\section{9 専臭}

専門家(センモンカ)expert; specialist

負(フ)negative; minus.
負う(おう)to bear; to carry on one's back.
負かす(まかす)(vt) to defeat.
負ける(まける)(v1,vi)
to lose; to be defeated.
to succumb; to give in; to surrender; to yield.
to be inferior to.

臭: In China 10 strokes, in Japan 9 strokes.
The lower character is 犬 in China and 大 in Japan.
臭い(くさい)(adj-i) stinking; malodorous; ill-smelling.

\section{7 余 9 叙 10 除}

余(ヨ)leave over.

叙(ジョ)confer; relate; narrate; describe.

自叙伝(ジジョデン)autobiography.

除(ジョ、ジ、のぞ)exclude.

除く(のぞく)to exclude.

削除(サクジョ)elimination; cancellation; deletion.

\section{9 胃政}

胃(イ)stomach.

政(セイ、まつりごと)rule; government.

\section{9 面 11 異}

異(イ)uncommon; different; unusual.
異国(イコク)foreign country.
異性(イセイ)different sex; opposite sex.
異なる(ことなる)to differ; to vary; to disagree.

\section{9 首 12 道}

首(シュ、くび)neck

道(ドウ、みち)street; road.
鉄道(テツドウ)railway.

\section{9 乗}

乗(ジョウ)ride.

乗る(のる)(v1) to get on a public transport vehicle; to aboard; to embark.

\chapter{Kanji 10}

\section{10 唇}

唇(くちびる)(n) lips.

\section{10 笑料}

笑う(わらう)to laugh.
笑む(えむ)to smile.

料(リョウ)(suffix) material; charge; rate; fee

\section{10 帰}

帰(キ)homecoming.
帰る(かえる)(v5r,vi)
to return; to come home; to go home; to go back.

\section{10 鬼 14 魂 21 魔}

鬼(キ、おに)
ogre; demon.
spirit of a deceased person.
吸血鬼(キュウケツキ)vampire; blood-sucking demon.

魂(たましい)soul.
魂 is made of 云 (cloud) and 鬼 (ghost; demon; spirit).

魔(マ)witch; demon; evil spirit.
魔女(マジョ)witch.
魔法(マホウ)magic (magick, not the magical tricks); witchcraft; sorcery.
悪魔(アクマ)evil spirit.
邪魔(ジャマ)intrusion.
邪魔するto intrude.

\chapter{Kanji 11}

\section{11 曽 13 僧 14 増}

曽(ソウ)formerly.

僧(ソウ)monk; priest.

増(ゾウ)increase.

\section{11 済啓祭}

済ませる(すませる)(v1,vt) to finish; to end.
経済(ケイザイ)economy.

啓(ケイ).
拝啓(ハイケイ):拝啓… Dear ...

祭り(まつり)feast; festival

\chapter{Kanji 12}

\section{12 傘}

傘(かさ)umbrella.

\section{12 最}

最(サイ)most.

最も(もっとも)most.

日本の最も高い山(ニホンのもっともたかいやま)Japan's highest mountain.

世界で最も太い人(セカイでもっともふといひと)The fattest person in the world.

最小(サイショウ)smallest.

最大(サイダイ)biggest.

最初(サイショ)first.

最後(サイゴ)last.

最新(サイシン)newest.

最高(サイコウ)best, highest, tallest.

\section{12 着}

着(チャク、き、ぎ)wear; clothing.
古着(ふるぎ)old clothes; second-hand clothes.
下着(シタぎ)underwear.
上着(うわぎ)coat; jacket; outer garment.
水着(みずぎ)bathing suit; swimsuit.
着用(チャクヨウ)wearing (uniform; seat belt).
着る(きる)(v1,vt) to wear (in modern Japanese, from the shoulders down).

着(チャク、つ)arrival.
先着(センチャク)first arrival.
新着(シンチャク)new arrival.
着く(つく)to arrive; to reach.

\section{12 報}

報(ホウ)report; information; news.
報じる(ホウじる)(v1,vt) to report; to inform.
報いる(むくいる)(v1,vt) to reward; to recompense; to repay.

\section{12 無}

無(ム) no, -less, without.
無駄(ムダ)uselessness.
無用(ムヨウ)uselessness.
無敵(ムテキ)invincible, unrivaled (lit. no-enemy).
無茶(ムチャ)absurd, unreasonable (lit. no-tea).
無人(ムジン)unmanned (lit. no-human).
無言(ムゴン)silence (lit. no-say).

\section{12 番 18 翻}

番(バン)number.
番組(バンぐみ)television program.

翻 飜 飛 18 S  flip ホン、ひるがえ-る、ひるがえ-す

翻訳(ホンヤク)translation.

\chapter{Kanji 13}

\section{13 数}

数(スウ、かず)number; amount; count.
数える(かぞえる)(v1,vt) to count; to enumerate.
算数(サンスウ)arithmetics.
数万(スウマン)tens of thousands.

\section{14 概}

概要(ガイヨウ)outline; summary

\section{14 際}

際(サイ)occasion; circumstances.

際限(サイゲン)limits; bounds.

学祭(ガクサイ)interdisciplinary.

国際(コクサイ)international.

\section{15 質}

質(シツ)(suffix) substance; quality; matter.
質問(シツモン)question.

\section{15 膝}

膝(ひざ)knee; lap

\section{15 監}

監(カン)government official; rule; administer.
監禁(カンキン)confinement; bondage.

\section{16 諦頭}

諦める(あきらめる)(v1,vt)
to give up; to abandon.

頭(トウ、あたま)head

\section{18 顔類}

顔(ガン、かお)face.

顔面(ガンメン)face (of a person).

類(ルイ)kind; sort; type.

人類(ジンルイ)mankind.

\section{15 暴 19 爆}

暴 depicts the antler of a buck, representing a savage attack, a violence.

暴動(ボウドウ)insurrection; rebellion; revolt; riot; uprising.

暴風(ボウフウ)storm; windstorm; gale.

暴れる(あばれる)(v1,vi) to rage; to act violently.

爆(バク)burst; explode; bomb.

自爆(ジバク)suicide bombing; self-destruct.

水爆(スイバク)hydrogen bomb.

原爆(ゲンバク)atomic bomb; nuclear bomb.

空爆(クウバク)aerial bombing; air raid.

爆殺(バクサツ)killing by bombing.

爆死(バクシ)death by explosion.

爆音(バクオン)sound of explosion or detonation.

爆発(バクハツ)explosion; detonation; eruption.

\section{15 趣}

趣味(シュミ)hobby; taste, preference.

\chapter{Grammar 1}

Throughout the series, we will build and understand increasingly complex constructions.

\section{Noun}

The adjective of ``noun'' is ``nominal''.

Nouns do not change form.

人(ひと) person

\section{Dictionary-form verb}

辞書形(ジショケイ)dictionary form

Every dictionary-form verb ends with a u-sound (u, ru, su, ku, etc.),
but not every word ending with u-sound is a dictionary-form verb.
Example: 夜(よる)is a noun, not a verb.

食べる(たべる)to eat

歩く(あるく)to walk

笑う(わらう)to laugh

\section{Verbal noun}

A verbal noun is a noun that can be turned into a dictionary-form verb
by appending suru, zuru, jiru, or something similar:

感(カン) becomes 感じる which is a dictionary-form verb.

信(シン) becomes 信じる.

安心(アンシン) becomes 安心する (but 心 on its own does not become 心する)

勉強(ベンキョウ) becomes 勉強する.

A (verbal noun + jiru/suru/zuru/etc.) is a dictionary-form verb.

\chapter{Conjugables}

\section{I-adjective}

黒いblack

細いthin

\section{Existence and possession: 有る、在る、居る}

有る(ある)possess

在る(ある)exist (inanimate)

居る(いる)exist (animate)

リンゴが有る。
The implied entity has an apple.

リンゴが在る。
``There is an apple,''
or ``An apple exists,''
depending on context.

犬は居る。
There is a dog.

妹が居る。
The implied entity has a younger sister.

\section{Negation: 無い}

Irregular:

ある becomes 無い(ない): not; not exist

Regular:

v1-ru-to-nai

居る becomes 居ない.

食べる becomes 食べない.

v5-u-to-anai

笑う becomes 笑わない.

Examples:

リンゴが無い。
``The implied entity does not have any apples,''
or ``There are no apples,''
depending on context.

妹が居ない。
The implied entity does not have a younger sister.

\section{Perfection of dictionary-form verb: た、だ}

``Perfect'' means ``complete'', as in ``past perfect'', not flawless.
Every dictionary-form verb can be perfected into an perfect-form verb.
The result is still a verb.

笑う。
The implied entity laughs.
The implied entity will laugh.

笑った。
The implied entity laughed (and is no longer laughing now).
The implied entity has laughed.
The implied entity had laughed.

\section{Perfection of i-adjective: i-to-katta}

速い becomes 速かった

\section{Negation of i-adjective: i-to-kunai}

速い becomes 速くない

\section{i-to-kunakatta = i-to-kunai + i-to-katta}

\section{Adverbiation: i-to-ku}

速い becomes 速く

楽しく歩いたwalked happily

\section{Volitionalization: u-to-itai}

笑う becomes 笑いたい

話す becomes 話したい

\section{Conjunction: i-form}

食べる becomes 食べ

話す becomes 話し

立つ becomes 立ち

\section{Concurrency: X-while-Y: i-form of Y + nagara + X}

Japanese is consistently head-final,
and while-Y explains X,
so while-Y comes before X.
The modifier always comes before the head.

食べながら while eating

立ちながら while standing

食べながら話すto speak while eating

立ちながら食べるto eat while standing

\section{In doubt}

A verbal can be modified by adverbials.
A nominal can be modified by adjectivals.

\chapter{Nominal}

A \emph{nominal} is a construction that functions like a noun.

Every noun is a nominal.

\section{Conjunction: と}

Unlike English ``and'', Japanese と conjoins \emph{nominals} only.
English ``and'' can conjoin noun phrases or sentences.
田中さんと中川さんは東京に行く。Tanaka-san and Nakagawa-san goes to Toukyou.

[〈鳥を食べる人〉と〈魚を食べる人〉]が良いです。
People who eat chicken and people who eat fish are okay.

彼と私は同じ夢を見ました。
He and I had the same dream.

\section{Disjunction: か}

鳥か魚は良いです。
Chicken or fish are okay.
(Whichever of chicken or fish you give me, I will eat it.)

\section{Possession: の}

カラスの羽feather of crow

父の車father's car

彼の友達の娘 his friend's daughter

彼の妹の娘の友達his younger sister's daughter's friend

父の黒い車father's black car

細い俳優の車skinny actor's car

私の家の右the house at the right side of my house

私の家の右の右the house at the right side of the house at the right side of my house

\section{Modification}

Every nominal can be modified by prepending an adjectival.

Every i-adjective is an adjectival.

黒い車black car

悲しい人sad person

細い悲しい人thin sad person

魚を食べない人person who does not eat fish

魚を食べなかった人person who did not eat fish

結婚出来ない男a man who cannot marry

日本人が知らない日本語the Japanese language that the Japanese people do not know

日本人の知らない日本語the Japanese language that the Japanese people do not know

妹が居ない人
people who do not have younger sisters

お金がない人
people who do not have money

Every dictionary-form verb is an adjectival.

夜で魚を食べる人
people who eat fish at night

石を投げる子供stone-throwing child

愛されなかった人a person who was not loved

Every dictionary-form verb is an adjectival.

笑う人 smiling person

Every perfect-form verb is an adjectival.

笑った人 the person who smiled

黒い is an i-adjective.
車 is a noun.

車car.

黒いblack

黒い車 is an i-adjectival-modified nominal. It means ``black car''.

高い人tall person

黒くない車non-black car

黒かった車formerly-black car (a car that was black)

黒くなかった車formerly-nonblack car (a car that was not black)

Start with 黒い. Negate it to 黒くない
and then perfect it to 黒くなかった.

高い黒い車expensive black car

高い黒くない車expensive non-black car

高くない黒い車non-expensive black car

高くない黒くない車non-expensive non-black car

高かった黒かった車formerly-expensive formerly-black car

人person

笑う人laughing person

笑わない人non-laughing person; person who does not laugh; person who is not laughing

笑わなかった人formerly-non-laughing person; person who did not laugh

笑った人person who laughed

愛される人loved person

愛されない人unloved person

愛された人formerly-loved person

愛されなかった人formerly-nonloved person; a person who was not loved

Begin with 愛 (love).
Append する, producing 愛する (to love; who loves).
Passivate する to される by u-to-areru, producing 愛される (to be loved; who is loved).
Negate される to されない by v1-ru-to-nai, producing 愛されない (who is not loved).
Perfect されない to されなかった by i-to-katta, producing 愛されなかった (who was not loved).

話されなかった言葉
words that were not spoken

\section{In doubt}

愛される田中さんの女性Beloved Tanaka-san's lady (or ladies)

田中さんの愛される女性Tanaka-san's beloved lady (or ladies)

\chapter{Statement}

A statement can be assigned truth value.

\section{Predication}

車は高い。
Cars are expensive.

黒い車は良い。
Black cars are good.

中村さんは魚を食べる。
Nakamura-san eats fish.

\section{Subsumption (is-a)}

犬は動物。
Dog is a mammal.

中村さんは俳優。Nakamura-san is an actor.

\section{Equation}

あの人は田中。
That person is Nakamura.

\section{Existential quantification}

Relative clause is adjectival.

人が居る。
There is a human.
There are humans.

お金がない人が居る。
There are people who do not have money.

お金がない人が悲しい。
People who do not have money are sad (feel sad).
(This is just an example sentence.
It has nothing to do with the real world.)

\section{Example constructions: が、は}

お金がありません。The implied entity does not have money.

この山は高いです。This mountain is tall.

この川は広いです。This river is wide.

名前(なまえ):お名前は?
And your name is...?

\section{Example constructions: て下さい}

戸(と)、開く(ひらく)、下さい(ください):
戸を開いてく下さい。Please open the door.

\chapter{Grammar 2}

All truth values are to be interpreted probabilistically.
The statement ``everybody has a chicken'' is neither true nor false;
it has truth value somewhere between 0 and 1.

Probabilistic temporal modal logic?

I-phrase

I-clause

U-phrase

U-clause

A nominal is a thing that acts like a noun.

An adjectival is a thing that modifies a nominal.

\section{Predication and is-a}

車は高い。
Cars are expensive.
\[
    車(x) \vdash 高い(x)
\]

車は高いか?
Are cars expensive?
\[
    ? : 車(x) \vdash 高い(x)
\]

黒い車は良いです。
Black cars are good.
\[
    車(x), 黒い(x) \vdash 良い(x)
\]

犬は動物。
Dog is a mammal.
\[
    犬(x) \vdash 動物(x)
\]

中村さんは俳優です。Nakamura-san is an actor.
\[
    \vdash 俳優(中村さん)
\]
or
\[
    中村さん(x) \vdash 俳優(x)
\]

\section{Equation}

あの人は田中さんです。

\[
    あの人 = 田中さん
\]

\section{Relative clauses}

妹が居ない人
people who do not have younger sisters
\[
    人(x), \neg \exists y (y \text{ is an 妹 of } x)
\]

お金がない人
people who do not have money
\[
    人(x), \neg \text{possess}(x,お金)
\]

お金がない人がいる。
There are people who do not have money.
\[
    \exists x (人(x) \wedge \neg \text{possess}(x,お金))
\]

お金がない人が悲しい。
People who do not have money are sad (feel sad).
(This is just an example sentence.
It has nothing to do with the real world.)
\[
    人(x), \neg \text{possess}(x,お金) \vdash 悲しい(x)
\]

\section{Example constructions: が、は}

お金(おかね):お金がありません。The implied entity does not have money.

山(やま)、高い(たかい):
この山は高いです。This mountain is tall.

川(かわ)、広い(ひろい):
この川は広いです。This river is wide.

石(いし)、子供(こども)、投げる(なげる):
石を投げる子供stone-throwing child

名前(なまえ):お名前は?
And your name is...?

\section{Example constructions: て下さい}

戸(と)、開く(ひらく)、下さい(ください):
戸を開いてく下さい。Please open the door.

\section{Nouns made by conjoining nouns: と}

Unlike English ``and'', Japanese と conjoins \emph{nouns} only.
English ``and'' can conjoin noun phrases or sentences.
田中さんと中川さんは東京に行く。Tanaka-san and Nakagawa-san goes to Toukyou.
\[
    \vdash \text{destination}(東京),行く(田中さん),行く(中川さん)
\]

彼(かれ)、私(わたし)、同じ(おなじ)、見る(みる):
彼と私は同じ夢を見ました。
He and I had the same dream.
\[
    夢(彼,x), 夢(私,y) \vdash 同じ(x,y)
\]

\section{Example constructions: Genitive: の}

羽(はね)、黒(くろ):カラスの羽は黒い。The color of a crow's feathers is black.

Crowのfeatherのcolorはblack。

\[
    羽(x), カラス(y), \text{belongs-to}(x,y) \vdash 黒い(x)
\]

\section{Example constructions: Nominalization: の}

\subsection{In doubt}

犬(いぬ)、来る(くる)、来た(きた):

田中さんは来た。Tanaka-san came.
\[
    \vdash 来た(田中さん)
\]
About Tanaka-san, he came.
What Tanaka-san did was coming.

来たのは田中さんです。
About having come,
it is Tanaka-san who did that (and not somebody else).
\[
    来た(x) \vdash x = 田中さん
\]
If anyone came, then it was Tanaka-san.
Who came is Tanaka-san.
Tanaka-san is the person who came (and not somebody else).

勉強(ベンキョウ)、貴方(あなた):
勉強するのは貴方にいい。
Studyingはyouにgood。
Studying is good for you.
Parse tree:
貴方にいいmodifiesの,
貴方にいいのis the topic,
貴方にいいのはmodifiesいい.
Logic: good-for(studying,you).

貴方にいいのは勉強するのです。
What is good for youはstudyingです。
What is good for you is studying.
Parse tree:((貴方)に・いい・の)は・勉強・する・の・です。
Another possible parse tree:
(貴方)に・(いい・の)は・勉強・する・の・です。
For you, what is good is studying.

勉強するのは何のため?
What is studying for?
What is the purpose of studying?

\section{Example constructions: たい}

魚(さかな)、食べる(たべる):魚を食べたくありません。The implied entity does not want to eat fish.

\section{Questionable}

良い(いい)、良くなかった(よくなかった):
良くなかった良い事things that was good that was not good.

Does AのBとC parse as Aの(BとC) or (AのB)とC?

\chapter{Grammar 3}

All truth values are to be interpreted probabilistically.
The statement ``everybody has a chicken'' is neither true nor false;
it has truth value somewhere between 0 and 1.

Probabilistic temporal modal logic?

An adjectival is a thing that modifies a nominal.

If 「笑った。」 is true, then 「笑う。」 is false,
but there exists a past time interval where 「笑う。」 is true.

The negative form is an i-form, so the i-to-katta rule perfects it:
笑わない becomes 笑わなかった.

笑わなかった。The implied entity did not laugh.

Iff 「笑わない。」, then not 「笑う。」.

Iff 「笑わなかった。」, then 「笑わない。」.

Iff 「笑わなかった。」, then not 「笑った。」.

\subsection{In doubt: Nominalization: の}

犬(いぬ)、来る(くる)、来た(きた):

田中さんは来た。Tanaka-san came.
\[
    \vdash 来た(田中さん)
\]
About Tanaka-san, he came.
What Tanaka-san did was coming.

来たのは田中さんです。
About having come,
it is Tanaka-san who did that (and not somebody else).
\[
    来た(x) \vdash x = 田中さん
\]
If anyone came, then it was Tanaka-san.
Who came is Tanaka-san.
Tanaka-san is the person who came (and not somebody else).

勉強(ベンキョウ)、貴方(あなた):
勉強するのは貴方にいい。
Studyingはyouにgood。
Studying is good for you.
Parse tree:
貴方にいいmodifiesの,
貴方にいいのis the topic,
貴方にいいのはmodifiesいい.
Logic: good-for(studying,you).

貴方にいいのは勉強するのです。
What is good for youはstudyingです。
What is good for you is studying.
Parse tree:((貴方)に・いい・の)は・勉強・する・の・です。
Another possible parse tree:
(貴方)に・(いい・の)は・勉強・する・の・です。
For you, what is good is studying.

勉強するのは何のため?
What is studying for?
What is the purpose of studying?

\section{Example constructions: たい}

魚(さかな)、食べる(たべる):魚を食べたくありません。The implied entity does not want to eat fish.

\section{Questionable}

良い(いい)、良くなかった(よくなかった):
良くなかった良い事things that was good that was not good.

Does AのBとC parse as Aの(BとC) or (AのB)とC?

\section{In doubt}

If X is a clause, then Xの is a nominal. (?)
Or is this a のは particle?
Example:
〈魚を食べる・の〉は・田中さんです。
or
〈魚を食べる〉のは・田中さんです。
?

\section{Adjectival}

If X is an i-adjective, then X is an adjectival.

If X is an no-adjectival, then X is an adjectival.

\subsection{I-adjectival}

\subsection{Verbal adjectival}

A verb by itself readily forms an adjectival.

笑う

笑った

\subsection{No-adjectival}

If X is a nominal, then Xの is a no-adjectival.

\section{Predicate}

A predicate is a nominal, an i-adjectival, or a verbal.

\section{Clause}

A clause is an adjectival.

食べる(たべる)by itself can be the main clause
``The implied entity eats.'' or the relative clause ``who eats''.
「食べる。」The implied entity will eat.
「食べる人」The person who eats.

\section{What?}

Phrases.

形容詞(ケイヨウシ)

i-adjective

連用形(レンヨウケイ)

continuative form?

How verbs change forms.

Conjugation is inflection of verb.
Inflection is a change of form that does not change syntactic category.
Derivation is a change of form that changes syntactic category.

\chapter{Polite}

\section{Politification: u-to-imasu}

Irregular:

する becomes します

来る(くる) becomes 来ます(きます)

Regular:

在る becomes 在ります

居る becomes 居ます

笑う becomes 笑います

\section{Polite negation: masu-to-masen}

In polite speech, to negate a verb,
politely-negate the polite form.

笑わない becomes 笑いません.
Politify 笑う to 笑います.
Politely-negate 笑います to 笑いません.

Do not politify the negative form.
Doing so would turn 笑わない to 笑わないです,
which is not the polite counterpart of 笑わない.

Do not casually-negate the polite form.
Doing so would turn 笑います to 笑いましない,
which is not Japanese.

\section{Polite perfection: masu-to-mashita}

Politely-perfect the polite form.

笑います becomes 笑いました

Politify, politely-negate, and then politely-perfect: imasen-to-imasendeshita.

笑わなかった corresponds to 笑いませんでした

\chapter{Logic}

\section{Counting}

\subsection{Numbers: 1 一 2 二十八七九 3 三千万 4 五六 5 半四 6 両百}

一(イチ、ひと)one

二(ニ、ふた)two

三(サン、み)three

四 four

五 five

六 six

七 seven

八 eight

九 nine

十 ten

半(ハン)half

両(リョウ)both

百(ヒャク)hundred

千(セン)thousand

万(マン)ten thousand

\subsection{Rotation: 6 回}

回 depicts a spiral.
回す(まわす)(vt) to turn; to rotate.
一回(イッカイ)once; one time.
一回目(イッカイめ)first.

\section{Contrast and opposition}

\subsection{Negation: 4 不反 12 無}

不(フ)(prefix) not; bad; poor.
不安(フアン)anxiety; insecurity.
不明(フメイ)unknown; obscure; anonymous; unidentified.

反(ハン)anti-.
反する(ハンする)to oppose; to rebel; to revolt.
反体制(ハンタイセイ)anti-establishment.

無(ム) no, -less, without.
無駄(ムダ)uselessness.
無用(ムヨウ)uselessness.
無敵(ムテキ)invincible, unrivaled (lit. no-enemy).
無茶(ムチャ)absurd, unreasonable (lit. no-tea).
無人(ムジン)unmanned (lit. no-human).
無言(ムゴン)silence (lit. no-say).

\section{Opposite: 6 向}

向 depicts a house and a window.
向かい(むかい)(n) facing; opposite; across the street; other side.
向く(むく)to face; to turn toward.
向ける(むける)(v1,vt) to turn towards.
向こう(むこう)opposite side; other side; opposite direction.
向上(コウジョウ)improvement; advancement; progress.

\subsection{Versus: 7 対}

対(タイ)versus...
対する(タイする)to face each other.

\section{Productive abstract concepts}

\subsection{Turning-into: 4 化}

化(カ)(suffix) -ization, -ification.
グローバル化(グローバルカ)globalization.
化ける(ばける)(v1,vi) to take the form of.
化学(カガク)chemistry.
化石(カセキ)fossilization.
分化(ブンカ)specialization.

\subsection{Self: 6 自}

自(ジ)self.
自ら(みずから)(adv) personally.
自在(ジザイ)freely (at will).
自分(ジブン)self (context? example usage?).

\subsection{Repetition: 6 再}

再(サイ)again, re-

再生(サイセイ)playback; rebirth

再開(サイカイ)reopening

再来(サイライ)return, comeback

\subsection{Same: 6 同}

同じ(おなじ)same.
同性愛(ドウセイアイ)same-sex love.

\subsection{Whole: 6 全}

全 depicts a whole piece of jade.

全(ゼン) whole

全部(ゼンブ)altogether; everything

全く(まったく)(adv) completely, entirely, wholly, totally

\subsection{Every: 7 毎}

毎(マイ)every.
毎日(マイニチ)everyday.
毎月(マイゲツ、マイつき)every month.
毎時(マイジ)every hour.
毎回(マイカイ)every time (every time it happens); every occurrence.
毎年(マイネン、マイとし)every year.

\subsection{Most: 12 最}

最(サイ)most.
最も(もっとも)most.
日本の最も高い山(ニホンのもっともたかいやま)Japan's highest mountain.
世界で最も太い人(セカイでもっともふといひと)The fattest person in the world.
最小(サイショウ)smallest.
最大(サイダイ)biggest.
最初(サイショ)first.
最後(サイゴ)last.
最新(サイシン)newest.
最高(サイコウ)best, highest, tallest.

\section{Foreign: 6 外}

6 外(ガイ)foreign (not from somewhere nearby).
外人(ガイジン)foreigner, foreign person.
外国(ガイコク)foreign country.
外界(ガイカイ)outside world.
海外(カイガイ)foreign; abroad; overseas.

\section{Existence and truth}

\subsection{Existence: 6 在存有}

有 depicts a hand holding a piece of 肉(meat).
有る(ある)to exist.

存じる(ゾンじる)(v1,humble) to think, feel, consider, know.
存在(ソンザイ)existence; being.
共存(キョウゾン)coexistence.
存亡(ソンボウ)life-or-death; existence; destiny.

\subsection{Truth: 10 真}

真(シン)truth; reality

\section{Cause and reason: 5 由 6 因}

由(よし)cause; reason.

因(イン)cause; factor

\section{Geometry}

\subsection{Flat: 5 平}

平 can mean flat, level (not tilted), ordinary, plain, non-special.
平ら(たいら)flatness.
平たい(ひらたい)(adj-i) flat; even; level; simple.
平皿(ひらざら)flat dish.
平安(ヘイアン)peace; tranquility.
平気(ヘイキ)coolness; calmness; composure; unconcern.
平日(ヘイジツ)weekday; ordinary day (non-holiday).
平年(ヘイネン)normal (non-leap) year; normal year (related to harvest; weather).
公平(コウヘイ)fairness; impartiality; justice
水平(スイヘイ)level; horizontally
平文(ヘイブン)plain (non-encrypted) text.
平面(ヘイメン)level (flat and not-tilted) surface.

\subsection{Point: 9 点}

点(テン)point; spot; speck; mark.

\subsection{Shape: 7 形}

形(ケイ、かたち)shape; form.

\subsection{Circle: 3 丸}

丸(まる)circle

\subsection{Intersect: 6 交 10 校}

交わる(まじわる)cross; intersect; join; meet

国交(コッコウ)diplomatic relations

学校(ガッコウ)school

\subsection{Corner: 7 角}

角(カク、かど)corner

角(つの)horn (head protrusion)

\section{Spacetime points and intervals: 5 古 7 近 8 若長 9 前 13 新遠}

古 consists of 十(ten) and 口(mouth, generation).
古い(ふるい)old (not of person); ancient; obsolete.

近い(ちかい)(adj-i) near (spatial distance).
近々(ちかぢか)soon.
近作(キンサク)recent work.
最近(サイキン)most recent; recently; these days; nowadays.

若い(わかい)young; at an early time in life.
若年(ジャクネン)the time when one was young.

長(チョウ)
long (distance or time).
leader.
eldest.
長い(ながい)long (distance); long (time).
長女(チョウジョ)eldest daughter; first-born daughter.
市長(シチョウ)mayor (a government official).
身長(シンチョウ)height (of body).
最長(サイチョウ)longest, tallest.
社長(シャチョウ)company president.

老い(おい)old age; old (of person); at a late time in life.
老人(ロウジン)old person.
老若(ロウニャク)old and young; all ages.

前(まえ)before (time), in front of.
午前(ゴゼン)morning; before noon; a.m. (ante meridien).

新 depicts cutting tree down with axe.
新(シン)new.
新しい(あたらしい)new.
新聞(シンブン)news.
新車(シンシャ)new car.

遠い(とおい)(adj-i) far (spatial distance).

\chapter{Scratch space}

同人(ドウジン)literary group (coterie); same people; clique; fraternity; comrade; colleague.

1252	騒	騷	馬	18	S		boisterous	ソウ、さわ-ぐ

353	詰		言	13	S		packed	キツ、つ-める、つ-まる、つ-む

追い詰める(おいつめる)(v1,vt) to corner; to drive to the wall.

1783	憤		心	15	S		aroused	フン、いきどお-る

憤り(いきどおり)resentment; indignation.

1261	即	卽	卩	7	S		instant	ソク

1013	状	狀	犬	7	5		form	ジョウ

402	況		水	8	S		condition	キョウ

状況(ジョウキョウ)state of affairs (around you).

1709	膝		肉	15	S	2010	knee	ひざ

883	収	收	攴	4	6		take in	シュウ、おさ-める、おさ-まる

収まる(おさまる)to be in one's (supposed/intended) place.

1054	振		手	10	S		shake	シン、ふ-る、ふ-るう、ふ-れる

1809	返		辵	7	3		return	ヘン、かえ-す、かえ-る

振り返る(ふりかえる)(v5r,vi)to turn head; to look over one's shoulder. to turn around. to look back.

1359	恥		心	10	S		shame	チ、は-じる、はじ、は-じらう、は-ずかしい

43	逸	逸 [4]	辵	11	S		deviate	イツ

逸らす(そらす)(vt) to turn away; to avert.

問い(とい)(n) question.

全く(まったく)(adv) really; truly; entirely; completely; wholly; indeed.

2004	腰		肉	13	S		loins	ヨウ、こし

腰(こし)back; lower back; loins; hip; waist; lumbar region.

28	移		禾	11	5		shift	イ、うつ-る、うつ-す

乱暴(ランボウ)rude; lawless.

抓る(つねる)(v5r,vt) to pinch.

1807	片		片	4	6		one-sided	ヘン、かた

1043	伸		人	7	S		lengthen	シン、の-びる、の-ばす、の-べる

掴まる(つかまる)(v5r,vi)to be caught; to be arrested. to hold on to; to grasp.

巡る(めぐる)(v5r,vi) to go around.

1213	訴		言	12	S		sue	ソ、うった-える

訴える(うったえる)(v1,vt) to raise; to bring to (someone's attention).

1917	妙		女	7	S		exquisite	ミョウ

どんどんdrumming (noise). rapidly; steadily.

1035	触	觸	角	13	S		contact	ショク、ふ-れる、さわ-る

触れる(ふれる)(v1,vi) to touch; to feel.

1400	張		弓	11	5		stretch	チョウ、は-る

637	絞		糸	12	S		strangle	コウ、しぼ-る、し-める、し-まる

絞る(しぼる)to wring; to squeeze; to press; to extract.

730	搾		手	13	S		squeeze	サク、しぼ-る

しぼる:
絞る vs 搾る

1727	描		手	11	S		sketch	ビョウ、えが-く、か-く

565	弧		弓	9	S		arc	コ

1412	嘲 [7]		口	15	S	2010	ridicule	チョウ、あざけ-る

741	擦		手	17	S		grate	サツ、す-る、す-れる

1491	塗		土	13	S		paint	ト、ぬ-る

布団(ふとん)(ateji) futon (Japanese mattress)

404	挟	挾	手	9	S	1981	pinch	キョウ、はさ-む、はさ-まる

立てる(たてる)(v1,vt) to stand up.

2026	落		艸	12	3		fall	ラク、お-ちる、お-とす

落ちる(おちる)

落とす(おとす)

2117	弄		廾	7	S	2010	tamper with	ロウ、もてあそ-ぶ

弄る(いじる)(v5r,vt) to tamper with.

1603	濃		水	16	S		concentrated	ノウ、こ-い

756	残	殘	歹	10	4		remainder	ザン、のこ-る、のこ-す

煽る(あおる)(v5r) to fan; to agitate; to stir up.

1136	醒		酉	16	S	2010	be disillusioned	セイ

醒める(さめる)(v1,vi) to be disillusioned.

317	起		走	10	3		wake up	キ、お-きる、お-こる、お-こす

121	憶		心	16	S		recollection	オク

記憶(キオク)memory; recollection; remembrance. storage.

愛撫(アイブ)caress

1330	奪		大	14	S		rob	ダツ、うば-う

奪う(うばう)(v5u,vt) to snatch away.

すらすらsmoothly

426	極		木	12	4		poles	キョク、ゴク、きわ-める、きわ-まる、きわ-み

極まる(きわまる)(v5r,vi) to terminate; to reach an extreme.

1457	締		糸	15	S		tighten	テイ、し-まる、し-める

反応(ハンノウ)reaction; response.

顔(かお)face (person).

固い(かたい)(adj-i) hard; solid; tough.

1638	薄		艸	16	S		dilute	ハク、うす-い、うす-める、うす-まる、うす-らぐ、うす-れる

薄い(うすい)(adj-i) thin. pale; light. watery; dilute; sparse.

ごりごりscraping; scratching. hard (to the bite, to the touch).

抉る(えぐる)(v5r,vt) to gouge; to hollow out.

29	萎		艸	11	S	2010	wither	イ、な-える

萎える(なえる)(v1,vi) to wither. to droop. to be lame.

1421	沈		水	7	S		sink	チン、しず-む、しず-める

沈む(しずむ)(v5m,vi) to sink; to feel depressed.

408	胸		肉	10	6		bosom	キョウ、むね、(むな)

1916	脈		肉	10	4		vein	ミャク

2025	絡		糸	12	S		entwine	ラク、から-む、から-まる、から-める

深々(シンシン)(adj-t,adv-to)silent (especially of the passing of the night). piercing.

深々(フカブカ)(adv-to) very deeply.

1229	挿	插	手	10	S	1981	insert	ソウ、さ-す

挿入(ソウニュウ)insertion.

2046	律		彳	9	6		law	リツ、(リチ)

脈絡(ミャクラク)chain of reasoning; logical connection; coherence.

310	奇		大	8	S		strange	キ

1130	精		米	14	5		refined	セイ、(ショウ)

落書き(ラクがき)scrawl.

搾乳(サクニュウ)milking.

記念(キネン)commemoration.

2 乃

1513	透		辵	10	S		transparent	トウ、す-く、す-かす、す-ける

556	厳	嚴	口	17	6		strict	ゲン、(ゴン)、おごそ-か、きび-しい

1256	増	增	土	14	5		increase	ゾウ、ま-す、ふ-える、ふ-やす

増す(ます)to increase.

厳しさ(きびしさ)strictness.

含む(ふくむ)(v5m,vt) to contain.

1211	組		糸	11	2		association	ソ、く-む、くみ

取り組む(とりくむ)(v5m,vi) to tackle; to deal with.

1427	追		辵	9	3		follow	ツイ、お-う

追い付く(おいつく)(v5,vi) to catch up with.

1256	増	增	土	14	5		increase	ゾウ、ま-す、ふ-える、ふ-やす

増やす(ふやす)(v5,vt) to increase.

対応(タイオウ)interaction; correspondence.

実態(ジッタイ)true state; actual condition; reality.

653	拷		手	9	S		torture	ゴウ

拷問(ゴウモン)torture.

公開(コウカイ)open to the public; exhibit.

聞き取る(ききとる)to catch (a person's words); to follow; to understand.

拘束(コウソク)restriction.

806	示		示	5	5		indicate	ジ、シ、しめ-す

示す(しめす)(v5,vt) to denote; to show; to indicate.

明らか(あきらか)obvious.

告発(コクハツ)inditement; prosecution; complaint.

判明(ハンメイ)establishing; proving; identifying; confirming.

新た(あらた)new; fresh.

1529	踏		足	15	S		step	トウ、ふ-む、ふ-まえる

踏む(ふむ)(v5,vt) to step on.

1060	進		辵	11	3		advance	シン、すす-む、すす-める

前進(ゼンシン)advance; drive; progress.

待合室(まちあいシツ)waiting room.

低速(テイソク)low gear; slow speed.

走行(ソウコウ)running a wheeled vehicle.

1726	病		疒	10	3		sick	ビョウ、(ヘイ)、や-む、やまい

52	院		阜	10	3		institution	イン

病院(ビョウイン)hospital.

突入(トツニュウ)rushing; breaking into.

566	故		攴	9	5		circumstances	コ、ゆえ

事故(ジコ)accident; incident; trouble.

1437	低		人	7	4		low	テイ、ひく-い、ひく-める、ひく-まる

大分(ダイブン)considerably; greatly; a lot.

261	患		心	11	S		afflicted	カン、わずら-う

1002	障		阜	14	6		hurt	ショウ、さわ-る

507	穴		穴	5	6		hole	ケツ、あな

2104	裂		衣	12	S		split	レツ、さ-く、さ-ける

1607	破		石	10	5		rend	ハ、やぶ-る、やぶ-れる

破る(やぶる)to tear (such as paper).

1282	損		手	13	5		loss	ソン、そこ-なう、そこ-ねる

破損(ハソン)damage.

1415	調		言	15	3		investigate	チョウ、しら-べる、ととの-う、ととの-える

110	押		手	8	S		push	オウ、お-す、お-さえる

押す(おす)(v5,vt) to push.

押し退ける(おしのける)(v1,vt) to push aside.

調べる(しらべる)(v1,vt) to investigate.

1451	停		人	11	4		halt	テイ

喰う(くう)(male,vulgar) to eat

1430	通		辵	10	2		pass through	ツウ、(ツ)、とお-る、とお-す、かよ-う

妊娠(ニンシン)conception; pregnancy.

伝える(つたえる)(vt) to convey.

ふくよかplump; well-rounded.

聞く(きく)to ask; to enquire; to query.

無神経(ムシンケイ)thick-skinned; insensitive to criticism or insults.

質問(シツモン)question.

妊娠していないふくよかな女性
a plump woman who is not pregnant

無神経な質問
insensible question

妊娠していないふくよかな女性に「予定日はいつ?」と聞く無神経な質問。

関連企業(カンレンキギョウ)associated company; affiliated business.

387	挙	擧	手	10	4		raise	キョ、あ-げる、あ-がる

\ruby{4}{ヨ}\ruby{人}{ニン}に\ruby{1}{ひと}\ruby{人}{り}がX 1 in 4 people X.

新社会人(シンシャカイジン)new members of society (especially after turning 20 or joining a company); new working adults

997	傷		人	13	6		wound	ショウ、きず、いた-む、いた-める

傷付ける(きずつける)to hurt someone's feelings or pride

多過ぎる(おおすぎる)to be too many; to be too much

1497	怒		心	9	S		angry	ド、いか-る、おこ-る

怒り(いかり)anger.

挙げる(あげる)(vt)

上げる(あげる)

予定日(ヨテイび)scheduled date; expected date.

660	刻		刀	8	6		engrave	コク、きざ-む

深刻(シンコク)serious.

面倒臭い(メンドウくさい)bothersome (to do); troublesome.

愛情(アイジョウ)love; affection

安心感(アンシンカン)sense of security

1550	得		彳	11	4		acquire	トク、え-る、う-る

得る(える)(v1,vt) to get; to obtain.

560	呼		口	8	6		call	コ、よ-ぶ

悩む(なやむ)to be worried.

多い(おおい)many; numerous.

物言う(ものいう)to talk; to carry meaning.

気分(キブン)feeling; mood.

気持ち(きもち)feeling; sensation; mood.

対等(タイトウ)equality (especially of status or terms).

清楚(セイソ)neat and clean; tidy; trim.

生返事(なまヘンジ)half-hearted reply; vague answer; reluctant answer.

645	興		臼	16	5		entertain	コウ、キョウ、おこ-る、おこ-す

興味(キョウミ)interest (in something).

私に興味ないの?Aren't you interested in me?

533	献	獻	犬	13	S		offering	ケン、(コン)

1766	服		月	8	3		clothes	フク

1241	装	裝	衣	12	6		attire	ソウ、ショウ、よそお-う

1850	褒	襃	衣	15	S	1981	praise	ホウ、ほ-める

理想的(リソウテキ)ideal.

219	較		車	13	S		contrast	カク

比較(ヒカク)comparison.

共通(キョウツウ)

語らう(かたらう)(vt) to talk; to tell

155	暇		日	13	S		spare time	カ、ひま

あまりremainder; rest; residue; remnant

792	視	視 [4]	見	11	6		look at	シ

視点(シテン)opinion; point of view.

よりvsから?

293	含		口	7	S		include	ガン、ふく-む、ふく-める

含む(ふくむ)(vt) to contain

含める(ふくめる)(vt) to include

1836	放		攴	8	3		release	ホウ、はな-す、はな-つ、はな-れる、ほう-る

457	屈		尸	8	S		yield	クツ

1039	辱		辰	10	S		embarrass	ジョク、はずかし-める

屈辱的(クツジョクテキ)humiliating.

???
前で後で
前に後に
???

行動(コウドウ)action

感性(カンセイ)sensitivity

存分(ゾンブン)to one's heart's content; as much as one wants.

無意味(ムイミ)meaningless; nonsense.

料金(リョウキン)fee; price.

偏見(ヘンケン)prejudice.

外人は外国の方。

迷惑(メイワク)trouble; bother; annoyance.

自身(ジシン)self-confidence.

支配(シハイ)control.

太め(ふとめ)chubby; plump.

1191	遷		辵	15	S		transition	セン

1192	選		辵	15	4		choose	セン、えら-ぶ

1619	敗		攴	11	4		failure	ハイ、やぶ-れる

447	苦		艸	8	3		suffer	ク、くる-しい、くる-しむ、くる-しめる、にが-い、にが-る

閉じる(とじる) vs 閉める(しめる)?

窓を閉めるclose window (of a building)

タブを閉じるclose tab

1802	壁		土	16	S		wall	ヘキ、かべ

27	異		田	11	6		uncommon	イ、こと

481	掲	揭	手	11	S		put up (a notice)	ケイ、かか-げる

「小説を読もう!」は約471,337作品の小説が無料で読める小説サイトです。
``Shousetsu-wo yomou!''is a site of about 471,337 freely-readable literary-work novels/short-stories.

読める(よめる)(v1,vi)to be legible; to be readable.
to be pronounceable.
to be predictable.

9 耶(ヤ)question mark

有耶無耶(ウヤムヤ)indefinite; hazy; vague; unsettled; undecided.

1468	哲		口	10	S		philosophy	テツ

哲学者(テツガクシャ)philosopher

1192	選		辵	15	4		choose	セン、えら-ぶ

1273	属	屬	尸	12	5		belong	ゾク

無所属(ムショゾク)independent (in politics); non-partisan.

1770	福	福 [4]	示	13	3		luck	フク

幸福(コウフク)happiness.

先発(センパツ)forerunner

987	勝		力	12	3		win	ショウ、か-つ、まさ-る

決勝(ケッショウ)decision of a contest; finals (in sports).

大会(タイカイ)convention; tournament; mass meeting; rally

375	宮		宀	10	3		Shinto shrine	キュウ、グウ、(ク)、みや

23	為	爲	爪	9	S		do	イ

為(ため)sake; purpose; benefit.

お前の為に!
It's for your own good!

僕はお前の為にこんな事をしている。
I'm doing things like this for your sake.

What is the difference between 者 and 家(カ)?

掛ける、さえ

1704	眉		目	9	S	2010	eyebrow	ビ、(ミ)、まゆ

眉間(ミケン)glabella; middle forehead; area between the eyebrows.

13 睨

睨み(にらみ)glare; sharp look.

睨み付ける(にらみつける)(v1,vt) to glare at; to scowl at.

2133	惑		心	12	S		beguile	ワク、まど-う

231	掛		手	11	S		hang	か-ける、か-かる、かかり

874	種		禾	14	4		kind	シュ、たね

流石(さすが)(ateji)as one would expect.

布団(フトン)(ateji)futon (Japanese mattress).

「X」と「Y」は、どう違う?
How does X and Y differ?

「X」と「Y」の違いは?
What is the difference between X and Y?
Technically, we can parse it as
``What can you say about X and 「Y」の違い (the difference of Y)?''

素  糸 10 5  elementary ソ、ス
遊  辵 12 3  play ユウ、(ユ)、あそ-ぶ
織  糸 18 5  weave ショク、シキ、お-る
状 狀 犬 7 5  form ジョウ

現状

このページを翻訳しますか?Translate this page?

すべてall.

すべてのタブをブックマークに\ruby{追}{つい}\ruby{加}{か}Add all tabs to bookmarks.

次回(ジカイ)next time.

上書き(うわがき)(n,vs) overwrite.

ぴりりtingling; stinging; pungently.

報告(ホウコク)report; information.

1846 報  土 12 5  report ホウ、むく-いる

1343 端  立 14 S  edge タン、はし、は、はた
端末(タンマツ)(comp) (abbr) terminal; computer terminal.

為 爲 爪 9 S  do イ

行為(コウイ)act; deed; conduct.

指示(シジ)(n,vs) instruction

Xように(exp) in order to X

在日(ザイニチ)in Japan; people in Japan.

現状(ゲンジョウ)present condition; existing state; status quo.

そもそもin the first place

解決(カイケツ)(n,vs) solution.

取得(シュトク)acquisition.
資格(シカク)qualifications.
取得資格vocational certificate.

979	称	稱	禾	10	S		appellation	ショウ
愛称(アイショウ)pet name.

舞台(ブタイ)
stage (theater).

写真集(シャシンシュウ)photo album.

1153	籍		竹	20	S		enroll	セキ
書籍(ショセキ)
book; publication.

Xきっかけに(exp) with X as a start

291	鑑		金	23	S		specimen	カン、かんが-みる
鑑賞(カンショウ)appreciation (of art).

人口(ジンコウ)population.
人口10,000人(ジインコウイチマンニン)
Population: 10,000 people.

1720	票		示	11	4		ballot	ヒョウ
投票(トウヒョウ)voting; poll.

944	初		刀	7	4		first	ショ、はじ-め、はじ-めて、はつ、うい、そ-める
初出(ショシュツ)first appearance.
初め(はじめ)first doing of something.
Compare: 始め(はじめ)beginning of something.

努める(つとめる)(v1,vt)to endeavor

2048	略		田	11	5		abbreviation	リャク
2100	歴	歷	止	14	4		curriculum	レキ

略歴(リャクレキ)
brief personal record;
short curriculum vitae.

1038	職		耳	18	5		employment	ショク
就職(シュウショク)finding employment.

801	誌		言	14	6		document	シ
雑誌(ザッシ)magazine (a periodical publication).

387	挙	擧	手	10	4		raise	キョ、あ-げる、あ-がる

153	過		辵	12	5		go beyond	カ、す-ぎる、す-ごす、あやま-つ、あやま-ち

過去 (カコ) (n-adv,n)
past; bygone days.
a past (a personal history one would prefer remained secret).

過程(カテイ)progress; mechanism.

過ごす(すごす)(v5,vt) to pass time. to spend. to overdo.

1507	倒		人	10	S		overthrow	トウ、たお-れる、たお-す

1945	猛		犬	11	S		fierce	モウ

627	降		阜	10	6		descend	コウ、お-りる、お-ろす、ふ-る
降る(ふる)to come down.
雨降り(あめふり)rainfall; rainy weather.
雨が降る。It's raining.
雪が降る。It's snowing.
雨だ。It's raining.

事実上(ジジツジョウ)(n,adj-no)
as a matter of fact; actually; in reality.

長編小説(チョウヘンショウセツ)novel.

需 demand

要 need

一般(イッパン)general; universal; ordinary; average.

ぐるぐる

すてきlovely; beautiful; dreamy; great; superb; cool

% https://en.wiktionary.org/wiki/%E8%BE%9B%E3%81%84#Japanese
% https://en.wiktionary.org/wiki/%E8%BE%9B#Japanese

子供を作るto make a child

建築家(ケンチクカ)architect.

制作(セイサク)
work (film, book).
production; creation.

注目(チュウモク)attention.

活動(カツドウ)activity.

1594	念		心	8	4		thought	ネン

1700	碑	碑 [4]	石	14	S		tombstone	ヒ

懺悔(ザンゲ)repentance; confession; penitence

値打ち(ねうち)value; worth; price; dignity

寒い(さむい)cold (weather; wind); chilly

警察(ケイサツ)police

素晴らしい(すばらしい)

卒業(ソツギョウ)

擦り傷(すりきず)(n) scratch; graze; abrasion

…階建て(…カイだて)(suf) ...-story building.
7階建て 7-story building.

14 増える(ふえる)(v1,vi) to increase; to multiply

15 選 elect; select

13 連携(レンケイ)collaboration; cooperation

喧嘩(ケンカ)fight; brawl

博打(バクチ)gambling

お絵描き(おエかき)oekaki; painting; drawing

ネタバレspoiler (of a movie, a story, etc.); something that spoils the end of a movie, a story, etc.

% https://en.wikipedia.org/wiki/Administrative_divisions_of_Japan

\chapter{Grammar}

% https://en.wikipedia.org/wiki/Japanese_verb_conjugation

\section{Noun clause}

「文字を読み書き出来ない子供たちの未来」means
``the future of children who cannot read and write Chinese characters''.
The primary noun is 「未来」 (``future''),
which is modified by the adjective
「文字を読み書き出来ない子供たちの」
(``of children who cannot read and write Chinese characters'').
「文字を読み書き出来ない」(``unable to read and write Chinese characters'')
is an adjective that modifies 「子供たち」 (children).

\section{Vocabulary}

本(ほん)book

魚(さかな)fish

食べる(たべる)eat

読む(よむ)read

(S) is the implied subject.

\section{Present imperfect}

魚を食べる。(S) eats fish.

魚を食べない。(S) doesn't eat fish.

\section{Past perfect}

魚を食べなかった。(S) didn't eat fish.

本を読んだ。(S) read a book.

\section{Command}

魚を食べなさい。Eat fish.

魚を食べな。(childspeak?) Eat fish.

魚を食べて。Eat fish.

魚を食べないで。Don't eat fish.

\section{Passive}

食べられた。(S) was eaten.

読まれた。(S) was read.

\section{I want to ...}

食べたい。I want to eat.

読みたい。I want to read.

\section{Polite}

You have to learn to say the same thing all over again in different registers.

食べます(polite) to eat

食べません(polite) not eat

食べました(polite) ate

魚を食べました。(polite) (S) ate fish.

魚が食べられました。(polite)Fish was eaten. (???)

魚を食べません。(polite) (S) doesn't eat fish.

魚を食べないでください。(polite command) Please do not eat fish.

\section{But, although, despite, in spite of}

でも

けど

けれど

\section{Then}

そして

\section{So, thus, therefore}

だから

\section{To, for}

\section{From, to (place)}

から

より

\section{From, until (time)}

\section{In order to}

\section{Nevertheless}

\section{Whereas}

\section{On the other hand}

\section{Too bad ...}

\section{Because, because of, due to}

から

たら

ば

\section{About, regarding}

ついて

\section{About, approximately}

くらい

\section{Hmm...}

あの

えと

\section{Huh?}

え?

おれ?

あれ?

\section{Let's ...}

\section{Don't ...}

\section{Please ...}

\section{Are you sure?} 

\section{Why not ...}

\section{I think ...}

\section{Perhaps ...}

たぶん

もしかして

\section{I mean..., What I'm trying to say is..., How do I say this...}

\section{Absolutely}

\section{Precisely}

\section{Actually}

\section{Frankly}

\section{Indeed}

\section{Anyway}

\section{I'm sorry}

すま

すみません

ごめん

ごめんなさい

\section{Excuse me}

しつれいします

おじゃまします

\section{Long time no see}

おひさしぶり

\section{How dare you...}

\section{Like it or not}

\section{Or else}

\section{Questions: 7 何 15 誰}

何(なに)what

誰(だれ)who

\section{Uncategorized}

として

さすが

まじで


\end{document}
