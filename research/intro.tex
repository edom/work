\chapter{How to benefit from this book}

\epigraph{Young man, in mathematics you don't understand things. You just get used to them.}{John von Neumann (1903--1957) \cite{zukav1979dancing} (p. 208)}

This book is optimized for slow thoughtful focused sequential reading.
The author tries to make the book as short as possible.

\section{Notation}

A \emph{mathematical notation}, like the Latin alphabet,
is a way of writing \emph{English} or any other natural language.
When you read ``math'',
you are really reading the same language
that you speak everyday.

When you encounter symbols in a sentence,
think about how they should read in English
to make the whole sentence grammatically correct.

Beware that mathematical notations also have slangs and inconsistencies.

A large part of mathematics is about defining and naming things.
``\(A\) is \(B\) with some additional properties. \(B\) is \(C\) with some other properties.''

The reader is assumed to know the Greek alphabet.

\section{Why math seems hard}

Math seems hard because:
\begin{itemize}
    \item
        Learning a new language is hard.
    \item
        Understanding a mathematical sentence
        requires knowing a lot (tens to hundreds) of related definitions.
\end{itemize}

\section{Math is fun}

In mathematics, right is right, and wrong is wrong,
regardless of age, skin color, gender, weight, wealth,
nationality, religion, or political affiliation.
In mathematics, we don't have to worry about offending anyone.

John von Neumann:
``If people do not believe that mathematics is simple, it is only because they do not realize how complicated life is.''
