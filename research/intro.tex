\chapter{How to benefit from this book}

\epigraph{Young man, in mathematics you don't understand things. You just get used to them.}{John von Neumann (1903--1957) \cite{zukav1979dancing} (p. 208)}

This book is optimized for slow thoughtful focused sequential reading.
The author tries to make the book as short as possible.

\section{Notation}

A \emph{mathematical notation}, like the Latin alphabet,
is a way of writing \emph{English} or any other natural language.
When you read ``math'',
you are really reading the same language
that you speak everyday.

When you encounter symbols in a sentence,
think about how they should read in English
to make the whole sentence grammatically correct.

Beware that mathematical notations also have slangs and inconsistencies.

A large part of mathematics is about defining and naming things.
``\(A\) is \(B\) with some additional properties. \(B\) is \(C\) with some other properties.''

The reader is assumed to know the Greek alphabet.

\section{Some motivation}

Math seems hard because learning a new language is hard.

\emph{Math is fun.}
Indeed you can brainwash yourself into thinking anything is fun,
including physical exercise and healthy eating,
just by repeatedly shouting in your mind that it is fun.

\emph{Every time} you think you suck at math,
shout in your head, ``I love math!''
Even if you feel you can't,
no matter however phony you feel,
just shout it.
Repeat it until it becomes a reflex.
Remember that you \emph{just} want to love math;
you don't want to be a math expert.
You can love math without being a math expert.
It's perfectly logical.

In mathematics, right is right, and wrong is wrong,
regardless of age, skin color, gender, weight, wealth,
nationality, religion, or political affiliation.
In mathematics, we don't have to worry about offending anyone.

John von Neumann:
``If people do not believe that mathematics is simple, it is only because they do not realize how complicated life is.''
