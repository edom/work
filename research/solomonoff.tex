\chapter{Solomonoff theory}

Why was Raymond J. Solomonoff \cite{SolAlpProb2011, GacsVitanyiSolomonoff}
interested in predicting sequences of bits?

Marcus Hutter approached intelligence from \emph{algorithmic} complexity theory (Solomonoff induction)
\cite{DefineMachIntel}.
Warren D. Smith approached intelligence from \emph{computational} complexity theory
(NP-completeness)
\cite{WdsIntel, WdsIntelSlide}

\section{Generalization of Solomonoff theory}

Let \(\Bit = \{0,1\}\).
Let \(\Bits = \Bit^*\) where \(*\) is the Kleene star.
Let \(f : \Bits \to \Bits\).
Let \(p : \Bits \to [0,1]\) be the input distribution.
Define \(q~y\) as the probability of finding an input \(x\) such that \(f~x = y\).
Solomonoff used a universal Turing machine for \(f\),
but I'm interested on the consequences of relaxing this constraint.

Bijection allows us to use \(\Nat\) or any other countable set instead of \(\Bits\).
Let \(f : \Nat \to \Nat\) be a function.
We say that \emph{\(x\) \(f\)-explains \(y\)} iff \(f~x = y\).
Let \(p : \Nat \to [0,1]\) be the input distribution.
Define \(q~y\) as the probability of finding an \(x\) such that \(f~x = y\).
We call an input such as \(x\) a \emph{hypothesis}.

\newcommand\preimage{\fun{preimage}}

Let \(f\) be a binary relation.
We define \(\preimage~f~y = \{ x ~|~ f~x~y \}\).

\section{Generalizing Kolmogorov complexity}

Schmidh\"uber generalized Kolmogorov complexity to super Turing machines \cite{SchmidhuberKolmogorov}.

The Kolmogorov complexity of a string
is the length of the shortest program that generates that string.
This complexity depends on the machine.
Formally, we define \(\kolmogorov~f~y\),
the Kolmogorov complexity of \(y\) with respect to \(f\),
as the length of the shortest \(x\) that satisfies \(f~x = y\).

We can relax the constraint of \(f\) so that \(f\)
now only needs to be a surjective relation.
We can relax it further so that it is a relation.
However, if \(f\) is not surjective,
the Kolmogorov complexity will be partial
(undefined for some inputs).
That seems to be as general as logic allows.

We can relax the measure \(m : A \to \Real\) into a function \(m : A \to B\)
and an ordering \(c\).

\[
    \kolmogorov~c~m~f = \minimumBy~c \circ \map~m \circ \preimage~f
\]

Note the notation:
If \(f\) is a binary relation,
we overload the notation \(f\)
to also mean the binary predicate
that is the indicator function of the relation \(f\).
We conflate a predicate and its extension.
We write \(f~x~y\) to mean \((x,y) \in f\).
The type of the expression \(f~x~y\) is \(\Bool\).

We can generalize Kolmogorov complexity.
Instead of a space of strings and the length of strings,
we can use a measure space.
Furthermore, we can generalize the function into relation.
Let \(m : A \to \Real\) be a measure on \(A\).
Let \(f \subseteq A \times A\) be a binary relation.
Formally,
\[
    \kolmogorov~m~f~y = \minimum~(\map~m~(\preimage~f~y))
\]
or, more pointfreely,
\begin{equation}
    \kolmogorov~m~f = \minimum \circ \map~m \circ \preimage~f
\end{equation}

We define \(\kolmogorov~m~f~y\) as the measure of an \(x\)
that both minimizes \(m\) and satisfies \(f~x~y\).
Therefore \(\kolmogorov\) is a constrained optimization problem.
The classical Kolmogorov complexity is a special case of \(\kolmogorov\)
where \(m\) computes the length of a bitstring
and \(f\) is a universal Turing machine.

The Kolmogorov complexity is incomputable due to the halting problem.
It does not even have a brute-force algorithm.
However, Levin search allows us to approximate it.
