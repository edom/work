\chapter{Collatz conjecture}

Let \(f(n) = n/2\) if \(n\) is even or \(3n+1\) if \(n\) is odd.

\(f(n) = (n \bmod 2) (n/2) + (1 - n \bmod 2) (3n + 1)\).

\(f(n) = (n \bmod 2) (n/2) + (3n + 1) - (n \bmod 2) (3n + 1)\).

\(f(n) = (n \bmod 2) (n/2 - (3n + 1)) + (3n + 1)\).

\(f(n) = (n \bmod 2) (2 - 5n)/2 + (3n + 1)\).

\(2 f(n) = (n \bmod 2) (2 - 5n) + 6n + 2\).

\(f(2n) = n\)

\(f(2n+1) = 3(2n+1)+1 = 6n+4 = 2(3n+2)\)

\(f(2n+2) = n+1\)

\(f(2n+3) = 3(2n+3)+1 = 6n+10 = 2(3n+5)\)

\(f(2n+2k) = n+k\)

\(f(2n+2k+1) = 2(3n+2+3k)\)

\(f(2n+2k+1) = 2(3(n+k)+2)\)

\(f(2(n+k)+1) = 2(3(n+k)+2)\)

\(f(2m+1) = 2(3m+2)\)... back to square 0

Let \(q\) be odd.
\begin{align*}
    f(2^{p+1} q) = 2^p q
    \\
    f(2^0 q) = 3q + 1
    \\
    f(2k+1) = 3(2k+1) + 1 = 6k+4 = 2(3k+2)
    \\
    f(2(2k)+1) = 3(4k+1) + 1 = 12k+4 = 4(3k+1)
    \\
    f(2(2k+1)+1) = 3(4k+3) + 1 = 12k+10 = 2(6k+5)
\end{align*}

\begin{align*}
    f(6n) &= 3n
    \\
    f(6n+1) &= 3(6n+1)+1 = 18n+4
    \\
    f(6n+2) &= 3n+1
    \\
    f(6n+3) &= 3(6n+3) + 1 = 18n+10
    \\
    f(6n+4) &= 3n+2
    \\
    f(6n+5) &= 3(6n+5) + 1 = 18n+16
\end{align*}

Let \(g(n) = f(2n)\) and \(h(n) = f(2n+1)\).

Let \(a(n,0) = n\) and \(a(n,k+1) = f(a(n,k))\).

Corollary: \(a(n,k+1) = a(f(n),k)\).

Corollary: If \(n\) is odd, then \(f(n)\) is even.

\section{Recursion}

Lemma: For every \(n\) and \(d \neq 0\), there exists \(q,d\) such that \(n = qd+m\) and \(0 \le m < d\).

\begin{align*}
    f(2n+0) &= n+0
    \\
    f(2n+1) &= 6n+4 = 2(3n+2)
    \\
    f(4n+0) &= 2n+0
    \\
    f(4n+1) &= 12n+4 = 4(3n+1)
    \\
    f(4n+2) &= 2n+1
    \\
    f(4n+3) &= 12n+10 = 2(6n+5)
    \\
    f(8n+0) &= 4n+0
    \\
    f(8n+1) &= 24n+4 = 4(6n+1)
    \\
    f(8n+2) &= 4n+1
    \\
    f(8n+3) &= 24n+10 = 2(12n+5)
    \\
    f(8n+4) &= 4n+2 = 2(2n+1)
    \\
    f(8n+5) &= 24n+16 = 8(3n+2)
    \\
    f(8n+6) &= 4n+3
    \\
    f(8n+7) &= 24n+22 = 2(12n+11)
    \\
    f(16n+0) &= 8n+0
    \\
    f(16n+1) &= 48n+4 = 4(12n+1)
    \\
    f(16n+2) &= 8n+1
    \\
    f(16n+3) &= 48n+10 = 2(24n+5)
    \\
    f(16n+4) &= 8n+2 = 2(4n+1)
    \\
    f(16n+5) &= 48n+16 = 16(3n+1)
    \\
    f(16n+6) &= 8n+3
    \\
    f(16n+7) &= 48n+22 = 2(24n+11)
    \\
    f(16n+8) &= 8n+4 = 4(2n+1)
    \\
    f(16n+9) &= 48n+28 = 4(12n+7)
    \\
    f(16n+10) &= 8n+5
    \\
    f(16n+11) &= 48n+34 = 2(24n+17)
    \\
    f(16n+12) &= 8n+6 = 2(4n+3)
    \\
    f(16n+13) &= 48n+40 = 8(6n+5)
    \\
    f(16n+14) &= 8n+7
    \\
    f(16n+15) &= 48n+46 = 2(24n+23)
    \\
    f(16n-1) &= 48n-2 = 2(24n-1)
    \\
    f(16n-5) &= 48n-14 = 2(24n-7)
\end{align*}

\begin{align*}
    f^2(4n) &= (gg)(n) = n
    \\
    f^2(4n+1) &= (gh)(n) = 6n+2 = 2(3n+1)
    \\
    f^2(4n+2) &= (hg)(n) = 6n+4 = 2(3n+2)
    \\
    f^2(4n+3) &= (gh)(n) = 6n+5
    \\
    f^3(8n) &= (ggg)(n) = n
    \\
    f^3(8n+1) &= f^2(24n+4) = (ggh)(n) = 6n+1
    \\
    f^3(8n+2) &= f^2(4n+1) = (ghg)(n) = 6n+2 = 2(3n+1)
    \\
    f^3(8n+3) &= f^2(24n+10) = (hgh)(n) = 36n+16 = 4(9n+4)
    \\
    f^3(8n+4) &= f(2n+1) = (ggg)(n) = 6n+4 = 2(3n+2)
    \\
    f^3(8n+5) &= f^2(24n+16) = (ggh)(n) = 6n+4 = 2(3n+2)
    \\
    f^3(8n+6) &= f^2(4n+3) = (hgh)(n) = 6n+5
    \\
    f^3(8n+7) &= f^2(24n+22) = f(12n+11) = (ghg)(n) = 36n+34 = 2(18n+17)
    \\
    f^p(2^p q+0) &= f^{p-1}(2^{p-1} q+0)
    \\
    f^p(2^p q+1) &= f^{p-1}(3 \cdot 2^{p-1} q+4)
\end{align*}

Lemma: \(f^3(8n+4) = f^3(8n+5)\).

Let \(g(n) = n/2\) and \(h(n) = 3n+1\).

\section{Equation}

\(2^p q = 3 n + 1\) where \(q\) is odd.

\(2^{p_0} q_0 = 3 n_0 + 1\) where \(q\) is odd.

\(2^{p_1} q_1 = 3 n_1 + 1\) where \(q\) is odd.

\(n_{k+1} = 2^{p_k} q_k\)

\section{Sequence}

Consider three mutual sequences \(p\), \(q\), and \(n\).

Input: \(q_0\).

Constraint: \(q_k\) odd.

\begin{align*}
    n_0 &= q_0
    \\
    n_{k+1} &= 3 q_k + 1
    \\
    n_k &= 2^{p_k} q_k
\end{align*}

Collatz conjecture: For every odd \(q_0 \in \Nat\), there exists \(k\) such that \(q_k = 1\).

Alternative: For every odd \(a\), there exists \(k\) and odd \(q_0\) such that \(q_k = a\).

Lemma: If \(n\) is even, then \(3n+1\) is odd.

Lemma: If \(n\) is odd, then \(3n+1\) is even.

\section{Composition}

\(u(n) = 1\) where \(u \in (f|g)^*\)

\(3(2n)+1 = 3(2n+1)-2\)

\section{Composition}

\((h \circ g)(n) = 3(n/2) + 1 = (3n+2)/2\)

\((g \circ h)(n) = (3n + 1)/2 = (3n+1)/2\)

\((h \circ g - g \circ h)(n) = 1/2\)

\section{Path}

Let \(P(a,b)\) mean that there is a path from \(a\) to \(b\) in the forward Collatz graph.

(Domain of discourse is \(\Nat\).
All unbound variables are universally quantified.)

Corollary: If \(P(f(a),b)\) then \(P(a,b)\).

Corollary: If \(n\) is odd and \(P(3n+1,a)\) then \(P(n,a)\).

Corollary: \(P(6n+4,a)\) then \(P(2n+1,a)\).

Corollary: If \(P(n,a)\) then \(P(2n,a)\).

Corollary: If \(n \bmod 3 = 1\) and \(P(n,a)\) then \(P(\frac{n-1}{3},a)\).

Lemma: There is no \(n\) such that \(n \bmod 2^p = 1\) for all \(p > 0\).

Lemma: \(P(2^p n, n)\).

Corollary: \(P(2^p,1)\).

Lemma: If \(P(a,2^p)\) then \(P(a,1)\).

Lemma: If \(P(q, 1)\) then \(P(2^p q,1)\).

Conjecture: If \(P(a,1)\) and \(P(b,1)\) then \(P(ab,1)\)?

Conjecture: If \(P(a,1)\) and \(P(b,1)\) then \(P(a+b,1)\)?

Lemma: If \(P(3n+2,1)\) then \(P(2n+1,1)\).

Lemma: If \(P(3n+1,1)\) then \(P(4n+1,1)\).

Lemma: If \(n\) is odd, then \(f(4n+1) = 4f(n)\).

Lemma: If \(p \ge 2\) then \(f(2^p n + 3) = 2(3 \cdot 2^{p-1} + 5)\).

Suppose \(P(n,1)\). Prove \(P(n+1,1)\).

If \(P(k,1)\) then \(P(2k,1)\).

If \(P(6k+4,1)\) then \(P(2k+1,1)\).

\section{Equivalence}

Lemma: Iff every \(6n+4\) has a path to \(2^p\), then the Collatz conjecture is true.

If \(3n+1 < 4\) and \(3n \bmod 2 = 1\) then \(P(n+1,1)\).

If \(3n+1 < 8\) and \(3n \bmod 4 = 1\) then \(P(3n+1,1)\).

If \(3n+1 < 16\) and \(3n \bmod 8 = 1\) then \(P(3n+1,1)\).

If \(3n+1 < 32\) and \(3n \bmod 16 = 1\) then \(P(3n+1,1)\).

\(\{ (p,N) ~|~ n \in N, ~ 3n \bmod 2^p = 1 \} = \{ (1,\{1,3,5,\ldots\}), (2,\{3,7,11,\ldots\}), (3,\{3,11,19,\ldots\}) \}\)

For all \(n,p\): Statement \(S(p)\) is: if \(3n+1 < 2^{p+1}\) and \(3n \bmod 2^p = 1\) then \(P(3n+1,1)\).

An inductive proof of \(S(p)\) is a proof of the Collatz conjecture.

\section{Representation}

In this section, the domain of discourse is \(\Nat\).

Assume \(n > 0\).
Define \(t(n)\) as the largest \(p\) such that \(n = 2^p q\).
Equivalent:
Define \(t(n)\) as the \(p\) such that \(n = 2^p q\) and \(q\) is odd.

Lemma:
For every positive \(n\), there exists \(p\) and odd \(q\) such that \(n = 2^p q\).

Lemma:
For every positive \(n\), there exists \(p\) and \(q < 2^p\) such that \(n = 2^p + q\).

Lemma: If \(b < 2^a\) then \(t(2^a + b) = \min(a,t(b))\).

Lemma: For every \(n\), there exist \(p,q\) such that \(3(2n+1)+1 = 2^p q\).

Corollary: Iff \(n\) is a power of two, then \(q = 1\).

If \(m = 2^a b\) and \(n = 2^c d\) then \(mn = 2^{a+c} (bd)\).

\begin{align*}
    (2^a b + 2^c d) \bmod 2^p = ?
\end{align*}

\(2^p \cdot 1 + 1 = 2^0 (2^p + 1)\).

If \(k\) is even then \(2^0 (2k + 1) + 1 = 2k + 2 = 2^1 (k+1)\).

\(2^0 (2(2k) + 1) + 1 = 4k+2 = 2^0(2k+1)\)

If \(k\) is even then \(2^0 (2(2k) + 3) + 1 = 4k+4 = 2^2(k+1)\)

\(t(1,2,3,4,5,\ldots) = 0,1,0,2,0,1,0,3,0,1,0,2,0,1,0,4,\ldots\)

\(t(2,4,6,8,10,\ldots) = 1,2,1,3,1,2,1,4,1,2,1,3,1,2,1,5,\ldots\)

\(t(k)\) is the number of trailing zero bits in the binary representation of \(k\)

Lemma: \(t(k) \cdot t(k+1) = 0\)

Lemma: \(t(ab) = t(a) + t(b)\)

If \(t(k) \ge 1\) then \(2^0 (2^p k+1) + 1 = 2^p k + 2 = 2^p (k+1)\)

If \(t(k) \ge 1\) then \(t(2^a k+1) = a\)

\(t(3k) = t(k)\)

Under what circumstances are \(3(2n+1)+1 = 2^p\) (that is \(q=1\))?

\(6n+4 = 2^p\).

\(1 \equiv 2^p \pmod 3\).
Corollary: Iff \(p\) is even then \(2^p+2\) is even and divisible by 3.

Corollary: If \(t(n) > 0\) then \(t(f(n)) = t(n)-1\).

\[
    g(n) =
    \begin{cases}
        \{2n\} & \text{if } n \bmod 3 \in \{0,2\}
        \\
        \{2n, \frac{n-1}{3}\} & \text{if } n \bmod 3 = 1
    \end{cases}
\]

Let \(G(N) = \bigcup_{n \in N} g(n)\).

Let \(H_0 = \{1\}\) and \(H_{k+1} = H_k \cup G(H_k)\).

\(H \sim \{1\}, \{2,1,0\}, \{4,2,1,0\}, \{8,4,2,1,0\}, \{16,8,5,4,2,1,0\}, \{32,16,10,8,5,4,2,1,0\}\)

\(\forall k : H_k \subset H_{k+1}\)

Collatz conjecture: for every \(N \subseteq \Nat - \{1\}\), there exists \(p \in \Nat\) such that \(N \subseteq G^p(\{1\})\).

Conjecture: \(G(\Nat) = \Nat\).

\section{Collatz graph}

Graph \((V,E)\)

\((2n,n) \in E\)

\((2n+1,3(2n+1)+1) \in E\), that is
\((2n+1,6n+4) \in E\)

\((2,1),(4,2),(6,3),(8,4),(10,5),\ldots\)

\((1,4),(3,10),(5,16),(7,22),(9,28),\ldots\)

Prove that there is a path from every positive \(n\) to 1.

Inverse graph

\((n,2n) \in E'\)

\((6n+4,2n+1) \in E'\)

Prove that there is a path from 1 to every positive \(n\).

Can we transform \(f\) into a monotonically increasing function?
