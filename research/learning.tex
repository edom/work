\chapter{Learning}

\section{The artificial dog brain}

This system is basic reinforcement learning.

The operator institutes external reward
(virtual snack) or punishment (virtual pain)
by a simple switch.

The behavior function, given sensors, drives the actuators.

Uses gradient descent for backpropagation.

Let \( x : I^m \), \( y : I^n \).

The operator determines the reward function;
it can be \(-1,0,1\).

The equation governs the system:
\[
    d~f = g~r
\]
where \(d : (A \to A) \to (A \to A)\) is the differentiation operator,
\( f : I^m \to I \) is the behavior,
and \( r : I \) is the external reward.
If we can feed the reward back from
the environment to the input of the system,
then we have an adaptive system that
learns on its own throughout its lifetime without operator supervision.
In English: the change of behavior is a function of the reward.

\[
    f'~x - f~x = r
\]
where
\[
    f' = f + d~f
\]
The definition of \(d\):
\[
    h \cdot d~f~x = f~(x+h) - f~x
\]
as \(h\) vanishes,
but the catch is that
here \(x\) and \(h\) are functions, not the usual numbers.

Hypothetical system:
\begin{align*}
    f'~x - f~x &= r
    \\
    (f' - f)~x &= r
    \\
    ((f + h) - f)~x &= r
    \\
    h~x &= r
    \\
    f'~x &= f~x + h~x
\end{align*}

The reward is the projection of the behavior difference to the real line.
The function \(h\) is the behavior difference function.
We can choose any \(h\) that satisfies \((f + h)~x = r\).
The problem is: which one?
There are infinitely many such functions.

Another thing that makes humans different from machines is that
\emph{over time, human changes how they respond to rewards.}
There is always a hardwired mechanism that interacts with the environment
to reinforce certain behaviors by driving internal satisfaction.

It's possible that mathematics may uncover a better architecture for learning,
or even uncover the limits of learning.

A possible reward mechanism in human brain is dopamine level.
If this hypothesis is true, then it should be possible to change human behavior,
possibly against the brain's will, by dopamine therapy.
Dopamine therapy, combined with conditioning (like what Pavlov did to dogs),
should be able to turn anyone into someone else entirely different,
someone who enjoys what he hated, and hates what he enjoyed.

In biology, there is \emph{reward system}.

\begin{align*}
    \frac{(f + h)~x - f~x}{h} = r
\end{align*}

We understand what it means to change a number a little.
But what does it mean to change a function a little?

We can define

\[
    f + g = \lambda x \to f~x+g~x
\]

Indeed we can lift any binary operator elementwise that way.
The difference between two functions is also a function.
There is a concept of infinitesimals in the set of hyperreal numbers.
We can extend functions to hyperfunctions in the same way
we extend real numbers to hyperreal numbers;
then we define infinitesimals in function space.
But probably we'll fare better if we just approach this from topology
in the first place instead of from nonstandard analysis.

The problem is, when punished, the dog knows that
\(x\) should not be mapped to \(y\),
but it does not know what \(x\) should be mapped to.

The dog is hardwired with the preconception that the output is proportional to the input.

How does the reward function relate \(f_1\) and \(f_2\)?

The brain samples the input at certain time intervals.

\section{Precisely defining vague words}

The most difficult task in establishing a learning theory is defining learning mathematically.
With the right definition, things will fall into place on their own.

A learning equation with goal function \(g\)
is an equation \( f~x = y \) where \( y \) may contain \( f \)
and that equation maximizes \( g \).
\emph{Learning is always relative to a goal function.}
There is nothing that `just learns'.
It learns \emph{something};
that is, `learn' is a transitive verb.

A learning system is a control system that minimizes \(g~x\), subject to
\[
    \bmat{z \\ y} = \bmat{r~z~y \\ c~y~x}.
\]

A learning system is a simultaneous functional equation and optimization problem.
Learning is a special case of \emph{constrained optimization}.
Every learning has the form ``Minimize \(g~x\), subject to \(x = f~x\).''

The goal function is a function of the input signal.
There is no other definition of the goal function that makes sense.

In real world, we don't know the goal function;
we can only observe behavior.
The agent may seem to change its high-level goal,
but there has to be one hardwired goal that causes the agent to change high-level goal.
The agent changes its high-level goal because it is hardwired to maximize its hardwired goal.
Every behavior of the agent can be understood in terms of maximizing its hardwired goal.

Every normal embryo of a species grows in the same way,
following the same sequence,
with the same number of cells.
This suggests that every member of a species
shares the same `bootstrapping' program. (?)

\section{General learning}

How do we define learning mathematically?
A dog learns. A human learns. An organization learns. A machine learns.

To learn is to purposefully change behavior?

Purposeful can be defined as the optimization (extremization) of a mathematical function.

Define behavior mathematically?

\section{Learning systems are special cases of control systems}

A learning system is a control system that alters
its own feedback mechanism to optimize its goal function.

An intelligent system can reach its goal in various environments.

A measure of intelligence: diff. environment / diff. goal ?

\section{Supervised learning is function approximation or statistical estimation?}

Every supervised learning problem is a special case of this:
\begin{quote}
Given a few \((x,y) \in f\), approximate \(f\).
\end{quote}

\section{Information gain}

Shannon's definition of \emph{surprisal} (self-information)?

A neuron \emph{reduces entropy}?

\section{Neuron on a finite field}

Who says we have to use real numbers?

There are two ways of thinking about a function \(f : A^n \to A\) where \(A\) is finite:
as a \emph{projection} from a hyperplane to a line,
or as a \emph{height map}.
Thinking it as a projection (a dimension-reducing function)
gives rise to the question about the information gain.
Thinking it as a height map gives rise to the question about the gradient.

The question is: can we approximate a complex function as a combination of simple functions?

Let \(L\) be a subset of \( A^2 \to A \).
We call \(L\) the \emph{allowed} set.
The question is: can we compose arbitrarily many functions from \(L\)
to form \( A^2 \to A \)?
If yes, then we say that \(L\) \emph{generates} \( A^2 \to A \).

Conjecture: \(L\) generates \(A^2 \to A\) iff
there is no triple \((f,g,h)\in F^3\) that satisfies \( \forall x,y,z: f~x~(g~y~z) = h~x~y \).

Translate to free monoid.
There are 4 ways to wire a 2-function: \( fxx, fxy, fyx, fyy \).
There are 2 patterns to wire two 2-functions: \( ffxxx, fxfxx \),
where an \(f\) is a 2-function and an \(x\) is a variable.
In total, there are \( 2^2 \times 3^3 = 108 \) ways to wire two 2-functions.
However, \( |A^3 \to A| = |A|^{|A|^3} \).
There are ? patterns to wire three 2-functions: \( fffxxxx, ffxfxxx, ffxxfxx, fxffxxx, \ldots \).

There are two combinations: \(\fb{x}{\fb{y}{f~x~y}}\) and \(\fb{x}{\fb{y}{f~y~x}}\).

Wiring combinations 2: \(f~(g~x~y)~z\), \(f~x~(g~y~z)\).

Can we build \emph{every} neural network from 2-input neurons only?

\section{Learning is special case of feedback}

\section{Neuron, neural network layer, neural network}

\index{neuron}%
\index{neural network}%
\index{neural network layer}%
The type of a \emph{neuron} is \(\Real^\infty \to \Real\).
The type of a \emph{neural network layer} is \(\Real^\infty \to \Real^\infty\).
The type of a \emph{neural network} is \(\Real^\infty \to \Real^\infty\).
If the neurons are \(N_0,N_1,\ldots\), the layer is \(L~x = [N_0~x, N_1~x, \ldots]\).
If the layers are \(L_0,L_1,\ldots\), the network is \(W~x = \ldots L_1~(L_0~x) \).

\section{Neuron}

The type of a recurrent neuron is \(\Real \to \Real^\infty \to \Real\),
but that type has the cardinality as \(\Real^\infty \to \Real\) does.
A \emph{recurrent neuron} is an equation \(y_t = f~m_t~x_t\)
and a recurrence relation \(m_t = y_{t-1}\).
An infinite impulse response filter is a recurrence relation.

A \emph{neuron} \(n : \Real^\infty \to \Real\)
has weight \(w \in \Real^\infty\), input \(x \in \Real^\infty\),
nonlinear activation function \(f : \Real \to \Real\),
and output \(y = f~(w \cdot x)\) where \(w \cdot x = \sum_{k=0}^\infty w_k \cdot x_k\).
Elsewhere the term \emph{artificial neuron} is used to disambiguate
such neuron from \emph{biological neuron}.

A neuron is analytically a function approximation
and topologically a continuous mapping.

A neuron with two inputs computes \(y = f~(w_0 \cdot x_0 + w_1 \cdot x_1)\).
An infinite neuron computes \(y = f~w~x\)
where each of \(y,w,x\) is an infinite sequence of real numbers.

\index{McCulloch-Pitts neuron}%
\index{neuron!McCulloch-Pitts}%
A \emph{McCulloch-Pitts neuron} is a neuron whose activation function is the step function.

A Hopfield network.

\section{Neural network layer}

A \emph{neural network layer} is
\([y_1,\ldots,y_n]\) where each \(y_k = f~(w_k \cdot x) \).
A neural network layer is
\(Y = F~(W \vec{x}) \).
\(F\) is nonlinear.

\section{Radial basis function neural network}

Non-incremental least-square-error approximation by a sum involving a radial basis function.

The \emph{approximator} is \(f(x) = \sum_{k=1}^n w_k \cdot b(s_k,x-x_k)\)
where \(n\) is fixed beforehand
where \(b\) is the basis function,
\(s_k\) is a spread,
and \(x_k\) is a center.
Error is \(e = \sum_k \norm{f(x_k) - y_k}^2\).
The \emph{approximand} is \(g\).
To obtain \(f\),
get \(n\) samples \((x_1,y_1), ~ \ldots , ~ (x_n,y_n)\) where each \(y_k = g~x_k\),
and then
solve this linear equation to obtain \(\vec{w}\):
\begin{align}
    \bmat{g~x_1 \\ \vdots \\ g~x_n} = \bmat{b~(x_1-x_1) & \cdots & b~(x_1-x_n) \\ \vdots & \ddots & \vdots \\ b~(x_n-x_1) & \cdots & b~(x_n-x_n)} \bmat{w_1 \\ \vdots \\ w_n}
\end{align}
The resulting \(\vec{w}\) minimizes the
\emph{square error} (the norm-2 of the error vector).

Incremental approximation.
Update the approximator to \(f' = f + \Delta f\)
where
\(\Delta f~x = - b ~ (x - x_0) \cdot \Delta y\).
Use genetic algorithm to optimize the width of \(b\).
See that \(f'~x_0 = y_0\).
Relate \(\Delta f = f' - f\) and \(\Delta y\).
\(\int_{-\infty}^\infty \Delta f~x ~ dx = - c \cdot \Delta y\).
If \(g\) is integrable, then \(f\) will converge if given enough samples.
Let \(y = f~x_0\).
The error at \(x_0\) is \(\Delta y = y - y_0\)
where \(y = f~x_0\) and \(y_0 = g~x_0\).
Let \(b\) be a probability density function (an integrable function with one unit of area under the curve) whose peak is at zero.

%\section{Adversarial networks}
%\section{Generative adversarial networks}

% https://en.wikipedia.org/wiki/Feature_learning
% https://en.wikipedia.org/wiki/Autoencoder
% https://en.wikipedia.org/wiki/Recurrent_neural_network
% http://cs.stanford.edu/people/eroberts/courses/soco/projects/neural-networks/Architecture/index.html
% https://en.wikipedia.org/wiki/Autoreceptor

\section{FitzHugh-Nagumo neuron}

\index{FitzHugh-Nagumo neuron}%
\index{neuron!FitzHugh-Nagumo}%
A \emph{FitzHugh-Nagumo neuron} is the following system of equations \cite{izhikevich2006fhn}
\begin{align*}
    \dot v &= v - \frac{v^3}{3} - w + I_{ext}
    \\
    \tau \dot w &= v + a - b w
\end{align*}

\section{Bibliography}

\emph{HLearn: a machine learning library for Haskell} \cite{izbicki2013hlearn}

Kaggle: get paid to solve machine learning problems.
