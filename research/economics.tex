\chapter{Economics}

\epigraph{It is sometimes said, common sense is very rare.}{François-Marie Arouet a.k.a. Voltaire (1694--1778)\\in \emph{Philosophical Dictionary} (1767)}

\section{Reading list}

David Ricardo

Henry George

Adam Smith

John Maynard Keynes

Von Neumann-Morgenstern

\section{Higher-order belief}

\index{higher-order belief}%
\index{higher-order belief!first-order}%
\index{belief!first-order}%
\index{first-order belief}%
Example of the effect of \emph{first-order belief}:
Crisis happens because people think that there is a crisis.
Money is valuable because people think that money is valuable.
\index{higher-order belief!second-order}%
\index{belief!second-order}%
\index{second-order belief}%
Example of the effect of \emph{second-order belief}:
People think that money is valuable because
they believe that other people also think that money is valuable.

\section{Price}

The \emph{price} of something is the amount the buyer
pays the seller to make the seller give that thing to the buyer.
The price of something is the amount of money exchanged with that thing.

``A buyer \emph{pays} a seller an amount of money for something'' means
that the buyer gives the seller that amount of money
to change the ownership of that thing from the seller to the buyer.

Money has \emph{currency} and \emph{amount}.

To \emph{own} something is to have exclusive access to that thing.

\section{Capitalism}

A sign of capitalism is the private ownership of the means of production
for profiting from voluntary trades.

Capital is everything that is not labor.

Capitalism is capital above labor.

Example of low-capital high-labor:
food stalls, home bakeries,
art freelancing,
research in pure mathematics,
and other small and medium enterprises (SMEs).

\section{Definitions}

\index{economy}%
An \emph{economy} is a government $G$,
a set of banks $B = \{B_0,B_1,\ldots\}$ of which $B_0$ is the central bank,
and a set of other actors $A = \{A_0,A_1,\ldots\}$.

Money can be exchanged with many other things.

The price of debt is the interest
(the amount of money that will be exchanged for obtaining the loan now).

Central bank creates money.
A commercial bank does not have to have
central bank money in order to create a loan.
Electronic money makes it even easier.

\section{Initial money supply}

How is money distributed for the first time?

\section{Money supply}

If there are more things exchanged, more money is required.

A government's domestic debt is that nation's wealth.
This is just accounting.
However, this doesn't say anything about the \emph{distribution}
of that wealth.

As long as there are humans, food will always be in demand.
If we have more humans, either we need more food or they need to eat less.
If we want more food, we need more land or a better way of producing food.

All organizations, including corporations and governments, employ people.
These organizations pay these employees for these employees' labor.

\section{Money circulation}

There are two aspects of the money in circulation: the amount and the speed.

Money moves faster in certain areas.

The government increases the amount of money in circulation
by spending new money by buying things from a few people,
with the hope that these few people will quickly spend that money,
and thus that money will circulate to more people; or give that money to people.
However, to make money circulate faster,
people, especially the owners of those new money,
have to be more willing to spend,
and there must be the things (goods or services)
that the money can actually be spent on.

The amount of money in circulation can be reduced by tax or destruction.

To increase the amount of money in circulation now
at the cost of reducing a bigger amount of money in circulation in the future,
people borrow from banks.

Principal plus interest (money retired) is greater than principal (money created),
so the net effect of lending money now
is a delayed reduction of a greater amount of money in circulation.

If left alone, eventually the money collects at the banks,
but banks have expenses too.
Banks need to pay taxes, pay their utility bills,
pay their staffs.

To reduce the amount of money in circulation now
at the cost of increasing a bigger amount of money in circulation in the future,
government offers bonds, and a few people buy those bonds.
This depend on the trust the public has on the government.

Where does the money to pay the interest come from?

Tax allows government to redistribute money.

If the total amount of money is constant,
then the only way for you to have more money
is to make someone else has less money.
