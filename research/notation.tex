\chapter{Notation}

\section{Phrases}

We assume that you know lambda calculus, currying, and partial function application.

\begin{center}
    \begin{tabular}{ll}
        \(f~x\) & the result of applying \(f\) to \(x\); the value of \(f\) at \(x\)
        \\
        \(f~x~y\) & the same as \((f~x)~y\)
        \\
        \(f~x~y~z\) & the same as \(((f~x)~y)~z\)
        \\
        \(x_k\) & the same as \(x~k\); component \(k\) of vector \(x\)
        \\
        \(f~x_k\) & the same as \(f~(x~k)\)
        \\
        \(x^2\) & \(x \cdot x\)
        \\
        \(f^2\) & \(f \circ f\)
    \end{tabular}
\end{center}

\section{Context-dependent phrases}

Depending on context,
\(x^2\) can mean the number \(x\) multiplied by itself
(that is \(x^2 = x \cdot x\))
or the function \(x\) composed with itself (that is \(x^2~y = x~(x~y)\)).
The letters \(f,g,h\) aren't always functions,
and the letters \(x,y,z\) are sometimes functions.

We also overload operators.
We use the same operator to mean
several different things.
For example, in the equation \((f + g)~x = f~x + g~x\),
the left \(+\) means function addition and the right \(+\) means number addition.

\begin{center}
    \begin{tabular}{ll}
        \(x^2\) & \(x \cdot x\)
        \\
        \(f^2\) & \(f \circ f\)
    \end{tabular}
\end{center}

\section{Clauses}

\begin{center}
    \begin{tabular}{ll}
        \(x : a\) & the type of \(x\) is \(a\)
        \\
        \(x : a\) & \(x\) is an inhabitant of \(a\)
        \\
        \(x : a\) & \(x\) is an instance of \(a\)
        \\
        \(x : a\) & \(x\) is a member of \(a\)
        \\
        \(p~x\) & \(p~x\) is true; \(x\) satisfies \(p\)
    \end{tabular}
\end{center}

\section{Types}

\begin{center}
    \begin{tabular}{ll}
        \(\Nat\) & the type of natural numbers: \(\{ 0, 1, 2, \ldots \}\)
        \\
        \(\Real\) & the type of real numbers
        \\
        \(I\) & the unit interval: \([0,1]\)
        \\
        \(A^\infty\) & infinite sequence of \(A\)s
        \\
        \(a \to b\) & the type of functions from \(a\) to \(b\)
        \\
        \(a \to b \to c\) & the same as \(a \to (b \to c)\)
    \end{tabular}
\end{center}

We write \(f~x~y = x+y\) to mean \((f(x))(y) = x+y\).
In this case, \(f~x\) is a function that returns a one-argument function.

We use the standard lambda calculus abstraction notation
\(\fb{x}{y}\) to mean the unnamed function \(\cdot(x) = y\).
The notation \(f~x = y\) and \(f = \fb{x}{y}\) means the same thing,
but the former is more readable.
