\chapter{Abstract algebra}

\section{Group}

\index{binary operation}%
The type of a \emph{binary operation} is \(a \to b \to c\).
\index{binary operation!closed}%
\index{closed binary operation}%
The type of a \emph{closed binary operation} is \(a \to a \to a\).
A binary operation \(\cdot\)
is \emph{associative} iff \(x\cdot(y\cdot z) = (x\cdot y)\cdot z\),
is \emph{commutative} iff \(x \cdot y = y \cdot x\).

\index{magma}%
A \emph{magma} is a set and a closed binary operation.
\index{semigroup}%
A \emph{semigroup} is an associative magma (a magma whose operation is associative).
\index{monoid}%
A \emph{monoid} is a semigroup with an identity element.
\index{group}%
A \emph{group} is a monoid where each element has an inverse.
\index{commutative!group}%
\index{group!abelian}%
\index{group!commutative}%
An \emph{abelian group} or a \emph{commutative group}
is a group whose operation is commutative.

\(-x\) is the additive inverse of \(x\).

\section{Ring}

A
\index{ring}%
\emph{ring} is \((S,+,\cdot,-,0,1)\) where
\((S,+,-,0)\) is a commutative group,
\((S,\cdot,1)\) is a monoid,
and multiplication distributes addition:
\begin{align*}
    x \cdot (y+z) &= (x \cdot y) + (x \cdot z)
    \\
    (x+y) \cdot z &= (x \cdot z) + (y \cdot z).
\end{align*}
A ring is
\index{commutative!ring}%
\index{ring!commutative}%
\emph{commutative} iff its multiplicative monoid is commutative.

If \(R\) is a ring and \(A \to R\) is a function space,
then the \emph{natural ring of the function space} is:
\begin{align*}
    0_{A\to R}(x) &= 0_R
    \\ 1_{A\to R}(x) &= 1_R
    \\ (- f)(x) &= - f(x)
    \\ (f + g)(x) &= f(x) + g(x)
    \\ (f \cdot g)(x) &= f(x) \cdot g(x).
\end{align*}

\section{Field}

\(x^{-1}\) is the multiplicative inverse of \(x\).

\index{field}%
\((F,+,\cdot,-,{}^{-1},0,1)\) is a \emph{field} iff
\((F,+,-,0)\) is a commutative group,
\((F-\{0\},\cdot,{}^{-1},1)\) is a commutative group,
and \(\cdot\) distributes \(+\) (that is \(x \cdot (y + z) = (x \cdot y) + (x \cdot z)\)) \cite{wpfield}.
For brevity, if \(F\) is a set, ``\(F\) is a field'' means that \((F,+,\cdot,-,{}^{-1},0,1)\) is a field.
That is, for brevity, we blur the distinction between a field and its underlying set.
