\chapter{Differentials}

\section{Differentiator}

\index{differentiator}%
The \emph{differentiator}
\(d : (\Real \to \Real) \to (\Real \to \Real)\)
is \(d(f) = \lim_{h\to 0} \frac{f(x+h)-f(x)}{h}\).

If \(d\) is the differentiator and \(f\) is a function,
we will abuse the notation \(df\) to mean \(d(f)\).

Write \(d^{n+1}\) to mean \(d \circ d^n\)
where \(\circ\) is function composition
and \(n \in \Nat\).
Thus \(d^{n+1}(f) = d(d^n(f))\).

\section{Fixed point of differentiator}

Let \(e\) be
\index{Euler's constant}%
\index{base of natural logarithm}%
Euler's constant, the base of natural logarithm.

\index{derivative!fixed point}%
Iff \(f(x) = e^x\), then \(df = f\).

\section{Linearity}

\index{derivative!linearity}%
If \(c\) is a constant function, then \(d(c \cdot f) = c \cdot df\).

\(d(f+g) = df + dg\).

\section{Product rule}

\index{derivative!product rule}%
\index{product rule}%
Product rule: \(d(f \cdot g) = df \cdot g + f \cdot dg\).

\section{Chain rule}

\index{derivative!chain rule}%
\index{chain rule}%
Chain rule: \(d(f \circ g) = df \cdot g + f \cdot dg\).

\section{Power rule}

\index{derivative!power rule}%
\index{power rule}%
Power rule: \(d(x \to x^n) = n x^{n-1}\).

\section{Directional derivative}

\index{directional derivative}%
\index{derivative!directional}%
The derivative of \(f\) at direction \(v\) at \(x\) is
\((d_v f)(x) = \lim_{h\to 0}\frac{f(x+h v)-f(x)}{h}\)
where \(x:\Real^n\).

\section{Gradient}

\index{gradient}%
The \emph{gradient} of \(f\) is the \(\nabla f\)
that satisfies \((\nabla f)(x) \cdot v = (d_v f)(x)\).
It is also written \(\fgrad(f)\).

The type of \(\nabla\) is \((\Real^n \to \Real) \to (\Real^n \to \Real^n)\).

The symbol \(\nabla\) is called \emph{del} or \emph{nabla}.

\section{Differential equation}

\index{differential equation}%
A differential equation is something like \(d^2f = -cf\).

\section{Partial derivative}

Let \(f : \Real^n \to \Real^m\).
Let \(f_k(x) = (f(x))_k\).
Define the
\index{partial derivative}%
\emph{partial derivative of \(f\) with respect to the \(k\)th variable} as
\((d_k f)(x) = \lim_{h\to 0}\frac{f(x+he_k)-f(x)}{h}\).

\section{Jacobian operator}

Define the
\index{Jacobian operator}%
\emph{Jacobian operator} \(J : (\Real^n \to \Real^m) \to (\Real^n \to \Real^{m \times n})\) as:
\[
    [(Jf)(x)]_{ij} = (d_j f_i)(x)
\]
The matrix \((Jf)(x)\) is called the
\index{Jacobian matrix}%
\emph{Jacobian matrix of \(f\) at \(x\)}.
